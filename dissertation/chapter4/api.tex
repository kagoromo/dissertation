\section{The LVish library interface for application writers}\label{s:lvish-api}

In this section \either{I}{we} illustrate the use of the LVish library
from the point of view of the application writer, through a series of
short example programs.\footnote{The examples in this section, among
  others, are available at
  \url{https://github.com/lkuper/lvar-examples/}.}

\subsection{A simple example: IVars in LVish}

\ifdefined\DISSERTATION
As mentioned in the previous section, the LVish library extends the
approach of Haskell's monad-par library for deterministic parallelism,
which allows communication between parallel tasks through IVars.
Recall that IVars can only be assigned to once.  Therefore the
following program written using monad-par will deterministically raise an
error, because it tries to write to the IVar @num@ twice:

\singlespacing
\lstinputlisting{chapter4/code/ivar-example.hs}
\doublespacing

Here, @p@ is a computation of type @Par Int@, meaning that it runs in
the @Par@ monad (via the call to @runPar@) and returns a value of
@Int@ type.  @num@ is an IVar, created with a call to @new@ and then
assigned to twice, via two calls to @put@, each of which runs in a separately
@fork@ed task.  The @runPar@ function is an implicit global barrier:
all @fork@s have to complete before @runPar@ can return.

The program raises a ``multiple put'' error at runtime, which is as it
should be: differing writes to the same shared location could cause
the subsequent call to @get@ to behave nondeterministically.  (Since
we are using monad-par and not LVish, @get@ has IVar semantics, not
LVar semantics: rather than performing a threshold read, it blocks
until @num@ has been written, then unblocks and evaluates to the exact
contents of @num@.)

However, when using monad-par, even multiple writes of the \emph{same}
value to an IVar will raise a ``multiple put'' error:

\singlespacing
\lstinputlisting{chapter4/code/repeated-4-ivar.hs}
\doublespacing

This program differs from the previous one only in that the two @put@s
are writing @4@ and @4@, rather than @3@ and @4@.  Even though the
call to @get@ would produce a deterministic result regardless of which
write happened first, the program nevertheless raises an error because
of monad-par's single-write restriction on IVars.

Now let us consider a version of this program written using the LVish
library. (Of course, in LVish we are not limited to IVars, but we
will consider IVars first as an interesting special case of LVars, and
then go on to consider some more sophisticated LVars later in this
section.)  A version of the above program written using LVish will
write @4@ to an IVar twice and then deterministically print @4@
instead of raising an error:
\fi

\ifdefined\JOURNAL
First, let us consider an LVish program that writes the same value to
an IVar twice.  (Of course, in LVish we are not limited to IVars, but
we will consider IVars first as an interesting special case of LVars,
and then go on to consider some more sophisticated LVars later in this
section.)
\fi

\singlespacing
\lstinputlisting{chapter4/code/repeated-4-lvar.hs}
\doublespacing

\ifdefined\JOURNAL
\noindent Compared to a monad-par implementation, the differences are minor.
\fi
\either{Here, we}{We} need to import the @Control.LVish@ module rather than
@Control.Monad.Par@ (that is, we are using LVish instead of
monad-par), and we must specifically import @Data.LVar.IVar@ in order
to specify which LVar data structure we want to work with (since we
are no longer limited to IVars).  Just as with monad-par, the LVish
@runPar@ function is a global barrier: both @fork@s must compete
before @runPar@ can return.  Also,\either{as before}{as in monad-par}, we have @new@, @put@,
and @get@ operations that respectively create, update, and read from
@num@.  However, these operations now have LVar semantics: the @put@
operation computes a lub (with respect to a lattice like that of
Figure~\ref{f:lvars-example-lattices}(b), except including all the
@Int@s), and the @get@ operation performs a threshold read, where the
threshold set is implicitly the set of all @Int@s.  We do not need to
explicitly write down the threshold set in the code; rather, it is the
obligation of the @Data.LVar.IVar@ module to provide operations (@put@
and @get@) that have the semantic effect of lub writes and threshold
reads (as \either{I}{we} touched on earlier in
Section~\ref{subsection:lvars-the-model-versus-reality}).

There are two other important differences between \either{the
  monad-par program}{a monad-par implementation} and the LVish
program: the @Par@ type constructor has gained two new type
parameters, @e@ and @s@, and @p@'s type annotation now has a
\emph{type class constraint} of @(HasPut e, HasGet e)@.  Furthermore,
we have added a @LANGUAGE@ pragma, instructing the compiler that we
are now using the @TypeFamilies@ language extension.  In the following
section, \either{I}{we} explain these changes.

\subsection{The \il{e} and \il{s} type parameters: effect tracking and session tracking}

In order to support both deterministic and quasi-deterministic
programming in LVish, we need a way to specify which LVar effects can
occur within a given @Par@ computation.  In a deterministic
computation, only update operations (such as @put@) and threshold
reads should be allowed; in a quasi-deterministic computation,
@freeze@ operations should be allowed as well.  Yet other combinations
may be desirable: for instance, we may want a computation to perform
\emph{only} writes, and not reads.
\ifdefined\DISSERTATION
Furthermore, we want to be able to
specifically allow or disallow \emph{non-idempotent} update
operations, which \either{I}{we} discuss in more detail in
Section~\ref{subsection:lvish-discussion-leveraging-idempotency}.
\fi

In order to capture these constraints and make them explicit in the
types of LVar computations, LVish indexes @Par@ computations with a
\emph{phantom type} @e@ that indicates their \emph{effect level}.  The
@Par@ type becomes, instead, @Par e@, where @e@ is a type-level
encoding of Booleans indicating which operations, such as writes,
reads, or freeze operations, are allowed to occur inside it.  LVish
follows the precedent of Kiselyov \etal~on extensible effects in
Haskell~\cite{oleg-amr-haskell-2013}: it abstracts away the specific
structure of @e@ into \emph{type class constraints}, which allow a
@Par@ computation to be annotated with the \emph{interface} that its
@e@ type parameter is expected to satisfy.  This approach allows us to
define ``effect shorthands'' and use them as Haskell type class
constraints.  For example, a @Par@ computation where @e@ is annotated
with the effect level constraint @HasPut@ can perform @put@s.  In our
example above, @e@ is annotated with both @HasPut@ and @HasGet@ and
therefore the @Par@ computation in question can perform both @put@s
and @get@s.  We will see several more examples of effect level
constraints in LVish @Par@ computations shortly.

The effect tracking infrastructure is also the reason why we need to
use the @TypeFamilies@ language extension in our LVish programs.  For
brevity, \either{I}{we} will elide the @LANGUAGE@ pragmas in the rest
of the example LVish programs in this section.

The LVish @Par@ type constructor also has a second type parameter,
@s@, making @Par e s a@ the complete type of a @Par@ computation that
returns a result of type @a@.  The @s@ parameter ensures that, when a
computation in the Par monad is run using the provided @runPar@
operation (or using a variant of @runPar@, which \either{I}{we} will discuss
below), it is not possible to return an LVar from @runPar@ and reuse
it in another call to @runPar@.  The @s@ type parameter also appears
in the types of LVars themselves, and the universal quantification of
@s@ in @runPar@ and its variants forces each LVar to be tied to a
single ``session'', \ie, a single use of a @run@ function, in the same
way that the @ST@ monad in Haskell prevents an @STRef@ from escaping
@runST@.  Doing so allows the LVish implementation to assume that
LVars are created and used within the same session.\footnote{The
  addition of the \il{s} type parameter to \il{Par} in the LVish
  library has nothing to do with LVars in particular; it would also be
  a useful addition to the original \il{Par} library to prevent
  programmers from reusing an IVar from one \il{Par} computation to
  another, which is, as Simon Marlow has noted, ``a Very Bad Idea;
  don't do it''~\cite{marlow-book}.}

% \subsection{Non-idempotent writes: an increment-only counter}

% \TODO{I want to put an example here, but the last time I tried
%   actually using \il{Data.LVar.Counter} in LVish, it didn't work. :(
%   Need to figure this out!}

\subsection{Container LVars: a shopping cart example}\label{subsection:lvish-container-lvars}

For our next few examples, let us consider concurrently adding items
to a shopping cart.  Suppose we have an @Item@ data type for items
that can be added to the cart.  For the sake of this example, suppose
that only two items are on offer:

\singlespacing
\begin{lstlisting}
data Item = Book | Shoes
  deriving (Show, Ord, Eq)
\end{lstlisting}
\doublespacing

\noindent The cart itself can be represented using the @IMap@ LVar
type (provided by the @Data.LVar.PureMap@\footnote{The ``Pure'' in
  \il{Data.LVar.PureMap} distinguishes it from LVish's other map data
  structure, which is also called \il{IMap}, but is provided by the
  \il{Data.LVar.SLMap} module and is a lock-free data structure based
  on concurrent skip lists.  The \il{IMap} provided by
  \il{Data.LVar.PureMap}, on the other hand, is a reference
  implementation of a map, which uses a pure \il{Data.Map} wrapped in
  a mutable container.  Both \il{IMap}s present the same API, and
  either implementation of \il{IMap} would have worked for this
  example, but the lock-free version is designed to scale as parallel
  resources are added.  \either{I}{We} discuss the role of lock-free
  data structures in LVish in more detail in
  Section~\ref{subsection:lvish-parallel-speedup-results}.} module),
which is a key-value map where the keys are @Item@s and the values are
the quantities of each item.  The name \il{IMap} is by analogy with
\il{IVar}, but here, it is individual entries in the map that are
immutable, not the map itself.  If a key is inserted multiple times,
the values must be equal (according to \il{==}), or a ``multiple put''
error will be raised.

\singlespacing
\lstinputlisting{chapter4/code/map-lvar-getkey-lib.hs}
\doublespacing

\noindent Here, the @newEmptyMap@ operation creates a new @IMap@, and the
@insert@ operation allows us to add new key-value pairs to the cart.
In this case, we are concurrently adding the @Book@ item with a
quantity of @2@, and the @Shoes@ item with a quantity of @1@.  The
call to @getKey@ will be able to unblock as soon as the first @insert@
operation has completed, and the program will deterministically print
@2@ regardless of whether the second @insert@ has completed at the
time that @getKey@ unblocks.

The @getKey@ operation allows us to threshold on a key---in this case
@Book@---and get back the value associated with that key, once it has
been written.  The (implicit) threshold set of a call to @getKey@ is
the set of all values that might be associated with a key; in this
case, the set of all @Int@s.  This is a legal threshold set because
@IMap@ entries are \emph{immutable}: we cannot, for instance, insert a
key of @Book@ with a quantity of @2@ and then later change the @2@ to
@3@.  In a more realistic shopping cart, the values in the cart could
themselves be LVars representing incrementable counters, as in the
previous section.  \TODO{I'd like there to be some kind of footnote
  here about the problem that LVars-that-contain-LVars presents for
  determinism.  I don't actually understand the problem, though.}
However, a shopping cart from which we can \emph{delete} items is not
possible with LVars, because it would go against the principle of
monotonic growth.\footnote{On the other hand, one way to implement a
  container that allows both insertion and removal of elements is to
  represent it internally with \emph{two} containers, one for the
  inserted elements and one for the removed elements, where both
  containers grow monotonically.  \emph{Conflict-free replicated data
    types} (CRDTs)~\cite{crdts} use variations on this approach to
  implement various data structures that support seemingly
  non-monotonic operations.  \either{I discuss the relationship of
    LVars to CRDTs in more detail in
    Chapter~\ref{ch:distributed}.}{Bringing these ideas from the
    literature on CRDTs to the setting of LVars is a topic of ongoing
    work~\cite{joining-wodet}.}}

\subsection{A quasi-deterministic shopping cart example}

So far, the examples in this section have been fully deterministic;
they do not use @freeze@.  Next, let us consider a program that
freezes and reads the exact contents of a shopping cart, concurrently
with inserting into it.

\singlespacing
\lstinputlisting{chapter4/code/map-lvar-quasi.hs}
\doublespacing

\noindent Here, we are @insert@ing items into our cart, as in the
previous example.  But, instead of returning the result of a call to
@getKey@, this time @p@ returns the result of a call to @freezeMap@.
Note that the return type of @p@ is a @Par@ computation containing not
an @Int@, but rather an entire map from @Item@s to @Int@s.  In fact,
this map is not the @IMap@ that @Data.LVar.PureMap@ provides, but
rather the standard @Map@ from the @Data.Map@ module (imported as
@M@).  This is possible because @Data.LVar.PureMap@ is implemented
using @Data.Map@, and so freezing its @IMap@ simply returns the
underlying @Data.Map@.

Because @p@ performs a freezing operation, the effect level of its
return type must reflect the fact that it is allowed to perform
freezes.  Therefore, in addition to @HasPut@, we now have the
additional type class constraint of @HasFreeze@ on @e@.  Furthermore,
because @p@ is allowed to perform a freeze, we cannot run it with
@runPar@, as in our previous examples, but must instead use a special
variant, @runParQuasiDet@, whose type signature allows @Par@
computations that allow freezing to be passed to it.

The quasi-determinism in this program arises from the fact that the
call to @freezeMap@ may run before both @fork@ed computations have
completed.  In this example, one or both calls to @insert@ may run
after the call to @freezeMap@.  If this happens, the program will
raise a write-after-freeze exception.  The other possibility is that
both items are already in the cart at the time it is frozen, in which
case the program will run without error and print both items.  There
are therefore two possible outcomes: a cart with both items, or a
write-after-freeze error.  The advantage of quasi-determinism is that
it is not possible to get multiple \emph{non-error} outcomes, such as,
for instance, an empty cart or a cart in which only the @Book@ has
been written.

\subsection{Regaining full determinism with \il{runParThenFreeze}}\label{subsection:lvish-regaining-full-determinism-with-runparthenfreeze}

The advantage of freezing is that it allows us to observe the exact,
complete contents of an LVar; the disadvantage is that it introduces
quasi-determinism due to the possibility of a write racing with a
freeze, as in the example above.  But, if we could ensure that a
@freeze@ operation happened \emph{last}, we would be able to freeze
LVars with no risk to determinism.  In fact, the LVish library offers
a straightforward solution to this problem: instead of writing
@freeze@ operations ourselves (and perhaps accidentally writing an
undersynchronized program that freezes an LVar too early), we can ask
LVish to freeze an LVar for us itself while ``on the way out'' of a
@Par@ computation.  The mechanism that allows this is a special
@runParThenFreeze@ function.  A version of the above program written
using @runParThenFreeze@ is as follows:

\singlespacing
\lstinputlisting{chapter4/code/map-lvar-freezeafter.hs}
\doublespacing

An interesting thing to note about this @Par@ computation is that it
\emph{only} performs writes (as evidenced by its effect level, which
is only constrained by @HasPut@).  Also, unlike the previous version,
where the @freeze@ took place inside the @Par@ computation, this
computation returns an @IMap@ rather than a @Map@.

Because there is no synchronization operation after the two @fork@
calls, @p@ may well return @cart@ before both (or either) of the items
have been added to it.  However, since @runParThenFreeze@ is an
implicit global barrier (just as @runPar@ and @runParQuasiDet@ are),
both calls to @insert@ \emph{must} complete before @runParThenFreeze@
can return---which means that the result of the program is
deterministic.

\subsection{Event-driven programming: a deterministic parallel graph traversal}

Finally, we'll look at an example that uses event handlers as well as
freezing.  The function @traverse@ takes a graph @g@ and a vertex
@startNode@ and finds the set of all vertices reachable from
@startNode@, in parallel.

\singlespacing
\lstinputlisting{chapter4/code/graph-traversal-explicit-freeze.hs}
\doublespacing

@traverse@ works by first creating a new LVar of set type, called
@seen@.  Its next step is to attach an event handler to @seen@, using
the @newHandler@ function.  @newHandler@ takes two arguments: an LVar
and the callback that we want to run every time an event occurs on
that LVar (in this case, every time a new node is added to the
set).\footnote{\il{newHandler} is not provided by LVish, but we can
  easily implement it using LVish's built-in \il{newPool} and
  \il{addHandler} operations.\lk{Actually, is there any good reason
    why LVish \emph{doesn't} provide something like \il{newHandler}?}}
We respond to such events by looking up the neighbors of the newly
arrived node (with a call to the @neighbors@ operation, which takes a
graph and a vertex and returns a list of the vertex's neighbor
vertices), then mapping the @insert@ function over that list of
neighbors.

Finally, @traverse@ adds the starting node to the @seen@ set by
calling \il{insert startNode seen}---and the event handler does the
rest of the work.  We know that we are done handling events when the
call to @quiesce h@ returns; it will block until all events have been
handled.  Finally, we freeze and return the LVar, which by this point
is a set of all reachable nodes.

The good news is that this particular graph traversal is
deterministic.  The bad news is that, in general, freezing introduces
quasi-determinism, since we could have accidentally forgotten to call
@quiesce@ before the freeze---which is why @traverse@ must be run with
@runParQuasiDet@, rather than @runPar@.  Although the program is
deterministic, the \emph{language-level} guarantee is merely of
quasi-determinism, not determinism.

However, just as with the final shopping cart example above, we can
use @runParThenFreeze@ to ensure that freezing happens last.  Here is
a version of @traverse@ that is guaranteed to be deterministic at the
language level:

\singlespacing
\lstinputlisting{chapter4/code/graph-traversal-implicit-freeze.hs}
\doublespacing

Here, since freezing is performed by @runParThenFreeze@ rather by an
explicit call to @freeze@, it is no longer necessary to explicitly
call @quiesce@, either!  The reason for this is that the implicit
barrier of @runParThenFreeze@ will ensure that all outstanding events
that can be handled will be handled before it can return.
