We want the programming model of \either{Chapters}{Sections}~\ref{ch:lvars}
and~\ref{ch:quasi} to be realizable in practice.  If the determinism
guarantee offered by LVars is to do us any good, however, we need to
add LVars to a programming environment that is already deterministic.

The \emph{monad-par} Haskell library~\cite{monad-par}, which provides
the @Par@ monad, is one such deterministic parallel programming
environment.  Haskell is in general an appealing substrate for
implementing guaranteed-deterministic parallel programming models
because it is pure by default, and its type system enforces separation
of pure and effectful code via monads.  In order for the determinism
guarantee of any parallel programming model to hold, the only side
effects allowed must be those sanctioned by the programming
model.\footnote{Haskell is often advertised as a purely functional
programming language, that is, one without side effects, but it is
perhaps more useful to think of it as a language that gets other
effects out of the way so that one can add one's own effects!}  In the
case of the basic LVars model of \either{Chapter}{Section}~\ref{ch:lvars}, those effects
are @put@ and @get@ operations on LVars;
\rn{What about addHandler?}
 \either{Chapter}{Section}~\ref{ch:quasi} adds
the @freeze@ operation and arbitrary update operations.  Implementing
the LVars model as a Haskell library makes it possible to provide
compile-time guarantees about determinism and quasi-determinism,
because programs written using the library run in a particular monad
and can therefore only perform the side effects that we choose to
sanction.

Another reason why the existing @Par@ monad is an appealing starting
point for a practical implementation of LVars is that it already allows
inter-task communication through IVars, which, as we have seen, are a
special case of LVars.  % Implementing the basic LVars model thus
% becomes a matter of generalizing the existing @Par@ monad
% implementation.  
% 
%% RN: Nixing this because we didn't end up actually USING any piece of the
%% monad-par implementation, did we?  Besides, space.
%
Finally, the @Par@ monad approach is appealing
because it is implemented entirely as a library in Haskell, with a
library-level scheduler.  This modularity makes it possible to make
changes to the @Par@ scheduling strategy (which we will need to do in
order to support LVars) without having to make any modifications to
GHC or its run-time system.

\lk{Ugh, I hate roadmap paragraphs, and this is just about the worst of
all time. Maybe I can make this better.}

In this \either{chapter}{section}, \either{I}{we} describe the \emph{LVish} library, a Haskell
library for practical deterministic and quasi-deterministic parallel
programming with LVars.  We have already seen a taste of what it is
like to use the LVish library; Section~\ref{s:quasi-informal} gave an
example of an LVish Haskell program.  Here, \either{I}{we} go on to give a
systematic introduction to the LVish library.  
\rn{What about just nixing most of this?  I think the paper shouldn't be soooo
long that there's more than just one roadmap para in the intro, if there is one
at all.}
\if{0}
First, in
Section~\ref{s:lvish-big-picture}, \either{I}{we} discuss the library at a high
level.  Then, in Section~\ref{s:lvish-api}, \either{I}{we} describe the LVish API,
and \either{I}{we} discuss the parallel breadth-first traversal of
Section~\ref{s:quasi-informal} in more detail.  In
Section~\ref{s:lvish-disjoint}, \either{I}{we} discuss how LVish
supports \emph{imperative disjoint parallelism} (in the style of
DPJ~\cite{dpj-oopsla}) alongside LVars, which we accomplish using a
monad transformer, @ParST@.  In Section~\ref{s:lvish-internals}, \either{I}{we}
describe how the LVish library itself is implemented.  Finally, in
Sections~\ref{s:lvish-k-cfa} and~\ref{s:lvish-phybin}, \either{I}{we} present two
case studies of parallelizing existing Haskell programs by porting
them to LVish.  First, in Section~\ref{s:lvish-k-cfa}, \either{I}{we} describe
using LVish to parallelize a control-flow analysis ($k$-CFA)
algorithm.  Second, in Section~\ref{s:lvish-phybin}, \either{I}{we} describe using
LVish to parallelize
\emph{PhyBin}, a bioinformatics application that relies heavily on a
(parallelizable) tree-edit distance algorithm.
\fi{}