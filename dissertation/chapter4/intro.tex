We want the programming model of \either{Chapters}{Sections}~\ref{ch:lvars}
and~\ref{ch:quasi} to be realizable in practice.  If the determinism
guarantee offered by LVars is to do us any good, however, we need to
add LVars to a programming environment that is already deterministic.

The \emph{monad-par} Haskell library~\cite{monad-par}, which provides
the @Par@ monad, is one such deterministic parallel programming
environment.  Haskell is in general an appealing substrate for
implementing guaranteed-deterministic parallel programming models
because it is pure by default, and its type system enforces separation
of pure and effectful code via monads.  In order for the determinism
guarantee of any parallel programming model to hold, the only side
effects allowed must be those sanctioned by the programming
model.\footnote{Haskell is often advertised as a purely functional
programming language, that is, one without side effects, but it is
perhaps more useful to think of it as a language that keeps other
effects out of the way so that one can use only the effects that one
wants to use!}  In the case of the basic LVars model
of \either{Chapter}{Section}~\ref{ch:lvars}, those allowed effects are
@put@ and @get@ operations on
LVars; \either{Chapter}{Section}~\ref{ch:quasi} adds the @freeze@
operation and arbitrary update operations to the set of allowed
effects.  Implementing these operations as monadic effects in Haskell
makes it possible to provide compile-time guarantees about determinism
and quasi-determinism, because as long as the only monad that programs
use is the one in which LVar operations are allowed to run, we know
that they can only perform the side effects that we have chosen to
allow.

Another reason why the existing @Par@ monad is an appealing conceptual
starting point for a practical implementation of LVars is that it
already allows inter-task communication through IVars, which, as we
have seen, are a special case of LVars.  Finally, the @Par@ monad
approach is appealing because it is implemented entirely as a library
in Haskell, with a library-level scheduler.  This modularity makes it
possible to make changes to the @Par@ scheduling strategy without
having to make any modifications to GHC or its runtime system.

In this \either{chapter}{section}, \either{I}{we} describe
the \emph{LVish} library, a Haskell library for practical
deterministic and quasi-deterministic parallel programming with LVars.
We have already seen an example of an LVish Haskell program in
Section~\ref{s:quasi-informal}; in the following two sections, we will
take a more extensive tour of what LVish offers.  Then, in
Section~\ref{s:lvish-disjoint}, we will consider adding support for
DPJ-style imperative disjoint parallelism to LVish.  Finally, in
Sections~\ref{s:lvish-k-cfa} and~\ref{s:lvish-phybin}, we will look at
two case studies that illustrate how LVish and LVars can be used in
practice.
