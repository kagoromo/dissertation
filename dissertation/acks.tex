\TODO{Some kind of introductory paragraph...}

Dan Friedman invited me to be a teaching assistant for his programming languages course, known as C311 at IU, in spring 2009, which eventually led to my first research project in grad school: working on miniKanren\footnote{\url{http://www.minikanren.org} --- not that the website existed at the time!} over summer 2009 with Dan, Will Byrd, and a group of relational programming aficionados fueled by intense curiosity and Mother Bear's pizza.  I was in way over my head, but we had a great time, and I went on to help out with C311 for two more semesters under Dan and Will's expert guidance.

Amal Ahmed joined the IU faculty in 2009 and was a tremendous influence on me as I began truly learning how to read PL research papers and, eventually, write them.  Although we never worked on determinism as such, it was nevertheless from Amal that I learned how to design and prove properties of core calculi mcuh like the $\lambdaLVar$ and $\lambdaLVish$ calculi in this dissertation.  Beyond that, Amal had a huge influence on the culture of the PL group at IU.  Weekly ``PL Wonks'' meetings are a given these days, but it wasn't so very long ago that the PL group didn't have particularly regular meetings, or even much of a group identity.  It was Amal who gathered us all in a room and told us that from then on, we were all going to give talks every semester.  It was also at Amal's urging that I attended my first PL conference, ICFP 2010 in Baltimore.  I don't expect any of them to remember it, but that conference was where I first met an astonishing number of the people who have become my mentors, colleagues, and dear friends in the PL community, including Neel Krishnaswami, Chung-chieh Shan, Tim Chevalier, Rob Simmons, Jason Reed, Ron Garcia, Stevie Strickland, Dan Licata, and Chris Martens.

When Amal left IU for Northeastern in 2011, I had to make the difficult decision between going with her or staying, and although I chose to stay at IU and work with Ryan Newton on what would eventually become this dissertation, I think that her influence on the work I ended up doing is evident.  In fact, it was Amal who originally pointed out the similarity between the Independence Lemma and the frame rule that I discuss in Section~\ref{subsection:lvars-independence}.

Ryan Newton, my advisor, runs at a faster clock speed than most people.  His first act as a new faculty member at IU in fall 2011 was to arrange for our department to get a fancy espresso machine --- and to kick-start a community of student espresso enthusiasts who lovingly maintain it.  In fact, now that I think about it, I'm not sure it was a coincidence that Ryan had the espresso machine installed in the kitchen directly next to my office... Anyway, I took Ryan's seminar course that fall, which was how I started learning about deterministic parallel programming models.  In November 2011, we started discussing the idea of generalizing single-assignment models --- we got so caught up in talking about it one morning that I actually made him late for lunch with our visiting colloquium speaker, Bjarne Stroustrup --- and Ryan suggested that I could help him write an NSF grant proposal to continue working on the idea.  We somehow managed to get our proposal together as 2011 came to a close, and to my astonishment, the proposal was funded.  I gratefully acknowledge everyone at the NSF who took a chance on grant CCF-1218375 --- without that early vote of confidence, I'm not sure I would have been able to keep my spirits up during the year that followed, in which it took us four attempts to get our first LVars paper published.  (In retrospect, I am also deeply grateful to the anonymous reviewers of POPL 2013, ESOP 2013, and ICFP 2013, whose advice helped us turn LVars from a half-baked idea into a convincing research contribution.)

In fall 2012, I got in touch with Aaron Turon (whom I'd previously met in 2011 while on a visit to Northeastern) and Neel Krishnaswami.  I had become interested in separation logic as a result of working on LVars, and wanted to come up with a separation logic for deterministic parallelism --- but I couldn't do it by myself, and who better to collaborate with than Aaron and Neel?  When we all met up in Saarbr{\"u}cken in January 2013, though, Aaron and Neel wanted to first focus on making the LVars programming model more expressive.  That train of thought eventually took us all the way to event handlers, quiescence, freezing, and quasi-determinism, the topics of Chapter~\ref{ch:quasi}, and to our POPL 2014 paper.  It also led to the LVish Haskell library (Chapter~\ref{ch:lvish}), which was mostly the work of Aaron and Ryan.  In other words, this dissertation as it is today could not have existed without the contributions of Aaron and Neel.  I'm thankful also to Derek Dreyer for helping facilitate our collaboration, as well as for giving me an excuse to give a \emph{Big Lebowski}-themed talk at MPI-SWS.

Before we managed to publish anything, though, I had started to hear from people in the distributed systems community who were interested in LVars.  Pursuing the connection between LVars and distributed data consistency --- the topic of Chapter~\ref{ch:distributed} --- has been one of the most exciting and rewarding parts of this whole journey for me, largely because of all the distributed systems researchers who have met me more than halfway.  I especially want to thank the BOOM group at Berkeley, particularly Joe Hellerstein and his students Peter Alvaro, Peter Bailis, and Neil Conway; and the fine folks from Basho Technologies, particularly Sam Elliott and Chris Meiklejohn.  Speaking about LVars at RICON 2013 was one of the highlights of my Ph.D. experience; thanks to everyone at Basho who made it possible.

\TODO{Lots more people to acknowledge:
Amr Sabry;
Larry Moss;
Chung-chieh Shan;
Mozilla Research folks, especially Dave Herman;
The original Rust interns: Tim Chevalier, Michael ``Sully'' Sullivan, Paul Stansifer, Eric Holk;
Sam Tobin-Hochstadt; PL Wonks; and probably some others I'm forgetting.}

Finally, this dissertation is dedicated to my husband, Alex Rudnick, without whose unflagging love, support, advice, and encouragement I doubt I would have \emph{started} grad school, let alone finished.
