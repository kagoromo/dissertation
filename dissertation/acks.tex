Not long ago, someone asked me if knew I wanted to work on LVars at
the time I applied to grad school, and I had to laugh.  Not only did I
not know then that I wanted to work on LVars, I didn't even know what
subfield of computer science I wanted to study.  I had an idea of what
kinds of problems I thought were interesting, but I didn't really
grasp that there was actually a field of study devoted to those
problems and that ``programming languages'' was what it was called.
As it happened, one of the Ph.D. programs to which I applied turned
out to be a very good place to study PL, and that program was also the
only one that accepted me.  So the first acknowledgement I want to
make here is to the Indiana University computer science program for
taking a chance on me and making this dissertation possible.

I took Dan Friedman's programming languages course in my first
semester at IU in fall 2008, and the following spring, Dan invited me
to be a teaching assistant for the undergraduate edition of the
course, known as C311, together with his student and collaborator Will
Byrd.  This led to my first PL research experience, which consisted of
playing with a fantastic little language called
miniKanren\footnote{\url{http://www.minikanren.org} --- not that any
such website existed at the time!} during the summer of 2009 with Dan,
Will, and a group of relational programming aficionados fueled by
intense curiosity and Mother Bear's pizza.  At the time, I was utterly
unprepared to do research and was in way over my head, but we had a
lot of fun, and I went on to help out with C311 and another course,
H211, for three more semesters under Dan and Will's expert guidance.

Amal Ahmed joined the IU faculty in 2009 and was a tremendous
influence on me as I began truly learning how to read research papers
and, eventually, write them.  Although Amal and I never worked on
determinism as such, it was nevertheless from Amal that I learned how
to design and prove properties of calculi much like the $\lambdaLVar$
and $\lambdaLVish$ calculi in this dissertation.  Amal also had a huge
influence on the culture of the PL group at IU.  Weekly ``PL Wonks''
meetings are a given these days, but it wasn't so very long ago that
the PL group didn't have regular meetings or even much of a group
identity.  It was Amal who rounded us all up and told us that from
then on, we were all going to give talks every semester.  (Happily,
this process has now become entirely student-driven and
self-sustaining.)  It was also at Amal's urging that I attended my
first PL conference, ICFP 2010 in Baltimore, and got to know many of
the people who have since become my mentors, colleagues,
collaborators, and dear friends in the PL community, including Neel
Krishnaswami, Chung-chieh Shan, Tim Chevalier, Rob Simmons, Jason
Reed, Ron Garcia, Stevie Strickland, Dan Licata, and Chris Martens.

When Amal left IU for Northeastern in 2011, I had to make the
difficult decision between staying at IU or leaving, and although I
chose to stay and work with Ryan Newton on what would eventually
become this dissertation, I think that Amal's influence on the flavor
of the work I ended up doing is evident.  (In fact, it was Amal who
originally pointed out the similarity between the Independence Lemma
and the frame rule that I discuss in
Section~\ref{subsection:lvars-independence}.)

Ryan's first act as a new faculty member at Indiana in 2011 was to
arrange for our building to get a fancy espresso machine (and to
kick-start a community of espresso enthusiasts who lovingly maintain
it).  Clearly, we were going to get along well!  That fall I took
Ryan's seminar course on high-performance DSLs, which was how I
started learning about deterministic parallel programming models.
Soon, we began discussing the idea of generalizing single-assignment
languages, and Ryan suggested that I help him write a grant proposal
to continue working on the idea.  We submitted our proposal in
December 2011, and to my surprise, it was funded on our first try.  I
gratefully acknowledge everyone at the National Science Foundation who
was involved in the decision to fund grant CCF-1218375, since without
that early vote of confidence, I'm not sure I would have been able to
keep my spirits up during the year that followed, in which it took us
four attempts to get our first LVars paper published.  Ryan's non-stop
energy and enthusiasm were crucial ingredients, too.  (And, although I
didn't feel thankful at the time, I'm now also thankful to the
anonymous reviewers of POPL 2013, ESOP 2013, and ICFP 2013, whose
constructive criticism helped us turn LVars from a half-baked idea
into a convincing research contribution.)

In addition to Ryan, I was lucky to have a wonderful dissertation
committee consisting of Amr Sabry, Larry Moss, and Chung-chieh Shan.
Amr, Larry, and Ken all played important roles in the early
development of LVars and helped resolve subtle issues with the
semantics of $\lambdaLVar$ and its determinism proof.  What appears
now in Chapter~\ref{ch:lvars} is relatively simple --- but it is only
simple because of considerable effort spent making it so!  I'm also
grateful to all my committee members for showing up to my thesis
proposal at eight-thirty in the morning in the middle of an Indiana
December snowstorm.  Additionally, I want to offer special thanks to
Sam Tobin-Hochstadt, Neel Krishnaswami, and Aaron Turon --- all of
whom gave me so much guidance, encouragement, and support throughout
this project that I think of them as \emph{de facto} committee
members, too.

By fall 2012, working on LVars had gotten me interested in the idea of
developing a separation logic for deterministic parallelism.  Aaron
and Neel liked the idea, too, and expressed interest in working on it
with me.  When we all met up in Saarbr{\"u}cken in January 2013,
though, we realized that we should first focus on making the LVars
programming model more expressive, and we put the separation logic
idea on hold while we went on an exciting side trip into language
design.  That ``side trip'' eventually took us all the way to event
handlers, quiescence, freezing, and quasi-determinism, the topics of
Chapter~\ref{ch:quasi} and of our POPL 2014 paper.  It also led to the
LVish Haskell library, the topic of Chapter~\ref{ch:lvish}, which was
largely implemented by Aaron and Ryan.  Neel and I collaborated on the
original proof of quasi-determinism for $\lambdaLVish$, and Neel also
provided lots of essential guidance as I worked on the revised
versions of the proofs that appear in this dissertation.  I also want
to acknowledge Derek Dreyer for helping facilitate our collaboration,
as well as for giving me an excuse to give a \emph{Big
Lebowski}-themed talk at MPI-SWS.

One of the most exciting and rewarding parts of my dissertation work
has been pursuing the connection between LVars and distributed data
consistency, the topic of Chapter~\ref{ch:distributed}.  It can be a
challenge for people coming from different subfields to find common
ground, and I'm indebted to the many distributed systems researchers
and practitioners who have met me much more than halfway, particularly
the BOOM group at Berkeley and the Riak folks at Basho.  Speaking at
RICON 2013 about LVars was one of the highlights of my
Ph.D. experience; thanks to everyone who made it so.

Katerina Barone-Adesi, Jos\'{e} Valim, and Zach Allaun --- not
coincidentally, all members of the Recurse Center community, which is
the best programming community in the world --- gave feedback on
drafts of this dissertation and helped me improve the presentation.
Thanks to all of them.

Jason Reed contributed the wonderful spot illustrations that appear
throughout.  My hope is that they'll make the pages a bit more
inviting and offer some comic relief to the frustrated student.
They've certainly done that for me.

Finally, this dissertation is dedicated to my amazing husband, Alex
Rudnick, without whose love, support, advice, and encouragement I
would never have \emph{started} grad school, let alone finished.
