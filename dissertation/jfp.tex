\NeedsTeXFormat{LaTeX2e}

\RequirePackage{amsmath}
\documentclass{../latex_common/jfp1}
\bibliographystyle{../latex_common/jfp}

% This is the journal article, not the dissertation
\newcommand{\JOURNAL}{}
\newcommand{\either}[2]{#2}

\usepackage{color}
\usepackage{hyperref}
\usepackage{graphicx}
\usepackage{stmaryrd}
\usepackage{listings}
\usepackage{lscape}
\usepackage{setspace}
\usepackage{multirow}
\usepackage{tabularx}
\usepackage{algorithm}
\usepackage{algpseudocode}
\renewcommand{\algorithmicforall}{\textbf{for each}}

\let\oldsection\section
\let\oldsubsection\subsection
\let\oldsubsubsection\subsubsection
\renewcommand{\chapter}[1]{\oldsection{#1}}
\renewcommand{\section}[1]{\oldsubsection{#1}}
\renewcommand{\subsection}[1]{\oldsubsubsection{#1}}

% Stuff needed for typesetting lambdaLVish and its metatheory
%%% Macros for typesetting terms of the LVish language itself.

% Sans-serif font for terms of the language
\newcommand\termfont[1]{\mbox{\texttt{#1}}}

% Amount to indent the next line of a 'let' expression.
\newcommand{\letspace}{\hspace{1.1em}}
\newcommand{\letparspace}{\hspace{3.7em}}

%%% Language forms.

\newcommand{\lam}[2]{\ensuremath{\lambda#1.\,#2}}
\newcommand{\app}[2]{\ensuremath{#1~#2}}

\newcommand{\unit}{\termfont{()}}

\newcommand{\NEW}{\termfont{new}}

\newcommand{\PUT}{\termfont{put}}
\newcommand{\putexp}[2]{\ensuremath{\PUT~#1~#2}}

\newcommand{\GET}{\termfont{get}}
\newcommand{\getexp}[2]{\ensuremath{\GET~#1~#2}}

\newcommand{\BUMP}{\termfont{bump}}
\newcommand{\bumpexp}[1]{\ensuremath{\BUMP~#1}}

\newcommand{\GETFSTORSND}{\termfont{getFstOrSnd}}
\newcommand{\GETFST}{\termfont{getFst}}
\newcommand{\getfstexp}[1]{\ensuremath{\GETFST~#1}}
\newcommand{\GETSND}{\termfont{getSnd}}
\newcommand{\getsndexp}[1]{\ensuremath{\GETSND~#1}}

\newcommand{\FREEZE}{\termfont{freeze}}
\newcommand{\AFTER}{~{\termfont{after}}}
\newcommand{\WITH}{~{\termfont{with}}}

\newcommand{\ADDHANDLER}{{\termfont{addHandler}}}
\newcommand{\NEWPOOL}{{\termfont{newPool}}}
\newcommand{\ADDHANDLERINPOOL}{{\termfont{addInPool}}}
\newcommand{\QUIESCE}{{\termfont{quiesce}}}

\newcommand{\FAW}{{\termfont{freeze}-\termfont{after}-\termfont{with}}}
\newcommand{\freeze}[1]{\ensuremath{\FREEZE~#1}}
\newcommand{\freezeafter}[3]{\ensuremath{\FREEZE~#1~\AFTER~#2~\WITH~#3}}
\newcommand{\freezeafterfull}[5]{\ensuremath{\FREEZE~#1~\AFTER~#2~\WITH~#3, #4, #5}}

\newcommand{\LET}{\termfont{let}}
\newcommand{\PAR}{\termfont{par}}
\newcommand{\LETPAR}{\termfont{\LET~\PAR}}
\newcommand{\IN}{\termfont{in}}
\newcommand{\letexp}[3]{\ensuremath{\LET~#1=#2~\IN~#3}}
\newcommand{\letparexp}[5]{\ensuremath{\LETPAR~#1=#2;~#3=#4~\IN~#5}}

\newcommand{\stateset}[1]{\ensuremath{\lbrace #1 \rbrace}}

\newcommand{\UNIQUE}{\termfont{unique}}

\newcommand{\lambdalvar}{\ensuremath{\lambda_{\textsf{LVar}}}}


%%% Macros for typesetting lambdaLVar and lambdaLVish semantics and metatheory.

%%% Names of the languages.
\newcommand{\lambdaLVar}{\ensuremath{\lambda_{\textrm{LVar}}}}
\newcommand{\lambdaLVish}{\ensuremath{\lambda_{\textrm{LVish}}}}

%%% BNF grammar stuff
\newcommand{\bnfdef}{\ensuremath{::=}}
\newcommand{\bnfsep}{\ensuremath{\ \ | \ \ }}
\newcommand{\setsep}{\ensuremath{\ | \ }}
\newcommand{\sep}{\bnfsep}

%%% Metafunctions
\newcommand{\subst}[3]{\ensuremath{#1[#2 := #3]}}
\newcommand{\dom}[1]{\ensuremath{\mathit{dom}(#1)}}
\newcommand{\incomp}[1]{\ensuremath{\mathit{incomp}(#1)}}
\newcommand{\pred}[1]{\ensuremath{\mathit{pred}(#1)}}
\newcommand{\userlub}[2]{\ensuremath{#1 \sqcup #2}}
\newcommand{\lubp}[2]{\ensuremath{#1 \sqcup_p #2}}
\newcommand{\qexist}[2]{\ensuremath{\exists\,#1.~#2}}
\newcommand{\qforall}[2]{\ensuremath{\forall\,#1.~#2}}
\newcommand{\userelements}{\ensuremath{\mathit{Elements}}}
\newcommand{\userleq}{\ensuremath{\sqsubseteq}}
\newcommand{\nuserleq}{\ensuremath{\not \sqsubseteq}}
\newcommand{\usergeq}{\ensuremath{\sqsupseteq}}
\newcommand{\userlt}{\ensuremath{\sqsubset}}
\newcommand{\leqp}{\ensuremath{\sqsubseteq_p}}
\newcommand{\ltp}{\ensuremath{\sqsubset_p}}
\newcommand{\botp}{\ensuremath{\bot_{p}}}
\newcommand{\topp}{\ensuremath{\top\hspace{-0.85mm}_{p}}}

%%% Evaluation / operational semantics
\newcommand*{\longhookrightarrow}{\ensuremath{\lhook\joinrel\relbar\joinrel\rightarrow}}

%%% Reduction semantics evaluation relation
\newcommand{\parstepsto}{\ensuremath{\longhookrightarrow}}
\newcommand{\parstepstoeq}{\ensuremath{\longhookrightarrow^{?}}}

%%% Context semantics evaluation relation
\newcommand{\ctxstepsto}{\ensuremath{\longmapsto}}

\newcommand{\evalctxt}[2]{\ensuremath{#1\hspace{-0.5mm}\left[#2\right]}}
\newcommand{\E}[1]{\ensuremath{\evalctxt{E}{#1}}}

%%% Stores and store operations
\newcommand{\storeS}{\ensuremath{\mathrm{S}}}
\newcommand{\StoreVal}{\ensuremath{\mathit{StoreVal}}}
\newcommand{\fmap}{\ensuremath{\stackrel{\textrm{fin}}{\rightarrow}}}
\newcommand{\eqstore}[2]{\ensuremath{#1 =_{\storeS} #2}}
\newcommand{\leqstore}[2]{\ensuremath{#1 \userleq_{\storeS} #2}}
\newcommand{\neqstore}[2]{\ensuremath{#1 \not =_{\storeS} #2}}
\newcommand{\nleqstore}[2]{\ensuremath{#1 \not \userleq_{\storeS} #2}}
\newcommand{\ltstore}[2]{\ensuremath{#1 \userlt_{\storeS} #2}}
\newcommand{\lubstore}[2]{\ensuremath{#1 \sqcup_{\storeS} #2}}
\newcommand{\glbstore}[2]{\ensuremath{#1 \sqcap_{\storeS} #2}}
\newcommand{\storebindingRaw}[2]{\ensuremath{#1 \mapsto #2}}
\newcommand{\storebinding}[3]{\storebindingRaw{#1}{\state{#2}{#3}}}
\newcommand{\state}[2]{\ensuremath{(#1, #2)}}
\newcommand{\store}[1]{\left[ #1 \right]}
\newcommand{\extS}[4]{#1[\storebinding{#2}{#3}{#4}]}
\newcommand{\extSRaw}[3]{#1[\storebindingRaw{#2}{#3}]}
\newcommand{\storeset}{\mathcal{S}}
\newcommand{\status}{\ensuremath{\mathit{frz}}}
\newcommand{\frozentrue}{\ensuremath{\textsf{true}}}
\newcommand{\frozenfalse}{\ensuremath{\textsf{false}}}
\newcommand{\Loc}{\mathit{Loc}}
\newcommand{\topS}{\ensuremath{\top\hspace{-0.85mm}_{\storeS}}}
\newcommand{\LVar}{\mathit{LVar}}

%% It would be really nice to be able to typeset stores like

%%        \store{l_1, l_2, l_3}{3, 4, 5}

%% producing

%%        {l_1 -> 3, l_2 -> 4, l_3 -> 5}

%% but to do that, I think we need something like what's done in
%% http://stackoverflow.com/questions/2402354/split-comma-separated-parameters-in-latex,
%% and I haven't figured out how to do it just yet :(

%%% States and configurations
\newcommand{\conf}{\ensuremath{\sigma}}
\newcommand{\config}[2]{\ensuremath{\langle #1;\, #2 \rangle}}
\newcommand{\error}{\ensuremath{\textbf{\textsf{error}}}}

\newcommand{\handledBy}{\hookrightarrow}

%%% Assorted math stuff
\newcommand{\defeq}{\stackrel{\triangle}{=}} % "defined-as"
\newcommand{\pfn}{\rightharpoonup}
\newcommand{\bag}[1]{\Lbag #1 \Rbag}
\newcommand{\setof}[1]{\left\{#1\right\}}
\newcommand{\comprehend}[2]{\setof{{#1}\;\middle|\;{#2}}}
\newcommand{\power}[1]{\mathcal{P}(#1)}
\newcommand{\inlop}{\mathsf{inl}}
\newcommand{\inl}[1]{\inlop\,{#1}}
\newcommand{\inrop}{\mathsf{inr}}
\newcommand{\inr}[1]{\inrop\,{#1}}
\newcommand{\letv}[2]{\mathsf{let}\;{#1} = {#2}}
\newcommand{\Freeze}[1]{\mathrm{Freeze}(#1)}
\newcommand{\piinv}{\pi^{-1}}
\newcommand{\piprimeinv}{\pi'^{-1}}
\newcommand{\andlv}[2]{\ensuremath{(\texttt{#1},\texttt{#2})}}
\newcommand{\id}{\mathrm{id}}
\newcommand{\eventually}{\lozenge}
\newcommand{\block}{\textsf{block}}


% For typesetting inference rules
\usepackage{../latex_common/mathpartir}

% Syntax and semantics of lambdaLVar and lambdaLVish
%%% The syntax and semantics of lambdaLVar.

\newcommand{\FigLambdaLVarGrammar}[1][t]{
\begin{figure}[#1]
  Given a lattice $(D, \userleq, \bot, \top)$ with elements $d \in D$:
    \[
    \begin{array}{rlcl}
      \mbox{configurations} & \conf & \bnfdef & \config{S}{e} \sep \error \\
      \mbox{expressions} & e & \bnfdef & 
           x \sep 
           v \sep 
           \app{e}{e} \sep 
           \getexp{e}{e} \sep 
           \putexp{e}{e} \sep
           \NEW \\
      % N.B. unit and d are part of the set of values because (unlike
      % in the FHPC paper) get and put don't take and return sets.
      % Before, put returned an empty set (we overloaded the idea of
      % an empty threshold set) and get returned a singleton set.
      % This was to keep the grammar clean, but it's silly; it makes
      % more sense to just have two more value forms.
      \mbox{values} & v & \bnfdef & \unit \sep d \sep l \sep T \sep \lam{x}{e} \\
      \mbox{threshold sets} & T & \bnfdef & \stateset{d_1,\,d_2,\,\dots} \\
      % N.B. In Redex we actually rule out store values being Top in
      % the grammar, and have a special StoreVal type for elements
      % other than Top.  Here we don't bother, and we just say that
      % stores contain bindings from locations l to pairs p.
      \mbox{stores} & S & \bnfdef &
        \store{\storebindingRaw{l_1}{d_1},\,\dots, \storebindingRaw{l_n}{d_n}} \sep \topS \\
      \mbox{evaluation contexts} & E & \bnfdef &
           [~] \sep
           \app{E}{e} \sep
           \app{e}{E} \sep
           \getexp{E}{e} \sep
           \getexp{e}{E} \sep 
           \putexp{E}{e} \sep
           \putexp{e}{E}
    \end{array}
    \]
  \caption{Syntax for $\lambdaLVar$.}
  \label{f:lvars-lambdaLVar-syntax}
\end{figure}
}

\newcommand{\FigLambdaLVarSemantics}[1][t]{
\begin{figure*}[#1]
  Given a lattice $(D, \userleq, \bot, \top)$ with elements $d \in D$:
  \bigskip
  $\incomp{T} \defeq \qforall{d_1,d_2 \in T}{(d_1 \neq d_2 \implies \userlub{d_1}{d_2}
    = \top)}$\hfill \fbox{$\conf \parstepsto \conf'$}
  \bigskip
  \begin{mathpar}
    \inferrule*[lab=E-Eval-Ctxt]
        {\config{S}{e} \parstepsto \config{S'}{e'}}
        {\config{S}{\E{e}} \parstepsto \config{S'}{\E{e'}}}

    \inferrule*[lab=E-Beta]
        {~}
        {\config{S}{\app{(\lam{x}{e})}{v}} \parstepsto \config{S}{\subst{e}{x}{v}}}

    \inferrule*[lab=E-New, right=\textnormal{($l \notin \dom{S}$)}]
        {~}
        {\config{S}{\NEW} \parstepsto \config{\extSRaw{S}{l}{\bot}}{l}}

    \inferrule*[lab=E-Put]
        {S(l) = d_1 \\ \userlub{d_1}{d_2} \neq \top}
        {\config{S}{\putexp{l}{d_2}} \parstepsto
          \config{\extSRaw{S}{l}{\userlub{d_1}{d_2}}}{\unit}}

    \inferrule*[lab=E-Put-Err]
        {S(l) = d_1 \\ \userlub{d_1}{d_2} = \top}
        {\config{S}{\putexp{l}{d_2}} \parstepsto 
          \error}

    \inferrule*[lab=E-Get]
        {S(l) = d_1 \\ \incomp{T} \\ d_2 \in T \\ d_2 \userleq d_1}
        {\config{S}{\getexp{l}{T}} \parstepsto \config{S}{d_2}}
  \end{mathpar}
  \caption{An operational semantics for $\lambdaLVar$.}
  \label{f:lvars-lambdaLVar-semantics}
\end{figure*}
}

%%% The syntax and semantics of lambdaLVish.

\newcommand{\FigLambdaLVishGrammar}[1][t]{
\begin{figure}[#1]
  Given a lattice $(D, \userleq, \bot, \top)$ with elements $d \in D$:
    \[
    \begin{array}{rlcl}
      \mbox{configurations} & \conf & \bnfdef & \config{S}{e} \sep \error \\
      \mbox{expressions} & e & \bnfdef & 
           x \sep 
           v \sep 
           \app{e}{e} \sep 
           \getexp{e}{e} \sep 
           \putiexp{e} \sep
           \NEW \sep
           \freeze{e} \\
           & & \sep &
           \freezeafter{e}{e}{e} \\
           & & \sep &
           \freezeafterfull{l}{Q}{\lam{x}{e}}{\setof{e, \dots}}{H} \\
      \mbox{values} & v & \bnfdef & \unit \sep d \sep p \sep l \sep P \sep Q \sep \lam{x}{e} \\
      \mbox{threshold sets} & P & \bnfdef & \stateset{p_1,\,p_2,\,\dots} \\
      \mbox{event sets} & Q & \bnfdef & \stateset{d_1,\,d_2,\,\dots} \\
      \mbox{``handled'' sets} & H & \bnfdef & \setof{d_1,\,\dots, d_n} \\
      %% N.B. In Redex we actually rule out store values being Top in
      %% the grammar, and have a special StoreVal type for elements
      %% other than Top.  Here we don't bother, and we just say that
      %% stores contain bindings from locations l to pairs p.
      \mbox{stores} & S & \bnfdef &
        \store{\storebindingRaw{l_1}{p_1},\,\dots, \storebindingRaw{l_n}{p_n}} \sep \topS \\
      \mbox{states} & p & \bnfdef & \state{d}{\status} \\
      \mbox{status bits} & \status & \bnfdef & \frozentrue \sep \frozenfalse \\
      \mbox{evaluation contexts} & E & \bnfdef &
           [~] \sep
           \app{E}{e} \sep
           \app{e}{E} \sep
           \getexp{E}{e} \sep
           \getexp{e}{E} \sep 
           \putiexp{E} \\
           & & \sep &
           \freeze {E} \sep
           \freezeafter{E}{e}{e} \\
           & & \sep &
           \freezeafter{e}{E}{e} \sep
           \freezeafter{e}{e}{E} \\
           & & \sep &
           \freezeafterfull{v}{v}{v}{\setof{e\dots~E~e\dots}}{H}
    \end{array}           
    \]
  \caption{Syntax for $\lambdaLVish$.}
  \label{f:lambdaLVish-syntax}
\end{figure}
}

\newcommand{\FigLambdaLVishReductionSemantics}[1][t]{
\begin{landscape}
\begin{figure*}[#1]
  Given a lattice $(D, \userleq, \bot, \top)$ with elements $d \in D$, \\
  and a set of $U$ of update operations $u_i: D \rightarrow D$: \\
  \bigskip
  $\incomp{P} \defeq \qforall{p_1,p_2 \in P}{(p_1 \neq p_2 \implies
    \lubp{p_1}{p_2} = \topp)}$\hfill \fbox{$\conf \parstepsto \conf'$}
  \\
  \begin{mathpar}
    \inferrule*[lab=E-Beta]
        {~}
        {\config{S}{\app{(\lam{x}{e})}{v}} \parstepsto \config{S}{\subst{e}{x}{v}}}

    \inferrule*[lab=E-New, right=\textnormal{($l \notin \dom{S}$)}]
        {~}
        {\config{S}{\NEW} \parstepsto \config{\extS{S}{l}{\bot}{\frozenfalse}}{l}}

    \inferrule*[lab=E-Put]
        {S(l) = p_1 \\ u_{p_i}(p_1) \neq \topp}
        {\config{S}{\putiexp{l}} \parstepsto
          \config{\extSRaw{S}{l}{u_{p_i}(p_1)}}{\unit}}

    \inferrule*[lab=E-Put-Err]
        {S(l) = p_1 \\ u_{p_i}(p_1) = \topp}
        {\config{S}{\putiexp{l}} \parstepsto \error}

    \inferrule*[lab=E-Get]
        {S(l) = p_1 \\ \incomp{P} \\ p_2 \in P \\ p_2 \leqp p_1}
        {\config{S}{\getexp{l}{P}} \parstepsto \config{S}{p_2}}

    \inferrule*[lab=E-Freeze-Init]
        {~}
        {\config{S}{\freezeafter{l}{Q}{\lam{x}{e}}} \parstepsto
          \config{S}{\freezeafterfull{l}{Q}{\lam{x}{e}}{\setof{}}{\setof{}}}}

    \inferrule*[lab=E-Spawn-Handler]
        { S(l) = \state{d_1}{\status_1} \\ 
          d_2 \userleq d_1 \\
          d_2 \notin H \\
          d_2 \in Q
        }
        {
          \config{S}{\freezeafterfull{l}{Q}{\lam{x}{e_0}}{\setof{e, \dots}}{H}}
          \parstepsto
          \config{S}{\freezeafterfull{l}{Q}{\lam{x}{e_0}}{\setof{\subst{e_0}{x}{d_2}, e, \dots}}
            {\{d_2\}\cup H}}
        }

    \inferrule*[lab=E-Freeze-Final]
        { S(l) = \state{d_1}{\status_1} \\ 
          \forall{d_2} ~.~ ( {d_2 \userleq d_1 \land d_2 \in Q} \Rightarrow 
             d_2 \in H) }
        {
          \config{S}{\freezeafterfull{l}{Q}{v}{\setof{v\dots}}{H}}
          \parstepsto
          \config{\extS{S}{l}{d_1}{\frozentrue}}{d_1}
        }

    \inferrule*[lab=E-Freeze-Simple]
        { S(l) = \state{d_1}{\status_1} }
        {
          \config{S}{\freeze{l}}
          \parstepsto
          \config{\extS{S}{l}{d_1}{\frozentrue}}{d_1}
        }
  \end{mathpar}
  \caption{Reduction semantics for $\lambdaLVish$.}
  \label{f:lambdaLVish-reduction-semantics}
\end{figure*}
\end{landscape}
}

\newcommand{\FigLambdaLVishContextSemantics}[1][t]{
\begin{figure*}[#1]
  \hfill \fbox{$\conf \ctxstepsto \conf'$}
  \begin{mathpar}
    \inferrule*[lab=E-Eval-Ctxt]
        {\config{S}{e} \parstepsto \config{S'}{e'}}
        {\config{S}{\E{e}} \ctxstepsto \config{S'}{\E{e'}}}
  \end{mathpar}
  \caption{Context semantics for $\lambdaLVish$.}
  \label{f:lambdaLVish-context-semantics}
\end{figure*}
}


% Definitions, lemmas, theorems
\newcommand{\LVarsDefEqStore}{
\begin{definition}\label{def:lvars-eqstore}
  Two stores $S$ and $S'$ are \emph{equal} iff:
  \begin{enumerate}
    \item $S = \topS$ and $S' = \topS$, or
    \item $\dom{S} = \dom{S'}$ and for all $l \in \dom{S}$, $S(l) =
      S'(l)$.
  \end{enumerate}
\end{definition}
}

\newcommand{\LVarsDefLeqStore}{
\begin{definition}\label{def:lvars-leqstore}
  A store $S$ is \emph{less than or equal to} a store $S'$ (written
  $\leqstore{S}{S'}$) iff:
  \begin{itemize}
  \item $S' = \topS$, or
  \item $\dom{S} \subseteq \dom{S'}$ and for all $l
    \in \dom{S}$, $S(l) \userleq S'(l)$.
  \end{itemize}
\end{definition}
}

\newcommand{\LVarsDefLeqStoreCnC}{
\begin{definition}[store ordering, Featherweight CnC]\label{def:lvars-leqstore-cnc}
  A store $S$ is \emph{less than or equal to} a store $S'$ (written
  $\leqstore{S}{S'}$) iff $\dom{S} \subseteq \dom{S'}$ and for all $l
  \in \dom{S}$, $S(l) = S'(l)$.
\end{definition}
}

\newcommand{\LVarsDefLubStore}{
\begin{definition}\label{def:lvars-lubstore}

  The \emph{least upper bound (lub)} of two stores $S_1$ and $S_2$ (written
  $\lubstore{S_1}{S_2}$) is defined as follows:
  
  \begin{itemize}
  \item $\lubstore{S_1}{S_2} = \topS$ iff there exists some $l \in
    \dom{S_1} \cap \dom{S_2}$ such that $\userlub{S_1(l)}{S_2(l)} = \top$.
  \item Otherwise, $\lubstore{S_1}{S_2}$ is the store $S$ such that:

  \begin{itemize}
  \item $\dom{S} = \dom{S_1} \cup \dom{S_2}$, and
  \item For all $l \in \dom{S}$:
  \end{itemize}
    \begin{displaymath}
      S(l) = \left\{ \begin{array}{ll}
        \userlub{S_1(l)}{S_2(l)} & \textrm{if $l \in \dom{S_1} \cap \dom{S_2}$} \\
        S_1(l) & \textrm{if $l \notin \dom{S_2}$} \\
        S_2(l) & \textrm{if $l \notin \dom{S_1}$}
        \end{array} \right.
    \end{displaymath}
  \end{itemize}
\end{definition}
}

\newcommand{\LVarsDefPermutation}{
\begin{definition}\label{def:lvars-permutation}
  A \emph{permutation} is a function $\pi : \Loc \rightarrow \Loc$
  such that:
  \begin{enumerate}
  \item it is invertible, that is, there is an inverse function
    $\piinv : \Loc \rightarrow \Loc$ with the property that $\pi(l) =
    l'$ iff $\piinv(l') = l$; and
    \item it is the identity on all but finitely many elements of
      $Loc$.
  \end{enumerate}
\end{definition}
}

\newcommand{\LVarsDefPermutationExpression}{
\begin{definition}\label{def:lvars-permutation-expression}
  A permutation of an expression $e$ is a function $\pi$ defined as
  follows:
  \begin{displaymath}
    \begin{array}{ll}
      \pi(x) &= x \\
      \pi(\unit) &= \unit \\
      \pi(d) &= d \\
      \pi(l) &= \pi(l) \\
      \pi(T) &= T \\
      \pi(\lam{x}{e}) &= \lam{x}{\pi(e)} \\
      \pi(\app{e_1}{e_2}) &= \app{\pi(e_1)}{\pi(e_2)} \\
      \pi(\getexp{e_1}{e_2}) &= \getexp{\pi(e_1)}{\pi(e_2)} \\
      \pi(\putexp{e_1}{e_2}) &= \putexp{\pi(e_1)}{\pi(e_2)} \\
      \pi(\NEW) &= \NEW \\
    \end{array}
  \end{displaymath}
  \lk{It's fine to just say that $\pi(\NEW)$ is $\NEW$; we only care
    about renaming it if it has already been allocated!  If it's just
    an unevaluated $\NEW$ expression, then there's nothing to do.}
\end{definition}
}

\newcommand{\LVarsDefPermutationStore}{
\begin{definition}\label{def:lvars-permutation-store}
  A permutation of an evaluation $S$ is a function $\pi$ defined as
  follows:
  \begin{displaymath}
    \begin{array}{ll}
      \pi(\topS) &= \topS \\
      \pi(\store{\storebindingRaw{l_1}{d_1}, \dots,
        \storebindingRaw{l_n}{d_n}}) &=
      \store{\storebindingRaw{\pi(l_1)}{d_1}, \dots,
        \storebindingRaw{\pi(l_n)}{d_n}} \\
    \end{array}
  \end{displaymath}
\end{definition}
}

\newcommand{\LVarsDefPermutationConfiguration}{
\begin{definition}\label{def:lvars-permutation-configuration}
  A permutation of a configuration $\config{S}{e}$ is a function $\pi$
  defined as follows: if $\config{S}{e} = \error$, then
  $\pi(\config{S}{e}) = \error$; otherwise, $\pi(\config{S}{e}) =
  \config{\pi(S)}{\pi(e)}$.
\end{definition}
}

\newcommand{\LVarsDefNonConflicting}{
\begin{definition}\label{def:lvars-non-conflicting}
  A store $S''$ is \emph{non-conflicting} with the transition
  $\config{S}{e} \parstepsto \config{S'}{e'}$ iff $(\dom{S'} -
  \dom{S}) \cap \dom{S''} = \emptyset$.
\end{definition}
}

\newcommand{\LVarsLemPermutability}{
\begin{lem}[Permutability]\label{lem:lvars-permutability}
  For any finite permutation $\pi$,
  \begin{enumerate}
  \item \label{thm:permutable-reduction-transitions} $\conf
    \parstepsto \conf'$ if and only if $\pi(\conf) \parstepsto
    \pi(\conf')$.
  \item \label{thm:permutable-context-transitions} $\conf \ctxstepsto
    \conf'$ if and only if $\pi(\conf) \ctxstepsto \pi(\conf')$.
  \end{enumerate}
\end{lem}
}

\newcommand{\LVarsLemInternalDeterminism}{
\begin{lem}[Internal Determinism of the Reduction Semantics]\label{lem:lvars-internal-determinism}
  If $\conf \parstepsto \conf'$ and $\conf \parstepsto \conf''$, then
  there is a permutation $\pi$ such that $\conf' = \pi(\conf'')$.
\end{lem}
}

\newcommand{\LVarsLemLocality}{
\begin{lem}[Locality]\label{lem:lvars-locality}
  If $\config{S}{\evalctxt{E_1}{e_1}} \ctxstepsto
  \config{S_1}{\evalctxt{E_1}{e'_1}}$ and
  $\config{S}{\evalctxt{E_2}{e_2}} \ctxstepsto
  \config{S_2}{\evalctxt{E_2}{e'_2}}$ and $\evalctxt{E_1}{e_1} =
  \evalctxt{E_2}{e_2}$, then there exist evaluation contexts $E'_1$
  and $E'_2$ such that:
  \begin{itemize}
  \item $\evalctxt{E'_1}{e_1} = \evalctxt{E_2}{e'_2}$, and
  \item $\evalctxt{E'_2}{e_2} = \evalctxt{E_1}{e'_1}$, and
  \item $\evalctxt{E'_1}{e'_1} = \evalctxt{E'_2}{e'_2}$.
  \end{itemize}
\end{lem}
}

\newcommand{\LVarsLemMonotonicity}{
\begin{lem}[Monotonicity]\label{lem:lvars-monotonicity}
  If $\config{S}{e} \parstepsto \config{S'}{e'}$,
  then $\leqstore{S}{S'}$.
\end{lem}
}

\newcommand{\LVarsLemIndependence}{
\begin{lem}[Independence]\label{lem:lvars-independence}
  If $\config{S}{e} \parstepsto \config{S'}{e'}$ (where
  $\config{S'}{e'} \neq \textup{\error}$), then for all $S''$ such
  that $S''$ is non-conflicting with $\config{S}{e} \parstepsto
  \config{S'}{e'}$ and $\lubstore{S'}{S''} \neq \topS$:

  $\config{\lubstore{S}{S''}}{e} \parstepsto \config{\lubstore{S'}{S''}}{e'}$.
\end{lem}
}

\newcommand{\LVarsLemClash}{
\begin{lem}[Clash]\label{lem:lvars-clash}
  If $\config{S}{e} \parstepsto \config{S'}{e'}$ (where
  $\config{S'}{e'} \neq \textup{\error}$), then for all $S''$ such
  that $S''$ is non-conflicting with $\config{S}{e} \parstepsto
  \config{S'}{e'}$ and $\lubstore{S'}{S''} = \topS$:

  $\config{\lubstore{S}{S''}}{e} \parstepsto^i \textup{\error}$, where
  $i \leq 1$.
\end{lem}
}

\newcommand{\LemStorePartition}{
\begin{lem}[Store Partition]\label{lem:store-partition}
  Let $S$ and $S'$ be stores.  Then, for all $l_i$ such that $S(l_i)
  \mylt S'(l_i)$, there exists $S''$ such that
  $\eqstore{S'}{\extS{S''}{l_i}{S'(l_i)}}$ and $\myltstore{S''}{S'}$.
\end{lem}
}

\newcommand{\LVarsLemErrorPreservation}{
\begin{lem}[Error Preservation]\label{lem:lvars-error-preservation}
  If $\config{S}{e} \parstepsto \textup{\error}$ and $\leqstore{S}{S'}$, then
  $\config{S'}{e} \parstepsto \textup{\error}$.

\end{lem}
}

\newcommand{\LVarsLemStrongLocalConfluence}{
\begin{lem}[Strong Local Confluence]\label{lem:lvars-strong-local-confluence}
  If $\conf \ctxstepsto \conf_a$ and $\conf \ctxstepsto \conf_b$, then
  there exist $\conf_c, i, j, \pi$ such that $\conf_a \ctxstepsto^i
  \conf_c$ and $\pi(\conf_b) \ctxstepsto^j \pi(\conf_c)$ and $i \leq
  1$ and $j \leq 1$.
\end{lem}
}

\newcommand{\LVarsLemStrongOneSidedConfluence}{
\begin{lem}[Strong One-Sided Confluence]\label{lem:lvars-strong-one-sided-confluence}
  If $\conf \parstepsto \conf'$ and $\conf \parstepsto^m \conf''$,
  where $1 \leq m$, then there exist $\conf_c, i, j$ such that $\conf'
  \parstepsto^i \conf_c$ and $\conf'' \parstepsto^j \conf_c$ and $i
  \leq m$ and $j \leq 1$.
  \TODO{If the statement of
    Lemma~\ref{lem:lvars-strong-local-confluence} changes to mention
    permutations, then figure out how the statement of this one has to
    change as well.}
\end{lem}
}

\newcommand{\LVarsLemStrongConfluence}{
\begin{lem}[Strong Confluence]\label{lem:lvars-strong-confluence}
  If $\conf \parstepsto^n \conf'$ and $\conf \parstepsto^m \conf''$,
  where $1 \leq n$ and $1 \leq m$, then there exist $\conf_c, i, j$
  such that $\conf' \parstepsto^i \conf_c$ and $\conf'' \parstepsto^j
  \conf_c$ and $i \leq m$ and $j \leq n$.
  \TODO{If the statement of
    Lemma~\ref{lem:lvars-strong-local-confluence} changes to mention
    permutations, then figure out how the statement of this one has to
    change as well.}
\end{lem}
}

\newcommand{\LVarsLemConfluence}{
\begin{lem}[Confluence]\label{lem:lvars-confluence}
  If $\conf \parstepsto^* \conf'$ and $\conf \parstepsto^* \conf''$,
  then there exists $\conf_c$ such that $\conf' \parstepsto^* \conf_c$
  and $\conf'' \parstepsto^* \conf_c$.
  \TODO{If the statement of
    Lemma~\ref{lem:lvars-strong-local-confluence} changes to mention
    permutations, then figure out how the statement of this one has to
    change as well.}
\end{lem}
\begin{proof}
  Strong Confluence (Lemma~\ref{lem:lvars-strong-confluence}) implies
  Confluence.
\end{proof}
}

\newcommand{\LVarsThmDeterminism}{
\begin{thm}[Determinism]\label{thm:lvars-determinism}
  If $\conf \parstepsto^* \conf'$ and $\conf \parstepsto^* \conf''$,
  and neither $\conf'$ nor $\conf''$ can take a step, then $\conf' =
  \conf''$.
  \TODO{If the statement of
    Lemma~\ref{lem:lvars-strong-local-confluence} changes to mention
    permutations, then figure out how the statement (and proof) of
    this one has to change as well.}
\end{thm}
\begin{proof}
  We have from Lemma~\ref{lem:lvars-confluence} that there exists
  $\conf_c$ such that $\conf' \parstepsto^* \conf_c$ and $\conf''
  \parstepsto^* \conf_c$.  Since neither $\conf'$ nor $\conf''$ can
  take a step, we must have $\conf' = \conf_c$ and $\conf'' =
  \conf_c$, hence $\conf' = \conf''$.
\end{proof}
}


\newcommand{\DefStability}{
  \begin{definition}[Stable Configurations]
    A configuration $\config{S}{e}$ is \emph{stable} if the free locations of $e$ are a subset of $\dom{S}$. 
  \end{definition}
}

\newcommand{\DefLatticeWithStatusBits}{
\begin{definition}\label{def:lattice-with-status-bits}
Suppose $(D, \userleq, \bot, \top)$ is a lattice.  We define an
operation $\Freeze{D, \userleq, \bot, \top} \defeq (D_p, \leqp, \botp,
\topp)$ as follows:
\begin{enumerate}
\item $D_p$ is a set defined as follows:
\begin{equation*}
\begin{split}
  D_p  \defeq & \setof{ \state{d}{\status} \sep d \in (D - \setof{\top}) \land \status
    \in \setof{\frozentrue, \frozenfalse} } \\
  & \cup \setof{(\top, \frozenfalse)}
\end{split}
\end{equation*}

\item $\leqp \;\in\, \power{D_p \times D_p}$ is a binary relation
  defined as follows:
\begin{displaymath}
  \begin{array}{lclcl}
    \state{d}{\frozenfalse} & \leqp & \state{d'}{\frozenfalse} & \iff & d \userleq d' \\ 
    \state{d}{\frozentrue} & \leqp & \state{d'}{\frozentrue} & \iff & d = d' \\
    \state{d}{\frozenfalse} & \leqp & \state{d'}{\frozentrue}  & \iff & d \userleq d' \\
    \state{d}{\frozentrue} & \leqp & \state{d'}{\frozenfalse}  & \iff & d' = \top 
  \end{array}
\end{displaymath}

\item $\botp \defeq \state{\bot}{\frozenfalse}$.

\item $\topp \defeq \state{\top}{\frozenfalse}$.
\end{enumerate}
\end{definition}
}

\newcommand{\DefLubP}{
\begin{definition}\label{def:lubp}
We define a binary operator $\lubp{}{} \in D_p \times D_p \to D_p$ as
follows:
  \begin{displaymath}
    \begin{array}{lcl}
      \lubp{\state{d_1}{\frozenfalse}}{\state{d_2}{\frozenfalse}} & = & \state{\userlub{d_1}{d_2}}{\frozenfalse} \\ 
      \lubp{\state{d_1}{\frozentrue}}{\state{d_2}{\frozentrue}} & = & \left\{
                                       \begin{array}{ll}
                                         \state{d_1}{\frozentrue} & \mbox{if } d_1 = d_2 \\ 
                                         \state{\top}{\frozenfalse} & \mbox{otherwise}
                                       \end{array}
                                     \right. \\
      \lubp{\state{d_1}{\frozenfalse}}{\state{d_2}{\frozentrue}}    & = & \left\{
                                       \begin{array}{ll}
                                         \state{d_2}{\frozentrue} & \mbox{if } d_1 \userleq d_2 \\ 
                                         \state{\top}{\frozenfalse} & \mbox{otherwise}
                                       \end{array}
                                     \right. \\
      \lubp{\state{d_1}{\frozentrue}}{\state{d_2}{\frozenfalse}}    & = & \left\{
                                       \begin{array}{ll}
                                         \state{d_1}{\frozentrue} & \mbox{if } d_2 \userleq d_1 \\ 
                                         \state{\top}{\frozenfalse} & \mbox{otherwise}
                                       \end{array}
                                     \right. \\
    \end{array}
  \end{displaymath}
\end{definition}
}


\newcommand{\DefStore}{
\begin{definition}\label{def:store}
A \emph{store} is either a finite partial mapping $S : \Loc \fmap (D_p
- \setof{\topp})$, or the distinguished element $\topS$.
\end{definition}
}

\newcommand{\DefEqStore}{
\begin{definition}\label{def:eqstore}
  Two stores $S$ and $S'$ are \emph{equal} iff:
  \begin{enumerate}
    \item $S = \topS$ and $S' = \topS$, or
    \item $\dom{S} = \dom{S'}$ and for all $l \in \dom{S}$, $S(l) =
      S'(l)$.
  \end{enumerate}
\end{definition}
}

\newcommand{\DefLeqStore}{
\begin{definition}\label{def:leqstore}
  A store $S$ is \emph{less than or equal to} a store $S'$ (written
  $\leqstore{S}{S'}$) iff:
  \begin{itemize}
  \item $S' = \topS$, or
  \item $\dom{S} \subseteq \dom{S'}$ and for all $l
    \in \dom{S}$, $S(l) \leqp S'(l)$.
  \end{itemize}
\end{definition}
}

\newcommand{\DefLubStore}{
\begin{definition}\label{def:lubstore}

  The \emph{least upper bound (lub)} of two stores $S_1$ and $S_2$ (written
  $\lubstore{S_1}{S_2}$) is defined as follows:

  \begin{itemize}
  \item $\lubstore{S_1}{S_2} = \topS$ iff $S_1 = \topS$ or $S_2 = \topS$.
  \item $\lubstore{S_1}{S_2} = \topS$ iff there exists some $l \in
    \dom{S_1} \cap \dom{S_2}$ such that $\lubp{S_1(l)}{S_2(l)} = \topp$.
  \item Otherwise, $\lubstore{S_1}{S_2}$ is the store $S$ such that:

  \begin{itemize}
  \item $\dom{S} = \dom{S_1} \cup \dom{S_2}$, and
  \item For all $l \in \dom{S}$:
  \end{itemize}
    \begin{displaymath}
      S(l) = \left\{ \begin{array}{ll}
        \lubp{S_1(l)}{S_2(l)} & \textrm{if $l \in \dom{S_1} \cap \dom{S_2}$} \\
        S_1(l) & \textrm{if $l \notin \dom{S_2}$} \\
        S_2(l) & \textrm{if $l \notin \dom{S_1}$}
        \end{array} \right.
    \end{displaymath}
  \end{itemize}
\end{definition}
}

\newcommand{\DefPermutation}{
\begin{definition}[permutation]\label{def:lvars-permutation}
  A \emph{permutation} is a function $\pi : \Loc \rightarrow \Loc$
  such that:
  \begin{enumerate}
  \item it is invertible, that is, there is an inverse function
    $\piinv : \Loc \rightarrow \Loc$ with the property that $\pi(l) =
    l'$ iff $\piinv(l') = l$; and
    \item it is the identity on all but finitely many elements of
      $Loc$.
  \end{enumerate}
\end{definition}
}

\newcommand{\DefPermutationExpression}{
\begin{definition}[permutation (expressions)]\label{def:lvars-permutation-expression}
  A \emph{permutation} of an expression $e$ is a function $\pi$ defined as
  follows:
  \begin{displaymath}
    \begin{array}{ll}
      \pi(x) &= x \\
      \pi(\unit) &= \unit \\
      \pi(d) &= d \\
      \pi(p) &= p \\
      \pi(l) &= \pi(l) \\
      \pi(P) &= P \\
      \pi(Q) &= Q \\
      \pi(\lam{x}{e}) &= \lam{x}{\pi(e)} \\
      \pi(\app{e_1}{e_2}) &= \app{\pi(e_1)}{\pi(e_2)} \\
      \pi(\getexp{e_1}{e_2}) &= \getexp{\pi(e_1)}{\pi(e_2)} \\
      \pi(\putiexp{e}) &= \putiexp{\pi(e)} \\
      \pi(\NEW) &= \NEW \\
      \pi(\freeze{e}) &= \freeze{\pi(e)} \\
      \pi(\freezeafter{e_1}{e_2}{e_3}) &= \freezeafter{\pi(e_1)}{\pi(e_2)}{\pi(e_3)} \\
      \pi(\freezeafterfull{l}{Q}{\lam{x}{e}}{\setof{e, \dots}{H}}) &= \freezeafterfull{\pi(l)}{Q}{\lam{x}{\pi(e)}}{\setof{\pi(e), \dots}}{H} \\
    \end{array}
  \end{displaymath}
\end{definition}
}

\newcommand{\DefPermutationStore}{
\begin{definition}[permutation (stores)]\label{def:lvars-permutation-store}
  A \emph{permutation} of a store $S$ is a function $\pi$ defined as
  follows:
  \begin{displaymath}
    \begin{array}{ll}
      \pi(\topS) &= \topS \\
      \pi(\store{\storebindingRaw{l_1}{p_1}, \dots,
        \storebindingRaw{l_n}{p_n}}) &=
      \store{\storebindingRaw{\pi(l_1)}{p_1}, \dots,
        \storebindingRaw{\pi(l_n)}{p_n}} \\
    \end{array}
  \end{displaymath}
\end{definition}
}

\newcommand{\DefPermutationConfiguration}{
\begin{definition}[permutation (configurations)]\label{def:lvars-permutation-configuration}
  A \emph{permutation} of a configuration $\config{S}{e}$ is a function $\pi$
  defined as follows: if $\config{S}{e} = \error$, then
  $\pi(\config{S}{e}) = \error$; otherwise, $\pi(\config{S}{e}) =
  \config{\pi(S)}{\pi(e)}$.
\end{definition}
}

\newcommand{\DefEqualStatus}{
\begin{definition}\label{def:equal-status}
  Two stores $S$ and $S'$ are \emph{equal in status}
  (written $S \statuseq S'$) iff for all $l \in (\dom{S} \cap \dom{S'})$, \\
  if $S(l) = \state{d}{\status}$ and $S'(l) = \state{d'}{\status'}$, then $\status = \status'$.
\end{definition}
}

\newcommand{\DefNonConflicting}{
\begin{definition}\label{def:non-conflicting}
  A store $S''$ is \emph{non-conflicting} with the transition
  $\config{S}{e} \parstepsto \config{S'}{e'}$ iff $(\dom{S'} -
  \dom{S}) \cap \dom{S''} = \emptyset$.
\end{definition}
}

\newcommand{\LemPartitionOfDp}{
\begin{lem}[Partition of $D_p$]\label{lem:partition-of-Dp}
If $(D, \userleq, \bot, \top)$ is a lattice and $(D_p, \leqp, \botp,
\topp) = \Freeze{D, \userleq, \bot, \top}$, and $X = D -
\setof{\top}$, then every member of $D_p$ is either
\begin{itemize}
  \item $(d, \textup{\frozenfalse})$, with $d \in D$, or
  \item $(x, \textup{\frozentrue})$, with $x \in X$.
\end{itemize}
\end{lem}
}

\newcommand{\LemLatticeStructure}{
\begin{lem}[Lattice structure]\label{lem:lattice-structure}
If $(D, \userleq, \bot, \top)$ is a lattice and $(D_p, \leqp, \botp,
\topp) = \Freeze{D, \userleq, \bot, \top}$, then:

\begin{enumerate}
  \item $\leqp$ is a partial order over $D_p$.
   
  \item Every nonempty finite subset of $D_p$ has a lub.

  \item $\botp$ is the least element of $D_p$.

  \item $\topp$ is the greatest element of $D_p$.
\end{enumerate}

Therefore $(D_p, \leqp, \botp, \topp)$ is a lattice.
\end{lem}
}

\newcommand{\LemPermutability}{
\begin{lem}[Permutability, $\lambdaLVish$]\label{lem:permutability}
  For any finite permutation $\pi$,
  \begin{enumerate}
  \item \label{thm:quasi-permutable-reduction-transitions} $\conf
    \parstepsto \conf'$ if and only if $\pi(\conf) \parstepsto
    \pi(\conf')$.
  \item \label{thm:quasi-permutable-context-transitions} $\conf
    \ctxstepsto \conf'$ if and only if $\pi(\conf) \ctxstepsto
    \pi(\conf')$.
  \end{enumerate}
\end{lem}
}

\newcommand{\LemInternalDeterminism}{
\begin{lem}[Internal Determinism, $\lambdaLVish$]\label{lem:internal-determinism}
  If $\conf \parstepsto \conf'$ and $\conf \parstepsto \conf''$, then
  there is a permutation $\pi$ such that $\conf' = \pi(\conf'')$.
\end{lem}
}

\newcommand{\LemLocality}{
\begin{lem}[Locality, $\lambdaLVish$]\label{lem:locality}
  If $\config{S}{\evalctxt{E_1}{e_1}} \ctxstepsto
  \config{S_1}{\evalctxt{E_1}{e'_1}}$ and
  $\config{S}{\evalctxt{E_2}{e_2}} \ctxstepsto
  \config{S_2}{\evalctxt{E_2}{e'_2}}$ and $\evalctxt{E_1}{e_1} =
  \evalctxt{E_2}{e_2}$, then:

  If $E_1 \neq E_2$, then there exist evaluation contexts $E'_1$ and
  $E'_2$ such that:
  \begin{itemize}
  \item $\evalctxt{E'_1}{e_1} = \evalctxt{E_2}{e'_2}$, and
  \item $\evalctxt{E'_2}{e_2} = \evalctxt{E_1}{e'_1}$, and
  \item $\evalctxt{E'_1}{e'_1} = \evalctxt{E'_2}{e'_2}$.
  \end{itemize}
\end{lem}
}

\newcommand{\LemMonotonicity}{
\begin{lem}[Monotonicity, $\lambdaLVish$]\label{lem:monotonicity}
  If $\config{S}{e} \parstepsto \config{S'}{e'}$,
  then $\leqstore{S}{S'}$.
\end{lem}
}

\newcommand{\LemGeneralizedIndependence}{
\begin{lem}[Generalized Independence]\label{lem:generalized-independence}
  If $\config{S}{e} \parstepsto \config{S'}{e'}$ (where
  $\config{S'}{e'} \neq \textup{\error}$), then we have that:
   
  $\config{U_S(S)}{e} \parstepsto \config{U_S(S')}{e'}$,

  where $U_S$ is a store update operation meeting the following conditions:
  \begin{itemize}
    \item $U_S$ is non-conflicting with $\config{S}{e} \parstepsto \config{S'}{e'}$,
    \item $U_S(S') \neq \topS$, and
    \item $U_S$ is freeze-safe with $\config{S}{e} \parstepsto \config{S'}{e'}$.
  \end{itemize}
\end{lem}
}

\newcommand{\LemGeneralizedClash}{
\begin{lem}[Generalized Clash]\label{lem:generalized-clash}
  If $\config{S}{e} \parstepsto \config{S'}{e'}$ (where
  $\config{S'}{e'} \neq \textup{\error}$), then we have that:

  $\config{U_S(S)}{e} \parstepsto^i \error$, where $i \leq 1$,

  and where $U_S$ is a store update operation meeting the following
  conditions:
  \begin{itemize}
    \item $U_S$ is non-conflicting with $\config{S}{e} \parstepsto \config{S'}{e'}$,
    \item $U_S(S') = \topS$, and
    \item $U_S$ is freeze-safe with $\config{S}{e} \parstepsto \config{S'}{e'}$.
  \end{itemize}
\end{lem}
}

\newcommand{\LemErrorPreservation}{
\begin{lem}[Error Preservation, $\lambdaLVish$]\label{lem:error-preservation}
  If $\config{S}{e} \parstepsto \textup{\error}$ and
  $\leqstore{S}{S'}$, then $\config{S'}{e} \parstepsto
  \textup{\error}$.
\end{lem}
}

\newcommand{\LemStrongLocalQuasiConfluence}{
\begin{lem}[Strong Local Quasi-Confluence]\label{lem:strong-local-quasi-confluence}
 If $\conf \ctxstepsto \conf_a$ and $\conf \ctxstepsto \conf_b$, then
 either:
  \begin{enumerate}
  \item there exist $\conf_c, i, j, \pi$ such that $\conf_a
    \ctxstepsto^i \conf_c$ and $\pi(\conf_b) \ctxstepsto^j \conf_c$
    and $i \leq 1$ and $j \leq 1$, or
  \item $\conf_a \ctxstepsto \textup{\error}$ or $\conf_b \ctxstepsto
    \textup{\error}$.
  \end{enumerate}
\end{lem}
}

\newcommand{\LemStrongOneSidedQuasiConfluence}{
\begin{lem}[Strong One-Sided Quasi-Confluence]\label{lem:strong-one-sided-quasi-confluence}
  If $\conf \ctxstepsto \conf'$ and $\conf \ctxstepsto^m \conf''$,
  where $1 \leq m$, then either:
  \begin{enumerate}
    \item there exist $\conf_c, i, j, \pi$ such that $\conf'
      \ctxstepsto^i \conf_c$ and $\pi(\conf'') \ctxstepsto^j \conf_c$
      and $i \leq m$ and $j \leq 1$, or
    \item there exists $k \leq m$ such that $\conf' \ctxstepsto^k
      \textup{\error}$, or there exists $k \leq 1$ such that $\conf''
      \ctxstepsto^k \textup{\error}$.
  \end{enumerate}
\end{lem}
}

\newcommand{\LemStrongQuasiConfluence}{
\begin{lem}[Strong Quasi-Confluence]\label{lem:strong-quasi-confluence}
  If $\conf \ctxstepsto^n \conf'$ and $\conf \ctxstepsto^m \conf''$,
  where $1 \leq n$ and $1 \leq m$, then either:
  \begin{enumerate}
    \item there exist $\conf_c, i, j, \pi$ such that $\conf'
      \ctxstepsto^i \conf_c$ and $\pi(\conf'') \ctxstepsto^j \conf_c$
      and $i \leq m$ and $j \leq n$, or
  \item there exists $k \leq m$ such that $\conf' \ctxstepsto^k
    \textup{\error}$, or there exists $k \leq n$ such that $\conf''
    \ctxstepsto^k \textup{\error}$.
  \end{enumerate}
\end{lem}
}

\newcommand{\LemQuasiConfluence}{
\begin{lem}[Quasi-Confluence]\label{lem:quasi-confluence}
  If $\conf \ctxstepsto^* \conf'$ and $\conf \ctxstepsto^* \conf''$,
  then either:
  \begin{enumerate}
  \item there exist $\conf_c$ and $\pi$ such that $\conf'
    \ctxstepsto^* \conf_c$ and $\pi(\conf'') \ctxstepsto^* \conf_c$,
    or
  \item $\conf' = \textup{\error}$ or $\conf'' = \textup{\error}$.
  \end{enumerate}
\end{lem}
}

\newcommand{\ThmQuasiDeterminism}{
\begin{thm}[Quasi-Determinism]\label{thm:quasi-determinism}
  If $\conf \ctxstepsto^* \conf'$ and $\conf \ctxstepsto^* \conf''$,
  and neither $\conf'$ nor $\conf''$ can take a step, then either:
  \begin{enumerate}
    \item there exists $\pi$ such that $\conf' = \pi(\conf'')$, or
    \item $\conf' = \textup{\error}$ or $\conf'' = \textup{\error}$.
  \end{enumerate}
\end{thm}
}


% Editing marks
%%% Macros for editing marks.

\newcommand{\TBD}[0]{{\color{red}TBD}}
\newcommand{\TODO}[1]{\noindent{\color{red} TODO: #1}\\}

\newcommand{\lk}[1]{\pgwrapper{LK}{#1}}
\newcommand{\rn}[1]{\pgwrapper{RRN}{#1}}
\newcommand{\rnote}[1]{\pgwrapper{RRN}{#1}}
\newcommand{\ajt}[1]{\pgwrapper{AJT}{#1}}

\InputIfFileExists{editingmarks}{}{
    \def\noeditingmarks{}
}

\definecolor{comment-red}{rgb}{0.8,0,0}
\ifx\noeditingmarks\undefined
   \newcommand{\textred}[1]{\textcolor{comment-red}{#1}}
   \newcommand{\pgwrapper}[2]{\textred{#1: #2}}
   \newcommand{\new}[1]{\textcolor{blue}{#1}}
\else
   \newcommand{\textred}[1]{\textcolor{comment-red}{#1}}
   \newcommand{\pgwrapper}[2]{}
   \newcommand{\new}[1]{#1}
\fi


%%% Formatting of definitions, theorems, etc.
\newtheorem{lem}{Lemma}[section]
\newtheorem{cor}{Corollary}[section]
\newtheorem{thm}{Theorem}[section]

\renewcommand{\theequation}{Example \thesection.\arabic{equation}}

% Defining some extra theoremstyle stuff for documents that need it
% (i.e., non-LNCS-style documents)
%\theoremstyle{definition}
\newtheorem{definition}{Definition}[section]
\newtheorem{example}{Example}[section]
\newtheorem{remark}{Remark}[section]

% Other presentational stuff
%% Assorted style and presentation stuff.

\newcommand{\figsepr}{\vspace{0.5em}\hrulefill\vspace{0.5em}}
\newcommand{\etal}{\textit{et al}.}
\newcommand{\ie}{\textit{i}.\textit{e}.}
\newcommand{\eg}{\textit{e}.\textit{g}.}
\newcommand{\il}[1]{\lstinline{#1}}

\newenvironment{blockquote}{%
  \par%
  \medskip
  \leftskip=4em\rightskip=2em%
  \noindent\ignorespaces}{%
  \par\medskip}

%% LK: See the definition of `\theequation`.
\newcommand{\eref}[1]{\ref{#1}}

%====Set up Listings===============================================================

\definecolor{darkgreen}{rgb}{0,0.5,0}
\definecolor{darkred}{rgb}{0.5,0,0}
\lstloadlanguages{Haskell}
\lstnewenvironment{code}
    { % \centering
      \lstset{}%
      \csname lst@SetFirstLabel\endcsname}
    { %\centering
      \csname lst@SaveFirstLabel\endcsname}
    \lstset{
      language=Haskell,
%      basicstyle=\footnotesize\ttfamily,
      basicstyle=\small\ttfamily,
      flexiblecolumns=false,
      basewidth={0.5em,0.45em},
      aboveskip={3pt},
      belowskip={3pt},
      keywordstyle=\color{black},  
      commentstyle=\color{darkgreen},
      literate={+}{{$+$}}1 {/}{{$/$}}1 {*}{{$*$}}1 % {=}{{$=$}}1
               {>}{{$>$}}1 {<}{{$<$}}1 {\\}{{$\lambda$}}1
               {\\\\}{{\char`\\\char`\\}}1
               {->}{{$\rightarrow$}}2 {>=}{{$\geq$}}2 {<-}{{$\leftarrow$}}2
               {<=}{{$\leq$}}2 {=>}{{$\Rightarrow$}}2 
               {\ .}{{$\circ$}}2 {\ .\ }{{$\circ$}}2
               {>>}{{>>}}2 {>>=}{{>>=}}2 {=<<}{{=<<}}2
               {|}{{$\mid$}}1
               {`member`}{{$\in$}}1
               {leftBrace}{\{}1
               {rightBrace}{\}}1
               {\$singleton\$startV}{{ \hspace{2.4em} \{{\tt startV}\}}}1
               {\$singleton\$n}{{  \{{\tt n}\}}}1
               {dotdotdot}{{$\ldots$}}3
    }

%================================================================================

% Penalty for line-breaking inline math
\relpenalty=9999
\binoppenalty=9999


% Specific to amsthm
\newenvironment{proofsketch}[1][\proofname]{\proof[#1]\mbox{}}{\endproof}

% Ignore badness
\renewcommand{\doublespacing}{}

% Use the @ symbol for simple inline code within prose:
\lstMakeShortInline[]@

% Seutup for hyperref.
\hypersetup{
    pdftitle={LVars: Lattice-based Data Structures for Deterministic Parallel Programming},
    pdfauthor={Lindsey Kuper \etal},
    colorlinks=true,
    linkcolor=blue,
    citecolor=blue,
    urlcolor=blue,
}

\title{LVars: Lattice-based Data Structures for Deterministic Parallel Programming}

\author[L. Kuper, A. Turon, N. R. Krishnaswami, and R. R. Newton]
       {LINDSEY KUPER\\
          Intel Labs\\
        \and\ AARON TURON\\
          Mozilla Research\\
        \and\ NEELAKANTAN R. KRISHNASWAMI\\
          University of Birmingham\\
        \and\ RYAN R. NEWTON\\
        Indiana University}

\begin{document}

\label{firstpage}

\maketitle

\begin{abstract}
\begin{abstract}

  Programs written using a {\em deterministic-by-construction} model
  of parallel computation are guaranteed to always produce the same
  observable results, offering programmers freedom from subtle,
  hard-to-reproduce nondeterministic bugs that are the scourge of
  parallel software.
  We present \emph{LVars}, a new model for deterministic-by-construction parallel
  programming that generalizes existing single-assignment models to
  allow multiple assignments that are monotonically increasing with
  respect to a user-specified lattice.  
  LVars ensure determinism by
  allowing only monotonic writes and ``threshold'' reads that block
  until a lower bound is reached.  We give a proof of determinism
  and a prototype implementation
  for a language with LVars
  and describe how to extend the LVars model to support a limited form of
  nondeterminism that admits failures but never wrong answers.
%   \rn{I'm a bit worried about saying ``both those with shared state''
%     since shared write-once state is quite a bit different than what
%     most people usually think of.}

\if{0}
  There is a need for new proposals that add diversity to the current
  menu of determistic parallel programming options.
  This paper presents a variant of the $\lambda$-calculus, $\lambdapar$: a
  parallel, call-by-value calculus that includes shared mutable
  variables whose states occupy a join semilattice and move
  monotonically upwards in that lattice.  
  In this paper we prove $\lambdapar$ deterministic.  
  Further, we explain how $\lambdapar$ generalizes a number of
  existing deterministic parallel programming models, and provides a
  new foundation for exploring {limited} forms of nondeterminism.
\fi{}

\end{abstract}

\end{abstract}

\chapter{Proofs}\label{app:proofs}

\section{Proof of Lemma~\ref{lem:lvars-permutability}}\label{section:lvars-permutability-proof}
\begin{proof}
  Consider an arbitrary permutation $\pi$.  For
  part~\ref{thm:permutable-reduction-transitions}, we have to show
  that if $\conf \parstepsto \conf'$ then $\pi(\conf) \parstepsto
  \pi(\conf')$, and that if $\pi(\conf) \parstepsto \pi(\conf')$ then
  $\conf \parstepsto \conf'$.

  For the forward direction of
  part~\ref{thm:permutable-reduction-transitions}, suppose $\conf
  \parstepsto \conf'$.  We have to show that $\pi(\conf) \parstepsto
  \pi(\conf')$.  We proceed by cases on the rule by which $\conf$
  steps to $\conf'$.

  \begin{itemize}
    \item Case {\sc E-Beta}: $\conf =
      \config{S}{\app{(\lam{x}{e})}{v}}$, and $\conf' =
      \config{S}{\subst{e}{x}{v}}$.

      To show: $\pi(\config{S}{\app{(\lam{x}{e})}{v}}) \parstepsto
      \pi(\config{S}{\subst{e}{x}{v}})$.

      By Definitions~\ref{def:lvars-permutation-configuration}
      and~\ref{def:lvars-permutation-expression}, $\pi(\conf) =
      \config{\pi(S)}{\app{(\lam{x}{\pi(e)})}{\pi(v)}}$.

      By {\sc E-Beta},
      $\config{\pi(S)}{\app{(\lam{x}{\pi(e)})}{\pi(v)}}$ steps to
      $\config{\pi(S)}{\subst{\pi(e)}{x}{\pi(v)}}$.

      By Definition~\ref{def:lvars-permutation-expression},
      $\config{\pi(S)}{\subst{\pi(e)}{x}{\pi(v)}}$ is equal to
      $\config{\pi(S)}{\pi(\subst{e}{x}{v})}$.

      Hence $\config{\pi(S)}{\app{(\lam{x}{\pi(e)})}{\pi(v)}}$ steps
      to $\config{\pi(S)}{\pi(\subst{e}{x}{v})}$,

      which is equal to $\pi(\config{S}{\subst{e}{x}{v}})$ by
      Definition~\ref{def:lvars-permutation-configuration}.  Hence the
      case is satisfied.

    \item Case {\sc E-New}: $\conf = \config{S}{\NEW}$, and $\conf' =
      \config{\extSRaw{S}{l}{\bot}}{l}$.

      To show: $\pi(\config{S}{\NEW}) \parstepsto
      \pi(\config{\extSRaw{S}{l}{\bot}}{l})$.

      By Definitions~\ref{def:lvars-permutation-configuration}
      and~\ref{def:lvars-permutation-expression}, $\pi(\conf) =
      \config{\pi(S)}{\NEW}$.

      By {\sc E-New}, $\config{\pi(S)}{\NEW}$ steps to
      $\config{\extSRaw{(\pi(S))}{l'}{\bot}}{l'}$, where $l' \notin
      \dom{\pi(S)}$.
      
      It remains to show that
      $\config{\extSRaw{(\pi(S))}{l'}{\bot}}{l'}$ is equal to
      $\pi(\config{\extSRaw{S}{l}{\bot}}{l})$.

      By Definition~\ref{def:lvars-permutation-configuration},
      $\pi(\config{\extSRaw{S}{l}{\bot}}{l})$ is equal to
      $\config{\pi(\extSRaw{S}{l}{\bot})}{\pi(l)}$,

      which is equal to
      $\config{\extSRaw{(\pi(S))}{\pi(l)}{\bot}}{\pi(l)}$.

      So, we have to show that
      $\config{\extSRaw{(\pi(S))}{l'}{\bot}}{l'}$ is equal to
      $\config{\extSRaw{(\pi(S))}{\pi(l)}{\bot}}{\pi(l)}$.  Since we
      know (from the side condition of {\sc E-New}) that $l \notin
      \dom{S}$, it follows that $\pi(l) \notin \pi(\dom{S})$.
      Therefore, in $\config{\extSRaw{(\pi(S))}{l'}{\bot}}{l'}$, we
      can $\alpha$-rename $l'$ to $\pi(l)$, and so the two
      configurations are equal and the case is satisfied.

    \item Case {\sc E-Put}: $\conf = \config{S}{\putexp{l}{d_2}}$, and
      $\conf' = \config{\extSRaw{S}{l}{\userlub{d_1}{d_2}}}{\unit}$.

      To show: $\pi(\config{S}{\putexp{l}{d_2}}) \parstepsto
      \pi(\config{\extSRaw{S}{l}{\userlub{d_1}{d_2}}}{\unit})$.

      By Definitions~\ref{def:lvars-permutation-configuration}
      and~\ref{def:lvars-permutation-expression}, $\pi(\conf) =
      \config{\pi(S)}{\putexp{\pi(l)}{d_2}}$.

      By {\sc E-Put}, $\config{\pi(S)}{\putexp{\pi(l)}{d_2}}$ steps to
      $\config{\extSRaw{(\pi(S))}{\pi(l)}{\userlub{d_1}{d_2}}}{\unit}$,

      since $S(l) = (\pi(S))(\pi(l)) = d_1$.

      It remains to show that
      $\config{\extSRaw{(\pi(S))}{\pi(l)}{\userlub{d_1}{d_2}}}{\unit}$
      is equal to
      $\pi(\config{\extSRaw{S}{l}{\userlub{d_1}{d_2}}}{\unit})$.

      By Definitions~\ref{def:lvars-permutation-configuration}
      and~\ref{def:lvars-permutation-expression},
      $\pi(\config{\extSRaw{S}{l}{\userlub{d_1}{d_2}}}{\unit})$ is
      equal to
      $\config{\extSRaw{(\pi(S))}{\pi(l)}{\userlub{d_1}{d_2}}}{\unit}$,
      and so the two configurations are equal and the case is
      satisfied.

    \item Case {\sc E-Put-Err}: $\conf = \config{S}{\putexp{l}{d_2}}$,
      and $\conf' = \error$.

      To show: $\pi(\config{S}{\putexp{l}{d_2}}) \parstepsto
      \pi(\error)$.

      By Definitions~\ref{def:lvars-permutation-configuration}
      and~\ref{def:lvars-permutation-expression}, $\pi(\conf) =
      \config{\pi(S)}{\putexp{\pi(l)}{d_2}}$.

      By {\sc E-Put-Err}, $\config{\pi(S)}{\putexp{\pi(l)}{d_2}}$
      steps to $\error$,

      since $S(l) = (\pi(S))(\pi(l)) = d_1$.

      Since $\pi(\error) = \error$ by
      Definition~\ref{def:lvars-permutation-configuration}, the case
      is complete.

    \item Case {\sc E-Get}: $\conf = \config{S}{\getexp{l}{T}}$, and
      $\conf' = \config{S}{d_2}$.

      To show: $\pi(\config{S}{\getexp{l}{T}}) \parstepsto
      \pi(\config{S}{d_2})$.

      By Definitions~\ref{def:lvars-permutation-configuration}
      and~\ref{def:lvars-permutation-expression}, $\pi(\conf) =
      \config{\pi(S)}{\getexp{\pi(l)}{T}}$.

      By {\sc E-Get}, $\config{\pi(S)}{\getexp{\pi(l)}{T}}$ steps to
      $\config{\pi(S)}{d_2}$,

      since $S(l) = (\pi(S))(\pi(l)) = d_1$.

      By Definitions~\ref{def:lvars-permutation-configuration}
      and~\ref{def:lvars-permutation-expression},
      $\pi(\config{S}{d_2}) \config{\pi(S)}{d_2}$.  Therefore the case
      is complete.
  \end{itemize}

  For the reverse direction of
  part~\ref{thm:permutable-reduction-transitions}, suppose $\pi(\conf)
  \parstepsto \pi(\conf')$.  We have to show that $\conf \parstepsto
  \conf'$.

  We know from the forward direction of the proof that for all
  configurations $\conf$ and $\conf'$ and permutations $\pi$, if
  $\conf \parstepsto \conf'$ then $\pi(\conf) \parstepsto
  \pi(\conf')$.  Hence since $\pi(\conf) \parstepsto \pi(\conf')$, and
  since $\piinv$ is also a permutation, we have that
  $\piinv(\pi(\conf)) \parstepsto \piinv(\pi(\conf'))$.  Since
  $\piinv(\pi(l)) = l$ for every $l \in \Loc$, and that property lifts
  to configurations as well, we have that $\conf \parstepsto \conf'$.

  \lk{Is the above enough of a proof?}

  For the forward direction of
  part~\ref{thm:permutable-context-transitions}, suppose $\conf
  \ctxstepsto \conf'$.  We have to show that $\pi(\conf) \ctxstepsto
  \pi(\conf')$.

  By inspection of the operational semantics, $\conf$ must be of the
  form $\config{S}{\E{e}}$, and $\conf'$ must be of the form
  $\config{S'}{\E{e'}}$.  Hence we have to show that
  $\pi(\config{S}{\E{e}}) \ctxstepsto \pi(\config{S'}{\E{e'}})$.

  By Definition~\ref{def:lvars-permutation-configuration},
  $\pi(\config{S}{\E{e}})$ is equal to $\config{\pi(S)}{\pi(\E{e})}$.

  Also by Definition~\ref{def:lvars-permutation-configuration},
  $\pi(\config{S'}{\E{e'}})$ is equal to
  $\config{\pi(S')}{\pi(\E{e'})}$.

  Furthermore, $\config{\pi(S)}{\pi(\E{e})}$ is equal to
  $\config{\pi(S)}{\evalctxt{(\pi(E))}{\pi(e)}}$ and
  $\config{\pi(S')}{\pi(\E{e'})}$ is equal to
  $\config{\pi(S')}{\evalctxt{(\pi(E))}{\pi(e')}}$.

  So we have to show that
  $\config{\pi(S)}{\evalctxt{(\pi(E))}{\pi(e)}} \ctxstepsto
  \config{\pi(S')}{\evalctxt{(\pi(E))}{\pi(e')}}$.

  From the premise of {\sc E-Eval-Ctxt}, $\config{S}{e} \parstepsto
  \config{S'}{e'}$.  Hence, by
  part~\ref{thm:permutable-reduction-transitions}, $\pi(\config{S}{e})
  \parstepsto \pi(\config{S'}{e'})$.  By
  Definition~\ref{def:lvars-permutation-configuration},
  $\pi(\config{S}{e})$ is equal to $\config{\pi(S)}{\pi(e)}$ and
  $\pi(\config{S'}{e'})$ is equal to $\config{\pi(S')}{\pi(e')}$.

  Hence $\config{\pi(S)}{\pi(e)} \parstepsto
  \config{\pi(S')}{\pi(e')}$.

  Therefore, by {\sc E-Eval-Ctxt}, $\config{\pi(S)}{\E{\pi(e)}}
  \ctxstepsto \config{\pi(S')}{\E{\pi(e')}}$ for all evaluation
  contexts $E$.

  In particular, it is true that
  $\config{\pi(S)}{\evalctxt{(\pi(E))}{\pi(e)}} \ctxstepsto
  \config{\pi(S')}{\evalctxt{(\pi(E))}{\pi(e')}}$, as we were required
  to show.

  For the reverse direction of
  part~\ref{thm:permutable-context-transitions}, suppose $\pi(\conf)
  \ctxstepsto \pi(\conf')$.  We have to show that $\conf \ctxstepsto
  \conf'$.

  We know from the forward direction of the proof that for all
  configurations $\conf$ and $\conf'$ and permutations $\pi$, if
  $\conf \ctxstepsto \conf'$ then $\pi(\conf) \ctxstepsto
  \pi(\conf')$.  Hence since $\pi(\conf) \ctxstepsto \pi(\conf')$, and
  since $\piinv$ is also a permutation, we have that
  $\piinv(\pi(\conf)) \ctxstepsto \piinv(\pi(\conf'))$.  Since
  $\piinv(\pi(l)) = l$ for every $l \in \Loc$, and that property lifts
  to configurations as well, we have that $\conf \ctxstepsto \conf'$.

  \lk{Is the above enough of a proof?}
\end{proof}


\section{Proof of Lemma~\ref{lem:lvars-internal-determinism}}\label{section:lvars-internal-determinism-proof}
\begin{proof}
  Suppose $\conf \parstepsto \conf'$ and $\conf \parstepsto \conf''$.

  We have to show that there is a permutation $\pi$ such that $\conf'
  = \pi(\conf'')$.

  The proof is by cases on the rule by which $\conf$ steps to
  $\conf'$.

  \begin{itemize}

  \item Case {\sc E-Beta}:

    Given: $\config{S}{\app{(\lam{x}{e})}{v}} \parstepsto
    \config{S}{\subst{e}{x}{v}}$, and
    $\config{S}{\app{(\lam{x}{e})}{v}} \parstepsto \conf''$.

    To show: There exists a $\pi$ such that
    $\config{S}{\subst{e}{x}{v}} = \pi(\conf'')$.

    By inspection of the operational semantics, the only reduction
    rule by which $\config{S}{\app{(\lam{x}{e})}{v}}$ can step is {\sc
      E-Beta}.

    Hence $\conf'' = \config{S}{\subst{e}{x}{v}}$, and the case is
    satisfied by choosing $\pi$ to be the identity function.

  \item Case {\sc E-New}: 

    Given: $\config{S}{\NEW} \parstepsto
    \config{\extSRaw{S}{l}{\bot}}{l}$, and $\config{S}{\NEW}
    \parstepsto \conf''$.

    To show: There exists a $\pi$ such that
    $\config{\extSRaw{S}{l}{\bot}}{l} = \pi(\conf'')$.

    By inspection of the operational semantics, the only reduction
    rule by which $\config{S}{\NEW}$ can step is {\sc E-New}.

    Hence $\conf'' = \config{\extSRaw{S}{l'}{\bot}}{l'}$.

    Since, by the side condition of {\sc E-New}, neither $l$ nor $l'$
    occur in $\dom{S}$, the case is satisfied by choosing $\pi$ to be
    the permutation that maps $l'$ to $l$ and is the identity on every
    other element of $\Loc$.

  \item Case {\sc E-Put}:

    Given: $\config{S}{\putexp{l}{d_2}} \parstepsto
    \config{\extSRaw{S}{l}{\userlub{d_1}{d_2}}}{\unit}$, and
    $\config{S}{\putexp{l}{d_2}} \parstepsto \conf''$.

    To show: There exists a $\pi$ such that
    $\config{\extSRaw{S}{l}{\userlub{d_1}{d_2}}}{\unit} =
    \pi(\conf'')$.

    By inspection of the operational semantics, and since
    $\userlub{d_1}{d_2} \neq \top$ (from the premise of {\sc E-Put}),
    the only reduction rule by which $\config{S}{\putexp{l}{d_2}}$ can
    step is {\sc E-Put}.

    Hence $\conf'' =
    \config{\extSRaw{S}{l}{\userlub{d_1}{d_2}}}{\unit}$, and the case
    is satisfied by choosing $\pi$ to be the identity function.

  \item Case {\sc E-Put-Err}:

    Given: $\config{S}{\putexp{l}{d_2}} \parstepsto \error$, and
    $\config{S}{\putexp{l}{d_2}} \parstepsto \conf''$.

    To show: There exists a $\pi$ such that $\error = \pi(\conf'')$.

    By inspection of the operational semantics, and since
    $\userlub{d_1}{d_2} = \top$ (from the premise of {\sc E-Put-Err}),
    the only reduction rule by which $\config{S}{\putexp{l}{d_2}}$ can
    step is {\sc E-Put-Err}.

    Hence $\conf'' = \error$, and the case is satisfied by choosing
    $\pi$ to be the identity function.

  \item Case {\sc E-Get}:

    Given: $\config{S}{\getexp{l}{T}} \parstepsto \config{S}{d_2}$,
    and $\config{S}{\getexp{l}{T}} \parstepsto \conf''$.

    To show: There exists a $\pi$ such that $\config{S}{d_2} =
    \pi(\conf'')$.

    By inspection of the operational semantics, the only reduction
    rule by which $\config{S}{\getexp{l}{T}}$ can step is {\sc
      E-Get}.

    Hence $\conf'' = \config{S}{d_2}$, and the case is satisfied by
    choosing $\pi$ to be the identity function.

  \end{itemize}
\end{proof}



\section{Proof of Lemma~\ref{lem:lvars-monotonicity}}\label{section:lvars-monotonicity-proof}
\begin{proof}
  Suppose $\config{S}{e} \parstepsto \config{S'}{e'}$.  We are
  required to show that $\leqstore{S}{S'}$.  The proof is by cases on
  the rule by which $\config{S}{e}$ steps to $\config{S'}{e'}$.

  \begin{itemize}
    \item Case {\sc E-Beta}:

      Immediate by the definition of $\leqstore{}{}$, since $S$ does
      not change.

    \item Case {\sc E-New}:

      Given: $\config{S}{\NEW} \parstepsto
      \config{\extSRaw{S}{l}{\bot}}{l}$.

      To show: $\leqstore{S}{\extSRaw{S}{l}{\bot}}$.

      By Definition~\ref{def:lvars-leqstore}, we have to show that
      $\dom{S} \subseteq \dom{\extSRaw{S}{l}{\bot}}$ and
      that for all $l' \in \dom{S}$, $S(l') \userleq
      (\extSRaw{S}{l}{\bot})(l')$.

      By definition, a store update operation on $S$ can only either
      update an existing binding in $S$ or extend $S$ with a new
      binding.  Hence $\dom{S} \subseteq \dom{\extSRaw{S}{l}{\bot}}$.

      From the side condition of {\sc E-New}, $l \notin \dom{S}$.
      Hence $\extSRaw{S}{l}{\bot}$ adds a new binding for $l$ in $S$.

      Hence $\extSRaw{S}{l}{\bot}$ does not update any existing
      bindings in $S$.

      Hence, for all $l' \in \dom{S}, S(l') \userleq
      (\extSRaw{S}{l}{\bot})(l')$.

      Therefore $\leqstore{S}{\extSRaw{S}{l}{\bot}}$, as
      required.

    \item Case {\sc E-Put}:

      Given: $\config{S}{\putexp{l}{d_2}} \parstepsto
      \config{\extSRaw{S}{l}{\userlub{d_1}{d_2}}}{\unit}$.

      To show: $\leqstore{S}{\extSRaw{S}{l}{\userlub{d_1}{d_2}}}$.

      By Definition~\ref{def:lvars-leqstore}, we have to show that
      $\dom{S} \subseteq \dom{\extSRaw{S}{l}{\userlub{d_1}{d_2}}}$ and
      that for all $l' \in \dom{S}$, $S(l') \userleq
      (\extSRaw{S}{l}{\userlub{d_1}{d_2}})(l')$.

      By definition, a store update operation on $S$ can only either
      update an existing binding in $S$ or extend $S$ with a new
      binding.  Hence $\dom{S} \subseteq
      \dom{\extSRaw{S}{l}{\userlub{d_1}{d_2}}}$.

      From the premises of {\sc E-Put}, $S(l) = d_1$.  Therefore $l
      \in \dom{S}$.

      Hence $\extSRaw{S}{l}{\userlub{d_1}{d_2}}$ updates the existing
      binding for $l$ in $S$ from $d_1$ to $\userlub{d_1}{d_2}$.

      By the definition of $\userlub{}{}$, $d_1 \userleq
      (\userlub{d_1}{d_2})$.  $\extSRaw{S}{l}{\userlub{d_1}{d_2}}$
      does not update any other bindings in $S$, hence, for all $l'
      \in \dom{S}, S(l') \userleq
      (\extSRaw{S}{l}{\userlub{d_1}{d_2}})(l')$.

      Hence $\leqstore{S}{\extSRaw{S}{l}{\userlub{d_1}{d_2}}}$, as
      required.

    \item Case {\sc E-Put-Err}:

      Given: $\config{S}{\putexp{l}{d_2}} \parstepsto \error$.

      By the definition of $\error$, $\error$ is equal to
      $\config{\topS}{e}$ for all $e$.

      To show: $\leqstore{S}{\topS}$.

      Immediate by the definition of $\leqstore{}{}$.

    \item Case {\sc E-Get}:

      Immediate by the definition of $\leqstore{}{}$, since $S$ does
      not change.

  \end{itemize}

\end{proof}


\section{Proof of Lemma~\ref{lem:lvars-independence}}\label{section:lvars-independence-proof}
\begin{proof}
  Consider arbitrary $S''$ such that $S''$ is non-conflicting with
  $\config{S}{e} \parstepsto \config{S'}{e'}$ and $\lubstore{S'}{S''}
  \neq \topS$.

  To show: $\config{\lubstore{S}{S''}}{e} \parstepsto
  \config{\lubstore{S'}{S''}}{e'}$.

  The proof is by induction on the derivation of $\config{S}{e}
  \parstepsto \config{S'}{e'}$, by cases on the last rule in the
  derivation.  In every case we may assume that $\config{S'}{e'} \neq
  \error$.  Since $\config{S'}{e'} \neq \error$, we do not need to
  consider the {\sc E-Put-Err} rule.
  \begin{itemize}

    \item Case {\sc E-Eval-Ctxt}:

      Given: $\config{S}{\E{e}} \parstepsto \config{S'}{\E{e'}}$.

      To show: $\config{\lubstore{S}{S''}}{\E{e}} \parstepsto
      \config{\lubstore{S'}{S''}}{\E{e'}}$.

      From the premise of {\sc E-Eval-Ctxt}, we have that
      $\config{S}{e} \parstepsto \config{S'}{e'}$.

      Therefore, by IH, we have that $\config{\lubstore{S}{S''}}{e}
      \parstepsto \config{\lubstore{S'}{S''}}{e'}$.

      Therefore, by {\sc E-Eval-Ctxt}, we have that
      $\config{\lubstore{S}{S''}}{\E{e}} \parstepsto
      \config{\lubstore{S'}{S''}}{\E{e'}}$, as we were required to
      show.

    \item Case {\sc E-Beta}:

      Given: $\config{S}{\app{(\lam{x}{e})}{v}} \parstepsto
      \config{S}{\subst{e}{x}{v}}$.

      To show: $\config{\lubstore{S}{S''}}{\app{(\lam{x}{e})}{v}}
      \parstepsto \config{\lubstore{S}{S''}}{\subst{e}{x}{v}}$.

      Immediate by {\sc E-Beta}.

    \item Case {\sc E-New}:

      Given: $\config{S}{\NEW} \parstepsto
      \config{\extSRaw{S}{l}{\bot}}{l}$.

      To show: $\config{\lubstore{S}{S''}}{\NEW} \parstepsto
      \config{\lubstore{(\extSRaw{S}{l}{\bot})}{S''}}{l}$.

      By {\sc E-New}, we have that $\config{\lubstore{S}{S''}}{\NEW}
      \parstepsto \config{\extSRaw{(\lubstore{S}{S''})}{l'}{\bot}}{l'}$,
      where $l' \notin \dom{\lubstore{S}{S''}}$.

      By assumption, $S''$ is non-conflicting with $\config{S}{\NEW}
      \parstepsto \config{\extSRaw{S}{l}{\bot}}{l}$.
 
      Therefore $l \notin \dom{S''}$.

      From the side condition of {\sc E-New}, $l \notin \dom{S}$.

      Therefore $l \notin \dom{\lubstore{S}{S''}}$.

      Therefore, in
      $\config{\extSRaw{(\lubstore{S}{S''})}{l'}{\bot}}{l'}$, we can
      $\alpha$-rename $l'$ to $l$, \\ resulting in
      $\config{\extSRaw{(\lubstore{S}{S''})}{l}{\bot}}{l}$.

      Therefore $\config{\lubstore{S}{S''}}{\NEW} \parstepsto
      \config{\extSRaw{(\lubstore{S}{S''})}{l}{\bot}}{l}$.

      Note that:
      \begin{align*}
        \extSRaw{(\lubstore{S}{S''})}{l}{\bot} &=
        \lubstore{\extSRaw{S}{l}{\bot}}{\extSRaw{S''}{l}{\bot}} \\ &=
        \lubstore{\lubstore{S}{\store{\storebindingRaw{l}{\bot}}}}{\lubstore{S''}{\store{\storebindingRaw{l}{\bot}}}}
        \\ &=
        \lubstore{\lubstore{S}{\store{\storebindingRaw{l}{\bot}}}}{S''}
        \\ &= \lubstore{\extSRaw{S}{l}{\bot}}{S''}.
      \end{align*}
      Therefore $\config{\lubstore{S}{S''}}{\NEW} \parstepsto
      \config{\lubstore{\extSRaw{S}{l}{\bot}}{S''}}{l}$, as we were
      required to show.

    \item Case {\sc E-Put}:

      Given: $\config{S}{\putexp{l}{d_2}} \parstepsto
      \config{\extSRaw{S}{l}{d_2}}{\unit}$.

      To show: $\config{\lubstore{S}{S''}}{\putexp{l}{d_2}}
      \parstepsto
      \config{\lubstore{\extSRaw{S}{l}{d_2}}{S''}}{\unit}$.

      We will first show that

      $\config{\lubstore{S}{S''}}{\putexp{l}{d_2}} \parstepsto
      \config{\extSRaw{(\lubstore{S}{S''})}{l}{d_2}}{\unit}$

      and then show why this is sufficient.

      We proceed by cases on $l$:

      \begin{itemize}
        \item $l \notin \dom{S''}$:

          By assumption, $\lubstore{\extSRaw{S}{l}{d_2}}{S''} \neq
          \topS$.

          By Lemma~\ref{lem:lvars-monotonicity},
          $\leqstore{S}{\extSRaw{S}{l}{d_2}}$.

          Hence $\lubstore{S}{S''} \neq \topS$.

          Therefore, by Definition~\ref{def:lvars-lubstore},
          $(\lubstore{S}{S''})(l) = S(l)$.

          From the premises of {\sc E-Put}, $S(l) = d_1$.

          Hence $(\lubstore{S}{S''})(l) = d_1$.

          From the premises of {\sc E-Put}, $d_2 = \userlub{d_1}{d_2}$
          and $d_2 \neq \top$.

          Therefore, by {\sc E-Put}, we have:
          $\config{\lubstore{S}{S''}}{\putexp{l}{d_2}} \parstepsto
          \config{\extSRaw{(\lubstore{S}{S''})}{l}{d_2}}{\unit}$.

        \item $l \in \dom{S''}$:

          By assumption, $\lubstore{\extSRaw{S}{l}{d_2}}{S''} \neq
          \topS$.

          By Lemma~\ref{lem:lvars-monotonicity},
          $\leqstore{S}{\extSRaw{S}{l}{d_2}}$.

          Hence $\lubstore{S}{S''} \neq \topS$.

          Therefore $(\lubstore{S}{S''})(l) = \userlub{S(l)}{S''(l)}$.

          From the premises of {\sc E-Put}, $S(l) = d_1$.
          
          Hence $(\lubstore{S}{S''})(l) = d'_1$, where $d_1 \userleq
          d'_1$.

          From the premises of {\sc E-Put}, $d_2 =
          \userlub{d_1}{d_2}$.

          Let $d'_2 = \userlub{d'_1}{d_2}$.

          Hence $d_2 \userleq d'_2$.

          By assumption, $\lubstore{\extSRaw{S}{l}{d_2}}{S''} \neq
          \topS$.

          Therefore, by Definition~\ref{def:lvars-lubstore},
          $\lubstore{d_2}{S''(l)} \neq \top$.

          Note that:
          \begin{align*}
            \top &\neq \lubstore{d_2}{S''(l)} \\ &=
            \userlub{\userlub{d_1}{d_2}}{S''(l)} \\ &=
            \userlub{\userlub{S(l)}{d_2}}{S''(l)} \\ &=
            \userlub{\userlub{S(l)}{S''(l)}}{d_2} \\ &=
            \userlub{(\lubstore{S}{S''})(l)}{d_2} \\ &=
            \userlub{d'_1}{d_2} \\ &= d'_2. \\
          \end{align*}
          Hence $d'_2 \neq \top$.

          Hence $(\lubstore{S}{S''})(l) = d'_1$ and $d'_2 =
          \userlub{d'_1}{d_2}$ and $d'_2 \neq \top$.

          Therefore, by {\sc E-Put} we have:
          $\config{\lubstore{S}{S''}}{\putexp{l}{d_2}} \parstepsto
          \config{\extSRaw{(\lubstore{S}{S''})}{l}{d'_2}}{\unit}$.

          \lk{If we really wanted to be pedantic here, we'd actually
            prove that the stores are equal.  I'm assuming that if I
            can show that $\extSRaw{(\lubstore{S}{S''})}{l}{d'_2}$ and
            $\extSRaw{(\lubstore{S}{S''})}{l}{d_2}$ bind $l$ to the
            same value, then it will be obvious that they're equal.}

          Note that:
          \begin{align*}
            (\extSRaw{(\lubstore{S}{S''})}{l}{d'_2})(l) &=
            \userlub{(\lubstore{S}{S''})(l)}{(\store{\storebindingRaw{l}{d'_2}})(l)}
            \\ &= \userlub{d'_1}{d'_2} \\ &=
            \userlub{d'_1}{\userlub{d'_1}{d_2}} \\ &=
            \userlub{d'_1}{d_2}
          \end{align*}
          and
          \begin{align*}
            (\extSRaw{(\lubstore{S}{S''})}{l}{d_2})(l) &=
            \userlub{(\lubstore{S}{S''})(l)}{(\store{\storebindingRaw{l}{d_2}})(l)}
            \\ &= \userlub{d'_1}{d_2} \\ &=
            \userlub{d'_1}{\userlub{d_1}{d_2}} \\ &=
            \userlub{d'_1}{d_2} & \textrm{(since $d_1 \userleq
              d'_1$).}
          \end{align*}
          Therefore $\extSRaw{(\lubstore{S}{S''})}{l}{d'_2} =
          \extSRaw{(\lubstore{S}{S''})}{l}{d_2}$.

          Therefore, $\config{\lubstore{S}{S''}}{\putexp{l}{d_2}}
          \parstepsto
          \config{\extSRaw{(\lubstore{S}{S''})}{l}{d_2}}{\unit}$.
      \end{itemize}

      Note that:
      \begin{align*}
        \extSRaw{(\lubstore{S}{S''})}{l}{d_2} &=
        \lubstore{\extSRaw{S}{l}{d_2}}{\extSRaw{S''}{l}{d_2}} \\ &=
        \lubstore{\lubstore{S}{\store{\storebindingRaw{l}{d_2}}}}{\lubstore{S''}{\store{\storebindingRaw{l}{d_2}}}}
        \\ &=
        \lubstore{\lubstore{S}{\store{\storebindingRaw{l}{d_2}}}}{S''}
        \\ &= \lubstore{\extSRaw{S}{l}{d_2}}{S''}.
      \end{align*}
      Therefore $\config{\lubstore{S}{S''}}{\putexp{l}{d_2}}
      \parstepsto
      \config{\lubstore{\extSRaw{S}{l}{d_2}}{S''}}{\unit}$, as we were
      required to show.

    \item Case {\sc E-Get}:

      Given: $\config{S}{\getexp{l}{T}} \parstepsto \config{S}{d_2}$.

      To show: $\config{\lubstore{S}{S''}}{\getexp{l}{T}} \parstepsto
      \config{\lubstore{S}{S''}}{d_2}$.

      From the premises of {\sc E-Get}, $S(l) = d_1$ and $\incomp{T}$
      and $d_2 \in T$ and $d_2 \userleq d_1$.

      By assumption, $\lubstore{S}{S''} \neq \topS$.

      Hence $(\lubstore{S}{S''}) = d'_1$, where $d_1 \userleq d'_1$.

      By the transitivity of $\userleq$, $d_2 \userleq d'_1$.

      Hence, $S(l) = d'_1$ and $\incomp{T}$ and $d_2 \in T$ and $d_2
      \userleq d'_1$.

      Therefore, by {\sc E-Get},

      $\config{\lubstore{S}{S''}}{\getexp{l}{T}} \parstepsto
      \config{\lubstore{S}{S''}}{d_2}$,

      as we were required to show.
  \end{itemize}
\end{proof}


\section{Proof of Lemma~\ref{lem:lvars-clash}}\label{section:lvars-clash-proof}
\begin{proof}
  Consider arbitrary $S''$ such that $S''$ is non-conflicting with
  $\config{S}{e} \parstepsto \config{S'}{e'}$ and $\lubstore{S'}{S''}
  = \topS$.

  To show: $\config{\lubstore{S}{S''}}{e} \parstepsto \error$.

  The proof is by induction on the derivation of $\config{S}{e}
  \parstepsto \config{S'}{e'}$, by cases on the last rule in the
  derivation.  In every case we may assume that $\config{S'}{e'} \neq
  \error$.  Since $\config{S'}{e'} \neq \error$, we do not need to
  consider the {\sc E-Put-Err} rule.

  \begin{itemize}

    \item Case {\sc E-Eval-Ctxt}:

      Given: $\config{S}{\E{e}} \parstepsto \config{S'}{\E{e'}}$.

      To show: $\config{\lubstore{S}{S''}}{\E{e}} \parstepsto^i
      \error$, where $i \leq 1$.

      From the premise of {\sc E-Eval-Ctxt}, we have that
      $\config{S}{e} \parstepsto \config{S'}{e'}$.

      Therefore, by IH, we have that $\config{\lubstore{S}{S''}}{e}
      \parstepsto^{i'} \error$, where $i' \leq 1$.

      We proceed by cases on $i'$:

      \begin{itemize}
        \item $i' = 0$:

          In this case, $\config{\lubstore{S}{S''}}{e} = \error$.

          Hence, by the definition of $\error$, $\lubstore{S}{S''} =
          \topS$.

          Hence $\config{\lubstore{S}{S''}}{\E{e}} = \error$.

          Hence $\config{\lubstore{S}{S''}}{\E{e}} \parstepsto^i
          \error$, with $i = 0$.

        \item $i' = 1$:

          In this case, $\config{\lubstore{S}{S''}}{e} \parstepsto
          \error$.

          By the definition of $\error$, $\error =
          \config{\topS}{e''}$ for any $e''$.

          Hence $\config{\lubstore{S}{S''}}{e} \parstepsto
          \config{\topS}{e''}$.

          Hence, by {\sc E-Eval-Ctxt},
          $\config{\lubstore{S}{S''}}{\E{e}} \parstepsto
          \config{\topS}{\E{e''}}$.

          By the definition of $\error$, $\config{\topS}{\E{e''}} =
          \error$.

          Hence $\config{\lubstore{S}{S''}}{\E{e}} \parstepsto
          \error$.

          Hence $\config{\lubstore{S}{S''}}{\E{e}} \parstepsto^i
          \error$, with $i = 1$.

      \end{itemize}

    \item Case {\sc E-Beta}:

      Given: $\config{S}{\app{(\lam{x}{e})}{v}} \parstepsto
      \config{S}{\subst{e}{x}{v}}$.

      To show: $\config{\lubstore{S}{S''}}{\app{(\lam{x}{e})}{v}}
      \parstepsto^i \error$, where $i \leq 1$.

      By assumption, $\lubstore{S}{S''} = \topS$.

      Hence, by the definition of $\error$,
      $\config{\lubstore{S}{S''}}{\app{(\lam{x}{e})}{v}} = \error$.

      Hence $\config{\lubstore{S}{S''}}{\app{(\lam{x}{e})}{v}}
      \parstepsto^i \error$, with $i = 0$.

    \item Case {\sc E-New}:

      Given: $\config{S}{\NEW} \parstepsto
      \config{\extSRaw{S}{l}{\bot}}{l}$.

      To show: $\config{\lubstore{S}{S''}}{\NEW} \parstepsto^i
      \error$, where $i \leq 1$.

      By {\sc E-New}, $\config{\lubstore{S}{S''}}{\NEW} \parstepsto
      \config{\extSRaw{(\lubstore{S}{S''})}{l'}{\bot}}{l'}$, where $l'
      \notin \dom{\lubstore{S}{S''}}$.

      By assumption, $S''$ is non-conflicting with $\config{S}{\NEW}
      \parstepsto \config{\extSRaw{S}{l}{\bot}}{l}$.
 
      Therefore $l \notin \dom{S''}$.

      From the side condition of {\sc E-New}, $l \notin \dom{S}$.

      Therefore $l \notin \dom{\lubstore{S}{S''}}$.

      Therefore, in
      $\config{\extSRaw{(\lubstore{S}{S''})}{l'}{\bot}}{l'}$, we can
      $\alpha$-rename $l'$ to $l$, \\ resulting in
      $\config{\extSRaw{(\lubstore{S}{S''})}{l}{\bot}}{l}$.

      Therefore $\config{\lubstore{S}{S''}}{\NEW} \parstepsto
      \config{\extSRaw{(\lubstore{S}{S''})}{l}{\bot}}{l}$.

      By assumption, $\lubstore{\extSRaw{S}{l}{\bot}}{S''}
      = \topS$.

      Note that:
      \begin{align*}
        \topS &= \lubstore{\extSRaw{S}{l}{\bot}}{S''} \\ &=
        \lubstore{\lubstore{S}{\store{\storebindingRaw{l}{\bot}}}}{S''}
        \\ &=
        \lubstore{\lubstore{S}{S''}}{\store{\storebindingRaw{l}{\bot}}}
        \\ &=
        \lubstore{(\lubstore{S}{S''})}{\store{\storebindingRaw{l}{\bot}}}
        \\ &= \extSRaw{(\lubstore{S}{S''})}{l}{\bot} .
      \end{align*}

      Hence $\config{\lubstore{S}{S''}}{\NEW} \parstepsto
      \config{\topS}{l}$.

      Hence, by the definition of $\error$,
      $\config{\lubstore{S}{S''}}{\NEW} \parstepsto \error$.

      Hence $\config{\lubstore{S}{S''}}{\NEW} \parstepsto^i \error$,
      with $i = 1$.

    \item Case {\sc E-Put}:

      Given: $\config{S}{\putexp{l}{d_2}} \parstepsto
      \config{\extSRaw{S}{l}{d_2}}{\unit}$.

      To show: $\config{\lubstore{S}{S''}}{\putexp{l}{d_2}}
      \parstepsto^i \error$, where $i \leq 1$.

      We proceed by cases on $\lubstore{S}{S''}$:

      \begin{itemize}

        \item $\lubstore{S}{S''} = \topS$:

          In this case, by the definition of $\error$,
          $\config{\lubstore{S}{S''}}{\putexp{l}{d_2}} = \error$.

          Hence $\config{\lubstore{S}{S''}}{\putexp{l}{d_2}}
          \parstepsto^i \error$, with $i = 0$.

        \item $\lubstore{S}{S''} \neq \topS$:

          From the premises of {\sc E-Put}, we have that $S(l) = d_1$.

          Hence $(\lubstore{S}{S''})(l) = d'_1$, where $d_1 \userleq
          d'_1$.

          We show that $\userlub{d'_1}{d_2} =
          \top$, as follows:

          By assumption, $\lubstore{\extSRaw{S}{l}{d_2}}{S''} = \topS$.

          Hence, by Definition~\ref{def:lvars-lubstore}, there exists
          some $l' \in \dom{\extSRaw{S}{l}{d_2}} \cap \dom{S''}$ such
          that $\userlub{(\extSRaw{S}{l}{d_2})(l')}{S''(l')} = \top$.

          Now case on $l'$:

          \begin{itemize}
            \item $l' \neq l$:

              In this case, $(\extSRaw{S}{l}{d_2})(l') = S(l')$.

              Since $\userlub{(\extSRaw{S}{l}{d_2})(l')}{S''(l')} = \top$,
              we then have that $\userlub{S(l')}{S''(l')} = \top$.

              However, this is a contradiction since
              $\lubstore{S}{S''} \neq \topS$.

              Hence this case cannot occur.

            \item $l' = l$:

              Then $\userlub{(\extSRaw{S}{l}{d_2})(l)}{S''(l)} = \top$.

              Note that:
              \begin{align*}
                \top &= \userlub{(\extSRaw{S}{l}{d_2})(l)}{S''(l)} \\ &=
                \userlub{d_2}{S''(l)} \\ &=
                \userlub{\userlub{d_1}{d_2}}{S''(l)}
                \\ &=
                \userlub{\userlub{S(l)}{d_2}}{S''(l)}
                \\ &=
                \userlub{\userlub{S(l)}{S''(l)}}{d_2}
                \\ &=
                \userlub{(\lubstore{S}{S''})(l)}{d_2}
                \\ &= \userlub{d'_1}{d_2}.
              \end{align*}
              Hence $\userlub{d'_1}{d_2} = \top$.

              Hence, by {\sc E-Put-Err},
              $\config{\lubstore{S}{S''}}{\putexp{l}{d_2}} \parstepsto
              \error$.

              Hence $\config{\lubstore{S}{S''}}{\putexp{l}{d_2}}
              \parstepsto^i \error$, with $i = 1$.

          \end{itemize}

      \end{itemize}

    \item Case {\sc E-Get}:

      Given: $\config{S}{\getexp{l}{T}} \parstepsto \config{S}{d_2}$.

      To show: $\config{\lubstore{S}{S''}}{\getexp{l}{T}}
      \parstepsto^i \error$, where $i \leq 1$.

      By assumption, $\lubstore{S}{S''} = \topS$.

      Hence, by the definition of $\error$,
      $\config{\lubstore{S}{S''}}{\getexp{l}{T}} = \error$.

      Hence $\config{\lubstore{S}{S''}}{\getexp{l}{T}} \parstepsto^i
      \error$, with $i = 0$.
  \end{itemize}
\end{proof}


\section{Proof of Lemma~\ref{lem:lvars-error-preservation}}\label{section:lvars-error-preservation-proof}
\begin{proof}

  Given: $\config{S}{e} \parstepsto \error$ and $\leqstore{S}{S'}$.

  To show: $\config{S'}{e} \parstepsto \error$.

  \TODO{Figure out what to do here.  I think we need to handle both
    E-Eval-Ctxt and E-Put-Err.}
\end{proof}


\section{Proof of Lemma~\ref{lem:lvars-strong-local-confluence}}\label{section:lvars-strong-local-confluence-proof}
\begin{proof}
  Suppose $\conf \ctxstepsto \conf_a$ and $\conf \ctxstepsto \conf_b$.
  We have to show that there exist $\conf_c, i, j, \pi$ such that
  $\conf_a \ctxstepsto^i \conf_c$ and $\pi(\conf_b) \ctxstepsto^j
  \pi(\conf_c)$ and $i \leq 1$ and $j \leq 1$.

  By inspection of the operational semantics, it must be the case that
  $\conf$ steps to $\conf_a$ by the {\sc E-Eval-Ctxt} rule.  Let
  $\conf = \config{S}{\evalctxt{E_a}{e_{a_1}}}$ and let $\conf_a =
  \config{S_a}{\evalctxt{E_a}{e_{a_2}}}$.

  Likewise, it must be the case that $\conf$ steps to $\conf_b$ by the
  {\sc E-Eval-Ctxt} rule.  Let $\conf =
  \config{S}{\evalctxt{E_b}{e_{b_1}}}$ and let $\conf_b =
  \config{S_b}{\evalctxt{E_b}{e_{b_2}}}$.

  Note that $\conf = \config{S}{\evalctxt{E_a}{e_{a_1}}} =
  \config{S}{\evalctxt{E_b}{e_{b_1}}}$, and so
  $\evalctxt{E_a}{e_{a_1}} = \evalctxt{E_b}{e_{b_1}}$, but $E_a$ and
  $E_b$ may differ and $e_{a_1}$ and $e_{b_1}$ may differ.

  Since $\config{S}{\evalctxt{E_a}{e_{a_1}}} \ctxstepsto
  \config{S_a}{\evalctxt{E_a}{e_{a_2}}}$ and
  $\config{S}{\evalctxt{E_b}{e_{b_1}}} \ctxstepsto
  \config{S_b}{\evalctxt{E_b}{e_{b_2}}}$ and $\evalctxt{E_a}{e_{a_1}}
  = \evalctxt{E_b}{e_{b_1}}$, we have from
  Lemma~\ref{lem:lvars-locality} (Locality) that there exist
  evaluation contexts $E'_a$ and $E'_b$ such that:

  \begin{itemize}
  \item $\evalctxt{E'_a}{e_{a_1}} = \evalctxt{E_b}{e_{b_2}}$, and
  \item $\evalctxt{E'_b}{e_{b_1}} = \evalctxt{E_a}{e_{a_2}}$, and
  \item $\evalctxt{E'_a}{e_{a_2}} =
  \evalctxt{E'_b}{e_{b_2}}$.
  \end{itemize}

  Our approach will be to show that there exist $S', i, j, \pi$ such
  that:
  \begin{itemize}
  \item $\config{S_a}{\evalctxt{E_a}{e_{a_2}}} \ctxstepsto^i
    \config{S'}{\evalctxt{E'_a}{e_{a_2}}}$, and
  \item $\pi(\config{S_b}{\evalctxt{E_b}{e_{b_2}}}) \ctxstepsto^j
    \pi(\config{S'}{\evalctxt{E'_a}{e_{a_2}}})$.
  \end{itemize}
  Since $\evalctxt{E'_a}{e_{a_1}} = \evalctxt{E_b}{e_{b_2}}$,
  $\evalctxt{E'_b}{e_{b_1}} = \evalctxt{E_a}{e_{a_2}}$, and
  $\evalctxt{E'_a}{e_{a_2}} = \evalctxt{E'_b}{e_{b_2}}$, it suffices
  to show that:
  \begin{itemize}
  \item $\config{S_a}{\evalctxt{E'_b}{e_{b_1}}} \ctxstepsto^i
    \config{S'}{\evalctxt{E'_b}{e_{b_2}}}$, and
  \item $\pi(\config{S_b}{\evalctxt{E'_a}{e_{a_1}}}) \ctxstepsto^j
    \pi(\config{S'}{\evalctxt{E'_a}{e_{a_2}}})$.
  \end{itemize}

  From the premise of {\sc E-Eval-Ctxt}, we have that
  $\config{S}{e_{a_1}} \parstepsto \config{S_a}{e_{a_2}}$ and
  $\config{S}{e_{b_1}} \parstepsto \config{S_b}{e_{b_2}}$.  We proceed
  by case analysis on the rule by which $\config{S}{e_{a_1}}$ steps to
  $\config{S_a}{e_{a_2}}$.

  \begin{enumerate}
  \item Case {\sc E-Beta}:

    We have:
    \begin{itemize}
      \item $e_{a_1} = \app{\lam{x}{e'_a}}{v_a}$,
      \item $e_{a_2} = \subst{e'_a}{x}{v_a}$, and
      \item $S_a = S$.
    \end{itemize}

    Now, we proceed by case analysis on the rule by which
    $\config{S}{e_{b_1}}$ steps to $\config{S_b}{e_{b_2}}$:
    \begin{enumerate}
    \item Case {\sc E-Beta}:

      We have:
      \begin{itemize}
      \item $e_{b_1} = \app{\lam{x}{e'_b}}{v_b}$,
      \item $e_{b_2} = \subst{e'_b}{x}{v_b}$, and
      \item $S_b = S$.
      \end{itemize}

      Choose $S' = S$, $i = 1$, $j = 1$, and $\pi = \id$.

      We have to show that:

      \begin{itemize}
      \item $\config{S}{\evalctxt{E'_b}{e_{b_1}}} \ctxstepsto
        \config{S}{\evalctxt{E'_b}{e_{b_2}}}$, and
      \item $\config{S}{\evalctxt{E'_a}{e_{a_1}}} \ctxstepsto
        \config{S}{\evalctxt{E'_a}{e_{a_2}}}$, 
      \end{itemize}

      both of which follow immediately from $\config{S}{e_{a_1}}
      \parstepsto \config{S_a}{e_{a_2}}$ and $\config{S}{e_{b_1}}
      \parstepsto \config{S_b}{e_{b_2}}$ and {\sc E-Eval-Ctxt}.

    \item Case {\sc E-New}:

      We have:
      \begin{itemize}
      \item $e_{b_1} = \NEW$,
      \item $e_{b_2} = l$, and
      \item $S_b = \extSRaw{S}{l}{\bot}$.
      \end{itemize}

      Choose $S' = S_b$, $i = 1$, $j = 1$, and $\pi = \id$.

      We have to show that:

      \begin{itemize}
      \item $\config{S}{\evalctxt{E'_b}{e_{b_1}}} \ctxstepsto
        \config{S_b}{\evalctxt{E'_b}{e_{b_2}}}$, and
      \item
        $\config{S_b}{\evalctxt{E'_a}{e_{a_1}}} \ctxstepsto
        \config{S_b}{\evalctxt{E'_a}{e_{a_2}}}$.
      \end{itemize}

      The first of these follows immediately from $\config{S}{e_{b_1}}
      \parstepsto \config{S_b}{e_{b_2}}$ and {\sc E-Eval-Ctxt}.  For
      the second, consider that $S_b = \extSRaw{S}{l}{\bot} =
      \lubstore{S}{\store{\storebindingRaw{l}{\bot}}}$.  Furthermore, we
      know from the side condition of {\sc E-New} that $l \notin
      \dom{S}$, so $\store{\storebindingRaw{l}{\bot}}$ is non-conflicting
      with the transition $\config{S}{e_{a_1}} \parstepsto
      \config{S_a}{e_{a_2}}$, and we know that
      $\lubstore{S_a}{\store{\storebindingRaw{l}{\bot}}} \neq \topS$
      since $S_a$ is just $S$.  Therefore, by
      Lemma~\ref{lem:lvars-independence} (Independence), we have that
      $\config{\lubstore{S}{\store{\storebindingRaw{l}{\bot}}}}{e_{a_1}}
      \parstepsto
      \config{\lubstore{S_a}{\store{\storebindingRaw{l}{\bot}}}}{e_{a_2}}$.
      Hence $\config{S_b}{e_{a_1}} \parstepsto \config{S_b}{e_{a_2}}$.
      By {\sc E-Eval-Ctxt}, it follows that
      $\config{S_b}{\evalctxt{E'_a}{e_{a_1}}} \ctxstepsto
      \config{S_b}{\evalctxt{E'_a}{e_{a_2}}}$, as we were required to
      show.

    \item Case {\sc E-Put}: \TODO{}
    \item Case {\sc E-Put-Err}: \TODO{}
    \item Case {\sc E-Get}:\TODO{}
    \end{enumerate}
  \item Case {\sc E-New}:

    Now, we proceed by case analysis on the rule by which
    $\config{S}{e_{b_1}}$ steps to $\config{S_b}{e_{b_2}}$:
    \begin{enumerate}
    \item Case {\sc E-Beta}: \TODO{}
    \item Case {\sc E-New}: \TODO{}
    \item Case {\sc E-Put}: \TODO{}
    \item Case {\sc E-Put-Err}: \TODO{}
    \item Case {\sc E-Get}: \TODO{}
    \end{enumerate}
  \item Case {\sc E-Put}:

    Now, we proceed by case analysis on the rule by which
    $\config{S}{e_{b_1}}$ steps to $\config{S_b}{e_{b_2}}$:
    \begin{enumerate}
    \item Case {\sc E-Beta}: \TODO{}
    \item Case {\sc E-New}: \TODO{}
    \item Case {\sc E-Put}: \TODO{}
    \item Case {\sc E-Put-Err}: \TODO{}
    \item Case {\sc E-Get}: \TODO{}
    \end{enumerate}
  \item Case {\sc E-Put-Err}:

    Now, we proceed by case analysis on the rule by which
    $\config{S}{e_{b_1}}$ steps to $\config{S_b}{e_{b_2}}$:
    \begin{enumerate}
    \item Case {\sc E-Beta}: \TODO{}
    \item Case {\sc E-New}: \TODO{}
    \item Case {\sc E-Put}: \TODO{}
    \item Case {\sc E-Put-Err}: \TODO{}
    \item Case {\sc E-Get}: \TODO{}
    \end{enumerate}
  \item Case {\sc E-Get}:

    Now, we proceed by case analysis on the rule by which
    $\config{S}{e_{b_1}}$ steps to $\config{S_b}{e_{b_2}}$:
    \begin{enumerate}
    \item Case {\sc E-Beta}: \TODO{}
    \item Case {\sc E-New}: \TODO{}
    \item Case {\sc E-Put}: \TODO{}
    \item Case {\sc E-Put-Err}: \TODO{}
    \item Case {\sc E-Get}: \TODO{}
    \end{enumerate}
  \end{enumerate}

  \lk{I think we also still have to separately deal with cases where
    $\conf_a = \error$ or $\conf_b = \error$.}
\end{proof}


\section{Proof of Lemma~\ref{lem:lvars-strong-one-sided-confluence}}\label{section:lvars-strong-one-sided-confluence-proof}
\begin{proof}
  Suppose $\conf \ctxstepsto \conf'$ and $\conf \ctxstepsto^m
  \conf''$, where $1 \leq m$.  We have to show that there exist
  $\conf_c, i, j, \pi$ such that $\conf' \ctxstepsto^i \conf_c$ and
  $\pi(\conf'') \ctxstepsto^j \conf_c$ and $i \leq m$ and $j \leq 1$.

  We proceed by induction on $m$.  In the base case of $m = 1$, the
  result is immediate from
  Lemma~\ref{lem:lvars-strong-local-confluence}.

  For the induction step, suppose $\conf \ctxstepsto^m \conf''
  \ctxstepsto \conf'''$ and suppose the lemma holds for $m$.

  We show that it holds for $m + 1$, as follows.

  We are required to show that there exist $\conf_c, i, j, \pi$ such
  that $\conf' \ctxstepsto^{i} \conf_c$ and $\pi(\conf''')
  \ctxstepsto^{j} \conf_c$ and $i \leq m + 1$ and $j \leq 1$.

  From the induction hypothesis, there exist $\conf_c', i', j', \pi'$
  such that $\conf' \ctxstepsto^{i'} \conf_c'$ and $\pi'(\conf'')
  \ctxstepsto^{j'} \conf_c'$ and $i' \leq m$ and $j' \leq 1$.

  We proceed by cases on $j'$:
  \begin{itemize}

  \item If $j' = 0$, then $\pi'(\conf'') = \conf_c'$.

    Since $\conf'' \ctxstepsto \conf'''$, we have that $\pi'(\conf'')
    \ctxstepsto \pi'(\conf''')$ by
    Lemma~\ref{lem:lvars-permutability} (Permutability).

    We can then choose $\conf_c = \pi'(\conf''')$ and $i = i' + 1$ and
    $j = 0$ and $\pi = \pi'$.  The key is that $\conf'
    \ctxstepsto^{i'} \conf'_c = \pi'(\conf'') \ctxstepsto
    \pi'(\conf''')$ for a total of $i' + 1$ steps.
    
  \item If $j' = 1$:

    First, since $\pi'(\conf'') \ctxstepsto^{j'} \conf'_c$, then by
    Lemma~\ref{lem:lvars-permutability} (Permutability) we have that
    $\conf'' \ctxstepsto^{j'} \piprimeinv(\conf'_c)$.

    Then, by $\conf'' \ctxstepsto^{j'} \piprimeinv(\conf'_c)$ and
    $\conf'' \ctxstepsto \conf'''$ and
    Lemma~\ref{lem:lvars-strong-local-confluence} (Strong Local
    Confluence), we have that there exist $\conf_c''$ and $i''$ and
    $j''$ and $\pi''$ such that $\piprimeinv(\conf'_c)
    \ctxstepsto^{i''} \conf_c''$ and $\pi''(\conf''')
    \ctxstepsto^{j''} \conf_c''$ and $i'' \leq 1$ and $j'' \leq 1$.

    Since $\piprimeinv(\conf'_c) \ctxstepsto^{i''} \conf_c''$, by
    Lemma~\ref{lem:lvars-permutability} (Permutability) we have that
    $\conf'_c \ctxstepsto^{i''} \pi'(\conf_c'')$.

    So we also have $\conf' \ctxstepsto^{i'} \conf_c'
    \ctxstepsto^{i''} \pi'(\conf_c'')$.

    Since $\pi''(\conf''') \ctxstepsto^{j''} \conf_c''$, by
    Lemma~\ref{lem:lvars-permutability} (Permutability) we have that
    $\pi'(\pi''(\conf''')) \ctxstepsto^{j''} \pi'(\conf_c'')$.

    In summary, we pick $\conf_c = \pi'(\conf_c'')$ and $i = i' + i''$
    and $j = j''$ and $\pi = \pi'' \circ \pi'$, which is sufficient
    because $i = i' + i'' \leq m + 1$ and $j = j'' \leq 1$.
  \end{itemize}

 \end{proof}


\section{Proof of Lemma~\ref{lem:lvars-strong-confluence}}\label{section:lvars-strong-confluence-proof}
\begin{proof}
  We proceed by induction on $n$.  In the base case of $n = 1$, the
  result is immediate from
  Lemma~\ref{lem:lvars-strong-one-sided-confluence}.

  For the induction step, suppose $\conf \parstepsto^n \conf'
  \parstepsto \conf'''$ and suppose the lemma holds for $n$.

  We show that it holds for $n + 1$, as follows.

  We are required to show that there exist $\conf_c, i, j$ such that
  $\conf''' \parstepsto^i \conf_c$ and $\conf'' \parstepsto^j \conf_c$
  and $i \leq m$ and $j \leq n + 1$.

  From the induction hypothesis, we have that there exist $\conf'_c,
  i', j'$ such that $\conf' \parstepsto^{i'} \conf'_c$ and $\conf''
  \parstepsto^{j'} \conf'_c$ and $i' \leq m$ and $j' \leq n$.

  We proceed by cases on $i'$:
  \begin{itemize}

  \item If $i' = 0$, then $\conf' = \conf_c'$.  We can then choose
    $\conf_c = \conf'''$ and $i = 0$ and $j = j' + 1$.

  \item If $i' \geq 1$:

    From $\conf' \parstepsto \conf'''$ and $\conf' \parstepsto^{i'}
    \conf_c'$ and Lemma~\ref{lem:lvars-strong-one-sided-confluence},
    we have that there exist $\conf_c''$ and $i''$ and $j''$ such that
    $\conf''' \parstepsto^{i''} \conf_c''$ and $\conf_c'
    \parstepsto^{j''} \conf_c''$ and $i'' \leq i'$ and $j'' \leq 1$.
    So we also have $\conf'' \parstepsto^{j'} \conf_c'
    \parstepsto^{j''} \conf_c''$.  In summary, we pick $\conf_c =
    \conf_c''$ and $i = i''$ and $j = j' + j''$, which is sufficient
    because $i = i'' \leq i' \leq m$ and $j = j' + j'' \leq n + 1$.
  \end{itemize}

\end{proof}


\section{Proof of Lemma~\ref{lem:lattice-structure}}\label{section:lattice-structure-proof}
\begin{proof}
  Suppose that $(D, \userleq, \bot, \top)$ is a lattice and $(D_p,
  \leqp, \botp, \topp) = \Freeze{D, \userleq, \bot, \top}$.

  In order to show that $(D_p, \leqp, \botp, \topp)$ is a lattice, we
  have to show that:
  \begin{enumerate}
  \item $\leqp$ is a partial order over $D_p$.

  \item Every nonempty finite subset of $D_p$ has a lub.

  \item $\botp$ is the least element of $D_p$.

  \item $\topp$ is the greatest element of $D_p$.
  \end{enumerate}

  We prove each of these properties in turn:

  \begin{enumerate}
  \item $\leqp$ is a partial order over $D_p$.

    To show this, we need to show that $\leqp$ is reflexive, transitive,
    and antisymmetric. 
    \begin{enumerate}
    \item $\leqp$ is reflexive.

      Suppose $v \in D_p$.

      Then, by Lemma~\ref{lem:partition-of-Dp}, either $v =
      \state{d}{\frozenfalse}$ with $d \in D$, or $v =
      \state{x}{\frozentrue}$ with $x \in X$, where $X = D -
      \setof{\top}$.
      \begin{itemize}
      \item Suppose $v = \state{d}{\frozenfalse}$:

        By the reflexivity of $\userleq$, we know $d \userleq d$.

        By the definition of $\leqp$, we know $\state{d}{\frozenfalse}
        \leqp \state{d}{\frozenfalse}$.

      \item Suppose $v = \state{x}{\frozentrue}$: 
        
        By the reflexivity of equality, $x = x$.

        By the definition of $\leqp$, we know $\state{x}{\frozentrue}
        \leqp \state{x}{\frozentrue}$.
      \end{itemize}

    \item $\leqp$ is transitive. 

      Suppose $v_1 \leqp v_2$ and $v_2 \leqp v_3$.

      We want to show that $v_1 \leqp v_3$.

      We proceed by case analysis on $v_1, v_2$, and $v_3$.
      \begin{itemize}
      \item Case $v_1 = \state{d_1}{\frozenfalse}$ and $v_2 =
        \state{d_2}{\frozenfalse}$ and $v_3 =
        \state{d_3}{\frozenfalse}$:
        
        By inversion on $\leqp$, it follows that $d_1 \userleq d_2$.

        By inversion on $\leqp$, it follows that $d_2 \userleq d_3$.

        By the transitivity of $\userleq$, we know $d_1 \userleq d_3$.

        By the definition of $\leqp$, it follows that
        $\state{d_1}{\frozenfalse} \leqp \state{d_3}{\frozenfalse}$.

        Hence $v_1 \leqp v_3$.

      \item Case $v_1 = \state{d_1}{\frozenfalse}$ and $v_2 =
        \state{d_2}{\frozenfalse}$ and $v_3 =
        \state{x_3}{\frozentrue}$:

        By inversion on $\leqp$, it follows that $d_1 \userleq d_2$.

        By inversion on $\leqp$, it follows that $d_2 \userleq x_3$.

        By the transitivity of $\userleq$, we know $d_1 \userleq x_3$.

        By the definition of $\leqp$, it follows that
        $\state{d_1}{\frozenfalse} \leqp \state{x_3}{\frozentrue}$.

        Hence $v_1 \leqp v_3$.

      \item Case $v_1 = \state{d_1}{\frozenfalse}$ and $v_2 =
        \state{x_2}{\frozentrue}$ and $v_3 =
        \state{d_3}{\frozenfalse}$:

        By inversion on $\leqp$, it follows that $d_1 \userleq x_2$.

        By inversion on $\leqp$, it follows that $d_3 = \top$.

        Since $\top$ is the maximal element of $D$, we know $d_1
        \userleq \top \equiv d_3$.

        By the definition of $\leqp$, it follows that
        $\state{d_1}{\frozenfalse} \leqp \state{d_3}{\frozenfalse}$.

        Hence $v_1 \leqp v_3$.

      \item Case $v_1 = \state{d_1}{\frozenfalse}$ and $v_2 =
        \state{x_2}{\frozentrue}$ and $v_3 =
        \state{x_3}{\frozentrue}$:

        By inversion on $\leqp$, it follows that $d_1 \userleq x_2$.

        By inversion on $\leqp$, it follows that $x_2 = x_3$.

        Hence $d_1 \userleq x_3$.

        By the definition of $\leqp$, it follows that
        $\state{d_1}{\frozenfalse} \leqp \state{x_3}{\frozentrue}$.

        Hence $v_1 \leqp v_3$.

      \item Case $v_1 = \state{x_1}{\frozentrue}$ and $v_2 =
        \state{d_2}{\frozenfalse}$ and $v_3 =
        \state{d_3}{\frozenfalse}$:

        By inversion on $\leqp$, it follows that $d_2 = \top$.

        By inversion on $\leqp$, it follows that $d_2 \userleq d_3$.

        Since $\top$ is maximal, it follows that $d_3 = \top$.

        By the definition of $\leqp$, it follows that
        $\state{x_1}{\frozentrue} \leqp \state{d_3}{\frozenfalse}$.

        Hence $v_1 \leqp v_3$. 

      \item Case $v_1 = \state{x_1}{\frozentrue}$ and $v_2 =
        \state{d_2}{\frozenfalse}$ and $v_3 =
        \state{x_3}{\frozentrue}$:

        By inversion on $\leqp$, it follows that $d_2 = \top$.

        By inversion on $\leqp$, it follows that $d_2 \userleq x_3$.

        Since $\top$ is maximal, it follows that $x_3 = \top$.

        But since $x_3 \in X \subseteq D/\setof{\top}$, we know $x_3
        \not= \top$.

        This is a contradiction. \\

        Hence $v_1 \leqp v_3$. 

      \item Case $v_1 = \state{x_1}{\frozentrue}$ and $v_2 =
        \state{x_2}{\frozentrue}$ and $v_3 =
        \state{d_3}{\frozenfalse}$:

        By inversion on $\leqp$, it follows that $x_1 = x_2$.

        By inversion on $\leqp$, it follows that $d_3 = \top$.

        By the definition of $\leqp$, it follows that
        $\state{x_1}{\frozentrue} \leqp \state{d_3}{\frozenfalse}$.

        Hence $v_1 \leqp v_3$. 

      \item Case $v_1 = \state{x_1}{\frozentrue}$ and $v_2 =
        \state{x_2}{\frozentrue}$ and $v_3 =
        \state{x_3}{\frozentrue}$:

        By inversion on $\leqp$, it follows that $x_1 = x_2$.

        By inversion on $\leqp$, it follows that $x_2 = x_3$.

        By transitivity of $=$, $x_1 = x_3$.

        By the definition of $\leqp$, it follows that
        $\state{x_1}{\frozentrue} \leqp \state{x_3}{\frozentrue}$.

        Hence $v_1 \leqp v_3$. 
        
      \end{itemize}

    \item $\leqp$ is antisymmetric. 

      Suppose $v_1 \leqp v_2$ and $v_2 \leqp v_1$. Now, we proceed by
      cases on $v_1$ and $v_2$.
      \begin{itemize}
      \item Case $v_1 = \state{d_1}{\frozenfalse}$ and $v_2 =
        \state{d_2}{\frozenfalse}$:
        
        By inversion on $v_1 \leqp v_2$, we know that $d_1 \userleq
        d_2$.

        By inversion on $v_2 \leqp v_1$, we know that $d_2 \userleq
        d_1$.

        By the antisymmetry of $\leq$, we know $d_1 = d_2$.

        Hence $v_1 = v_2$. 

      \item Case $v_1 = \state{d_1}{\frozenfalse}$ and $v_2 =
        \state{x_2}{\frozentrue}$:

        By inversion on $v_1 \leqp v_2$, we know that $d_1 \userleq x_2$.

        By inversion on $v_2 \leqp v_1$, we know that $d_1 = \top$.

        Since $\top$ is maximal in $D$, we know $x_2 = \top$.

        But since $x_2 \in X \subseteq D/\setof{\top}$, we know $x_2 \not= \top$.

        This is a contradiction.

        Hence $v_1 = v_2$. 
        
      \item Case $v_1 = \state{x_1}{\frozentrue}$ and $v_2 =
        \state{d_2}{\frozenfalse}$:

        Similar to the previous case. 

      \item Case $v_1 = \state{x_1}{\frozentrue}$ and $v_2 =
        \state{x_2}{\frozentrue}$:

        By inversion on $v_1 \leqp v_2$, we know that $x_1 = x_2$.

        Hence $v_1 = v_2$. 
      \end{itemize}
    \end{enumerate}

  \item Every nonempty finite subset of $D_p$ has a lub.

    To show this, it is sufficient to show that every two elements of
    $D_p$ have a lub, since a binary lub operation can be repeatedly
    applied to compute the lub of any finite set.

    We will show that every two elements of $D_p$ have a lub by
    showing that the $\lubp{}{}$ operation defined by
    Definition~\ref{def:lubp} computes their lub.

    It suffices to show the following two properties:
    \begin{enumerate}
    \item For all $v_1, v_2, v \in D_p$, if $v_1 \leqp v$ and $v_2
      \leqp v$, then $(\lubp{v_1}{v_2}) \leqp v$.
    \item For all $v_1, v_2 \in D_p$, $v_1 \leqp (\lubp{v_1}{v_2})$
      and $v_2 \leqp (\lubp{v_1}{v_2})$.
    \end{enumerate}
    \begin{enumerate}
    \item For all $v_1, v_2, v \in D_p$, if $v_1 \leqp v$ and $v_2
      \leqp v$, then $\lubp{v_1}{v_2} \leqp v$.
      
      Assume $v_1, v_2, v \in D_p$, and $v_1 \leqp v$ and $v_2 \leqp
      v$.

      Now we do a case analysis on $v_1$ and $v_2$.
      \begin{itemize}
      \item Case $v_1 = \state{d_1}{\frozenfalse}$ and $v_2 =
        \state{d_2}{\frozenfalse}$.
        
        Now case on $v$: 
        \begin{itemize}
        \item Case $v = \state{d}{\frozenfalse}$: 

          By the definition of $\lubp{}{}$,
          $\lubp{\state{d_1}{\frozenfalse}}{\state{d_2}{\frozenfalse}}
          = \state{\userlub{d_1}{d_2}}{\frozenfalse}$.

          By inversion on $\state{d_1}{\frozenfalse} \leqp
          \state{d}{\frozenfalse}$, $d_1 \userleq l$.

          By inversion on $\state{d_2}{\frozenfalse} \leqp
          \state{d}{\frozenfalse}$, $d_2 \userleq l$.

          Hence $l$ is an upper bound for $d_1$ and $d_2$.

          Hence $\userlub{d_1}{d_2} \userleq l$.

          Hence $\state{\userlub{d_1}{d_2}}{\frozenfalse} \leqp
          \state{d}{\frozenfalse}$.

          Hence $\lubp{v_1}{v_2} \leqp v$.
          
        \item Case $v = \state{x}{\frozentrue}$: 
          
          By the definition of $\lubp{}{}$, $\state{d_1}{\frozenfalse}
          \lubp{}{} \state{d_2}{\frozenfalse} =
          \state{\userlub{d_1}{d_2}}{\frozenfalse}$.

          By inversion on $\state{d_1}{\frozenfalse} \leqp
          \state{x}{\frozentrue}$, $d_1 \userleq x$.

          By inversion on $\state{d_2}{\frozenfalse} \leqp
          \state{x}{\frozentrue}$, $d_2 \userleq x$.
     
          Hence $x$ is an upper bound for $d_1$ and $d_2$.

          Hence $\userlub{d_1}{d_2} \userleq x$.

          Hence $\state{\userlub{d_1}{d_2}}{\frozenfalse} \leqp
          \state{x}{\frozentrue}$.

          Hence $\lubp{v_1}{v_2} \leqp v$.
        \end{itemize}
        
      \item Case $v_1 = \state{x_1}{\frozentrue}$ and $v_2 =
        \state{x_2}{\frozentrue}$:
        
        Now case on $v$: 
        \begin{itemize}
        \item Case $v = \state{d}{\frozenfalse}$: 
          
          By inversion on $\state{x_1}{\frozentrue} \leqp
          \state{d}{\frozenfalse}$, we know $l = \top$.

          By inversion on $\state{x_2}{\frozentrue} \leqp
          \state{d}{\frozenfalse}$, we know $l = \top$.

          Now consider whether $x_1 = x_2$ or not.
        
          If it does, then by the definition of $\lubp{}{}$,
          $\state{x_1}{\frozentrue} \lubp{}{} \state{x_2}{\frozentrue}
          = \state{x_1}{\frozentrue}$.

          By definition of $\leqp$, we have $\state{x_1}{\frozentrue}
          \leqp \state{\top}{\frozenfalse}$.

          So $\lubp{v_1}{v_2} \leqp v$.

          If it does not, then $\lubp{v_1}{v_2} =
          \state{\top}{\frozenfalse}$.

          By the definition of $\leqp$, we have
          $\state{\top}{\frozenfalse} \leqp
          \state{\top}{\frozenfalse}$.

          So $\lubp{v_1}{v_2} \leqp v$.
          
        \item Case $v = \state{x}{\frozentrue}$: 
          
          By inversion on $\state{x_1}{\frozentrue} \leqp
          \state{x}{\frozentrue}$, we know $x = x_1$.

          By inversion on $\state{x_2}{\frozentrue} \leqp
          \state{x}{\frozentrue}$, we know $x = x_2$.

          Hence $x_1 = x_2$.

          By the definition of $\lubp{}{}$, $\state{x_1}{\frozentrue}
          \lubp{}{} \state{x_2}{\frozentrue} =
          \state{x_1}{\frozentrue}$.

          Hence $\lubp{v_1}{v_2} \leqp v$.
        \end{itemize}
        
      \item Case $v_1 = \state{x_1}{\frozentrue}$ and $v_2 =
        \state{d_2}{\frozenfalse}$:
        
        Now case on $v$:
        \begin{itemize}
        \item Case $v = \state{d}{\frozenfalse}$:
          
          Now consider whether $d_2 \userleq x_1$.

          If it is, then $\state{x_1}{\frozentrue} \lubp{}{}
          \state{d_2}{\frozenfalse} = \state{x_1}{\frozentrue} = v_1$.

          Hence $\lubp{v_1}{v_2} \leqp v$.

          Otherwise, $\state{x_1}{\frozentrue} \lubp{}{}
          \state{d_2}{\frozenfalse} = \state{\top}{\frozenfalse}$.

          By inversion on $\state{x_1}{\frozentrue} \leqp
          \state{d}{\frozenfalse}$, we know $l = \top$.

          By reflexivity, $\state{\top}{\frozenfalse} \leqp
          \state{\top}{\frozenfalse}$.

          Hence $\lubp{v_1}{v_2} \leqp v$. 
          
        \item Case $v = \state{x}{\frozentrue}$:  
          
          By inversion on $\state{x_1}{\frozentrue} \leqp
          \state{x}{\frozentrue}$, we know that $x_1 = x$.

          By inversion on $\state{d_2}{\frozenfalse} \leqp
          \state{x}{\frozentrue}$, we know that $d_2 \userleq x$.

          By transitivity, $d_2 \userleq x_1$.

          By the definition of $\lubp{}{}$, it follows that
          $\state{x_1}{\frozentrue} \lubp{}{}
          \state{d_2}{\frozenfalse} = \state{x_1}{\frozentrue}$.

          By definition of $\leqp$, $\state{x_1}{\frozentrue} \leqp
          \state{x_1}{\frozentrue}$.

          Hence $\lubp{v_1}{v_2} \leqp v$. 
        \end{itemize}
        
      \item Case $v_1 = \state{d_1}{\frozenfalse}$ and $v_2 =
        \state{x_2}{\frozentrue}$:
        
        Symmetric with the previous case. 
      \end{itemize}
    \item For all $v_1, v_2 \in D_p$, $v_1 \leqp \lubp{v_1}{v_2}$ and
      $v_2 \leqp \lubp{v_1}{v_2}$.
      
      Assume $v_1, v_2 \in D_p$, and proceed by case analysis. 
      \begin{itemize}
      \item Case $v_1 = \state{d_1}{\frozenfalse}$ and $v_2 =
        \state{d_2}{\frozenfalse}$:

        Since $\userlub{}{}$ is a join operator, we know $d_1 \userleq
        \userlub{d_1}{d_2}$.

        By the definition of $\leqp$, $\state{d_1}{\frozenfalse}
        \userleq \state{\userlub{d_1}{d_2}}{\frozenfalse}$.

        By the definition of $\lubp{}{}$, $\lubp{v_1}{v_2} =
        \state{\userlub{d_1}{d_2}}{\frozenfalse}$.

        Hence $v_1 \leqp \lubp{v_1}{v_2}$.

        Since $\userlub{}{}$ is a join operator, we know $d_1 \userleq
        \userlub{d_1}{d_2}$.

        By the definition of $\leqp$, $\state{d_2}{\frozenfalse}
        \userleq \state{\userlub{d_1}{d_2}}{\frozenfalse}$.

        By the definition of $\lubp{}{}$, $\lubp{v_1}{v_2} =
        \state{\userlub{d_1}{d_2}}{\frozenfalse}$.

        Hence $v_2 \leqp \lubp{v_1}{v_2}$. 

        Therefore $v_1 \leqp v_1 \userlub{}{} v_2$ and $v_2 \leqp v_1
        \userlub{}{} v_2$.
 
      \item Case $v_1 = \state{d_1}{\frozenfalse}$ and $v_2 = \state{x_2}{\frozentrue}$:

        Consider whether $d_1 \userleq x_2$. 
        \begin{itemize}
        \item Case  $d_1 \userleq x_2$:

          By the definition of $\lubp{}{}$, we know
          $\state{d_1}{\frozenfalse} \lubp{}{}
          \state{x_2}{\frozentrue} = \state{x_2}{\frozentrue}$.

          By the definition of $\lubp{}{}$, we know
          $\state{d_1}{\frozenfalse} \leqp \state{x_2}{\frozentrue}$.

          Hence $v_1 \leqp \lubp{v_1}{v_2}$.

          By reflexivity, $\state{x_2}{\frozentrue} \leqp
          \state{x_2}{\frozentrue}$.

          Hence $v_2 \leqp \lubp{v_1}{v_2}$.

          Therefore $v_1 \leqp v_1 \userlub{}{} v_2$ and $v_2 \leqp
          v_1 \userlub{}{} v_2$.

        \item Case $d_1 \not\userleq x_2$:

          By the definition of $\lubp{}{}$, we know
          $\state{d_1}{\frozenfalse} \lubp{}{}
          \state{x_2}{\frozentrue} = \state{\top}{\frozenfalse}$.

          Since $d_1 \userleq \top$, by the definition of $\leqp$ we
          know $\state{d_1}{\frozenfalse} \userleq
          \state{\top}{\frozenfalse}$.

          Hence $v_1 \leqp \lubp{v_1}{v_2}$.

          By the definition of $\leqp$, we know
          $\state{x_2}{\frozentrue} \userleq
          \state{\top}{\frozenfalse}$.

          Hence $v_2 \leqp \lubp{v_1}{v_2}$.

          Therefore $v_1 \leqp v_1 \userlub{}{} v_2$ and $v_2 \leqp
          v_1 \userlub{}{} v_2$.
        \end{itemize}
      \item Case $v_1 = \state{x_1}{\frozentrue}$ and $v_2 =
        \state{d_2}{\frozenfalse}$:

        Symmetric with the previous case. 
      \item Case $v_1 = \state{x_1}{\frozentrue}$ and $v_2 =
        \state{x_2}{\frozentrue}$:

        Consider whether $x_1$ equals $x_2$. 
        \begin{itemize}
        \item Case $x_1 = x_2$:
          
          By the definition $\lubp{}{}$, $\state{x_1}{\frozentrue}
          \lubp{}{} \state{x_2}{\frozentrue} =
          \state{x_1}{\frozentrue}$.
 
          By reflexivity, $\state{x_1}{\frozentrue} \leqp
          \state{x_1}{\frozentrue}$.

          Hence $v_1 \leqp \lubp{v_1}{v_2}$.

          By reflexivity, $\state{x_2}{\frozentrue} \leqp
          \state{x_1}{\frozentrue}$.

          Hence $v_2 \leqp \lubp{v_1}{v_2}$.

          Therefore $v_1 \leqp v_1 \userlub{}{} v_2$ and $v_2 \leqp
          v_1 \userlub{}{} v_2$.

        \item Case $x_1 \not= x_2$: 

          By the definition $\lubp{}{}$, $\state{x_1}{\frozentrue}
          \lubp{}{} \state{x_2}{\frozentrue} =
          \state{\top}{\frozenfalse}$.

          By the definition of $\leqp$, $\state{x_1}{\frozentrue}
          \leqp \state{\top}{\frozenfalse}$.

          Hence $v_1 \leqp \lubp{v_1}{v_2}$.

          By the definition of $\leqp$, $\state{x_2}{\frozentrue}
          \leqp \state{\top}{\frozenfalse}$.

          Hence $v_2 \leqp \lubp{v_1}{v_2}$.

          Therefore $v_1 \leqp v_1 \userlub{}{} v_2$ and $v_2 \leqp
          v_1 \userlub{}{} v_2$.
        \end{itemize}
      \end{itemize}
    \end{enumerate}

  \item $\botp$ is the least element of $D_p$. 

    $\botp$ is defined to be $\state{\bot}{\frozenfalse}$.

    In order to be the least element of $D_p$, it must be less than or
    equal to every element of $D_p$.

    By Lemma~\ref{lem:partition-of-Dp}, the elements of $D_p$
    partition into $\state{d}{\frozenfalse}$ for all $d \in D$, and
    $\state{x}{\frozentrue}$ for all $x \in X$, where $X = D -
    \setof{\top}$.

    We consider both cases:

    \begin{itemize}
    \item $\state{d}{\frozenfalse}$ for all $d \in D$:

      By the definition of $\leqp$, $\state{\bot}{\frozenfalse} \leqp
      \state{d}{\frozenfalse}$ iff $\bot \userleq d$.

      Since $\bot$ is the least element of $D$, $\bot \userleq d$.

      Therefore $\botp = \state{\bot}{\frozenfalse} \leqp
      \state{d}{\frozenfalse}$.

    \item $\state{x}{\frozentrue}$ for all $x \in X$:

      By the definition of $\leqp$, $\state{\bot}{\frozenfalse} \leqp
      \state{x}{\frozentrue}$ iff $\bot \userleq x$.

      Since $\bot$ is the least element of $D$, $\bot \userleq x$.

      Therefore $\botp = \state{\bot}{\frozenfalse} \leqp
      \state{x}{\frozentrue}$.

    \end{itemize}

    Therefore $\botp$ is less than or equal to all elements of $D_p$.

  \item $\topp$ is the greatest element of $D_p$.

    $\topp$ is defined to be $\state{\top}{\frozenfalse}$.

    In order to be the greatest element of $D_p$, every element of
    $D_p$ must be less than or equal to it.

    By Lemma~\ref{lem:partition-of-Dp}, the elements of $D_p$
    partition into $\state{d}{\frozenfalse}$ for all $d \in D$, and
    $\state{x}{\frozentrue}$ for all $x \in X$, where $X = D -
    \setof{\top}$.

    We consider both cases:

    \begin{itemize}
    \item $\state{d}{\frozenfalse}$ for all $d \in D$:

      By the definition of $\leqp$, $\state{d}{\frozenfalse} \leqp
      \state{\top}{\frozenfalse}$ iff $d \userleq \top$.

      Since $\top$ is the greatest element of $D$, $d \userleq \top$.

      Therefore $\state{d}{\frozenfalse} \leqp
      \state{\top}{\frozenfalse} = \topp$.

    \item $\state{x}{\frozentrue}$ for all $x \in X$:

      By the definition of $\leqp$, $\state{x}{\frozentrue} \leqp
      \state{\top}{\frozenfalse}$ iff $\top \userleq \top$.

      Therefore $\state{x}{\frozentrue} \leqp
      \state{\top}{\frozenfalse} = \topp$.

    \end{itemize}

    Therefore all elements of $D_p$ are less than or equal to $\topp$.
  \end{enumerate}
\end{proof}


\section{Proof of Lemma~\ref{lem:monotonicity}}\label{section:monotonicity-proof}
\begin{proof}
  \TODO{Fix the typos I found in this.}

  \begin{itemize}

    \item Case {\sc E-Eval-Ctxt}:

      Given: $\config{S}{\E{e}} \parstepsto \config{S'}{\E{e'}}$.

      To show: $\leqstore{S}{S'}$.

      From the premise of {\sc E-Eval-Ctxt}, $\config{S}{e}
      \parstepsto \config{S'}{e'}$.

      Hence by IH, $\leqstore{S}{S'}$, as we were required to show.

    \item Case {\sc E-Beta}:

      Immediate by the definition of $\leqstore{}{}$, since $S$ does
      not change.

    \item Case {\sc E-New}:

      Given: $\config{S}{\NEW} \parstepsto
      \config{\extS{S}{l}{\bot}{\frozenfalse}}{l}$.

      To show: $\leqstore{S}{\extS{S}{l}{\bot}{\frozenfalse}}$.

      By Definition~\ref{def:leqstore}, we have to show that $\dom{S}
      \subseteq \dom{\extS{S}{l}{\bot}{\frozenfalse}}$ and that for
      all $l' \in \dom{S}, \\
      S(l') \leqp (\extS{S}{l}{\bot}{\frozenfalse})(l')$.

      By the definition of store update,
      $\extS{S}{l}{d_1}{\frozentrue}$ can only either update an
      existing binding in $S$ or extend $S$ with a new binding.

      Hence $\dom{S} \subseteq \dom{\extS{S}{l}{\bot}{\frozenfalse}}$.

      From the side condition of {\sc E-New}, $l \notin \dom{S}$.

      Hence $\extS{S}{l}{\bot}{\frozenfalse}$ adds a new binding for
      $l$ in $S$.

      Hence $\extS{S}{l}{d_1}{\frozentrue}$ does not update any
      existing bindings in $S$.

      Hence, for all $l' \in \dom{S}, S(l') \leqp
      (\extS{S}{l}{d_1}{\frozentrue})(l')$.

      Therefore $\leqstore{S}{\extS{S}{l}{\bot}{\frozenfalse}}$, as
      required.

    \item Case {\sc E-Put}:

      Given: $\config{S}{\putexp{l}{d_2}} \parstepsto
      \config{\extSRaw{S}{l}{p_2}}{\unit}$.

      To show: $\leqstore{S}{\extSRaw{S}{l}{p_2}}$.

      By Definition~\ref{def:leqstore}, we have to show that $\dom{S}
      \subseteq \dom{\extSRaw{S}{l}{p_2}}$ and that for all $l' \in
      \dom{S}, \\
      S(l') \leqp (\extSRaw{S}{l}{p_2})(l')$.

      By the definition of store update, $\extSRaw{S}{l}{p_2}$ can only
      either update an existing binding in $S$ or extend $S$ with a
      new binding.

      Hence $\dom{S} \subseteq \dom{\extSRaw{S}{l}{p_2}}$.

      From the premises of {\sc E-Put}, $S(l) = p_1$.  Therefore $l
      \in \dom{S}$.

      Hence $\extSRaw{S}{l}{p_2}$ updates the existing binding for $l$
      in $S$ from $p_1$ to $p_2$.

      From the premises of {\sc E-Put}, $p_2 =
      \lubp{p_1}{\state{d_2}{\frozenfalse}}$.

      Hence, by the definition of $\lubp{}{}$, $p_1 \leqp p_2$.

      $\extSRaw{S}{l}{p_2}$ does not update any other bindings in $S$,
      hence, for all $l' \in \dom{S}, S(l') \leqp
      (\extSRaw{S}{l}{p_2})(l')$.

      Hence $\leqstore{S}{\extSRaw{S}{l}{p_2}}$, as required.

    \item Case {\sc E-Put-Err}:

      Given: $\config{S}{\putexp{l}{d_2}} \parstepsto \error$.

      By the definition of $\error$, $\error = \config{\topS}{e}$ for
      any $e$.

      To show: $\leqstore{S}{\topS}$.

      Immediate by the definition of $\leqstore{}{}$.

    \item Case {\sc E-Get}:

      Immediate by the definition of $\leqstore{}{}$, since $S$ does
      not change.

    \item Case {\sc E-Freeze-Init}:

      Immediate by the definition of $\leqstore{}{}$, since $S$ does
      not change.

    \item Case {\sc E-Spawn-Handler}:

      Immediate by the definition of $\leqstore{}{}$, since $S$ does
      not change.

    \item Case {\sc E-Freeze-Final}:

      Given: $\config{S}{\freezeafterfull{l}{Q}{v}{\setof{v\dots}}{H}}
      \parstepsto \config{\extS{S}{l}{d_1}{\frozentrue}}{d_1}$.

      To show: $\leqstore{S}{\extS{S}{l}{d_1}{\frozentrue}}$.

      By Definition~\ref{def:leqstore}, we have to show that $\dom{S}
      \subseteq \dom{\extS{S}{l}{d_1}{\frozentrue}}$ and that for all
      $l' \in \dom{S}, \\
      S(l') \leqp (\extS{S}{l}{d_1}{\frozentrue})(l')$.

      \lk{We could spell this out in even more excruciating detail,
        but I think it's obvious enough.}

      By the definition of store update,
      $\extS{S}{l}{d_1}{\frozentrue}$ can only either update an
      existing binding in $S$ or extend $S$ with a new binding.

      Hence $\dom{S} \subseteq \dom{\extS{S}{l}{d_1}{\frozentrue}}$.

      From the premises of {\sc E-Freeze-Final}, $S(l) =
      \state{d_1}{\status_1}$.  Therefore $l \in \dom{S}$.

      Hence $\extS{S}{l}{d_1}{\frozentrue}$ updates the existing
      binding for $l$ in $S$ from $\state{d_1}{\status_1}$ to
      $\state{d_1}{\frozentrue}$.

      By the definition of $\leqp$, $\state{d_1}{\status_1} \leqp
      \state{d_1}{\frozentrue}$.

      $\extS{S}{l}{d_1}{\frozentrue}$ does not update any other
      bindings in $S$, hence, for all $l' \in \dom{S}, \\
      S(l') \leqp (\extS{S}{l}{d_1}{\frozentrue})(l')$.

      Hence $\leqstore{S}{\extS{S}{l}{d_1}{\frozentrue}}$, as
      required.

    \item Case {\sc E-Freeze-Simple}:

      Given: $\config{S}{\freeze{l}} \parstepsto
      \config{\extS{S}{l}{d_1}{\frozentrue}}{d_1}$.

      To show: $\leqstore{S}{\extS{S}{l}{d_1}{\frozentrue}}$.

      Similar to the previous case.

  \end{itemize}

\end{proof}


\section{Proof of Lemma~\ref{lem:independence}}\label{section:independence-proof}
\begin{proof}
  Consider arbitrary $S''$ such that $S''$ is non-conflicting with
  $\config{S}{e} \parstepsto \config{S'}{e'}$ and $\lubstore{S'}{S''}
  \statuseq S$ and $\lubstore{S'}{S''} \neq \topS$.

  To show: $\config{\lubstore{S}{S''}}{e} \parstepsto
  \config{\lubstore{S'}{S''}}{e'}$.

  The proof is by cases on the rule of the reduction semantics by
  which $\config{S}{e}$ steps to $\config{S'}{e'}$.  Since
  $\config{S'}{e'} \neq \error$, we do not need to consider the {\sc
    E-Put-Err} rule.

  The assumption that $\lubstore{S'}{S''} \statuseq S$ is only needed
  in the {\sc E-Freeze-Final} and {\sc E-Freeze-Simple} cases.

  \begin{itemize}

    \item Case {\sc E-Beta}:

      Given: $\config{S}{\app{(\lam{x}{e})}{v}} \parstepsto
      \config{S}{\subst{e}{x}{v}}$.

      To show: $\config{\lubstore{S}{S''}}{\app{(\lam{x}{e})}{v}} \parstepsto
      \config{\lubstore{S}{S''}}{\subst{e}{x}{v}}$.

      Immediate by {\sc E-Beta}.

    \item Case {\sc E-New}:

      Given: $\config{S}{\NEW} \parstepsto
      \config{\extS{S}{l}{\bot}{\frozenfalse}}{l}$.

      To show: $\config{\lubstore{S}{S''}}{\NEW} \parstepsto
      \config{\lubstore{(\extS{S}{l}{\bot}{\frozenfalse})}{S''}}{l}$.

      By {\sc E-New}, we have that $\config{\lubstore{S}{S''}}{\NEW}
      \parstepsto
      \config{\extS{(\lubstore{S}{S''})}{l'}{\bot}{\frozenfalse}}{l'}$,
      where $l' \notin \dom{\lubstore{S}{S''}}$.

      By assumption, $S''$ is non-conflicting with $\config{S}{\NEW}
      \parstepsto \config{\extS{S}{l}{\bot}{\frozenfalse}}{l}$.
 
      Therefore $l \notin \dom{S''}$.

      From the side condition of {\sc E-New}, $l \notin \dom{S}$.

      Therefore $l \notin \dom{\lubstore{S}{S''}}$.

      Therefore, in
      $\config{\extS{(\lubstore{S}{S''})}{l'}{\bot}{\frozenfalse}}{l'}$,
      we can $\alpha$-rename $l'$ to $l$, resulting in
      $\config{\extS{(\lubstore{S}{S''})}{l}{\bot}{\frozenfalse}}{l}$.

      Therefore $\config{\lubstore{S}{S''}}{\NEW} \parstepsto
      \config{\extS{(\lubstore{S}{S''})}{l}{\bot}{\frozenfalse}}{l}$.

      Note that:
      \begin{align*}
        \extS{(\lubstore{S}{S''})}{l}{\bot}{\frozenfalse} &=
        \lubstore{\extS{S}{l}{\bot}{\frozenfalse}}{\extS{S''}{l}{\bot}{\frozenfalse}} \\
        &= \lubstore{\lubstore{S}{\store{\storebinding{l}{\bot}{\frozenfalse}}}}{\lubstore{S''}{\store{\storebinding{l}{\bot}{\frozenfalse}}}} \\
        &= \lubstore{\lubstore{S}{\store{\storebinding{l}{\bot}{\frozenfalse}}}}{S''} \\
        &= \lubstore{\extS{S}{l}{\bot}{\frozenfalse}}{S''}.
      \end{align*}
      Therefore $\config{\lubstore{S}{S''}}{\NEW} \parstepsto
      \config{\lubstore{\extS{S}{l}{\bot}{\frozenfalse}}{S''}}{l}$, as we were
      required to show.

    \item Case {\sc E-Put}:

      Given: $\config{S}{\putiexp{l}} \parstepsto
      \config{\extSRaw{S}{l}{u_{p_i}(p_1)}}{\unit}$.

      To show: $\config{\lubstore{S}{S''}}{\putiexp{l}{d_2}}
      \parstepsto
      \config{\lubstore{\extSRaw{S}{l}{u_{p_i}(p_1)}}{S''}}{\unit}$.

      We will first show that

      $\config{\lubstore{S}{S''}}{\putiexp{l}{d_2}} \parstepsto
      \config{\extSRaw{(\lubstore{S}{S''})}{l}{u_{p_i}(p_1)}}{\unit}$

      and then show why this is sufficient.

      We proceed by cases on $l$:

      \begin{itemize}
        \item $l \notin \dom{S''}$:

          By assumption, $\lubstore{\extSRaw{S}{l}{u_{p_i}(p_1)}}{S''}
          \neq \topS$.

          By Lemma~\ref{lem:monotonicity},
          $\leqstore{S}{\extSRaw{S}{l}{u_{p_i}(p_1)}}$.

          Hence $\lubstore{S}{S''} \neq \topS$.

          Therefore, by Definition~\ref{def:lubstore},
          $(\lubstore{S}{S''})(l) = S(l)$.

          From the premises of {\sc E-Put}, $S(l) = p_1$.

          Hence $(\lubstore{S}{S''})(l) = p_1$.

          From the premises of {\sc E-Put}, $u_{p_i}(p_1) \neq \topp$.

          Therefore, by {\sc E-Put}, we have:
          $\config{\lubstore{S}{S''}}{\putiexp{l}} \parstepsto
          \config{\extSRaw{(\lubstore{S}{S''})}{l}{u_{p_i}(p_1)}}{\unit}$.

        \item $l \in \dom{S''}$:

          By assumption, $\lubstore{\extSRaw{S}{l}{u_{p_i}(p_1)}}{S''} \neq
          \topS$.

          By Lemma~\ref{lem:monotonicity},
          $\leqstore{S}{\extSRaw{S}{l}{u_{p_i}(p_1)}}$.

          Hence $\lubstore{S}{S''} \neq \topS$.

          Therefore $(\lubstore{S}{S''})(l) = \lubp{S(l)}{S''(l)}$.

          From the premises of {\sc E-Put}, $S(l) = p_1$.
          
          Hence $(\lubstore{S}{S''})(l) = p'_1$, where $p_1 \leqp
          p'_1$.

          \TODO{From here forward, this subcase still needs to be
            fixed.}

          By assumption, $\lubstore{\extSRaw{S}{l}{p_2}}{S''} \neq
          \topS$.

          Therefore, by Definition~\ref{def:lubstore},
          $\lubp{p_2}{S''(l)} \neq \topp$.

          Note that:
          \begin{align*}
            \topp &\neq \lubp{p_2}{S''(l)} \\
            &= \lubp{\lubp{p_1}{\state{d_2}{\frozenfalse}}}{S''(l)} \\
            &= \lubp{\lubp{S(l)}{\state{d_2}{\frozenfalse}}}{S''(l)} \\
            &= \lubp{\lubp{S(l)}{S''(l)}}{\state{d_2}{\frozenfalse}} \\
            &= \lubp{(\lubstore{S}{S''})(l)}{\state{d_2}{\frozenfalse}} \\
            &= \lubp{p'_1}{\state{d_2}{\frozenfalse}} \\
            &= p'_2. \\
          \end{align*}
          Hence $p'_2 \neq \topp$.

          Hence $(\lubstore{S}{S''})(l) = p'_1$ and $p'_2 =
          \lubp{p'_1}{\state{d_2}{\frozenfalse}}$ and $p'_2 \neq
          \topp$.

          Therefore, by {\sc E-Put} we have:
          $\config{\lubstore{S}{S''}}{\putiexp{l}{d_2}} \parstepsto
          \config{\extSRaw{(\lubstore{S}{S''})}{l}{p'_2}}{\unit}$.

          \lk{If we really wanted to be pedantic here, we'd actually
            prove that the stores are equal.  I'm assuming that if I
            can show that $\extSRaw{(\lubstore{S}{S''})}{l}{p'_2}$ and
            $\extSRaw{(\lubstore{S}{S''})}{l}{p_2}$ bind $l$ to the
            same value, then it will be obvious that they're equal.}

          Note that:
          \begin{align*}
            (\extSRaw{(\lubstore{S}{S''})}{l}{p'_2})(l) &= \lubp{(\lubstore{S}{S''})(l)}{(\store{\storebindingRaw{l}{p'_2}})(l)} \\
            &= \lubp{p'_1}{p'_2} \\
            &= \lubp{p'_1}{\lubp{p'_1}{\state{d_2}{\frozenfalse}}} \\
            &= \lubp{p'_1}{\state{d_2}{\frozenfalse}}
          \end{align*}
          and
          \begin{align*}
            (\extSRaw{(\lubstore{S}{S''})}{l}{p_2})(l) &= \lubp{(\lubstore{S}{S''})(l)}{(\store{\storebindingRaw{l}{p_2}})(l)} \\
            &= \lubp{p'_1}{p_2} \\
            &= \lubp{p'_1}{\lubp{p_1}{\state{d_2}{\frozenfalse}}} \\
            &= \lubp{p'_1}{\state{d_2}{\frozenfalse}} & \textrm{(since $p_1 \leqp p'_1$).}
          \end{align*}
          Therefore $\extSRaw{(\lubstore{S}{S''})}{l}{p'_2} =
          \extSRaw{(\lubstore{S}{S''})}{l}{p_2}$.

          Therefore, $\config{\lubstore{S}{S''}}{\putiexp{l}{d_2}}
          \parstepsto
          \config{\extSRaw{(\lubstore{S}{S''})}{l}{p_2}}{\unit}$.
      \end{itemize}

      Note that:
      \begin{align*}
        \extSRaw{(\lubstore{S}{S''})}{l}{p_2} &= \lubstore{\extSRaw{S}{l}{p_2}}{\extSRaw{S''}{l}{p_2}} \\
        &= \lubstore{\lubstore{S}{\store{\storebindingRaw{l}{p_2}}}}{\lubstore{S''}{\store{\storebindingRaw{l}{p_2}}}} \\
        &= \lubstore{\lubstore{S}{\store{\storebindingRaw{l}{p_2}}}}{S''} \\
        &= \lubstore{\extSRaw{S}{l}{p_2}}{S''}.
      \end{align*}
      Therefore $\config{\lubstore{S}{S''}}{\putiexp{l}{d_2}}
      \parstepsto \config{\lubstore{\extSRaw{S}{l}{p_2}}{S''}}{\unit}$,
      as we were required to show.

    \item Case {\sc E-Get}:

      Given: $\config{S}{\getexp{l}{P}} \parstepsto \config{S}{p_2}$.

      To show: $\config{\lubstore{S}{S''}}{\getexp{l}{P}} \parstepsto
      \config{\lubstore{S}{S''}}{p_2}$.

      From the premises of {\sc E-Get}, $S(l) = p_1$ and $\incomp{P}$
      and $p_2 \in P$ and $p_2 \leqp p_1$.

      By assumption, $\lubstore{S}{S''} \neq \topS$.

      Hence $(\lubstore{S}{S''}) = p'_1$, where $p_1 \leqp p'_1$.

      By the transitivity of $\leqp$, $p_2 \leqp p'_1$.

      Hence, $S(l) = p'_1$ and $\incomp{P}$ and $p_2 \in P$ and $p_2
      \leqp p'_1$.

      Therefore, by {\sc E-Get},

      $\config{\lubstore{S}{S''}}{\getexp{l}{P}} \parstepsto
      \config{\lubstore{S}{S''}}{p_2}$,

      as we were required to show.

    \item Case {\sc E-Freeze-Init}:

      Given: $\config{S}{\freezeafter{l}{Q}{\lam{x}{e}}} \parstepsto
      \config{S}{\freezeafterfull{l}{Q}{\lam{x}{e}}{\setof{}}{\setof{}}}$.

      To show:
      $\config{\lubstore{S}{S''}}{\freezeafter{l}{Q}{\lam{x}{e}}}
      \parstepsto
      \config{\lubstore{S}{S''}}{\freezeafterfull{l}{Q}{\lam{x}{e}}{\setof{}}{\setof{}}}$.

      Immediate by {\sc E-Freeze-Init}.

    \item Case {\sc E-Spawn-Handler}:

      Given:

      $\config{S}{\freezeafterfull{l}{Q}{\lam{x}{e_0}}{\setof{e,
            \dots}}{H}} \parstepsto
      \config{S}{\freezeafterfull{l}{Q}{\lam{x}{e_0}}{\setof{\subst{e_0}{x}{d_2},
            e, \dots}} {\{d_2\}\cup H}}$.

      To show:

      $\config{\lubstore{S}{S''}}{\freezeafterfull{l}{Q}{\lam{x}{e_0}}{\setof{e,
            \dots}}{H}} \parstepsto
      \config{\lubstore{S}{S''}}{\freezeafterfull{l}{Q}{\lam{x}{e_0}}{\setof{\subst{e_0}{x}{d_2},
            e, \dots}} {\{d_2\}\cup H}}$.

      From the premises of {\sc E-Spawn-Handler}, $S(l) =
      \state{d_1}{\status_1}$ and $d_2 \userleq d_1$ and $d_2 \notin
      H$ and $d_2 \in Q$.

      By assumption, $\lubstore{S}{S''} \neq \topS$.

      Hence $(\lubstore{S}{S''})(l) = \state{d'_1}{\status'_1}$ where
      $\state{d_1}{\status_1} \leqp \state{d'_1}{\status'_1}$.

      By Definition~\ref{def:lattice-with-status-bits}, $d_1 \userleq
      d'_1$.

      By the transitivity of $\userleq$, $d_2 \userleq d'_1$.

      Hence $(\lubstore{S}{S''})(l) =
      \state{d'_1}{\status'_1}$ and $d_2 \userleq d'_1$ and $d_2 \notin
      H$ and $d_2 \in Q$.

      Therefore, by {\sc E-Spawn-Handler},

      $\config{\lubstore{S}{S''}}{\freezeafterfull{l}{Q}{\lam{x}{e_0}}{\setof{e,
            \dots}}{H}} \parstepsto
      \config{\lubstore{S}{S''}}{\freezeafterfull{l}{Q}{\lam{x}{e_0}}{\setof{\subst{e_0}{x}{d_2},
            e, \dots}} {\{d_2\}\cup H}}$,

      as we were required to show.

    \item Case {\sc E-Freeze-Final}:

      \lk{This case wouldn't work but for the $\lubstore{S'}{S''}
        \statuseq S$ requirement, which makes it a no-op freeze.}

      Given:
      $\config{S}{\freezeafterfull{l}{Q}{\lam{x}{e_0}}{\setof{v,
            \dots}}{H}} \parstepsto
      \config{\extS{S}{l}{d_1}{\frozentrue}}{d_1}$.

      To show:
      $\config{\lubstore{S}{S''}}{\freezeafterfull{l}{Q}{\lam{x}{e_0}}{\setof{v,
            \dots}}{H}} \parstepsto
      \config{\lubstore{\extS{S}{l}{d_1}{\frozentrue}}{S''}}{d_1}$.

      We will first show that

      $\config{\lubstore{S}{S''}}{\freezeafterfull{l}{Q}{\lam{x}{e_0}}{\setof{v,
            \dots}}{H}} \parstepsto
      \config{\extS{(\lubstore{S}{S''})}{l}{d_1}{\frozentrue}}{d_1}$

      and then show why this is sufficient.

      We proceed by cases on $l$:
      \begin{itemize}
      \item $l \notin \dom{S''}$:

        By assumption, $\lubstore{\extS{S}{l}{d_1}{\frozentrue}}{S''}
        \neq \topS$.

        By Lemma~\ref{lem:monotonicity},
        $\leqstore{S}{\extS{S}{l}{d_1}{\frozentrue}}$.

        Therefore $\lubstore{S}{S''} \neq \topS$.

        Hence, by Definition~\ref{def:lubstore},
        $(\lubstore{S}{S''})(l) = S(l)$.

        From the premises of {\sc E-Freeze-Final}, we have that $S(l)
        = \state{d_1}{\status_1}$.

        Hence $(\lubstore{S}{S''})(l) = \state{d_1}{\status_1}$.

        From the premises of {\sc E-Freeze-Final}, we have that
        $\forall{d_2} ~.~ ( {d_2 \userleq d_1 \land d_2 \in Q} \Rightarrow d_2 \in
        H)$.

        Therefore, by {\sc E-Freeze-Final}, we have that

        $\config{\lubstore{S}{S''}}{\freezeafterfull{l}{Q}{\lam{x}{e_0}}{\setof{v,
              \dots}}{H}} \parstepsto
        \config{\extS{(\lubstore{S}{S''})}{l}{d_1}{\frozentrue}}{d_1}$.


      \item $l \in \dom{S''}$:

        By assumption, $\lubstore{\extS{S}{l}{d_1}{\frozentrue}}{S''}
        \neq \topS$.

        By Lemma~\ref{lem:monotonicity},
        $\leqstore{S}{\extS{S}{l}{d_1}{\frozentrue}}$.

        Therefore $\lubstore{S}{S''} \neq \topS$.

        Hence, by Definition~\ref{def:lubstore},
        $(\lubstore{S}{S''})(l) = \lubp{S(l)}{S''(l)}$.

        From the premises of {\sc E-Freeze-Final}, we have that
        $S(l) = \state{d_1}{\status_1}$.

        By assumption, $\lubstore{\extS{S}{l}{d_1}{\frozentrue}}{S''}
        \statuseq S$.

        Therefore $\status_1 = \frozentrue$.

        Therefore $S(l) = \state{d_1}{\frozentrue}$.

        Therefore $(\lubstore{S}{S''})(l) =
        \lubp{\state{d_1}{\frozentrue}}{S''(l)}$.

        We proceed by cases on $S''(l)$:
        \begin{itemize}
        \item $S''(l) = \state{d_3}{\frozenfalse}$, where $d_3 \userleq d_1$:

          By Definition~\ref{def:lubp},
          $\lubp{\state{d_1}{\frozentrue}}{\state{d_3}{\frozenfalse}}
          = \state{d_1}{\frozentrue}$.

          Therefore $(\lubstore{S}{S''})(l) =
          \state{d_1}{\frozentrue}$.

          From the premises of {\sc E-Freeze-Final}, we have that
          $\forall{d_2} ~.~ ( {d_2 \userleq d_1 \land d_2 \in Q} \Rightarrow d_2 \in
          H)$.

          Therefore, by {\sc E-Freeze-Final}, we have that

          $\config{\lubstore{S}{S''}}{\freezeafterfull{l}{Q}{\lam{x}{e_0}}{\setof{v,
                \dots}}{H}} \parstepsto
          \config{\extS{(\lubstore{S}{S''})}{l}{d_1}{\frozentrue}}{d_1}$.

        \item $S''(l) = \state{d_3}{\frozenfalse}$, where $d_3 \nuserleq d_1$:

          By Definition~\ref{def:lubp},
          $\lubp{\state{d_1}{\frozentrue}}{\state{d_3}{\frozenfalse}}
          = \state{\top}{\frozenfalse}$.

          Therefore $\lubp{S(l)}{S''(l)} =
          \state{\top}{\frozenfalse}$.

          By Definition~\ref{def:lattice-with-status-bits},
          $\state{\top}{\frozenfalse} = \topp$.

          Therefore $\lubp{S(l)}{S''(l)} = \topp$.

          Therefore, by Definition~\ref{def:lubstore},
          $\lubstore{S}{S''} = \topS$.

          This is a contradiction.

          Therefore,

          $\config{\lubstore{S}{S''}}{\freezeafterfull{l}{Q}{\lam{x}{e_0}}{\setof{v,
                \dots}}{H}} \parstepsto
          \config{\extS{(\lubstore{S}{S''})}{l}{d_1}{\frozentrue}}{d_1}$.

        \item $S''(l) = \state{d_3}{\frozentrue}$, where $d_3 = d_1$:

          By Definition~\ref{def:lubp},
          $\lubp{\state{d_1}{\frozentrue}}{\state{d_3}{\frozentrue}} =
          \state{d_1}{\frozentrue}$.

          Therefore $(\lubstore{S}{S''})(l) = \state{d_1}{\frozentrue}$.

          From the premises of {\sc E-Freeze-Final}, we have that
          $\forall{d_2} ~.~ ( {d_2 \userleq d_1 \land d_2 \in Q} \Rightarrow d_2 \in
          H)$.

          Therefore, by {\sc E-Freeze-Final}, we have that

          $\config{\lubstore{S}{S''}}{\freezeafterfull{l}{Q}{\lam{x}{e_0}}{\setof{v,
                \dots}}{H}} \parstepsto
          \config{\extS{(\lubstore{S}{S''})}{l}{d_1}{\frozentrue}}{d_1}$.

        \item $S''(l) = \state{d_3}{\frozentrue}$, where $d_3 \neq d_1$:

          By Definition~\ref{def:lubp},
          $\lubp{\state{d_1}{\frozentrue}}{\state{d_3}{\frozentrue}}
          = \state{\top}{\frozenfalse}$.

          Therefore $\lubp{S(l)}{S''(l)} = \state{\top}{\frozenfalse}$.

          By Definition~\ref{def:lattice-with-status-bits},
          $\state{\top}{\frozenfalse} = \topp$.

          Therefore $\lubp{S(l)}{S''(l)} = \topp$.

          Therefore, by Definition~\ref{def:lubstore},
          $\lubstore{S}{S''} = \topS$.

          This is a contradiction.

          Therefore,

          $\config{\lubstore{S}{S''}}{\freezeafterfull{l}{Q}{\lam{x}{e_0}}{\setof{v,
                \dots}}{H}} \parstepsto
          \config{\extS{(\lubstore{S}{S''})}{l}{d_1}{\frozentrue}}{d_1}$.
        \end{itemize}
      \end{itemize}

      In each case we have shown that

      $\config{\lubstore{S}{S''}}{\freezeafterfull{l}{Q}{\lam{x}{e_0}}{\setof{v,
            \dots}}{H}} \parstepsto
      \config{\extS{(\lubstore{S}{S''})}{l}{d_1}{\frozentrue}}{d_1}$.

      Note that:
      \begin{align*}
        \extS{(\lubstore{S}{S''})}{l}{d_1}{\frozentrue} &=
        \lubstore{\extS{S}{l}{d_1}{\frozentrue}}{\extS{S''}{l}{d_1}{\frozentrue}} \\
        &= \lubstore{\lubstore{S}{\store{\storebinding{l}{d_1}{\frozentrue}}}}{\lubstore{S''}{\store{\storebinding{l}{d_1}{\frozentrue}}}} \\
        &= \lubstore{\lubstore{S}{\store{\storebinding{l}{d_1}{\frozentrue}}}}{S''} \\
        &= \lubstore{\extS{S}{l}{d_1}{\frozentrue}}{S''}.
      \end{align*}
      Therefore

      $\config{\lubstore{S}{S''}}{\freezeafterfull{l}{Q}{\lam{x}{e_0}}{\setof{v,
            \dots}}{H}} \parstepsto
      \config{\lubstore{\extS{S}{l}{d_1}{\frozentrue}}{S''}}{d_1}$,

      as we were required to show.

    \item Case {\sc E-Freeze-Simple}:

      Given: $\config{S}{\freeze{l}} \parstepsto
      \config{\extS{S}{l}{d_1}{\frozentrue}}{d_1}$.

      To show: $\config{\lubstore{S}{S''}}{\freeze{l}}
      \parstepsto
      \config{\lubstore{\extS{S}{l}{d_1}{\frozentrue}}{S''}}{d_1}$.

      We will first show that

      $\config{\lubstore{S}{S''}}{\freeze{l}} \parstepsto
      \config{\extS{(\lubstore{S}{S''})}{l}{d_1}{\frozentrue}}{d_1}$

      and then show why this is sufficient.

      We proceed by cases on $l$:
      \begin{itemize}
      \item $l \notin \dom{S''}$:

        By assumption, $\lubstore{\extS{S}{l}{d_1}{\frozentrue}}{S''}
        \neq \topS$.

        By Lemma~\ref{lem:monotonicity},
        $\leqstore{S}{\extS{S}{l}{d_1}{\frozentrue}}$.

        Therefore $\lubstore{S}{S''} \neq \topS$.

        Hence, by Definition~\ref{def:lubstore},
        $(\lubstore{S}{S''})(l) = S(l)$.

        From the premise of {\sc E-Freeze-Simple}, we have that
        $S(l) = \state{d_1}{\status_1}$.

        Therefore, by {\sc E-Freeze-Simple}, we have that

        $\config{\lubstore{S}{S''}}{\freeze{l}}
        \parstepsto
        \config{\extS{(\lubstore{S}{S''})}{l}{d_1}{\frozentrue}}{d_1}$.

      \item $l \in \dom{S''}$:

        By assumption, $\lubstore{\extS{S}{l}{d_1}{\frozentrue}}{S''}
        \neq \topS$.

        By Lemma~\ref{lem:monotonicity},
        $\leqstore{S}{\extS{S}{l}{d_1}{\frozentrue}}$.

        Therefore $\lubstore{S}{S''} \neq \topS$.

        Hence, by Definition~\ref{def:lubstore},
        $(\lubstore{S}{S''})(l) = \lubp{S(l)}{S''(l)}$.

        From the premise of {\sc E-Freeze-Simple}, we have that
        $S(l) = \state{d_1}{\status_1}$.

        By assumption, $\lubstore{\extS{S}{l}{d_1}{\frozentrue}}{S''}
        \statuseq S$.

        Therefore $\status_1 = \frozentrue$.

        Therefore $(\lubstore{S}{S''})(l) =
        \lubp{\state{d_1}{\frozentrue}}{S''(l)}$.
        
        We proceed by cases on $S''(l)$:
        \begin{itemize}
        \item $S''(l) = \state{d_2}{\frozenfalse}$, where $d_2 \userleq d_1$:

          By Definition~\ref{def:lubp},
          $\lubp{\state{d_1}{\frozentrue}}{\state{d_2}{\frozenfalse}} =
          \state{d_1}{\frozentrue}$.

          Therefore $(\lubstore{S}{S''})(l) =
          \state{d_1}{\frozentrue}$.

          Therefore, by {\sc E-Freeze-Simple}, we have that

          $\config{\lubstore{S}{S''}}{\freeze{l}}
          \parstepsto
          \config{\extS{(\lubstore{S}{S''})}{l}{d_1}{\frozentrue}}{d_1}$.

        \item $S''(l) = \state{d_2}{\frozenfalse}$, where $d_2 \nuserleq d_1$:

          By Definition~\ref{def:lubp},
          $\lubp{\state{d_1}{\frozentrue}}{\state{d_2}{\frozenfalse}}
          = \state{\top}{\frozenfalse}$.

          By Definition~\ref{def:lattice-with-status-bits},
          $\state{\top}{\frozenfalse} = \topp$.

          Therefore $\lubp{S(l)}{S''(l)} = \topp$.

          Therefore, by Definition~\ref{def:lubstore},
          $\lubstore{S}{S''} = \topS$.

          This is a contradiction.

          Therefore,

          $\config{\lubstore{S}{S''}}{\freeze{l}}
          \parstepsto
          \config{\extS{(\lubstore{S}{S''})}{l}{d_1}{\frozentrue}}{d_1}$.

        \item $S''(l) = \state{d_2}{\frozentrue}$, where $d_2 = d_1$:

          Therefore $(\lubstore{S}{S''})(l) =
          \lubp{\state{d_1}{\frozentrue}}{\state{d_2}{\frozentrue}}$.

          By Definition~\ref{def:lubp},
          $\lubp{\state{d_1}{\frozentrue}}{\state{d_2}{\frozentrue}} =
          \state{d_1}{\frozentrue}$.

          Therefore $(\lubstore{S}{S''})(l) =
          \state{d_1}{\frozentrue}$.

          Therefore, by {\sc E-Freeze-Simple}, we have that

          $\config{\lubstore{S}{S''}}{\freeze{l}}
          \parstepsto
          \config{\extS{(\lubstore{S}{S''})}{l}{d_1}{\frozentrue}}{d_1}$.

        \item $S''(l) = \state{d_2}{\frozentrue}$, where $d_2 \neq d_1$:

          By Definition~\ref{def:lubp},
          $\lubp{\state{d_1}{\frozentrue}}{\state{d_2}{\frozentrue}}
          = \state{\top}{\frozenfalse}$.

          By Definition~\ref{def:lattice-with-status-bits},
          $\state{\top}{\frozenfalse} = \topp$.

          Therefore $\lubp{S(l)}{S''(l)} = \topp$.

          Therefore, by Definition~\ref{def:lubstore},
          $\lubstore{S}{S''} = \topS$.

          This is a contradiction.

          Therefore,

          $\config{\lubstore{S}{S''}}{\freeze{l}}
          \parstepsto
          \config{\extS{(\lubstore{S}{S''})}{l}{d_1}{\frozentrue}}{d_1}$.
        \end{itemize}
      \end{itemize}

      In each case we have shown that

      $\config{\lubstore{S}{S''}}{\freeze{l}} \parstepsto
      \config{\extS{(\lubstore{S}{S''})}{l}{d_1}{\frozentrue}}{d_1}$.

      Note that:
      \begin{align*}
        \extS{(\lubstore{S}{S''})}{l}{d_1}{\frozentrue} &=
        \lubstore{\extS{S}{l}{d_1}{\frozentrue}}{\extS{S''}{l}{d_1}{\frozentrue}} \\
        &= \lubstore{\lubstore{S}{\store{\storebinding{l}{d_1}{\frozentrue}}}}{\lubstore{S''}{\store{\storebinding{l}{d_1}{\frozentrue}}}} \\
        &= \lubstore{\lubstore{S}{\store{\storebinding{l}{d_1}{\frozentrue}}}}{S''} \\
        &= \lubstore{\extS{S}{l}{d_1}{\frozentrue}}{S''}.
      \end{align*}
      Therefore
      $\config{\lubstore{S}{S''}}{\freeze{l}}
      \parstepsto
      \config{\lubstore{\extS{S}{l}{d_1}{\frozentrue}}{S''}}{d_1}$,
      as we were required to show.
  \end{itemize}
\end{proof}


\section{Proof of Lemma~\ref{lem:strong-local-quasi-confluence}}\label{section:strong-local-quasi-confluence-proof}
\begin{proof}
  Suppose $\conf \ctxstepsto \conf_a$ and $\conf \ctxstepsto \conf_b$.

  We have to show that either there exist $\conf_c, i, j, \pi$ such
  that $\conf_a \ctxstepsto^i \conf_c$ and $\pi(\conf_b) \ctxstepsto^j
  \conf_c$ and $i \leq 1$ and $j \leq 1$, or that $\conf_a \ctxstepsto
  \error$ or $\conf_b \ctxstepsto \error$.

  By inspection of the operational semantics, it must be the case that
  $\conf$ steps to $\conf_a$ by the {\sc E-Eval-Ctxt} rule.

  Let $\conf = \config{S}{\evalctxt{E_a}{e_{a_1}}}$ and let $\conf_a =
  \config{S_a}{\evalctxt{E_a}{e_{a_2}}}$.

  Likewise, it must be the case that $\conf$ steps to $\conf_b$ by the
  {\sc E-Eval-Ctxt} rule.

  Let $\conf = \config{S}{\evalctxt{E_b}{e_{b_1}}}$ and let $\conf_b =
  \config{S_b}{\evalctxt{E_b}{e_{b_2}}}$.

  Note that $\conf = \config{S}{\evalctxt{E_a}{e_{a_1}}} =
  \config{S}{\evalctxt{E_b}{e_{b_1}}}$, and so
  $\evalctxt{E_a}{e_{a_1}} = \evalctxt{E_b}{e_{b_1}}$, but $E_a$ and
  $E_b$ may differ and $e_{a_1}$ and $e_{b_1}$ may differ.

  First, consider the possibility that $E_a = E_b$ (and $e_{a_1} =
  e_{b_1}$).

  Since $\config{S}{\evalctxt{E_a}{e_{a_1}}} \ctxstepsto
  \config{S_a}{\evalctxt{E_a}{e_{a_2}}}$ by {\sc E-Eval-Ctxt} and
  $\config{S}{\evalctxt{E_b}{e_{b_1}}} \ctxstepsto
  \config{S_b}{\evalctxt{E_b}{e_{b_2}}}$ by {\sc E-Eval-Ctxt}, we have
  from the premise of {\sc E-Eval-Ctxt} that $\config{S}{e_{a_1}}
  \parstepsto \config{S_a}{e_{a_2}}$ and $\config{S}{e_{b_1}}
  \parstepsto \config{S_b}{e_{b_2}}$.

  But then, since $e_{a_1} = e_{b_1}$, by Internal Determinism
  (Lemma~\ref{lem:internal-determinism}) there is a permutation $\pi'$
  such that $\config{S_a}{e_{a_2}} = \pi'(\config{S_b}{e_{b_2}})$,
  modulo choice of events.

  We have two cases:

  \begin{itemize}
  \item In the case where the steps $\conf \ctxstepsto \conf_a$ and
    $\conf \ctxstepsto \conf_b$ are both by {\sc E-Spawn-Handler} and
    they handle different events $d_2$ and $d'_2$, then we can satisfy
    the proof by choosing the final configuration $\conf_c$ as the
    configuration where both $d_2$ and $d'_2$ have been handled.

    Both $\conf_a$ and $\conf_b$ can step to this configuration by
    {\sc E-Spawn-Handler}: if the step from $\conf$ to $\conf_a$
    handles $d_2$ then the step from $\conf_a$ to $\conf_c$ handles
    $d'_2$, while if the step from $\conf$ to $\conf_b$ handles $d'_2$
    then the step from $\conf_b$ to $\conf_c$ handles $d_2$.

    The store in the final configuration is $S_a$ or $S_b$, which are
    equal because {\sc E-Spawn-Handler} does not affect the store, and
    we can satisfy the proof by choosing $i = 1$ and $j = 0$ and $\pi
    = \id$.

  \item Otherwise, we can satisfy the proof by choosing $\conf_c =
    \config{S_a}{e_{a_2}}$ and $i = 0$ and $j = 0$ and $\pi = \id$.
  \end{itemize}

  The rest of this proof deals with the more interesting case in which
  $E_a \neq E_b$ (and $e_{a_1} \neq e_{b_1}$).

  Since $\config{S}{\evalctxt{E_a}{e_{a_1}}} \ctxstepsto
  \config{S_a}{\evalctxt{E_a}{e_{a_2}}}$ and
  $\config{S}{\evalctxt{E_b}{e_{b_1}}} \ctxstepsto
  \config{S_b}{\evalctxt{E_b}{e_{b_2}}}$ and $\evalctxt{E_a}{e_{a_1}}
  = \evalctxt{E_b}{e_{b_1}}$, and since $E_a \neq E_b$, we have from
  Lemma~\ref{lem:locality} (Locality) that there exist evaluation
  contexts $E'_a$ and $E'_b$ such that:

  \begin{itemize}
  \item $\evalctxt{E'_a}{e_{a_1}} = \evalctxt{E_b}{e_{b_2}}$, and
  \item $\evalctxt{E'_b}{e_{b_1}} = \evalctxt{E_a}{e_{a_2}}$, and
  \item $\evalctxt{E'_a}{e_{a_2}} =
    \evalctxt{E'_b}{e_{b_2}}$.
  \end{itemize}

  In some of the cases that follow, we will choose $\conf_c = \error$,
  and in some we will prove that one of $\conf_a$ or $\conf_b$ steps
  to $\error$.

  In most cases, however, our approach will be to show that there
  exist $S', i, j, \pi$ such that:
  \begin{itemize}
  \item $\config{S_a}{\evalctxt{E_a}{e_{a_2}}} \ctxstepsto^i
    \config{S'}{\evalctxt{E'_a}{e_{a_2}}}$, and
  \item $\pi(\config{S_b}{\evalctxt{E_b}{e_{b_2}}}) \ctxstepsto^j
    \config{S'}{\evalctxt{E'_a}{e_{a_2}}}$.
  \end{itemize}
  Since $\evalctxt{E'_a}{e_{a_1}} = \evalctxt{E_b}{e_{b_2}}$,
  $\evalctxt{E'_b}{e_{b_1}} = \evalctxt{E_a}{e_{a_2}}$, and
  $\evalctxt{E'_a}{e_{a_2}} = \evalctxt{E'_b}{e_{b_2}}$, it suffices
  to show that:
  \begin{itemize}
  \item $\config{S_a}{\evalctxt{E'_b}{e_{b_1}}} \ctxstepsto^i
    \config{S'}{\evalctxt{E'_b}{e_{b_2}}}$, and
  \item $\pi(\config{S_b}{\evalctxt{E'_a}{e_{a_1}}}) \ctxstepsto^j
    \config{S'}{\evalctxt{E'_a}{e_{a_2}}}$.
  \end{itemize}
  From the premise of {\sc E-Eval-Ctxt}, we have that
  $\config{S}{e_{a_1}} \parstepsto \config{S_a}{e_{a_2}}$ and
  $\config{S}{e_{b_1}} \parstepsto \config{S_b}{e_{b_2}}$.

  We proceed by case analysis on the rule by which
  $\config{S}{e_{a_1}}$ steps to $\config{S_a}{e_{a_2}}$.

  Since the only way an $\error$ configuration can arise is by the
  {\sc E-Put-Err} rule, we can assume in all other cases that $\conf_a
  \neq \error$.

  \begin{enumerate}
  \item Case {\sc E-Beta}: We have $S_a = S$.

    We proceed by case analysis on the rule by which
    $\config{S}{e_{b_1}}$ steps to $\config{S_b}{e_{b_2}}$.

    Since the only way an $\error$ configuration can arise is by the
    {\sc E-Put-Err} rule, we can assume in all other cases that
    $\conf_b \neq \error$.
    \begin{enumerate}
    \item \label{slqc-beta-beta}Case {\sc E-Beta}: We have $S_a = S$
      and $S_b = S$.

      Choose $S' = S = S_a = S_b$, $i = 1$, $j = 1$, and $\pi = \id$.

      We have to show that:
      \begin{itemize}
      \item $\config{S}{\evalctxt{E'_b}{e_{b_1}}} \ctxstepsto
        \config{S_a}{\evalctxt{E'_b}{e_{b_2}}}$, and
      \item $\config{S}{\evalctxt{E'_a}{e_{a_1}}} \ctxstepsto
        \config{S_b}{\evalctxt{E'_a}{e_{a_2}}}$, 
      \end{itemize}

      both of which follow immediately from $\config{S}{e_{a_1}}
      \parstepsto \config{S_a}{e_{a_2}}$ and $\config{S}{e_{b_1}}
      \parstepsto \config{S_b}{e_{b_2}}$ and {\sc E-Eval-Ctxt}.

    \item \label{slqc-beta-new}Case {\sc E-New}: We have $S_a = S$ and
      $S_b = \extS{S}{l}{\bot}{\frozenfalse}$.

      Choose $S' = S_b$, $i = 1$, $j = 1$, and $\pi = \id$.

      We have to show that:
      \begin{itemize}
      \item $\config{S}{\evalctxt{E'_b}{e_{b_1}}} \ctxstepsto
        \config{S_b}{\evalctxt{E'_b}{e_{b_2}}}$, and
      \item $\config{S_b}{\evalctxt{E'_a}{e_{a_1}}} \ctxstepsto
        \config{S_b}{\evalctxt{E'_a}{e_{a_2}}}$.
      \end{itemize}

      The first of these follows immediately from $\config{S}{e_{b_1}}
      \parstepsto \config{S_b}{e_{b_2}}$ and {\sc E-Eval-Ctxt}.

      For the second, consider that $S_b =
      \extS{S}{l}{\bot}{\frozenfalse} = U_S(S)$, where $U_S$ is the
      store update operation that acts as the identity on the contents
      of all existing locations, and adds the binding
      $\storebinding{l}{\bot}{\frozenfalse}$ if no binding for $l$
      exists.

      Note that:
      \begin{itemize}
      \item $U_S$ is non-conflicting with $\config{S}{e_{a_1}}
        \parstepsto \config{S_a}{e_{a_2}}$, since no locations are
        allocated in the transition;
      \item $U_S(S_a) \neq \topS$, since $U_S(S_a) = U_S(S) = S_b$
        and we know that $\conf_b \neq \error$; and
      \item $U_S$ is freeze-safe with $\config{S}{e_{a_1}}
        \parstepsto \config{S_a}{e_{a_2}}$, since $S_a = S$, so
        there are no locations whose contents differ in status
        between them.
      \end{itemize}

      Therefore, by Lemma~\ref{lem:generalized-independence}
      (Generalized Independence), we have that

      $\config{U_S(S)}{e_{a_1}} \parstepsto
      \config{U_S(S_a)}{e_{a_2}}$.

      Hence $\config{S_b}{e_{a_1}} \parstepsto \config{S_b}{e_{a_2}}$.

      By {\sc E-Eval-Ctxt}, it follows that
      $\config{S_b}{\evalctxt{E'_a}{e_{a_1}}} \ctxstepsto
      \config{S_b}{\evalctxt{E'_a}{e_{a_2}}}$,
      as we were required to show.

    \item \label{slqc-beta-put}Case {\sc E-Put}: We have $S_a = S$ and
      $S_b = \extSRaw{S}{l}{u_{p_i}(p_1)}$.

      Choose $S' = S_b$, $i = 1$, $j = 1$, and $\pi = \id$.

      We have to show that:
      \begin{itemize}
      \item $\config{S}{\evalctxt{E'_b}{e_{b_1}}} \ctxstepsto
        \config{S_b}{\evalctxt{E'_b}{e_{b_2}}}$, and
      \item $\config{S_b}{\evalctxt{E'_a}{e_{a_1}}} \ctxstepsto
        \config{S_b}{\evalctxt{E'_a}{e_{a_2}}}$.
      \end{itemize}

      The first of these follows immediately from $\config{S}{e_{b_1}}
      \parstepsto \config{S_b}{e_{b_2}}$ and {\sc E-Eval-Ctxt}.

      For the second, consider that $S_b = U_S(S)$, where $U_S$ is the
      store update operation that applies $u_{p_i}$ to the contents of
      $l$ and acts as the identity on all other locations.

      Note that:
      \begin{itemize}
      \item $U_S$ is non-conflicting with $\config{S}{e_{a_1}}
        \parstepsto \config{S_a}{e_{a_2}}$, since no locations are
        allocated in the transition;
      \item $U_S(S_a) \neq \topS$, since $U_S(S_a) = U_S(S) = S_b$
        and we know that $\conf_b \neq \error$; and
      \item $U_S$ is freeze-safe with $\config{S}{e_{a_1}}
        \parstepsto \config{S_a}{e_{a_2}}$, since $S_a = S$, so
        there are no locations whose contents differ in status
        between them.
      \end{itemize}

      Therefore, by Lemma~\ref{lem:generalized-independence}
      (Generalized Independence), we have that

      $\config{U_S(S)}{e_{a_1}} \parstepsto
      \config{U_S(S_a)}{e_{a_2}}$.

      Hence $\config{S_b}{e_{a_1}} \parstepsto \config{S_b}{e_{a_2}}$.

      By {\sc E-Eval-Ctxt}, it follows that
      $\config{S_b}{\evalctxt{E'_a}{e_{a_1}}} \ctxstepsto
      \config{S_b}{\evalctxt{E'_a}{e_{a_2}}}$, as we were required to
      show.

    \item \label{slqc-beta-put-err}Case {\sc E-Put-Err}: We have $S_a
      = S$ and $\config{S_b}{e_{b_2}} = \error$, and so we choose
      $\conf_c = \error$, $i = 1$, $j = 0$, and $\pi = \id$.

      We have to show that:
      \begin{itemize}
      \item $\config{S}{\evalctxt{E'_b}{e_{b_1}}} \ctxstepsto \error$,
        and
      \item $\config{S_b}{\evalctxt{E'_a}{e_{a_1}}} = \error$.
      \end{itemize}

      The second of these is immediately true because since
      $\config{S_b}{e_{b_2}} = \error$, $S_b = \topS$, and so
      $\config{S_b}{\evalctxt{E'_a}{e_{a_1}}}$ is equal to $\error$ as
      well.

      For the first, observe that $\config{S}{e_{b_1}} \parstepsto
      \config{S_b}{e_{b_2}}$, hence by {\sc E-Eval-Ctxt},
      $\config{S}{\evalctxt{E'_b}{e_{b_1}}} \ctxstepsto
      \config{S_b}{\evalctxt{E'_b}{e_{b_2}}}$.

      But $S_b = \topS$, so $\config{S_b}{\evalctxt{E'_b}{e_{b_2}}}$
      is equal to $\error$, and so
      $\config{S}{\evalctxt{E'_b}{e_{b_1}}} \ctxstepsto \error$, as
      required.

    \item \label{slqc-beta-get}Case {\sc E-Get}: Similar to
      case~\ref{slqc-beta-beta}, since $S_a = S$ and $S_b = S$.
    \item \label{slqc-beta-freeze-init}Case {\sc E-Freeze-Init}:
      Similar to case~\ref{slqc-beta-beta}, since $S_a = S$ and $S_b =
      S$.
    \item \label{slqc-beta-spawn-handler}Case {\sc E-Spawn-Handler}:
      Similar to case~\ref{slqc-beta-beta}, since $S_a = S$ and $S_b =
      S$.
    \item \label{slqc-beta-freeze-final}Case {\sc E-Freeze-Final}: We
      have $S_a = S$ and $S_b = \extS{S}{l}{d_1}{\frozentrue}$.

      Choose $S' = S_b$, $i = 1$, $j = 1$, and $\pi = \id$.

      We have to show that:
      \begin{itemize}
      \item $\config{S}{\evalctxt{E'_b}{e_{b_1}}} \ctxstepsto
        \config{S_b}{\evalctxt{E'_b}{e_{b_2}}}$, and
      \item $\config{S_b}{\evalctxt{E'_a}{e_{a_1}}} \ctxstepsto
        \config{S_b}{\evalctxt{E'_a}{e_{a_2}}}$.
      \end{itemize}

      The first of these follows immediately from $\config{S}{e_{b_1}}
      \parstepsto \config{S_b}{e_{b_2}}$ and {\sc E-Eval-Ctxt}.

      For the second, note that $S_b = U_S(S)$, where $U_S$ is the
      store update operation that freezes the contents of $l$ and acts
      as the identity on the contents of all other locations.

      Note that:
      \begin{itemize}
      \item $U_S$ is non-conflicting with $\config{S}{e_{a_1}}
        \parstepsto \config{S_a}{e_{a_2}}$, since no locations are
        allocated in the transition;
      \item $U_S(S_a) \neq \topS$, since $U_S(S_a) = U_S(S) = S_b$
        and we know that $\conf_b \neq \error$; and
      \item $U_S$ is freeze-safe with $\config{S}{e_{a_1}}
        \parstepsto \config{S_a}{e_{a_2}}$, since $S_a = S$, so
        there are no locations whose contents differ in status
        between them.
      \end{itemize}

      Therefore, by Lemma~\ref{lem:generalized-independence}
      (Generalized Independence), we have that

      $\config{U_S(S)}{e_{a_1}} \parstepsto
      \config{U_S(S_a)}{e_{a_2}}$.

      Hence $\config{S_b}{e_{a_1}} \parstepsto \config{S_b}{e_{a_2}}$.

      By {\sc E-Eval-Ctxt}, it follows that
      $\config{S_b}{\evalctxt{E'_a}{e_{a_1}}} \ctxstepsto
      \config{S_b}{\evalctxt{E'_a}{e_{a_2}}}$, as we were required to
      show.

    \item \label{slqc-beta-freeze-simple}Case {\sc E-Freeze-Simple}:
      Similar to case~\ref{slqc-beta-freeze-final}, since $S_b =
      \extS{S}{l}{d_1}{\frozentrue}$.

    \end{enumerate}
  \item Case {\sc E-New}: We have $S_a = \extS{S}{l}{\bot}{\frozenfalse}$.

    We proceed by case analysis on the rule by which
    $\config{S}{e_{b_1}}$ steps to $\config{S_b}{e_{b_2}}$.

    Since the only way an $\error$ configuration can arise is by the
    {\sc E-Put-Err} rule, we can assume in all other cases that
    $\conf_b \neq \error$.
    \begin{enumerate}
    \item \label{slqc-new-beta}Case {\sc E-Beta}: By symmetry with case~\ref{slqc-beta-new}.
    \item \label{slqc-new-new}Case {\sc E-New}: We have $S_a =
      \extS{S}{l}{\bot}{\frozenfalse}$ and $S_b =
      \extS{S}{l'}{\bot}{\frozenfalse}$.

      Now consider whether $l = l'$:
      \begin{itemize}
      \item If $l \neq l'$:

        Choose $S' =
        \extS{\extS{S}{l'}{\bot}{\frozenfalse}}{l}{\bot}{\frozenfalse}$,
        $i = 1$, $j = 1$, and $\pi = \id$.

        We have to show that:
        \begin{itemize}
        \item $\config{S_a}{\evalctxt{E'_b}{e_{b_1}}} \ctxstepsto
          \config{\extS{\extS{S}{l'}{\bot}{\frozenfalse}}{l}{\bot}{\frozenfalse}}{\evalctxt{E'_b}{e_{b_2}}}$,
          and
        \item $\config{S_b}{\evalctxt{E'_a}{e_{a_1}}} \ctxstepsto
          \config{\extS{\extS{S}{l'}{\bot}{\frozenfalse}}{l}{\bot}{\frozenfalse}}{\evalctxt{E'_a}{e_{a_2}}}$.
        \end{itemize}

        For the first of these, consider that $S_a =
        \extS{S}{l}{\bot}{\frozenfalse} = U_S(S)$, where $U_S$ is
        the store update operation that acts as the identity on the
        contents of all existing locations, and adds the binding
        $\storebinding{l}{\bot}{\frozenfalse}$ if no binding for $l$
        exists.

        Note that:
        \begin{itemize}
        \item $U_S$ is non-conflicting with $\config{S}{e_{b_1}}
          \parstepsto \config{S_b}{e_{b_2}}$, since the only
          location allocated in the transition is $l'$, and $l
          \neq l'$ in this case;
        \item $U_S(S_b) \neq \topS$, since $U_S(S_b) =
          \extS{\extS{S}{l'}{\bot}{\frozenfalse}}{l}{\bot}{\frozenfalse}$
          and we know $S \neq \topS$ and the addition of new
          bindings $\storebinding{l}{\bot}{\frozenfalse}$ and
          $\storebinding{l'}{\bot}{\frozenfalse}$ cannot cause it to
          become $\topS$; and
        \item $U_S$ is freeze-safe with $\config{S}{e_{b_1}}
          \parstepsto \config{S_b}{e_{b_2}}$, since $S_b =
          \extS{S}{l'}{\bot}{\frozenfalse}$ and $l' \notin \dom{S}$,
          so there are no locations whose contents differ in status
          between $S$ and $S_b$.
        \end{itemize}

        Therefore, by Lemma~\ref{lem:generalized-independence}
        (Generalized Independence), we have that

        $\config{U_S(S)}{e_{b_1}} \parstepsto
        \config{U_S(S_b)}{e_{b_2}}$.

        Hence $\config{\extS{S}{l}{\bot}{\frozenfalse}}{e_{b_1}}
        \parstepsto
        \config{\extS{S_b}{l}{\bot}{\frozenfalse}}{e_{b_2}}$.

        By {\sc E-Eval-Ctxt} it follows that

        $\config{\extS{S}{l}{\bot}{\frozenfalse}}{\evalctxt{E'_b}{e_{b_1}}}
        \parstepsto
        \config{\extS{S_b}{l}{\bot}{\frozenfalse}}{\evalctxt{E'_b}{e_{b_2}}}$,
        which, since $S_b = \extS{S}{l'}{\bot}{\frozenfalse}$, is what
        we were required to show.

        The argument for the second is symmetrical.

      \item If $l = l'$:

        In this case, observe that we do \emph{not} want the
        expression in the final configuration to be
        $\evalctxt{E'_a}{e_{a_2}}$ (nor its equivalent,
        $\evalctxt{E'_b}{e_{b_2}}$).

        The reason for this is that $\evalctxt{E'_a}{e_{a_2}}$
        contains both occurrences of $l$.

        Rather, we want both configurations to step to a configuration
        in which exactly one occurrence of $l$ has been renamed to a
        fresh location $l''$.

        Let $l''$ be a location such that $l'' \notin \dom{S}$ and
        $l'' \neq l$ (and hence $l'' \neq l'$, as well).

        Then choose $S' =
        \extS{\extS{S}{l''}{\bot}{\frozenfalse}}{l}{\bot}{\frozenfalse}$,
        $i = 1$, $j = 1$, and $\pi = \setof{(l, l'')}$.

        Either
        $\config{\extS{\extS{S}{l''}{\bot}{\frozenfalse}}{l}{\bot}{\frozenfalse}}{\evalctxt{E'_a}{\pi(e_{a_2})}}$
        or
        $\config{\extS{\extS{S}{l''}{\bot}{\frozenfalse}}{l}{\bot}{\frozenfalse}}{\evalctxt{E'_b}{\pi(e_{b_2})}}$
        would work as a final configuration; we choose

        $\config{\extS{\extS{S}{l''}{\bot}{\frozenfalse}}{l}{\bot}{\frozenfalse}}{\evalctxt{E'_b}{\pi(e_{b_2})}}$.

        We have to show that:
        \begin{itemize}
        \item $\config{S_a}{\evalctxt{E'_b}{e_{b_1}}} \ctxstepsto
          \config{\extS{\extS{S}{l''}{\bot}{\frozenfalse}}{l}{\bot}{\frozenfalse}}{\evalctxt{E'_b}{\pi(e_{b_2})}}$,
          and
        \item $\pi(\config{S_b}{\evalctxt{E'_a}{e_{a_1}}})
          \ctxstepsto
          \config{\extS{\extS{S}{l''}{\bot}{\frozenfalse}}{l}{\bot}{\frozenfalse}}{\evalctxt{E'_b}{\pi(e_{b_2})}}$.
        \end{itemize}

        For the first of these, since $\config{S}{e_{b_1}}
        \parstepsto \config{S_b}{e_{b_2}}$, we have by
        Lemma~\ref{lem:permutability} (Permutability) that
        $\pi(\config{S}{e_{b_1}}) \parstepsto
        \pi(\config{S_b}{e_{b_2}})$.

        Since $\pi = \setof{(l, l'')}$, but $l \notin S$ (from the
        side condition on {\sc E-New}), we have that
        $\pi(\config{S}{e_{b_1}}) = \config{S}{e_{b_1}}$.

        Since $\config{S_b}{e_{b_2}} =
        \config{\extS{S}{l'}{\bot}{\frozenfalse}}{l'}$, and $l = l'$,
        we have that $\pi(\config{S_b}{e_{b_2}}) =
        \config{\extS{S}{l''}{\bot}{\frozenfalse}}{\pi(e_{b_2})}$.

        Hence $\config{S}{e_{b_1}} \parstepsto
        \config{\extS{S}{l''}{\bot}{\frozenfalse}}{\pi(e_{b_2})}$.

        Let $U_S$ be the store update operation that acts as the
        identity on the contents of all existing locations, and adds
        the binding $\storebinding{l}{\bot}{\frozenfalse}$ if no
        binding for $l$ exists.

        Note that:
        \begin{itemize}
        \item $U_S$ is non-conflicting with $\config{S}{e_{b_1}}
          \parstepsto
          \config{\extS{S}{l''}{\bot}{\frozenfalse}}{\pi(e_{b_2})}$,
          since the only location allocated in the transition is
          $l''$;
        \item $U_S(\extS{S}{l''}{\bot}{\frozenfalse}) \neq \topS$,
          since $U_S(\extS{S}{l''}{\bot}{\frozenfalse}) = \\
          \extS{\extS{S}{l''}{\bot}{\frozenfalse}}{l}{\bot}{\frozenfalse}$
          and we know $S \neq \topS$ and the addition of new
          bindings $\storebinding{l}{\bot}{\frozenfalse}$ and
          $\storebinding{l''}{\bot}{\frozenfalse}$ cannot cause it
          to become $\topS$; and
        \item $U_S$ is freeze-safe with $\config{S}{e_{b_1}}
          \parstepsto
          \config{\extS{S}{l''}{\bot}{\frozenfalse}}{\pi(e_{b_2})}$,
          since $l'' \notin \dom{S}$, so there are no locations
          whose contents differ in status between $S$ and
          $\extS{S}{l''}{\bot}{\frozenfalse}$.
        \end{itemize}

        Therefore, by Lemma~\ref{lem:generalized-independence}
        (Generalized Independence), we have that

        $\config{U_S(S)}{e_{b_1}} \parstepsto
        \config{U_S(\extS{S}{l''}{\bot}{\frozenfalse})}{\pi(e_{b_2})}$.

        Hence $\config{\extS{S}{l}{\bot}{\frozenfalse}}{e_{b_1}}
        \parstepsto
        \config{\extS{\extS{S}{l''}{\bot}{\frozenfalse}}{l}{\bot}{\frozenfalse}}{\pi(e_{b_2})}$.

        By {\sc E-Eval-Ctxt} it follows that

        $\config{\extS{S}{l}{\bot}{\frozenfalse}}{\evalctxt{E'_b}{e_{b_1}}}
        \parstepsto
        \config{\extS{\extS{S}{l''}{\bot}{\frozenfalse}}{l}{\bot}{\frozenfalse}}{\evalctxt{E'_b}{\pi(e_{b_2})}}$,

        which, since $\extS{S}{l}{\bot}{\frozenfalse} = S_a$, is what
        we were required to show.

        For the second, observe that since $S_b =
        \extS{S}{l}{\bot}{\frozenfalse}$, we have that $\pi(S_b) =
        \extS{S}{l''}{\bot}{\frozenfalse}$.

        Also, since $l$ does not occur in $e_{a_1}$, we have that
        $\pi(\evalctxt{E'_a}{e_{a_1}}) =
        \evalctxt{(\pi(E'_a))}{e_{a_1}}$.

        Hence we have to show that

        $\config{\extS{S}{l''}{\bot}{\frozenfalse}}{\evalctxt{(\pi(E'_a))}{e_{a_1}}}
        \ctxstepsto \\
        \config{\extS{\extS{S}{l''}{\bot}{\frozenfalse}}{l}{\bot}{\frozenfalse}}{\evalctxt{E'_b}{\pi(e_{b_2})}}$.

        Let $U_S$ be the store update operation that acts as the
        identity on the contents of all existing locations, and adds
        the binding $\storebinding{l''}{\bot}{\frozenfalse}$ if no
        binding for $l''$ exists.

        Note that:
        \begin{itemize}
        \item $U_S$ is non-conflicting with $\config{S}{e_{a_1}}
          \parstepsto \config{S_a}{e_{a_2}}$, since the only
          location allocated in the transition is $l$;
        \item $U_S(S_a) \neq \topS$, since $U_S(S_a) =
          \extS{\extS{S}{l''}{\bot}{\frozenfalse}}{l}{\bot}{\frozenfalse}$
          and we know $S \neq \topS$ and the addition of new
          bindings $\storebinding{l}{\bot}{\frozenfalse}$ and
          $\storebinding{l''}{\bot}{\frozenfalse}$ cannot cause it
          to become $\topS$; and
        \item $U_S$ is freeze-safe with $\config{S}{e_{a_1}}
          \parstepsto \config{S_a}{e_{a_2}}$, since $S_a =
          \extS{S}{l}{\bot}{\frozenfalse}$ and $l \notin \dom{S}$,
          so there are no locations whose contents differ in status
          between $S$ and $S_a$.
        \end{itemize}

        Therefore, by Lemma~\ref{lem:generalized-independence}
        (Generalized Independence), we have that

        $\config{U_S(S)}{e_{a_1}} \parstepsto
        \config{U_S(S_a)}{e_{a_2}}$.

        Hence $\config{\extS{S}{l''}{\bot}{\frozenfalse}}{e_{a_1}}
        \parstepsto
        \config{\extS{\extS{S}{l''}{\bot}{\frozenfalse}}{l}{\bot}{\frozenfalse}}{e_{a_2}}$.

        By {\sc E-Eval-Ctxt} it follows that
        
        $\config{\extS{S}{l''}{\bot}{\frozenfalse}}{\evalctxt{(\pi(E'_a))}{e_{a_1}}}
        \ctxstepsto \\
        \config{\extS{\extS{S}{l''}{\bot}{\frozenfalse}}{l}{\bot}{\frozenfalse}}{\evalctxt{(\pi(E'_a))}{e_{a_2}}}$,

        which completes the case since $\evalctxt{E'_b}{\pi(e_{b_2})}
        = \evalctxt{(\pi(E'_a))}{e_{a_2}}$.

        \lk{This assumes that you believe that
          $\evalctxt{E'_b}{\pi(e_{b_2})} =
          \evalctxt{(\pi(E'_a))}{e_{a_2}}$.}

      \end{itemize}

    \item \label{slqc-new-put}Case {\sc E-Put}: We have $S_a =
      \extS{S}{l}{\bot}{\frozenfalse}$ and $S_b =
      \extSRaw{S}{l'}{u_{p_i}(p_1)}$, where $l \neq l'$ (since $l
      \notin \dom{S}$, but $l' \in \dom{S}$).

      We have to show that:
      \begin{itemize}
      \item $\config{S_a}{\evalctxt{E'_b}{e_{b_1}}} \ctxstepsto
        \config{\extS{S_b}{l}{\bot}{\frozenfalse}}{\evalctxt{E'_b}{e_{b_2}}}$,
        and
      \item $\config{S_b}{\evalctxt{E'_a}{e_{a_1}}} \ctxstepsto
        \config{\extS{S_b}{l}{\bot}{\frozenfalse}}{\evalctxt{E'_a}{e_{a_2}}}$.
      \end{itemize}

      For the first of these, consider that $S_a =
      \extS{S}{l}{\bot}{\frozenfalse} = U_S(S)$, where $U_S$ is the
      store update operation that acts as the identity on the contents
      of all existing locations, and adds the binding
      $\storebinding{l}{\bot}{\frozenfalse}$ if no binding for $l$
      exists.

      Note that:
      \begin{itemize}
      \item $U_S$ is non-conflicting with $\config{S}{e_{b_1}}
        \parstepsto \config{S_b}{e_{b_2}}$, since no locations are
        allocated in the transition;
      \item $U_S(S_b) \neq \topS$, since $U_S(S_b) =
        \extS{S_b}{l}{\bot}{\frozenfalse}$, and we know $S_b \neq
        \topS$ and the addition of a new binding
        $\storebinding{l}{\bot}{\frozenfalse}$ cannot cause it to
        become $\topS$; and
      \item $U_S$ is freeze-safe with $\config{S}{e_{b_1}} \parstepsto
        \config{S_b}{e_{b_2}}$, since $S_b =
        \extSRaw{S}{l'}{u_{p_i}(p_1)}$ and $u_{p_i}$ does not alter
        the status of $p_1$.

        (By Definition~\ref{def:set-of-state-update-operations},
        $u_{p_i}$ can only change the status bit of a location if its
        contents are $\state{d}{\frozentrue}$ and $u_i(d) \neq d$, in
        which case $u_{p_i}$ changes the contents of the location to
        $\state{\top}{\frozenfalse}$; however, that cannot be the case
        here since then $u_{p_i}(p_1)$ would be $\topp$, contradicting
        the premise of {\sc E-Put}.)
      \end{itemize}

      Therefore, by Lemma~\ref{lem:generalized-independence}
      (Generalized Independence), we have that

      $\config{U_S(S)}{e_{b_1}} \parstepsto
      \config{U_S(S_b)}{e_{b_2}}$.

      Hence $\config{\extS{S}{l}{\bot}{\frozenfalse}}{e_{b_1}}
      \parstepsto
      \config{\extS{S_b}{l}{\bot}{\frozenfalse}}{e_{b_2}}$.

      By {\sc E-Eval-Ctxt}, it follows that

      $\config{\extS{S}{l}{\bot}{\frozenfalse}}{\evalctxt{E'_b}{e_{b_1}}}
      \ctxstepsto
      \config{\extS{S_b}{l}{\bot}{\frozenfalse}}{\evalctxt{E'_b}{e_{b_2}}}$,
      
      which, since $S_a = \extS{S}{l}{\bot}{\frozenfalse}$, is what we
      were required to show.

      For the second, let $U_S$ be the store update operation that
      applies $u_{p_i}$ to the contents of $l'$ if it exists, and adds
      a binding $\storebindingRaw{l'}{u_{p_i}(p_1)}$ if no binding for
      $l'$ exists.

      Consider that $S_b = U_S(S)$, and
      $\extS{S_b}{l}{\bot}{\frozenfalse} =
      \extSRaw{S_a}{l'}{u_{p_i}(p_1)} = U_S(S_a)$.

      Note that:
      \begin{itemize}
      \item $U_S$ is non-conflicting with $\config{S}{e_{a_1}}
        \parstepsto \config{S_a}{e_{a_2}}$, since the only location
        allocated in the transition is $l$;
      \item $U_S(S_a) \neq \topS$, since $U_S(S_a) =
        \extSRaw{\extS{S}{l}{\bot}{\frozenfalse}}{l'}{u_{p_i}(p_1)}$
        and we know $S \neq \topS$ and the addition of a new binding
        $\storebinding{l}{\bot}{\frozenfalse}$ and updating the
        contents of location $l'$ to $u_{p_i}(p_1)$ in $S$ cannot
        cause it to become $\topS$ (since if $u_{p_i}(p_1) = \topp$,
        $\config{S}{e_{b_1}}$ would not have been able to step by {\sc
          E-Put}); and
      \item $U_S$ is freeze-safe with $\config{S}{e_{a_1}} \parstepsto
        \config{S_a}{e_{a_2}}$, since $S_a =
        \extS{S}{l}{\bot}{\frozenfalse}$ and $l \notin \dom{S}$, so
        there are no locations whose contents differ in status between
        $S$ and $S_a$.
      \end{itemize}

      Therefore, by Lemma~\ref{lem:generalized-independence}
      (Generalized Independence), we have that

      $\config{U_S(S)}{e_{a_1}} \parstepsto
      \config{U_S(S_a)}{e_{a_2}}$.

      Hence $\config{S_b}{e_{a_1}}
      \parstepsto
      \config{\extS{S_b}{l}{\bot}{\frozenfalse}}{e_{a_2}}$.

      By {\sc E-Eval-Ctxt}, it follows that
      
      $\config{S_b}{\evalctxt{E'_a}{e_{a_1}}} \ctxstepsto
      \config{\extS{S_b}{l}{\bot}{\frozenfalse}}{\evalctxt{E'_a}{e_{a_2}}}$,
      
      as we were required to show.

    \item \label{slqc-new-put-err}Case {\sc E-Put-Err}: We have $S_a =
      \extS{S}{l}{\bot}{\frozenfalse}$ and $\config{S_b}{e_{b_2}} =
      \error$, and so we choose $\conf_c = \error$, $i = 1$, $j = 0$,
      and $\pi = \id$.

      We have to show that:
      \begin{itemize}
      \item $\config{S_a}{\evalctxt{E'_b}{e_{b_1}}} \ctxstepsto
        \error$, and
      \item $\config{S_b}{\evalctxt{E'_a}{e_{a_1}}} = \error$.
      \end{itemize}

      The second of these is immediately true because since
      $\config{S_b}{e_{b_2}} = \error$, $S_b = \topS$, and so
      $\config{S_b}{\evalctxt{E'_a}{e_{a_1}}}$ is equal to $\error$ as
      well.

      For the first, observe that since $\config{S}{e_{a_1}}
      \parstepsto \config{S_a}{e_{a_2}}$, we have by
      Lemma~\ref{lem:monotonicity} (Monotonicity) that
      $\leqstore{S}{S_a}$.

      Therefore, since $\config{S}{e_{b_1}} \parstepsto \error$,

      we have by Lemma~\ref{lem:error-preservation} (Error
      Preservation) that $\config{S_a}{e_{b_1}} \parstepsto \error$.

      Since $\error$ is equal to $\config{\topS}{e}$ for all
      expressions $e$, $\config{S_a}{e_{b_1}} \parstepsto
      \config{\topS}{e}$ for all $e$.

      Therefore, by {\sc E-Eval-Ctxt},
      $\config{S_a}{\evalctxt{E'_b}{e_{b_1}}} \ctxstepsto
      \config{\topS}{\evalctxt{E'_b}{e}}$ for all $e$.

      Since $\config{\topS}{\evalctxt{E'_b}{e}}$ is equal to $\error$,
      we have that $\config{S_a}{\evalctxt{E'_b}{e_{b_1}}} \ctxstepsto
      \error$, as we were required to show.

    \item \label{slqc-new-get}Case {\sc E-Get}: Similar to
      case~\ref{slqc-new-beta}, since $S_a =
      \extS{S}{l}{\bot}{\frozenfalse}$ and $S_b = S$.
    \item \label{slqc-new-freeze-init}Case {\sc E-Freeze-Init}:
      Similar to case~\ref{slqc-new-beta}, since $S_a =
      \extS{S}{l}{\bot}{\frozenfalse}$ and $S_b = S$.
    \item \label{slqc-new-spawn-handler}Case {\sc E-Spawn-Handler}:
      Similar to case~\ref{slqc-new-beta}, since $S_a =
      \extS{S}{l}{\bot}{\frozenfalse}$ and $S_b = S$.
    \item \label{slqc-new-freeze-final}Case {\sc E-Freeze-Final}: We
      have $S_a = \extS{S}{l}{\bot}{\frozenfalse}$ and $S_b =
      \extS{S}{l'}{d_1}{\frozentrue}$, where $l \neq l'$ (since $l
      \notin \dom{S}$, but $l' \in \dom{S}$).

      Choose $S' =
      \extS{\extS{S}{l}{\bot}{\frozenfalse}}{l'}{d_1}{\frozentrue}$,
      $i = i$, $j = 1$, and $\pi = \id$.

      We have to show that:
      \begin{itemize}
      \item
        $\config{\extS{S}{l}{\bot}{\frozenfalse}}{\evalctxt{E'_b}{e_{b_1}}}
        \ctxstepsto
        \config{\extS{\extS{S}{l}{\bot}{\frozenfalse}}{l'}{d_1}{\frozentrue}}{\evalctxt{E'_b}{e_{b_2}}}$,
        and
      \item
        $\config{\extS{S}{l'}{d_1}{\frozentrue}}{\evalctxt{E'_a}{e_{a_1}}}
        \ctxstepsto
        \config{\extS{\extS{S}{l}{\bot}{\frozenfalse}}{l'}{d_1}{\frozentrue}}{\evalctxt{E'_a}{e_{a_2}}}$.
      \end{itemize}

      For the first of these, consider that
      $\extS{S}{l}{\bot}{\frozenfalse} = U_S(S)$, where $U_S$ is the
      store update operation that acts as the identity on the contents
      of all existing locations, and adds the binding
      $\storebinding{l}{\bot}{\frozenfalse}$ if no binding for $l$
      exists.

      Note that:
      \begin{itemize}
      \item $U_S$ is non-conflicting with $\config{S}{e_{b_1}}
        \parstepsto \config{S_b}{e_{b_2}}$, since no locations are
        allocated in the transition;
      \item $U_S(S_b) \neq \topS$, since $U_S(S_b) =
        \extS{S_b}{l}{\bot}{\frozenfalse}$, and we know $S_b \neq
        \topS$ and the addition of a new binding
        $\storebinding{l}{\bot}{\frozenfalse}$ cannot cause it to
        become $\topS$; and
      \item $U_S$ is freeze-safe with $\config{S}{e_{b_1}}
        \parstepsto \config{S_b}{e_{b_2}}$, since $S_b =
        \extS{S}{l'}{d_1}{\frozentrue}$ and so the only location
        that can change in status between $S$ and $S_b$ is $l'$, and
        $U_S$ acts as the identity on $l'$.
      \end{itemize}
      Therefore, by Lemma~\ref{lem:generalized-independence}
      (Generalized Independence), we have that

      $\config{U_S(S)}{e_{b_1}} \parstepsto
      \config{U_S(S_b)}{e_{b_2}}$.

      Hence $\config{\extS{S}{l}{\bot}{\frozenfalse}}{e_{b_1}}
      \parstepsto
      \config{\extS{\extS{S}{l}{\bot}{\frozenfalse}}{l'}{d_1}{\frozentrue}}{e_{b_2}}$.

      By {\sc E-Eval-Ctxt}, it follows that

      $\config{\extS{S}{l}{\bot}{\frozenfalse}}{\evalctxt{E'_b}{e_{b_1}}}
      \ctxstepsto
      \config{\extS{\extS{S}{l}{\bot}{\frozenfalse}}{l'}{d_1}{\frozentrue}}{\evalctxt{E'_b}{e_{b_2}}}$,

      as we were required to show.

      For the second, consider that $\extS{S}{l'}{d_1}{\frozentrue} =
      U_S(S)$, where $U_S$ is the store update operation that freezes
      the contents of $l'$ and acts as the identity on the contents of
      all other locations.

      Note that:
      \begin{itemize}
      \item $U_S$ is non-conflicting with $\config{S}{e_{a_1}}
        \parstepsto \config{S_a}{e_{a_2}}$, since the only location
        allocated in the transition is $l$, and $l \neq l'$;
      \item $U_S(S_a) \neq \topS$, since $U_S(S_a) =
        \extS{S_a}{l'}{d_1}{\frozentrue} =
        \extS{S_b}{l}{\bot}{\frozenfalse}$, and we know $S_b \neq
        \topS$ and the addition of a new binding
        $\storebinding{l}{\bot}{\frozenfalse}$ cannot cause it to
        become $\topS$; and
      \item $U_S$ is freeze-safe with $\config{S}{e_{a_1}}
        \parstepsto \config{S_a}{e_{a_2}}$, since $S_a =
        \extS{S}{l}{\bot}{\frozenfalse}$ and $l \notin \dom{S}$, so
        there are no locations whose contents differ in status
        between $S$ and $S_a$.
      \end{itemize}

      Therefore, by Lemma~\ref{lem:generalized-independence}
      (Generalized Independence), we have that

      $\config{U_S(S)}{e_{a_1}} \parstepsto
      \config{U_S(S_a)}{e_{a_2}}$.

      Hence $\config{\extS{S}{l'}{d_1}{\frozentrue}}{e_{a_1}}
      \parstepsto
      \config{\extS{\extS{S}{l}{\bot}{\frozenfalse}}{l'}{d_1}{\frozentrue}}{e_{a_2}}$.

      By {\sc E-Eval-Ctxt} it follows that

      $\config{\extS{S}{l'}{d_1}{\frozentrue}}{\evalctxt{E'_a}{e_{a_1}}}
      \ctxstepsto
      \config{\extS{\extS{S}{l}{\bot}{\frozenfalse}}{l'}{d_1}{\frozentrue}}{\evalctxt{E'_a}{e_{a_2}}}$,

      as we were required to show.

    \item \label{slqc-new-freeze-simple}Case {\sc E-Freeze-Simple}:
      Similar to case~\ref{slqc-new-freeze-final}, since $S_a =
      \extS{S}{l}{\bot}{\frozenfalse}$ and $S_b =
      \extS{S}{l'}{d_1}{\frozentrue}$, where $l \neq l'$ (since $l
      \notin \dom{S}$, but $l' \in \dom{S}$).

    \end{enumerate}
  \item Case {\sc E-Put}: We have $S_a =
    \extSRaw{S}{l}{u_{p_i}(p_1)}$.

    We proceed by case analysis on the rule by which
    $\config{S}{e_{b_1}}$ steps to $\config{S_b}{e_{b_2}}$.

    Since the only way an $\error$ configuration can arise is by the
    {\sc E-Put-Err} rule, we can assume in all other cases that
    $\conf_b \neq \error$.
    \begin{enumerate}
    \item \label{slqc-put-beta}Case {\sc E-Beta}: By symmetry with case~\ref{slqc-beta-put}.
    \item \label{slqc-put-new}Case {\sc E-New}: By symmetry with case~\ref{slqc-new-put}.
    \item \label{slqc-put-put}Case {\sc E-Put}: We have $S_a =
      \extSRaw{S}{l}{u_{p_i}(p_1)}$ and $S_b =
      \extSRaw{S}{l'}{u_{p_j}(p'_1)}$, where $p'_1 = S(l')$.

      Now consider whether $l = l'$:
      \begin{itemize}
      \item If $l \neq l'$:

        Choose $S' =
        \extSRaw{\extSRaw{S}{l'}{u_{p_j}(p'_1)}}{l}{u_{p_i}(p_1)}$,
        $i = 1$, $j = 1$, and $\pi = \id$.

        We have to show that:
        \begin{itemize}
        \item
          $\config{\extSRaw{S}{l}{u_{p_i}(p_1)}}{\evalctxt{E'_b}{e_{b_1}}}
          \ctxstepsto
          \config{\extSRaw{\extSRaw{S}{l'}{u_{p_j}(p'_1)}}{l}{u_{p_i}(p_1)}}{\evalctxt{E'_b}{e_{b_2}}}$,
          and
        \item
          $\config{\extSRaw{S}{l'}{u_{p_j}(p'_1)}}{\evalctxt{E'_a}{e_{a_1}}}
          \ctxstepsto
          \config{\extSRaw{\extSRaw{S}{l'}{u_{p_j}(p'_1)}}{l}{u_{p_i}(p_1)}}{\evalctxt{E'_a}{e_{a_2}}}$.
        \end{itemize}

        For the first of these, consider that
        $\extSRaw{S}{l}{u_{p_i}(p_1)} = U_S(S)$, where $U_S$ is the
        store update operation that applies $u_{p_i}$ to the
        contents of $l$ if it exists, and adds a binding
        $\storebindingRaw{l}{u_{p_i}(p_1)}$ if no binding for $l$
        exists.

        Note that:
        \begin{itemize}
        \item $U_S$ is non-conflicting with $\config{S}{e_{b_1}}
          \parstepsto
          \config{\extSRaw{S}{l'}{u_{p_j}(p'_1)}}{e_{b_2}}$, since
          no locations are allocated in the transition;
        \item $U_S(\extSRaw{S}{l'}{u_{p_j}(p'_1)}) \neq \topS$,
          since $U_S(\extSRaw{S}{l'}{u_{p_j}(p'_1)}) =
          \extSRaw{\extSRaw{S}{l'}{u_{p_j}(p'_1)}}{l}{u_{p_i}(p_1)}$
          and we know $S \neq \topS$ and updating the contents of
          location $l$ to $u_{p_i}(p_1)$ and the contents of
          location $l'$ to $u_{p_j}(p'_1)$ in $S$ cannot cause it to
          become $\topS$ (because if so, then we would have $S_a =
          \topS$ or $S_b = \topS$, which we know are not the case);
          and
        \item $U_S$ is freeze-safe with $\config{S}{e_{b_1}}
          \parstepsto
          \config{\extSRaw{S}{l'}{u_{p_j}(p'_1)}}{e_{b_2}}$, since
          $u_{p_j}$ does not alter the status of $p'_1$.

          (By Definition~\ref{def:set-of-state-update-operations},
          $u_{p_j}$ can only change the status bit of a location if
          its contents are $\state{d}{\frozentrue}$ and $u_j(d) \neq
          d$, in which case $u_{p_j}$ changes the contents of the
          location to $\state{\top}{\frozenfalse}$; however, that
          cannot be the case here since then $u_{p_j}(p'_1)$ would be
          $\topp$, contradicting the premise of {\sc E-Put}.)
        \end{itemize}

        Therefore, by Lemma~\ref{lem:generalized-independence}
        (Generalized Independence), we have that

        $\config{U_S(S)}{e_{b_1}} \parstepsto
        \config{U_S(\extSRaw{S}{l'}{u_{p_j}(p'_1)})}{e_{b_2}}$.

        Hence $\config{\extSRaw{S}{l}{u_{p_i}(p_1)}}{e_{b_1}}
        \parstepsto
        \config{\extSRaw{\extSRaw{S}{l'}{u_{p_j}(p'_1)}}{l}{u_{p_i}(p_1)}}{e_{b_2}}$.

        By {\sc E-Eval-Ctxt}, it follows that

        $\config{\extSRaw{S}{l}{u_{p_i}(p_1)}}{\evalctxt{E'_b}{e_{b_1}}}
        \ctxstepsto
        \config{\extSRaw{\extSRaw{S}{l'}{u_{p_j}(p'_1)}}{l}{u_{p_i}(p_1)}}{\evalctxt{E'_b}{e_{b_2}}}$,

        as we were required to show.

        The argument for the second is symmetrical.

      \item If $l = l'$:
        Note that since $l = l'$, $p_1 = p'_1$ as well.

        Consider whether $u_{p_i}(u_{p_j}(p_1)) = \topp$:
        \begin{itemize}
        \item If $u_{p_i}(u_{p_j}(p_1)) = \topp$:

          Choose $\conf_c = \error$, $i = 1$, $j = 1$, and $\pi =
          \id$.

          We have to show that:

          \begin{itemize}
          \item
            $\config{\extSRaw{S}{l}{u_{p_i}(p_1)}}{\evalctxt{E'_b}{e_{b_1}}}
            \ctxstepsto \error$, and
          \item
            $\config{\extSRaw{S}{l}{u_{p_j}(p_1)}}{\evalctxt{E'_a}{e_{a_1}}}
            \ctxstepsto \error$.
          \end{itemize}

          For the first of these, consider that
          $\extSRaw{S}{l}{u_{p_i}(p_1)} = U_S(S)$, where $U_S$ is the
          store update operation that applies $u_{p_i}$ to the
          contents of $l$ if it exists.

          Note that:
          \begin{itemize}
          \item $U_S$ is non-conflicting with $\config{S}{e_{b_1}}
            \parstepsto
            \config{\extSRaw{S}{l}{u_{p_j}(p_1)}}{e_{b_2}}$, since
            no locations are allocated in the transition;
          \item $U_S(\extSRaw{S}{l}{u_{p_j}(p_1)}) = \topS$, since
            $U_S(\extSRaw{S}{l}{u_{p_j}(p_1)}) =
            \extSRaw{S}{l}{u_{p_i}(u_{p_j}(p_1))}$ and we know
            $u_{p_i}(u_{p_j}(p_1)) = \topp$ in this case;
          \item $U_S$ is freeze-safe with $\config{S}{e_{b_1}}
            \parstepsto
            \config{\extSRaw{S}{l}{u_{p_j}(p_1)}}{e_{b_2}}$, since
            $u_{p_j}$ does not alter the status of $p_1$.

            (By Definition~\ref{def:set-of-state-update-operations},
            $u_{p_j}$ can only change the status bit of a location if
            its contents are $\state{d}{\frozentrue}$ and $u_j(d) \neq
            d$, in which case $u_{p_j}$ changes the contents of the
            location to $\state{\top}{\frozenfalse}$; however, that
            cannot be the case here since then $u_{p_j}(p_1)$ would be
            $\topp$, contradicting the premise of {\sc E-Put}.)
          \end{itemize}

          Therefore, by Lemma~\ref{lem:generalized-clash}
          (Generalized Clash), we have that there exists $i' \leq 1$
          such that $\config{U_S(S)}{e_{b_1}} \parstepsto^{i'}
          \error$.

          Hence $\config{\extSRaw{S}{l}{u_{p_i}(p_1)}}{e_{b_1}}
          \parstepsto^{i'} \error$.

          If $i' = 0$, we would have
          $\config{\extSRaw{S}{l}{u_{p_i}(p_1)}}{e_{b_1}} =
          \config{S_a}{e_{b_1}} = \error$.

          So we would have $S_a = \topS$ by the definition of
          $\error$, but then we would have $\conf_a = \error$, a
          contradiction.

          Therefore $i' = 1$, and so we have
          $\config{\extSRaw{S}{l}{u_{p_i}(p_1)}}{e_{b_1}} \parstepsto
          \error$.

          Since $\error = \config{\topS}{e}$ for all $e$, we have
          $\config{\extSRaw{S}{l}{u_{p_i}(p_1)}}{e_{b_1}}
          \parstepsto \config{\topS}{e}$ for all $e$.

          So, by {\sc E-Eval-Ctxt}, we have that
          $\config{\extSRaw{S}{l}{u_{p_i}(p_1)}}{\evalctxt{E'_b}{e_{b_1}}}
          \parstepsto \config{\topS}{\evalctxt{E'_b}{e}}$ for all $e$.

          Hence
          $\config{\extSRaw{S}{l}{u_{p_i}(p_1)}}{\evalctxt{E'_b}{e_{b_1}}}
          \parstepsto \error$.

          The argument for the second is symmetrical.

        \item If $u_{p_i}(u_{p_j}(p_1)) \neq \topp$:

          Choose $S' = \extSRaw{S}{l}{u_{p_i}(u_{p_j}(p_1))}$, $i =
          1$, $j = 1$, and $\pi = \id$.

          We have to show that:
          \begin{itemize}
          \item
            $\config{\extSRaw{S}{l}{u_{p_i}(p_1)}}{\evalctxt{E'_b}{e_{b_1}}}
            \ctxstepsto
            \config{\extSRaw{S}{l}{u_{p_i}(u_{p_j}(p_1))}}{\evalctxt{E'_b}{e_{b_2}}}$,
            and
          \item
            $\config{\extSRaw{S}{l}{u_{p_j}(p_1)}}{\evalctxt{E'_a}{e_{a_1}}}
            \ctxstepsto
            \config{\extSRaw{S}{l}{u_{p_i}(u_{p_j}(p_1))}}{\evalctxt{E'_a}{e_{a_2}}}$.
          \end{itemize}

          For the first of these, consider that
          $\extSRaw{S}{l}{u_{p_i}(p_1)} = U_S(S)$, where $U_S$ is the
          store update operation that applies $u_{p_i}$ to the
          contents of $l$ if it exists.

          Note that:
          \begin{itemize}
          \item $U_S$ is non-conflicting with $\config{S}{e_{b_1}}
            \parstepsto
            \config{\extSRaw{S}{l}{u_{p_j}(p_1)}}{e_{b_2}}$, since no
            locations are allocated in the transition;
          \item $U_S(\extSRaw{S}{l}{u_{p_j}(p_1)}) \neq \topS$, since
            $U_S(\extSRaw{S}{l}{u_{p_j}(p_1)}) =
            \extSRaw{S}{l}{u_{p_i}(u_{p_j}(p_1))}$ and we know $S \neq
            \topS$ and $u_{p_i}(u_{p_j}(p_1)) \neq \topp$ in this
            case;
          \item $U_S$ is freeze-safe with $\config{S}{e_{b_1}}
            \parstepsto
            \config{\extSRaw{S}{l}{u_{p_j}(p_1)}}{e_{b_2}}$, since
            $u_{p_j}$ does not alter the status of $p_1$.

            (By Definition~\ref{def:set-of-state-update-operations},
            $u_{p_j}$ can only change the status bit of a location if
            its contents are $\state{d}{\frozentrue}$ and $u_j(d) \neq
            d$, in which case $u_{p_j}$ changes the contents of the
            location to $\state{\top}{\frozenfalse}$; however, that
            cannot be the case here since then $u_{p_j}(p_1)$ would be
            $\topp$, contradicting the premise of {\sc E-Put}.)
          \end{itemize}

          Therefore, by Lemma~\ref{lem:generalized-independence}
          (Generalized Independence), we have that

          $\config{U_S(S)}{e_{b_1}} \parstepsto
          \config{U_S(\extSRaw{S}{l}{u_{p_j}(p_1)})}{e_{b_2}}$.

          Hence $\config{\extSRaw{S}{l}{u_{p_i}(p_1)}}{e_{b_1}}
          \parstepsto
          \config{\extSRaw{S}{l}{u_{p_i}(u_{p_j}(p_1))}}{e_{b_2}}$.

          By {\sc E-Eval-Ctxt}, it follows that

          $\config{\extSRaw{S}{l}{u_{p_i}(p_1)}}{\evalctxt{E'_b}{e_{b_1}}}
          \ctxstepsto
          \config{\extSRaw{S}{l}{u_{p_i}(u_{p_j}(p_1))}}{\evalctxt{E'_b}{e_{b_2}}}$,

          as we were required to show.

          The argument for the second is symmetrical.

        \end{itemize}

      \end{itemize}

    \item \label{slqc-put-put-err}Case {\sc E-Put-Err}: We have $S_a =
      \extSRaw{S}{l}{u_{p_i}(p_1)}$ and $\config{S_b}{e_{b_2}} =
      \error$, and so we choose $\conf_c = \error$, $i = 1$, $j = 0$,
      and $\pi = \id$.

      We have to show that:
      \begin{itemize}
      \item $\config{S_a}{\evalctxt{E'_b}{e_{b_1}}} \ctxstepsto
        \error$, and
      \item $\config{S_b}{\evalctxt{E'_a}{e_{a_1}}} = \error$.
      \end{itemize}

      The second of these is immediately true because since
      $\config{S_b}{e_{b_2}} = \error$, $S_b = \topS$, and so
      $\config{S_b}{\evalctxt{E'_a}{e_{a_1}}}$ is equal to $\error$ as
      well.

      For the first, observe that since $\config{S}{e_{a_1}}
      \parstepsto \config{S_a}{e_{a_2}}$, we have by
      Lemma~\ref{lem:monotonicity} (Monotonicity) that
      $\leqstore{S}{S_a}$.

      Therefore, since $\config{S}{e_{b_1}} \parstepsto \error$,

      we have by Lemma~\ref{lem:error-preservation} (Error
      Preservation) that $\config{S_a}{e_{b_1}} \parstepsto \error$.
      
      Since $\error$ is equal to $\config{\topS}{e}$ for all
      expressions $e$, $\config{S_a}{e_{b_1}} \parstepsto
      \config{\topS}{e}$ for all $e$.

      Therefore, by {\sc E-Eval-Ctxt},
      $\config{S_a}{\evalctxt{E'_b}{e_{b_1}}} \ctxstepsto
      \config{\topS}{\evalctxt{E'_b}{e}}$ for all $e$.

      Since $\config{\topS}{\evalctxt{E'_b}{e}}$ is equal to $\error$,
      we have that $\config{S_a}{\evalctxt{E'_b}{e_{b_1}}} \ctxstepsto
      \error$, as we were required to show.

    \item \label{slqc-put-get}Case {\sc E-Get}: Similar to
      case~\ref{slqc-put-beta}, since $S_a =
      \extSRaw{S}{l}{u_{p_i}(p_1)}$ and $S_b = S$.
    \item \label{slqc-put-freeze-init}Case {\sc E-Freeze-Init}:
      Similar to case~\ref{slqc-put-beta}, since $S_a =
      \extSRaw{S}{l}{u_{p_i}(p_1)}$ and $S_b = S$.
    \item \label{slqc-put-spawn-handler}Case {\sc E-Spawn-Handler}:
      Similar to case~\ref{slqc-put-beta}, since $S_a =
      \extSRaw{S}{l}{u_{p_i}(p_1)}$ and $S_b = S$.
    \item \label{slqc-put-freeze-final}Case {\sc E-Freeze-Final}: We
      have $S_a = \extSRaw{S}{l}{u_{p_i}(p_1)}$ and $S_b =
      \extS{S}{l'}{d_1}{\frozentrue}$.

      Now consider whether $l = l'$:
      \begin{itemize}
      \item If $l \neq l'$:

        Choose $S' =
        \extS{\extSRaw{S}{l}{u_{p_i}(p_1)}}{l'}{d_1}{\frozentrue}$,
        $i = 1$, $j = 1$, and $\pi = \id$.

        We have to show that:
        \begin{itemize}
        \item
          $\config{\extSRaw{S}{l}{u_{p_i}(p_1)}}{\evalctxt{E'_b}{e_{b_1}}}
          \ctxstepsto
          \config{\extS{\extSRaw{S}{l}{u_{p_i}(p_1)}}{l'}{d_1}{\frozentrue}}{\evalctxt{E'_b}{e_{b_2}}}$,
          and
        \item
          $\config{\extS{S}{l'}{d_1}{\frozentrue}}{\evalctxt{E'_a}{e_{a_1}}}
          \ctxstepsto
          \config{\extS{\extSRaw{S}{l}{u_{p_i}(p_1)}}{l'}{d_1}{\frozentrue}}{\evalctxt{E'_a}{e_{a_2}}}$.
        \end{itemize}

        For the first of these, consider that
        $\extSRaw{S}{l}{u_{p_i}(p_1)} = U_S(S)$, where $U_S$ is the
        store update operation that applies $u_{p_i}$ to the
        contents of $l$ if it exists, and adds a binding
        $\storebindingRaw{l}{u_{p_i}(p_1)}$ if no binding for $l$
        exists, and acts as the identity on all other locations.

        Note that:
        \begin{itemize}
        \item $U_S$ is non-conflicting with $\config{S}{e_{b_1}}
          \parstepsto
          \config{\extS{S}{l'}{d_1}{\frozentrue}}{e_{b_2}}$, since
          no locations are allocated in the transition;
        \item $U_S(\extS{S}{l'}{d_1}{\frozentrue}) \neq \topS$,

          since $U_S(\extS{S}{l'}{d_1}{\frozentrue}) =
          \extSRaw{\extS{S}{l'}{d_1}{\frozentrue}}{l}{u_{p_i}(p_1)}$
          and we know $S \neq \topS$ and updating the contents of
          location $l$ to $u_{p_i}(p_1)$ and freezing the contents
          of location $l'$ in $S$ cannot cause it to become $\topS$
          (because if so, then we would have $S_a = \topS$ or $S_b =
          \topS$, which we know are not the case); and
        \item $U_S$ is freeze-safe with $\config{S}{e_{b_1}}
          \parstepsto
          \config{\extS{S}{l'}{d_1}{\frozentrue}}{e_{b_2}}$, since
          the only location that can change in status between $S$
          and $\extS{S}{l'}{d_1}{\frozentrue}$ is $l'$, and $U_S$
          acts as the identity on $l'$.
        \end{itemize}
        Therefore, by Lemma~\ref{lem:generalized-independence}
        (Generalized Independence), we have that

        $\config{U_S(S)}{e_{b_1}} \parstepsto
        \config{U_S(\extS{S}{l'}{d_1}{\frozentrue})}{e_{b_2}}$.

        Hence $\config{\extSRaw{S}{l}{u_{p_i}(p_1)}}{e_{b_1}}
        \parstepsto
        \config{\extSRaw{\extS{S}{l'}{d_1}{\frozentrue}}{l}{u_{p_i}(p_1)}}{e_{b_2}}$.

        By {\sc E-Eval-Ctxt}, it follows that

        $\config{\extSRaw{S}{l}{u_{p_i}(p_1)}}{\evalctxt{E'_b}{e_{b_1}}}
        \ctxstepsto
        \config{\extSRaw{\extS{S}{l'}{d_1}{\frozentrue}}{l}{u_{p_i}(p_1)}}{\evalctxt{E'_b}{e_{b_2}}}$,
        
        as we were required to show.

        For the second, consider that
        $\extS{S}{l'}{d_1}{\frozentrue} = U_S(S)$, where $U_S$ is
        the store update operation that freezes the contents of $l'$
        and acts as the identity on the contents of all other
        locations.

        Note that:
        \begin{itemize}
        \item $U_S$ is non-conflicting with $\config{S}{e_{a_1}}
          \parstepsto
          \config{\extSRaw{S}{l}{u_{p_i}(p_1)}}{e_{a_2}}$, since no
          locations are allocated in the transition;
        \item $U_S(\extSRaw{S}{l}{u_{p_i}(p_1)}) \neq \topS$, since
          $U_S(\extSRaw{S}{l}{u_{p_i}(p_1)}) =
          \extS{\extSRaw{S}{l}{u_{p_i}(p_1)}}{l'}{d_1}{\frozentrue}$,
          and we know $S \neq \topS$ and updating the contents of
          location $l$ to $u_{p_i}(p_1)$ and freezing the contents
          of location $l$ in $S$ cannot cause it to become $\topS$
          (because if so, then we would have $S_a = \topS$ or $S_b =
          \topS$, which we know are not the case); and
        \item $U_S$ is freeze-safe with $\config{S}{e_{a_1}}
          \parstepsto
          \config{\extSRaw{S}{l}{u_{p_i}(p_1)}}{e_{a_2}}$, since
          $u_{p_i}$ does not alter the status of $p_1$.

          (By Definition~\ref{def:set-of-state-update-operations},
          $u_{p_i}$ can only change the status bit of a location if
          its contents are $\state{d}{\frozentrue}$ and $u_i(d) \neq
          d$, in which case $u_{p_i}$ changes the contents of the
          location to $\state{\top}{\frozenfalse}$; however, that
          cannot be the case here since then $u_{p_i}(p_1)$ would be
          $\topp$, and we would have $S_a = \topS$, a contradiction.)
        \end{itemize}
        Therefore, by Lemma~\ref{lem:generalized-independence}
        (Generalized Independence), we have that

        $\config{U_S(S)}{e_{a_1}} \parstepsto
        \config{U_S(\extSRaw{S}{l}{u_{p_i}(p_1)})}{e_{a_2}}$.

        Hence $\config{\extS{S}{l'}{d_1}{\frozentrue}}{e_{a_1}}
        \parstepsto
        \config{\extS{\extSRaw{S}{l}{u_{p_i}(p_1)}}{l'}{d_1}{\frozentrue}}{e_{a_2}}$.

        By {\sc E-Eval-Ctxt}, it follows that
        $\config{\extS{S}{l'}{d_1}{\frozentrue}}{\evalctxt{E'_a}{e_{a_1}}}
        \ctxstepsto
        \config{\extS{\extSRaw{S}{l}{u_{p_i}(p_1)}}{l'}{d_1}{\frozentrue}}{\evalctxt{E'_a}{e_{a_2}}}$,
        
        as we were required to show.

      \item If $l = l'$:

        We have two cases to consider:

        \begin{itemize}
        \item $u_{p_i}(\state{d_1}{\frozentrue}) = \topp$:

          \lk{This is the interesting case: the potential
            put-after-freeze case.  It's important to note that this
            case doesn't necessarily end in a put-after-freeze (and
            hence an error); all we're required to show is that it
            \emph{can} end that way.}

          Since $(\extSRaw{S}{l}{\state{d_1}{\frozentrue}})(l) =
          \state{d_1}{\frozentrue}$ and
          $u_{p_i}(\state{d_1}{\frozentrue}) = \topp$, by {\sc
            E-Put-Err} we have that
          $\config{\extSRaw{S}{l}{\state{d_1}{\frozentrue}}}{\putiexp{l}}
          \parstepsto \error$.

          Since $S_b = \extSRaw{S}{l}{\state{d_1}{\frozentrue}}$,
          we have that $\config{S_b}{\putiexp{l}} \parstepsto
          \error$.

          Since $\config{S}{e_{a_1}} \parstepsto
          \config{S_a}{e_{a_2}}$ by {\sc E-Put}, it must be the
          case that $e_{a_1} = \putiexp{l}$.

          Hence $\config{S_b}{e_{a_1}} \parstepsto \error$.

          Since $\error$ is equal to $\config{\topS}{e}$ for all
          expressions $e$, $\config{S_b}{e_{a_1}} \parstepsto
          \config{\topS}{e}$ for all $e$.

          Therefore, by {\sc E-Eval-Ctxt},
          $\config{S_b}{\evalctxt{E'_a}{e_{a_1}}} \ctxstepsto
          \config{\topS}{\evalctxt{E'_a}{e}}$ for all $e$.

          Since $\config{\topS}{\evalctxt{E'_a}{e}}$ is equal to
          $\error$, we have that
          $\config{S_b}{\evalctxt{E'_a}{e_{a_1}}} \ctxstepsto \error$.

          Since $\evalctxt{E'_a}{e_{a_1}} =
          \evalctxt{E_b}{e_{b_2}}$, we have that
          $\config{S_b}{\evalctxt{E_b}{e_{b_2}}} \ctxstepsto
          \error$.

          Since $\conf_b = \config{S_b}{\evalctxt{E_b}{e_{b_2}}}$,
          we therefore have that $\conf_b \ctxstepsto \error$, and
          the case is satisfied.

        \item $u_{p_i}(\state{d_1}{\frozentrue}) \neq \topp$:

          \lk{This is the case where there's a conflicting put and
            freeze, but the put is a no-op, so it doesn't matter.}

          In this case, by the definition of $U_p$
          (Definition~\ref{def:set-of-state-update-operations}),
          
          it must be the case that $u_{p_i}(\state{d_1}{\frozentrue})
          = \state{d_1}{\frozentrue}$.

          Choose $S' = \extS{S}{l}{d_1}{\frozentrue}$, $i = 1$, $j
          = 1$, and $\pi = \id$.

          We have to show that:
          \begin{itemize}
          \item
            $\config{\extSRaw{S}{l}{u_{p_i}(p_1)}}{\evalctxt{E'_b}{e_{b_1}}}
            \ctxstepsto
            \config{\extS{S}{l}{d_1}{\frozentrue}}{\evalctxt{E'_b}{e_{b_2}}}$,
            and
          \item
            $\config{\extS{S}{l}{d_1}{\frozentrue}}{\evalctxt{E'_a}{e_{a_1}}}
            \ctxstepsto
            \config{\extS{S}{l}{d_1}{\frozentrue}}{\evalctxt{E'_a}{e_{a_2}}}$.
          \end{itemize}

          For the first of these, consider that
          $\extSRaw{S}{l}{u_{p_i}(p_1)} = U_S(S)$, where $U_S$ is
          the store update operation that applies $u_{p_i}$ to the
          contents of $l$ if it exists, and adds a binding
          $\storebindingRaw{l}{u_{p_i}(p_1)}$ if no binding for
          $l$ exists, and acts as the identity on all other
          locations.

          Note that:
          \begin{itemize}
          \item $U_S$ is non-conflicting with $\config{S}{e_{b_1}}
            \parstepsto
            \config{\extS{S}{l}{d_1}{\frozentrue}}{e_{b_2}}$,
            since no locations are allocated in the
            transition;
          \item $U_S(\extS{S}{l}{d_1}{\frozentrue}) \neq \topS$,
            
            since $U_S(\extS{S}{l}{d_1}{\frozentrue}) =
            \extSRaw{S}{l}{u_{p_i}(\state{d_1}{\frozentrue})}$ and
            we know $S \neq \topS$ and
            $u_{p_i}(\state{d_1}{\frozentrue}) \neq \topp$; and
          \item $U_S$ is freeze-safe with $\config{S}{e_{b_1}}
            \parstepsto
            \config{\extS{S}{l}{d_1}{\frozentrue}}{e_{b_2}}$, since
            the only location that can change in status between $S$
            and $\extS{S}{l}{d_1}{\frozentrue}$ is $l$, and $U_S$
            acts as the identity on $l$.
          \end{itemize}
          Therefore, by Lemma~\ref{lem:generalized-independence}
          (Generalized Independence), we have that

          $\config{U_S(S)}{e_{b_1}} \parstepsto
          \config{U_S(\extS{S}{l}{d_1}{\frozentrue})}{e_{b_2}}$.

          Hence $\config{\extSRaw{S}{l}{u_{p_i}(p_1)}}{e_{b_1}}
          \parstepsto
          \config{\extSRaw{S}{l}{u_{p_i}(\state{d_1}{\frozentrue})}}{e_{b_2}}$.

          Since $u_{p_i}(\state{d_1}{\frozentrue}) =
          \state{d_1}{\frozentrue}$,

          we have that
          $\config{\extSRaw{S}{l}{u_{p_i}(p_1)}}{e_{b_1}}
          \parstepsto
          \config{\extS{S}{l}{d_1}{\frozentrue}}{e_{b_2}}$.

          By {\sc E-Eval-Ctxt}, it follows that

          $\config{\extSRaw{S}{l}{u_{p_i}(p_1)}}{\evalctxt{E'_b}{e_{b_1}}}
          \ctxstepsto
          \config{\extS{S}{l}{d_1}{\frozentrue}}{\evalctxt{E'_b}{e_{b_2}}}$,

          as we were required to show.

          For the second, consider that
          $\extS{S}{l}{d_1}{\frozentrue} = U_S(S)$, where $U_S$ is
          the store update operation that freezes the contents of $l$
          and acts as the identity on the contents of all other
          locations.

          Note that:
          \begin{itemize}
          \item $U_S$ is non-conflicting with $\config{S}{e_{a_1}}
            \parstepsto
            \config{\extSRaw{S}{l}{u_{p_i}(p_1)}}{e_{a_2}}$, since no
            locations are allocated in the transition;
          \item $U_S(\extSRaw{S}{l}{u_{p_i}(p_1)}) \neq \topS$,
            since $U_S(\extSRaw{S}{l}{u_{p_i}(p_1)}) =
            \extS{S}{l}{d_1}{\frozentrue}$ (since, by
            Definition~\ref{def:set-of-state-update-operations},
            $u_i(d_1) = d_1$; otherwise we would have
            $u_{p_i}(\state{d_1}{\frozentrue}) = \topp$, a
            contradiction), and we know $S \neq \topS$ and
            freezing the contents of location $l$ in $S$ cannot
            cause it to become $\topS$; and
          \item $U_S$ is freeze-safe with $\config{S}{e_{a_1}}
            \parstepsto
            \config{\extSRaw{S}{l}{u_{p_i}(p_1)}}{e_{a_2}}$, since
            $u_{p_i}$ does not alter the status of $p_1$.

            (By Definition~\ref{def:set-of-state-update-operations},
            $u_{p_i}$ can only change the status bit of a location if
            its contents are $\state{d}{\frozentrue}$ and $u_i(d) \neq
            d$, in which case $u_{p_i}$ changes the contents of the
            location to $\state{\top}{\frozenfalse}$; however, that
            cannot be the case here since then $u_{p_i}(p_1)$ would be
            $\topp$, and we would have $S_a = \topS$, a
            contradiction.)
          \end{itemize}
          Therefore, by Lemma~\ref{lem:generalized-independence}
          (Generalized Independence), we have that

          $\config{U_S(S)}{e_{a_1}} \parstepsto
          \config{U_S(\extSRaw{S}{l}{u_{p_i}(p_1)})}{e_{a_2}}$.

          Hence $\config{\extS{S}{l}{d_1}{\frozentrue}}{e_{a_1}}
          \parstepsto
          \config{\extS{S}{l}{d_1}{\frozentrue}}{e_{a_2}}$.

          By {\sc E-Eval-Ctxt}, it follows that

          $\config{\extS{S}{l}{d_1}{\frozentrue}}{\evalctxt{E'_a}{e_{a_1}}}
          \ctxstepsto
          \config{\extS{S}{l}{d_1}{\frozentrue}}{\evalctxt{E'_a}{e_{a_2}}}$,

          as we were required to show.
        \end{itemize}

      \end{itemize}

    \item \label{slqc-put-freeze-simple}Case {\sc E-Freeze-Simple}:
      Similar to case~\ref{slqc-put-freeze-final}, since $S_a =
      \extSRaw{S}{l}{u_{p_i}(p_1)}$ and $S_b =
      \extS{S}{l'}{d_1}{\frozentrue}$.

    \end{enumerate}
  \item Case {\sc E-Put-Err}: We have $\config{S_a}{e_{a_2}} =
    \error$.

    We proceed by case analysis on the rule by which
    $\config{S}{e_{b_1}}$ steps to $\config{S_b}{e_{b_2}}$.

    Since the only way an $\error$ configuration can arise is by the
    {\sc E-Put-Err} rule, we can assume in all other cases that
    $\conf_b \neq \error$.
    \begin{enumerate}
    \item \label{slqc-put-err-beta}Case {\sc E-Beta}: By symmetry with case~\ref{slqc-beta-put-err}.
    \item \label{slqc-put-err-new}Case {\sc E-New}: By symmetry with case~\ref{slqc-new-put-err}.
    \item \label{slqc-put-err-put}Case {\sc E-Put}: By symmetry with case~\ref{slqc-put-put-err}.
    \item \label{slqc-put-err-put-err}Case {\sc E-Put-Err}: We have
      $\config{S_a}{e_{a_2}} = \error$ and $\config{S_b}{e_{b_2}} =
      \error$, and so we choose $\conf_c = \error$, $i = 0$, $j = 0$,
      and $\pi = \id$.

      We have to show that:
      \begin{itemize}
      \item $\config{S_a}{\evalctxt{E'_b}{e_{b_1}}} = \error$, and
      \item $\config{S_b}{\evalctxt{E'_a}{e_{a_1}}} = \error$.
      \end{itemize}

      Since $\config{S_a}{e_{a_2}} = \error$, $S_a = \topS$, and since
      $\config{S_b}{e_{b_2}} = \error$, $S_b = \topS$, so both of the
      above follow immediately.

    \item \label{slqc-put-err-get}Case {\sc E-Get}: Similar to
      case~\ref{slqc-put-err-beta}, since $\config{S_a}{e_{a_2}} =
      \error$ and $S_b = S$.
    \item \label{slqc-put-err-freeze-init}Case {\sc E-Freeze-Init}:
      Similar to case~\ref{slqc-put-err-beta}, since
      $\config{S_a}{e_{a_2}} = \error$ and $S_b = S$.
    \item \label{slqc-put-err-spawn-handler}Case {\sc
      E-Spawn-Handler}: Similar to case~\ref{slqc-put-err-beta}, since
      $\config{S_a}{e_{a_2}} = \error$ and $S_b = S$.
    \item \label{slqc-put-err-freeze-final}Case {\sc E-Freeze-Final}:
      We have $\config{S_a}{e_{a_2}} = \error$ and $S_b =
      \extS{S}{l}{d_1}{\frozentrue}$, and so we choose $\conf_c =
      \error$, $i = 0$, $j = 1$, and $\pi = \id$.

      We have to show that:
      \begin{itemize}
      \item $\config{S_a}{\evalctxt{E'_b}{e_{b_1}}} = \error$,
        and
      \item $\config{S_b}{\evalctxt{E'_a}{e_{a_1}}} \ctxstepsto
        \error$.
      \end{itemize}

      The first of these is immediately true because since
      $\config{S_a}{e_{a_2}} = \error$, $S_a = \topS$, and so
      $\config{S_a}{\evalctxt{E'_b}{e_{b_1}}}$ is equal to $\error$ as
      well.

      For the second, observe that since $\config{S}{e_{b_1}}
      \parstepsto \config{S_b}{e_{b_2}}$, we have by
      Lemma~\ref{lem:monotonicity} (Monotonicity) that
      $\leqstore{S}{S_b}$.

      Therefore, since $\config{S}{e_{a_1}} \parstepsto \error$, we
      have by Lemma~\ref{lem:error-preservation} that
      $\config{S_b}{e_{a_1}} \parstepsto \error$.

      Since $\error$ is equal to $\config{\topS}{e}$ for all
      expressions $e$, $\config{S_b}{e_{a_1}} \parstepsto
      \config{\topS}{e}$ for all $e$.

      Therefore, by {\sc E-Eval-Ctxt},
      $\config{S_b}{\evalctxt{E'_a}{e_{a_1}}} \ctxstepsto
      \config{\topS}{\evalctxt{E'_a}{e}}$ for all $e$.

      Since $\config{\topS}{\evalctxt{E'_a}{e}}$ is equal to $\error$,
      we have that $\config{S_b}{\evalctxt{E'_a}{e_{a_1}}} \ctxstepsto
      \error$, as we were required to show.

    \item \label{slqc-put-err-freeze-simple}Case {\sc
      E-Freeze-Simple}: Similar to
      case~\ref{slqc-put-err-freeze-final}, since $S_b =
      \extS{S}{l}{d_1}{\frozentrue}$.

    \end{enumerate}
  \item Case {\sc E-Get}: We have $S_a = S$.

    We proceed by case analysis on the rule by which
    $\config{S}{e_{b_1}}$ steps to $\config{S_b}{e_{b_2}}$.

    Since the only way an $\error$ configuration can arise is by the
    {\sc E-Put-Err} rule, we can assume in all other cases that
    $\conf_b \neq \error$.
    \begin{enumerate}
    \item \label{slqc-get-beta}Case {\sc E-Beta}: By symmetry with case~\ref{slqc-beta-get}.
    \item \label{slqc-get-new}Case {\sc E-New}: By symmetry with case~\ref{slqc-new-get}.
    \item \label{slqc-get-put}Case {\sc E-Put}: By symmetry with case~\ref{slqc-put-get}.
    \item \label{slqc-get-put-err}Case {\sc E-Put-Err}: By symmetry with case~\ref{slqc-put-err-get}.
    \item \label{slqc-get-get}Case {\sc E-Get}: Similar to
      case~\ref{slqc-get-beta}, since $S_a = S$ and $S_b = S$.
    \item \label{slqc-get-freeze-init}Case {\sc E-Freeze-Init}:
      Similar to case~\ref{slqc-get-beta}, since $S_a = S$ and $S_b = S$.
    \item \label{slqc-get-spawn-handler}Case {\sc E-Spawn-Handler}:
      Similar to case~\ref{slqc-get-beta}, since $S_a = S$ and $S_b = S$.
    \item \label{slqc-get-freeze-final}Case {\sc E-Freeze-Final}:
      Similar to case~\ref{slqc-beta-freeze-final}, since $S_a = S$
      and $S_b = \extS{S}{l}{d_1}{\frozentrue}$.
    \item \label{slqc-get-freeze-simple}Case {\sc E-Freeze-Simple}:
      Similar to case~\ref{slqc-beta-freeze-simple}, since $S_a = S$
      and $S_b = \extS{S}{l}{d_1}{\frozentrue}$.
    \end{enumerate}

  \item Case {\sc E-Freeze-Init}: We have $S_a = S$.

    We proceed by case analysis on the rule by which
    $\config{S}{e_{b_1}}$ steps to $\config{S_b}{e_{b_2}}$.

    Since the only way an $\error$ configuration can arise is by the
    {\sc E-Put-Err} rule, we can assume in all other cases that
    $\conf_b \neq \error$.
    \begin{enumerate}
    \item \label{slqc-freeze-init-beta}Case {\sc E-Beta}: By symmetry with case~\ref{slqc-beta-freeze-init}.
    \item \label{slqc-freeze-init-new}Case {\sc E-New}: By symmetry with case~\ref{slqc-new-freeze-init}.
    \item \label{slqc-freeze-init-put}Case {\sc E-Put}: By symmetry with case~\ref{slqc-put-freeze-init}.
    \item \label{slqc-freeze-init-put-err}Case {\sc E-Put-Err}: By symmetry with case~\ref{slqc-put-err-freeze-init}.
    \item \label{slqc-freeze-init-get}Case {\sc E-Get}: By symmetry with case~\ref{slqc-get-freeze-init}.
    \item \label{slqc-freeze-init-freeze-init}Case {\sc
      E-Freeze-Init}: Similar to case~\ref{slqc-freeze-init-beta},
      since $S_a = S$ and $S_b = S$.
    \item \label{slqc-freeze-init-spawn-handler}Case {\sc
      E-Spawn-Handler}: Similar to case~\ref{slqc-freeze-init-beta},
      since $S_a = S$ and $S_b = S$.
    \item \label{slqc-freeze-init-freeze-final}Case {\sc
      E-Freeze-Final}: Similar to case~\ref{slqc-beta-freeze-final},
      since $S_a = S$ and $S_b = \extS{S}{l}{d_1}{\frozentrue}$.
    \item \label{slqc-freeze-init-freeze-simple}Case {\sc
      E-Freeze-Simple}: Similar to case~\ref{slqc-beta-freeze-simple},
      since $S_a = S$ and $S_b = \extS{S}{l}{d_1}{\frozentrue}$.
    \end{enumerate}

  \item Case {\sc E-Spawn-Handler}: We have $S_a = S$.

    We proceed by case analysis on the rule by which
    $\config{S}{e_{b_1}}$ steps to $\config{S_b}{e_{b_2}}$.

    Since the only way an $\error$ configuration can arise is by the
    {\sc E-Put-Err} rule, we can assume in all other cases that
    $\conf_b \neq \error$.
    \begin{enumerate}
    \item \label{slqc-spawn-handler-beta}Case {\sc E-Beta}: By symmetry with case~\ref{slqc-beta-spawn-handler}.
    \item \label{slqc-spawn-handler-new}Case {\sc E-New}: By symmetry with case~\ref{slqc-new-spawn-handler}.
    \item \label{slqc-spawn-handler-put}Case {\sc E-Put}: By symmetry with case~\ref{slqc-put-spawn-handler}.
    \item \label{slqc-spawn-handler-put-err}Case {\sc E-Put-Err}: By symmetry with case~\ref{slqc-put-err-spawn-handler}.
    \item \label{slqc-spawn-handler-get}Case {\sc E-Get}: By symmetry with case~\ref{slqc-get-spawn-handler}.
    \item \label{slqc-spawn-handler-freeze-init}Case {\sc E-Freeze-Init}: By symmetry with case~\ref{slqc-freeze-init-spawn-handler}.
    \item \label{slqc-spawn-handler-spawn-handler}Case {\sc
      E-Spawn-Handler}: Similar to case~\ref{slqc-spawn-handler-beta},
      since $S_a = S$ and $S_b = S$.
    \item \label{slqc-spawn-handler-freeze-final}Case {\sc
      E-Freeze-Final}: Similar to case~\ref{slqc-beta-freeze-final},
      since $S_a = S$ and $S_b = \extS{S}{l}{d_1}{\frozentrue}$.
    \item \label{slqc-spawn-handler-freeze-simple}Case {\sc
      E-Freeze-Simple}: Similar to case~\ref{slqc-beta-freeze-simple},
      since $S_a = S$ and $S_b = \extS{S}{l}{d_1}{\frozentrue}$.
    \end{enumerate}

  \item Case {\sc E-Freeze-Final}: We have $S_a =
    \extS{S}{l}{d_1}{\frozentrue}$.

    We proceed by case analysis on the rule by which
    $\config{S}{e_{b_1}}$ steps to $\config{S_b}{e_{b_2}}$.

    Since the only way an $\error$ configuration can arise is by the
    {\sc E-Put-Err} rule, we can assume in all other cases that
    $\conf_b \neq \error$.
    \begin{enumerate}
    \item \label{slqc-freeze-final-beta}Case {\sc E-Beta}: By symmetry with case~\ref{slqc-beta-freeze-final}.
    \item \label{slqc-freeze-final-new}Case {\sc E-New}: By symmetry with case~\ref{slqc-new-freeze-final}.
    \item \label{slqc-freeze-final-put}Case {\sc E-Put}: By symmetry with case~\ref{slqc-put-freeze-final}.
    \item \label{slqc-freeze-final-put-err}Case {\sc E-Put-Err}: By symmetry with case~\ref{slqc-put-err-freeze-final}.
    \item \label{slqc-freeze-final-get}Case {\sc E-Get}: By symmetry with case~\ref{slqc-get-freeze-final}.
    \item \label{slqc-freeze-final-freeze-init}Case {\sc E-Freeze-Init}: By symmetry with case~\ref{slqc-freeze-init-freeze-final}.
    \item \label{slqc-freeze-final-spawn-handler}Case {\sc E-Spawn-Handler}: By symmetry with case~\ref{slqc-spawn-handler-freeze-final}.
    \item \label{slqc-freeze-final-freeze-final}Case {\sc
      E-Freeze-Final}: We have $S_a = \extS{S}{l}{d_1}{\frozentrue}$
      and $S_b = \extS{S}{l'}{d'_1}{\frozentrue}$.

      Now consider whether $l = l'$:
      \begin{itemize}
      \item If $l \neq l'$:

        Choose $S' =
        \extS{\extS{S}{l'}{d'_1}{\frozentrue}}{l}{d_1}{\frozentrue}$,
        $i = 1$, $j = 1$, and $\pi = \id$.

        We have to show that:
        \begin{itemize}
        \item
          $\config{\extS{S}{l}{d_1}{\frozentrue}}{\evalctxt{E'_b}{e_{b_1}}}
          \ctxstepsto
          \config{\extS{\extS{S}{l'}{d'_1}{\frozentrue}}{l}{d_1}{\frozentrue}}{\evalctxt{E'_b}{e_{b_2}}}$,
          and
        \item
          $\config{\extS{S}{l'}{d'_1}{\frozentrue}}{\evalctxt{E'_a}{e_{a_1}}}
          \ctxstepsto
          \config{\extS{\extS{S}{l'}{d'_1}{\frozentrue}}{l}{d_1}{\frozentrue}}{\evalctxt{E'_a}{e_{a_2}}}$.
        \end{itemize}

        For the first of these, consider that
        $\extS{S}{l}{d_1}{\frozentrue} = U_S(S)$, where $U_S$ is the
        store update operation that freezes the contents of $l$
        and acts as the identity on the contents of all other
        locations.

        Note that:
        \begin{itemize}
        \item $U_S$ is non-conflicting with $\config{S}{e_{b_1}}
          \parstepsto
          \config{\extS{S}{l'}{d'_1}{\frozentrue}}{e_{b_2}}$, since
          no locations are allocated in the transition;
        \item $U_S(\extS{S}{l'}{d'_1}{\frozentrue}) \neq \topS$,

          since $U_S(\extS{S}{l'}{d'_1}{\frozentrue}) =
          \extS{\extS{S}{l'}{d'_1}{\frozentrue}}{l}{d_1}{\frozentrue}$
          and we know $S \neq \topS$ and freezing the contents of
          locations $l$ and $l'$ in $S$ cannot cause it to become
          $\topS$ (because if so, then we would have $S_a = \topS$
          or $S_b = \topS$, which we know are not the case); and
        \item $U_S$ is freeze-safe with $\config{S}{e_{b_1}}
          \parstepsto
          \config{\extS{S}{l'}{d'_1}{\frozentrue}}{e_{b_2}}$, since
          the only location that can change in status between $S$
          and $\extS{S}{l'}{d'_1}{\frozentrue}$ is $l'$, and $U_S$
          acts as the identity on $l'$.
        \end{itemize}
        Therefore, by Lemma~\ref{lem:generalized-independence}
        (Generalized Independence), we have that

        $\config{U_S(S)}{e_{b_1}} \parstepsto
        \config{U_S(\extS{S}{l'}{d'_1}{\frozentrue})}{e_{b_2}}$.

        Hence $\config{\extS{S}{l}{d_1}{\frozentrue}}{e_{b_1}}
        \parstepsto
        \config{\extS{\extS{S}{l'}{d'_1}{\frozentrue}}{l}{d_1}{\frozentrue}}{e_{b_2}}$.

        By {\sc E-Eval-Ctxt}, it follows that

        $\config{\extS{S}{l}{d_1}{\frozentrue}}{\evalctxt{E'_b}{e_{b_1}}}
        \ctxstepsto
        \config{\extS{\extS{S}{l'}{d'_1}{\frozentrue}}{l}{d_1}{\frozentrue}}{\evalctxt{E'_b}{e_{b_2}}}$,
        
        as we were required to show.

        The argument for the second is symmetrical.

      \item If $l = l'$:

        \lk{This is the case where we freeze the same location twice,
          which is no problem; the second freeze is a no-op.}

        Note that since $l = l'$, $d_1 = d'_1$ as well.

        Choose $S' = \extS{S}{l}{d_1}{\frozentrue}$, $i = 1$, $j =
        1$, and $\pi = \id$.

        We have to show that:
        \begin{itemize}
        \item
          $\config{\extS{S}{l}{d_1}{\frozentrue}}{\evalctxt{E'_b}{e_{b_1}}}
          \ctxstepsto
          \config{\extS{S}{l}{d_1}{\frozentrue}}{\evalctxt{E'_b}{e_{b_2}}}$,
          and
        \item
          $\config{\extS{S}{l'}{d'_1}{\frozentrue}}{\evalctxt{E'_a}{e_{a_1}}}
          \ctxstepsto
          \config{\extS{S}{l}{d_1}{\frozentrue}}{\evalctxt{E'_a}{e_{a_2}}}$.
        \end{itemize}

        For the first of these, consider that
        $\extS{S}{l}{d_1}{\frozentrue} = U_S(S)$, where $U_S$ is the
        store update operation that freezes the contents of $l$ and
        acts as the identity on the contents of all other locations.

        Note that:
        \begin{itemize}
        \item $U_S$ is non-conflicting with $\config{S}{e_{b_1}}
          \parstepsto
          \config{\extS{S}{l}{d_1}{\frozentrue}}{e_{b_2}}$, since no
          locations are allocated in the transition;
        \item $U_S(\extS{S}{l}{d_1}{\frozentrue}) \neq \topS$, since
          $U_S(\extS{S}{l}{d_1}{\frozentrue}) =
          \extS{S}{l}{d_1}{\frozentrue}$, and we know $S \neq \topS$
          and freezing the contents of location $l$ in $S$ cannot
          cause it to become $\topS$; and
        \item $U_S$ is freeze-safe with $\config{S}{e_{b_1}}
          \parstepsto
          \config{\extS{S}{l}{d_1}{\frozentrue}}{e_{b_2}}$, since
          the only location that can change in status between $S$
          and $\extS{S}{l}{d_1}{\frozentrue}$ is $l$, and $U_S$
          freezes the contents of $l$ but has no other effect on
          them.
        \end{itemize}

        Therefore, by Lemma~\ref{lem:generalized-independence}
        (Generalized Independence), we have that

        $\config{U_S(S)}{e_{b_1}} \parstepsto
        \config{U_S(\extS{S}{l}{d_1}{\frozentrue})}{e_{b_2}}$.

        Hence $\config{\extS{S}{l}{d_1}{\frozentrue}}{e_{b_1}}
        \parstepsto
        \config{\extS{S}{l}{d_1}{\frozentrue}}{e_{b_2}}$.

        By {\sc E-Eval-Ctxt}, it follows that

        $\config{\extS{S}{l}{d_1}{\frozentrue}}{\evalctxt{E'_b}{e_{b_1}}}
        \ctxstepsto
        \config{\extS{S}{l}{d_1}{\frozentrue}}{\evalctxt{E'_b}{e_{b_2}}}$,

        as we were required to show.

        The argument for the second is symmetrical.

      \end{itemize}

    \item \label{slqc-freeze-final-freeze-simple}Case {\sc
      E-Freeze-Simple}: Similar to
      case~\ref{slqc-freeze-final-freeze-final}, since $S_a =
      \extS{S}{l}{d_1}{\frozentrue}$ and $S_b =
      \extS{S}{l'}{d'_1}{\frozentrue}$.
    \end{enumerate}

  \item Case {\sc E-Freeze-Simple}: We have $S_a =
    \extS{S}{l}{d_1}{\frozentrue}$.

    \begin{enumerate}
    \item \label{slqc-freeze-simple-beta}Case {\sc E-Beta}: By symmetry with case~\ref{slqc-beta-freeze-simple}.
    \item \label{slqc-freeze-simple-new}Case {\sc E-New}: By symmetry with case~\ref{slqc-new-freeze-simple}.
    \item \label{slqc-freeze-simple-put}Case {\sc E-Put}: By symmetry with case~\ref{slqc-put-freeze-simple}.
    \item \label{slqc-freeze-simple-put-err}Case {\sc E-Put-Err}: By symmetry with case~\ref{slqc-put-err-freeze-simple}.
    \item \label{slqc-freeze-simple-get}Case {\sc E-Get}: By symmetry with case~\ref{slqc-get-freeze-simple}.
    \item \label{slqc-freeze-simple-freeze-init}Case {\sc E-Freeze-Init}: By symmetry with case~\ref{slqc-freeze-init-freeze-simple}.
    \item \label{slqc-freeze-simple-spawn-handler}Case {\sc E-Spawn-Handler}: By symmetry with case~\ref{slqc-spawn-handler-freeze-simple}.
    \item \label{slqc-freeze-simple-freeze-final}Case {\sc E-Freeze-Final}: By symmetry with case~\ref{slqc-freeze-final-freeze-simple}.
    \item \label{slqc-freeze-simple-freeze-simple}Case {\sc
      E-Freeze-Simple}: Similar to
      case~\ref{slqc-freeze-simple-freeze-final}, since $S_a =
      \extS{S}{l}{d_1}{\frozentrue}$ and $S_b =
      \extS{S}{l'}{d'_1}{\frozentrue}$.
    \end{enumerate}

  \end{enumerate}
\end{proof}



\section{Proof of Lemma~\ref{lem:strong-one-sided-quasi-confluence}}\label{section:strong-one-sided-quasi-confluence-proof}
\begin{proof}
  Suppose $\conf \ctxstepsto \conf'$ and $\conf \ctxstepsto^m
  \conf''$, where $1 \leq m$.

  We are required to show that either:
  \begin{enumerate}
  \item there exist $\conf_c, i, j, \pi$ such that $\conf'
    \ctxstepsto^i \conf_c$ and $\pi(\conf'') \ctxstepsto^j \conf_c$
    and $i \leq m$ and $j \leq 1$, or
  \item there exists $k \leq m$ such that $\conf' \ctxstepsto^k
    \textup{\error}$, or there exists $k \leq 1$ such that $\conf''
    \ctxstepsto^k \textup{\error}$.
  \end{enumerate}

  We proceed by induction on $m$.

  In the base case of $m = 1$, the result is immediate from
  Lemma~\ref{lem:strong-local-quasi-confluence}, with $k = 1$.

  For the induction step, suppose $\conf \ctxstepsto^m \conf''
  \ctxstepsto \conf'''$ and suppose the lemma holds for $m$.

  We show that it holds for $m + 1$, as follows.

  From the induction hypothesis, we have that either:
  \begin{enumerate}
  \item there exist $\conf_c', i', j', \pi'$ such that $\conf'
    \ctxstepsto^{i'} \conf_c'$ and $\pi'(\conf'') \ctxstepsto^{j'}
    \conf_c'$ and $i' \leq m$ and $j' \leq 1$, or
  \item there exists $k' \leq m$ such that $\conf'
    \ctxstepsto^{k'} \error$, or there exists $k' \leq 1$ such that
    $\conf'' \ctxstepsto^{k'} \error$.
  \end{enumerate}

  We consider these two cases in turn:
  \begin{enumerate}
  \item There exist $\conf_c', i', j', \pi'$ such that $\conf'
    \ctxstepsto^{i'} \conf_c'$ and $\pi'(\conf'') \ctxstepsto^{j'}
    \conf_c'$ and $i' \leq m$ and $j' \leq 1$:

    We proceed by cases on $j'$:
    \begin{itemize}

    \item If $j' = 0$, then $\pi'(\conf'') = \conf_c'$.

      Since $\conf'' \ctxstepsto \conf'''$, we have that
      $\pi'(\conf'') \ctxstepsto \pi'(\conf''')$ by
      Lemma~\ref{lem:permutability} (Permutability).

      We can then choose $\conf_c = \pi'(\conf''')$ and $i = i' + 1$
      and $j = 0$ and $\pi = \pi'$.

      The key is that $\conf' \ctxstepsto^{i'} \conf'_c =
      \pi'(\conf'') \ctxstepsto \pi'(\conf''')$ for a total of $i' +
      1$ steps.
      
    \item If $j' = 1$:

      First, since $\pi'(\conf'') \ctxstepsto^{j'} \conf'_c$, then
      by Lemma~\ref{lem:permutability} (Permutability) we have that
      $\conf'' \ctxstepsto^{j'} \piprimeinv(\conf'_c)$.
      
      Then, by $\conf'' \ctxstepsto^{j'} \piprimeinv(\conf'_c)$ and
      $\conf'' \ctxstepsto \conf'''$ and
      Lemma~\ref{lem:strong-local-quasi-confluence}, one of the
      following two cases is true:
      \begin{enumerate}
      \item There exist $\conf_c''$ and $i''$ and $j''$ and $\pi''$
        such that $\piprimeinv(\conf'_c) \ctxstepsto^{i''}
        \conf_c''$ and $\pi''(\conf''') \ctxstepsto^{j''} \conf_c''$
        and $i'' \leq 1$ and $j'' \leq 1$.

        Since $\piprimeinv(\conf'_c) \ctxstepsto^{i''} \conf_c''$,
        by Lemma~\ref{lem:permutability} (Permutability) we have
        that $\conf'_c \ctxstepsto^{i''} \pi'(\conf_c'')$.

        So we also have $\conf' \ctxstepsto^{i'} \conf_c'
        \ctxstepsto^{i''} \pi'(\conf_c'')$.

        Since $\pi''(\conf''') \ctxstepsto^{j''} \conf_c''$, by
        Lemma~\ref{lem:permutability} (Permutability) we have that
        $\pi'(\pi''(\conf''')) \ctxstepsto^{j''} \pi'(\conf_c'')$.

        In summary, we pick $\conf_c = \pi'(\conf_c'')$ and $i = i' + i''$
        and $j = j''$ and $\pi = \pi'' \circ \pi'$, which is sufficient
        because $i = i' + i'' \leq m + 1$ and $j = j'' \leq 1$.

      \item $\piprimeinv(\conf'_c) \ctxstepsto \error$ or $\conf'''
        \ctxstepsto \error$.

        If $\conf''' \ctxstepsto \error$, then choosing $k = 1$
        satisfies the proof.

        Otherwise, $\piprimeinv(\conf'_c) \ctxstepsto \error$.

        Then, by Lemma~\ref{lem:permutability} we have that
        $\conf'_c \ctxstepsto \pi'(\error)$.

        By Definition~\ref{def:permutation-configuration},
        $\pi'(\error) = \error$, and so $\conf'_c \ctxstepsto
        \error$.

        Therefore $\conf' \ctxstepsto^{i'} \conf'_c \ctxstepsto
        \error$.

        Hence $\conf' \ctxstepsto^{i'+1} \error$.

        Since $i' \leq m$, we have that $i' + 1 \leq m + 1$, and
        so choosing $k = i' + 1$ satisfies the proof.
        
      \end{enumerate}

    \end{itemize}

  \item There exists $k' \leq m$ such that $\conf' \ctxstepsto^{k'}
    \error$, or there exists $k' \leq 1$ such that $\conf''
    \ctxstepsto^{k'} \error$:

    If there exists $k' \leq m$ such that $\conf' \ctxstepsto^{k'}
    \error$, then choosing $k = k'$ satisfies the proof.

    Otherwise, there exists $k' \leq 1$ such that $\conf''
    \ctxstepsto^{k'} \error$.

    We proceed by cases on $k'$:

    \begin{itemize}

    \item If $k' = 0$, then $\conf'' = \error$.

      Hence this case is not possible, since $\conf'' \ctxstepsto
      \conf'''$ and $\error$ cannot step.

    \item If $k' = 1$:

      From $\conf'' \ctxstepsto \conf'''$ and $\conf''
      \ctxstepsto^{k'} \error$ and
      Lemma~\ref{lem:strong-local-quasi-confluence}, one of the
      following two cases is true:

      \begin{enumerate}
      \item There exist $\conf_c''$ and $i''$ and $j''$ and $\pi''$
        such that $\error \ctxstepsto^{i''} \conf_c''$ and
        $\pi''(\conf''') \ctxstepsto^{j''} \conf_c''$ and $i'' \leq
        1$ and $j'' \leq 1$.

        Since $\error$ cannot step, $i'' = 0$ and $\conf''_c =
        \error$.

        By Definition~\ref{def:permutation-configuration},
        $\pi''(\conf''') = \conf'''$.

        Hence $\conf''' \ctxstepsto^{j''} \error$.

        \lk{This is the one place that we need to allow $k$ to be
          $\leq$ 1 instead of exactly 1.}

        Since $j'' \leq 1$, choosing $k = j''$ satisfies the proof.

      \item $\error \ctxstepsto \error$ or $\conf''' \ctxstepsto
        \error$.

        Since $\error$ cannot step, $\conf''' \ctxstepsto \error$.

        Hence choosing $k = 1$ satisfies the proof.

      \end{enumerate}

    \end{itemize}

  \end{enumerate}

\end{proof}


\section{Proof of Lemma~\ref{lem:strong-quasi-confluence}}\label{section:strong-quasi-confluence-proof}
\begin{proof}
  We proceed by induction on $n$.  In the base case of $n = 1$, the
  result is immediate from Lemma~\ref{lem:strong-one-sided-quasi-confluence}.

  For the induction step, suppose $\conf \parstepsto^n \conf'
  \parstepsto \conf'''$ and suppose the lemma holds for $n$.

  We show that it holds for $n + 1$, as follows.

  We are required to show that either:
  \begin{enumerate}
  \item there exist $\conf_c, i, j$ such that $\conf''' \parstepsto^i
    \conf_c$ and $\conf'' \parstepsto^j \conf_c$ and $i \leq m$ and $j
    \leq n + 1$, or
  \item there exists $k \leq m$ such that $\conf''' \parstepsto^k
    \error$, or there exists $k \leq n + 1$ such that $\conf''
    \parstepsto^k \error$.
  \end{enumerate}

  From the induction hypothesis, we have that either:
  \begin{enumerate}
  \item there exist $\conf'_c, i', j'$ such that $\conf'
    \parstepsto^{i'} \conf'_c$ and $\conf'' \parstepsto^{j'} \conf'_c$
    and $i' \leq m$ and $j' \leq n$, or
  \item there exists $k' \leq m$ such that $\conf' \parstepsto^{k'}
    \error$, or there exists $k' \leq n$ such that $\conf''
    \parstepsto^{k'} \error$.
  \end{enumerate}

  We consider these two cases in turn:

  \begin{enumerate}
  \item There exist $\conf'_c, i', j'$ such that $\conf'
    \parstepsto^{i'} \conf'_c$ and $\conf'' \parstepsto^{j'} \conf'_c$
    and $i' \leq m$ and $j' \leq n$:

    We proceed by cases on $i'$:
    \begin{itemize}

    \item If $i' = 0$, then $\conf' = \conf_c'$.  We can then choose
      $\conf_c = \conf'''$ and $i = 0$ and $j = j' + 1$.

    \item If $i' \geq 1$:

      From $\conf' \parstepsto \conf'''$ and $\conf' \parstepsto^{i'}
      \conf_c'$ and Lemma~\ref{lem:strong-one-sided-quasi-confluence},
      one of the following two cases is true:
      \begin{enumerate}
        \item There exist $\conf_c''$ and $i''$ and $j''$ such that
          $\conf''' \parstepsto^{i''} \conf_c''$ and $\conf_c'
          \parstepsto^{j''} \conf_c''$ and $i'' \leq i'$ and $j'' \leq
          1$.  So we also have $\conf'' \parstepsto^{j'} \conf_c'
          \parstepsto^{j''} \conf_c''$.  In summary, we pick $\conf_c
          = \conf_c''$ and $i = i''$ and $j = j' + j''$, which is
          sufficient because $i = i'' \leq i' \leq m$ and $j = j' +
          j'' \leq n + 1$.
        \item There exists $k'' \leq i'$ such that $\conf'''
          \parstepsto^{k''} \error$, or there exists $k'' \leq 1$ such
          that $\conf'_c \parstepsto^{k''} \error$.

          If there exists $k'' \leq i'$ such that $\conf'''
          \parstepsto^{k''} \error$, then choosing $k = k''$ satisfies
          the proof, since $k'' \leq i' \leq m$.

          Otherwise, there exists $k'' \leq 1$ such
          that $\conf'_c \parstepsto^{k''} \error$.

          Therefore, $\conf'' \parstepsto^{j'} \conf_c'
          \parstepsto^{k''} \error$.

          Hence $\conf'' \parstepsto^{j' + k''} \error$.

          Since $j' \leq n$ and $k'' \leq 1$, $j' + k'' \leq n + 1$.

          Hence choosing $k = j' + k''$ satisfies the proof.

      \end{enumerate}
    \end{itemize}

  \item There exists $k' \leq m$ such that $\conf' \parstepsto^{k'}
    \error$, or there exists $k' \leq n$ such that $\conf''
    \parstepsto^{k'} \error$:

    If there exists $k' \leq n$ such that $\conf'' \parstepsto^{k'}
    \error$, then choosing $k = k'$ satisfies the proof.

    Otherwise, there exists $k' \leq m$ such that $\conf'
    \parstepsto^{k'} \error$.  We proceed by cases on $k'$:

    \begin{itemize}

    \item If $k' = 0$, then $\conf' = \error$.

      Hence this case is not possible, since $\conf' \parstepsto
      \conf'''$ and $\error$ cannot step.

    \item If $k' \geq 1$:

      From $\conf' \parstepsto \conf'''$ and $\conf' \parstepsto^{k'}
      \error$ and Lemma~\ref{lem:strong-one-sided-quasi-confluence},
      one of the following two cases is true:

      \begin{enumerate}
        \item There exist $\conf''_c$ and $i''$ and $j''$ such that
          $\conf''' \parstepsto^{i''} \conf''_c$ and $\error
          \parstepsto^{j''} \conf''_c$ and $i'' \leq k'$ and $j'' \leq
          1$.

          Since $\error$ cannot step, $j'' = 0$ and $\conf''_c =
          \error$.

          Hence $\conf''' \parstepsto^{i''} \error$.

          Since $i'' \leq k' \leq m$, choosing $k = i''$ satisfies the
          proof.

        \item There exists $k'' \leq k'$ such that $\conf'''
          \parstepsto^{k''} \error$, or there exists $k'' \leq 1$ such
          that $\error \parstepsto^{k''} \error$.

          Since $\error$ cannot step, there exists $k'' \leq k'$ such
          that $\conf''' \parstepsto^{k''} \error$.

          Since $k'' \leq k' \leq m$, choosing $k = k''$ satisfies the
          proof.
      \end{enumerate}
    \end{itemize}
  \end{enumerate}

\end{proof}


\section{Proof of Theorem~\ref{thm:determinism-of-threshold-queries}}\label{section:determinism-of-threshold-queries-proof}
\begin{proof}
  Consider replica $i$ of a threshold CvRDT $(S, \leq, s^0, q, t, u,
  m)$.

  Let $\mathcal{S}$ be a threshold set with respect to
  $(S, \leq)$.

  Consider a method execution $t^{k+1}_i(\mathcal{S})$ (\ie, a
  threshold query that is the $k+1$th method execution on replica $i$,
  with threshold set $\mathcal{S}$ as its argument) that returns some
  set of activation states $S_a \in \mathcal{S}$.

  For part~\ref{thm:this-replica} of the theorem, we have to show that
  threshold queries with $\mathcal{S}$ as their argument will always
  return $S_a$ on subsequent executions at $i$.

  That is, we have to show that, for all $k' > (k+1)$, the threshold
  query $t^{k'}_i(\mathcal{S})$ on $i$ returns $S_a$.

  Since $t^{k+1}_i(\mathcal{S})$ returns $S_a$, from
  Definition~\ref{def:cvrdt-with-threshold-queries} we have that for
  some activation state $s_a \in S_a$, the condition $s_a \leq s^k_i$
  holds.

  Consider arbitrary $k' > (k+1)$.

  Since state is inflationary across updates, we know that the state
  $s^{k'}_i$ after method execution $k'$ is at least $s^k_i$.

  That is, $s^k_i \leq s^{k'}_i$.

  By transitivity of $\leq$, then, $s_a \leq s^{k'}_i$.

  Hence, by Definition~\ref{def:cvrdt-with-threshold-queries},
  $t^{k'}_i(\mathcal{S})$ returns $S_a$.

  For part~\ref{thm:any-replica} of the theorem, consider some replica
  $j$ of $(S, \leq, s^0, q, t, u, m)$, located at process $p_j$.

  We are required to show that, for all $x \geq 0$, the threshold
  query $t^{x+1}_j(\mathcal{S})$ returns $S_a$ eventually, and blocks
  until it does.\footnote{The occurrences of $k+1$ and $x+1$ in this
    proof are an artifact of how we index method executions starting
    from $1$, but states starting from $0$.  The initial state (of
    every replica) is $s^0$, and so $s^k_i$ is the state of replica
    $i$ after method execution $k$ has completed at $i$.}

  That is, we must show that, for all $x \geq 0$, there exists some
  finite $n \geq 0$ such that
  \begin{itemize}
  \item 
    for all $i$ in the range $0 \leq i \leq n-1$, the threshold query
    $t^{x+1+i}_j(\mathcal{S})$ returns $\block$, and
  \item
    for all $i \geq n$, the threshold query $t^{x+1+i}_j(\mathcal{S})$
    returns $S_a$.
  \end{itemize}
  Consider arbitrary $x \geq 0$.

  Recall that $s^x_j$ is the state of replica $j$ after the $x$th
  method execution, and therefore $s^x_j$ is also the state of $j$
  when $t^{x+1}_j(\mathcal{S})$ runs.
  %
  We have three cases to consider:
  \begin{itemize}
  \item $s^k_i \leq s^x_j$.

    (That is, replica $i$'s state after the $k$th method execution on $i$
    is \emph{at or below} replica $j$'s state after the $x$th method
    execution on $j$.)

    Choose $n = 0$.

    We have to show that, for all $i \geq n$, the threshold query
    $t^{x+1+i}_j(\mathcal{S})$ returns $S_a$.

    Since $t^{k+1}_i(\mathcal{S})$ returns $S_a$, we know that there
    exists an $s_a \in S_a$ such that $s_a \leq s^k_i$.

    Since $s^k_i \leq s^x_j$, we have by transitivity of $\leq$ that
    $s_a \leq s^x_j$.

    Therefore, by Definition~\ref{def:cvrdt-with-threshold-queries},
    $t^{x+1}_j(\mathcal{S})$ returns $S_a$.

    Then, by part~\ref{thm:this-replica} of the theorem, we have that
    subsequent executions $t^{x+1+i}_j(\mathcal{S})$ at replica $j$
    will also return $S_a$, and so the case holds.

    (Note that this case includes the possibility $s^k_i \equiv s^0$,
    in which no updates have executed at replica $i$.)

  \item $s^k_i > s^x_j$.

    (That is, replica $i$'s state after the $k$th method execution on $i$
    is \emph{above} replica $j$'s state after the $x$th method execution
    on $j$.)

    We have two subcases:

    \begin{itemize}
    \item
      There exists some activation state $s'_a \in S_a$ for which $s'_a \leq
      s^x_j$.

      In this case, we choose $n = 0$.

      We have to show that, for all $i \geq n$, the threshold query
      $t^{x+1+i}_j(\mathcal{S})$ returns $S_a$.

      Since $s'_a \leq s^x_j$, by
      Definition~\ref{def:cvrdt-with-threshold-queries},
      $t^{x+1}_j(\mathcal{S})$ returns $S_a$.

      Then, by part~\ref{thm:this-replica} of the theorem, we have
      that subsequent executions $t^{x+1+i}_j(\mathcal{S})$ at replica
      $j$ will also return $S_a$, and so the case holds.

    \item
      There is no activation state $s'_a \in S_a$ for which $s'_a \leq
      s^x_j$.

      Since $t^{k+1}_i(\mathcal{S})$ returns $S_a$, we know that there
      is some update $u^{k'}_i(a)$ in $i$'s causal history, for some
      $k' < (k+1)$, that updates $i$ from a state at or below $s^x_j$
      to $s^k_i$.\footnote{We know that $i$'s state was once at or
        below $s^x_j$, because $i$ and $j$ started at the same state
        $s^0$ and can both only grow.  Hence the least that $s^x_j$
        can be is $s^0$, and we know that $i$ was originally $s^0$ as
        well.}

      By eventual delivery, $u^{k'}_i(a)$ is eventually delivered at
      $j$.

      Hence some update or updates that will increase $j$'s state from
      $s^x_j$ to a state at or above some $s'_a$ must reach replica
      $j$.\footnote{We say ``some update or updates'' because the
        exact update $u^{k'}_i(a)$ may not be the update that causes
        the threshold query at $j$ to unblock; a different update or
        updates could do it.  Nevertheless, the existence of
        $u^{k'}_i(a)$ means that there is at least one update that
        will suffice to unblock the threshold query.}

      Let the $x+1+r$th method execution on $j$ be the first update on $j$
      that updates its state to some $s^{x+1+r}_j \geq s'_a$, for some
      activation state $s'_a \in S_a$.

      Choose $n = r+1$.

      We have to show that, for all $i$ in the range $0 \leq i \leq
      r$, the threshold query $t^{x+1+i}_j(\mathcal{S})$ returns
      $\block$, and that for all $i \geq r+1$, the threshold query
      $t^{x+1+i}_j(\mathcal{S})$ returns $S_a$.

      For the former, since the $x+1+r$th method execution on $j$ is the
      first one that updates its state to $s^{x+1+r}_j \geq s'_a$, we have
      by Definition~\ref{def:cvrdt-with-threshold-queries} that for all $i$
      in the range $0 \leq i \leq r$, the threshold query
      $t^{x+1+i}_j(\mathcal{S})$ returns $\block$.

      For the latter, since $s^{x+1+r}_j \geq s'_a$, by
      Definition~\ref{def:cvrdt-with-threshold-queries} we have that
      $t^{x+1+r+1}_j(\mathcal{S})$ returns $S_a$, and by
      part~\ref{thm:this-replica} of the theorem, we have that for $i \geq
      r+1$, subsequent executions $t^{x+1+i}_j(\mathcal{S})$ at replica $j$
      will also return $S_a$, and so the case holds.
    \end{itemize}

  \item $s^k_i \nleq s^x_j$ and $s^x_j \nleq s^k_i$.

    (That is, replica $i$'s state after the $k$th method execution on $i$
    is \emph{not comparable} to replica $j$'s state after the $x$th method
    execution on $j$.)

    Similar to the previous case.
  \end{itemize}
\end{proof}



\chapter{Proofs}\label{app:proofs}

\section{Proof of Lemma~\ref{lem:lvars-permutability}}\label{section:lvars-permutability-proof}
\begin{proof}
  Consider an arbitrary permutation $\pi$.  For
  part~\ref{thm:permutable-reduction-transitions}, we have to show
  that if $\conf \parstepsto \conf'$ then $\pi(\conf) \parstepsto
  \pi(\conf')$, and that if $\pi(\conf) \parstepsto \pi(\conf')$ then
  $\conf \parstepsto \conf'$.

  For the forward direction of
  part~\ref{thm:permutable-reduction-transitions}, suppose $\conf
  \parstepsto \conf'$.  We have to show that $\pi(\conf) \parstepsto
  \pi(\conf')$.  We proceed by cases on the rule by which $\conf$
  steps to $\conf'$.

  \begin{itemize}
    \item Case {\sc E-Beta}: $\conf =
      \config{S}{\app{(\lam{x}{e})}{v}}$, and $\conf' =
      \config{S}{\subst{e}{x}{v}}$.

      To show: $\pi(\config{S}{\app{(\lam{x}{e})}{v}}) \parstepsto
      \pi(\config{S}{\subst{e}{x}{v}})$.

      By Definitions~\ref{def:lvars-permutation-configuration}
      and~\ref{def:lvars-permutation-expression}, $\pi(\conf) =
      \config{\pi(S)}{\app{(\lam{x}{\pi(e)})}{\pi(v)}}$.

      By {\sc E-Beta},
      $\config{\pi(S)}{\app{(\lam{x}{\pi(e)})}{\pi(v)}}$ steps to
      $\config{\pi(S)}{\subst{\pi(e)}{x}{\pi(v)}}$.

      By Definition~\ref{def:lvars-permutation-expression},
      $\config{\pi(S)}{\subst{\pi(e)}{x}{\pi(v)}}$ is equal to
      $\config{\pi(S)}{\pi(\subst{e}{x}{v})}$.

      Hence $\config{\pi(S)}{\app{(\lam{x}{\pi(e)})}{\pi(v)}}$ steps
      to $\config{\pi(S)}{\pi(\subst{e}{x}{v})}$,

      which is equal to $\pi(\config{S}{\subst{e}{x}{v}})$ by
      Definition~\ref{def:lvars-permutation-configuration}.  Hence the
      case is satisfied.

    \item Case {\sc E-New}: $\conf = \config{S}{\NEW}$, and $\conf' =
      \config{\extSRaw{S}{l}{\bot}}{l}$.

      To show: $\pi(\config{S}{\NEW}) \parstepsto
      \pi(\config{\extSRaw{S}{l}{\bot}}{l})$.

      By Definitions~\ref{def:lvars-permutation-configuration}
      and~\ref{def:lvars-permutation-expression}, $\pi(\conf) =
      \config{\pi(S)}{\NEW}$.

      By {\sc E-New}, $\config{\pi(S)}{\NEW}$ steps to
      $\config{\extSRaw{(\pi(S))}{l'}{\bot}}{l'}$, where $l' \notin
      \dom{\pi(S)}$.
      
      It remains to show that
      $\config{\extSRaw{(\pi(S))}{l'}{\bot}}{l'}$ is equal to
      $\pi(\config{\extSRaw{S}{l}{\bot}}{l})$.

      By Definition~\ref{def:lvars-permutation-configuration},
      $\pi(\config{\extSRaw{S}{l}{\bot}}{l})$ is equal to
      $\config{\pi(\extSRaw{S}{l}{\bot})}{\pi(l)}$,

      which is equal to
      $\config{\extSRaw{(\pi(S))}{\pi(l)}{\bot}}{\pi(l)}$.

      So, we have to show that
      $\config{\extSRaw{(\pi(S))}{l'}{\bot}}{l'}$ is equal to
      $\config{\extSRaw{(\pi(S))}{\pi(l)}{\bot}}{\pi(l)}$.  Since we
      know (from the side condition of {\sc E-New}) that $l \notin
      \dom{S}$, it follows that $\pi(l) \notin \pi(\dom{S})$.
      Therefore, in $\config{\extSRaw{(\pi(S))}{l'}{\bot}}{l'}$, we
      can $\alpha$-rename $l'$ to $\pi(l)$, and so the two
      configurations are equal and the case is satisfied.

    \item Case {\sc E-Put}: $\conf = \config{S}{\putexp{l}{d_2}}$, and
      $\conf' = \config{\extSRaw{S}{l}{\userlub{d_1}{d_2}}}{\unit}$.

      To show: $\pi(\config{S}{\putexp{l}{d_2}}) \parstepsto
      \pi(\config{\extSRaw{S}{l}{\userlub{d_1}{d_2}}}{\unit})$.

      By Definitions~\ref{def:lvars-permutation-configuration}
      and~\ref{def:lvars-permutation-expression}, $\pi(\conf) =
      \config{\pi(S)}{\putexp{\pi(l)}{d_2}}$.

      By {\sc E-Put}, $\config{\pi(S)}{\putexp{\pi(l)}{d_2}}$ steps to
      $\config{\extSRaw{(\pi(S))}{\pi(l)}{\userlub{d_1}{d_2}}}{\unit}$,

      since $S(l) = (\pi(S))(\pi(l)) = d_1$.

      It remains to show that
      $\config{\extSRaw{(\pi(S))}{\pi(l)}{\userlub{d_1}{d_2}}}{\unit}$
      is equal to
      $\pi(\config{\extSRaw{S}{l}{\userlub{d_1}{d_2}}}{\unit})$.

      By Definitions~\ref{def:lvars-permutation-configuration}
      and~\ref{def:lvars-permutation-expression},
      $\pi(\config{\extSRaw{S}{l}{\userlub{d_1}{d_2}}}{\unit})$ is
      equal to
      $\config{\extSRaw{(\pi(S))}{\pi(l)}{\userlub{d_1}{d_2}}}{\unit}$,
      and so the two configurations are equal and the case is
      satisfied.

    \item Case {\sc E-Put-Err}: $\conf = \config{S}{\putexp{l}{d_2}}$,
      and $\conf' = \error$.

      To show: $\pi(\config{S}{\putexp{l}{d_2}}) \parstepsto
      \pi(\error)$.

      By Definitions~\ref{def:lvars-permutation-configuration}
      and~\ref{def:lvars-permutation-expression}, $\pi(\conf) =
      \config{\pi(S)}{\putexp{\pi(l)}{d_2}}$.

      By {\sc E-Put-Err}, $\config{\pi(S)}{\putexp{\pi(l)}{d_2}}$
      steps to $\error$,

      since $S(l) = (\pi(S))(\pi(l)) = d_1$.

      Since $\pi(\error) = \error$ by
      Definition~\ref{def:lvars-permutation-configuration}, the case
      is complete.

    \item Case {\sc E-Get}: $\conf = \config{S}{\getexp{l}{T}}$, and
      $\conf' = \config{S}{d_2}$.

      To show: $\pi(\config{S}{\getexp{l}{T}}) \parstepsto
      \pi(\config{S}{d_2})$.

      By Definitions~\ref{def:lvars-permutation-configuration}
      and~\ref{def:lvars-permutation-expression}, $\pi(\conf) =
      \config{\pi(S)}{\getexp{\pi(l)}{T}}$.

      By {\sc E-Get}, $\config{\pi(S)}{\getexp{\pi(l)}{T}}$ steps to
      $\config{\pi(S)}{d_2}$,

      since $S(l) = (\pi(S))(\pi(l)) = d_1$.

      By Definitions~\ref{def:lvars-permutation-configuration}
      and~\ref{def:lvars-permutation-expression},
      $\pi(\config{S}{d_2}) \config{\pi(S)}{d_2}$.  Therefore the case
      is complete.
  \end{itemize}

  For the reverse direction of
  part~\ref{thm:permutable-reduction-transitions}, suppose $\pi(\conf)
  \parstepsto \pi(\conf')$.  We have to show that $\conf \parstepsto
  \conf'$.

  We know from the forward direction of the proof that for all
  configurations $\conf$ and $\conf'$ and permutations $\pi$, if
  $\conf \parstepsto \conf'$ then $\pi(\conf) \parstepsto
  \pi(\conf')$.  Hence since $\pi(\conf) \parstepsto \pi(\conf')$, and
  since $\piinv$ is also a permutation, we have that
  $\piinv(\pi(\conf)) \parstepsto \piinv(\pi(\conf'))$.  Since
  $\piinv(\pi(l)) = l$ for every $l \in \Loc$, and that property lifts
  to configurations as well, we have that $\conf \parstepsto \conf'$.

  \lk{Is the above enough of a proof?}

  For the forward direction of
  part~\ref{thm:permutable-context-transitions}, suppose $\conf
  \ctxstepsto \conf'$.  We have to show that $\pi(\conf) \ctxstepsto
  \pi(\conf')$.

  By inspection of the operational semantics, $\conf$ must be of the
  form $\config{S}{\E{e}}$, and $\conf'$ must be of the form
  $\config{S'}{\E{e'}}$.  Hence we have to show that
  $\pi(\config{S}{\E{e}}) \ctxstepsto \pi(\config{S'}{\E{e'}})$.

  By Definition~\ref{def:lvars-permutation-configuration},
  $\pi(\config{S}{\E{e}})$ is equal to $\config{\pi(S)}{\pi(\E{e})}$.

  Also by Definition~\ref{def:lvars-permutation-configuration},
  $\pi(\config{S'}{\E{e'}})$ is equal to
  $\config{\pi(S')}{\pi(\E{e'})}$.

  Furthermore, $\config{\pi(S)}{\pi(\E{e})}$ is equal to
  $\config{\pi(S)}{\evalctxt{(\pi(E))}{\pi(e)}}$ and
  $\config{\pi(S')}{\pi(\E{e'})}$ is equal to
  $\config{\pi(S')}{\evalctxt{(\pi(E))}{\pi(e')}}$.

  So we have to show that
  $\config{\pi(S)}{\evalctxt{(\pi(E))}{\pi(e)}} \ctxstepsto
  \config{\pi(S')}{\evalctxt{(\pi(E))}{\pi(e')}}$.

  From the premise of {\sc E-Eval-Ctxt}, $\config{S}{e} \parstepsto
  \config{S'}{e'}$.  Hence, by
  part~\ref{thm:permutable-reduction-transitions}, $\pi(\config{S}{e})
  \parstepsto \pi(\config{S'}{e'})$.  By
  Definition~\ref{def:lvars-permutation-configuration},
  $\pi(\config{S}{e})$ is equal to $\config{\pi(S)}{\pi(e)}$ and
  $\pi(\config{S'}{e'})$ is equal to $\config{\pi(S')}{\pi(e')}$.

  Hence $\config{\pi(S)}{\pi(e)} \parstepsto
  \config{\pi(S')}{\pi(e')}$.

  Therefore, by {\sc E-Eval-Ctxt}, $\config{\pi(S)}{\E{\pi(e)}}
  \ctxstepsto \config{\pi(S')}{\E{\pi(e')}}$ for all evaluation
  contexts $E$.

  In particular, it is true that
  $\config{\pi(S)}{\evalctxt{(\pi(E))}{\pi(e)}} \ctxstepsto
  \config{\pi(S')}{\evalctxt{(\pi(E))}{\pi(e')}}$, as we were required
  to show.

  For the reverse direction of
  part~\ref{thm:permutable-context-transitions}, suppose $\pi(\conf)
  \ctxstepsto \pi(\conf')$.  We have to show that $\conf \ctxstepsto
  \conf'$.

  We know from the forward direction of the proof that for all
  configurations $\conf$ and $\conf'$ and permutations $\pi$, if
  $\conf \ctxstepsto \conf'$ then $\pi(\conf) \ctxstepsto
  \pi(\conf')$.  Hence since $\pi(\conf) \ctxstepsto \pi(\conf')$, and
  since $\piinv$ is also a permutation, we have that
  $\piinv(\pi(\conf)) \ctxstepsto \piinv(\pi(\conf'))$.  Since
  $\piinv(\pi(l)) = l$ for every $l \in \Loc$, and that property lifts
  to configurations as well, we have that $\conf \ctxstepsto \conf'$.

  \lk{Is the above enough of a proof?}
\end{proof}


\section{Proof of Lemma~\ref{lem:lvars-internal-determinism}}\label{section:lvars-internal-determinism-proof}
\begin{proof}
  Suppose $\conf \parstepsto \conf'$ and $\conf \parstepsto \conf''$.

  We have to show that there is a permutation $\pi$ such that $\conf'
  = \pi(\conf'')$.

  The proof is by cases on the rule by which $\conf$ steps to
  $\conf'$.

  \begin{itemize}

  \item Case {\sc E-Beta}:

    Given: $\config{S}{\app{(\lam{x}{e})}{v}} \parstepsto
    \config{S}{\subst{e}{x}{v}}$, and
    $\config{S}{\app{(\lam{x}{e})}{v}} \parstepsto \conf''$.

    To show: There exists a $\pi$ such that
    $\config{S}{\subst{e}{x}{v}} = \pi(\conf'')$.

    By inspection of the operational semantics, the only reduction
    rule by which $\config{S}{\app{(\lam{x}{e})}{v}}$ can step is {\sc
      E-Beta}.

    Hence $\conf'' = \config{S}{\subst{e}{x}{v}}$, and the case is
    satisfied by choosing $\pi$ to be the identity function.

  \item Case {\sc E-New}: 

    Given: $\config{S}{\NEW} \parstepsto
    \config{\extSRaw{S}{l}{\bot}}{l}$, and $\config{S}{\NEW}
    \parstepsto \conf''$.

    To show: There exists a $\pi$ such that
    $\config{\extSRaw{S}{l}{\bot}}{l} = \pi(\conf'')$.

    By inspection of the operational semantics, the only reduction
    rule by which $\config{S}{\NEW}$ can step is {\sc E-New}.

    Hence $\conf'' = \config{\extSRaw{S}{l'}{\bot}}{l'}$.

    Since, by the side condition of {\sc E-New}, neither $l$ nor $l'$
    occur in $\dom{S}$, the case is satisfied by choosing $\pi$ to be
    the permutation that maps $l'$ to $l$ and is the identity on every
    other element of $\Loc$.

  \item Case {\sc E-Put}:

    Given: $\config{S}{\putexp{l}{d_2}} \parstepsto
    \config{\extSRaw{S}{l}{\userlub{d_1}{d_2}}}{\unit}$, and
    $\config{S}{\putexp{l}{d_2}} \parstepsto \conf''$.

    To show: There exists a $\pi$ such that
    $\config{\extSRaw{S}{l}{\userlub{d_1}{d_2}}}{\unit} =
    \pi(\conf'')$.

    By inspection of the operational semantics, and since
    $\userlub{d_1}{d_2} \neq \top$ (from the premise of {\sc E-Put}),
    the only reduction rule by which $\config{S}{\putexp{l}{d_2}}$ can
    step is {\sc E-Put}.

    Hence $\conf'' =
    \config{\extSRaw{S}{l}{\userlub{d_1}{d_2}}}{\unit}$, and the case
    is satisfied by choosing $\pi$ to be the identity function.

  \item Case {\sc E-Put-Err}:

    Given: $\config{S}{\putexp{l}{d_2}} \parstepsto \error$, and
    $\config{S}{\putexp{l}{d_2}} \parstepsto \conf''$.

    To show: There exists a $\pi$ such that $\error = \pi(\conf'')$.

    By inspection of the operational semantics, and since
    $\userlub{d_1}{d_2} = \top$ (from the premise of {\sc E-Put-Err}),
    the only reduction rule by which $\config{S}{\putexp{l}{d_2}}$ can
    step is {\sc E-Put-Err}.

    Hence $\conf'' = \error$, and the case is satisfied by choosing
    $\pi$ to be the identity function.

  \item Case {\sc E-Get}:

    Given: $\config{S}{\getexp{l}{T}} \parstepsto \config{S}{d_2}$,
    and $\config{S}{\getexp{l}{T}} \parstepsto \conf''$.

    To show: There exists a $\pi$ such that $\config{S}{d_2} =
    \pi(\conf'')$.

    By inspection of the operational semantics, the only reduction
    rule by which $\config{S}{\getexp{l}{T}}$ can step is {\sc
      E-Get}.

    Hence $\conf'' = \config{S}{d_2}$, and the case is satisfied by
    choosing $\pi$ to be the identity function.

  \end{itemize}
\end{proof}



\section{Proof of Lemma~\ref{lem:lvars-monotonicity}}\label{section:lvars-monotonicity-proof}
\begin{proof}
  Suppose $\config{S}{e} \parstepsto \config{S'}{e'}$.  We are
  required to show that $\leqstore{S}{S'}$.  The proof is by cases on
  the rule by which $\config{S}{e}$ steps to $\config{S'}{e'}$.

  \begin{itemize}
    \item Case {\sc E-Beta}:

      Immediate by the definition of $\leqstore{}{}$, since $S$ does
      not change.

    \item Case {\sc E-New}:

      Given: $\config{S}{\NEW} \parstepsto
      \config{\extSRaw{S}{l}{\bot}}{l}$.

      To show: $\leqstore{S}{\extSRaw{S}{l}{\bot}}$.

      By Definition~\ref{def:lvars-leqstore}, we have to show that
      $\dom{S} \subseteq \dom{\extSRaw{S}{l}{\bot}}$ and
      that for all $l' \in \dom{S}$, $S(l') \userleq
      (\extSRaw{S}{l}{\bot})(l')$.

      By definition, a store update operation on $S$ can only either
      update an existing binding in $S$ or extend $S$ with a new
      binding.  Hence $\dom{S} \subseteq \dom{\extSRaw{S}{l}{\bot}}$.

      From the side condition of {\sc E-New}, $l \notin \dom{S}$.
      Hence $\extSRaw{S}{l}{\bot}$ adds a new binding for $l$ in $S$.

      Hence $\extSRaw{S}{l}{\bot}$ does not update any existing
      bindings in $S$.

      Hence, for all $l' \in \dom{S}, S(l') \userleq
      (\extSRaw{S}{l}{\bot})(l')$.

      Therefore $\leqstore{S}{\extSRaw{S}{l}{\bot}}$, as
      required.

    \item Case {\sc E-Put}:

      Given: $\config{S}{\putexp{l}{d_2}} \parstepsto
      \config{\extSRaw{S}{l}{\userlub{d_1}{d_2}}}{\unit}$.

      To show: $\leqstore{S}{\extSRaw{S}{l}{\userlub{d_1}{d_2}}}$.

      By Definition~\ref{def:lvars-leqstore}, we have to show that
      $\dom{S} \subseteq \dom{\extSRaw{S}{l}{\userlub{d_1}{d_2}}}$ and
      that for all $l' \in \dom{S}$, $S(l') \userleq
      (\extSRaw{S}{l}{\userlub{d_1}{d_2}})(l')$.

      By definition, a store update operation on $S$ can only either
      update an existing binding in $S$ or extend $S$ with a new
      binding.  Hence $\dom{S} \subseteq
      \dom{\extSRaw{S}{l}{\userlub{d_1}{d_2}}}$.

      From the premises of {\sc E-Put}, $S(l) = d_1$.  Therefore $l
      \in \dom{S}$.

      Hence $\extSRaw{S}{l}{\userlub{d_1}{d_2}}$ updates the existing
      binding for $l$ in $S$ from $d_1$ to $\userlub{d_1}{d_2}$.

      By the definition of $\userlub{}{}$, $d_1 \userleq
      (\userlub{d_1}{d_2})$.  $\extSRaw{S}{l}{\userlub{d_1}{d_2}}$
      does not update any other bindings in $S$, hence, for all $l'
      \in \dom{S}, S(l') \userleq
      (\extSRaw{S}{l}{\userlub{d_1}{d_2}})(l')$.

      Hence $\leqstore{S}{\extSRaw{S}{l}{\userlub{d_1}{d_2}}}$, as
      required.

    \item Case {\sc E-Put-Err}:

      Given: $\config{S}{\putexp{l}{d_2}} \parstepsto \error$.

      By the definition of $\error$, $\error$ is equal to
      $\config{\topS}{e}$ for all $e$.

      To show: $\leqstore{S}{\topS}$.

      Immediate by the definition of $\leqstore{}{}$.

    \item Case {\sc E-Get}:

      Immediate by the definition of $\leqstore{}{}$, since $S$ does
      not change.

  \end{itemize}

\end{proof}


\section{Proof of Lemma~\ref{lem:lvars-independence}}\label{section:lvars-independence-proof}
\begin{proof}
  Consider arbitrary $S''$ such that $S''$ is non-conflicting with
  $\config{S}{e} \parstepsto \config{S'}{e'}$ and $\lubstore{S'}{S''}
  \neq \topS$.

  To show: $\config{\lubstore{S}{S''}}{e} \parstepsto
  \config{\lubstore{S'}{S''}}{e'}$.

  The proof is by induction on the derivation of $\config{S}{e}
  \parstepsto \config{S'}{e'}$, by cases on the last rule in the
  derivation.  In every case we may assume that $\config{S'}{e'} \neq
  \error$.  Since $\config{S'}{e'} \neq \error$, we do not need to
  consider the {\sc E-Put-Err} rule.
  \begin{itemize}

    \item Case {\sc E-Eval-Ctxt}:

      Given: $\config{S}{\E{e}} \parstepsto \config{S'}{\E{e'}}$.

      To show: $\config{\lubstore{S}{S''}}{\E{e}} \parstepsto
      \config{\lubstore{S'}{S''}}{\E{e'}}$.

      From the premise of {\sc E-Eval-Ctxt}, we have that
      $\config{S}{e} \parstepsto \config{S'}{e'}$.

      Therefore, by IH, we have that $\config{\lubstore{S}{S''}}{e}
      \parstepsto \config{\lubstore{S'}{S''}}{e'}$.

      Therefore, by {\sc E-Eval-Ctxt}, we have that
      $\config{\lubstore{S}{S''}}{\E{e}} \parstepsto
      \config{\lubstore{S'}{S''}}{\E{e'}}$, as we were required to
      show.

    \item Case {\sc E-Beta}:

      Given: $\config{S}{\app{(\lam{x}{e})}{v}} \parstepsto
      \config{S}{\subst{e}{x}{v}}$.

      To show: $\config{\lubstore{S}{S''}}{\app{(\lam{x}{e})}{v}}
      \parstepsto \config{\lubstore{S}{S''}}{\subst{e}{x}{v}}$.

      Immediate by {\sc E-Beta}.

    \item Case {\sc E-New}:

      Given: $\config{S}{\NEW} \parstepsto
      \config{\extSRaw{S}{l}{\bot}}{l}$.

      To show: $\config{\lubstore{S}{S''}}{\NEW} \parstepsto
      \config{\lubstore{(\extSRaw{S}{l}{\bot})}{S''}}{l}$.

      By {\sc E-New}, we have that $\config{\lubstore{S}{S''}}{\NEW}
      \parstepsto \config{\extSRaw{(\lubstore{S}{S''})}{l'}{\bot}}{l'}$,
      where $l' \notin \dom{\lubstore{S}{S''}}$.

      By assumption, $S''$ is non-conflicting with $\config{S}{\NEW}
      \parstepsto \config{\extSRaw{S}{l}{\bot}}{l}$.
 
      Therefore $l \notin \dom{S''}$.

      From the side condition of {\sc E-New}, $l \notin \dom{S}$.

      Therefore $l \notin \dom{\lubstore{S}{S''}}$.

      Therefore, in
      $\config{\extSRaw{(\lubstore{S}{S''})}{l'}{\bot}}{l'}$, we can
      $\alpha$-rename $l'$ to $l$, \\ resulting in
      $\config{\extSRaw{(\lubstore{S}{S''})}{l}{\bot}}{l}$.

      Therefore $\config{\lubstore{S}{S''}}{\NEW} \parstepsto
      \config{\extSRaw{(\lubstore{S}{S''})}{l}{\bot}}{l}$.

      Note that:
      \begin{align*}
        \extSRaw{(\lubstore{S}{S''})}{l}{\bot} &=
        \lubstore{\extSRaw{S}{l}{\bot}}{\extSRaw{S''}{l}{\bot}} \\ &=
        \lubstore{\lubstore{S}{\store{\storebindingRaw{l}{\bot}}}}{\lubstore{S''}{\store{\storebindingRaw{l}{\bot}}}}
        \\ &=
        \lubstore{\lubstore{S}{\store{\storebindingRaw{l}{\bot}}}}{S''}
        \\ &= \lubstore{\extSRaw{S}{l}{\bot}}{S''}.
      \end{align*}
      Therefore $\config{\lubstore{S}{S''}}{\NEW} \parstepsto
      \config{\lubstore{\extSRaw{S}{l}{\bot}}{S''}}{l}$, as we were
      required to show.

    \item Case {\sc E-Put}:

      Given: $\config{S}{\putexp{l}{d_2}} \parstepsto
      \config{\extSRaw{S}{l}{d_2}}{\unit}$.

      To show: $\config{\lubstore{S}{S''}}{\putexp{l}{d_2}}
      \parstepsto
      \config{\lubstore{\extSRaw{S}{l}{d_2}}{S''}}{\unit}$.

      We will first show that

      $\config{\lubstore{S}{S''}}{\putexp{l}{d_2}} \parstepsto
      \config{\extSRaw{(\lubstore{S}{S''})}{l}{d_2}}{\unit}$

      and then show why this is sufficient.

      We proceed by cases on $l$:

      \begin{itemize}
        \item $l \notin \dom{S''}$:

          By assumption, $\lubstore{\extSRaw{S}{l}{d_2}}{S''} \neq
          \topS$.

          By Lemma~\ref{lem:lvars-monotonicity},
          $\leqstore{S}{\extSRaw{S}{l}{d_2}}$.

          Hence $\lubstore{S}{S''} \neq \topS$.

          Therefore, by Definition~\ref{def:lvars-lubstore},
          $(\lubstore{S}{S''})(l) = S(l)$.

          From the premises of {\sc E-Put}, $S(l) = d_1$.

          Hence $(\lubstore{S}{S''})(l) = d_1$.

          From the premises of {\sc E-Put}, $d_2 = \userlub{d_1}{d_2}$
          and $d_2 \neq \top$.

          Therefore, by {\sc E-Put}, we have:
          $\config{\lubstore{S}{S''}}{\putexp{l}{d_2}} \parstepsto
          \config{\extSRaw{(\lubstore{S}{S''})}{l}{d_2}}{\unit}$.

        \item $l \in \dom{S''}$:

          By assumption, $\lubstore{\extSRaw{S}{l}{d_2}}{S''} \neq
          \topS$.

          By Lemma~\ref{lem:lvars-monotonicity},
          $\leqstore{S}{\extSRaw{S}{l}{d_2}}$.

          Hence $\lubstore{S}{S''} \neq \topS$.

          Therefore $(\lubstore{S}{S''})(l) = \userlub{S(l)}{S''(l)}$.

          From the premises of {\sc E-Put}, $S(l) = d_1$.
          
          Hence $(\lubstore{S}{S''})(l) = d'_1$, where $d_1 \userleq
          d'_1$.

          From the premises of {\sc E-Put}, $d_2 =
          \userlub{d_1}{d_2}$.

          Let $d'_2 = \userlub{d'_1}{d_2}$.

          Hence $d_2 \userleq d'_2$.

          By assumption, $\lubstore{\extSRaw{S}{l}{d_2}}{S''} \neq
          \topS$.

          Therefore, by Definition~\ref{def:lvars-lubstore},
          $\lubstore{d_2}{S''(l)} \neq \top$.

          Note that:
          \begin{align*}
            \top &\neq \lubstore{d_2}{S''(l)} \\ &=
            \userlub{\userlub{d_1}{d_2}}{S''(l)} \\ &=
            \userlub{\userlub{S(l)}{d_2}}{S''(l)} \\ &=
            \userlub{\userlub{S(l)}{S''(l)}}{d_2} \\ &=
            \userlub{(\lubstore{S}{S''})(l)}{d_2} \\ &=
            \userlub{d'_1}{d_2} \\ &= d'_2. \\
          \end{align*}
          Hence $d'_2 \neq \top$.

          Hence $(\lubstore{S}{S''})(l) = d'_1$ and $d'_2 =
          \userlub{d'_1}{d_2}$ and $d'_2 \neq \top$.

          Therefore, by {\sc E-Put} we have:
          $\config{\lubstore{S}{S''}}{\putexp{l}{d_2}} \parstepsto
          \config{\extSRaw{(\lubstore{S}{S''})}{l}{d'_2}}{\unit}$.

          \lk{If we really wanted to be pedantic here, we'd actually
            prove that the stores are equal.  I'm assuming that if I
            can show that $\extSRaw{(\lubstore{S}{S''})}{l}{d'_2}$ and
            $\extSRaw{(\lubstore{S}{S''})}{l}{d_2}$ bind $l$ to the
            same value, then it will be obvious that they're equal.}

          Note that:
          \begin{align*}
            (\extSRaw{(\lubstore{S}{S''})}{l}{d'_2})(l) &=
            \userlub{(\lubstore{S}{S''})(l)}{(\store{\storebindingRaw{l}{d'_2}})(l)}
            \\ &= \userlub{d'_1}{d'_2} \\ &=
            \userlub{d'_1}{\userlub{d'_1}{d_2}} \\ &=
            \userlub{d'_1}{d_2}
          \end{align*}
          and
          \begin{align*}
            (\extSRaw{(\lubstore{S}{S''})}{l}{d_2})(l) &=
            \userlub{(\lubstore{S}{S''})(l)}{(\store{\storebindingRaw{l}{d_2}})(l)}
            \\ &= \userlub{d'_1}{d_2} \\ &=
            \userlub{d'_1}{\userlub{d_1}{d_2}} \\ &=
            \userlub{d'_1}{d_2} & \textrm{(since $d_1 \userleq
              d'_1$).}
          \end{align*}
          Therefore $\extSRaw{(\lubstore{S}{S''})}{l}{d'_2} =
          \extSRaw{(\lubstore{S}{S''})}{l}{d_2}$.

          Therefore, $\config{\lubstore{S}{S''}}{\putexp{l}{d_2}}
          \parstepsto
          \config{\extSRaw{(\lubstore{S}{S''})}{l}{d_2}}{\unit}$.
      \end{itemize}

      Note that:
      \begin{align*}
        \extSRaw{(\lubstore{S}{S''})}{l}{d_2} &=
        \lubstore{\extSRaw{S}{l}{d_2}}{\extSRaw{S''}{l}{d_2}} \\ &=
        \lubstore{\lubstore{S}{\store{\storebindingRaw{l}{d_2}}}}{\lubstore{S''}{\store{\storebindingRaw{l}{d_2}}}}
        \\ &=
        \lubstore{\lubstore{S}{\store{\storebindingRaw{l}{d_2}}}}{S''}
        \\ &= \lubstore{\extSRaw{S}{l}{d_2}}{S''}.
      \end{align*}
      Therefore $\config{\lubstore{S}{S''}}{\putexp{l}{d_2}}
      \parstepsto
      \config{\lubstore{\extSRaw{S}{l}{d_2}}{S''}}{\unit}$, as we were
      required to show.

    \item Case {\sc E-Get}:

      Given: $\config{S}{\getexp{l}{T}} \parstepsto \config{S}{d_2}$.

      To show: $\config{\lubstore{S}{S''}}{\getexp{l}{T}} \parstepsto
      \config{\lubstore{S}{S''}}{d_2}$.

      From the premises of {\sc E-Get}, $S(l) = d_1$ and $\incomp{T}$
      and $d_2 \in T$ and $d_2 \userleq d_1$.

      By assumption, $\lubstore{S}{S''} \neq \topS$.

      Hence $(\lubstore{S}{S''}) = d'_1$, where $d_1 \userleq d'_1$.

      By the transitivity of $\userleq$, $d_2 \userleq d'_1$.

      Hence, $S(l) = d'_1$ and $\incomp{T}$ and $d_2 \in T$ and $d_2
      \userleq d'_1$.

      Therefore, by {\sc E-Get},

      $\config{\lubstore{S}{S''}}{\getexp{l}{T}} \parstepsto
      \config{\lubstore{S}{S''}}{d_2}$,

      as we were required to show.
  \end{itemize}
\end{proof}


\section{Proof of Lemma~\ref{lem:lvars-clash}}\label{section:lvars-clash-proof}
\begin{proof}
  Consider arbitrary $S''$ such that $S''$ is non-conflicting with
  $\config{S}{e} \parstepsto \config{S'}{e'}$ and $\lubstore{S'}{S''}
  = \topS$.

  To show: $\config{\lubstore{S}{S''}}{e} \parstepsto \error$.

  The proof is by induction on the derivation of $\config{S}{e}
  \parstepsto \config{S'}{e'}$, by cases on the last rule in the
  derivation.  In every case we may assume that $\config{S'}{e'} \neq
  \error$.  Since $\config{S'}{e'} \neq \error$, we do not need to
  consider the {\sc E-Put-Err} rule.

  \begin{itemize}

    \item Case {\sc E-Eval-Ctxt}:

      Given: $\config{S}{\E{e}} \parstepsto \config{S'}{\E{e'}}$.

      To show: $\config{\lubstore{S}{S''}}{\E{e}} \parstepsto^i
      \error$, where $i \leq 1$.

      From the premise of {\sc E-Eval-Ctxt}, we have that
      $\config{S}{e} \parstepsto \config{S'}{e'}$.

      Therefore, by IH, we have that $\config{\lubstore{S}{S''}}{e}
      \parstepsto^{i'} \error$, where $i' \leq 1$.

      We proceed by cases on $i'$:

      \begin{itemize}
        \item $i' = 0$:

          In this case, $\config{\lubstore{S}{S''}}{e} = \error$.

          Hence, by the definition of $\error$, $\lubstore{S}{S''} =
          \topS$.

          Hence $\config{\lubstore{S}{S''}}{\E{e}} = \error$.

          Hence $\config{\lubstore{S}{S''}}{\E{e}} \parstepsto^i
          \error$, with $i = 0$.

        \item $i' = 1$:

          In this case, $\config{\lubstore{S}{S''}}{e} \parstepsto
          \error$.

          By the definition of $\error$, $\error =
          \config{\topS}{e''}$ for any $e''$.

          Hence $\config{\lubstore{S}{S''}}{e} \parstepsto
          \config{\topS}{e''}$.

          Hence, by {\sc E-Eval-Ctxt},
          $\config{\lubstore{S}{S''}}{\E{e}} \parstepsto
          \config{\topS}{\E{e''}}$.

          By the definition of $\error$, $\config{\topS}{\E{e''}} =
          \error$.

          Hence $\config{\lubstore{S}{S''}}{\E{e}} \parstepsto
          \error$.

          Hence $\config{\lubstore{S}{S''}}{\E{e}} \parstepsto^i
          \error$, with $i = 1$.

      \end{itemize}

    \item Case {\sc E-Beta}:

      Given: $\config{S}{\app{(\lam{x}{e})}{v}} \parstepsto
      \config{S}{\subst{e}{x}{v}}$.

      To show: $\config{\lubstore{S}{S''}}{\app{(\lam{x}{e})}{v}}
      \parstepsto^i \error$, where $i \leq 1$.

      By assumption, $\lubstore{S}{S''} = \topS$.

      Hence, by the definition of $\error$,
      $\config{\lubstore{S}{S''}}{\app{(\lam{x}{e})}{v}} = \error$.

      Hence $\config{\lubstore{S}{S''}}{\app{(\lam{x}{e})}{v}}
      \parstepsto^i \error$, with $i = 0$.

    \item Case {\sc E-New}:

      Given: $\config{S}{\NEW} \parstepsto
      \config{\extSRaw{S}{l}{\bot}}{l}$.

      To show: $\config{\lubstore{S}{S''}}{\NEW} \parstepsto^i
      \error$, where $i \leq 1$.

      By {\sc E-New}, $\config{\lubstore{S}{S''}}{\NEW} \parstepsto
      \config{\extSRaw{(\lubstore{S}{S''})}{l'}{\bot}}{l'}$, where $l'
      \notin \dom{\lubstore{S}{S''}}$.

      By assumption, $S''$ is non-conflicting with $\config{S}{\NEW}
      \parstepsto \config{\extSRaw{S}{l}{\bot}}{l}$.
 
      Therefore $l \notin \dom{S''}$.

      From the side condition of {\sc E-New}, $l \notin \dom{S}$.

      Therefore $l \notin \dom{\lubstore{S}{S''}}$.

      Therefore, in
      $\config{\extSRaw{(\lubstore{S}{S''})}{l'}{\bot}}{l'}$, we can
      $\alpha$-rename $l'$ to $l$, \\ resulting in
      $\config{\extSRaw{(\lubstore{S}{S''})}{l}{\bot}}{l}$.

      Therefore $\config{\lubstore{S}{S''}}{\NEW} \parstepsto
      \config{\extSRaw{(\lubstore{S}{S''})}{l}{\bot}}{l}$.

      By assumption, $\lubstore{\extSRaw{S}{l}{\bot}}{S''}
      = \topS$.

      Note that:
      \begin{align*}
        \topS &= \lubstore{\extSRaw{S}{l}{\bot}}{S''} \\ &=
        \lubstore{\lubstore{S}{\store{\storebindingRaw{l}{\bot}}}}{S''}
        \\ &=
        \lubstore{\lubstore{S}{S''}}{\store{\storebindingRaw{l}{\bot}}}
        \\ &=
        \lubstore{(\lubstore{S}{S''})}{\store{\storebindingRaw{l}{\bot}}}
        \\ &= \extSRaw{(\lubstore{S}{S''})}{l}{\bot} .
      \end{align*}

      Hence $\config{\lubstore{S}{S''}}{\NEW} \parstepsto
      \config{\topS}{l}$.

      Hence, by the definition of $\error$,
      $\config{\lubstore{S}{S''}}{\NEW} \parstepsto \error$.

      Hence $\config{\lubstore{S}{S''}}{\NEW} \parstepsto^i \error$,
      with $i = 1$.

    \item Case {\sc E-Put}:

      Given: $\config{S}{\putexp{l}{d_2}} \parstepsto
      \config{\extSRaw{S}{l}{d_2}}{\unit}$.

      To show: $\config{\lubstore{S}{S''}}{\putexp{l}{d_2}}
      \parstepsto^i \error$, where $i \leq 1$.

      We proceed by cases on $\lubstore{S}{S''}$:

      \begin{itemize}

        \item $\lubstore{S}{S''} = \topS$:

          In this case, by the definition of $\error$,
          $\config{\lubstore{S}{S''}}{\putexp{l}{d_2}} = \error$.

          Hence $\config{\lubstore{S}{S''}}{\putexp{l}{d_2}}
          \parstepsto^i \error$, with $i = 0$.

        \item $\lubstore{S}{S''} \neq \topS$:

          From the premises of {\sc E-Put}, we have that $S(l) = d_1$.

          Hence $(\lubstore{S}{S''})(l) = d'_1$, where $d_1 \userleq
          d'_1$.

          We show that $\userlub{d'_1}{d_2} =
          \top$, as follows:

          By assumption, $\lubstore{\extSRaw{S}{l}{d_2}}{S''} = \topS$.

          Hence, by Definition~\ref{def:lvars-lubstore}, there exists
          some $l' \in \dom{\extSRaw{S}{l}{d_2}} \cap \dom{S''}$ such
          that $\userlub{(\extSRaw{S}{l}{d_2})(l')}{S''(l')} = \top$.

          Now case on $l'$:

          \begin{itemize}
            \item $l' \neq l$:

              In this case, $(\extSRaw{S}{l}{d_2})(l') = S(l')$.

              Since $\userlub{(\extSRaw{S}{l}{d_2})(l')}{S''(l')} = \top$,
              we then have that $\userlub{S(l')}{S''(l')} = \top$.

              However, this is a contradiction since
              $\lubstore{S}{S''} \neq \topS$.

              Hence this case cannot occur.

            \item $l' = l$:

              Then $\userlub{(\extSRaw{S}{l}{d_2})(l)}{S''(l)} = \top$.

              Note that:
              \begin{align*}
                \top &= \userlub{(\extSRaw{S}{l}{d_2})(l)}{S''(l)} \\ &=
                \userlub{d_2}{S''(l)} \\ &=
                \userlub{\userlub{d_1}{d_2}}{S''(l)}
                \\ &=
                \userlub{\userlub{S(l)}{d_2}}{S''(l)}
                \\ &=
                \userlub{\userlub{S(l)}{S''(l)}}{d_2}
                \\ &=
                \userlub{(\lubstore{S}{S''})(l)}{d_2}
                \\ &= \userlub{d'_1}{d_2}.
              \end{align*}
              Hence $\userlub{d'_1}{d_2} = \top$.

              Hence, by {\sc E-Put-Err},
              $\config{\lubstore{S}{S''}}{\putexp{l}{d_2}} \parstepsto
              \error$.

              Hence $\config{\lubstore{S}{S''}}{\putexp{l}{d_2}}
              \parstepsto^i \error$, with $i = 1$.

          \end{itemize}

      \end{itemize}

    \item Case {\sc E-Get}:

      Given: $\config{S}{\getexp{l}{T}} \parstepsto \config{S}{d_2}$.

      To show: $\config{\lubstore{S}{S''}}{\getexp{l}{T}}
      \parstepsto^i \error$, where $i \leq 1$.

      By assumption, $\lubstore{S}{S''} = \topS$.

      Hence, by the definition of $\error$,
      $\config{\lubstore{S}{S''}}{\getexp{l}{T}} = \error$.

      Hence $\config{\lubstore{S}{S''}}{\getexp{l}{T}} \parstepsto^i
      \error$, with $i = 0$.
  \end{itemize}
\end{proof}


\section{Proof of Lemma~\ref{lem:lvars-error-preservation}}\label{section:lvars-error-preservation-proof}
\begin{proof}

  Given: $\config{S}{e} \parstepsto \error$ and $\leqstore{S}{S'}$.

  To show: $\config{S'}{e} \parstepsto \error$.

  \TODO{Figure out what to do here.  I think we need to handle both
    E-Eval-Ctxt and E-Put-Err.}
\end{proof}


\section{Proof of Lemma~\ref{lem:lvars-strong-local-confluence}}\label{section:lvars-strong-local-confluence-proof}
\begin{proof}
  Suppose $\conf \ctxstepsto \conf_a$ and $\conf \ctxstepsto \conf_b$.
  We have to show that there exist $\conf_c, i, j, \pi$ such that
  $\conf_a \ctxstepsto^i \conf_c$ and $\pi(\conf_b) \ctxstepsto^j
  \pi(\conf_c)$ and $i \leq 1$ and $j \leq 1$.

  By inspection of the operational semantics, it must be the case that
  $\conf$ steps to $\conf_a$ by the {\sc E-Eval-Ctxt} rule.  Let
  $\conf = \config{S}{\evalctxt{E_a}{e_{a_1}}}$ and let $\conf_a =
  \config{S_a}{\evalctxt{E_a}{e_{a_2}}}$.

  Likewise, it must be the case that $\conf$ steps to $\conf_b$ by the
  {\sc E-Eval-Ctxt} rule.  Let $\conf =
  \config{S}{\evalctxt{E_b}{e_{b_1}}}$ and let $\conf_b =
  \config{S_b}{\evalctxt{E_b}{e_{b_2}}}$.

  Note that $\conf = \config{S}{\evalctxt{E_a}{e_{a_1}}} =
  \config{S}{\evalctxt{E_b}{e_{b_1}}}$, and so
  $\evalctxt{E_a}{e_{a_1}} = \evalctxt{E_b}{e_{b_1}}$, but $E_a$ and
  $E_b$ may differ and $e_{a_1}$ and $e_{b_1}$ may differ.

  Since $\config{S}{\evalctxt{E_a}{e_{a_1}}} \ctxstepsto
  \config{S_a}{\evalctxt{E_a}{e_{a_2}}}$ and
  $\config{S}{\evalctxt{E_b}{e_{b_1}}} \ctxstepsto
  \config{S_b}{\evalctxt{E_b}{e_{b_2}}}$ and $\evalctxt{E_a}{e_{a_1}}
  = \evalctxt{E_b}{e_{b_1}}$, we have from
  Lemma~\ref{lem:lvars-locality} (Locality) that there exist
  evaluation contexts $E'_a$ and $E'_b$ such that:

  \begin{itemize}
  \item $\evalctxt{E'_a}{e_{a_1}} = \evalctxt{E_b}{e_{b_2}}$, and
  \item $\evalctxt{E'_b}{e_{b_1}} = \evalctxt{E_a}{e_{a_2}}$, and
  \item $\evalctxt{E'_a}{e_{a_2}} =
  \evalctxt{E'_b}{e_{b_2}}$.
  \end{itemize}

  Our approach will be to show that there exist $S', i, j, \pi$ such
  that:
  \begin{itemize}
  \item $\config{S_a}{\evalctxt{E_a}{e_{a_2}}} \ctxstepsto^i
    \config{S'}{\evalctxt{E'_a}{e_{a_2}}}$, and
  \item $\pi(\config{S_b}{\evalctxt{E_b}{e_{b_2}}}) \ctxstepsto^j
    \pi(\config{S'}{\evalctxt{E'_a}{e_{a_2}}})$.
  \end{itemize}
  Since $\evalctxt{E'_a}{e_{a_1}} = \evalctxt{E_b}{e_{b_2}}$,
  $\evalctxt{E'_b}{e_{b_1}} = \evalctxt{E_a}{e_{a_2}}$, and
  $\evalctxt{E'_a}{e_{a_2}} = \evalctxt{E'_b}{e_{b_2}}$, it suffices
  to show that:
  \begin{itemize}
  \item $\config{S_a}{\evalctxt{E'_b}{e_{b_1}}} \ctxstepsto^i
    \config{S'}{\evalctxt{E'_b}{e_{b_2}}}$, and
  \item $\pi(\config{S_b}{\evalctxt{E'_a}{e_{a_1}}}) \ctxstepsto^j
    \pi(\config{S'}{\evalctxt{E'_a}{e_{a_2}}})$.
  \end{itemize}

  From the premise of {\sc E-Eval-Ctxt}, we have that
  $\config{S}{e_{a_1}} \parstepsto \config{S_a}{e_{a_2}}$ and
  $\config{S}{e_{b_1}} \parstepsto \config{S_b}{e_{b_2}}$.  We proceed
  by case analysis on the rule by which $\config{S}{e_{a_1}}$ steps to
  $\config{S_a}{e_{a_2}}$.

  \begin{enumerate}
  \item Case {\sc E-Beta}:

    We have:
    \begin{itemize}
      \item $e_{a_1} = \app{\lam{x}{e'_a}}{v_a}$,
      \item $e_{a_2} = \subst{e'_a}{x}{v_a}$, and
      \item $S_a = S$.
    \end{itemize}

    Now, we proceed by case analysis on the rule by which
    $\config{S}{e_{b_1}}$ steps to $\config{S_b}{e_{b_2}}$:
    \begin{enumerate}
    \item Case {\sc E-Beta}:

      We have:
      \begin{itemize}
      \item $e_{b_1} = \app{\lam{x}{e'_b}}{v_b}$,
      \item $e_{b_2} = \subst{e'_b}{x}{v_b}$, and
      \item $S_b = S$.
      \end{itemize}

      Choose $S' = S$, $i = 1$, $j = 1$, and $\pi = \id$.

      We have to show that:

      \begin{itemize}
      \item $\config{S}{\evalctxt{E'_b}{e_{b_1}}} \ctxstepsto
        \config{S}{\evalctxt{E'_b}{e_{b_2}}}$, and
      \item $\config{S}{\evalctxt{E'_a}{e_{a_1}}} \ctxstepsto
        \config{S}{\evalctxt{E'_a}{e_{a_2}}}$, 
      \end{itemize}

      both of which follow immediately from $\config{S}{e_{a_1}}
      \parstepsto \config{S_a}{e_{a_2}}$ and $\config{S}{e_{b_1}}
      \parstepsto \config{S_b}{e_{b_2}}$ and {\sc E-Eval-Ctxt}.

    \item Case {\sc E-New}:

      We have:
      \begin{itemize}
      \item $e_{b_1} = \NEW$,
      \item $e_{b_2} = l$, and
      \item $S_b = \extSRaw{S}{l}{\bot}$.
      \end{itemize}

      Choose $S' = S_b$, $i = 1$, $j = 1$, and $\pi = \id$.

      We have to show that:

      \begin{itemize}
      \item $\config{S}{\evalctxt{E'_b}{e_{b_1}}} \ctxstepsto
        \config{S_b}{\evalctxt{E'_b}{e_{b_2}}}$, and
      \item
        $\config{S_b}{\evalctxt{E'_a}{e_{a_1}}} \ctxstepsto
        \config{S_b}{\evalctxt{E'_a}{e_{a_2}}}$.
      \end{itemize}

      The first of these follows immediately from $\config{S}{e_{b_1}}
      \parstepsto \config{S_b}{e_{b_2}}$ and {\sc E-Eval-Ctxt}.  For
      the second, consider that $S_b = \extSRaw{S}{l}{\bot} =
      \lubstore{S}{\store{\storebindingRaw{l}{\bot}}}$.  Furthermore, we
      know from the side condition of {\sc E-New} that $l \notin
      \dom{S}$, so $\store{\storebindingRaw{l}{\bot}}$ is non-conflicting
      with the transition $\config{S}{e_{a_1}} \parstepsto
      \config{S_a}{e_{a_2}}$, and we know that
      $\lubstore{S_a}{\store{\storebindingRaw{l}{\bot}}} \neq \topS$
      since $S_a$ is just $S$.  Therefore, by
      Lemma~\ref{lem:lvars-independence} (Independence), we have that
      $\config{\lubstore{S}{\store{\storebindingRaw{l}{\bot}}}}{e_{a_1}}
      \parstepsto
      \config{\lubstore{S_a}{\store{\storebindingRaw{l}{\bot}}}}{e_{a_2}}$.
      Hence $\config{S_b}{e_{a_1}} \parstepsto \config{S_b}{e_{a_2}}$.
      By {\sc E-Eval-Ctxt}, it follows that
      $\config{S_b}{\evalctxt{E'_a}{e_{a_1}}} \ctxstepsto
      \config{S_b}{\evalctxt{E'_a}{e_{a_2}}}$, as we were required to
      show.

    \item Case {\sc E-Put}: \TODO{}
    \item Case {\sc E-Put-Err}: \TODO{}
    \item Case {\sc E-Get}:\TODO{}
    \end{enumerate}
  \item Case {\sc E-New}:

    Now, we proceed by case analysis on the rule by which
    $\config{S}{e_{b_1}}$ steps to $\config{S_b}{e_{b_2}}$:
    \begin{enumerate}
    \item Case {\sc E-Beta}: \TODO{}
    \item Case {\sc E-New}: \TODO{}
    \item Case {\sc E-Put}: \TODO{}
    \item Case {\sc E-Put-Err}: \TODO{}
    \item Case {\sc E-Get}: \TODO{}
    \end{enumerate}
  \item Case {\sc E-Put}:

    Now, we proceed by case analysis on the rule by which
    $\config{S}{e_{b_1}}$ steps to $\config{S_b}{e_{b_2}}$:
    \begin{enumerate}
    \item Case {\sc E-Beta}: \TODO{}
    \item Case {\sc E-New}: \TODO{}
    \item Case {\sc E-Put}: \TODO{}
    \item Case {\sc E-Put-Err}: \TODO{}
    \item Case {\sc E-Get}: \TODO{}
    \end{enumerate}
  \item Case {\sc E-Put-Err}:

    Now, we proceed by case analysis on the rule by which
    $\config{S}{e_{b_1}}$ steps to $\config{S_b}{e_{b_2}}$:
    \begin{enumerate}
    \item Case {\sc E-Beta}: \TODO{}
    \item Case {\sc E-New}: \TODO{}
    \item Case {\sc E-Put}: \TODO{}
    \item Case {\sc E-Put-Err}: \TODO{}
    \item Case {\sc E-Get}: \TODO{}
    \end{enumerate}
  \item Case {\sc E-Get}:

    Now, we proceed by case analysis on the rule by which
    $\config{S}{e_{b_1}}$ steps to $\config{S_b}{e_{b_2}}$:
    \begin{enumerate}
    \item Case {\sc E-Beta}: \TODO{}
    \item Case {\sc E-New}: \TODO{}
    \item Case {\sc E-Put}: \TODO{}
    \item Case {\sc E-Put-Err}: \TODO{}
    \item Case {\sc E-Get}: \TODO{}
    \end{enumerate}
  \end{enumerate}

  \lk{I think we also still have to separately deal with cases where
    $\conf_a = \error$ or $\conf_b = \error$.}
\end{proof}


\section{Proof of Lemma~\ref{lem:lvars-strong-one-sided-confluence}}\label{section:lvars-strong-one-sided-confluence-proof}
\begin{proof}
  Suppose $\conf \ctxstepsto \conf'$ and $\conf \ctxstepsto^m
  \conf''$, where $1 \leq m$.  We have to show that there exist
  $\conf_c, i, j, \pi$ such that $\conf' \ctxstepsto^i \conf_c$ and
  $\pi(\conf'') \ctxstepsto^j \conf_c$ and $i \leq m$ and $j \leq 1$.

  We proceed by induction on $m$.  In the base case of $m = 1$, the
  result is immediate from
  Lemma~\ref{lem:lvars-strong-local-confluence}.

  For the induction step, suppose $\conf \ctxstepsto^m \conf''
  \ctxstepsto \conf'''$ and suppose the lemma holds for $m$.

  We show that it holds for $m + 1$, as follows.

  We are required to show that there exist $\conf_c, i, j, \pi$ such
  that $\conf' \ctxstepsto^{i} \conf_c$ and $\pi(\conf''')
  \ctxstepsto^{j} \conf_c$ and $i \leq m + 1$ and $j \leq 1$.

  From the induction hypothesis, there exist $\conf_c', i', j', \pi'$
  such that $\conf' \ctxstepsto^{i'} \conf_c'$ and $\pi'(\conf'')
  \ctxstepsto^{j'} \conf_c'$ and $i' \leq m$ and $j' \leq 1$.

  We proceed by cases on $j'$:
  \begin{itemize}

  \item If $j' = 0$, then $\pi'(\conf'') = \conf_c'$.

    Since $\conf'' \ctxstepsto \conf'''$, we have that $\pi'(\conf'')
    \ctxstepsto \pi'(\conf''')$ by
    Lemma~\ref{lem:lvars-permutability} (Permutability).

    We can then choose $\conf_c = \pi'(\conf''')$ and $i = i' + 1$ and
    $j = 0$ and $\pi = \pi'$.  The key is that $\conf'
    \ctxstepsto^{i'} \conf'_c = \pi'(\conf'') \ctxstepsto
    \pi'(\conf''')$ for a total of $i' + 1$ steps.
    
  \item If $j' = 1$:

    First, since $\pi'(\conf'') \ctxstepsto^{j'} \conf'_c$, then by
    Lemma~\ref{lem:lvars-permutability} (Permutability) we have that
    $\conf'' \ctxstepsto^{j'} \piprimeinv(\conf'_c)$.

    Then, by $\conf'' \ctxstepsto^{j'} \piprimeinv(\conf'_c)$ and
    $\conf'' \ctxstepsto \conf'''$ and
    Lemma~\ref{lem:lvars-strong-local-confluence} (Strong Local
    Confluence), we have that there exist $\conf_c''$ and $i''$ and
    $j''$ and $\pi''$ such that $\piprimeinv(\conf'_c)
    \ctxstepsto^{i''} \conf_c''$ and $\pi''(\conf''')
    \ctxstepsto^{j''} \conf_c''$ and $i'' \leq 1$ and $j'' \leq 1$.

    Since $\piprimeinv(\conf'_c) \ctxstepsto^{i''} \conf_c''$, by
    Lemma~\ref{lem:lvars-permutability} (Permutability) we have that
    $\conf'_c \ctxstepsto^{i''} \pi'(\conf_c'')$.

    So we also have $\conf' \ctxstepsto^{i'} \conf_c'
    \ctxstepsto^{i''} \pi'(\conf_c'')$.

    Since $\pi''(\conf''') \ctxstepsto^{j''} \conf_c''$, by
    Lemma~\ref{lem:lvars-permutability} (Permutability) we have that
    $\pi'(\pi''(\conf''')) \ctxstepsto^{j''} \pi'(\conf_c'')$.

    In summary, we pick $\conf_c = \pi'(\conf_c'')$ and $i = i' + i''$
    and $j = j''$ and $\pi = \pi'' \circ \pi'$, which is sufficient
    because $i = i' + i'' \leq m + 1$ and $j = j'' \leq 1$.
  \end{itemize}

 \end{proof}


\section{Proof of Lemma~\ref{lem:lvars-strong-confluence}}\label{section:lvars-strong-confluence-proof}
\begin{proof}
  We proceed by induction on $n$.  In the base case of $n = 1$, the
  result is immediate from
  Lemma~\ref{lem:lvars-strong-one-sided-confluence}.

  For the induction step, suppose $\conf \parstepsto^n \conf'
  \parstepsto \conf'''$ and suppose the lemma holds for $n$.

  We show that it holds for $n + 1$, as follows.

  We are required to show that there exist $\conf_c, i, j$ such that
  $\conf''' \parstepsto^i \conf_c$ and $\conf'' \parstepsto^j \conf_c$
  and $i \leq m$ and $j \leq n + 1$.

  From the induction hypothesis, we have that there exist $\conf'_c,
  i', j'$ such that $\conf' \parstepsto^{i'} \conf'_c$ and $\conf''
  \parstepsto^{j'} \conf'_c$ and $i' \leq m$ and $j' \leq n$.

  We proceed by cases on $i'$:
  \begin{itemize}

  \item If $i' = 0$, then $\conf' = \conf_c'$.  We can then choose
    $\conf_c = \conf'''$ and $i = 0$ and $j = j' + 1$.

  \item If $i' \geq 1$:

    From $\conf' \parstepsto \conf'''$ and $\conf' \parstepsto^{i'}
    \conf_c'$ and Lemma~\ref{lem:lvars-strong-one-sided-confluence},
    we have that there exist $\conf_c''$ and $i''$ and $j''$ such that
    $\conf''' \parstepsto^{i''} \conf_c''$ and $\conf_c'
    \parstepsto^{j''} \conf_c''$ and $i'' \leq i'$ and $j'' \leq 1$.
    So we also have $\conf'' \parstepsto^{j'} \conf_c'
    \parstepsto^{j''} \conf_c''$.  In summary, we pick $\conf_c =
    \conf_c''$ and $i = i''$ and $j = j' + j''$, which is sufficient
    because $i = i'' \leq i' \leq m$ and $j = j' + j'' \leq n + 1$.
  \end{itemize}

\end{proof}


\section{Proof of Lemma~\ref{lem:lattice-structure}}\label{section:lattice-structure-proof}
\begin{proof}
  Suppose that $(D, \userleq, \bot, \top)$ is a lattice and $(D_p,
  \leqp, \botp, \topp) = \Freeze{D, \userleq, \bot, \top}$.

  In order to show that $(D_p, \leqp, \botp, \topp)$ is a lattice, we
  have to show that:
  \begin{enumerate}
  \item $\leqp$ is a partial order over $D_p$.

  \item Every nonempty finite subset of $D_p$ has a lub.

  \item $\botp$ is the least element of $D_p$.

  \item $\topp$ is the greatest element of $D_p$.
  \end{enumerate}

  We prove each of these properties in turn:

  \begin{enumerate}
  \item $\leqp$ is a partial order over $D_p$.

    To show this, we need to show that $\leqp$ is reflexive, transitive,
    and antisymmetric. 
    \begin{enumerate}
    \item $\leqp$ is reflexive.

      Suppose $v \in D_p$.

      Then, by Lemma~\ref{lem:partition-of-Dp}, either $v =
      \state{d}{\frozenfalse}$ with $d \in D$, or $v =
      \state{x}{\frozentrue}$ with $x \in X$, where $X = D -
      \setof{\top}$.
      \begin{itemize}
      \item Suppose $v = \state{d}{\frozenfalse}$:

        By the reflexivity of $\userleq$, we know $d \userleq d$.

        By the definition of $\leqp$, we know $\state{d}{\frozenfalse}
        \leqp \state{d}{\frozenfalse}$.

      \item Suppose $v = \state{x}{\frozentrue}$: 
        
        By the reflexivity of equality, $x = x$.

        By the definition of $\leqp$, we know $\state{x}{\frozentrue}
        \leqp \state{x}{\frozentrue}$.
      \end{itemize}

    \item $\leqp$ is transitive. 

      Suppose $v_1 \leqp v_2$ and $v_2 \leqp v_3$.

      We want to show that $v_1 \leqp v_3$.

      We proceed by case analysis on $v_1, v_2$, and $v_3$.
      \begin{itemize}
      \item Case $v_1 = \state{d_1}{\frozenfalse}$ and $v_2 =
        \state{d_2}{\frozenfalse}$ and $v_3 =
        \state{d_3}{\frozenfalse}$:
        
        By inversion on $\leqp$, it follows that $d_1 \userleq d_2$.

        By inversion on $\leqp$, it follows that $d_2 \userleq d_3$.

        By the transitivity of $\userleq$, we know $d_1 \userleq d_3$.

        By the definition of $\leqp$, it follows that
        $\state{d_1}{\frozenfalse} \leqp \state{d_3}{\frozenfalse}$.

        Hence $v_1 \leqp v_3$.

      \item Case $v_1 = \state{d_1}{\frozenfalse}$ and $v_2 =
        \state{d_2}{\frozenfalse}$ and $v_3 =
        \state{x_3}{\frozentrue}$:

        By inversion on $\leqp$, it follows that $d_1 \userleq d_2$.

        By inversion on $\leqp$, it follows that $d_2 \userleq x_3$.

        By the transitivity of $\userleq$, we know $d_1 \userleq x_3$.

        By the definition of $\leqp$, it follows that
        $\state{d_1}{\frozenfalse} \leqp \state{x_3}{\frozentrue}$.

        Hence $v_1 \leqp v_3$.

      \item Case $v_1 = \state{d_1}{\frozenfalse}$ and $v_2 =
        \state{x_2}{\frozentrue}$ and $v_3 =
        \state{d_3}{\frozenfalse}$:

        By inversion on $\leqp$, it follows that $d_1 \userleq x_2$.

        By inversion on $\leqp$, it follows that $d_3 = \top$.

        Since $\top$ is the maximal element of $D$, we know $d_1
        \userleq \top \equiv d_3$.

        By the definition of $\leqp$, it follows that
        $\state{d_1}{\frozenfalse} \leqp \state{d_3}{\frozenfalse}$.

        Hence $v_1 \leqp v_3$.

      \item Case $v_1 = \state{d_1}{\frozenfalse}$ and $v_2 =
        \state{x_2}{\frozentrue}$ and $v_3 =
        \state{x_3}{\frozentrue}$:

        By inversion on $\leqp$, it follows that $d_1 \userleq x_2$.

        By inversion on $\leqp$, it follows that $x_2 = x_3$.

        Hence $d_1 \userleq x_3$.

        By the definition of $\leqp$, it follows that
        $\state{d_1}{\frozenfalse} \leqp \state{x_3}{\frozentrue}$.

        Hence $v_1 \leqp v_3$.

      \item Case $v_1 = \state{x_1}{\frozentrue}$ and $v_2 =
        \state{d_2}{\frozenfalse}$ and $v_3 =
        \state{d_3}{\frozenfalse}$:

        By inversion on $\leqp$, it follows that $d_2 = \top$.

        By inversion on $\leqp$, it follows that $d_2 \userleq d_3$.

        Since $\top$ is maximal, it follows that $d_3 = \top$.

        By the definition of $\leqp$, it follows that
        $\state{x_1}{\frozentrue} \leqp \state{d_3}{\frozenfalse}$.

        Hence $v_1 \leqp v_3$. 

      \item Case $v_1 = \state{x_1}{\frozentrue}$ and $v_2 =
        \state{d_2}{\frozenfalse}$ and $v_3 =
        \state{x_3}{\frozentrue}$:

        By inversion on $\leqp$, it follows that $d_2 = \top$.

        By inversion on $\leqp$, it follows that $d_2 \userleq x_3$.

        Since $\top$ is maximal, it follows that $x_3 = \top$.

        But since $x_3 \in X \subseteq D/\setof{\top}$, we know $x_3
        \not= \top$.

        This is a contradiction. \\

        Hence $v_1 \leqp v_3$. 

      \item Case $v_1 = \state{x_1}{\frozentrue}$ and $v_2 =
        \state{x_2}{\frozentrue}$ and $v_3 =
        \state{d_3}{\frozenfalse}$:

        By inversion on $\leqp$, it follows that $x_1 = x_2$.

        By inversion on $\leqp$, it follows that $d_3 = \top$.

        By the definition of $\leqp$, it follows that
        $\state{x_1}{\frozentrue} \leqp \state{d_3}{\frozenfalse}$.

        Hence $v_1 \leqp v_3$. 

      \item Case $v_1 = \state{x_1}{\frozentrue}$ and $v_2 =
        \state{x_2}{\frozentrue}$ and $v_3 =
        \state{x_3}{\frozentrue}$:

        By inversion on $\leqp$, it follows that $x_1 = x_2$.

        By inversion on $\leqp$, it follows that $x_2 = x_3$.

        By transitivity of $=$, $x_1 = x_3$.

        By the definition of $\leqp$, it follows that
        $\state{x_1}{\frozentrue} \leqp \state{x_3}{\frozentrue}$.

        Hence $v_1 \leqp v_3$. 
        
      \end{itemize}

    \item $\leqp$ is antisymmetric. 

      Suppose $v_1 \leqp v_2$ and $v_2 \leqp v_1$. Now, we proceed by
      cases on $v_1$ and $v_2$.
      \begin{itemize}
      \item Case $v_1 = \state{d_1}{\frozenfalse}$ and $v_2 =
        \state{d_2}{\frozenfalse}$:
        
        By inversion on $v_1 \leqp v_2$, we know that $d_1 \userleq
        d_2$.

        By inversion on $v_2 \leqp v_1$, we know that $d_2 \userleq
        d_1$.

        By the antisymmetry of $\leq$, we know $d_1 = d_2$.

        Hence $v_1 = v_2$. 

      \item Case $v_1 = \state{d_1}{\frozenfalse}$ and $v_2 =
        \state{x_2}{\frozentrue}$:

        By inversion on $v_1 \leqp v_2$, we know that $d_1 \userleq x_2$.

        By inversion on $v_2 \leqp v_1$, we know that $d_1 = \top$.

        Since $\top$ is maximal in $D$, we know $x_2 = \top$.

        But since $x_2 \in X \subseteq D/\setof{\top}$, we know $x_2 \not= \top$.

        This is a contradiction.

        Hence $v_1 = v_2$. 
        
      \item Case $v_1 = \state{x_1}{\frozentrue}$ and $v_2 =
        \state{d_2}{\frozenfalse}$:

        Similar to the previous case. 

      \item Case $v_1 = \state{x_1}{\frozentrue}$ and $v_2 =
        \state{x_2}{\frozentrue}$:

        By inversion on $v_1 \leqp v_2$, we know that $x_1 = x_2$.

        Hence $v_1 = v_2$. 
      \end{itemize}
    \end{enumerate}

  \item Every nonempty finite subset of $D_p$ has a lub.

    To show this, it is sufficient to show that every two elements of
    $D_p$ have a lub, since a binary lub operation can be repeatedly
    applied to compute the lub of any finite set.

    We will show that every two elements of $D_p$ have a lub by
    showing that the $\lubp{}{}$ operation defined by
    Definition~\ref{def:lubp} computes their lub.

    It suffices to show the following two properties:
    \begin{enumerate}
    \item For all $v_1, v_2, v \in D_p$, if $v_1 \leqp v$ and $v_2
      \leqp v$, then $(\lubp{v_1}{v_2}) \leqp v$.
    \item For all $v_1, v_2 \in D_p$, $v_1 \leqp (\lubp{v_1}{v_2})$
      and $v_2 \leqp (\lubp{v_1}{v_2})$.
    \end{enumerate}
    \begin{enumerate}
    \item For all $v_1, v_2, v \in D_p$, if $v_1 \leqp v$ and $v_2
      \leqp v$, then $\lubp{v_1}{v_2} \leqp v$.
      
      Assume $v_1, v_2, v \in D_p$, and $v_1 \leqp v$ and $v_2 \leqp
      v$.

      Now we do a case analysis on $v_1$ and $v_2$.
      \begin{itemize}
      \item Case $v_1 = \state{d_1}{\frozenfalse}$ and $v_2 =
        \state{d_2}{\frozenfalse}$.
        
        Now case on $v$: 
        \begin{itemize}
        \item Case $v = \state{d}{\frozenfalse}$: 

          By the definition of $\lubp{}{}$,
          $\lubp{\state{d_1}{\frozenfalse}}{\state{d_2}{\frozenfalse}}
          = \state{\userlub{d_1}{d_2}}{\frozenfalse}$.

          By inversion on $\state{d_1}{\frozenfalse} \leqp
          \state{d}{\frozenfalse}$, $d_1 \userleq l$.

          By inversion on $\state{d_2}{\frozenfalse} \leqp
          \state{d}{\frozenfalse}$, $d_2 \userleq l$.

          Hence $l$ is an upper bound for $d_1$ and $d_2$.

          Hence $\userlub{d_1}{d_2} \userleq l$.

          Hence $\state{\userlub{d_1}{d_2}}{\frozenfalse} \leqp
          \state{d}{\frozenfalse}$.

          Hence $\lubp{v_1}{v_2} \leqp v$.
          
        \item Case $v = \state{x}{\frozentrue}$: 
          
          By the definition of $\lubp{}{}$, $\state{d_1}{\frozenfalse}
          \lubp{}{} \state{d_2}{\frozenfalse} =
          \state{\userlub{d_1}{d_2}}{\frozenfalse}$.

          By inversion on $\state{d_1}{\frozenfalse} \leqp
          \state{x}{\frozentrue}$, $d_1 \userleq x$.

          By inversion on $\state{d_2}{\frozenfalse} \leqp
          \state{x}{\frozentrue}$, $d_2 \userleq x$.
     
          Hence $x$ is an upper bound for $d_1$ and $d_2$.

          Hence $\userlub{d_1}{d_2} \userleq x$.

          Hence $\state{\userlub{d_1}{d_2}}{\frozenfalse} \leqp
          \state{x}{\frozentrue}$.

          Hence $\lubp{v_1}{v_2} \leqp v$.
        \end{itemize}
        
      \item Case $v_1 = \state{x_1}{\frozentrue}$ and $v_2 =
        \state{x_2}{\frozentrue}$:
        
        Now case on $v$: 
        \begin{itemize}
        \item Case $v = \state{d}{\frozenfalse}$: 
          
          By inversion on $\state{x_1}{\frozentrue} \leqp
          \state{d}{\frozenfalse}$, we know $l = \top$.

          By inversion on $\state{x_2}{\frozentrue} \leqp
          \state{d}{\frozenfalse}$, we know $l = \top$.

          Now consider whether $x_1 = x_2$ or not.
        
          If it does, then by the definition of $\lubp{}{}$,
          $\state{x_1}{\frozentrue} \lubp{}{} \state{x_2}{\frozentrue}
          = \state{x_1}{\frozentrue}$.

          By definition of $\leqp$, we have $\state{x_1}{\frozentrue}
          \leqp \state{\top}{\frozenfalse}$.

          So $\lubp{v_1}{v_2} \leqp v$.

          If it does not, then $\lubp{v_1}{v_2} =
          \state{\top}{\frozenfalse}$.

          By the definition of $\leqp$, we have
          $\state{\top}{\frozenfalse} \leqp
          \state{\top}{\frozenfalse}$.

          So $\lubp{v_1}{v_2} \leqp v$.
          
        \item Case $v = \state{x}{\frozentrue}$: 
          
          By inversion on $\state{x_1}{\frozentrue} \leqp
          \state{x}{\frozentrue}$, we know $x = x_1$.

          By inversion on $\state{x_2}{\frozentrue} \leqp
          \state{x}{\frozentrue}$, we know $x = x_2$.

          Hence $x_1 = x_2$.

          By the definition of $\lubp{}{}$, $\state{x_1}{\frozentrue}
          \lubp{}{} \state{x_2}{\frozentrue} =
          \state{x_1}{\frozentrue}$.

          Hence $\lubp{v_1}{v_2} \leqp v$.
        \end{itemize}
        
      \item Case $v_1 = \state{x_1}{\frozentrue}$ and $v_2 =
        \state{d_2}{\frozenfalse}$:
        
        Now case on $v$:
        \begin{itemize}
        \item Case $v = \state{d}{\frozenfalse}$:
          
          Now consider whether $d_2 \userleq x_1$.

          If it is, then $\state{x_1}{\frozentrue} \lubp{}{}
          \state{d_2}{\frozenfalse} = \state{x_1}{\frozentrue} = v_1$.

          Hence $\lubp{v_1}{v_2} \leqp v$.

          Otherwise, $\state{x_1}{\frozentrue} \lubp{}{}
          \state{d_2}{\frozenfalse} = \state{\top}{\frozenfalse}$.

          By inversion on $\state{x_1}{\frozentrue} \leqp
          \state{d}{\frozenfalse}$, we know $l = \top$.

          By reflexivity, $\state{\top}{\frozenfalse} \leqp
          \state{\top}{\frozenfalse}$.

          Hence $\lubp{v_1}{v_2} \leqp v$. 
          
        \item Case $v = \state{x}{\frozentrue}$:  
          
          By inversion on $\state{x_1}{\frozentrue} \leqp
          \state{x}{\frozentrue}$, we know that $x_1 = x$.

          By inversion on $\state{d_2}{\frozenfalse} \leqp
          \state{x}{\frozentrue}$, we know that $d_2 \userleq x$.

          By transitivity, $d_2 \userleq x_1$.

          By the definition of $\lubp{}{}$, it follows that
          $\state{x_1}{\frozentrue} \lubp{}{}
          \state{d_2}{\frozenfalse} = \state{x_1}{\frozentrue}$.

          By definition of $\leqp$, $\state{x_1}{\frozentrue} \leqp
          \state{x_1}{\frozentrue}$.

          Hence $\lubp{v_1}{v_2} \leqp v$. 
        \end{itemize}
        
      \item Case $v_1 = \state{d_1}{\frozenfalse}$ and $v_2 =
        \state{x_2}{\frozentrue}$:
        
        Symmetric with the previous case. 
      \end{itemize}
    \item For all $v_1, v_2 \in D_p$, $v_1 \leqp \lubp{v_1}{v_2}$ and
      $v_2 \leqp \lubp{v_1}{v_2}$.
      
      Assume $v_1, v_2 \in D_p$, and proceed by case analysis. 
      \begin{itemize}
      \item Case $v_1 = \state{d_1}{\frozenfalse}$ and $v_2 =
        \state{d_2}{\frozenfalse}$:

        Since $\userlub{}{}$ is a join operator, we know $d_1 \userleq
        \userlub{d_1}{d_2}$.

        By the definition of $\leqp$, $\state{d_1}{\frozenfalse}
        \userleq \state{\userlub{d_1}{d_2}}{\frozenfalse}$.

        By the definition of $\lubp{}{}$, $\lubp{v_1}{v_2} =
        \state{\userlub{d_1}{d_2}}{\frozenfalse}$.

        Hence $v_1 \leqp \lubp{v_1}{v_2}$.

        Since $\userlub{}{}$ is a join operator, we know $d_1 \userleq
        \userlub{d_1}{d_2}$.

        By the definition of $\leqp$, $\state{d_2}{\frozenfalse}
        \userleq \state{\userlub{d_1}{d_2}}{\frozenfalse}$.

        By the definition of $\lubp{}{}$, $\lubp{v_1}{v_2} =
        \state{\userlub{d_1}{d_2}}{\frozenfalse}$.

        Hence $v_2 \leqp \lubp{v_1}{v_2}$. 

        Therefore $v_1 \leqp v_1 \userlub{}{} v_2$ and $v_2 \leqp v_1
        \userlub{}{} v_2$.
 
      \item Case $v_1 = \state{d_1}{\frozenfalse}$ and $v_2 = \state{x_2}{\frozentrue}$:

        Consider whether $d_1 \userleq x_2$. 
        \begin{itemize}
        \item Case  $d_1 \userleq x_2$:

          By the definition of $\lubp{}{}$, we know
          $\state{d_1}{\frozenfalse} \lubp{}{}
          \state{x_2}{\frozentrue} = \state{x_2}{\frozentrue}$.

          By the definition of $\lubp{}{}$, we know
          $\state{d_1}{\frozenfalse} \leqp \state{x_2}{\frozentrue}$.

          Hence $v_1 \leqp \lubp{v_1}{v_2}$.

          By reflexivity, $\state{x_2}{\frozentrue} \leqp
          \state{x_2}{\frozentrue}$.

          Hence $v_2 \leqp \lubp{v_1}{v_2}$.

          Therefore $v_1 \leqp v_1 \userlub{}{} v_2$ and $v_2 \leqp
          v_1 \userlub{}{} v_2$.

        \item Case $d_1 \not\userleq x_2$:

          By the definition of $\lubp{}{}$, we know
          $\state{d_1}{\frozenfalse} \lubp{}{}
          \state{x_2}{\frozentrue} = \state{\top}{\frozenfalse}$.

          Since $d_1 \userleq \top$, by the definition of $\leqp$ we
          know $\state{d_1}{\frozenfalse} \userleq
          \state{\top}{\frozenfalse}$.

          Hence $v_1 \leqp \lubp{v_1}{v_2}$.

          By the definition of $\leqp$, we know
          $\state{x_2}{\frozentrue} \userleq
          \state{\top}{\frozenfalse}$.

          Hence $v_2 \leqp \lubp{v_1}{v_2}$.

          Therefore $v_1 \leqp v_1 \userlub{}{} v_2$ and $v_2 \leqp
          v_1 \userlub{}{} v_2$.
        \end{itemize}
      \item Case $v_1 = \state{x_1}{\frozentrue}$ and $v_2 =
        \state{d_2}{\frozenfalse}$:

        Symmetric with the previous case. 
      \item Case $v_1 = \state{x_1}{\frozentrue}$ and $v_2 =
        \state{x_2}{\frozentrue}$:

        Consider whether $x_1$ equals $x_2$. 
        \begin{itemize}
        \item Case $x_1 = x_2$:
          
          By the definition $\lubp{}{}$, $\state{x_1}{\frozentrue}
          \lubp{}{} \state{x_2}{\frozentrue} =
          \state{x_1}{\frozentrue}$.
 
          By reflexivity, $\state{x_1}{\frozentrue} \leqp
          \state{x_1}{\frozentrue}$.

          Hence $v_1 \leqp \lubp{v_1}{v_2}$.

          By reflexivity, $\state{x_2}{\frozentrue} \leqp
          \state{x_1}{\frozentrue}$.

          Hence $v_2 \leqp \lubp{v_1}{v_2}$.

          Therefore $v_1 \leqp v_1 \userlub{}{} v_2$ and $v_2 \leqp
          v_1 \userlub{}{} v_2$.

        \item Case $x_1 \not= x_2$: 

          By the definition $\lubp{}{}$, $\state{x_1}{\frozentrue}
          \lubp{}{} \state{x_2}{\frozentrue} =
          \state{\top}{\frozenfalse}$.

          By the definition of $\leqp$, $\state{x_1}{\frozentrue}
          \leqp \state{\top}{\frozenfalse}$.

          Hence $v_1 \leqp \lubp{v_1}{v_2}$.

          By the definition of $\leqp$, $\state{x_2}{\frozentrue}
          \leqp \state{\top}{\frozenfalse}$.

          Hence $v_2 \leqp \lubp{v_1}{v_2}$.

          Therefore $v_1 \leqp v_1 \userlub{}{} v_2$ and $v_2 \leqp
          v_1 \userlub{}{} v_2$.
        \end{itemize}
      \end{itemize}
    \end{enumerate}

  \item $\botp$ is the least element of $D_p$. 

    $\botp$ is defined to be $\state{\bot}{\frozenfalse}$.

    In order to be the least element of $D_p$, it must be less than or
    equal to every element of $D_p$.

    By Lemma~\ref{lem:partition-of-Dp}, the elements of $D_p$
    partition into $\state{d}{\frozenfalse}$ for all $d \in D$, and
    $\state{x}{\frozentrue}$ for all $x \in X$, where $X = D -
    \setof{\top}$.

    We consider both cases:

    \begin{itemize}
    \item $\state{d}{\frozenfalse}$ for all $d \in D$:

      By the definition of $\leqp$, $\state{\bot}{\frozenfalse} \leqp
      \state{d}{\frozenfalse}$ iff $\bot \userleq d$.

      Since $\bot$ is the least element of $D$, $\bot \userleq d$.

      Therefore $\botp = \state{\bot}{\frozenfalse} \leqp
      \state{d}{\frozenfalse}$.

    \item $\state{x}{\frozentrue}$ for all $x \in X$:

      By the definition of $\leqp$, $\state{\bot}{\frozenfalse} \leqp
      \state{x}{\frozentrue}$ iff $\bot \userleq x$.

      Since $\bot$ is the least element of $D$, $\bot \userleq x$.

      Therefore $\botp = \state{\bot}{\frozenfalse} \leqp
      \state{x}{\frozentrue}$.

    \end{itemize}

    Therefore $\botp$ is less than or equal to all elements of $D_p$.

  \item $\topp$ is the greatest element of $D_p$.

    $\topp$ is defined to be $\state{\top}{\frozenfalse}$.

    In order to be the greatest element of $D_p$, every element of
    $D_p$ must be less than or equal to it.

    By Lemma~\ref{lem:partition-of-Dp}, the elements of $D_p$
    partition into $\state{d}{\frozenfalse}$ for all $d \in D$, and
    $\state{x}{\frozentrue}$ for all $x \in X$, where $X = D -
    \setof{\top}$.

    We consider both cases:

    \begin{itemize}
    \item $\state{d}{\frozenfalse}$ for all $d \in D$:

      By the definition of $\leqp$, $\state{d}{\frozenfalse} \leqp
      \state{\top}{\frozenfalse}$ iff $d \userleq \top$.

      Since $\top$ is the greatest element of $D$, $d \userleq \top$.

      Therefore $\state{d}{\frozenfalse} \leqp
      \state{\top}{\frozenfalse} = \topp$.

    \item $\state{x}{\frozentrue}$ for all $x \in X$:

      By the definition of $\leqp$, $\state{x}{\frozentrue} \leqp
      \state{\top}{\frozenfalse}$ iff $\top \userleq \top$.

      Therefore $\state{x}{\frozentrue} \leqp
      \state{\top}{\frozenfalse} = \topp$.

    \end{itemize}

    Therefore all elements of $D_p$ are less than or equal to $\topp$.
  \end{enumerate}
\end{proof}


\section{Proof of Lemma~\ref{lem:monotonicity}}\label{section:monotonicity-proof}
\begin{proof}
  \TODO{Fix the typos I found in this.}

  \begin{itemize}

    \item Case {\sc E-Eval-Ctxt}:

      Given: $\config{S}{\E{e}} \parstepsto \config{S'}{\E{e'}}$.

      To show: $\leqstore{S}{S'}$.

      From the premise of {\sc E-Eval-Ctxt}, $\config{S}{e}
      \parstepsto \config{S'}{e'}$.

      Hence by IH, $\leqstore{S}{S'}$, as we were required to show.

    \item Case {\sc E-Beta}:

      Immediate by the definition of $\leqstore{}{}$, since $S$ does
      not change.

    \item Case {\sc E-New}:

      Given: $\config{S}{\NEW} \parstepsto
      \config{\extS{S}{l}{\bot}{\frozenfalse}}{l}$.

      To show: $\leqstore{S}{\extS{S}{l}{\bot}{\frozenfalse}}$.

      By Definition~\ref{def:leqstore}, we have to show that $\dom{S}
      \subseteq \dom{\extS{S}{l}{\bot}{\frozenfalse}}$ and that for
      all $l' \in \dom{S}, \\
      S(l') \leqp (\extS{S}{l}{\bot}{\frozenfalse})(l')$.

      By the definition of store update,
      $\extS{S}{l}{d_1}{\frozentrue}$ can only either update an
      existing binding in $S$ or extend $S$ with a new binding.

      Hence $\dom{S} \subseteq \dom{\extS{S}{l}{\bot}{\frozenfalse}}$.

      From the side condition of {\sc E-New}, $l \notin \dom{S}$.

      Hence $\extS{S}{l}{\bot}{\frozenfalse}$ adds a new binding for
      $l$ in $S$.

      Hence $\extS{S}{l}{d_1}{\frozentrue}$ does not update any
      existing bindings in $S$.

      Hence, for all $l' \in \dom{S}, S(l') \leqp
      (\extS{S}{l}{d_1}{\frozentrue})(l')$.

      Therefore $\leqstore{S}{\extS{S}{l}{\bot}{\frozenfalse}}$, as
      required.

    \item Case {\sc E-Put}:

      Given: $\config{S}{\putexp{l}{d_2}} \parstepsto
      \config{\extSRaw{S}{l}{p_2}}{\unit}$.

      To show: $\leqstore{S}{\extSRaw{S}{l}{p_2}}$.

      By Definition~\ref{def:leqstore}, we have to show that $\dom{S}
      \subseteq \dom{\extSRaw{S}{l}{p_2}}$ and that for all $l' \in
      \dom{S}, \\
      S(l') \leqp (\extSRaw{S}{l}{p_2})(l')$.

      By the definition of store update, $\extSRaw{S}{l}{p_2}$ can only
      either update an existing binding in $S$ or extend $S$ with a
      new binding.

      Hence $\dom{S} \subseteq \dom{\extSRaw{S}{l}{p_2}}$.

      From the premises of {\sc E-Put}, $S(l) = p_1$.  Therefore $l
      \in \dom{S}$.

      Hence $\extSRaw{S}{l}{p_2}$ updates the existing binding for $l$
      in $S$ from $p_1$ to $p_2$.

      From the premises of {\sc E-Put}, $p_2 =
      \lubp{p_1}{\state{d_2}{\frozenfalse}}$.

      Hence, by the definition of $\lubp{}{}$, $p_1 \leqp p_2$.

      $\extSRaw{S}{l}{p_2}$ does not update any other bindings in $S$,
      hence, for all $l' \in \dom{S}, S(l') \leqp
      (\extSRaw{S}{l}{p_2})(l')$.

      Hence $\leqstore{S}{\extSRaw{S}{l}{p_2}}$, as required.

    \item Case {\sc E-Put-Err}:

      Given: $\config{S}{\putexp{l}{d_2}} \parstepsto \error$.

      By the definition of $\error$, $\error = \config{\topS}{e}$ for
      any $e$.

      To show: $\leqstore{S}{\topS}$.

      Immediate by the definition of $\leqstore{}{}$.

    \item Case {\sc E-Get}:

      Immediate by the definition of $\leqstore{}{}$, since $S$ does
      not change.

    \item Case {\sc E-Freeze-Init}:

      Immediate by the definition of $\leqstore{}{}$, since $S$ does
      not change.

    \item Case {\sc E-Spawn-Handler}:

      Immediate by the definition of $\leqstore{}{}$, since $S$ does
      not change.

    \item Case {\sc E-Freeze-Final}:

      Given: $\config{S}{\freezeafterfull{l}{Q}{v}{\setof{v\dots}}{H}}
      \parstepsto \config{\extS{S}{l}{d_1}{\frozentrue}}{d_1}$.

      To show: $\leqstore{S}{\extS{S}{l}{d_1}{\frozentrue}}$.

      By Definition~\ref{def:leqstore}, we have to show that $\dom{S}
      \subseteq \dom{\extS{S}{l}{d_1}{\frozentrue}}$ and that for all
      $l' \in \dom{S}, \\
      S(l') \leqp (\extS{S}{l}{d_1}{\frozentrue})(l')$.

      \lk{We could spell this out in even more excruciating detail,
        but I think it's obvious enough.}

      By the definition of store update,
      $\extS{S}{l}{d_1}{\frozentrue}$ can only either update an
      existing binding in $S$ or extend $S$ with a new binding.

      Hence $\dom{S} \subseteq \dom{\extS{S}{l}{d_1}{\frozentrue}}$.

      From the premises of {\sc E-Freeze-Final}, $S(l) =
      \state{d_1}{\status_1}$.  Therefore $l \in \dom{S}$.

      Hence $\extS{S}{l}{d_1}{\frozentrue}$ updates the existing
      binding for $l$ in $S$ from $\state{d_1}{\status_1}$ to
      $\state{d_1}{\frozentrue}$.

      By the definition of $\leqp$, $\state{d_1}{\status_1} \leqp
      \state{d_1}{\frozentrue}$.

      $\extS{S}{l}{d_1}{\frozentrue}$ does not update any other
      bindings in $S$, hence, for all $l' \in \dom{S}, \\
      S(l') \leqp (\extS{S}{l}{d_1}{\frozentrue})(l')$.

      Hence $\leqstore{S}{\extS{S}{l}{d_1}{\frozentrue}}$, as
      required.

    \item Case {\sc E-Freeze-Simple}:

      Given: $\config{S}{\freeze{l}} \parstepsto
      \config{\extS{S}{l}{d_1}{\frozentrue}}{d_1}$.

      To show: $\leqstore{S}{\extS{S}{l}{d_1}{\frozentrue}}$.

      Similar to the previous case.

  \end{itemize}

\end{proof}


\section{Proof of Lemma~\ref{lem:independence}}\label{section:independence-proof}
\begin{proof}
  Consider arbitrary $S''$ such that $S''$ is non-conflicting with
  $\config{S}{e} \parstepsto \config{S'}{e'}$ and $\lubstore{S'}{S''}
  \statuseq S$ and $\lubstore{S'}{S''} \neq \topS$.

  To show: $\config{\lubstore{S}{S''}}{e} \parstepsto
  \config{\lubstore{S'}{S''}}{e'}$.

  The proof is by cases on the rule of the reduction semantics by
  which $\config{S}{e}$ steps to $\config{S'}{e'}$.  Since
  $\config{S'}{e'} \neq \error$, we do not need to consider the {\sc
    E-Put-Err} rule.

  The assumption that $\lubstore{S'}{S''} \statuseq S$ is only needed
  in the {\sc E-Freeze-Final} and {\sc E-Freeze-Simple} cases.

  \begin{itemize}

    \item Case {\sc E-Beta}:

      Given: $\config{S}{\app{(\lam{x}{e})}{v}} \parstepsto
      \config{S}{\subst{e}{x}{v}}$.

      To show: $\config{\lubstore{S}{S''}}{\app{(\lam{x}{e})}{v}} \parstepsto
      \config{\lubstore{S}{S''}}{\subst{e}{x}{v}}$.

      Immediate by {\sc E-Beta}.

    \item Case {\sc E-New}:

      Given: $\config{S}{\NEW} \parstepsto
      \config{\extS{S}{l}{\bot}{\frozenfalse}}{l}$.

      To show: $\config{\lubstore{S}{S''}}{\NEW} \parstepsto
      \config{\lubstore{(\extS{S}{l}{\bot}{\frozenfalse})}{S''}}{l}$.

      By {\sc E-New}, we have that $\config{\lubstore{S}{S''}}{\NEW}
      \parstepsto
      \config{\extS{(\lubstore{S}{S''})}{l'}{\bot}{\frozenfalse}}{l'}$,
      where $l' \notin \dom{\lubstore{S}{S''}}$.

      By assumption, $S''$ is non-conflicting with $\config{S}{\NEW}
      \parstepsto \config{\extS{S}{l}{\bot}{\frozenfalse}}{l}$.
 
      Therefore $l \notin \dom{S''}$.

      From the side condition of {\sc E-New}, $l \notin \dom{S}$.

      Therefore $l \notin \dom{\lubstore{S}{S''}}$.

      Therefore, in
      $\config{\extS{(\lubstore{S}{S''})}{l'}{\bot}{\frozenfalse}}{l'}$,
      we can $\alpha$-rename $l'$ to $l$, resulting in
      $\config{\extS{(\lubstore{S}{S''})}{l}{\bot}{\frozenfalse}}{l}$.

      Therefore $\config{\lubstore{S}{S''}}{\NEW} \parstepsto
      \config{\extS{(\lubstore{S}{S''})}{l}{\bot}{\frozenfalse}}{l}$.

      Note that:
      \begin{align*}
        \extS{(\lubstore{S}{S''})}{l}{\bot}{\frozenfalse} &=
        \lubstore{\extS{S}{l}{\bot}{\frozenfalse}}{\extS{S''}{l}{\bot}{\frozenfalse}} \\
        &= \lubstore{\lubstore{S}{\store{\storebinding{l}{\bot}{\frozenfalse}}}}{\lubstore{S''}{\store{\storebinding{l}{\bot}{\frozenfalse}}}} \\
        &= \lubstore{\lubstore{S}{\store{\storebinding{l}{\bot}{\frozenfalse}}}}{S''} \\
        &= \lubstore{\extS{S}{l}{\bot}{\frozenfalse}}{S''}.
      \end{align*}
      Therefore $\config{\lubstore{S}{S''}}{\NEW} \parstepsto
      \config{\lubstore{\extS{S}{l}{\bot}{\frozenfalse}}{S''}}{l}$, as we were
      required to show.

    \item Case {\sc E-Put}:

      Given: $\config{S}{\putiexp{l}} \parstepsto
      \config{\extSRaw{S}{l}{u_{p_i}(p_1)}}{\unit}$.

      To show: $\config{\lubstore{S}{S''}}{\putiexp{l}{d_2}}
      \parstepsto
      \config{\lubstore{\extSRaw{S}{l}{u_{p_i}(p_1)}}{S''}}{\unit}$.

      We will first show that

      $\config{\lubstore{S}{S''}}{\putiexp{l}{d_2}} \parstepsto
      \config{\extSRaw{(\lubstore{S}{S''})}{l}{u_{p_i}(p_1)}}{\unit}$

      and then show why this is sufficient.

      We proceed by cases on $l$:

      \begin{itemize}
        \item $l \notin \dom{S''}$:

          By assumption, $\lubstore{\extSRaw{S}{l}{u_{p_i}(p_1)}}{S''}
          \neq \topS$.

          By Lemma~\ref{lem:monotonicity},
          $\leqstore{S}{\extSRaw{S}{l}{u_{p_i}(p_1)}}$.

          Hence $\lubstore{S}{S''} \neq \topS$.

          Therefore, by Definition~\ref{def:lubstore},
          $(\lubstore{S}{S''})(l) = S(l)$.

          From the premises of {\sc E-Put}, $S(l) = p_1$.

          Hence $(\lubstore{S}{S''})(l) = p_1$.

          From the premises of {\sc E-Put}, $u_{p_i}(p_1) \neq \topp$.

          Therefore, by {\sc E-Put}, we have:
          $\config{\lubstore{S}{S''}}{\putiexp{l}} \parstepsto
          \config{\extSRaw{(\lubstore{S}{S''})}{l}{u_{p_i}(p_1)}}{\unit}$.

        \item $l \in \dom{S''}$:

          By assumption, $\lubstore{\extSRaw{S}{l}{u_{p_i}(p_1)}}{S''} \neq
          \topS$.

          By Lemma~\ref{lem:monotonicity},
          $\leqstore{S}{\extSRaw{S}{l}{u_{p_i}(p_1)}}$.

          Hence $\lubstore{S}{S''} \neq \topS$.

          Therefore $(\lubstore{S}{S''})(l) = \lubp{S(l)}{S''(l)}$.

          From the premises of {\sc E-Put}, $S(l) = p_1$.
          
          Hence $(\lubstore{S}{S''})(l) = p'_1$, where $p_1 \leqp
          p'_1$.

          \TODO{From here forward, this subcase still needs to be
            fixed.}

          By assumption, $\lubstore{\extSRaw{S}{l}{p_2}}{S''} \neq
          \topS$.

          Therefore, by Definition~\ref{def:lubstore},
          $\lubp{p_2}{S''(l)} \neq \topp$.

          Note that:
          \begin{align*}
            \topp &\neq \lubp{p_2}{S''(l)} \\
            &= \lubp{\lubp{p_1}{\state{d_2}{\frozenfalse}}}{S''(l)} \\
            &= \lubp{\lubp{S(l)}{\state{d_2}{\frozenfalse}}}{S''(l)} \\
            &= \lubp{\lubp{S(l)}{S''(l)}}{\state{d_2}{\frozenfalse}} \\
            &= \lubp{(\lubstore{S}{S''})(l)}{\state{d_2}{\frozenfalse}} \\
            &= \lubp{p'_1}{\state{d_2}{\frozenfalse}} \\
            &= p'_2. \\
          \end{align*}
          Hence $p'_2 \neq \topp$.

          Hence $(\lubstore{S}{S''})(l) = p'_1$ and $p'_2 =
          \lubp{p'_1}{\state{d_2}{\frozenfalse}}$ and $p'_2 \neq
          \topp$.

          Therefore, by {\sc E-Put} we have:
          $\config{\lubstore{S}{S''}}{\putiexp{l}{d_2}} \parstepsto
          \config{\extSRaw{(\lubstore{S}{S''})}{l}{p'_2}}{\unit}$.

          \lk{If we really wanted to be pedantic here, we'd actually
            prove that the stores are equal.  I'm assuming that if I
            can show that $\extSRaw{(\lubstore{S}{S''})}{l}{p'_2}$ and
            $\extSRaw{(\lubstore{S}{S''})}{l}{p_2}$ bind $l$ to the
            same value, then it will be obvious that they're equal.}

          Note that:
          \begin{align*}
            (\extSRaw{(\lubstore{S}{S''})}{l}{p'_2})(l) &= \lubp{(\lubstore{S}{S''})(l)}{(\store{\storebindingRaw{l}{p'_2}})(l)} \\
            &= \lubp{p'_1}{p'_2} \\
            &= \lubp{p'_1}{\lubp{p'_1}{\state{d_2}{\frozenfalse}}} \\
            &= \lubp{p'_1}{\state{d_2}{\frozenfalse}}
          \end{align*}
          and
          \begin{align*}
            (\extSRaw{(\lubstore{S}{S''})}{l}{p_2})(l) &= \lubp{(\lubstore{S}{S''})(l)}{(\store{\storebindingRaw{l}{p_2}})(l)} \\
            &= \lubp{p'_1}{p_2} \\
            &= \lubp{p'_1}{\lubp{p_1}{\state{d_2}{\frozenfalse}}} \\
            &= \lubp{p'_1}{\state{d_2}{\frozenfalse}} & \textrm{(since $p_1 \leqp p'_1$).}
          \end{align*}
          Therefore $\extSRaw{(\lubstore{S}{S''})}{l}{p'_2} =
          \extSRaw{(\lubstore{S}{S''})}{l}{p_2}$.

          Therefore, $\config{\lubstore{S}{S''}}{\putiexp{l}{d_2}}
          \parstepsto
          \config{\extSRaw{(\lubstore{S}{S''})}{l}{p_2}}{\unit}$.
      \end{itemize}

      Note that:
      \begin{align*}
        \extSRaw{(\lubstore{S}{S''})}{l}{p_2} &= \lubstore{\extSRaw{S}{l}{p_2}}{\extSRaw{S''}{l}{p_2}} \\
        &= \lubstore{\lubstore{S}{\store{\storebindingRaw{l}{p_2}}}}{\lubstore{S''}{\store{\storebindingRaw{l}{p_2}}}} \\
        &= \lubstore{\lubstore{S}{\store{\storebindingRaw{l}{p_2}}}}{S''} \\
        &= \lubstore{\extSRaw{S}{l}{p_2}}{S''}.
      \end{align*}
      Therefore $\config{\lubstore{S}{S''}}{\putiexp{l}{d_2}}
      \parstepsto \config{\lubstore{\extSRaw{S}{l}{p_2}}{S''}}{\unit}$,
      as we were required to show.

    \item Case {\sc E-Get}:

      Given: $\config{S}{\getexp{l}{P}} \parstepsto \config{S}{p_2}$.

      To show: $\config{\lubstore{S}{S''}}{\getexp{l}{P}} \parstepsto
      \config{\lubstore{S}{S''}}{p_2}$.

      From the premises of {\sc E-Get}, $S(l) = p_1$ and $\incomp{P}$
      and $p_2 \in P$ and $p_2 \leqp p_1$.

      By assumption, $\lubstore{S}{S''} \neq \topS$.

      Hence $(\lubstore{S}{S''}) = p'_1$, where $p_1 \leqp p'_1$.

      By the transitivity of $\leqp$, $p_2 \leqp p'_1$.

      Hence, $S(l) = p'_1$ and $\incomp{P}$ and $p_2 \in P$ and $p_2
      \leqp p'_1$.

      Therefore, by {\sc E-Get},

      $\config{\lubstore{S}{S''}}{\getexp{l}{P}} \parstepsto
      \config{\lubstore{S}{S''}}{p_2}$,

      as we were required to show.

    \item Case {\sc E-Freeze-Init}:

      Given: $\config{S}{\freezeafter{l}{Q}{\lam{x}{e}}} \parstepsto
      \config{S}{\freezeafterfull{l}{Q}{\lam{x}{e}}{\setof{}}{\setof{}}}$.

      To show:
      $\config{\lubstore{S}{S''}}{\freezeafter{l}{Q}{\lam{x}{e}}}
      \parstepsto
      \config{\lubstore{S}{S''}}{\freezeafterfull{l}{Q}{\lam{x}{e}}{\setof{}}{\setof{}}}$.

      Immediate by {\sc E-Freeze-Init}.

    \item Case {\sc E-Spawn-Handler}:

      Given:

      $\config{S}{\freezeafterfull{l}{Q}{\lam{x}{e_0}}{\setof{e,
            \dots}}{H}} \parstepsto
      \config{S}{\freezeafterfull{l}{Q}{\lam{x}{e_0}}{\setof{\subst{e_0}{x}{d_2},
            e, \dots}} {\{d_2\}\cup H}}$.

      To show:

      $\config{\lubstore{S}{S''}}{\freezeafterfull{l}{Q}{\lam{x}{e_0}}{\setof{e,
            \dots}}{H}} \parstepsto
      \config{\lubstore{S}{S''}}{\freezeafterfull{l}{Q}{\lam{x}{e_0}}{\setof{\subst{e_0}{x}{d_2},
            e, \dots}} {\{d_2\}\cup H}}$.

      From the premises of {\sc E-Spawn-Handler}, $S(l) =
      \state{d_1}{\status_1}$ and $d_2 \userleq d_1$ and $d_2 \notin
      H$ and $d_2 \in Q$.

      By assumption, $\lubstore{S}{S''} \neq \topS$.

      Hence $(\lubstore{S}{S''})(l) = \state{d'_1}{\status'_1}$ where
      $\state{d_1}{\status_1} \leqp \state{d'_1}{\status'_1}$.

      By Definition~\ref{def:lattice-with-status-bits}, $d_1 \userleq
      d'_1$.

      By the transitivity of $\userleq$, $d_2 \userleq d'_1$.

      Hence $(\lubstore{S}{S''})(l) =
      \state{d'_1}{\status'_1}$ and $d_2 \userleq d'_1$ and $d_2 \notin
      H$ and $d_2 \in Q$.

      Therefore, by {\sc E-Spawn-Handler},

      $\config{\lubstore{S}{S''}}{\freezeafterfull{l}{Q}{\lam{x}{e_0}}{\setof{e,
            \dots}}{H}} \parstepsto
      \config{\lubstore{S}{S''}}{\freezeafterfull{l}{Q}{\lam{x}{e_0}}{\setof{\subst{e_0}{x}{d_2},
            e, \dots}} {\{d_2\}\cup H}}$,

      as we were required to show.

    \item Case {\sc E-Freeze-Final}:

      \lk{This case wouldn't work but for the $\lubstore{S'}{S''}
        \statuseq S$ requirement, which makes it a no-op freeze.}

      Given:
      $\config{S}{\freezeafterfull{l}{Q}{\lam{x}{e_0}}{\setof{v,
            \dots}}{H}} \parstepsto
      \config{\extS{S}{l}{d_1}{\frozentrue}}{d_1}$.

      To show:
      $\config{\lubstore{S}{S''}}{\freezeafterfull{l}{Q}{\lam{x}{e_0}}{\setof{v,
            \dots}}{H}} \parstepsto
      \config{\lubstore{\extS{S}{l}{d_1}{\frozentrue}}{S''}}{d_1}$.

      We will first show that

      $\config{\lubstore{S}{S''}}{\freezeafterfull{l}{Q}{\lam{x}{e_0}}{\setof{v,
            \dots}}{H}} \parstepsto
      \config{\extS{(\lubstore{S}{S''})}{l}{d_1}{\frozentrue}}{d_1}$

      and then show why this is sufficient.

      We proceed by cases on $l$:
      \begin{itemize}
      \item $l \notin \dom{S''}$:

        By assumption, $\lubstore{\extS{S}{l}{d_1}{\frozentrue}}{S''}
        \neq \topS$.

        By Lemma~\ref{lem:monotonicity},
        $\leqstore{S}{\extS{S}{l}{d_1}{\frozentrue}}$.

        Therefore $\lubstore{S}{S''} \neq \topS$.

        Hence, by Definition~\ref{def:lubstore},
        $(\lubstore{S}{S''})(l) = S(l)$.

        From the premises of {\sc E-Freeze-Final}, we have that $S(l)
        = \state{d_1}{\status_1}$.

        Hence $(\lubstore{S}{S''})(l) = \state{d_1}{\status_1}$.

        From the premises of {\sc E-Freeze-Final}, we have that
        $\forall{d_2} ~.~ ( {d_2 \userleq d_1 \land d_2 \in Q} \Rightarrow d_2 \in
        H)$.

        Therefore, by {\sc E-Freeze-Final}, we have that

        $\config{\lubstore{S}{S''}}{\freezeafterfull{l}{Q}{\lam{x}{e_0}}{\setof{v,
              \dots}}{H}} \parstepsto
        \config{\extS{(\lubstore{S}{S''})}{l}{d_1}{\frozentrue}}{d_1}$.


      \item $l \in \dom{S''}$:

        By assumption, $\lubstore{\extS{S}{l}{d_1}{\frozentrue}}{S''}
        \neq \topS$.

        By Lemma~\ref{lem:monotonicity},
        $\leqstore{S}{\extS{S}{l}{d_1}{\frozentrue}}$.

        Therefore $\lubstore{S}{S''} \neq \topS$.

        Hence, by Definition~\ref{def:lubstore},
        $(\lubstore{S}{S''})(l) = \lubp{S(l)}{S''(l)}$.

        From the premises of {\sc E-Freeze-Final}, we have that
        $S(l) = \state{d_1}{\status_1}$.

        By assumption, $\lubstore{\extS{S}{l}{d_1}{\frozentrue}}{S''}
        \statuseq S$.

        Therefore $\status_1 = \frozentrue$.

        Therefore $S(l) = \state{d_1}{\frozentrue}$.

        Therefore $(\lubstore{S}{S''})(l) =
        \lubp{\state{d_1}{\frozentrue}}{S''(l)}$.

        We proceed by cases on $S''(l)$:
        \begin{itemize}
        \item $S''(l) = \state{d_3}{\frozenfalse}$, where $d_3 \userleq d_1$:

          By Definition~\ref{def:lubp},
          $\lubp{\state{d_1}{\frozentrue}}{\state{d_3}{\frozenfalse}}
          = \state{d_1}{\frozentrue}$.

          Therefore $(\lubstore{S}{S''})(l) =
          \state{d_1}{\frozentrue}$.

          From the premises of {\sc E-Freeze-Final}, we have that
          $\forall{d_2} ~.~ ( {d_2 \userleq d_1 \land d_2 \in Q} \Rightarrow d_2 \in
          H)$.

          Therefore, by {\sc E-Freeze-Final}, we have that

          $\config{\lubstore{S}{S''}}{\freezeafterfull{l}{Q}{\lam{x}{e_0}}{\setof{v,
                \dots}}{H}} \parstepsto
          \config{\extS{(\lubstore{S}{S''})}{l}{d_1}{\frozentrue}}{d_1}$.

        \item $S''(l) = \state{d_3}{\frozenfalse}$, where $d_3 \nuserleq d_1$:

          By Definition~\ref{def:lubp},
          $\lubp{\state{d_1}{\frozentrue}}{\state{d_3}{\frozenfalse}}
          = \state{\top}{\frozenfalse}$.

          Therefore $\lubp{S(l)}{S''(l)} =
          \state{\top}{\frozenfalse}$.

          By Definition~\ref{def:lattice-with-status-bits},
          $\state{\top}{\frozenfalse} = \topp$.

          Therefore $\lubp{S(l)}{S''(l)} = \topp$.

          Therefore, by Definition~\ref{def:lubstore},
          $\lubstore{S}{S''} = \topS$.

          This is a contradiction.

          Therefore,

          $\config{\lubstore{S}{S''}}{\freezeafterfull{l}{Q}{\lam{x}{e_0}}{\setof{v,
                \dots}}{H}} \parstepsto
          \config{\extS{(\lubstore{S}{S''})}{l}{d_1}{\frozentrue}}{d_1}$.

        \item $S''(l) = \state{d_3}{\frozentrue}$, where $d_3 = d_1$:

          By Definition~\ref{def:lubp},
          $\lubp{\state{d_1}{\frozentrue}}{\state{d_3}{\frozentrue}} =
          \state{d_1}{\frozentrue}$.

          Therefore $(\lubstore{S}{S''})(l) = \state{d_1}{\frozentrue}$.

          From the premises of {\sc E-Freeze-Final}, we have that
          $\forall{d_2} ~.~ ( {d_2 \userleq d_1 \land d_2 \in Q} \Rightarrow d_2 \in
          H)$.

          Therefore, by {\sc E-Freeze-Final}, we have that

          $\config{\lubstore{S}{S''}}{\freezeafterfull{l}{Q}{\lam{x}{e_0}}{\setof{v,
                \dots}}{H}} \parstepsto
          \config{\extS{(\lubstore{S}{S''})}{l}{d_1}{\frozentrue}}{d_1}$.

        \item $S''(l) = \state{d_3}{\frozentrue}$, where $d_3 \neq d_1$:

          By Definition~\ref{def:lubp},
          $\lubp{\state{d_1}{\frozentrue}}{\state{d_3}{\frozentrue}}
          = \state{\top}{\frozenfalse}$.

          Therefore $\lubp{S(l)}{S''(l)} = \state{\top}{\frozenfalse}$.

          By Definition~\ref{def:lattice-with-status-bits},
          $\state{\top}{\frozenfalse} = \topp$.

          Therefore $\lubp{S(l)}{S''(l)} = \topp$.

          Therefore, by Definition~\ref{def:lubstore},
          $\lubstore{S}{S''} = \topS$.

          This is a contradiction.

          Therefore,

          $\config{\lubstore{S}{S''}}{\freezeafterfull{l}{Q}{\lam{x}{e_0}}{\setof{v,
                \dots}}{H}} \parstepsto
          \config{\extS{(\lubstore{S}{S''})}{l}{d_1}{\frozentrue}}{d_1}$.
        \end{itemize}
      \end{itemize}

      In each case we have shown that

      $\config{\lubstore{S}{S''}}{\freezeafterfull{l}{Q}{\lam{x}{e_0}}{\setof{v,
            \dots}}{H}} \parstepsto
      \config{\extS{(\lubstore{S}{S''})}{l}{d_1}{\frozentrue}}{d_1}$.

      Note that:
      \begin{align*}
        \extS{(\lubstore{S}{S''})}{l}{d_1}{\frozentrue} &=
        \lubstore{\extS{S}{l}{d_1}{\frozentrue}}{\extS{S''}{l}{d_1}{\frozentrue}} \\
        &= \lubstore{\lubstore{S}{\store{\storebinding{l}{d_1}{\frozentrue}}}}{\lubstore{S''}{\store{\storebinding{l}{d_1}{\frozentrue}}}} \\
        &= \lubstore{\lubstore{S}{\store{\storebinding{l}{d_1}{\frozentrue}}}}{S''} \\
        &= \lubstore{\extS{S}{l}{d_1}{\frozentrue}}{S''}.
      \end{align*}
      Therefore

      $\config{\lubstore{S}{S''}}{\freezeafterfull{l}{Q}{\lam{x}{e_0}}{\setof{v,
            \dots}}{H}} \parstepsto
      \config{\lubstore{\extS{S}{l}{d_1}{\frozentrue}}{S''}}{d_1}$,

      as we were required to show.

    \item Case {\sc E-Freeze-Simple}:

      Given: $\config{S}{\freeze{l}} \parstepsto
      \config{\extS{S}{l}{d_1}{\frozentrue}}{d_1}$.

      To show: $\config{\lubstore{S}{S''}}{\freeze{l}}
      \parstepsto
      \config{\lubstore{\extS{S}{l}{d_1}{\frozentrue}}{S''}}{d_1}$.

      We will first show that

      $\config{\lubstore{S}{S''}}{\freeze{l}} \parstepsto
      \config{\extS{(\lubstore{S}{S''})}{l}{d_1}{\frozentrue}}{d_1}$

      and then show why this is sufficient.

      We proceed by cases on $l$:
      \begin{itemize}
      \item $l \notin \dom{S''}$:

        By assumption, $\lubstore{\extS{S}{l}{d_1}{\frozentrue}}{S''}
        \neq \topS$.

        By Lemma~\ref{lem:monotonicity},
        $\leqstore{S}{\extS{S}{l}{d_1}{\frozentrue}}$.

        Therefore $\lubstore{S}{S''} \neq \topS$.

        Hence, by Definition~\ref{def:lubstore},
        $(\lubstore{S}{S''})(l) = S(l)$.

        From the premise of {\sc E-Freeze-Simple}, we have that
        $S(l) = \state{d_1}{\status_1}$.

        Therefore, by {\sc E-Freeze-Simple}, we have that

        $\config{\lubstore{S}{S''}}{\freeze{l}}
        \parstepsto
        \config{\extS{(\lubstore{S}{S''})}{l}{d_1}{\frozentrue}}{d_1}$.

      \item $l \in \dom{S''}$:

        By assumption, $\lubstore{\extS{S}{l}{d_1}{\frozentrue}}{S''}
        \neq \topS$.

        By Lemma~\ref{lem:monotonicity},
        $\leqstore{S}{\extS{S}{l}{d_1}{\frozentrue}}$.

        Therefore $\lubstore{S}{S''} \neq \topS$.

        Hence, by Definition~\ref{def:lubstore},
        $(\lubstore{S}{S''})(l) = \lubp{S(l)}{S''(l)}$.

        From the premise of {\sc E-Freeze-Simple}, we have that
        $S(l) = \state{d_1}{\status_1}$.

        By assumption, $\lubstore{\extS{S}{l}{d_1}{\frozentrue}}{S''}
        \statuseq S$.

        Therefore $\status_1 = \frozentrue$.

        Therefore $(\lubstore{S}{S''})(l) =
        \lubp{\state{d_1}{\frozentrue}}{S''(l)}$.
        
        We proceed by cases on $S''(l)$:
        \begin{itemize}
        \item $S''(l) = \state{d_2}{\frozenfalse}$, where $d_2 \userleq d_1$:

          By Definition~\ref{def:lubp},
          $\lubp{\state{d_1}{\frozentrue}}{\state{d_2}{\frozenfalse}} =
          \state{d_1}{\frozentrue}$.

          Therefore $(\lubstore{S}{S''})(l) =
          \state{d_1}{\frozentrue}$.

          Therefore, by {\sc E-Freeze-Simple}, we have that

          $\config{\lubstore{S}{S''}}{\freeze{l}}
          \parstepsto
          \config{\extS{(\lubstore{S}{S''})}{l}{d_1}{\frozentrue}}{d_1}$.

        \item $S''(l) = \state{d_2}{\frozenfalse}$, where $d_2 \nuserleq d_1$:

          By Definition~\ref{def:lubp},
          $\lubp{\state{d_1}{\frozentrue}}{\state{d_2}{\frozenfalse}}
          = \state{\top}{\frozenfalse}$.

          By Definition~\ref{def:lattice-with-status-bits},
          $\state{\top}{\frozenfalse} = \topp$.

          Therefore $\lubp{S(l)}{S''(l)} = \topp$.

          Therefore, by Definition~\ref{def:lubstore},
          $\lubstore{S}{S''} = \topS$.

          This is a contradiction.

          Therefore,

          $\config{\lubstore{S}{S''}}{\freeze{l}}
          \parstepsto
          \config{\extS{(\lubstore{S}{S''})}{l}{d_1}{\frozentrue}}{d_1}$.

        \item $S''(l) = \state{d_2}{\frozentrue}$, where $d_2 = d_1$:

          Therefore $(\lubstore{S}{S''})(l) =
          \lubp{\state{d_1}{\frozentrue}}{\state{d_2}{\frozentrue}}$.

          By Definition~\ref{def:lubp},
          $\lubp{\state{d_1}{\frozentrue}}{\state{d_2}{\frozentrue}} =
          \state{d_1}{\frozentrue}$.

          Therefore $(\lubstore{S}{S''})(l) =
          \state{d_1}{\frozentrue}$.

          Therefore, by {\sc E-Freeze-Simple}, we have that

          $\config{\lubstore{S}{S''}}{\freeze{l}}
          \parstepsto
          \config{\extS{(\lubstore{S}{S''})}{l}{d_1}{\frozentrue}}{d_1}$.

        \item $S''(l) = \state{d_2}{\frozentrue}$, where $d_2 \neq d_1$:

          By Definition~\ref{def:lubp},
          $\lubp{\state{d_1}{\frozentrue}}{\state{d_2}{\frozentrue}}
          = \state{\top}{\frozenfalse}$.

          By Definition~\ref{def:lattice-with-status-bits},
          $\state{\top}{\frozenfalse} = \topp$.

          Therefore $\lubp{S(l)}{S''(l)} = \topp$.

          Therefore, by Definition~\ref{def:lubstore},
          $\lubstore{S}{S''} = \topS$.

          This is a contradiction.

          Therefore,

          $\config{\lubstore{S}{S''}}{\freeze{l}}
          \parstepsto
          \config{\extS{(\lubstore{S}{S''})}{l}{d_1}{\frozentrue}}{d_1}$.
        \end{itemize}
      \end{itemize}

      In each case we have shown that

      $\config{\lubstore{S}{S''}}{\freeze{l}} \parstepsto
      \config{\extS{(\lubstore{S}{S''})}{l}{d_1}{\frozentrue}}{d_1}$.

      Note that:
      \begin{align*}
        \extS{(\lubstore{S}{S''})}{l}{d_1}{\frozentrue} &=
        \lubstore{\extS{S}{l}{d_1}{\frozentrue}}{\extS{S''}{l}{d_1}{\frozentrue}} \\
        &= \lubstore{\lubstore{S}{\store{\storebinding{l}{d_1}{\frozentrue}}}}{\lubstore{S''}{\store{\storebinding{l}{d_1}{\frozentrue}}}} \\
        &= \lubstore{\lubstore{S}{\store{\storebinding{l}{d_1}{\frozentrue}}}}{S''} \\
        &= \lubstore{\extS{S}{l}{d_1}{\frozentrue}}{S''}.
      \end{align*}
      Therefore
      $\config{\lubstore{S}{S''}}{\freeze{l}}
      \parstepsto
      \config{\lubstore{\extS{S}{l}{d_1}{\frozentrue}}{S''}}{d_1}$,
      as we were required to show.
  \end{itemize}
\end{proof}


\section{Proof of Lemma~\ref{lem:strong-local-quasi-confluence}}\label{section:strong-local-quasi-confluence-proof}
\begin{proof}
  Suppose $\conf \ctxstepsto \conf_a$ and $\conf \ctxstepsto \conf_b$.

  We have to show that either there exist $\conf_c, i, j, \pi$ such
  that $\conf_a \ctxstepsto^i \conf_c$ and $\pi(\conf_b) \ctxstepsto^j
  \conf_c$ and $i \leq 1$ and $j \leq 1$, or that $\conf_a \ctxstepsto
  \error$ or $\conf_b \ctxstepsto \error$.

  By inspection of the operational semantics, it must be the case that
  $\conf$ steps to $\conf_a$ by the {\sc E-Eval-Ctxt} rule.

  Let $\conf = \config{S}{\evalctxt{E_a}{e_{a_1}}}$ and let $\conf_a =
  \config{S_a}{\evalctxt{E_a}{e_{a_2}}}$.

  Likewise, it must be the case that $\conf$ steps to $\conf_b$ by the
  {\sc E-Eval-Ctxt} rule.

  Let $\conf = \config{S}{\evalctxt{E_b}{e_{b_1}}}$ and let $\conf_b =
  \config{S_b}{\evalctxt{E_b}{e_{b_2}}}$.

  Note that $\conf = \config{S}{\evalctxt{E_a}{e_{a_1}}} =
  \config{S}{\evalctxt{E_b}{e_{b_1}}}$, and so
  $\evalctxt{E_a}{e_{a_1}} = \evalctxt{E_b}{e_{b_1}}$, but $E_a$ and
  $E_b$ may differ and $e_{a_1}$ and $e_{b_1}$ may differ.

  First, consider the possibility that $E_a = E_b$ (and $e_{a_1} =
  e_{b_1}$).

  Since $\config{S}{\evalctxt{E_a}{e_{a_1}}} \ctxstepsto
  \config{S_a}{\evalctxt{E_a}{e_{a_2}}}$ by {\sc E-Eval-Ctxt} and
  $\config{S}{\evalctxt{E_b}{e_{b_1}}} \ctxstepsto
  \config{S_b}{\evalctxt{E_b}{e_{b_2}}}$ by {\sc E-Eval-Ctxt}, we have
  from the premise of {\sc E-Eval-Ctxt} that $\config{S}{e_{a_1}}
  \parstepsto \config{S_a}{e_{a_2}}$ and $\config{S}{e_{b_1}}
  \parstepsto \config{S_b}{e_{b_2}}$.

  But then, since $e_{a_1} = e_{b_1}$, by Internal Determinism
  (Lemma~\ref{lem:internal-determinism}) there is a permutation $\pi'$
  such that $\config{S_a}{e_{a_2}} = \pi'(\config{S_b}{e_{b_2}})$,
  modulo choice of events.

  We have two cases:

  \begin{itemize}
  \item In the case where the steps $\conf \ctxstepsto \conf_a$ and
    $\conf \ctxstepsto \conf_b$ are both by {\sc E-Spawn-Handler} and
    they handle different events $d_2$ and $d'_2$, then we can satisfy
    the proof by choosing the final configuration $\conf_c$ as the
    configuration where both $d_2$ and $d'_2$ have been handled.

    Both $\conf_a$ and $\conf_b$ can step to this configuration by
    {\sc E-Spawn-Handler}: if the step from $\conf$ to $\conf_a$
    handles $d_2$ then the step from $\conf_a$ to $\conf_c$ handles
    $d'_2$, while if the step from $\conf$ to $\conf_b$ handles $d'_2$
    then the step from $\conf_b$ to $\conf_c$ handles $d_2$.

    The store in the final configuration is $S_a$ or $S_b$, which are
    equal because {\sc E-Spawn-Handler} does not affect the store, and
    we can satisfy the proof by choosing $i = 1$ and $j = 0$ and $\pi
    = \id$.

  \item Otherwise, we can satisfy the proof by choosing $\conf_c =
    \config{S_a}{e_{a_2}}$ and $i = 0$ and $j = 0$ and $\pi = \id$.
  \end{itemize}

  The rest of this proof deals with the more interesting case in which
  $E_a \neq E_b$ (and $e_{a_1} \neq e_{b_1}$).

  Since $\config{S}{\evalctxt{E_a}{e_{a_1}}} \ctxstepsto
  \config{S_a}{\evalctxt{E_a}{e_{a_2}}}$ and
  $\config{S}{\evalctxt{E_b}{e_{b_1}}} \ctxstepsto
  \config{S_b}{\evalctxt{E_b}{e_{b_2}}}$ and $\evalctxt{E_a}{e_{a_1}}
  = \evalctxt{E_b}{e_{b_1}}$, and since $E_a \neq E_b$, we have from
  Lemma~\ref{lem:locality} (Locality) that there exist evaluation
  contexts $E'_a$ and $E'_b$ such that:

  \begin{itemize}
  \item $\evalctxt{E'_a}{e_{a_1}} = \evalctxt{E_b}{e_{b_2}}$, and
  \item $\evalctxt{E'_b}{e_{b_1}} = \evalctxt{E_a}{e_{a_2}}$, and
  \item $\evalctxt{E'_a}{e_{a_2}} =
    \evalctxt{E'_b}{e_{b_2}}$.
  \end{itemize}

  In some of the cases that follow, we will choose $\conf_c = \error$,
  and in some we will prove that one of $\conf_a$ or $\conf_b$ steps
  to $\error$.

  In most cases, however, our approach will be to show that there
  exist $S', i, j, \pi$ such that:
  \begin{itemize}
  \item $\config{S_a}{\evalctxt{E_a}{e_{a_2}}} \ctxstepsto^i
    \config{S'}{\evalctxt{E'_a}{e_{a_2}}}$, and
  \item $\pi(\config{S_b}{\evalctxt{E_b}{e_{b_2}}}) \ctxstepsto^j
    \config{S'}{\evalctxt{E'_a}{e_{a_2}}}$.
  \end{itemize}
  Since $\evalctxt{E'_a}{e_{a_1}} = \evalctxt{E_b}{e_{b_2}}$,
  $\evalctxt{E'_b}{e_{b_1}} = \evalctxt{E_a}{e_{a_2}}$, and
  $\evalctxt{E'_a}{e_{a_2}} = \evalctxt{E'_b}{e_{b_2}}$, it suffices
  to show that:
  \begin{itemize}
  \item $\config{S_a}{\evalctxt{E'_b}{e_{b_1}}} \ctxstepsto^i
    \config{S'}{\evalctxt{E'_b}{e_{b_2}}}$, and
  \item $\pi(\config{S_b}{\evalctxt{E'_a}{e_{a_1}}}) \ctxstepsto^j
    \config{S'}{\evalctxt{E'_a}{e_{a_2}}}$.
  \end{itemize}
  From the premise of {\sc E-Eval-Ctxt}, we have that
  $\config{S}{e_{a_1}} \parstepsto \config{S_a}{e_{a_2}}$ and
  $\config{S}{e_{b_1}} \parstepsto \config{S_b}{e_{b_2}}$.

  We proceed by case analysis on the rule by which
  $\config{S}{e_{a_1}}$ steps to $\config{S_a}{e_{a_2}}$.

  Since the only way an $\error$ configuration can arise is by the
  {\sc E-Put-Err} rule, we can assume in all other cases that $\conf_a
  \neq \error$.

  \begin{enumerate}
  \item Case {\sc E-Beta}: We have $S_a = S$.

    We proceed by case analysis on the rule by which
    $\config{S}{e_{b_1}}$ steps to $\config{S_b}{e_{b_2}}$.

    Since the only way an $\error$ configuration can arise is by the
    {\sc E-Put-Err} rule, we can assume in all other cases that
    $\conf_b \neq \error$.
    \begin{enumerate}
    \item \label{slqc-beta-beta}Case {\sc E-Beta}: We have $S_a = S$
      and $S_b = S$.

      Choose $S' = S = S_a = S_b$, $i = 1$, $j = 1$, and $\pi = \id$.

      We have to show that:
      \begin{itemize}
      \item $\config{S}{\evalctxt{E'_b}{e_{b_1}}} \ctxstepsto
        \config{S_a}{\evalctxt{E'_b}{e_{b_2}}}$, and
      \item $\config{S}{\evalctxt{E'_a}{e_{a_1}}} \ctxstepsto
        \config{S_b}{\evalctxt{E'_a}{e_{a_2}}}$, 
      \end{itemize}

      both of which follow immediately from $\config{S}{e_{a_1}}
      \parstepsto \config{S_a}{e_{a_2}}$ and $\config{S}{e_{b_1}}
      \parstepsto \config{S_b}{e_{b_2}}$ and {\sc E-Eval-Ctxt}.

    \item \label{slqc-beta-new}Case {\sc E-New}: We have $S_a = S$ and
      $S_b = \extS{S}{l}{\bot}{\frozenfalse}$.

      Choose $S' = S_b$, $i = 1$, $j = 1$, and $\pi = \id$.

      We have to show that:
      \begin{itemize}
      \item $\config{S}{\evalctxt{E'_b}{e_{b_1}}} \ctxstepsto
        \config{S_b}{\evalctxt{E'_b}{e_{b_2}}}$, and
      \item $\config{S_b}{\evalctxt{E'_a}{e_{a_1}}} \ctxstepsto
        \config{S_b}{\evalctxt{E'_a}{e_{a_2}}}$.
      \end{itemize}

      The first of these follows immediately from $\config{S}{e_{b_1}}
      \parstepsto \config{S_b}{e_{b_2}}$ and {\sc E-Eval-Ctxt}.

      For the second, consider that $S_b =
      \extS{S}{l}{\bot}{\frozenfalse} = U_S(S)$, where $U_S$ is the
      store update operation that acts as the identity on the contents
      of all existing locations, and adds the binding
      $\storebinding{l}{\bot}{\frozenfalse}$ if no binding for $l$
      exists.

      Note that:
      \begin{itemize}
      \item $U_S$ is non-conflicting with $\config{S}{e_{a_1}}
        \parstepsto \config{S_a}{e_{a_2}}$, since no locations are
        allocated in the transition;
      \item $U_S(S_a) \neq \topS$, since $U_S(S_a) = U_S(S) = S_b$
        and we know that $\conf_b \neq \error$; and
      \item $U_S$ is freeze-safe with $\config{S}{e_{a_1}}
        \parstepsto \config{S_a}{e_{a_2}}$, since $S_a = S$, so
        there are no locations whose contents differ in status
        between them.
      \end{itemize}

      Therefore, by Lemma~\ref{lem:generalized-independence}
      (Generalized Independence), we have that

      $\config{U_S(S)}{e_{a_1}} \parstepsto
      \config{U_S(S_a)}{e_{a_2}}$.

      Hence $\config{S_b}{e_{a_1}} \parstepsto \config{S_b}{e_{a_2}}$.

      By {\sc E-Eval-Ctxt}, it follows that
      $\config{S_b}{\evalctxt{E'_a}{e_{a_1}}} \ctxstepsto
      \config{S_b}{\evalctxt{E'_a}{e_{a_2}}}$,
      as we were required to show.

    \item \label{slqc-beta-put}Case {\sc E-Put}: We have $S_a = S$ and
      $S_b = \extSRaw{S}{l}{u_{p_i}(p_1)}$.

      Choose $S' = S_b$, $i = 1$, $j = 1$, and $\pi = \id$.

      We have to show that:
      \begin{itemize}
      \item $\config{S}{\evalctxt{E'_b}{e_{b_1}}} \ctxstepsto
        \config{S_b}{\evalctxt{E'_b}{e_{b_2}}}$, and
      \item $\config{S_b}{\evalctxt{E'_a}{e_{a_1}}} \ctxstepsto
        \config{S_b}{\evalctxt{E'_a}{e_{a_2}}}$.
      \end{itemize}

      The first of these follows immediately from $\config{S}{e_{b_1}}
      \parstepsto \config{S_b}{e_{b_2}}$ and {\sc E-Eval-Ctxt}.

      For the second, consider that $S_b = U_S(S)$, where $U_S$ is the
      store update operation that applies $u_{p_i}$ to the contents of
      $l$ and acts as the identity on all other locations.

      Note that:
      \begin{itemize}
      \item $U_S$ is non-conflicting with $\config{S}{e_{a_1}}
        \parstepsto \config{S_a}{e_{a_2}}$, since no locations are
        allocated in the transition;
      \item $U_S(S_a) \neq \topS$, since $U_S(S_a) = U_S(S) = S_b$
        and we know that $\conf_b \neq \error$; and
      \item $U_S$ is freeze-safe with $\config{S}{e_{a_1}}
        \parstepsto \config{S_a}{e_{a_2}}$, since $S_a = S$, so
        there are no locations whose contents differ in status
        between them.
      \end{itemize}

      Therefore, by Lemma~\ref{lem:generalized-independence}
      (Generalized Independence), we have that

      $\config{U_S(S)}{e_{a_1}} \parstepsto
      \config{U_S(S_a)}{e_{a_2}}$.

      Hence $\config{S_b}{e_{a_1}} \parstepsto \config{S_b}{e_{a_2}}$.

      By {\sc E-Eval-Ctxt}, it follows that
      $\config{S_b}{\evalctxt{E'_a}{e_{a_1}}} \ctxstepsto
      \config{S_b}{\evalctxt{E'_a}{e_{a_2}}}$, as we were required to
      show.

    \item \label{slqc-beta-put-err}Case {\sc E-Put-Err}: We have $S_a
      = S$ and $\config{S_b}{e_{b_2}} = \error$, and so we choose
      $\conf_c = \error$, $i = 1$, $j = 0$, and $\pi = \id$.

      We have to show that:
      \begin{itemize}
      \item $\config{S}{\evalctxt{E'_b}{e_{b_1}}} \ctxstepsto \error$,
        and
      \item $\config{S_b}{\evalctxt{E'_a}{e_{a_1}}} = \error$.
      \end{itemize}

      The second of these is immediately true because since
      $\config{S_b}{e_{b_2}} = \error$, $S_b = \topS$, and so
      $\config{S_b}{\evalctxt{E'_a}{e_{a_1}}}$ is equal to $\error$ as
      well.

      For the first, observe that $\config{S}{e_{b_1}} \parstepsto
      \config{S_b}{e_{b_2}}$, hence by {\sc E-Eval-Ctxt},
      $\config{S}{\evalctxt{E'_b}{e_{b_1}}} \ctxstepsto
      \config{S_b}{\evalctxt{E'_b}{e_{b_2}}}$.

      But $S_b = \topS$, so $\config{S_b}{\evalctxt{E'_b}{e_{b_2}}}$
      is equal to $\error$, and so
      $\config{S}{\evalctxt{E'_b}{e_{b_1}}} \ctxstepsto \error$, as
      required.

    \item \label{slqc-beta-get}Case {\sc E-Get}: Similar to
      case~\ref{slqc-beta-beta}, since $S_a = S$ and $S_b = S$.
    \item \label{slqc-beta-freeze-init}Case {\sc E-Freeze-Init}:
      Similar to case~\ref{slqc-beta-beta}, since $S_a = S$ and $S_b =
      S$.
    \item \label{slqc-beta-spawn-handler}Case {\sc E-Spawn-Handler}:
      Similar to case~\ref{slqc-beta-beta}, since $S_a = S$ and $S_b =
      S$.
    \item \label{slqc-beta-freeze-final}Case {\sc E-Freeze-Final}: We
      have $S_a = S$ and $S_b = \extS{S}{l}{d_1}{\frozentrue}$.

      Choose $S' = S_b$, $i = 1$, $j = 1$, and $\pi = \id$.

      We have to show that:
      \begin{itemize}
      \item $\config{S}{\evalctxt{E'_b}{e_{b_1}}} \ctxstepsto
        \config{S_b}{\evalctxt{E'_b}{e_{b_2}}}$, and
      \item $\config{S_b}{\evalctxt{E'_a}{e_{a_1}}} \ctxstepsto
        \config{S_b}{\evalctxt{E'_a}{e_{a_2}}}$.
      \end{itemize}

      The first of these follows immediately from $\config{S}{e_{b_1}}
      \parstepsto \config{S_b}{e_{b_2}}$ and {\sc E-Eval-Ctxt}.

      For the second, note that $S_b = U_S(S)$, where $U_S$ is the
      store update operation that freezes the contents of $l$ and acts
      as the identity on the contents of all other locations.

      Note that:
      \begin{itemize}
      \item $U_S$ is non-conflicting with $\config{S}{e_{a_1}}
        \parstepsto \config{S_a}{e_{a_2}}$, since no locations are
        allocated in the transition;
      \item $U_S(S_a) \neq \topS$, since $U_S(S_a) = U_S(S) = S_b$
        and we know that $\conf_b \neq \error$; and
      \item $U_S$ is freeze-safe with $\config{S}{e_{a_1}}
        \parstepsto \config{S_a}{e_{a_2}}$, since $S_a = S$, so
        there are no locations whose contents differ in status
        between them.
      \end{itemize}

      Therefore, by Lemma~\ref{lem:generalized-independence}
      (Generalized Independence), we have that

      $\config{U_S(S)}{e_{a_1}} \parstepsto
      \config{U_S(S_a)}{e_{a_2}}$.

      Hence $\config{S_b}{e_{a_1}} \parstepsto \config{S_b}{e_{a_2}}$.

      By {\sc E-Eval-Ctxt}, it follows that
      $\config{S_b}{\evalctxt{E'_a}{e_{a_1}}} \ctxstepsto
      \config{S_b}{\evalctxt{E'_a}{e_{a_2}}}$, as we were required to
      show.

    \item \label{slqc-beta-freeze-simple}Case {\sc E-Freeze-Simple}:
      Similar to case~\ref{slqc-beta-freeze-final}, since $S_b =
      \extS{S}{l}{d_1}{\frozentrue}$.

    \end{enumerate}
  \item Case {\sc E-New}: We have $S_a = \extS{S}{l}{\bot}{\frozenfalse}$.

    We proceed by case analysis on the rule by which
    $\config{S}{e_{b_1}}$ steps to $\config{S_b}{e_{b_2}}$.

    Since the only way an $\error$ configuration can arise is by the
    {\sc E-Put-Err} rule, we can assume in all other cases that
    $\conf_b \neq \error$.
    \begin{enumerate}
    \item \label{slqc-new-beta}Case {\sc E-Beta}: By symmetry with case~\ref{slqc-beta-new}.
    \item \label{slqc-new-new}Case {\sc E-New}: We have $S_a =
      \extS{S}{l}{\bot}{\frozenfalse}$ and $S_b =
      \extS{S}{l'}{\bot}{\frozenfalse}$.

      Now consider whether $l = l'$:
      \begin{itemize}
      \item If $l \neq l'$:

        Choose $S' =
        \extS{\extS{S}{l'}{\bot}{\frozenfalse}}{l}{\bot}{\frozenfalse}$,
        $i = 1$, $j = 1$, and $\pi = \id$.

        We have to show that:
        \begin{itemize}
        \item $\config{S_a}{\evalctxt{E'_b}{e_{b_1}}} \ctxstepsto
          \config{\extS{\extS{S}{l'}{\bot}{\frozenfalse}}{l}{\bot}{\frozenfalse}}{\evalctxt{E'_b}{e_{b_2}}}$,
          and
        \item $\config{S_b}{\evalctxt{E'_a}{e_{a_1}}} \ctxstepsto
          \config{\extS{\extS{S}{l'}{\bot}{\frozenfalse}}{l}{\bot}{\frozenfalse}}{\evalctxt{E'_a}{e_{a_2}}}$.
        \end{itemize}

        For the first of these, consider that $S_a =
        \extS{S}{l}{\bot}{\frozenfalse} = U_S(S)$, where $U_S$ is
        the store update operation that acts as the identity on the
        contents of all existing locations, and adds the binding
        $\storebinding{l}{\bot}{\frozenfalse}$ if no binding for $l$
        exists.

        Note that:
        \begin{itemize}
        \item $U_S$ is non-conflicting with $\config{S}{e_{b_1}}
          \parstepsto \config{S_b}{e_{b_2}}$, since the only
          location allocated in the transition is $l'$, and $l
          \neq l'$ in this case;
        \item $U_S(S_b) \neq \topS$, since $U_S(S_b) =
          \extS{\extS{S}{l'}{\bot}{\frozenfalse}}{l}{\bot}{\frozenfalse}$
          and we know $S \neq \topS$ and the addition of new
          bindings $\storebinding{l}{\bot}{\frozenfalse}$ and
          $\storebinding{l'}{\bot}{\frozenfalse}$ cannot cause it to
          become $\topS$; and
        \item $U_S$ is freeze-safe with $\config{S}{e_{b_1}}
          \parstepsto \config{S_b}{e_{b_2}}$, since $S_b =
          \extS{S}{l'}{\bot}{\frozenfalse}$ and $l' \notin \dom{S}$,
          so there are no locations whose contents differ in status
          between $S$ and $S_b$.
        \end{itemize}

        Therefore, by Lemma~\ref{lem:generalized-independence}
        (Generalized Independence), we have that

        $\config{U_S(S)}{e_{b_1}} \parstepsto
        \config{U_S(S_b)}{e_{b_2}}$.

        Hence $\config{\extS{S}{l}{\bot}{\frozenfalse}}{e_{b_1}}
        \parstepsto
        \config{\extS{S_b}{l}{\bot}{\frozenfalse}}{e_{b_2}}$.

        By {\sc E-Eval-Ctxt} it follows that

        $\config{\extS{S}{l}{\bot}{\frozenfalse}}{\evalctxt{E'_b}{e_{b_1}}}
        \parstepsto
        \config{\extS{S_b}{l}{\bot}{\frozenfalse}}{\evalctxt{E'_b}{e_{b_2}}}$,
        which, since $S_b = \extS{S}{l'}{\bot}{\frozenfalse}$, is what
        we were required to show.

        The argument for the second is symmetrical.

      \item If $l = l'$:

        In this case, observe that we do \emph{not} want the
        expression in the final configuration to be
        $\evalctxt{E'_a}{e_{a_2}}$ (nor its equivalent,
        $\evalctxt{E'_b}{e_{b_2}}$).

        The reason for this is that $\evalctxt{E'_a}{e_{a_2}}$
        contains both occurrences of $l$.

        Rather, we want both configurations to step to a configuration
        in which exactly one occurrence of $l$ has been renamed to a
        fresh location $l''$.

        Let $l''$ be a location such that $l'' \notin \dom{S}$ and
        $l'' \neq l$ (and hence $l'' \neq l'$, as well).

        Then choose $S' =
        \extS{\extS{S}{l''}{\bot}{\frozenfalse}}{l}{\bot}{\frozenfalse}$,
        $i = 1$, $j = 1$, and $\pi = \setof{(l, l'')}$.

        Either
        $\config{\extS{\extS{S}{l''}{\bot}{\frozenfalse}}{l}{\bot}{\frozenfalse}}{\evalctxt{E'_a}{\pi(e_{a_2})}}$
        or
        $\config{\extS{\extS{S}{l''}{\bot}{\frozenfalse}}{l}{\bot}{\frozenfalse}}{\evalctxt{E'_b}{\pi(e_{b_2})}}$
        would work as a final configuration; we choose

        $\config{\extS{\extS{S}{l''}{\bot}{\frozenfalse}}{l}{\bot}{\frozenfalse}}{\evalctxt{E'_b}{\pi(e_{b_2})}}$.

        We have to show that:
        \begin{itemize}
        \item $\config{S_a}{\evalctxt{E'_b}{e_{b_1}}} \ctxstepsto
          \config{\extS{\extS{S}{l''}{\bot}{\frozenfalse}}{l}{\bot}{\frozenfalse}}{\evalctxt{E'_b}{\pi(e_{b_2})}}$,
          and
        \item $\pi(\config{S_b}{\evalctxt{E'_a}{e_{a_1}}})
          \ctxstepsto
          \config{\extS{\extS{S}{l''}{\bot}{\frozenfalse}}{l}{\bot}{\frozenfalse}}{\evalctxt{E'_b}{\pi(e_{b_2})}}$.
        \end{itemize}

        For the first of these, since $\config{S}{e_{b_1}}
        \parstepsto \config{S_b}{e_{b_2}}$, we have by
        Lemma~\ref{lem:permutability} (Permutability) that
        $\pi(\config{S}{e_{b_1}}) \parstepsto
        \pi(\config{S_b}{e_{b_2}})$.

        Since $\pi = \setof{(l, l'')}$, but $l \notin S$ (from the
        side condition on {\sc E-New}), we have that
        $\pi(\config{S}{e_{b_1}}) = \config{S}{e_{b_1}}$.

        Since $\config{S_b}{e_{b_2}} =
        \config{\extS{S}{l'}{\bot}{\frozenfalse}}{l'}$, and $l = l'$,
        we have that $\pi(\config{S_b}{e_{b_2}}) =
        \config{\extS{S}{l''}{\bot}{\frozenfalse}}{\pi(e_{b_2})}$.

        Hence $\config{S}{e_{b_1}} \parstepsto
        \config{\extS{S}{l''}{\bot}{\frozenfalse}}{\pi(e_{b_2})}$.

        Let $U_S$ be the store update operation that acts as the
        identity on the contents of all existing locations, and adds
        the binding $\storebinding{l}{\bot}{\frozenfalse}$ if no
        binding for $l$ exists.

        Note that:
        \begin{itemize}
        \item $U_S$ is non-conflicting with $\config{S}{e_{b_1}}
          \parstepsto
          \config{\extS{S}{l''}{\bot}{\frozenfalse}}{\pi(e_{b_2})}$,
          since the only location allocated in the transition is
          $l''$;
        \item $U_S(\extS{S}{l''}{\bot}{\frozenfalse}) \neq \topS$,
          since $U_S(\extS{S}{l''}{\bot}{\frozenfalse}) = \\
          \extS{\extS{S}{l''}{\bot}{\frozenfalse}}{l}{\bot}{\frozenfalse}$
          and we know $S \neq \topS$ and the addition of new
          bindings $\storebinding{l}{\bot}{\frozenfalse}$ and
          $\storebinding{l''}{\bot}{\frozenfalse}$ cannot cause it
          to become $\topS$; and
        \item $U_S$ is freeze-safe with $\config{S}{e_{b_1}}
          \parstepsto
          \config{\extS{S}{l''}{\bot}{\frozenfalse}}{\pi(e_{b_2})}$,
          since $l'' \notin \dom{S}$, so there are no locations
          whose contents differ in status between $S$ and
          $\extS{S}{l''}{\bot}{\frozenfalse}$.
        \end{itemize}

        Therefore, by Lemma~\ref{lem:generalized-independence}
        (Generalized Independence), we have that

        $\config{U_S(S)}{e_{b_1}} \parstepsto
        \config{U_S(\extS{S}{l''}{\bot}{\frozenfalse})}{\pi(e_{b_2})}$.

        Hence $\config{\extS{S}{l}{\bot}{\frozenfalse}}{e_{b_1}}
        \parstepsto
        \config{\extS{\extS{S}{l''}{\bot}{\frozenfalse}}{l}{\bot}{\frozenfalse}}{\pi(e_{b_2})}$.

        By {\sc E-Eval-Ctxt} it follows that

        $\config{\extS{S}{l}{\bot}{\frozenfalse}}{\evalctxt{E'_b}{e_{b_1}}}
        \parstepsto
        \config{\extS{\extS{S}{l''}{\bot}{\frozenfalse}}{l}{\bot}{\frozenfalse}}{\evalctxt{E'_b}{\pi(e_{b_2})}}$,

        which, since $\extS{S}{l}{\bot}{\frozenfalse} = S_a$, is what
        we were required to show.

        For the second, observe that since $S_b =
        \extS{S}{l}{\bot}{\frozenfalse}$, we have that $\pi(S_b) =
        \extS{S}{l''}{\bot}{\frozenfalse}$.

        Also, since $l$ does not occur in $e_{a_1}$, we have that
        $\pi(\evalctxt{E'_a}{e_{a_1}}) =
        \evalctxt{(\pi(E'_a))}{e_{a_1}}$.

        Hence we have to show that

        $\config{\extS{S}{l''}{\bot}{\frozenfalse}}{\evalctxt{(\pi(E'_a))}{e_{a_1}}}
        \ctxstepsto \\
        \config{\extS{\extS{S}{l''}{\bot}{\frozenfalse}}{l}{\bot}{\frozenfalse}}{\evalctxt{E'_b}{\pi(e_{b_2})}}$.

        Let $U_S$ be the store update operation that acts as the
        identity on the contents of all existing locations, and adds
        the binding $\storebinding{l''}{\bot}{\frozenfalse}$ if no
        binding for $l''$ exists.

        Note that:
        \begin{itemize}
        \item $U_S$ is non-conflicting with $\config{S}{e_{a_1}}
          \parstepsto \config{S_a}{e_{a_2}}$, since the only
          location allocated in the transition is $l$;
        \item $U_S(S_a) \neq \topS$, since $U_S(S_a) =
          \extS{\extS{S}{l''}{\bot}{\frozenfalse}}{l}{\bot}{\frozenfalse}$
          and we know $S \neq \topS$ and the addition of new
          bindings $\storebinding{l}{\bot}{\frozenfalse}$ and
          $\storebinding{l''}{\bot}{\frozenfalse}$ cannot cause it
          to become $\topS$; and
        \item $U_S$ is freeze-safe with $\config{S}{e_{a_1}}
          \parstepsto \config{S_a}{e_{a_2}}$, since $S_a =
          \extS{S}{l}{\bot}{\frozenfalse}$ and $l \notin \dom{S}$,
          so there are no locations whose contents differ in status
          between $S$ and $S_a$.
        \end{itemize}

        Therefore, by Lemma~\ref{lem:generalized-independence}
        (Generalized Independence), we have that

        $\config{U_S(S)}{e_{a_1}} \parstepsto
        \config{U_S(S_a)}{e_{a_2}}$.

        Hence $\config{\extS{S}{l''}{\bot}{\frozenfalse}}{e_{a_1}}
        \parstepsto
        \config{\extS{\extS{S}{l''}{\bot}{\frozenfalse}}{l}{\bot}{\frozenfalse}}{e_{a_2}}$.

        By {\sc E-Eval-Ctxt} it follows that
        
        $\config{\extS{S}{l''}{\bot}{\frozenfalse}}{\evalctxt{(\pi(E'_a))}{e_{a_1}}}
        \ctxstepsto \\
        \config{\extS{\extS{S}{l''}{\bot}{\frozenfalse}}{l}{\bot}{\frozenfalse}}{\evalctxt{(\pi(E'_a))}{e_{a_2}}}$,

        which completes the case since $\evalctxt{E'_b}{\pi(e_{b_2})}
        = \evalctxt{(\pi(E'_a))}{e_{a_2}}$.

        \lk{This assumes that you believe that
          $\evalctxt{E'_b}{\pi(e_{b_2})} =
          \evalctxt{(\pi(E'_a))}{e_{a_2}}$.}

      \end{itemize}

    \item \label{slqc-new-put}Case {\sc E-Put}: We have $S_a =
      \extS{S}{l}{\bot}{\frozenfalse}$ and $S_b =
      \extSRaw{S}{l'}{u_{p_i}(p_1)}$, where $l \neq l'$ (since $l
      \notin \dom{S}$, but $l' \in \dom{S}$).

      We have to show that:
      \begin{itemize}
      \item $\config{S_a}{\evalctxt{E'_b}{e_{b_1}}} \ctxstepsto
        \config{\extS{S_b}{l}{\bot}{\frozenfalse}}{\evalctxt{E'_b}{e_{b_2}}}$,
        and
      \item $\config{S_b}{\evalctxt{E'_a}{e_{a_1}}} \ctxstepsto
        \config{\extS{S_b}{l}{\bot}{\frozenfalse}}{\evalctxt{E'_a}{e_{a_2}}}$.
      \end{itemize}

      For the first of these, consider that $S_a =
      \extS{S}{l}{\bot}{\frozenfalse} = U_S(S)$, where $U_S$ is the
      store update operation that acts as the identity on the contents
      of all existing locations, and adds the binding
      $\storebinding{l}{\bot}{\frozenfalse}$ if no binding for $l$
      exists.

      Note that:
      \begin{itemize}
      \item $U_S$ is non-conflicting with $\config{S}{e_{b_1}}
        \parstepsto \config{S_b}{e_{b_2}}$, since no locations are
        allocated in the transition;
      \item $U_S(S_b) \neq \topS$, since $U_S(S_b) =
        \extS{S_b}{l}{\bot}{\frozenfalse}$, and we know $S_b \neq
        \topS$ and the addition of a new binding
        $\storebinding{l}{\bot}{\frozenfalse}$ cannot cause it to
        become $\topS$; and
      \item $U_S$ is freeze-safe with $\config{S}{e_{b_1}} \parstepsto
        \config{S_b}{e_{b_2}}$, since $S_b =
        \extSRaw{S}{l'}{u_{p_i}(p_1)}$ and $u_{p_i}$ does not alter
        the status of $p_1$.

        (By Definition~\ref{def:set-of-state-update-operations},
        $u_{p_i}$ can only change the status bit of a location if its
        contents are $\state{d}{\frozentrue}$ and $u_i(d) \neq d$, in
        which case $u_{p_i}$ changes the contents of the location to
        $\state{\top}{\frozenfalse}$; however, that cannot be the case
        here since then $u_{p_i}(p_1)$ would be $\topp$, contradicting
        the premise of {\sc E-Put}.)
      \end{itemize}

      Therefore, by Lemma~\ref{lem:generalized-independence}
      (Generalized Independence), we have that

      $\config{U_S(S)}{e_{b_1}} \parstepsto
      \config{U_S(S_b)}{e_{b_2}}$.

      Hence $\config{\extS{S}{l}{\bot}{\frozenfalse}}{e_{b_1}}
      \parstepsto
      \config{\extS{S_b}{l}{\bot}{\frozenfalse}}{e_{b_2}}$.

      By {\sc E-Eval-Ctxt}, it follows that

      $\config{\extS{S}{l}{\bot}{\frozenfalse}}{\evalctxt{E'_b}{e_{b_1}}}
      \ctxstepsto
      \config{\extS{S_b}{l}{\bot}{\frozenfalse}}{\evalctxt{E'_b}{e_{b_2}}}$,
      
      which, since $S_a = \extS{S}{l}{\bot}{\frozenfalse}$, is what we
      were required to show.

      For the second, let $U_S$ be the store update operation that
      applies $u_{p_i}$ to the contents of $l'$ if it exists, and adds
      a binding $\storebindingRaw{l'}{u_{p_i}(p_1)}$ if no binding for
      $l'$ exists.

      Consider that $S_b = U_S(S)$, and
      $\extS{S_b}{l}{\bot}{\frozenfalse} =
      \extSRaw{S_a}{l'}{u_{p_i}(p_1)} = U_S(S_a)$.

      Note that:
      \begin{itemize}
      \item $U_S$ is non-conflicting with $\config{S}{e_{a_1}}
        \parstepsto \config{S_a}{e_{a_2}}$, since the only location
        allocated in the transition is $l$;
      \item $U_S(S_a) \neq \topS$, since $U_S(S_a) =
        \extSRaw{\extS{S}{l}{\bot}{\frozenfalse}}{l'}{u_{p_i}(p_1)}$
        and we know $S \neq \topS$ and the addition of a new binding
        $\storebinding{l}{\bot}{\frozenfalse}$ and updating the
        contents of location $l'$ to $u_{p_i}(p_1)$ in $S$ cannot
        cause it to become $\topS$ (since if $u_{p_i}(p_1) = \topp$,
        $\config{S}{e_{b_1}}$ would not have been able to step by {\sc
          E-Put}); and
      \item $U_S$ is freeze-safe with $\config{S}{e_{a_1}} \parstepsto
        \config{S_a}{e_{a_2}}$, since $S_a =
        \extS{S}{l}{\bot}{\frozenfalse}$ and $l \notin \dom{S}$, so
        there are no locations whose contents differ in status between
        $S$ and $S_a$.
      \end{itemize}

      Therefore, by Lemma~\ref{lem:generalized-independence}
      (Generalized Independence), we have that

      $\config{U_S(S)}{e_{a_1}} \parstepsto
      \config{U_S(S_a)}{e_{a_2}}$.

      Hence $\config{S_b}{e_{a_1}}
      \parstepsto
      \config{\extS{S_b}{l}{\bot}{\frozenfalse}}{e_{a_2}}$.

      By {\sc E-Eval-Ctxt}, it follows that
      
      $\config{S_b}{\evalctxt{E'_a}{e_{a_1}}} \ctxstepsto
      \config{\extS{S_b}{l}{\bot}{\frozenfalse}}{\evalctxt{E'_a}{e_{a_2}}}$,
      
      as we were required to show.

    \item \label{slqc-new-put-err}Case {\sc E-Put-Err}: We have $S_a =
      \extS{S}{l}{\bot}{\frozenfalse}$ and $\config{S_b}{e_{b_2}} =
      \error$, and so we choose $\conf_c = \error$, $i = 1$, $j = 0$,
      and $\pi = \id$.

      We have to show that:
      \begin{itemize}
      \item $\config{S_a}{\evalctxt{E'_b}{e_{b_1}}} \ctxstepsto
        \error$, and
      \item $\config{S_b}{\evalctxt{E'_a}{e_{a_1}}} = \error$.
      \end{itemize}

      The second of these is immediately true because since
      $\config{S_b}{e_{b_2}} = \error$, $S_b = \topS$, and so
      $\config{S_b}{\evalctxt{E'_a}{e_{a_1}}}$ is equal to $\error$ as
      well.

      For the first, observe that since $\config{S}{e_{a_1}}
      \parstepsto \config{S_a}{e_{a_2}}$, we have by
      Lemma~\ref{lem:monotonicity} (Monotonicity) that
      $\leqstore{S}{S_a}$.

      Therefore, since $\config{S}{e_{b_1}} \parstepsto \error$,

      we have by Lemma~\ref{lem:error-preservation} (Error
      Preservation) that $\config{S_a}{e_{b_1}} \parstepsto \error$.

      Since $\error$ is equal to $\config{\topS}{e}$ for all
      expressions $e$, $\config{S_a}{e_{b_1}} \parstepsto
      \config{\topS}{e}$ for all $e$.

      Therefore, by {\sc E-Eval-Ctxt},
      $\config{S_a}{\evalctxt{E'_b}{e_{b_1}}} \ctxstepsto
      \config{\topS}{\evalctxt{E'_b}{e}}$ for all $e$.

      Since $\config{\topS}{\evalctxt{E'_b}{e}}$ is equal to $\error$,
      we have that $\config{S_a}{\evalctxt{E'_b}{e_{b_1}}} \ctxstepsto
      \error$, as we were required to show.

    \item \label{slqc-new-get}Case {\sc E-Get}: Similar to
      case~\ref{slqc-new-beta}, since $S_a =
      \extS{S}{l}{\bot}{\frozenfalse}$ and $S_b = S$.
    \item \label{slqc-new-freeze-init}Case {\sc E-Freeze-Init}:
      Similar to case~\ref{slqc-new-beta}, since $S_a =
      \extS{S}{l}{\bot}{\frozenfalse}$ and $S_b = S$.
    \item \label{slqc-new-spawn-handler}Case {\sc E-Spawn-Handler}:
      Similar to case~\ref{slqc-new-beta}, since $S_a =
      \extS{S}{l}{\bot}{\frozenfalse}$ and $S_b = S$.
    \item \label{slqc-new-freeze-final}Case {\sc E-Freeze-Final}: We
      have $S_a = \extS{S}{l}{\bot}{\frozenfalse}$ and $S_b =
      \extS{S}{l'}{d_1}{\frozentrue}$, where $l \neq l'$ (since $l
      \notin \dom{S}$, but $l' \in \dom{S}$).

      Choose $S' =
      \extS{\extS{S}{l}{\bot}{\frozenfalse}}{l'}{d_1}{\frozentrue}$,
      $i = i$, $j = 1$, and $\pi = \id$.

      We have to show that:
      \begin{itemize}
      \item
        $\config{\extS{S}{l}{\bot}{\frozenfalse}}{\evalctxt{E'_b}{e_{b_1}}}
        \ctxstepsto
        \config{\extS{\extS{S}{l}{\bot}{\frozenfalse}}{l'}{d_1}{\frozentrue}}{\evalctxt{E'_b}{e_{b_2}}}$,
        and
      \item
        $\config{\extS{S}{l'}{d_1}{\frozentrue}}{\evalctxt{E'_a}{e_{a_1}}}
        \ctxstepsto
        \config{\extS{\extS{S}{l}{\bot}{\frozenfalse}}{l'}{d_1}{\frozentrue}}{\evalctxt{E'_a}{e_{a_2}}}$.
      \end{itemize}

      For the first of these, consider that
      $\extS{S}{l}{\bot}{\frozenfalse} = U_S(S)$, where $U_S$ is the
      store update operation that acts as the identity on the contents
      of all existing locations, and adds the binding
      $\storebinding{l}{\bot}{\frozenfalse}$ if no binding for $l$
      exists.

      Note that:
      \begin{itemize}
      \item $U_S$ is non-conflicting with $\config{S}{e_{b_1}}
        \parstepsto \config{S_b}{e_{b_2}}$, since no locations are
        allocated in the transition;
      \item $U_S(S_b) \neq \topS$, since $U_S(S_b) =
        \extS{S_b}{l}{\bot}{\frozenfalse}$, and we know $S_b \neq
        \topS$ and the addition of a new binding
        $\storebinding{l}{\bot}{\frozenfalse}$ cannot cause it to
        become $\topS$; and
      \item $U_S$ is freeze-safe with $\config{S}{e_{b_1}}
        \parstepsto \config{S_b}{e_{b_2}}$, since $S_b =
        \extS{S}{l'}{d_1}{\frozentrue}$ and so the only location
        that can change in status between $S$ and $S_b$ is $l'$, and
        $U_S$ acts as the identity on $l'$.
      \end{itemize}
      Therefore, by Lemma~\ref{lem:generalized-independence}
      (Generalized Independence), we have that

      $\config{U_S(S)}{e_{b_1}} \parstepsto
      \config{U_S(S_b)}{e_{b_2}}$.

      Hence $\config{\extS{S}{l}{\bot}{\frozenfalse}}{e_{b_1}}
      \parstepsto
      \config{\extS{\extS{S}{l}{\bot}{\frozenfalse}}{l'}{d_1}{\frozentrue}}{e_{b_2}}$.

      By {\sc E-Eval-Ctxt}, it follows that

      $\config{\extS{S}{l}{\bot}{\frozenfalse}}{\evalctxt{E'_b}{e_{b_1}}}
      \ctxstepsto
      \config{\extS{\extS{S}{l}{\bot}{\frozenfalse}}{l'}{d_1}{\frozentrue}}{\evalctxt{E'_b}{e_{b_2}}}$,

      as we were required to show.

      For the second, consider that $\extS{S}{l'}{d_1}{\frozentrue} =
      U_S(S)$, where $U_S$ is the store update operation that freezes
      the contents of $l'$ and acts as the identity on the contents of
      all other locations.

      Note that:
      \begin{itemize}
      \item $U_S$ is non-conflicting with $\config{S}{e_{a_1}}
        \parstepsto \config{S_a}{e_{a_2}}$, since the only location
        allocated in the transition is $l$, and $l \neq l'$;
      \item $U_S(S_a) \neq \topS$, since $U_S(S_a) =
        \extS{S_a}{l'}{d_1}{\frozentrue} =
        \extS{S_b}{l}{\bot}{\frozenfalse}$, and we know $S_b \neq
        \topS$ and the addition of a new binding
        $\storebinding{l}{\bot}{\frozenfalse}$ cannot cause it to
        become $\topS$; and
      \item $U_S$ is freeze-safe with $\config{S}{e_{a_1}}
        \parstepsto \config{S_a}{e_{a_2}}$, since $S_a =
        \extS{S}{l}{\bot}{\frozenfalse}$ and $l \notin \dom{S}$, so
        there are no locations whose contents differ in status
        between $S$ and $S_a$.
      \end{itemize}

      Therefore, by Lemma~\ref{lem:generalized-independence}
      (Generalized Independence), we have that

      $\config{U_S(S)}{e_{a_1}} \parstepsto
      \config{U_S(S_a)}{e_{a_2}}$.

      Hence $\config{\extS{S}{l'}{d_1}{\frozentrue}}{e_{a_1}}
      \parstepsto
      \config{\extS{\extS{S}{l}{\bot}{\frozenfalse}}{l'}{d_1}{\frozentrue}}{e_{a_2}}$.

      By {\sc E-Eval-Ctxt} it follows that

      $\config{\extS{S}{l'}{d_1}{\frozentrue}}{\evalctxt{E'_a}{e_{a_1}}}
      \ctxstepsto
      \config{\extS{\extS{S}{l}{\bot}{\frozenfalse}}{l'}{d_1}{\frozentrue}}{\evalctxt{E'_a}{e_{a_2}}}$,

      as we were required to show.

    \item \label{slqc-new-freeze-simple}Case {\sc E-Freeze-Simple}:
      Similar to case~\ref{slqc-new-freeze-final}, since $S_a =
      \extS{S}{l}{\bot}{\frozenfalse}$ and $S_b =
      \extS{S}{l'}{d_1}{\frozentrue}$, where $l \neq l'$ (since $l
      \notin \dom{S}$, but $l' \in \dom{S}$).

    \end{enumerate}
  \item Case {\sc E-Put}: We have $S_a =
    \extSRaw{S}{l}{u_{p_i}(p_1)}$.

    We proceed by case analysis on the rule by which
    $\config{S}{e_{b_1}}$ steps to $\config{S_b}{e_{b_2}}$.

    Since the only way an $\error$ configuration can arise is by the
    {\sc E-Put-Err} rule, we can assume in all other cases that
    $\conf_b \neq \error$.
    \begin{enumerate}
    \item \label{slqc-put-beta}Case {\sc E-Beta}: By symmetry with case~\ref{slqc-beta-put}.
    \item \label{slqc-put-new}Case {\sc E-New}: By symmetry with case~\ref{slqc-new-put}.
    \item \label{slqc-put-put}Case {\sc E-Put}: We have $S_a =
      \extSRaw{S}{l}{u_{p_i}(p_1)}$ and $S_b =
      \extSRaw{S}{l'}{u_{p_j}(p'_1)}$, where $p'_1 = S(l')$.

      Now consider whether $l = l'$:
      \begin{itemize}
      \item If $l \neq l'$:

        Choose $S' =
        \extSRaw{\extSRaw{S}{l'}{u_{p_j}(p'_1)}}{l}{u_{p_i}(p_1)}$,
        $i = 1$, $j = 1$, and $\pi = \id$.

        We have to show that:
        \begin{itemize}
        \item
          $\config{\extSRaw{S}{l}{u_{p_i}(p_1)}}{\evalctxt{E'_b}{e_{b_1}}}
          \ctxstepsto
          \config{\extSRaw{\extSRaw{S}{l'}{u_{p_j}(p'_1)}}{l}{u_{p_i}(p_1)}}{\evalctxt{E'_b}{e_{b_2}}}$,
          and
        \item
          $\config{\extSRaw{S}{l'}{u_{p_j}(p'_1)}}{\evalctxt{E'_a}{e_{a_1}}}
          \ctxstepsto
          \config{\extSRaw{\extSRaw{S}{l'}{u_{p_j}(p'_1)}}{l}{u_{p_i}(p_1)}}{\evalctxt{E'_a}{e_{a_2}}}$.
        \end{itemize}

        For the first of these, consider that
        $\extSRaw{S}{l}{u_{p_i}(p_1)} = U_S(S)$, where $U_S$ is the
        store update operation that applies $u_{p_i}$ to the
        contents of $l$ if it exists, and adds a binding
        $\storebindingRaw{l}{u_{p_i}(p_1)}$ if no binding for $l$
        exists.

        Note that:
        \begin{itemize}
        \item $U_S$ is non-conflicting with $\config{S}{e_{b_1}}
          \parstepsto
          \config{\extSRaw{S}{l'}{u_{p_j}(p'_1)}}{e_{b_2}}$, since
          no locations are allocated in the transition;
        \item $U_S(\extSRaw{S}{l'}{u_{p_j}(p'_1)}) \neq \topS$,
          since $U_S(\extSRaw{S}{l'}{u_{p_j}(p'_1)}) =
          \extSRaw{\extSRaw{S}{l'}{u_{p_j}(p'_1)}}{l}{u_{p_i}(p_1)}$
          and we know $S \neq \topS$ and updating the contents of
          location $l$ to $u_{p_i}(p_1)$ and the contents of
          location $l'$ to $u_{p_j}(p'_1)$ in $S$ cannot cause it to
          become $\topS$ (because if so, then we would have $S_a =
          \topS$ or $S_b = \topS$, which we know are not the case);
          and
        \item $U_S$ is freeze-safe with $\config{S}{e_{b_1}}
          \parstepsto
          \config{\extSRaw{S}{l'}{u_{p_j}(p'_1)}}{e_{b_2}}$, since
          $u_{p_j}$ does not alter the status of $p'_1$.

          (By Definition~\ref{def:set-of-state-update-operations},
          $u_{p_j}$ can only change the status bit of a location if
          its contents are $\state{d}{\frozentrue}$ and $u_j(d) \neq
          d$, in which case $u_{p_j}$ changes the contents of the
          location to $\state{\top}{\frozenfalse}$; however, that
          cannot be the case here since then $u_{p_j}(p'_1)$ would be
          $\topp$, contradicting the premise of {\sc E-Put}.)
        \end{itemize}

        Therefore, by Lemma~\ref{lem:generalized-independence}
        (Generalized Independence), we have that

        $\config{U_S(S)}{e_{b_1}} \parstepsto
        \config{U_S(\extSRaw{S}{l'}{u_{p_j}(p'_1)})}{e_{b_2}}$.

        Hence $\config{\extSRaw{S}{l}{u_{p_i}(p_1)}}{e_{b_1}}
        \parstepsto
        \config{\extSRaw{\extSRaw{S}{l'}{u_{p_j}(p'_1)}}{l}{u_{p_i}(p_1)}}{e_{b_2}}$.

        By {\sc E-Eval-Ctxt}, it follows that

        $\config{\extSRaw{S}{l}{u_{p_i}(p_1)}}{\evalctxt{E'_b}{e_{b_1}}}
        \ctxstepsto
        \config{\extSRaw{\extSRaw{S}{l'}{u_{p_j}(p'_1)}}{l}{u_{p_i}(p_1)}}{\evalctxt{E'_b}{e_{b_2}}}$,

        as we were required to show.

        The argument for the second is symmetrical.

      \item If $l = l'$:
        Note that since $l = l'$, $p_1 = p'_1$ as well.

        Consider whether $u_{p_i}(u_{p_j}(p_1)) = \topp$:
        \begin{itemize}
        \item If $u_{p_i}(u_{p_j}(p_1)) = \topp$:

          Choose $\conf_c = \error$, $i = 1$, $j = 1$, and $\pi =
          \id$.

          We have to show that:

          \begin{itemize}
          \item
            $\config{\extSRaw{S}{l}{u_{p_i}(p_1)}}{\evalctxt{E'_b}{e_{b_1}}}
            \ctxstepsto \error$, and
          \item
            $\config{\extSRaw{S}{l}{u_{p_j}(p_1)}}{\evalctxt{E'_a}{e_{a_1}}}
            \ctxstepsto \error$.
          \end{itemize}

          For the first of these, consider that
          $\extSRaw{S}{l}{u_{p_i}(p_1)} = U_S(S)$, where $U_S$ is the
          store update operation that applies $u_{p_i}$ to the
          contents of $l$ if it exists.

          Note that:
          \begin{itemize}
          \item $U_S$ is non-conflicting with $\config{S}{e_{b_1}}
            \parstepsto
            \config{\extSRaw{S}{l}{u_{p_j}(p_1)}}{e_{b_2}}$, since
            no locations are allocated in the transition;
          \item $U_S(\extSRaw{S}{l}{u_{p_j}(p_1)}) = \topS$, since
            $U_S(\extSRaw{S}{l}{u_{p_j}(p_1)}) =
            \extSRaw{S}{l}{u_{p_i}(u_{p_j}(p_1))}$ and we know
            $u_{p_i}(u_{p_j}(p_1)) = \topp$ in this case;
          \item $U_S$ is freeze-safe with $\config{S}{e_{b_1}}
            \parstepsto
            \config{\extSRaw{S}{l}{u_{p_j}(p_1)}}{e_{b_2}}$, since
            $u_{p_j}$ does not alter the status of $p_1$.

            (By Definition~\ref{def:set-of-state-update-operations},
            $u_{p_j}$ can only change the status bit of a location if
            its contents are $\state{d}{\frozentrue}$ and $u_j(d) \neq
            d$, in which case $u_{p_j}$ changes the contents of the
            location to $\state{\top}{\frozenfalse}$; however, that
            cannot be the case here since then $u_{p_j}(p_1)$ would be
            $\topp$, contradicting the premise of {\sc E-Put}.)
          \end{itemize}

          Therefore, by Lemma~\ref{lem:generalized-clash}
          (Generalized Clash), we have that there exists $i' \leq 1$
          such that $\config{U_S(S)}{e_{b_1}} \parstepsto^{i'}
          \error$.

          Hence $\config{\extSRaw{S}{l}{u_{p_i}(p_1)}}{e_{b_1}}
          \parstepsto^{i'} \error$.

          If $i' = 0$, we would have
          $\config{\extSRaw{S}{l}{u_{p_i}(p_1)}}{e_{b_1}} =
          \config{S_a}{e_{b_1}} = \error$.

          So we would have $S_a = \topS$ by the definition of
          $\error$, but then we would have $\conf_a = \error$, a
          contradiction.

          Therefore $i' = 1$, and so we have
          $\config{\extSRaw{S}{l}{u_{p_i}(p_1)}}{e_{b_1}} \parstepsto
          \error$.

          Since $\error = \config{\topS}{e}$ for all $e$, we have
          $\config{\extSRaw{S}{l}{u_{p_i}(p_1)}}{e_{b_1}}
          \parstepsto \config{\topS}{e}$ for all $e$.

          So, by {\sc E-Eval-Ctxt}, we have that
          $\config{\extSRaw{S}{l}{u_{p_i}(p_1)}}{\evalctxt{E'_b}{e_{b_1}}}
          \parstepsto \config{\topS}{\evalctxt{E'_b}{e}}$ for all $e$.

          Hence
          $\config{\extSRaw{S}{l}{u_{p_i}(p_1)}}{\evalctxt{E'_b}{e_{b_1}}}
          \parstepsto \error$.

          The argument for the second is symmetrical.

        \item If $u_{p_i}(u_{p_j}(p_1)) \neq \topp$:

          Choose $S' = \extSRaw{S}{l}{u_{p_i}(u_{p_j}(p_1))}$, $i =
          1$, $j = 1$, and $\pi = \id$.

          We have to show that:
          \begin{itemize}
          \item
            $\config{\extSRaw{S}{l}{u_{p_i}(p_1)}}{\evalctxt{E'_b}{e_{b_1}}}
            \ctxstepsto
            \config{\extSRaw{S}{l}{u_{p_i}(u_{p_j}(p_1))}}{\evalctxt{E'_b}{e_{b_2}}}$,
            and
          \item
            $\config{\extSRaw{S}{l}{u_{p_j}(p_1)}}{\evalctxt{E'_a}{e_{a_1}}}
            \ctxstepsto
            \config{\extSRaw{S}{l}{u_{p_i}(u_{p_j}(p_1))}}{\evalctxt{E'_a}{e_{a_2}}}$.
          \end{itemize}

          For the first of these, consider that
          $\extSRaw{S}{l}{u_{p_i}(p_1)} = U_S(S)$, where $U_S$ is the
          store update operation that applies $u_{p_i}$ to the
          contents of $l$ if it exists.

          Note that:
          \begin{itemize}
          \item $U_S$ is non-conflicting with $\config{S}{e_{b_1}}
            \parstepsto
            \config{\extSRaw{S}{l}{u_{p_j}(p_1)}}{e_{b_2}}$, since no
            locations are allocated in the transition;
          \item $U_S(\extSRaw{S}{l}{u_{p_j}(p_1)}) \neq \topS$, since
            $U_S(\extSRaw{S}{l}{u_{p_j}(p_1)}) =
            \extSRaw{S}{l}{u_{p_i}(u_{p_j}(p_1))}$ and we know $S \neq
            \topS$ and $u_{p_i}(u_{p_j}(p_1)) \neq \topp$ in this
            case;
          \item $U_S$ is freeze-safe with $\config{S}{e_{b_1}}
            \parstepsto
            \config{\extSRaw{S}{l}{u_{p_j}(p_1)}}{e_{b_2}}$, since
            $u_{p_j}$ does not alter the status of $p_1$.

            (By Definition~\ref{def:set-of-state-update-operations},
            $u_{p_j}$ can only change the status bit of a location if
            its contents are $\state{d}{\frozentrue}$ and $u_j(d) \neq
            d$, in which case $u_{p_j}$ changes the contents of the
            location to $\state{\top}{\frozenfalse}$; however, that
            cannot be the case here since then $u_{p_j}(p_1)$ would be
            $\topp$, contradicting the premise of {\sc E-Put}.)
          \end{itemize}

          Therefore, by Lemma~\ref{lem:generalized-independence}
          (Generalized Independence), we have that

          $\config{U_S(S)}{e_{b_1}} \parstepsto
          \config{U_S(\extSRaw{S}{l}{u_{p_j}(p_1)})}{e_{b_2}}$.

          Hence $\config{\extSRaw{S}{l}{u_{p_i}(p_1)}}{e_{b_1}}
          \parstepsto
          \config{\extSRaw{S}{l}{u_{p_i}(u_{p_j}(p_1))}}{e_{b_2}}$.

          By {\sc E-Eval-Ctxt}, it follows that

          $\config{\extSRaw{S}{l}{u_{p_i}(p_1)}}{\evalctxt{E'_b}{e_{b_1}}}
          \ctxstepsto
          \config{\extSRaw{S}{l}{u_{p_i}(u_{p_j}(p_1))}}{\evalctxt{E'_b}{e_{b_2}}}$,

          as we were required to show.

          The argument for the second is symmetrical.

        \end{itemize}

      \end{itemize}

    \item \label{slqc-put-put-err}Case {\sc E-Put-Err}: We have $S_a =
      \extSRaw{S}{l}{u_{p_i}(p_1)}$ and $\config{S_b}{e_{b_2}} =
      \error$, and so we choose $\conf_c = \error$, $i = 1$, $j = 0$,
      and $\pi = \id$.

      We have to show that:
      \begin{itemize}
      \item $\config{S_a}{\evalctxt{E'_b}{e_{b_1}}} \ctxstepsto
        \error$, and
      \item $\config{S_b}{\evalctxt{E'_a}{e_{a_1}}} = \error$.
      \end{itemize}

      The second of these is immediately true because since
      $\config{S_b}{e_{b_2}} = \error$, $S_b = \topS$, and so
      $\config{S_b}{\evalctxt{E'_a}{e_{a_1}}}$ is equal to $\error$ as
      well.

      For the first, observe that since $\config{S}{e_{a_1}}
      \parstepsto \config{S_a}{e_{a_2}}$, we have by
      Lemma~\ref{lem:monotonicity} (Monotonicity) that
      $\leqstore{S}{S_a}$.

      Therefore, since $\config{S}{e_{b_1}} \parstepsto \error$,

      we have by Lemma~\ref{lem:error-preservation} (Error
      Preservation) that $\config{S_a}{e_{b_1}} \parstepsto \error$.
      
      Since $\error$ is equal to $\config{\topS}{e}$ for all
      expressions $e$, $\config{S_a}{e_{b_1}} \parstepsto
      \config{\topS}{e}$ for all $e$.

      Therefore, by {\sc E-Eval-Ctxt},
      $\config{S_a}{\evalctxt{E'_b}{e_{b_1}}} \ctxstepsto
      \config{\topS}{\evalctxt{E'_b}{e}}$ for all $e$.

      Since $\config{\topS}{\evalctxt{E'_b}{e}}$ is equal to $\error$,
      we have that $\config{S_a}{\evalctxt{E'_b}{e_{b_1}}} \ctxstepsto
      \error$, as we were required to show.

    \item \label{slqc-put-get}Case {\sc E-Get}: Similar to
      case~\ref{slqc-put-beta}, since $S_a =
      \extSRaw{S}{l}{u_{p_i}(p_1)}$ and $S_b = S$.
    \item \label{slqc-put-freeze-init}Case {\sc E-Freeze-Init}:
      Similar to case~\ref{slqc-put-beta}, since $S_a =
      \extSRaw{S}{l}{u_{p_i}(p_1)}$ and $S_b = S$.
    \item \label{slqc-put-spawn-handler}Case {\sc E-Spawn-Handler}:
      Similar to case~\ref{slqc-put-beta}, since $S_a =
      \extSRaw{S}{l}{u_{p_i}(p_1)}$ and $S_b = S$.
    \item \label{slqc-put-freeze-final}Case {\sc E-Freeze-Final}: We
      have $S_a = \extSRaw{S}{l}{u_{p_i}(p_1)}$ and $S_b =
      \extS{S}{l'}{d_1}{\frozentrue}$.

      Now consider whether $l = l'$:
      \begin{itemize}
      \item If $l \neq l'$:

        Choose $S' =
        \extS{\extSRaw{S}{l}{u_{p_i}(p_1)}}{l'}{d_1}{\frozentrue}$,
        $i = 1$, $j = 1$, and $\pi = \id$.

        We have to show that:
        \begin{itemize}
        \item
          $\config{\extSRaw{S}{l}{u_{p_i}(p_1)}}{\evalctxt{E'_b}{e_{b_1}}}
          \ctxstepsto
          \config{\extS{\extSRaw{S}{l}{u_{p_i}(p_1)}}{l'}{d_1}{\frozentrue}}{\evalctxt{E'_b}{e_{b_2}}}$,
          and
        \item
          $\config{\extS{S}{l'}{d_1}{\frozentrue}}{\evalctxt{E'_a}{e_{a_1}}}
          \ctxstepsto
          \config{\extS{\extSRaw{S}{l}{u_{p_i}(p_1)}}{l'}{d_1}{\frozentrue}}{\evalctxt{E'_a}{e_{a_2}}}$.
        \end{itemize}

        For the first of these, consider that
        $\extSRaw{S}{l}{u_{p_i}(p_1)} = U_S(S)$, where $U_S$ is the
        store update operation that applies $u_{p_i}$ to the
        contents of $l$ if it exists, and adds a binding
        $\storebindingRaw{l}{u_{p_i}(p_1)}$ if no binding for $l$
        exists, and acts as the identity on all other locations.

        Note that:
        \begin{itemize}
        \item $U_S$ is non-conflicting with $\config{S}{e_{b_1}}
          \parstepsto
          \config{\extS{S}{l'}{d_1}{\frozentrue}}{e_{b_2}}$, since
          no locations are allocated in the transition;
        \item $U_S(\extS{S}{l'}{d_1}{\frozentrue}) \neq \topS$,

          since $U_S(\extS{S}{l'}{d_1}{\frozentrue}) =
          \extSRaw{\extS{S}{l'}{d_1}{\frozentrue}}{l}{u_{p_i}(p_1)}$
          and we know $S \neq \topS$ and updating the contents of
          location $l$ to $u_{p_i}(p_1)$ and freezing the contents
          of location $l'$ in $S$ cannot cause it to become $\topS$
          (because if so, then we would have $S_a = \topS$ or $S_b =
          \topS$, which we know are not the case); and
        \item $U_S$ is freeze-safe with $\config{S}{e_{b_1}}
          \parstepsto
          \config{\extS{S}{l'}{d_1}{\frozentrue}}{e_{b_2}}$, since
          the only location that can change in status between $S$
          and $\extS{S}{l'}{d_1}{\frozentrue}$ is $l'$, and $U_S$
          acts as the identity on $l'$.
        \end{itemize}
        Therefore, by Lemma~\ref{lem:generalized-independence}
        (Generalized Independence), we have that

        $\config{U_S(S)}{e_{b_1}} \parstepsto
        \config{U_S(\extS{S}{l'}{d_1}{\frozentrue})}{e_{b_2}}$.

        Hence $\config{\extSRaw{S}{l}{u_{p_i}(p_1)}}{e_{b_1}}
        \parstepsto
        \config{\extSRaw{\extS{S}{l'}{d_1}{\frozentrue}}{l}{u_{p_i}(p_1)}}{e_{b_2}}$.

        By {\sc E-Eval-Ctxt}, it follows that

        $\config{\extSRaw{S}{l}{u_{p_i}(p_1)}}{\evalctxt{E'_b}{e_{b_1}}}
        \ctxstepsto
        \config{\extSRaw{\extS{S}{l'}{d_1}{\frozentrue}}{l}{u_{p_i}(p_1)}}{\evalctxt{E'_b}{e_{b_2}}}$,
        
        as we were required to show.

        For the second, consider that
        $\extS{S}{l'}{d_1}{\frozentrue} = U_S(S)$, where $U_S$ is
        the store update operation that freezes the contents of $l'$
        and acts as the identity on the contents of all other
        locations.

        Note that:
        \begin{itemize}
        \item $U_S$ is non-conflicting with $\config{S}{e_{a_1}}
          \parstepsto
          \config{\extSRaw{S}{l}{u_{p_i}(p_1)}}{e_{a_2}}$, since no
          locations are allocated in the transition;
        \item $U_S(\extSRaw{S}{l}{u_{p_i}(p_1)}) \neq \topS$, since
          $U_S(\extSRaw{S}{l}{u_{p_i}(p_1)}) =
          \extS{\extSRaw{S}{l}{u_{p_i}(p_1)}}{l'}{d_1}{\frozentrue}$,
          and we know $S \neq \topS$ and updating the contents of
          location $l$ to $u_{p_i}(p_1)$ and freezing the contents
          of location $l$ in $S$ cannot cause it to become $\topS$
          (because if so, then we would have $S_a = \topS$ or $S_b =
          \topS$, which we know are not the case); and
        \item $U_S$ is freeze-safe with $\config{S}{e_{a_1}}
          \parstepsto
          \config{\extSRaw{S}{l}{u_{p_i}(p_1)}}{e_{a_2}}$, since
          $u_{p_i}$ does not alter the status of $p_1$.

          (By Definition~\ref{def:set-of-state-update-operations},
          $u_{p_i}$ can only change the status bit of a location if
          its contents are $\state{d}{\frozentrue}$ and $u_i(d) \neq
          d$, in which case $u_{p_i}$ changes the contents of the
          location to $\state{\top}{\frozenfalse}$; however, that
          cannot be the case here since then $u_{p_i}(p_1)$ would be
          $\topp$, and we would have $S_a = \topS$, a contradiction.)
        \end{itemize}
        Therefore, by Lemma~\ref{lem:generalized-independence}
        (Generalized Independence), we have that

        $\config{U_S(S)}{e_{a_1}} \parstepsto
        \config{U_S(\extSRaw{S}{l}{u_{p_i}(p_1)})}{e_{a_2}}$.

        Hence $\config{\extS{S}{l'}{d_1}{\frozentrue}}{e_{a_1}}
        \parstepsto
        \config{\extS{\extSRaw{S}{l}{u_{p_i}(p_1)}}{l'}{d_1}{\frozentrue}}{e_{a_2}}$.

        By {\sc E-Eval-Ctxt}, it follows that
        $\config{\extS{S}{l'}{d_1}{\frozentrue}}{\evalctxt{E'_a}{e_{a_1}}}
        \ctxstepsto
        \config{\extS{\extSRaw{S}{l}{u_{p_i}(p_1)}}{l'}{d_1}{\frozentrue}}{\evalctxt{E'_a}{e_{a_2}}}$,
        
        as we were required to show.

      \item If $l = l'$:

        We have two cases to consider:

        \begin{itemize}
        \item $u_{p_i}(\state{d_1}{\frozentrue}) = \topp$:

          \lk{This is the interesting case: the potential
            put-after-freeze case.  It's important to note that this
            case doesn't necessarily end in a put-after-freeze (and
            hence an error); all we're required to show is that it
            \emph{can} end that way.}

          Since $(\extSRaw{S}{l}{\state{d_1}{\frozentrue}})(l) =
          \state{d_1}{\frozentrue}$ and
          $u_{p_i}(\state{d_1}{\frozentrue}) = \topp$, by {\sc
            E-Put-Err} we have that
          $\config{\extSRaw{S}{l}{\state{d_1}{\frozentrue}}}{\putiexp{l}}
          \parstepsto \error$.

          Since $S_b = \extSRaw{S}{l}{\state{d_1}{\frozentrue}}$,
          we have that $\config{S_b}{\putiexp{l}} \parstepsto
          \error$.

          Since $\config{S}{e_{a_1}} \parstepsto
          \config{S_a}{e_{a_2}}$ by {\sc E-Put}, it must be the
          case that $e_{a_1} = \putiexp{l}$.

          Hence $\config{S_b}{e_{a_1}} \parstepsto \error$.

          Since $\error$ is equal to $\config{\topS}{e}$ for all
          expressions $e$, $\config{S_b}{e_{a_1}} \parstepsto
          \config{\topS}{e}$ for all $e$.

          Therefore, by {\sc E-Eval-Ctxt},
          $\config{S_b}{\evalctxt{E'_a}{e_{a_1}}} \ctxstepsto
          \config{\topS}{\evalctxt{E'_a}{e}}$ for all $e$.

          Since $\config{\topS}{\evalctxt{E'_a}{e}}$ is equal to
          $\error$, we have that
          $\config{S_b}{\evalctxt{E'_a}{e_{a_1}}} \ctxstepsto \error$.

          Since $\evalctxt{E'_a}{e_{a_1}} =
          \evalctxt{E_b}{e_{b_2}}$, we have that
          $\config{S_b}{\evalctxt{E_b}{e_{b_2}}} \ctxstepsto
          \error$.

          Since $\conf_b = \config{S_b}{\evalctxt{E_b}{e_{b_2}}}$,
          we therefore have that $\conf_b \ctxstepsto \error$, and
          the case is satisfied.

        \item $u_{p_i}(\state{d_1}{\frozentrue}) \neq \topp$:

          \lk{This is the case where there's a conflicting put and
            freeze, but the put is a no-op, so it doesn't matter.}

          In this case, by the definition of $U_p$
          (Definition~\ref{def:set-of-state-update-operations}),
          
          it must be the case that $u_{p_i}(\state{d_1}{\frozentrue})
          = \state{d_1}{\frozentrue}$.

          Choose $S' = \extS{S}{l}{d_1}{\frozentrue}$, $i = 1$, $j
          = 1$, and $\pi = \id$.

          We have to show that:
          \begin{itemize}
          \item
            $\config{\extSRaw{S}{l}{u_{p_i}(p_1)}}{\evalctxt{E'_b}{e_{b_1}}}
            \ctxstepsto
            \config{\extS{S}{l}{d_1}{\frozentrue}}{\evalctxt{E'_b}{e_{b_2}}}$,
            and
          \item
            $\config{\extS{S}{l}{d_1}{\frozentrue}}{\evalctxt{E'_a}{e_{a_1}}}
            \ctxstepsto
            \config{\extS{S}{l}{d_1}{\frozentrue}}{\evalctxt{E'_a}{e_{a_2}}}$.
          \end{itemize}

          For the first of these, consider that
          $\extSRaw{S}{l}{u_{p_i}(p_1)} = U_S(S)$, where $U_S$ is
          the store update operation that applies $u_{p_i}$ to the
          contents of $l$ if it exists, and adds a binding
          $\storebindingRaw{l}{u_{p_i}(p_1)}$ if no binding for
          $l$ exists, and acts as the identity on all other
          locations.

          Note that:
          \begin{itemize}
          \item $U_S$ is non-conflicting with $\config{S}{e_{b_1}}
            \parstepsto
            \config{\extS{S}{l}{d_1}{\frozentrue}}{e_{b_2}}$,
            since no locations are allocated in the
            transition;
          \item $U_S(\extS{S}{l}{d_1}{\frozentrue}) \neq \topS$,
            
            since $U_S(\extS{S}{l}{d_1}{\frozentrue}) =
            \extSRaw{S}{l}{u_{p_i}(\state{d_1}{\frozentrue})}$ and
            we know $S \neq \topS$ and
            $u_{p_i}(\state{d_1}{\frozentrue}) \neq \topp$; and
          \item $U_S$ is freeze-safe with $\config{S}{e_{b_1}}
            \parstepsto
            \config{\extS{S}{l}{d_1}{\frozentrue}}{e_{b_2}}$, since
            the only location that can change in status between $S$
            and $\extS{S}{l}{d_1}{\frozentrue}$ is $l$, and $U_S$
            acts as the identity on $l$.
          \end{itemize}
          Therefore, by Lemma~\ref{lem:generalized-independence}
          (Generalized Independence), we have that

          $\config{U_S(S)}{e_{b_1}} \parstepsto
          \config{U_S(\extS{S}{l}{d_1}{\frozentrue})}{e_{b_2}}$.

          Hence $\config{\extSRaw{S}{l}{u_{p_i}(p_1)}}{e_{b_1}}
          \parstepsto
          \config{\extSRaw{S}{l}{u_{p_i}(\state{d_1}{\frozentrue})}}{e_{b_2}}$.

          Since $u_{p_i}(\state{d_1}{\frozentrue}) =
          \state{d_1}{\frozentrue}$,

          we have that
          $\config{\extSRaw{S}{l}{u_{p_i}(p_1)}}{e_{b_1}}
          \parstepsto
          \config{\extS{S}{l}{d_1}{\frozentrue}}{e_{b_2}}$.

          By {\sc E-Eval-Ctxt}, it follows that

          $\config{\extSRaw{S}{l}{u_{p_i}(p_1)}}{\evalctxt{E'_b}{e_{b_1}}}
          \ctxstepsto
          \config{\extS{S}{l}{d_1}{\frozentrue}}{\evalctxt{E'_b}{e_{b_2}}}$,

          as we were required to show.

          For the second, consider that
          $\extS{S}{l}{d_1}{\frozentrue} = U_S(S)$, where $U_S$ is
          the store update operation that freezes the contents of $l$
          and acts as the identity on the contents of all other
          locations.

          Note that:
          \begin{itemize}
          \item $U_S$ is non-conflicting with $\config{S}{e_{a_1}}
            \parstepsto
            \config{\extSRaw{S}{l}{u_{p_i}(p_1)}}{e_{a_2}}$, since no
            locations are allocated in the transition;
          \item $U_S(\extSRaw{S}{l}{u_{p_i}(p_1)}) \neq \topS$,
            since $U_S(\extSRaw{S}{l}{u_{p_i}(p_1)}) =
            \extS{S}{l}{d_1}{\frozentrue}$ (since, by
            Definition~\ref{def:set-of-state-update-operations},
            $u_i(d_1) = d_1$; otherwise we would have
            $u_{p_i}(\state{d_1}{\frozentrue}) = \topp$, a
            contradiction), and we know $S \neq \topS$ and
            freezing the contents of location $l$ in $S$ cannot
            cause it to become $\topS$; and
          \item $U_S$ is freeze-safe with $\config{S}{e_{a_1}}
            \parstepsto
            \config{\extSRaw{S}{l}{u_{p_i}(p_1)}}{e_{a_2}}$, since
            $u_{p_i}$ does not alter the status of $p_1$.

            (By Definition~\ref{def:set-of-state-update-operations},
            $u_{p_i}$ can only change the status bit of a location if
            its contents are $\state{d}{\frozentrue}$ and $u_i(d) \neq
            d$, in which case $u_{p_i}$ changes the contents of the
            location to $\state{\top}{\frozenfalse}$; however, that
            cannot be the case here since then $u_{p_i}(p_1)$ would be
            $\topp$, and we would have $S_a = \topS$, a
            contradiction.)
          \end{itemize}
          Therefore, by Lemma~\ref{lem:generalized-independence}
          (Generalized Independence), we have that

          $\config{U_S(S)}{e_{a_1}} \parstepsto
          \config{U_S(\extSRaw{S}{l}{u_{p_i}(p_1)})}{e_{a_2}}$.

          Hence $\config{\extS{S}{l}{d_1}{\frozentrue}}{e_{a_1}}
          \parstepsto
          \config{\extS{S}{l}{d_1}{\frozentrue}}{e_{a_2}}$.

          By {\sc E-Eval-Ctxt}, it follows that

          $\config{\extS{S}{l}{d_1}{\frozentrue}}{\evalctxt{E'_a}{e_{a_1}}}
          \ctxstepsto
          \config{\extS{S}{l}{d_1}{\frozentrue}}{\evalctxt{E'_a}{e_{a_2}}}$,

          as we were required to show.
        \end{itemize}

      \end{itemize}

    \item \label{slqc-put-freeze-simple}Case {\sc E-Freeze-Simple}:
      Similar to case~\ref{slqc-put-freeze-final}, since $S_a =
      \extSRaw{S}{l}{u_{p_i}(p_1)}$ and $S_b =
      \extS{S}{l'}{d_1}{\frozentrue}$.

    \end{enumerate}
  \item Case {\sc E-Put-Err}: We have $\config{S_a}{e_{a_2}} =
    \error$.

    We proceed by case analysis on the rule by which
    $\config{S}{e_{b_1}}$ steps to $\config{S_b}{e_{b_2}}$.

    Since the only way an $\error$ configuration can arise is by the
    {\sc E-Put-Err} rule, we can assume in all other cases that
    $\conf_b \neq \error$.
    \begin{enumerate}
    \item \label{slqc-put-err-beta}Case {\sc E-Beta}: By symmetry with case~\ref{slqc-beta-put-err}.
    \item \label{slqc-put-err-new}Case {\sc E-New}: By symmetry with case~\ref{slqc-new-put-err}.
    \item \label{slqc-put-err-put}Case {\sc E-Put}: By symmetry with case~\ref{slqc-put-put-err}.
    \item \label{slqc-put-err-put-err}Case {\sc E-Put-Err}: We have
      $\config{S_a}{e_{a_2}} = \error$ and $\config{S_b}{e_{b_2}} =
      \error$, and so we choose $\conf_c = \error$, $i = 0$, $j = 0$,
      and $\pi = \id$.

      We have to show that:
      \begin{itemize}
      \item $\config{S_a}{\evalctxt{E'_b}{e_{b_1}}} = \error$, and
      \item $\config{S_b}{\evalctxt{E'_a}{e_{a_1}}} = \error$.
      \end{itemize}

      Since $\config{S_a}{e_{a_2}} = \error$, $S_a = \topS$, and since
      $\config{S_b}{e_{b_2}} = \error$, $S_b = \topS$, so both of the
      above follow immediately.

    \item \label{slqc-put-err-get}Case {\sc E-Get}: Similar to
      case~\ref{slqc-put-err-beta}, since $\config{S_a}{e_{a_2}} =
      \error$ and $S_b = S$.
    \item \label{slqc-put-err-freeze-init}Case {\sc E-Freeze-Init}:
      Similar to case~\ref{slqc-put-err-beta}, since
      $\config{S_a}{e_{a_2}} = \error$ and $S_b = S$.
    \item \label{slqc-put-err-spawn-handler}Case {\sc
      E-Spawn-Handler}: Similar to case~\ref{slqc-put-err-beta}, since
      $\config{S_a}{e_{a_2}} = \error$ and $S_b = S$.
    \item \label{slqc-put-err-freeze-final}Case {\sc E-Freeze-Final}:
      We have $\config{S_a}{e_{a_2}} = \error$ and $S_b =
      \extS{S}{l}{d_1}{\frozentrue}$, and so we choose $\conf_c =
      \error$, $i = 0$, $j = 1$, and $\pi = \id$.

      We have to show that:
      \begin{itemize}
      \item $\config{S_a}{\evalctxt{E'_b}{e_{b_1}}} = \error$,
        and
      \item $\config{S_b}{\evalctxt{E'_a}{e_{a_1}}} \ctxstepsto
        \error$.
      \end{itemize}

      The first of these is immediately true because since
      $\config{S_a}{e_{a_2}} = \error$, $S_a = \topS$, and so
      $\config{S_a}{\evalctxt{E'_b}{e_{b_1}}}$ is equal to $\error$ as
      well.

      For the second, observe that since $\config{S}{e_{b_1}}
      \parstepsto \config{S_b}{e_{b_2}}$, we have by
      Lemma~\ref{lem:monotonicity} (Monotonicity) that
      $\leqstore{S}{S_b}$.

      Therefore, since $\config{S}{e_{a_1}} \parstepsto \error$, we
      have by Lemma~\ref{lem:error-preservation} that
      $\config{S_b}{e_{a_1}} \parstepsto \error$.

      Since $\error$ is equal to $\config{\topS}{e}$ for all
      expressions $e$, $\config{S_b}{e_{a_1}} \parstepsto
      \config{\topS}{e}$ for all $e$.

      Therefore, by {\sc E-Eval-Ctxt},
      $\config{S_b}{\evalctxt{E'_a}{e_{a_1}}} \ctxstepsto
      \config{\topS}{\evalctxt{E'_a}{e}}$ for all $e$.

      Since $\config{\topS}{\evalctxt{E'_a}{e}}$ is equal to $\error$,
      we have that $\config{S_b}{\evalctxt{E'_a}{e_{a_1}}} \ctxstepsto
      \error$, as we were required to show.

    \item \label{slqc-put-err-freeze-simple}Case {\sc
      E-Freeze-Simple}: Similar to
      case~\ref{slqc-put-err-freeze-final}, since $S_b =
      \extS{S}{l}{d_1}{\frozentrue}$.

    \end{enumerate}
  \item Case {\sc E-Get}: We have $S_a = S$.

    We proceed by case analysis on the rule by which
    $\config{S}{e_{b_1}}$ steps to $\config{S_b}{e_{b_2}}$.

    Since the only way an $\error$ configuration can arise is by the
    {\sc E-Put-Err} rule, we can assume in all other cases that
    $\conf_b \neq \error$.
    \begin{enumerate}
    \item \label{slqc-get-beta}Case {\sc E-Beta}: By symmetry with case~\ref{slqc-beta-get}.
    \item \label{slqc-get-new}Case {\sc E-New}: By symmetry with case~\ref{slqc-new-get}.
    \item \label{slqc-get-put}Case {\sc E-Put}: By symmetry with case~\ref{slqc-put-get}.
    \item \label{slqc-get-put-err}Case {\sc E-Put-Err}: By symmetry with case~\ref{slqc-put-err-get}.
    \item \label{slqc-get-get}Case {\sc E-Get}: Similar to
      case~\ref{slqc-get-beta}, since $S_a = S$ and $S_b = S$.
    \item \label{slqc-get-freeze-init}Case {\sc E-Freeze-Init}:
      Similar to case~\ref{slqc-get-beta}, since $S_a = S$ and $S_b = S$.
    \item \label{slqc-get-spawn-handler}Case {\sc E-Spawn-Handler}:
      Similar to case~\ref{slqc-get-beta}, since $S_a = S$ and $S_b = S$.
    \item \label{slqc-get-freeze-final}Case {\sc E-Freeze-Final}:
      Similar to case~\ref{slqc-beta-freeze-final}, since $S_a = S$
      and $S_b = \extS{S}{l}{d_1}{\frozentrue}$.
    \item \label{slqc-get-freeze-simple}Case {\sc E-Freeze-Simple}:
      Similar to case~\ref{slqc-beta-freeze-simple}, since $S_a = S$
      and $S_b = \extS{S}{l}{d_1}{\frozentrue}$.
    \end{enumerate}

  \item Case {\sc E-Freeze-Init}: We have $S_a = S$.

    We proceed by case analysis on the rule by which
    $\config{S}{e_{b_1}}$ steps to $\config{S_b}{e_{b_2}}$.

    Since the only way an $\error$ configuration can arise is by the
    {\sc E-Put-Err} rule, we can assume in all other cases that
    $\conf_b \neq \error$.
    \begin{enumerate}
    \item \label{slqc-freeze-init-beta}Case {\sc E-Beta}: By symmetry with case~\ref{slqc-beta-freeze-init}.
    \item \label{slqc-freeze-init-new}Case {\sc E-New}: By symmetry with case~\ref{slqc-new-freeze-init}.
    \item \label{slqc-freeze-init-put}Case {\sc E-Put}: By symmetry with case~\ref{slqc-put-freeze-init}.
    \item \label{slqc-freeze-init-put-err}Case {\sc E-Put-Err}: By symmetry with case~\ref{slqc-put-err-freeze-init}.
    \item \label{slqc-freeze-init-get}Case {\sc E-Get}: By symmetry with case~\ref{slqc-get-freeze-init}.
    \item \label{slqc-freeze-init-freeze-init}Case {\sc
      E-Freeze-Init}: Similar to case~\ref{slqc-freeze-init-beta},
      since $S_a = S$ and $S_b = S$.
    \item \label{slqc-freeze-init-spawn-handler}Case {\sc
      E-Spawn-Handler}: Similar to case~\ref{slqc-freeze-init-beta},
      since $S_a = S$ and $S_b = S$.
    \item \label{slqc-freeze-init-freeze-final}Case {\sc
      E-Freeze-Final}: Similar to case~\ref{slqc-beta-freeze-final},
      since $S_a = S$ and $S_b = \extS{S}{l}{d_1}{\frozentrue}$.
    \item \label{slqc-freeze-init-freeze-simple}Case {\sc
      E-Freeze-Simple}: Similar to case~\ref{slqc-beta-freeze-simple},
      since $S_a = S$ and $S_b = \extS{S}{l}{d_1}{\frozentrue}$.
    \end{enumerate}

  \item Case {\sc E-Spawn-Handler}: We have $S_a = S$.

    We proceed by case analysis on the rule by which
    $\config{S}{e_{b_1}}$ steps to $\config{S_b}{e_{b_2}}$.

    Since the only way an $\error$ configuration can arise is by the
    {\sc E-Put-Err} rule, we can assume in all other cases that
    $\conf_b \neq \error$.
    \begin{enumerate}
    \item \label{slqc-spawn-handler-beta}Case {\sc E-Beta}: By symmetry with case~\ref{slqc-beta-spawn-handler}.
    \item \label{slqc-spawn-handler-new}Case {\sc E-New}: By symmetry with case~\ref{slqc-new-spawn-handler}.
    \item \label{slqc-spawn-handler-put}Case {\sc E-Put}: By symmetry with case~\ref{slqc-put-spawn-handler}.
    \item \label{slqc-spawn-handler-put-err}Case {\sc E-Put-Err}: By symmetry with case~\ref{slqc-put-err-spawn-handler}.
    \item \label{slqc-spawn-handler-get}Case {\sc E-Get}: By symmetry with case~\ref{slqc-get-spawn-handler}.
    \item \label{slqc-spawn-handler-freeze-init}Case {\sc E-Freeze-Init}: By symmetry with case~\ref{slqc-freeze-init-spawn-handler}.
    \item \label{slqc-spawn-handler-spawn-handler}Case {\sc
      E-Spawn-Handler}: Similar to case~\ref{slqc-spawn-handler-beta},
      since $S_a = S$ and $S_b = S$.
    \item \label{slqc-spawn-handler-freeze-final}Case {\sc
      E-Freeze-Final}: Similar to case~\ref{slqc-beta-freeze-final},
      since $S_a = S$ and $S_b = \extS{S}{l}{d_1}{\frozentrue}$.
    \item \label{slqc-spawn-handler-freeze-simple}Case {\sc
      E-Freeze-Simple}: Similar to case~\ref{slqc-beta-freeze-simple},
      since $S_a = S$ and $S_b = \extS{S}{l}{d_1}{\frozentrue}$.
    \end{enumerate}

  \item Case {\sc E-Freeze-Final}: We have $S_a =
    \extS{S}{l}{d_1}{\frozentrue}$.

    We proceed by case analysis on the rule by which
    $\config{S}{e_{b_1}}$ steps to $\config{S_b}{e_{b_2}}$.

    Since the only way an $\error$ configuration can arise is by the
    {\sc E-Put-Err} rule, we can assume in all other cases that
    $\conf_b \neq \error$.
    \begin{enumerate}
    \item \label{slqc-freeze-final-beta}Case {\sc E-Beta}: By symmetry with case~\ref{slqc-beta-freeze-final}.
    \item \label{slqc-freeze-final-new}Case {\sc E-New}: By symmetry with case~\ref{slqc-new-freeze-final}.
    \item \label{slqc-freeze-final-put}Case {\sc E-Put}: By symmetry with case~\ref{slqc-put-freeze-final}.
    \item \label{slqc-freeze-final-put-err}Case {\sc E-Put-Err}: By symmetry with case~\ref{slqc-put-err-freeze-final}.
    \item \label{slqc-freeze-final-get}Case {\sc E-Get}: By symmetry with case~\ref{slqc-get-freeze-final}.
    \item \label{slqc-freeze-final-freeze-init}Case {\sc E-Freeze-Init}: By symmetry with case~\ref{slqc-freeze-init-freeze-final}.
    \item \label{slqc-freeze-final-spawn-handler}Case {\sc E-Spawn-Handler}: By symmetry with case~\ref{slqc-spawn-handler-freeze-final}.
    \item \label{slqc-freeze-final-freeze-final}Case {\sc
      E-Freeze-Final}: We have $S_a = \extS{S}{l}{d_1}{\frozentrue}$
      and $S_b = \extS{S}{l'}{d'_1}{\frozentrue}$.

      Now consider whether $l = l'$:
      \begin{itemize}
      \item If $l \neq l'$:

        Choose $S' =
        \extS{\extS{S}{l'}{d'_1}{\frozentrue}}{l}{d_1}{\frozentrue}$,
        $i = 1$, $j = 1$, and $\pi = \id$.

        We have to show that:
        \begin{itemize}
        \item
          $\config{\extS{S}{l}{d_1}{\frozentrue}}{\evalctxt{E'_b}{e_{b_1}}}
          \ctxstepsto
          \config{\extS{\extS{S}{l'}{d'_1}{\frozentrue}}{l}{d_1}{\frozentrue}}{\evalctxt{E'_b}{e_{b_2}}}$,
          and
        \item
          $\config{\extS{S}{l'}{d'_1}{\frozentrue}}{\evalctxt{E'_a}{e_{a_1}}}
          \ctxstepsto
          \config{\extS{\extS{S}{l'}{d'_1}{\frozentrue}}{l}{d_1}{\frozentrue}}{\evalctxt{E'_a}{e_{a_2}}}$.
        \end{itemize}

        For the first of these, consider that
        $\extS{S}{l}{d_1}{\frozentrue} = U_S(S)$, where $U_S$ is the
        store update operation that freezes the contents of $l$
        and acts as the identity on the contents of all other
        locations.

        Note that:
        \begin{itemize}
        \item $U_S$ is non-conflicting with $\config{S}{e_{b_1}}
          \parstepsto
          \config{\extS{S}{l'}{d'_1}{\frozentrue}}{e_{b_2}}$, since
          no locations are allocated in the transition;
        \item $U_S(\extS{S}{l'}{d'_1}{\frozentrue}) \neq \topS$,

          since $U_S(\extS{S}{l'}{d'_1}{\frozentrue}) =
          \extS{\extS{S}{l'}{d'_1}{\frozentrue}}{l}{d_1}{\frozentrue}$
          and we know $S \neq \topS$ and freezing the contents of
          locations $l$ and $l'$ in $S$ cannot cause it to become
          $\topS$ (because if so, then we would have $S_a = \topS$
          or $S_b = \topS$, which we know are not the case); and
        \item $U_S$ is freeze-safe with $\config{S}{e_{b_1}}
          \parstepsto
          \config{\extS{S}{l'}{d'_1}{\frozentrue}}{e_{b_2}}$, since
          the only location that can change in status between $S$
          and $\extS{S}{l'}{d'_1}{\frozentrue}$ is $l'$, and $U_S$
          acts as the identity on $l'$.
        \end{itemize}
        Therefore, by Lemma~\ref{lem:generalized-independence}
        (Generalized Independence), we have that

        $\config{U_S(S)}{e_{b_1}} \parstepsto
        \config{U_S(\extS{S}{l'}{d'_1}{\frozentrue})}{e_{b_2}}$.

        Hence $\config{\extS{S}{l}{d_1}{\frozentrue}}{e_{b_1}}
        \parstepsto
        \config{\extS{\extS{S}{l'}{d'_1}{\frozentrue}}{l}{d_1}{\frozentrue}}{e_{b_2}}$.

        By {\sc E-Eval-Ctxt}, it follows that

        $\config{\extS{S}{l}{d_1}{\frozentrue}}{\evalctxt{E'_b}{e_{b_1}}}
        \ctxstepsto
        \config{\extS{\extS{S}{l'}{d'_1}{\frozentrue}}{l}{d_1}{\frozentrue}}{\evalctxt{E'_b}{e_{b_2}}}$,
        
        as we were required to show.

        The argument for the second is symmetrical.

      \item If $l = l'$:

        \lk{This is the case where we freeze the same location twice,
          which is no problem; the second freeze is a no-op.}

        Note that since $l = l'$, $d_1 = d'_1$ as well.

        Choose $S' = \extS{S}{l}{d_1}{\frozentrue}$, $i = 1$, $j =
        1$, and $\pi = \id$.

        We have to show that:
        \begin{itemize}
        \item
          $\config{\extS{S}{l}{d_1}{\frozentrue}}{\evalctxt{E'_b}{e_{b_1}}}
          \ctxstepsto
          \config{\extS{S}{l}{d_1}{\frozentrue}}{\evalctxt{E'_b}{e_{b_2}}}$,
          and
        \item
          $\config{\extS{S}{l'}{d'_1}{\frozentrue}}{\evalctxt{E'_a}{e_{a_1}}}
          \ctxstepsto
          \config{\extS{S}{l}{d_1}{\frozentrue}}{\evalctxt{E'_a}{e_{a_2}}}$.
        \end{itemize}

        For the first of these, consider that
        $\extS{S}{l}{d_1}{\frozentrue} = U_S(S)$, where $U_S$ is the
        store update operation that freezes the contents of $l$ and
        acts as the identity on the contents of all other locations.

        Note that:
        \begin{itemize}
        \item $U_S$ is non-conflicting with $\config{S}{e_{b_1}}
          \parstepsto
          \config{\extS{S}{l}{d_1}{\frozentrue}}{e_{b_2}}$, since no
          locations are allocated in the transition;
        \item $U_S(\extS{S}{l}{d_1}{\frozentrue}) \neq \topS$, since
          $U_S(\extS{S}{l}{d_1}{\frozentrue}) =
          \extS{S}{l}{d_1}{\frozentrue}$, and we know $S \neq \topS$
          and freezing the contents of location $l$ in $S$ cannot
          cause it to become $\topS$; and
        \item $U_S$ is freeze-safe with $\config{S}{e_{b_1}}
          \parstepsto
          \config{\extS{S}{l}{d_1}{\frozentrue}}{e_{b_2}}$, since
          the only location that can change in status between $S$
          and $\extS{S}{l}{d_1}{\frozentrue}$ is $l$, and $U_S$
          freezes the contents of $l$ but has no other effect on
          them.
        \end{itemize}

        Therefore, by Lemma~\ref{lem:generalized-independence}
        (Generalized Independence), we have that

        $\config{U_S(S)}{e_{b_1}} \parstepsto
        \config{U_S(\extS{S}{l}{d_1}{\frozentrue})}{e_{b_2}}$.

        Hence $\config{\extS{S}{l}{d_1}{\frozentrue}}{e_{b_1}}
        \parstepsto
        \config{\extS{S}{l}{d_1}{\frozentrue}}{e_{b_2}}$.

        By {\sc E-Eval-Ctxt}, it follows that

        $\config{\extS{S}{l}{d_1}{\frozentrue}}{\evalctxt{E'_b}{e_{b_1}}}
        \ctxstepsto
        \config{\extS{S}{l}{d_1}{\frozentrue}}{\evalctxt{E'_b}{e_{b_2}}}$,

        as we were required to show.

        The argument for the second is symmetrical.

      \end{itemize}

    \item \label{slqc-freeze-final-freeze-simple}Case {\sc
      E-Freeze-Simple}: Similar to
      case~\ref{slqc-freeze-final-freeze-final}, since $S_a =
      \extS{S}{l}{d_1}{\frozentrue}$ and $S_b =
      \extS{S}{l'}{d'_1}{\frozentrue}$.
    \end{enumerate}

  \item Case {\sc E-Freeze-Simple}: We have $S_a =
    \extS{S}{l}{d_1}{\frozentrue}$.

    \begin{enumerate}
    \item \label{slqc-freeze-simple-beta}Case {\sc E-Beta}: By symmetry with case~\ref{slqc-beta-freeze-simple}.
    \item \label{slqc-freeze-simple-new}Case {\sc E-New}: By symmetry with case~\ref{slqc-new-freeze-simple}.
    \item \label{slqc-freeze-simple-put}Case {\sc E-Put}: By symmetry with case~\ref{slqc-put-freeze-simple}.
    \item \label{slqc-freeze-simple-put-err}Case {\sc E-Put-Err}: By symmetry with case~\ref{slqc-put-err-freeze-simple}.
    \item \label{slqc-freeze-simple-get}Case {\sc E-Get}: By symmetry with case~\ref{slqc-get-freeze-simple}.
    \item \label{slqc-freeze-simple-freeze-init}Case {\sc E-Freeze-Init}: By symmetry with case~\ref{slqc-freeze-init-freeze-simple}.
    \item \label{slqc-freeze-simple-spawn-handler}Case {\sc E-Spawn-Handler}: By symmetry with case~\ref{slqc-spawn-handler-freeze-simple}.
    \item \label{slqc-freeze-simple-freeze-final}Case {\sc E-Freeze-Final}: By symmetry with case~\ref{slqc-freeze-final-freeze-simple}.
    \item \label{slqc-freeze-simple-freeze-simple}Case {\sc
      E-Freeze-Simple}: Similar to
      case~\ref{slqc-freeze-simple-freeze-final}, since $S_a =
      \extS{S}{l}{d_1}{\frozentrue}$ and $S_b =
      \extS{S}{l'}{d'_1}{\frozentrue}$.
    \end{enumerate}

  \end{enumerate}
\end{proof}



\section{Proof of Lemma~\ref{lem:strong-one-sided-quasi-confluence}}\label{section:strong-one-sided-quasi-confluence-proof}
\begin{proof}
  Suppose $\conf \ctxstepsto \conf'$ and $\conf \ctxstepsto^m
  \conf''$, where $1 \leq m$.

  We are required to show that either:
  \begin{enumerate}
  \item there exist $\conf_c, i, j, \pi$ such that $\conf'
    \ctxstepsto^i \conf_c$ and $\pi(\conf'') \ctxstepsto^j \conf_c$
    and $i \leq m$ and $j \leq 1$, or
  \item there exists $k \leq m$ such that $\conf' \ctxstepsto^k
    \textup{\error}$, or there exists $k \leq 1$ such that $\conf''
    \ctxstepsto^k \textup{\error}$.
  \end{enumerate}

  We proceed by induction on $m$.

  In the base case of $m = 1$, the result is immediate from
  Lemma~\ref{lem:strong-local-quasi-confluence}, with $k = 1$.

  For the induction step, suppose $\conf \ctxstepsto^m \conf''
  \ctxstepsto \conf'''$ and suppose the lemma holds for $m$.

  We show that it holds for $m + 1$, as follows.

  From the induction hypothesis, we have that either:
  \begin{enumerate}
  \item there exist $\conf_c', i', j', \pi'$ such that $\conf'
    \ctxstepsto^{i'} \conf_c'$ and $\pi'(\conf'') \ctxstepsto^{j'}
    \conf_c'$ and $i' \leq m$ and $j' \leq 1$, or
  \item there exists $k' \leq m$ such that $\conf'
    \ctxstepsto^{k'} \error$, or there exists $k' \leq 1$ such that
    $\conf'' \ctxstepsto^{k'} \error$.
  \end{enumerate}

  We consider these two cases in turn:
  \begin{enumerate}
  \item There exist $\conf_c', i', j', \pi'$ such that $\conf'
    \ctxstepsto^{i'} \conf_c'$ and $\pi'(\conf'') \ctxstepsto^{j'}
    \conf_c'$ and $i' \leq m$ and $j' \leq 1$:

    We proceed by cases on $j'$:
    \begin{itemize}

    \item If $j' = 0$, then $\pi'(\conf'') = \conf_c'$.

      Since $\conf'' \ctxstepsto \conf'''$, we have that
      $\pi'(\conf'') \ctxstepsto \pi'(\conf''')$ by
      Lemma~\ref{lem:permutability} (Permutability).

      We can then choose $\conf_c = \pi'(\conf''')$ and $i = i' + 1$
      and $j = 0$ and $\pi = \pi'$.

      The key is that $\conf' \ctxstepsto^{i'} \conf'_c =
      \pi'(\conf'') \ctxstepsto \pi'(\conf''')$ for a total of $i' +
      1$ steps.
      
    \item If $j' = 1$:

      First, since $\pi'(\conf'') \ctxstepsto^{j'} \conf'_c$, then
      by Lemma~\ref{lem:permutability} (Permutability) we have that
      $\conf'' \ctxstepsto^{j'} \piprimeinv(\conf'_c)$.
      
      Then, by $\conf'' \ctxstepsto^{j'} \piprimeinv(\conf'_c)$ and
      $\conf'' \ctxstepsto \conf'''$ and
      Lemma~\ref{lem:strong-local-quasi-confluence}, one of the
      following two cases is true:
      \begin{enumerate}
      \item There exist $\conf_c''$ and $i''$ and $j''$ and $\pi''$
        such that $\piprimeinv(\conf'_c) \ctxstepsto^{i''}
        \conf_c''$ and $\pi''(\conf''') \ctxstepsto^{j''} \conf_c''$
        and $i'' \leq 1$ and $j'' \leq 1$.

        Since $\piprimeinv(\conf'_c) \ctxstepsto^{i''} \conf_c''$,
        by Lemma~\ref{lem:permutability} (Permutability) we have
        that $\conf'_c \ctxstepsto^{i''} \pi'(\conf_c'')$.

        So we also have $\conf' \ctxstepsto^{i'} \conf_c'
        \ctxstepsto^{i''} \pi'(\conf_c'')$.

        Since $\pi''(\conf''') \ctxstepsto^{j''} \conf_c''$, by
        Lemma~\ref{lem:permutability} (Permutability) we have that
        $\pi'(\pi''(\conf''')) \ctxstepsto^{j''} \pi'(\conf_c'')$.

        In summary, we pick $\conf_c = \pi'(\conf_c'')$ and $i = i' + i''$
        and $j = j''$ and $\pi = \pi'' \circ \pi'$, which is sufficient
        because $i = i' + i'' \leq m + 1$ and $j = j'' \leq 1$.

      \item $\piprimeinv(\conf'_c) \ctxstepsto \error$ or $\conf'''
        \ctxstepsto \error$.

        If $\conf''' \ctxstepsto \error$, then choosing $k = 1$
        satisfies the proof.

        Otherwise, $\piprimeinv(\conf'_c) \ctxstepsto \error$.

        Then, by Lemma~\ref{lem:permutability} we have that
        $\conf'_c \ctxstepsto \pi'(\error)$.

        By Definition~\ref{def:permutation-configuration},
        $\pi'(\error) = \error$, and so $\conf'_c \ctxstepsto
        \error$.

        Therefore $\conf' \ctxstepsto^{i'} \conf'_c \ctxstepsto
        \error$.

        Hence $\conf' \ctxstepsto^{i'+1} \error$.

        Since $i' \leq m$, we have that $i' + 1 \leq m + 1$, and
        so choosing $k = i' + 1$ satisfies the proof.
        
      \end{enumerate}

    \end{itemize}

  \item There exists $k' \leq m$ such that $\conf' \ctxstepsto^{k'}
    \error$, or there exists $k' \leq 1$ such that $\conf''
    \ctxstepsto^{k'} \error$:

    If there exists $k' \leq m$ such that $\conf' \ctxstepsto^{k'}
    \error$, then choosing $k = k'$ satisfies the proof.

    Otherwise, there exists $k' \leq 1$ such that $\conf''
    \ctxstepsto^{k'} \error$.

    We proceed by cases on $k'$:

    \begin{itemize}

    \item If $k' = 0$, then $\conf'' = \error$.

      Hence this case is not possible, since $\conf'' \ctxstepsto
      \conf'''$ and $\error$ cannot step.

    \item If $k' = 1$:

      From $\conf'' \ctxstepsto \conf'''$ and $\conf''
      \ctxstepsto^{k'} \error$ and
      Lemma~\ref{lem:strong-local-quasi-confluence}, one of the
      following two cases is true:

      \begin{enumerate}
      \item There exist $\conf_c''$ and $i''$ and $j''$ and $\pi''$
        such that $\error \ctxstepsto^{i''} \conf_c''$ and
        $\pi''(\conf''') \ctxstepsto^{j''} \conf_c''$ and $i'' \leq
        1$ and $j'' \leq 1$.

        Since $\error$ cannot step, $i'' = 0$ and $\conf''_c =
        \error$.

        By Definition~\ref{def:permutation-configuration},
        $\pi''(\conf''') = \conf'''$.

        Hence $\conf''' \ctxstepsto^{j''} \error$.

        \lk{This is the one place that we need to allow $k$ to be
          $\leq$ 1 instead of exactly 1.}

        Since $j'' \leq 1$, choosing $k = j''$ satisfies the proof.

      \item $\error \ctxstepsto \error$ or $\conf''' \ctxstepsto
        \error$.

        Since $\error$ cannot step, $\conf''' \ctxstepsto \error$.

        Hence choosing $k = 1$ satisfies the proof.

      \end{enumerate}

    \end{itemize}

  \end{enumerate}

\end{proof}


\section{Proof of Lemma~\ref{lem:strong-quasi-confluence}}\label{section:strong-quasi-confluence-proof}
\begin{proof}
  We proceed by induction on $n$.  In the base case of $n = 1$, the
  result is immediate from Lemma~\ref{lem:strong-one-sided-quasi-confluence}.

  For the induction step, suppose $\conf \parstepsto^n \conf'
  \parstepsto \conf'''$ and suppose the lemma holds for $n$.

  We show that it holds for $n + 1$, as follows.

  We are required to show that either:
  \begin{enumerate}
  \item there exist $\conf_c, i, j$ such that $\conf''' \parstepsto^i
    \conf_c$ and $\conf'' \parstepsto^j \conf_c$ and $i \leq m$ and $j
    \leq n + 1$, or
  \item there exists $k \leq m$ such that $\conf''' \parstepsto^k
    \error$, or there exists $k \leq n + 1$ such that $\conf''
    \parstepsto^k \error$.
  \end{enumerate}

  From the induction hypothesis, we have that either:
  \begin{enumerate}
  \item there exist $\conf'_c, i', j'$ such that $\conf'
    \parstepsto^{i'} \conf'_c$ and $\conf'' \parstepsto^{j'} \conf'_c$
    and $i' \leq m$ and $j' \leq n$, or
  \item there exists $k' \leq m$ such that $\conf' \parstepsto^{k'}
    \error$, or there exists $k' \leq n$ such that $\conf''
    \parstepsto^{k'} \error$.
  \end{enumerate}

  We consider these two cases in turn:

  \begin{enumerate}
  \item There exist $\conf'_c, i', j'$ such that $\conf'
    \parstepsto^{i'} \conf'_c$ and $\conf'' \parstepsto^{j'} \conf'_c$
    and $i' \leq m$ and $j' \leq n$:

    We proceed by cases on $i'$:
    \begin{itemize}

    \item If $i' = 0$, then $\conf' = \conf_c'$.  We can then choose
      $\conf_c = \conf'''$ and $i = 0$ and $j = j' + 1$.

    \item If $i' \geq 1$:

      From $\conf' \parstepsto \conf'''$ and $\conf' \parstepsto^{i'}
      \conf_c'$ and Lemma~\ref{lem:strong-one-sided-quasi-confluence},
      one of the following two cases is true:
      \begin{enumerate}
        \item There exist $\conf_c''$ and $i''$ and $j''$ such that
          $\conf''' \parstepsto^{i''} \conf_c''$ and $\conf_c'
          \parstepsto^{j''} \conf_c''$ and $i'' \leq i'$ and $j'' \leq
          1$.  So we also have $\conf'' \parstepsto^{j'} \conf_c'
          \parstepsto^{j''} \conf_c''$.  In summary, we pick $\conf_c
          = \conf_c''$ and $i = i''$ and $j = j' + j''$, which is
          sufficient because $i = i'' \leq i' \leq m$ and $j = j' +
          j'' \leq n + 1$.
        \item There exists $k'' \leq i'$ such that $\conf'''
          \parstepsto^{k''} \error$, or there exists $k'' \leq 1$ such
          that $\conf'_c \parstepsto^{k''} \error$.

          If there exists $k'' \leq i'$ such that $\conf'''
          \parstepsto^{k''} \error$, then choosing $k = k''$ satisfies
          the proof, since $k'' \leq i' \leq m$.

          Otherwise, there exists $k'' \leq 1$ such
          that $\conf'_c \parstepsto^{k''} \error$.

          Therefore, $\conf'' \parstepsto^{j'} \conf_c'
          \parstepsto^{k''} \error$.

          Hence $\conf'' \parstepsto^{j' + k''} \error$.

          Since $j' \leq n$ and $k'' \leq 1$, $j' + k'' \leq n + 1$.

          Hence choosing $k = j' + k''$ satisfies the proof.

      \end{enumerate}
    \end{itemize}

  \item There exists $k' \leq m$ such that $\conf' \parstepsto^{k'}
    \error$, or there exists $k' \leq n$ such that $\conf''
    \parstepsto^{k'} \error$:

    If there exists $k' \leq n$ such that $\conf'' \parstepsto^{k'}
    \error$, then choosing $k = k'$ satisfies the proof.

    Otherwise, there exists $k' \leq m$ such that $\conf'
    \parstepsto^{k'} \error$.  We proceed by cases on $k'$:

    \begin{itemize}

    \item If $k' = 0$, then $\conf' = \error$.

      Hence this case is not possible, since $\conf' \parstepsto
      \conf'''$ and $\error$ cannot step.

    \item If $k' \geq 1$:

      From $\conf' \parstepsto \conf'''$ and $\conf' \parstepsto^{k'}
      \error$ and Lemma~\ref{lem:strong-one-sided-quasi-confluence},
      one of the following two cases is true:

      \begin{enumerate}
        \item There exist $\conf''_c$ and $i''$ and $j''$ such that
          $\conf''' \parstepsto^{i''} \conf''_c$ and $\error
          \parstepsto^{j''} \conf''_c$ and $i'' \leq k'$ and $j'' \leq
          1$.

          Since $\error$ cannot step, $j'' = 0$ and $\conf''_c =
          \error$.

          Hence $\conf''' \parstepsto^{i''} \error$.

          Since $i'' \leq k' \leq m$, choosing $k = i''$ satisfies the
          proof.

        \item There exists $k'' \leq k'$ such that $\conf'''
          \parstepsto^{k''} \error$, or there exists $k'' \leq 1$ such
          that $\error \parstepsto^{k''} \error$.

          Since $\error$ cannot step, there exists $k'' \leq k'$ such
          that $\conf''' \parstepsto^{k''} \error$.

          Since $k'' \leq k' \leq m$, choosing $k = k''$ satisfies the
          proof.
      \end{enumerate}
    \end{itemize}
  \end{enumerate}

\end{proof}


\section{Proof of Theorem~\ref{thm:determinism-of-threshold-queries}}\label{section:determinism-of-threshold-queries-proof}
\begin{proof}
  Consider replica $i$ of a threshold CvRDT $(S, \leq, s^0, q, t, u,
  m)$.

  Let $\mathcal{S}$ be a threshold set with respect to
  $(S, \leq)$.

  Consider a method execution $t^{k+1}_i(\mathcal{S})$ (\ie, a
  threshold query that is the $k+1$th method execution on replica $i$,
  with threshold set $\mathcal{S}$ as its argument) that returns some
  set of activation states $S_a \in \mathcal{S}$.

  For part~\ref{thm:this-replica} of the theorem, we have to show that
  threshold queries with $\mathcal{S}$ as their argument will always
  return $S_a$ on subsequent executions at $i$.

  That is, we have to show that, for all $k' > (k+1)$, the threshold
  query $t^{k'}_i(\mathcal{S})$ on $i$ returns $S_a$.

  Since $t^{k+1}_i(\mathcal{S})$ returns $S_a$, from
  Definition~\ref{def:cvrdt-with-threshold-queries} we have that for
  some activation state $s_a \in S_a$, the condition $s_a \leq s^k_i$
  holds.

  Consider arbitrary $k' > (k+1)$.

  Since state is inflationary across updates, we know that the state
  $s^{k'}_i$ after method execution $k'$ is at least $s^k_i$.

  That is, $s^k_i \leq s^{k'}_i$.

  By transitivity of $\leq$, then, $s_a \leq s^{k'}_i$.

  Hence, by Definition~\ref{def:cvrdt-with-threshold-queries},
  $t^{k'}_i(\mathcal{S})$ returns $S_a$.

  For part~\ref{thm:any-replica} of the theorem, consider some replica
  $j$ of $(S, \leq, s^0, q, t, u, m)$, located at process $p_j$.

  We are required to show that, for all $x \geq 0$, the threshold
  query $t^{x+1}_j(\mathcal{S})$ returns $S_a$ eventually, and blocks
  until it does.\footnote{The occurrences of $k+1$ and $x+1$ in this
    proof are an artifact of how we index method executions starting
    from $1$, but states starting from $0$.  The initial state (of
    every replica) is $s^0$, and so $s^k_i$ is the state of replica
    $i$ after method execution $k$ has completed at $i$.}

  That is, we must show that, for all $x \geq 0$, there exists some
  finite $n \geq 0$ such that
  \begin{itemize}
  \item 
    for all $i$ in the range $0 \leq i \leq n-1$, the threshold query
    $t^{x+1+i}_j(\mathcal{S})$ returns $\block$, and
  \item
    for all $i \geq n$, the threshold query $t^{x+1+i}_j(\mathcal{S})$
    returns $S_a$.
  \end{itemize}
  Consider arbitrary $x \geq 0$.

  Recall that $s^x_j$ is the state of replica $j$ after the $x$th
  method execution, and therefore $s^x_j$ is also the state of $j$
  when $t^{x+1}_j(\mathcal{S})$ runs.
  %
  We have three cases to consider:
  \begin{itemize}
  \item $s^k_i \leq s^x_j$.

    (That is, replica $i$'s state after the $k$th method execution on $i$
    is \emph{at or below} replica $j$'s state after the $x$th method
    execution on $j$.)

    Choose $n = 0$.

    We have to show that, for all $i \geq n$, the threshold query
    $t^{x+1+i}_j(\mathcal{S})$ returns $S_a$.

    Since $t^{k+1}_i(\mathcal{S})$ returns $S_a$, we know that there
    exists an $s_a \in S_a$ such that $s_a \leq s^k_i$.

    Since $s^k_i \leq s^x_j$, we have by transitivity of $\leq$ that
    $s_a \leq s^x_j$.

    Therefore, by Definition~\ref{def:cvrdt-with-threshold-queries},
    $t^{x+1}_j(\mathcal{S})$ returns $S_a$.

    Then, by part~\ref{thm:this-replica} of the theorem, we have that
    subsequent executions $t^{x+1+i}_j(\mathcal{S})$ at replica $j$
    will also return $S_a$, and so the case holds.

    (Note that this case includes the possibility $s^k_i \equiv s^0$,
    in which no updates have executed at replica $i$.)

  \item $s^k_i > s^x_j$.

    (That is, replica $i$'s state after the $k$th method execution on $i$
    is \emph{above} replica $j$'s state after the $x$th method execution
    on $j$.)

    We have two subcases:

    \begin{itemize}
    \item
      There exists some activation state $s'_a \in S_a$ for which $s'_a \leq
      s^x_j$.

      In this case, we choose $n = 0$.

      We have to show that, for all $i \geq n$, the threshold query
      $t^{x+1+i}_j(\mathcal{S})$ returns $S_a$.

      Since $s'_a \leq s^x_j$, by
      Definition~\ref{def:cvrdt-with-threshold-queries},
      $t^{x+1}_j(\mathcal{S})$ returns $S_a$.

      Then, by part~\ref{thm:this-replica} of the theorem, we have
      that subsequent executions $t^{x+1+i}_j(\mathcal{S})$ at replica
      $j$ will also return $S_a$, and so the case holds.

    \item
      There is no activation state $s'_a \in S_a$ for which $s'_a \leq
      s^x_j$.

      Since $t^{k+1}_i(\mathcal{S})$ returns $S_a$, we know that there
      is some update $u^{k'}_i(a)$ in $i$'s causal history, for some
      $k' < (k+1)$, that updates $i$ from a state at or below $s^x_j$
      to $s^k_i$.\footnote{We know that $i$'s state was once at or
        below $s^x_j$, because $i$ and $j$ started at the same state
        $s^0$ and can both only grow.  Hence the least that $s^x_j$
        can be is $s^0$, and we know that $i$ was originally $s^0$ as
        well.}

      By eventual delivery, $u^{k'}_i(a)$ is eventually delivered at
      $j$.

      Hence some update or updates that will increase $j$'s state from
      $s^x_j$ to a state at or above some $s'_a$ must reach replica
      $j$.\footnote{We say ``some update or updates'' because the
        exact update $u^{k'}_i(a)$ may not be the update that causes
        the threshold query at $j$ to unblock; a different update or
        updates could do it.  Nevertheless, the existence of
        $u^{k'}_i(a)$ means that there is at least one update that
        will suffice to unblock the threshold query.}

      Let the $x+1+r$th method execution on $j$ be the first update on $j$
      that updates its state to some $s^{x+1+r}_j \geq s'_a$, for some
      activation state $s'_a \in S_a$.

      Choose $n = r+1$.

      We have to show that, for all $i$ in the range $0 \leq i \leq
      r$, the threshold query $t^{x+1+i}_j(\mathcal{S})$ returns
      $\block$, and that for all $i \geq r+1$, the threshold query
      $t^{x+1+i}_j(\mathcal{S})$ returns $S_a$.

      For the former, since the $x+1+r$th method execution on $j$ is the
      first one that updates its state to $s^{x+1+r}_j \geq s'_a$, we have
      by Definition~\ref{def:cvrdt-with-threshold-queries} that for all $i$
      in the range $0 \leq i \leq r$, the threshold query
      $t^{x+1+i}_j(\mathcal{S})$ returns $\block$.

      For the latter, since $s^{x+1+r}_j \geq s'_a$, by
      Definition~\ref{def:cvrdt-with-threshold-queries} we have that
      $t^{x+1+r+1}_j(\mathcal{S})$ returns $S_a$, and by
      part~\ref{thm:this-replica} of the theorem, we have that for $i \geq
      r+1$, subsequent executions $t^{x+1+i}_j(\mathcal{S})$ at replica $j$
      will also return $S_a$, and so the case holds.
    \end{itemize}

  \item $s^k_i \nleq s^x_j$ and $s^x_j \nleq s^k_i$.

    (That is, replica $i$'s state after the $k$th method execution on $i$
    is \emph{not comparable} to replica $j$'s state after the $x$th method
    execution on $j$.)

    Similar to the previous case.
  \end{itemize}
\end{proof}



\chapter{Proofs}\label{app:proofs}

\section{Proof of Lemma~\ref{lem:lvars-permutability}}\label{section:lvars-permutability-proof}
\begin{proof}
  Consider an arbitrary permutation $\pi$.  For
  part~\ref{thm:permutable-reduction-transitions}, we have to show
  that if $\conf \parstepsto \conf'$ then $\pi(\conf) \parstepsto
  \pi(\conf')$, and that if $\pi(\conf) \parstepsto \pi(\conf')$ then
  $\conf \parstepsto \conf'$.

  For the forward direction of
  part~\ref{thm:permutable-reduction-transitions}, suppose $\conf
  \parstepsto \conf'$.  We have to show that $\pi(\conf) \parstepsto
  \pi(\conf')$.  We proceed by cases on the rule by which $\conf$
  steps to $\conf'$.

  \begin{itemize}
    \item Case {\sc E-Beta}: $\conf =
      \config{S}{\app{(\lam{x}{e})}{v}}$, and $\conf' =
      \config{S}{\subst{e}{x}{v}}$.

      To show: $\pi(\config{S}{\app{(\lam{x}{e})}{v}}) \parstepsto
      \pi(\config{S}{\subst{e}{x}{v}})$.

      By Definitions~\ref{def:lvars-permutation-configuration}
      and~\ref{def:lvars-permutation-expression}, $\pi(\conf) =
      \config{\pi(S)}{\app{(\lam{x}{\pi(e)})}{\pi(v)}}$.

      By {\sc E-Beta},
      $\config{\pi(S)}{\app{(\lam{x}{\pi(e)})}{\pi(v)}}$ steps to
      $\config{\pi(S)}{\subst{\pi(e)}{x}{\pi(v)}}$.

      By Definition~\ref{def:lvars-permutation-expression},
      $\config{\pi(S)}{\subst{\pi(e)}{x}{\pi(v)}}$ is equal to
      $\config{\pi(S)}{\pi(\subst{e}{x}{v})}$.

      Hence $\config{\pi(S)}{\app{(\lam{x}{\pi(e)})}{\pi(v)}}$ steps
      to $\config{\pi(S)}{\pi(\subst{e}{x}{v})}$,

      which is equal to $\pi(\config{S}{\subst{e}{x}{v}})$ by
      Definition~\ref{def:lvars-permutation-configuration}.  Hence the
      case is satisfied.

    \item Case {\sc E-New}: $\conf = \config{S}{\NEW}$, and $\conf' =
      \config{\extSRaw{S}{l}{\bot}}{l}$.

      To show: $\pi(\config{S}{\NEW}) \parstepsto
      \pi(\config{\extSRaw{S}{l}{\bot}}{l})$.

      By Definitions~\ref{def:lvars-permutation-configuration}
      and~\ref{def:lvars-permutation-expression}, $\pi(\conf) =
      \config{\pi(S)}{\NEW}$.

      By {\sc E-New}, $\config{\pi(S)}{\NEW}$ steps to
      $\config{\extSRaw{(\pi(S))}{l'}{\bot}}{l'}$, where $l' \notin
      \dom{\pi(S)}$.
      
      It remains to show that
      $\config{\extSRaw{(\pi(S))}{l'}{\bot}}{l'}$ is equal to
      $\pi(\config{\extSRaw{S}{l}{\bot}}{l})$.

      By Definition~\ref{def:lvars-permutation-configuration},
      $\pi(\config{\extSRaw{S}{l}{\bot}}{l})$ is equal to
      $\config{\pi(\extSRaw{S}{l}{\bot})}{\pi(l)}$,

      which is equal to
      $\config{\extSRaw{(\pi(S))}{\pi(l)}{\bot}}{\pi(l)}$.

      So, we have to show that
      $\config{\extSRaw{(\pi(S))}{l'}{\bot}}{l'}$ is equal to
      $\config{\extSRaw{(\pi(S))}{\pi(l)}{\bot}}{\pi(l)}$.  Since we
      know (from the side condition of {\sc E-New}) that $l \notin
      \dom{S}$, it follows that $\pi(l) \notin \pi(\dom{S})$.
      Therefore, in $\config{\extSRaw{(\pi(S))}{l'}{\bot}}{l'}$, we
      can $\alpha$-rename $l'$ to $\pi(l)$, and so the two
      configurations are equal and the case is satisfied.

    \item Case {\sc E-Put}: $\conf = \config{S}{\putexp{l}{d_2}}$, and
      $\conf' = \config{\extSRaw{S}{l}{\userlub{d_1}{d_2}}}{\unit}$.

      To show: $\pi(\config{S}{\putexp{l}{d_2}}) \parstepsto
      \pi(\config{\extSRaw{S}{l}{\userlub{d_1}{d_2}}}{\unit})$.

      By Definitions~\ref{def:lvars-permutation-configuration}
      and~\ref{def:lvars-permutation-expression}, $\pi(\conf) =
      \config{\pi(S)}{\putexp{\pi(l)}{d_2}}$.

      By {\sc E-Put}, $\config{\pi(S)}{\putexp{\pi(l)}{d_2}}$ steps to
      $\config{\extSRaw{(\pi(S))}{\pi(l)}{\userlub{d_1}{d_2}}}{\unit}$,

      since $S(l) = (\pi(S))(\pi(l)) = d_1$.

      It remains to show that
      $\config{\extSRaw{(\pi(S))}{\pi(l)}{\userlub{d_1}{d_2}}}{\unit}$
      is equal to
      $\pi(\config{\extSRaw{S}{l}{\userlub{d_1}{d_2}}}{\unit})$.

      By Definitions~\ref{def:lvars-permutation-configuration}
      and~\ref{def:lvars-permutation-expression},
      $\pi(\config{\extSRaw{S}{l}{\userlub{d_1}{d_2}}}{\unit})$ is
      equal to
      $\config{\extSRaw{(\pi(S))}{\pi(l)}{\userlub{d_1}{d_2}}}{\unit}$,
      and so the two configurations are equal and the case is
      satisfied.

    \item Case {\sc E-Put-Err}: $\conf = \config{S}{\putexp{l}{d_2}}$,
      and $\conf' = \error$.

      To show: $\pi(\config{S}{\putexp{l}{d_2}}) \parstepsto
      \pi(\error)$.

      By Definitions~\ref{def:lvars-permutation-configuration}
      and~\ref{def:lvars-permutation-expression}, $\pi(\conf) =
      \config{\pi(S)}{\putexp{\pi(l)}{d_2}}$.

      By {\sc E-Put-Err}, $\config{\pi(S)}{\putexp{\pi(l)}{d_2}}$
      steps to $\error$,

      since $S(l) = (\pi(S))(\pi(l)) = d_1$.

      Since $\pi(\error) = \error$ by
      Definition~\ref{def:lvars-permutation-configuration}, the case
      is complete.

    \item Case {\sc E-Get}: $\conf = \config{S}{\getexp{l}{T}}$, and
      $\conf' = \config{S}{d_2}$.

      To show: $\pi(\config{S}{\getexp{l}{T}}) \parstepsto
      \pi(\config{S}{d_2})$.

      By Definitions~\ref{def:lvars-permutation-configuration}
      and~\ref{def:lvars-permutation-expression}, $\pi(\conf) =
      \config{\pi(S)}{\getexp{\pi(l)}{T}}$.

      By {\sc E-Get}, $\config{\pi(S)}{\getexp{\pi(l)}{T}}$ steps to
      $\config{\pi(S)}{d_2}$,

      since $S(l) = (\pi(S))(\pi(l)) = d_1$.

      By Definitions~\ref{def:lvars-permutation-configuration}
      and~\ref{def:lvars-permutation-expression},
      $\pi(\config{S}{d_2}) \config{\pi(S)}{d_2}$.  Therefore the case
      is complete.
  \end{itemize}

  For the reverse direction of
  part~\ref{thm:permutable-reduction-transitions}, suppose $\pi(\conf)
  \parstepsto \pi(\conf')$.  We have to show that $\conf \parstepsto
  \conf'$.

  We know from the forward direction of the proof that for all
  configurations $\conf$ and $\conf'$ and permutations $\pi$, if
  $\conf \parstepsto \conf'$ then $\pi(\conf) \parstepsto
  \pi(\conf')$.  Hence since $\pi(\conf) \parstepsto \pi(\conf')$, and
  since $\piinv$ is also a permutation, we have that
  $\piinv(\pi(\conf)) \parstepsto \piinv(\pi(\conf'))$.  Since
  $\piinv(\pi(l)) = l$ for every $l \in \Loc$, and that property lifts
  to configurations as well, we have that $\conf \parstepsto \conf'$.

  \lk{Is the above enough of a proof?}

  For the forward direction of
  part~\ref{thm:permutable-context-transitions}, suppose $\conf
  \ctxstepsto \conf'$.  We have to show that $\pi(\conf) \ctxstepsto
  \pi(\conf')$.

  By inspection of the operational semantics, $\conf$ must be of the
  form $\config{S}{\E{e}}$, and $\conf'$ must be of the form
  $\config{S'}{\E{e'}}$.  Hence we have to show that
  $\pi(\config{S}{\E{e}}) \ctxstepsto \pi(\config{S'}{\E{e'}})$.

  By Definition~\ref{def:lvars-permutation-configuration},
  $\pi(\config{S}{\E{e}})$ is equal to $\config{\pi(S)}{\pi(\E{e})}$.

  Also by Definition~\ref{def:lvars-permutation-configuration},
  $\pi(\config{S'}{\E{e'}})$ is equal to
  $\config{\pi(S')}{\pi(\E{e'})}$.

  Furthermore, $\config{\pi(S)}{\pi(\E{e})}$ is equal to
  $\config{\pi(S)}{\evalctxt{(\pi(E))}{\pi(e)}}$ and
  $\config{\pi(S')}{\pi(\E{e'})}$ is equal to
  $\config{\pi(S')}{\evalctxt{(\pi(E))}{\pi(e')}}$.

  So we have to show that
  $\config{\pi(S)}{\evalctxt{(\pi(E))}{\pi(e)}} \ctxstepsto
  \config{\pi(S')}{\evalctxt{(\pi(E))}{\pi(e')}}$.

  From the premise of {\sc E-Eval-Ctxt}, $\config{S}{e} \parstepsto
  \config{S'}{e'}$.  Hence, by
  part~\ref{thm:permutable-reduction-transitions}, $\pi(\config{S}{e})
  \parstepsto \pi(\config{S'}{e'})$.  By
  Definition~\ref{def:lvars-permutation-configuration},
  $\pi(\config{S}{e})$ is equal to $\config{\pi(S)}{\pi(e)}$ and
  $\pi(\config{S'}{e'})$ is equal to $\config{\pi(S')}{\pi(e')}$.

  Hence $\config{\pi(S)}{\pi(e)} \parstepsto
  \config{\pi(S')}{\pi(e')}$.

  Therefore, by {\sc E-Eval-Ctxt}, $\config{\pi(S)}{\E{\pi(e)}}
  \ctxstepsto \config{\pi(S')}{\E{\pi(e')}}$ for all evaluation
  contexts $E$.

  In particular, it is true that
  $\config{\pi(S)}{\evalctxt{(\pi(E))}{\pi(e)}} \ctxstepsto
  \config{\pi(S')}{\evalctxt{(\pi(E))}{\pi(e')}}$, as we were required
  to show.

  For the reverse direction of
  part~\ref{thm:permutable-context-transitions}, suppose $\pi(\conf)
  \ctxstepsto \pi(\conf')$.  We have to show that $\conf \ctxstepsto
  \conf'$.

  We know from the forward direction of the proof that for all
  configurations $\conf$ and $\conf'$ and permutations $\pi$, if
  $\conf \ctxstepsto \conf'$ then $\pi(\conf) \ctxstepsto
  \pi(\conf')$.  Hence since $\pi(\conf) \ctxstepsto \pi(\conf')$, and
  since $\piinv$ is also a permutation, we have that
  $\piinv(\pi(\conf)) \ctxstepsto \piinv(\pi(\conf'))$.  Since
  $\piinv(\pi(l)) = l$ for every $l \in \Loc$, and that property lifts
  to configurations as well, we have that $\conf \ctxstepsto \conf'$.

  \lk{Is the above enough of a proof?}
\end{proof}


\section{Proof of Lemma~\ref{lem:lvars-internal-determinism}}\label{section:lvars-internal-determinism-proof}
\begin{proof}
  Suppose $\conf \parstepsto \conf'$ and $\conf \parstepsto \conf''$.

  We have to show that there is a permutation $\pi$ such that $\conf'
  = \pi(\conf'')$.

  The proof is by cases on the rule by which $\conf$ steps to
  $\conf'$.

  \begin{itemize}

  \item Case {\sc E-Beta}:

    Given: $\config{S}{\app{(\lam{x}{e})}{v}} \parstepsto
    \config{S}{\subst{e}{x}{v}}$, and
    $\config{S}{\app{(\lam{x}{e})}{v}} \parstepsto \conf''$.

    To show: There exists a $\pi$ such that
    $\config{S}{\subst{e}{x}{v}} = \pi(\conf'')$.

    By inspection of the operational semantics, the only reduction
    rule by which $\config{S}{\app{(\lam{x}{e})}{v}}$ can step is {\sc
      E-Beta}.

    Hence $\conf'' = \config{S}{\subst{e}{x}{v}}$, and the case is
    satisfied by choosing $\pi$ to be the identity function.

  \item Case {\sc E-New}: 

    Given: $\config{S}{\NEW} \parstepsto
    \config{\extSRaw{S}{l}{\bot}}{l}$, and $\config{S}{\NEW}
    \parstepsto \conf''$.

    To show: There exists a $\pi$ such that
    $\config{\extSRaw{S}{l}{\bot}}{l} = \pi(\conf'')$.

    By inspection of the operational semantics, the only reduction
    rule by which $\config{S}{\NEW}$ can step is {\sc E-New}.

    Hence $\conf'' = \config{\extSRaw{S}{l'}{\bot}}{l'}$.

    Since, by the side condition of {\sc E-New}, neither $l$ nor $l'$
    occur in $\dom{S}$, the case is satisfied by choosing $\pi$ to be
    the permutation that maps $l'$ to $l$ and is the identity on every
    other element of $\Loc$.

  \item Case {\sc E-Put}:

    Given: $\config{S}{\putexp{l}{d_2}} \parstepsto
    \config{\extSRaw{S}{l}{\userlub{d_1}{d_2}}}{\unit}$, and
    $\config{S}{\putexp{l}{d_2}} \parstepsto \conf''$.

    To show: There exists a $\pi$ such that
    $\config{\extSRaw{S}{l}{\userlub{d_1}{d_2}}}{\unit} =
    \pi(\conf'')$.

    By inspection of the operational semantics, and since
    $\userlub{d_1}{d_2} \neq \top$ (from the premise of {\sc E-Put}),
    the only reduction rule by which $\config{S}{\putexp{l}{d_2}}$ can
    step is {\sc E-Put}.

    Hence $\conf'' =
    \config{\extSRaw{S}{l}{\userlub{d_1}{d_2}}}{\unit}$, and the case
    is satisfied by choosing $\pi$ to be the identity function.

  \item Case {\sc E-Put-Err}:

    Given: $\config{S}{\putexp{l}{d_2}} \parstepsto \error$, and
    $\config{S}{\putexp{l}{d_2}} \parstepsto \conf''$.

    To show: There exists a $\pi$ such that $\error = \pi(\conf'')$.

    By inspection of the operational semantics, and since
    $\userlub{d_1}{d_2} = \top$ (from the premise of {\sc E-Put-Err}),
    the only reduction rule by which $\config{S}{\putexp{l}{d_2}}$ can
    step is {\sc E-Put-Err}.

    Hence $\conf'' = \error$, and the case is satisfied by choosing
    $\pi$ to be the identity function.

  \item Case {\sc E-Get}:

    Given: $\config{S}{\getexp{l}{T}} \parstepsto \config{S}{d_2}$,
    and $\config{S}{\getexp{l}{T}} \parstepsto \conf''$.

    To show: There exists a $\pi$ such that $\config{S}{d_2} =
    \pi(\conf'')$.

    By inspection of the operational semantics, the only reduction
    rule by which $\config{S}{\getexp{l}{T}}$ can step is {\sc
      E-Get}.

    Hence $\conf'' = \config{S}{d_2}$, and the case is satisfied by
    choosing $\pi$ to be the identity function.

  \end{itemize}
\end{proof}



\section{Proof of Lemma~\ref{lem:lvars-monotonicity}}\label{section:lvars-monotonicity-proof}
\begin{proof}
  Suppose $\config{S}{e} \parstepsto \config{S'}{e'}$.  We are
  required to show that $\leqstore{S}{S'}$.  The proof is by cases on
  the rule by which $\config{S}{e}$ steps to $\config{S'}{e'}$.

  \begin{itemize}
    \item Case {\sc E-Beta}:

      Immediate by the definition of $\leqstore{}{}$, since $S$ does
      not change.

    \item Case {\sc E-New}:

      Given: $\config{S}{\NEW} \parstepsto
      \config{\extSRaw{S}{l}{\bot}}{l}$.

      To show: $\leqstore{S}{\extSRaw{S}{l}{\bot}}$.

      By Definition~\ref{def:lvars-leqstore}, we have to show that
      $\dom{S} \subseteq \dom{\extSRaw{S}{l}{\bot}}$ and
      that for all $l' \in \dom{S}$, $S(l') \userleq
      (\extSRaw{S}{l}{\bot})(l')$.

      By definition, a store update operation on $S$ can only either
      update an existing binding in $S$ or extend $S$ with a new
      binding.  Hence $\dom{S} \subseteq \dom{\extSRaw{S}{l}{\bot}}$.

      From the side condition of {\sc E-New}, $l \notin \dom{S}$.
      Hence $\extSRaw{S}{l}{\bot}$ adds a new binding for $l$ in $S$.

      Hence $\extSRaw{S}{l}{\bot}$ does not update any existing
      bindings in $S$.

      Hence, for all $l' \in \dom{S}, S(l') \userleq
      (\extSRaw{S}{l}{\bot})(l')$.

      Therefore $\leqstore{S}{\extSRaw{S}{l}{\bot}}$, as
      required.

    \item Case {\sc E-Put}:

      Given: $\config{S}{\putexp{l}{d_2}} \parstepsto
      \config{\extSRaw{S}{l}{\userlub{d_1}{d_2}}}{\unit}$.

      To show: $\leqstore{S}{\extSRaw{S}{l}{\userlub{d_1}{d_2}}}$.

      By Definition~\ref{def:lvars-leqstore}, we have to show that
      $\dom{S} \subseteq \dom{\extSRaw{S}{l}{\userlub{d_1}{d_2}}}$ and
      that for all $l' \in \dom{S}$, $S(l') \userleq
      (\extSRaw{S}{l}{\userlub{d_1}{d_2}})(l')$.

      By definition, a store update operation on $S$ can only either
      update an existing binding in $S$ or extend $S$ with a new
      binding.  Hence $\dom{S} \subseteq
      \dom{\extSRaw{S}{l}{\userlub{d_1}{d_2}}}$.

      From the premises of {\sc E-Put}, $S(l) = d_1$.  Therefore $l
      \in \dom{S}$.

      Hence $\extSRaw{S}{l}{\userlub{d_1}{d_2}}$ updates the existing
      binding for $l$ in $S$ from $d_1$ to $\userlub{d_1}{d_2}$.

      By the definition of $\userlub{}{}$, $d_1 \userleq
      (\userlub{d_1}{d_2})$.  $\extSRaw{S}{l}{\userlub{d_1}{d_2}}$
      does not update any other bindings in $S$, hence, for all $l'
      \in \dom{S}, S(l') \userleq
      (\extSRaw{S}{l}{\userlub{d_1}{d_2}})(l')$.

      Hence $\leqstore{S}{\extSRaw{S}{l}{\userlub{d_1}{d_2}}}$, as
      required.

    \item Case {\sc E-Put-Err}:

      Given: $\config{S}{\putexp{l}{d_2}} \parstepsto \error$.

      By the definition of $\error$, $\error$ is equal to
      $\config{\topS}{e}$ for all $e$.

      To show: $\leqstore{S}{\topS}$.

      Immediate by the definition of $\leqstore{}{}$.

    \item Case {\sc E-Get}:

      Immediate by the definition of $\leqstore{}{}$, since $S$ does
      not change.

  \end{itemize}

\end{proof}


\section{Proof of Lemma~\ref{lem:lvars-independence}}\label{section:lvars-independence-proof}
\begin{proof}
  Consider arbitrary $S''$ such that $S''$ is non-conflicting with
  $\config{S}{e} \parstepsto \config{S'}{e'}$ and $\lubstore{S'}{S''}
  \neq \topS$.

  To show: $\config{\lubstore{S}{S''}}{e} \parstepsto
  \config{\lubstore{S'}{S''}}{e'}$.

  The proof is by induction on the derivation of $\config{S}{e}
  \parstepsto \config{S'}{e'}$, by cases on the last rule in the
  derivation.  In every case we may assume that $\config{S'}{e'} \neq
  \error$.  Since $\config{S'}{e'} \neq \error$, we do not need to
  consider the {\sc E-Put-Err} rule.
  \begin{itemize}

    \item Case {\sc E-Eval-Ctxt}:

      Given: $\config{S}{\E{e}} \parstepsto \config{S'}{\E{e'}}$.

      To show: $\config{\lubstore{S}{S''}}{\E{e}} \parstepsto
      \config{\lubstore{S'}{S''}}{\E{e'}}$.

      From the premise of {\sc E-Eval-Ctxt}, we have that
      $\config{S}{e} \parstepsto \config{S'}{e'}$.

      Therefore, by IH, we have that $\config{\lubstore{S}{S''}}{e}
      \parstepsto \config{\lubstore{S'}{S''}}{e'}$.

      Therefore, by {\sc E-Eval-Ctxt}, we have that
      $\config{\lubstore{S}{S''}}{\E{e}} \parstepsto
      \config{\lubstore{S'}{S''}}{\E{e'}}$, as we were required to
      show.

    \item Case {\sc E-Beta}:

      Given: $\config{S}{\app{(\lam{x}{e})}{v}} \parstepsto
      \config{S}{\subst{e}{x}{v}}$.

      To show: $\config{\lubstore{S}{S''}}{\app{(\lam{x}{e})}{v}}
      \parstepsto \config{\lubstore{S}{S''}}{\subst{e}{x}{v}}$.

      Immediate by {\sc E-Beta}.

    \item Case {\sc E-New}:

      Given: $\config{S}{\NEW} \parstepsto
      \config{\extSRaw{S}{l}{\bot}}{l}$.

      To show: $\config{\lubstore{S}{S''}}{\NEW} \parstepsto
      \config{\lubstore{(\extSRaw{S}{l}{\bot})}{S''}}{l}$.

      By {\sc E-New}, we have that $\config{\lubstore{S}{S''}}{\NEW}
      \parstepsto \config{\extSRaw{(\lubstore{S}{S''})}{l'}{\bot}}{l'}$,
      where $l' \notin \dom{\lubstore{S}{S''}}$.

      By assumption, $S''$ is non-conflicting with $\config{S}{\NEW}
      \parstepsto \config{\extSRaw{S}{l}{\bot}}{l}$.
 
      Therefore $l \notin \dom{S''}$.

      From the side condition of {\sc E-New}, $l \notin \dom{S}$.

      Therefore $l \notin \dom{\lubstore{S}{S''}}$.

      Therefore, in
      $\config{\extSRaw{(\lubstore{S}{S''})}{l'}{\bot}}{l'}$, we can
      $\alpha$-rename $l'$ to $l$, \\ resulting in
      $\config{\extSRaw{(\lubstore{S}{S''})}{l}{\bot}}{l}$.

      Therefore $\config{\lubstore{S}{S''}}{\NEW} \parstepsto
      \config{\extSRaw{(\lubstore{S}{S''})}{l}{\bot}}{l}$.

      Note that:
      \begin{align*}
        \extSRaw{(\lubstore{S}{S''})}{l}{\bot} &=
        \lubstore{\extSRaw{S}{l}{\bot}}{\extSRaw{S''}{l}{\bot}} \\ &=
        \lubstore{\lubstore{S}{\store{\storebindingRaw{l}{\bot}}}}{\lubstore{S''}{\store{\storebindingRaw{l}{\bot}}}}
        \\ &=
        \lubstore{\lubstore{S}{\store{\storebindingRaw{l}{\bot}}}}{S''}
        \\ &= \lubstore{\extSRaw{S}{l}{\bot}}{S''}.
      \end{align*}
      Therefore $\config{\lubstore{S}{S''}}{\NEW} \parstepsto
      \config{\lubstore{\extSRaw{S}{l}{\bot}}{S''}}{l}$, as we were
      required to show.

    \item Case {\sc E-Put}:

      Given: $\config{S}{\putexp{l}{d_2}} \parstepsto
      \config{\extSRaw{S}{l}{d_2}}{\unit}$.

      To show: $\config{\lubstore{S}{S''}}{\putexp{l}{d_2}}
      \parstepsto
      \config{\lubstore{\extSRaw{S}{l}{d_2}}{S''}}{\unit}$.

      We will first show that

      $\config{\lubstore{S}{S''}}{\putexp{l}{d_2}} \parstepsto
      \config{\extSRaw{(\lubstore{S}{S''})}{l}{d_2}}{\unit}$

      and then show why this is sufficient.

      We proceed by cases on $l$:

      \begin{itemize}
        \item $l \notin \dom{S''}$:

          By assumption, $\lubstore{\extSRaw{S}{l}{d_2}}{S''} \neq
          \topS$.

          By Lemma~\ref{lem:lvars-monotonicity},
          $\leqstore{S}{\extSRaw{S}{l}{d_2}}$.

          Hence $\lubstore{S}{S''} \neq \topS$.

          Therefore, by Definition~\ref{def:lvars-lubstore},
          $(\lubstore{S}{S''})(l) = S(l)$.

          From the premises of {\sc E-Put}, $S(l) = d_1$.

          Hence $(\lubstore{S}{S''})(l) = d_1$.

          From the premises of {\sc E-Put}, $d_2 = \userlub{d_1}{d_2}$
          and $d_2 \neq \top$.

          Therefore, by {\sc E-Put}, we have:
          $\config{\lubstore{S}{S''}}{\putexp{l}{d_2}} \parstepsto
          \config{\extSRaw{(\lubstore{S}{S''})}{l}{d_2}}{\unit}$.

        \item $l \in \dom{S''}$:

          By assumption, $\lubstore{\extSRaw{S}{l}{d_2}}{S''} \neq
          \topS$.

          By Lemma~\ref{lem:lvars-monotonicity},
          $\leqstore{S}{\extSRaw{S}{l}{d_2}}$.

          Hence $\lubstore{S}{S''} \neq \topS$.

          Therefore $(\lubstore{S}{S''})(l) = \userlub{S(l)}{S''(l)}$.

          From the premises of {\sc E-Put}, $S(l) = d_1$.
          
          Hence $(\lubstore{S}{S''})(l) = d'_1$, where $d_1 \userleq
          d'_1$.

          From the premises of {\sc E-Put}, $d_2 =
          \userlub{d_1}{d_2}$.

          Let $d'_2 = \userlub{d'_1}{d_2}$.

          Hence $d_2 \userleq d'_2$.

          By assumption, $\lubstore{\extSRaw{S}{l}{d_2}}{S''} \neq
          \topS$.

          Therefore, by Definition~\ref{def:lvars-lubstore},
          $\lubstore{d_2}{S''(l)} \neq \top$.

          Note that:
          \begin{align*}
            \top &\neq \lubstore{d_2}{S''(l)} \\ &=
            \userlub{\userlub{d_1}{d_2}}{S''(l)} \\ &=
            \userlub{\userlub{S(l)}{d_2}}{S''(l)} \\ &=
            \userlub{\userlub{S(l)}{S''(l)}}{d_2} \\ &=
            \userlub{(\lubstore{S}{S''})(l)}{d_2} \\ &=
            \userlub{d'_1}{d_2} \\ &= d'_2. \\
          \end{align*}
          Hence $d'_2 \neq \top$.

          Hence $(\lubstore{S}{S''})(l) = d'_1$ and $d'_2 =
          \userlub{d'_1}{d_2}$ and $d'_2 \neq \top$.

          Therefore, by {\sc E-Put} we have:
          $\config{\lubstore{S}{S''}}{\putexp{l}{d_2}} \parstepsto
          \config{\extSRaw{(\lubstore{S}{S''})}{l}{d'_2}}{\unit}$.

          \lk{If we really wanted to be pedantic here, we'd actually
            prove that the stores are equal.  I'm assuming that if I
            can show that $\extSRaw{(\lubstore{S}{S''})}{l}{d'_2}$ and
            $\extSRaw{(\lubstore{S}{S''})}{l}{d_2}$ bind $l$ to the
            same value, then it will be obvious that they're equal.}

          Note that:
          \begin{align*}
            (\extSRaw{(\lubstore{S}{S''})}{l}{d'_2})(l) &=
            \userlub{(\lubstore{S}{S''})(l)}{(\store{\storebindingRaw{l}{d'_2}})(l)}
            \\ &= \userlub{d'_1}{d'_2} \\ &=
            \userlub{d'_1}{\userlub{d'_1}{d_2}} \\ &=
            \userlub{d'_1}{d_2}
          \end{align*}
          and
          \begin{align*}
            (\extSRaw{(\lubstore{S}{S''})}{l}{d_2})(l) &=
            \userlub{(\lubstore{S}{S''})(l)}{(\store{\storebindingRaw{l}{d_2}})(l)}
            \\ &= \userlub{d'_1}{d_2} \\ &=
            \userlub{d'_1}{\userlub{d_1}{d_2}} \\ &=
            \userlub{d'_1}{d_2} & \textrm{(since $d_1 \userleq
              d'_1$).}
          \end{align*}
          Therefore $\extSRaw{(\lubstore{S}{S''})}{l}{d'_2} =
          \extSRaw{(\lubstore{S}{S''})}{l}{d_2}$.

          Therefore, $\config{\lubstore{S}{S''}}{\putexp{l}{d_2}}
          \parstepsto
          \config{\extSRaw{(\lubstore{S}{S''})}{l}{d_2}}{\unit}$.
      \end{itemize}

      Note that:
      \begin{align*}
        \extSRaw{(\lubstore{S}{S''})}{l}{d_2} &=
        \lubstore{\extSRaw{S}{l}{d_2}}{\extSRaw{S''}{l}{d_2}} \\ &=
        \lubstore{\lubstore{S}{\store{\storebindingRaw{l}{d_2}}}}{\lubstore{S''}{\store{\storebindingRaw{l}{d_2}}}}
        \\ &=
        \lubstore{\lubstore{S}{\store{\storebindingRaw{l}{d_2}}}}{S''}
        \\ &= \lubstore{\extSRaw{S}{l}{d_2}}{S''}.
      \end{align*}
      Therefore $\config{\lubstore{S}{S''}}{\putexp{l}{d_2}}
      \parstepsto
      \config{\lubstore{\extSRaw{S}{l}{d_2}}{S''}}{\unit}$, as we were
      required to show.

    \item Case {\sc E-Get}:

      Given: $\config{S}{\getexp{l}{T}} \parstepsto \config{S}{d_2}$.

      To show: $\config{\lubstore{S}{S''}}{\getexp{l}{T}} \parstepsto
      \config{\lubstore{S}{S''}}{d_2}$.

      From the premises of {\sc E-Get}, $S(l) = d_1$ and $\incomp{T}$
      and $d_2 \in T$ and $d_2 \userleq d_1$.

      By assumption, $\lubstore{S}{S''} \neq \topS$.

      Hence $(\lubstore{S}{S''}) = d'_1$, where $d_1 \userleq d'_1$.

      By the transitivity of $\userleq$, $d_2 \userleq d'_1$.

      Hence, $S(l) = d'_1$ and $\incomp{T}$ and $d_2 \in T$ and $d_2
      \userleq d'_1$.

      Therefore, by {\sc E-Get},

      $\config{\lubstore{S}{S''}}{\getexp{l}{T}} \parstepsto
      \config{\lubstore{S}{S''}}{d_2}$,

      as we were required to show.
  \end{itemize}
\end{proof}


\section{Proof of Lemma~\ref{lem:lvars-clash}}\label{section:lvars-clash-proof}
\begin{proof}
  Consider arbitrary $S''$ such that $S''$ is non-conflicting with
  $\config{S}{e} \parstepsto \config{S'}{e'}$ and $\lubstore{S'}{S''}
  = \topS$.

  To show: $\config{\lubstore{S}{S''}}{e} \parstepsto \error$.

  The proof is by induction on the derivation of $\config{S}{e}
  \parstepsto \config{S'}{e'}$, by cases on the last rule in the
  derivation.  In every case we may assume that $\config{S'}{e'} \neq
  \error$.  Since $\config{S'}{e'} \neq \error$, we do not need to
  consider the {\sc E-Put-Err} rule.

  \begin{itemize}

    \item Case {\sc E-Eval-Ctxt}:

      Given: $\config{S}{\E{e}} \parstepsto \config{S'}{\E{e'}}$.

      To show: $\config{\lubstore{S}{S''}}{\E{e}} \parstepsto^i
      \error$, where $i \leq 1$.

      From the premise of {\sc E-Eval-Ctxt}, we have that
      $\config{S}{e} \parstepsto \config{S'}{e'}$.

      Therefore, by IH, we have that $\config{\lubstore{S}{S''}}{e}
      \parstepsto^{i'} \error$, where $i' \leq 1$.

      We proceed by cases on $i'$:

      \begin{itemize}
        \item $i' = 0$:

          In this case, $\config{\lubstore{S}{S''}}{e} = \error$.

          Hence, by the definition of $\error$, $\lubstore{S}{S''} =
          \topS$.

          Hence $\config{\lubstore{S}{S''}}{\E{e}} = \error$.

          Hence $\config{\lubstore{S}{S''}}{\E{e}} \parstepsto^i
          \error$, with $i = 0$.

        \item $i' = 1$:

          In this case, $\config{\lubstore{S}{S''}}{e} \parstepsto
          \error$.

          By the definition of $\error$, $\error =
          \config{\topS}{e''}$ for any $e''$.

          Hence $\config{\lubstore{S}{S''}}{e} \parstepsto
          \config{\topS}{e''}$.

          Hence, by {\sc E-Eval-Ctxt},
          $\config{\lubstore{S}{S''}}{\E{e}} \parstepsto
          \config{\topS}{\E{e''}}$.

          By the definition of $\error$, $\config{\topS}{\E{e''}} =
          \error$.

          Hence $\config{\lubstore{S}{S''}}{\E{e}} \parstepsto
          \error$.

          Hence $\config{\lubstore{S}{S''}}{\E{e}} \parstepsto^i
          \error$, with $i = 1$.

      \end{itemize}

    \item Case {\sc E-Beta}:

      Given: $\config{S}{\app{(\lam{x}{e})}{v}} \parstepsto
      \config{S}{\subst{e}{x}{v}}$.

      To show: $\config{\lubstore{S}{S''}}{\app{(\lam{x}{e})}{v}}
      \parstepsto^i \error$, where $i \leq 1$.

      By assumption, $\lubstore{S}{S''} = \topS$.

      Hence, by the definition of $\error$,
      $\config{\lubstore{S}{S''}}{\app{(\lam{x}{e})}{v}} = \error$.

      Hence $\config{\lubstore{S}{S''}}{\app{(\lam{x}{e})}{v}}
      \parstepsto^i \error$, with $i = 0$.

    \item Case {\sc E-New}:

      Given: $\config{S}{\NEW} \parstepsto
      \config{\extSRaw{S}{l}{\bot}}{l}$.

      To show: $\config{\lubstore{S}{S''}}{\NEW} \parstepsto^i
      \error$, where $i \leq 1$.

      By {\sc E-New}, $\config{\lubstore{S}{S''}}{\NEW} \parstepsto
      \config{\extSRaw{(\lubstore{S}{S''})}{l'}{\bot}}{l'}$, where $l'
      \notin \dom{\lubstore{S}{S''}}$.

      By assumption, $S''$ is non-conflicting with $\config{S}{\NEW}
      \parstepsto \config{\extSRaw{S}{l}{\bot}}{l}$.
 
      Therefore $l \notin \dom{S''}$.

      From the side condition of {\sc E-New}, $l \notin \dom{S}$.

      Therefore $l \notin \dom{\lubstore{S}{S''}}$.

      Therefore, in
      $\config{\extSRaw{(\lubstore{S}{S''})}{l'}{\bot}}{l'}$, we can
      $\alpha$-rename $l'$ to $l$, \\ resulting in
      $\config{\extSRaw{(\lubstore{S}{S''})}{l}{\bot}}{l}$.

      Therefore $\config{\lubstore{S}{S''}}{\NEW} \parstepsto
      \config{\extSRaw{(\lubstore{S}{S''})}{l}{\bot}}{l}$.

      By assumption, $\lubstore{\extSRaw{S}{l}{\bot}}{S''}
      = \topS$.

      Note that:
      \begin{align*}
        \topS &= \lubstore{\extSRaw{S}{l}{\bot}}{S''} \\ &=
        \lubstore{\lubstore{S}{\store{\storebindingRaw{l}{\bot}}}}{S''}
        \\ &=
        \lubstore{\lubstore{S}{S''}}{\store{\storebindingRaw{l}{\bot}}}
        \\ &=
        \lubstore{(\lubstore{S}{S''})}{\store{\storebindingRaw{l}{\bot}}}
        \\ &= \extSRaw{(\lubstore{S}{S''})}{l}{\bot} .
      \end{align*}

      Hence $\config{\lubstore{S}{S''}}{\NEW} \parstepsto
      \config{\topS}{l}$.

      Hence, by the definition of $\error$,
      $\config{\lubstore{S}{S''}}{\NEW} \parstepsto \error$.

      Hence $\config{\lubstore{S}{S''}}{\NEW} \parstepsto^i \error$,
      with $i = 1$.

    \item Case {\sc E-Put}:

      Given: $\config{S}{\putexp{l}{d_2}} \parstepsto
      \config{\extSRaw{S}{l}{d_2}}{\unit}$.

      To show: $\config{\lubstore{S}{S''}}{\putexp{l}{d_2}}
      \parstepsto^i \error$, where $i \leq 1$.

      We proceed by cases on $\lubstore{S}{S''}$:

      \begin{itemize}

        \item $\lubstore{S}{S''} = \topS$:

          In this case, by the definition of $\error$,
          $\config{\lubstore{S}{S''}}{\putexp{l}{d_2}} = \error$.

          Hence $\config{\lubstore{S}{S''}}{\putexp{l}{d_2}}
          \parstepsto^i \error$, with $i = 0$.

        \item $\lubstore{S}{S''} \neq \topS$:

          From the premises of {\sc E-Put}, we have that $S(l) = d_1$.

          Hence $(\lubstore{S}{S''})(l) = d'_1$, where $d_1 \userleq
          d'_1$.

          We show that $\userlub{d'_1}{d_2} =
          \top$, as follows:

          By assumption, $\lubstore{\extSRaw{S}{l}{d_2}}{S''} = \topS$.

          Hence, by Definition~\ref{def:lvars-lubstore}, there exists
          some $l' \in \dom{\extSRaw{S}{l}{d_2}} \cap \dom{S''}$ such
          that $\userlub{(\extSRaw{S}{l}{d_2})(l')}{S''(l')} = \top$.

          Now case on $l'$:

          \begin{itemize}
            \item $l' \neq l$:

              In this case, $(\extSRaw{S}{l}{d_2})(l') = S(l')$.

              Since $\userlub{(\extSRaw{S}{l}{d_2})(l')}{S''(l')} = \top$,
              we then have that $\userlub{S(l')}{S''(l')} = \top$.

              However, this is a contradiction since
              $\lubstore{S}{S''} \neq \topS$.

              Hence this case cannot occur.

            \item $l' = l$:

              Then $\userlub{(\extSRaw{S}{l}{d_2})(l)}{S''(l)} = \top$.

              Note that:
              \begin{align*}
                \top &= \userlub{(\extSRaw{S}{l}{d_2})(l)}{S''(l)} \\ &=
                \userlub{d_2}{S''(l)} \\ &=
                \userlub{\userlub{d_1}{d_2}}{S''(l)}
                \\ &=
                \userlub{\userlub{S(l)}{d_2}}{S''(l)}
                \\ &=
                \userlub{\userlub{S(l)}{S''(l)}}{d_2}
                \\ &=
                \userlub{(\lubstore{S}{S''})(l)}{d_2}
                \\ &= \userlub{d'_1}{d_2}.
              \end{align*}
              Hence $\userlub{d'_1}{d_2} = \top$.

              Hence, by {\sc E-Put-Err},
              $\config{\lubstore{S}{S''}}{\putexp{l}{d_2}} \parstepsto
              \error$.

              Hence $\config{\lubstore{S}{S''}}{\putexp{l}{d_2}}
              \parstepsto^i \error$, with $i = 1$.

          \end{itemize}

      \end{itemize}

    \item Case {\sc E-Get}:

      Given: $\config{S}{\getexp{l}{T}} \parstepsto \config{S}{d_2}$.

      To show: $\config{\lubstore{S}{S''}}{\getexp{l}{T}}
      \parstepsto^i \error$, where $i \leq 1$.

      By assumption, $\lubstore{S}{S''} = \topS$.

      Hence, by the definition of $\error$,
      $\config{\lubstore{S}{S''}}{\getexp{l}{T}} = \error$.

      Hence $\config{\lubstore{S}{S''}}{\getexp{l}{T}} \parstepsto^i
      \error$, with $i = 0$.
  \end{itemize}
\end{proof}


\section{Proof of Lemma~\ref{lem:lvars-error-preservation}}\label{section:lvars-error-preservation-proof}
\begin{proof}

  Given: $\config{S}{e} \parstepsto \error$ and $\leqstore{S}{S'}$.

  To show: $\config{S'}{e} \parstepsto \error$.

  \TODO{Figure out what to do here.  I think we need to handle both
    E-Eval-Ctxt and E-Put-Err.}
\end{proof}


\section{Proof of Lemma~\ref{lem:lvars-strong-local-confluence}}\label{section:lvars-strong-local-confluence-proof}
\begin{proof}
  Suppose $\conf \ctxstepsto \conf_a$ and $\conf \ctxstepsto \conf_b$.
  We have to show that there exist $\conf_c, i, j, \pi$ such that
  $\conf_a \ctxstepsto^i \conf_c$ and $\pi(\conf_b) \ctxstepsto^j
  \pi(\conf_c)$ and $i \leq 1$ and $j \leq 1$.

  By inspection of the operational semantics, it must be the case that
  $\conf$ steps to $\conf_a$ by the {\sc E-Eval-Ctxt} rule.  Let
  $\conf = \config{S}{\evalctxt{E_a}{e_{a_1}}}$ and let $\conf_a =
  \config{S_a}{\evalctxt{E_a}{e_{a_2}}}$.

  Likewise, it must be the case that $\conf$ steps to $\conf_b$ by the
  {\sc E-Eval-Ctxt} rule.  Let $\conf =
  \config{S}{\evalctxt{E_b}{e_{b_1}}}$ and let $\conf_b =
  \config{S_b}{\evalctxt{E_b}{e_{b_2}}}$.

  Note that $\conf = \config{S}{\evalctxt{E_a}{e_{a_1}}} =
  \config{S}{\evalctxt{E_b}{e_{b_1}}}$, and so
  $\evalctxt{E_a}{e_{a_1}} = \evalctxt{E_b}{e_{b_1}}$, but $E_a$ and
  $E_b$ may differ and $e_{a_1}$ and $e_{b_1}$ may differ.

  Since $\config{S}{\evalctxt{E_a}{e_{a_1}}} \ctxstepsto
  \config{S_a}{\evalctxt{E_a}{e_{a_2}}}$ and
  $\config{S}{\evalctxt{E_b}{e_{b_1}}} \ctxstepsto
  \config{S_b}{\evalctxt{E_b}{e_{b_2}}}$ and $\evalctxt{E_a}{e_{a_1}}
  = \evalctxt{E_b}{e_{b_1}}$, we have from
  Lemma~\ref{lem:lvars-locality} (Locality) that there exist
  evaluation contexts $E'_a$ and $E'_b$ such that:

  \begin{itemize}
  \item $\evalctxt{E'_a}{e_{a_1}} = \evalctxt{E_b}{e_{b_2}}$, and
  \item $\evalctxt{E'_b}{e_{b_1}} = \evalctxt{E_a}{e_{a_2}}$, and
  \item $\evalctxt{E'_a}{e_{a_2}} =
  \evalctxt{E'_b}{e_{b_2}}$.
  \end{itemize}

  Our approach will be to show that there exist $S', i, j, \pi$ such
  that:
  \begin{itemize}
  \item $\config{S_a}{\evalctxt{E_a}{e_{a_2}}} \ctxstepsto^i
    \config{S'}{\evalctxt{E'_a}{e_{a_2}}}$, and
  \item $\pi(\config{S_b}{\evalctxt{E_b}{e_{b_2}}}) \ctxstepsto^j
    \pi(\config{S'}{\evalctxt{E'_a}{e_{a_2}}})$.
  \end{itemize}
  Since $\evalctxt{E'_a}{e_{a_1}} = \evalctxt{E_b}{e_{b_2}}$,
  $\evalctxt{E'_b}{e_{b_1}} = \evalctxt{E_a}{e_{a_2}}$, and
  $\evalctxt{E'_a}{e_{a_2}} = \evalctxt{E'_b}{e_{b_2}}$, it suffices
  to show that:
  \begin{itemize}
  \item $\config{S_a}{\evalctxt{E'_b}{e_{b_1}}} \ctxstepsto^i
    \config{S'}{\evalctxt{E'_b}{e_{b_2}}}$, and
  \item $\pi(\config{S_b}{\evalctxt{E'_a}{e_{a_1}}}) \ctxstepsto^j
    \pi(\config{S'}{\evalctxt{E'_a}{e_{a_2}}})$.
  \end{itemize}

  From the premise of {\sc E-Eval-Ctxt}, we have that
  $\config{S}{e_{a_1}} \parstepsto \config{S_a}{e_{a_2}}$ and
  $\config{S}{e_{b_1}} \parstepsto \config{S_b}{e_{b_2}}$.  We proceed
  by case analysis on the rule by which $\config{S}{e_{a_1}}$ steps to
  $\config{S_a}{e_{a_2}}$.

  \begin{enumerate}
  \item Case {\sc E-Beta}:

    We have:
    \begin{itemize}
      \item $e_{a_1} = \app{\lam{x}{e'_a}}{v_a}$,
      \item $e_{a_2} = \subst{e'_a}{x}{v_a}$, and
      \item $S_a = S$.
    \end{itemize}

    Now, we proceed by case analysis on the rule by which
    $\config{S}{e_{b_1}}$ steps to $\config{S_b}{e_{b_2}}$:
    \begin{enumerate}
    \item Case {\sc E-Beta}:

      We have:
      \begin{itemize}
      \item $e_{b_1} = \app{\lam{x}{e'_b}}{v_b}$,
      \item $e_{b_2} = \subst{e'_b}{x}{v_b}$, and
      \item $S_b = S$.
      \end{itemize}

      Choose $S' = S$, $i = 1$, $j = 1$, and $\pi = \id$.

      We have to show that:

      \begin{itemize}
      \item $\config{S}{\evalctxt{E'_b}{e_{b_1}}} \ctxstepsto
        \config{S}{\evalctxt{E'_b}{e_{b_2}}}$, and
      \item $\config{S}{\evalctxt{E'_a}{e_{a_1}}} \ctxstepsto
        \config{S}{\evalctxt{E'_a}{e_{a_2}}}$, 
      \end{itemize}

      both of which follow immediately from $\config{S}{e_{a_1}}
      \parstepsto \config{S_a}{e_{a_2}}$ and $\config{S}{e_{b_1}}
      \parstepsto \config{S_b}{e_{b_2}}$ and {\sc E-Eval-Ctxt}.

    \item Case {\sc E-New}:

      We have:
      \begin{itemize}
      \item $e_{b_1} = \NEW$,
      \item $e_{b_2} = l$, and
      \item $S_b = \extSRaw{S}{l}{\bot}$.
      \end{itemize}

      Choose $S' = S_b$, $i = 1$, $j = 1$, and $\pi = \id$.

      We have to show that:

      \begin{itemize}
      \item $\config{S}{\evalctxt{E'_b}{e_{b_1}}} \ctxstepsto
        \config{S_b}{\evalctxt{E'_b}{e_{b_2}}}$, and
      \item
        $\config{S_b}{\evalctxt{E'_a}{e_{a_1}}} \ctxstepsto
        \config{S_b}{\evalctxt{E'_a}{e_{a_2}}}$.
      \end{itemize}

      The first of these follows immediately from $\config{S}{e_{b_1}}
      \parstepsto \config{S_b}{e_{b_2}}$ and {\sc E-Eval-Ctxt}.  For
      the second, consider that $S_b = \extSRaw{S}{l}{\bot} =
      \lubstore{S}{\store{\storebindingRaw{l}{\bot}}}$.  Furthermore, we
      know from the side condition of {\sc E-New} that $l \notin
      \dom{S}$, so $\store{\storebindingRaw{l}{\bot}}$ is non-conflicting
      with the transition $\config{S}{e_{a_1}} \parstepsto
      \config{S_a}{e_{a_2}}$, and we know that
      $\lubstore{S_a}{\store{\storebindingRaw{l}{\bot}}} \neq \topS$
      since $S_a$ is just $S$.  Therefore, by
      Lemma~\ref{lem:lvars-independence} (Independence), we have that
      $\config{\lubstore{S}{\store{\storebindingRaw{l}{\bot}}}}{e_{a_1}}
      \parstepsto
      \config{\lubstore{S_a}{\store{\storebindingRaw{l}{\bot}}}}{e_{a_2}}$.
      Hence $\config{S_b}{e_{a_1}} \parstepsto \config{S_b}{e_{a_2}}$.
      By {\sc E-Eval-Ctxt}, it follows that
      $\config{S_b}{\evalctxt{E'_a}{e_{a_1}}} \ctxstepsto
      \config{S_b}{\evalctxt{E'_a}{e_{a_2}}}$, as we were required to
      show.

    \item Case {\sc E-Put}: \TODO{}
    \item Case {\sc E-Put-Err}: \TODO{}
    \item Case {\sc E-Get}:\TODO{}
    \end{enumerate}
  \item Case {\sc E-New}:

    Now, we proceed by case analysis on the rule by which
    $\config{S}{e_{b_1}}$ steps to $\config{S_b}{e_{b_2}}$:
    \begin{enumerate}
    \item Case {\sc E-Beta}: \TODO{}
    \item Case {\sc E-New}: \TODO{}
    \item Case {\sc E-Put}: \TODO{}
    \item Case {\sc E-Put-Err}: \TODO{}
    \item Case {\sc E-Get}: \TODO{}
    \end{enumerate}
  \item Case {\sc E-Put}:

    Now, we proceed by case analysis on the rule by which
    $\config{S}{e_{b_1}}$ steps to $\config{S_b}{e_{b_2}}$:
    \begin{enumerate}
    \item Case {\sc E-Beta}: \TODO{}
    \item Case {\sc E-New}: \TODO{}
    \item Case {\sc E-Put}: \TODO{}
    \item Case {\sc E-Put-Err}: \TODO{}
    \item Case {\sc E-Get}: \TODO{}
    \end{enumerate}
  \item Case {\sc E-Put-Err}:

    Now, we proceed by case analysis on the rule by which
    $\config{S}{e_{b_1}}$ steps to $\config{S_b}{e_{b_2}}$:
    \begin{enumerate}
    \item Case {\sc E-Beta}: \TODO{}
    \item Case {\sc E-New}: \TODO{}
    \item Case {\sc E-Put}: \TODO{}
    \item Case {\sc E-Put-Err}: \TODO{}
    \item Case {\sc E-Get}: \TODO{}
    \end{enumerate}
  \item Case {\sc E-Get}:

    Now, we proceed by case analysis on the rule by which
    $\config{S}{e_{b_1}}$ steps to $\config{S_b}{e_{b_2}}$:
    \begin{enumerate}
    \item Case {\sc E-Beta}: \TODO{}
    \item Case {\sc E-New}: \TODO{}
    \item Case {\sc E-Put}: \TODO{}
    \item Case {\sc E-Put-Err}: \TODO{}
    \item Case {\sc E-Get}: \TODO{}
    \end{enumerate}
  \end{enumerate}

  \lk{I think we also still have to separately deal with cases where
    $\conf_a = \error$ or $\conf_b = \error$.}
\end{proof}


\section{Proof of Lemma~\ref{lem:lvars-strong-one-sided-confluence}}\label{section:lvars-strong-one-sided-confluence-proof}
\begin{proof}
  Suppose $\conf \ctxstepsto \conf'$ and $\conf \ctxstepsto^m
  \conf''$, where $1 \leq m$.  We have to show that there exist
  $\conf_c, i, j, \pi$ such that $\conf' \ctxstepsto^i \conf_c$ and
  $\pi(\conf'') \ctxstepsto^j \conf_c$ and $i \leq m$ and $j \leq 1$.

  We proceed by induction on $m$.  In the base case of $m = 1$, the
  result is immediate from
  Lemma~\ref{lem:lvars-strong-local-confluence}.

  For the induction step, suppose $\conf \ctxstepsto^m \conf''
  \ctxstepsto \conf'''$ and suppose the lemma holds for $m$.

  We show that it holds for $m + 1$, as follows.

  We are required to show that there exist $\conf_c, i, j, \pi$ such
  that $\conf' \ctxstepsto^{i} \conf_c$ and $\pi(\conf''')
  \ctxstepsto^{j} \conf_c$ and $i \leq m + 1$ and $j \leq 1$.

  From the induction hypothesis, there exist $\conf_c', i', j', \pi'$
  such that $\conf' \ctxstepsto^{i'} \conf_c'$ and $\pi'(\conf'')
  \ctxstepsto^{j'} \conf_c'$ and $i' \leq m$ and $j' \leq 1$.

  We proceed by cases on $j'$:
  \begin{itemize}

  \item If $j' = 0$, then $\pi'(\conf'') = \conf_c'$.

    Since $\conf'' \ctxstepsto \conf'''$, we have that $\pi'(\conf'')
    \ctxstepsto \pi'(\conf''')$ by
    Lemma~\ref{lem:lvars-permutability} (Permutability).

    We can then choose $\conf_c = \pi'(\conf''')$ and $i = i' + 1$ and
    $j = 0$ and $\pi = \pi'$.  The key is that $\conf'
    \ctxstepsto^{i'} \conf'_c = \pi'(\conf'') \ctxstepsto
    \pi'(\conf''')$ for a total of $i' + 1$ steps.
    
  \item If $j' = 1$:

    First, since $\pi'(\conf'') \ctxstepsto^{j'} \conf'_c$, then by
    Lemma~\ref{lem:lvars-permutability} (Permutability) we have that
    $\conf'' \ctxstepsto^{j'} \piprimeinv(\conf'_c)$.

    Then, by $\conf'' \ctxstepsto^{j'} \piprimeinv(\conf'_c)$ and
    $\conf'' \ctxstepsto \conf'''$ and
    Lemma~\ref{lem:lvars-strong-local-confluence} (Strong Local
    Confluence), we have that there exist $\conf_c''$ and $i''$ and
    $j''$ and $\pi''$ such that $\piprimeinv(\conf'_c)
    \ctxstepsto^{i''} \conf_c''$ and $\pi''(\conf''')
    \ctxstepsto^{j''} \conf_c''$ and $i'' \leq 1$ and $j'' \leq 1$.

    Since $\piprimeinv(\conf'_c) \ctxstepsto^{i''} \conf_c''$, by
    Lemma~\ref{lem:lvars-permutability} (Permutability) we have that
    $\conf'_c \ctxstepsto^{i''} \pi'(\conf_c'')$.

    So we also have $\conf' \ctxstepsto^{i'} \conf_c'
    \ctxstepsto^{i''} \pi'(\conf_c'')$.

    Since $\pi''(\conf''') \ctxstepsto^{j''} \conf_c''$, by
    Lemma~\ref{lem:lvars-permutability} (Permutability) we have that
    $\pi'(\pi''(\conf''')) \ctxstepsto^{j''} \pi'(\conf_c'')$.

    In summary, we pick $\conf_c = \pi'(\conf_c'')$ and $i = i' + i''$
    and $j = j''$ and $\pi = \pi'' \circ \pi'$, which is sufficient
    because $i = i' + i'' \leq m + 1$ and $j = j'' \leq 1$.
  \end{itemize}

 \end{proof}


\section{Proof of Lemma~\ref{lem:lvars-strong-confluence}}\label{section:lvars-strong-confluence-proof}
\begin{proof}
  We proceed by induction on $n$.  In the base case of $n = 1$, the
  result is immediate from
  Lemma~\ref{lem:lvars-strong-one-sided-confluence}.

  For the induction step, suppose $\conf \parstepsto^n \conf'
  \parstepsto \conf'''$ and suppose the lemma holds for $n$.

  We show that it holds for $n + 1$, as follows.

  We are required to show that there exist $\conf_c, i, j$ such that
  $\conf''' \parstepsto^i \conf_c$ and $\conf'' \parstepsto^j \conf_c$
  and $i \leq m$ and $j \leq n + 1$.

  From the induction hypothesis, we have that there exist $\conf'_c,
  i', j'$ such that $\conf' \parstepsto^{i'} \conf'_c$ and $\conf''
  \parstepsto^{j'} \conf'_c$ and $i' \leq m$ and $j' \leq n$.

  We proceed by cases on $i'$:
  \begin{itemize}

  \item If $i' = 0$, then $\conf' = \conf_c'$.  We can then choose
    $\conf_c = \conf'''$ and $i = 0$ and $j = j' + 1$.

  \item If $i' \geq 1$:

    From $\conf' \parstepsto \conf'''$ and $\conf' \parstepsto^{i'}
    \conf_c'$ and Lemma~\ref{lem:lvars-strong-one-sided-confluence},
    we have that there exist $\conf_c''$ and $i''$ and $j''$ such that
    $\conf''' \parstepsto^{i''} \conf_c''$ and $\conf_c'
    \parstepsto^{j''} \conf_c''$ and $i'' \leq i'$ and $j'' \leq 1$.
    So we also have $\conf'' \parstepsto^{j'} \conf_c'
    \parstepsto^{j''} \conf_c''$.  In summary, we pick $\conf_c =
    \conf_c''$ and $i = i''$ and $j = j' + j''$, which is sufficient
    because $i = i'' \leq i' \leq m$ and $j = j' + j'' \leq n + 1$.
  \end{itemize}

\end{proof}


\section{Proof of Lemma~\ref{lem:lattice-structure}}\label{section:lattice-structure-proof}
\begin{proof}
  Suppose that $(D, \userleq, \bot, \top)$ is a lattice and $(D_p,
  \leqp, \botp, \topp) = \Freeze{D, \userleq, \bot, \top}$.

  In order to show that $(D_p, \leqp, \botp, \topp)$ is a lattice, we
  have to show that:
  \begin{enumerate}
  \item $\leqp$ is a partial order over $D_p$.

  \item Every nonempty finite subset of $D_p$ has a lub.

  \item $\botp$ is the least element of $D_p$.

  \item $\topp$ is the greatest element of $D_p$.
  \end{enumerate}

  We prove each of these properties in turn:

  \begin{enumerate}
  \item $\leqp$ is a partial order over $D_p$.

    To show this, we need to show that $\leqp$ is reflexive, transitive,
    and antisymmetric. 
    \begin{enumerate}
    \item $\leqp$ is reflexive.

      Suppose $v \in D_p$.

      Then, by Lemma~\ref{lem:partition-of-Dp}, either $v =
      \state{d}{\frozenfalse}$ with $d \in D$, or $v =
      \state{x}{\frozentrue}$ with $x \in X$, where $X = D -
      \setof{\top}$.
      \begin{itemize}
      \item Suppose $v = \state{d}{\frozenfalse}$:

        By the reflexivity of $\userleq$, we know $d \userleq d$.

        By the definition of $\leqp$, we know $\state{d}{\frozenfalse}
        \leqp \state{d}{\frozenfalse}$.

      \item Suppose $v = \state{x}{\frozentrue}$: 
        
        By the reflexivity of equality, $x = x$.

        By the definition of $\leqp$, we know $\state{x}{\frozentrue}
        \leqp \state{x}{\frozentrue}$.
      \end{itemize}

    \item $\leqp$ is transitive. 

      Suppose $v_1 \leqp v_2$ and $v_2 \leqp v_3$.

      We want to show that $v_1 \leqp v_3$.

      We proceed by case analysis on $v_1, v_2$, and $v_3$.
      \begin{itemize}
      \item Case $v_1 = \state{d_1}{\frozenfalse}$ and $v_2 =
        \state{d_2}{\frozenfalse}$ and $v_3 =
        \state{d_3}{\frozenfalse}$:
        
        By inversion on $\leqp$, it follows that $d_1 \userleq d_2$.

        By inversion on $\leqp$, it follows that $d_2 \userleq d_3$.

        By the transitivity of $\userleq$, we know $d_1 \userleq d_3$.

        By the definition of $\leqp$, it follows that
        $\state{d_1}{\frozenfalse} \leqp \state{d_3}{\frozenfalse}$.

        Hence $v_1 \leqp v_3$.

      \item Case $v_1 = \state{d_1}{\frozenfalse}$ and $v_2 =
        \state{d_2}{\frozenfalse}$ and $v_3 =
        \state{x_3}{\frozentrue}$:

        By inversion on $\leqp$, it follows that $d_1 \userleq d_2$.

        By inversion on $\leqp$, it follows that $d_2 \userleq x_3$.

        By the transitivity of $\userleq$, we know $d_1 \userleq x_3$.

        By the definition of $\leqp$, it follows that
        $\state{d_1}{\frozenfalse} \leqp \state{x_3}{\frozentrue}$.

        Hence $v_1 \leqp v_3$.

      \item Case $v_1 = \state{d_1}{\frozenfalse}$ and $v_2 =
        \state{x_2}{\frozentrue}$ and $v_3 =
        \state{d_3}{\frozenfalse}$:

        By inversion on $\leqp$, it follows that $d_1 \userleq x_2$.

        By inversion on $\leqp$, it follows that $d_3 = \top$.

        Since $\top$ is the maximal element of $D$, we know $d_1
        \userleq \top \equiv d_3$.

        By the definition of $\leqp$, it follows that
        $\state{d_1}{\frozenfalse} \leqp \state{d_3}{\frozenfalse}$.

        Hence $v_1 \leqp v_3$.

      \item Case $v_1 = \state{d_1}{\frozenfalse}$ and $v_2 =
        \state{x_2}{\frozentrue}$ and $v_3 =
        \state{x_3}{\frozentrue}$:

        By inversion on $\leqp$, it follows that $d_1 \userleq x_2$.

        By inversion on $\leqp$, it follows that $x_2 = x_3$.

        Hence $d_1 \userleq x_3$.

        By the definition of $\leqp$, it follows that
        $\state{d_1}{\frozenfalse} \leqp \state{x_3}{\frozentrue}$.

        Hence $v_1 \leqp v_3$.

      \item Case $v_1 = \state{x_1}{\frozentrue}$ and $v_2 =
        \state{d_2}{\frozenfalse}$ and $v_3 =
        \state{d_3}{\frozenfalse}$:

        By inversion on $\leqp$, it follows that $d_2 = \top$.

        By inversion on $\leqp$, it follows that $d_2 \userleq d_3$.

        Since $\top$ is maximal, it follows that $d_3 = \top$.

        By the definition of $\leqp$, it follows that
        $\state{x_1}{\frozentrue} \leqp \state{d_3}{\frozenfalse}$.

        Hence $v_1 \leqp v_3$. 

      \item Case $v_1 = \state{x_1}{\frozentrue}$ and $v_2 =
        \state{d_2}{\frozenfalse}$ and $v_3 =
        \state{x_3}{\frozentrue}$:

        By inversion on $\leqp$, it follows that $d_2 = \top$.

        By inversion on $\leqp$, it follows that $d_2 \userleq x_3$.

        Since $\top$ is maximal, it follows that $x_3 = \top$.

        But since $x_3 \in X \subseteq D/\setof{\top}$, we know $x_3
        \not= \top$.

        This is a contradiction. \\

        Hence $v_1 \leqp v_3$. 

      \item Case $v_1 = \state{x_1}{\frozentrue}$ and $v_2 =
        \state{x_2}{\frozentrue}$ and $v_3 =
        \state{d_3}{\frozenfalse}$:

        By inversion on $\leqp$, it follows that $x_1 = x_2$.

        By inversion on $\leqp$, it follows that $d_3 = \top$.

        By the definition of $\leqp$, it follows that
        $\state{x_1}{\frozentrue} \leqp \state{d_3}{\frozenfalse}$.

        Hence $v_1 \leqp v_3$. 

      \item Case $v_1 = \state{x_1}{\frozentrue}$ and $v_2 =
        \state{x_2}{\frozentrue}$ and $v_3 =
        \state{x_3}{\frozentrue}$:

        By inversion on $\leqp$, it follows that $x_1 = x_2$.

        By inversion on $\leqp$, it follows that $x_2 = x_3$.

        By transitivity of $=$, $x_1 = x_3$.

        By the definition of $\leqp$, it follows that
        $\state{x_1}{\frozentrue} \leqp \state{x_3}{\frozentrue}$.

        Hence $v_1 \leqp v_3$. 
        
      \end{itemize}

    \item $\leqp$ is antisymmetric. 

      Suppose $v_1 \leqp v_2$ and $v_2 \leqp v_1$. Now, we proceed by
      cases on $v_1$ and $v_2$.
      \begin{itemize}
      \item Case $v_1 = \state{d_1}{\frozenfalse}$ and $v_2 =
        \state{d_2}{\frozenfalse}$:
        
        By inversion on $v_1 \leqp v_2$, we know that $d_1 \userleq
        d_2$.

        By inversion on $v_2 \leqp v_1$, we know that $d_2 \userleq
        d_1$.

        By the antisymmetry of $\leq$, we know $d_1 = d_2$.

        Hence $v_1 = v_2$. 

      \item Case $v_1 = \state{d_1}{\frozenfalse}$ and $v_2 =
        \state{x_2}{\frozentrue}$:

        By inversion on $v_1 \leqp v_2$, we know that $d_1 \userleq x_2$.

        By inversion on $v_2 \leqp v_1$, we know that $d_1 = \top$.

        Since $\top$ is maximal in $D$, we know $x_2 = \top$.

        But since $x_2 \in X \subseteq D/\setof{\top}$, we know $x_2 \not= \top$.

        This is a contradiction.

        Hence $v_1 = v_2$. 
        
      \item Case $v_1 = \state{x_1}{\frozentrue}$ and $v_2 =
        \state{d_2}{\frozenfalse}$:

        Similar to the previous case. 

      \item Case $v_1 = \state{x_1}{\frozentrue}$ and $v_2 =
        \state{x_2}{\frozentrue}$:

        By inversion on $v_1 \leqp v_2$, we know that $x_1 = x_2$.

        Hence $v_1 = v_2$. 
      \end{itemize}
    \end{enumerate}

  \item Every nonempty finite subset of $D_p$ has a lub.

    To show this, it is sufficient to show that every two elements of
    $D_p$ have a lub, since a binary lub operation can be repeatedly
    applied to compute the lub of any finite set.

    We will show that every two elements of $D_p$ have a lub by
    showing that the $\lubp{}{}$ operation defined by
    Definition~\ref{def:lubp} computes their lub.

    It suffices to show the following two properties:
    \begin{enumerate}
    \item For all $v_1, v_2, v \in D_p$, if $v_1 \leqp v$ and $v_2
      \leqp v$, then $(\lubp{v_1}{v_2}) \leqp v$.
    \item For all $v_1, v_2 \in D_p$, $v_1 \leqp (\lubp{v_1}{v_2})$
      and $v_2 \leqp (\lubp{v_1}{v_2})$.
    \end{enumerate}
    \begin{enumerate}
    \item For all $v_1, v_2, v \in D_p$, if $v_1 \leqp v$ and $v_2
      \leqp v$, then $\lubp{v_1}{v_2} \leqp v$.
      
      Assume $v_1, v_2, v \in D_p$, and $v_1 \leqp v$ and $v_2 \leqp
      v$.

      Now we do a case analysis on $v_1$ and $v_2$.
      \begin{itemize}
      \item Case $v_1 = \state{d_1}{\frozenfalse}$ and $v_2 =
        \state{d_2}{\frozenfalse}$.
        
        Now case on $v$: 
        \begin{itemize}
        \item Case $v = \state{d}{\frozenfalse}$: 

          By the definition of $\lubp{}{}$,
          $\lubp{\state{d_1}{\frozenfalse}}{\state{d_2}{\frozenfalse}}
          = \state{\userlub{d_1}{d_2}}{\frozenfalse}$.

          By inversion on $\state{d_1}{\frozenfalse} \leqp
          \state{d}{\frozenfalse}$, $d_1 \userleq l$.

          By inversion on $\state{d_2}{\frozenfalse} \leqp
          \state{d}{\frozenfalse}$, $d_2 \userleq l$.

          Hence $l$ is an upper bound for $d_1$ and $d_2$.

          Hence $\userlub{d_1}{d_2} \userleq l$.

          Hence $\state{\userlub{d_1}{d_2}}{\frozenfalse} \leqp
          \state{d}{\frozenfalse}$.

          Hence $\lubp{v_1}{v_2} \leqp v$.
          
        \item Case $v = \state{x}{\frozentrue}$: 
          
          By the definition of $\lubp{}{}$, $\state{d_1}{\frozenfalse}
          \lubp{}{} \state{d_2}{\frozenfalse} =
          \state{\userlub{d_1}{d_2}}{\frozenfalse}$.

          By inversion on $\state{d_1}{\frozenfalse} \leqp
          \state{x}{\frozentrue}$, $d_1 \userleq x$.

          By inversion on $\state{d_2}{\frozenfalse} \leqp
          \state{x}{\frozentrue}$, $d_2 \userleq x$.
     
          Hence $x$ is an upper bound for $d_1$ and $d_2$.

          Hence $\userlub{d_1}{d_2} \userleq x$.

          Hence $\state{\userlub{d_1}{d_2}}{\frozenfalse} \leqp
          \state{x}{\frozentrue}$.

          Hence $\lubp{v_1}{v_2} \leqp v$.
        \end{itemize}
        
      \item Case $v_1 = \state{x_1}{\frozentrue}$ and $v_2 =
        \state{x_2}{\frozentrue}$:
        
        Now case on $v$: 
        \begin{itemize}
        \item Case $v = \state{d}{\frozenfalse}$: 
          
          By inversion on $\state{x_1}{\frozentrue} \leqp
          \state{d}{\frozenfalse}$, we know $l = \top$.

          By inversion on $\state{x_2}{\frozentrue} \leqp
          \state{d}{\frozenfalse}$, we know $l = \top$.

          Now consider whether $x_1 = x_2$ or not.
        
          If it does, then by the definition of $\lubp{}{}$,
          $\state{x_1}{\frozentrue} \lubp{}{} \state{x_2}{\frozentrue}
          = \state{x_1}{\frozentrue}$.

          By definition of $\leqp$, we have $\state{x_1}{\frozentrue}
          \leqp \state{\top}{\frozenfalse}$.

          So $\lubp{v_1}{v_2} \leqp v$.

          If it does not, then $\lubp{v_1}{v_2} =
          \state{\top}{\frozenfalse}$.

          By the definition of $\leqp$, we have
          $\state{\top}{\frozenfalse} \leqp
          \state{\top}{\frozenfalse}$.

          So $\lubp{v_1}{v_2} \leqp v$.
          
        \item Case $v = \state{x}{\frozentrue}$: 
          
          By inversion on $\state{x_1}{\frozentrue} \leqp
          \state{x}{\frozentrue}$, we know $x = x_1$.

          By inversion on $\state{x_2}{\frozentrue} \leqp
          \state{x}{\frozentrue}$, we know $x = x_2$.

          Hence $x_1 = x_2$.

          By the definition of $\lubp{}{}$, $\state{x_1}{\frozentrue}
          \lubp{}{} \state{x_2}{\frozentrue} =
          \state{x_1}{\frozentrue}$.

          Hence $\lubp{v_1}{v_2} \leqp v$.
        \end{itemize}
        
      \item Case $v_1 = \state{x_1}{\frozentrue}$ and $v_2 =
        \state{d_2}{\frozenfalse}$:
        
        Now case on $v$:
        \begin{itemize}
        \item Case $v = \state{d}{\frozenfalse}$:
          
          Now consider whether $d_2 \userleq x_1$.

          If it is, then $\state{x_1}{\frozentrue} \lubp{}{}
          \state{d_2}{\frozenfalse} = \state{x_1}{\frozentrue} = v_1$.

          Hence $\lubp{v_1}{v_2} \leqp v$.

          Otherwise, $\state{x_1}{\frozentrue} \lubp{}{}
          \state{d_2}{\frozenfalse} = \state{\top}{\frozenfalse}$.

          By inversion on $\state{x_1}{\frozentrue} \leqp
          \state{d}{\frozenfalse}$, we know $l = \top$.

          By reflexivity, $\state{\top}{\frozenfalse} \leqp
          \state{\top}{\frozenfalse}$.

          Hence $\lubp{v_1}{v_2} \leqp v$. 
          
        \item Case $v = \state{x}{\frozentrue}$:  
          
          By inversion on $\state{x_1}{\frozentrue} \leqp
          \state{x}{\frozentrue}$, we know that $x_1 = x$.

          By inversion on $\state{d_2}{\frozenfalse} \leqp
          \state{x}{\frozentrue}$, we know that $d_2 \userleq x$.

          By transitivity, $d_2 \userleq x_1$.

          By the definition of $\lubp{}{}$, it follows that
          $\state{x_1}{\frozentrue} \lubp{}{}
          \state{d_2}{\frozenfalse} = \state{x_1}{\frozentrue}$.

          By definition of $\leqp$, $\state{x_1}{\frozentrue} \leqp
          \state{x_1}{\frozentrue}$.

          Hence $\lubp{v_1}{v_2} \leqp v$. 
        \end{itemize}
        
      \item Case $v_1 = \state{d_1}{\frozenfalse}$ and $v_2 =
        \state{x_2}{\frozentrue}$:
        
        Symmetric with the previous case. 
      \end{itemize}
    \item For all $v_1, v_2 \in D_p$, $v_1 \leqp \lubp{v_1}{v_2}$ and
      $v_2 \leqp \lubp{v_1}{v_2}$.
      
      Assume $v_1, v_2 \in D_p$, and proceed by case analysis. 
      \begin{itemize}
      \item Case $v_1 = \state{d_1}{\frozenfalse}$ and $v_2 =
        \state{d_2}{\frozenfalse}$:

        Since $\userlub{}{}$ is a join operator, we know $d_1 \userleq
        \userlub{d_1}{d_2}$.

        By the definition of $\leqp$, $\state{d_1}{\frozenfalse}
        \userleq \state{\userlub{d_1}{d_2}}{\frozenfalse}$.

        By the definition of $\lubp{}{}$, $\lubp{v_1}{v_2} =
        \state{\userlub{d_1}{d_2}}{\frozenfalse}$.

        Hence $v_1 \leqp \lubp{v_1}{v_2}$.

        Since $\userlub{}{}$ is a join operator, we know $d_1 \userleq
        \userlub{d_1}{d_2}$.

        By the definition of $\leqp$, $\state{d_2}{\frozenfalse}
        \userleq \state{\userlub{d_1}{d_2}}{\frozenfalse}$.

        By the definition of $\lubp{}{}$, $\lubp{v_1}{v_2} =
        \state{\userlub{d_1}{d_2}}{\frozenfalse}$.

        Hence $v_2 \leqp \lubp{v_1}{v_2}$. 

        Therefore $v_1 \leqp v_1 \userlub{}{} v_2$ and $v_2 \leqp v_1
        \userlub{}{} v_2$.
 
      \item Case $v_1 = \state{d_1}{\frozenfalse}$ and $v_2 = \state{x_2}{\frozentrue}$:

        Consider whether $d_1 \userleq x_2$. 
        \begin{itemize}
        \item Case  $d_1 \userleq x_2$:

          By the definition of $\lubp{}{}$, we know
          $\state{d_1}{\frozenfalse} \lubp{}{}
          \state{x_2}{\frozentrue} = \state{x_2}{\frozentrue}$.

          By the definition of $\lubp{}{}$, we know
          $\state{d_1}{\frozenfalse} \leqp \state{x_2}{\frozentrue}$.

          Hence $v_1 \leqp \lubp{v_1}{v_2}$.

          By reflexivity, $\state{x_2}{\frozentrue} \leqp
          \state{x_2}{\frozentrue}$.

          Hence $v_2 \leqp \lubp{v_1}{v_2}$.

          Therefore $v_1 \leqp v_1 \userlub{}{} v_2$ and $v_2 \leqp
          v_1 \userlub{}{} v_2$.

        \item Case $d_1 \not\userleq x_2$:

          By the definition of $\lubp{}{}$, we know
          $\state{d_1}{\frozenfalse} \lubp{}{}
          \state{x_2}{\frozentrue} = \state{\top}{\frozenfalse}$.

          Since $d_1 \userleq \top$, by the definition of $\leqp$ we
          know $\state{d_1}{\frozenfalse} \userleq
          \state{\top}{\frozenfalse}$.

          Hence $v_1 \leqp \lubp{v_1}{v_2}$.

          By the definition of $\leqp$, we know
          $\state{x_2}{\frozentrue} \userleq
          \state{\top}{\frozenfalse}$.

          Hence $v_2 \leqp \lubp{v_1}{v_2}$.

          Therefore $v_1 \leqp v_1 \userlub{}{} v_2$ and $v_2 \leqp
          v_1 \userlub{}{} v_2$.
        \end{itemize}
      \item Case $v_1 = \state{x_1}{\frozentrue}$ and $v_2 =
        \state{d_2}{\frozenfalse}$:

        Symmetric with the previous case. 
      \item Case $v_1 = \state{x_1}{\frozentrue}$ and $v_2 =
        \state{x_2}{\frozentrue}$:

        Consider whether $x_1$ equals $x_2$. 
        \begin{itemize}
        \item Case $x_1 = x_2$:
          
          By the definition $\lubp{}{}$, $\state{x_1}{\frozentrue}
          \lubp{}{} \state{x_2}{\frozentrue} =
          \state{x_1}{\frozentrue}$.
 
          By reflexivity, $\state{x_1}{\frozentrue} \leqp
          \state{x_1}{\frozentrue}$.

          Hence $v_1 \leqp \lubp{v_1}{v_2}$.

          By reflexivity, $\state{x_2}{\frozentrue} \leqp
          \state{x_1}{\frozentrue}$.

          Hence $v_2 \leqp \lubp{v_1}{v_2}$.

          Therefore $v_1 \leqp v_1 \userlub{}{} v_2$ and $v_2 \leqp
          v_1 \userlub{}{} v_2$.

        \item Case $x_1 \not= x_2$: 

          By the definition $\lubp{}{}$, $\state{x_1}{\frozentrue}
          \lubp{}{} \state{x_2}{\frozentrue} =
          \state{\top}{\frozenfalse}$.

          By the definition of $\leqp$, $\state{x_1}{\frozentrue}
          \leqp \state{\top}{\frozenfalse}$.

          Hence $v_1 \leqp \lubp{v_1}{v_2}$.

          By the definition of $\leqp$, $\state{x_2}{\frozentrue}
          \leqp \state{\top}{\frozenfalse}$.

          Hence $v_2 \leqp \lubp{v_1}{v_2}$.

          Therefore $v_1 \leqp v_1 \userlub{}{} v_2$ and $v_2 \leqp
          v_1 \userlub{}{} v_2$.
        \end{itemize}
      \end{itemize}
    \end{enumerate}

  \item $\botp$ is the least element of $D_p$. 

    $\botp$ is defined to be $\state{\bot}{\frozenfalse}$.

    In order to be the least element of $D_p$, it must be less than or
    equal to every element of $D_p$.

    By Lemma~\ref{lem:partition-of-Dp}, the elements of $D_p$
    partition into $\state{d}{\frozenfalse}$ for all $d \in D$, and
    $\state{x}{\frozentrue}$ for all $x \in X$, where $X = D -
    \setof{\top}$.

    We consider both cases:

    \begin{itemize}
    \item $\state{d}{\frozenfalse}$ for all $d \in D$:

      By the definition of $\leqp$, $\state{\bot}{\frozenfalse} \leqp
      \state{d}{\frozenfalse}$ iff $\bot \userleq d$.

      Since $\bot$ is the least element of $D$, $\bot \userleq d$.

      Therefore $\botp = \state{\bot}{\frozenfalse} \leqp
      \state{d}{\frozenfalse}$.

    \item $\state{x}{\frozentrue}$ for all $x \in X$:

      By the definition of $\leqp$, $\state{\bot}{\frozenfalse} \leqp
      \state{x}{\frozentrue}$ iff $\bot \userleq x$.

      Since $\bot$ is the least element of $D$, $\bot \userleq x$.

      Therefore $\botp = \state{\bot}{\frozenfalse} \leqp
      \state{x}{\frozentrue}$.

    \end{itemize}

    Therefore $\botp$ is less than or equal to all elements of $D_p$.

  \item $\topp$ is the greatest element of $D_p$.

    $\topp$ is defined to be $\state{\top}{\frozenfalse}$.

    In order to be the greatest element of $D_p$, every element of
    $D_p$ must be less than or equal to it.

    By Lemma~\ref{lem:partition-of-Dp}, the elements of $D_p$
    partition into $\state{d}{\frozenfalse}$ for all $d \in D$, and
    $\state{x}{\frozentrue}$ for all $x \in X$, where $X = D -
    \setof{\top}$.

    We consider both cases:

    \begin{itemize}
    \item $\state{d}{\frozenfalse}$ for all $d \in D$:

      By the definition of $\leqp$, $\state{d}{\frozenfalse} \leqp
      \state{\top}{\frozenfalse}$ iff $d \userleq \top$.

      Since $\top$ is the greatest element of $D$, $d \userleq \top$.

      Therefore $\state{d}{\frozenfalse} \leqp
      \state{\top}{\frozenfalse} = \topp$.

    \item $\state{x}{\frozentrue}$ for all $x \in X$:

      By the definition of $\leqp$, $\state{x}{\frozentrue} \leqp
      \state{\top}{\frozenfalse}$ iff $\top \userleq \top$.

      Therefore $\state{x}{\frozentrue} \leqp
      \state{\top}{\frozenfalse} = \topp$.

    \end{itemize}

    Therefore all elements of $D_p$ are less than or equal to $\topp$.
  \end{enumerate}
\end{proof}


\section{Proof of Lemma~\ref{lem:monotonicity}}\label{section:monotonicity-proof}
\begin{proof}
  \TODO{Fix the typos I found in this.}

  \begin{itemize}

    \item Case {\sc E-Eval-Ctxt}:

      Given: $\config{S}{\E{e}} \parstepsto \config{S'}{\E{e'}}$.

      To show: $\leqstore{S}{S'}$.

      From the premise of {\sc E-Eval-Ctxt}, $\config{S}{e}
      \parstepsto \config{S'}{e'}$.

      Hence by IH, $\leqstore{S}{S'}$, as we were required to show.

    \item Case {\sc E-Beta}:

      Immediate by the definition of $\leqstore{}{}$, since $S$ does
      not change.

    \item Case {\sc E-New}:

      Given: $\config{S}{\NEW} \parstepsto
      \config{\extS{S}{l}{\bot}{\frozenfalse}}{l}$.

      To show: $\leqstore{S}{\extS{S}{l}{\bot}{\frozenfalse}}$.

      By Definition~\ref{def:leqstore}, we have to show that $\dom{S}
      \subseteq \dom{\extS{S}{l}{\bot}{\frozenfalse}}$ and that for
      all $l' \in \dom{S}, \\
      S(l') \leqp (\extS{S}{l}{\bot}{\frozenfalse})(l')$.

      By the definition of store update,
      $\extS{S}{l}{d_1}{\frozentrue}$ can only either update an
      existing binding in $S$ or extend $S$ with a new binding.

      Hence $\dom{S} \subseteq \dom{\extS{S}{l}{\bot}{\frozenfalse}}$.

      From the side condition of {\sc E-New}, $l \notin \dom{S}$.

      Hence $\extS{S}{l}{\bot}{\frozenfalse}$ adds a new binding for
      $l$ in $S$.

      Hence $\extS{S}{l}{d_1}{\frozentrue}$ does not update any
      existing bindings in $S$.

      Hence, for all $l' \in \dom{S}, S(l') \leqp
      (\extS{S}{l}{d_1}{\frozentrue})(l')$.

      Therefore $\leqstore{S}{\extS{S}{l}{\bot}{\frozenfalse}}$, as
      required.

    \item Case {\sc E-Put}:

      Given: $\config{S}{\putexp{l}{d_2}} \parstepsto
      \config{\extSRaw{S}{l}{p_2}}{\unit}$.

      To show: $\leqstore{S}{\extSRaw{S}{l}{p_2}}$.

      By Definition~\ref{def:leqstore}, we have to show that $\dom{S}
      \subseteq \dom{\extSRaw{S}{l}{p_2}}$ and that for all $l' \in
      \dom{S}, \\
      S(l') \leqp (\extSRaw{S}{l}{p_2})(l')$.

      By the definition of store update, $\extSRaw{S}{l}{p_2}$ can only
      either update an existing binding in $S$ or extend $S$ with a
      new binding.

      Hence $\dom{S} \subseteq \dom{\extSRaw{S}{l}{p_2}}$.

      From the premises of {\sc E-Put}, $S(l) = p_1$.  Therefore $l
      \in \dom{S}$.

      Hence $\extSRaw{S}{l}{p_2}$ updates the existing binding for $l$
      in $S$ from $p_1$ to $p_2$.

      From the premises of {\sc E-Put}, $p_2 =
      \lubp{p_1}{\state{d_2}{\frozenfalse}}$.

      Hence, by the definition of $\lubp{}{}$, $p_1 \leqp p_2$.

      $\extSRaw{S}{l}{p_2}$ does not update any other bindings in $S$,
      hence, for all $l' \in \dom{S}, S(l') \leqp
      (\extSRaw{S}{l}{p_2})(l')$.

      Hence $\leqstore{S}{\extSRaw{S}{l}{p_2}}$, as required.

    \item Case {\sc E-Put-Err}:

      Given: $\config{S}{\putexp{l}{d_2}} \parstepsto \error$.

      By the definition of $\error$, $\error = \config{\topS}{e}$ for
      any $e$.

      To show: $\leqstore{S}{\topS}$.

      Immediate by the definition of $\leqstore{}{}$.

    \item Case {\sc E-Get}:

      Immediate by the definition of $\leqstore{}{}$, since $S$ does
      not change.

    \item Case {\sc E-Freeze-Init}:

      Immediate by the definition of $\leqstore{}{}$, since $S$ does
      not change.

    \item Case {\sc E-Spawn-Handler}:

      Immediate by the definition of $\leqstore{}{}$, since $S$ does
      not change.

    \item Case {\sc E-Freeze-Final}:

      Given: $\config{S}{\freezeafterfull{l}{Q}{v}{\setof{v\dots}}{H}}
      \parstepsto \config{\extS{S}{l}{d_1}{\frozentrue}}{d_1}$.

      To show: $\leqstore{S}{\extS{S}{l}{d_1}{\frozentrue}}$.

      By Definition~\ref{def:leqstore}, we have to show that $\dom{S}
      \subseteq \dom{\extS{S}{l}{d_1}{\frozentrue}}$ and that for all
      $l' \in \dom{S}, \\
      S(l') \leqp (\extS{S}{l}{d_1}{\frozentrue})(l')$.

      \lk{We could spell this out in even more excruciating detail,
        but I think it's obvious enough.}

      By the definition of store update,
      $\extS{S}{l}{d_1}{\frozentrue}$ can only either update an
      existing binding in $S$ or extend $S$ with a new binding.

      Hence $\dom{S} \subseteq \dom{\extS{S}{l}{d_1}{\frozentrue}}$.

      From the premises of {\sc E-Freeze-Final}, $S(l) =
      \state{d_1}{\status_1}$.  Therefore $l \in \dom{S}$.

      Hence $\extS{S}{l}{d_1}{\frozentrue}$ updates the existing
      binding for $l$ in $S$ from $\state{d_1}{\status_1}$ to
      $\state{d_1}{\frozentrue}$.

      By the definition of $\leqp$, $\state{d_1}{\status_1} \leqp
      \state{d_1}{\frozentrue}$.

      $\extS{S}{l}{d_1}{\frozentrue}$ does not update any other
      bindings in $S$, hence, for all $l' \in \dom{S}, \\
      S(l') \leqp (\extS{S}{l}{d_1}{\frozentrue})(l')$.

      Hence $\leqstore{S}{\extS{S}{l}{d_1}{\frozentrue}}$, as
      required.

    \item Case {\sc E-Freeze-Simple}:

      Given: $\config{S}{\freeze{l}} \parstepsto
      \config{\extS{S}{l}{d_1}{\frozentrue}}{d_1}$.

      To show: $\leqstore{S}{\extS{S}{l}{d_1}{\frozentrue}}$.

      Similar to the previous case.

  \end{itemize}

\end{proof}


\section{Proof of Lemma~\ref{lem:independence}}\label{section:independence-proof}
\begin{proof}
  Consider arbitrary $S''$ such that $S''$ is non-conflicting with
  $\config{S}{e} \parstepsto \config{S'}{e'}$ and $\lubstore{S'}{S''}
  \statuseq S$ and $\lubstore{S'}{S''} \neq \topS$.

  To show: $\config{\lubstore{S}{S''}}{e} \parstepsto
  \config{\lubstore{S'}{S''}}{e'}$.

  The proof is by cases on the rule of the reduction semantics by
  which $\config{S}{e}$ steps to $\config{S'}{e'}$.  Since
  $\config{S'}{e'} \neq \error$, we do not need to consider the {\sc
    E-Put-Err} rule.

  The assumption that $\lubstore{S'}{S''} \statuseq S$ is only needed
  in the {\sc E-Freeze-Final} and {\sc E-Freeze-Simple} cases.

  \begin{itemize}

    \item Case {\sc E-Beta}:

      Given: $\config{S}{\app{(\lam{x}{e})}{v}} \parstepsto
      \config{S}{\subst{e}{x}{v}}$.

      To show: $\config{\lubstore{S}{S''}}{\app{(\lam{x}{e})}{v}} \parstepsto
      \config{\lubstore{S}{S''}}{\subst{e}{x}{v}}$.

      Immediate by {\sc E-Beta}.

    \item Case {\sc E-New}:

      Given: $\config{S}{\NEW} \parstepsto
      \config{\extS{S}{l}{\bot}{\frozenfalse}}{l}$.

      To show: $\config{\lubstore{S}{S''}}{\NEW} \parstepsto
      \config{\lubstore{(\extS{S}{l}{\bot}{\frozenfalse})}{S''}}{l}$.

      By {\sc E-New}, we have that $\config{\lubstore{S}{S''}}{\NEW}
      \parstepsto
      \config{\extS{(\lubstore{S}{S''})}{l'}{\bot}{\frozenfalse}}{l'}$,
      where $l' \notin \dom{\lubstore{S}{S''}}$.

      By assumption, $S''$ is non-conflicting with $\config{S}{\NEW}
      \parstepsto \config{\extS{S}{l}{\bot}{\frozenfalse}}{l}$.
 
      Therefore $l \notin \dom{S''}$.

      From the side condition of {\sc E-New}, $l \notin \dom{S}$.

      Therefore $l \notin \dom{\lubstore{S}{S''}}$.

      Therefore, in
      $\config{\extS{(\lubstore{S}{S''})}{l'}{\bot}{\frozenfalse}}{l'}$,
      we can $\alpha$-rename $l'$ to $l$, resulting in
      $\config{\extS{(\lubstore{S}{S''})}{l}{\bot}{\frozenfalse}}{l}$.

      Therefore $\config{\lubstore{S}{S''}}{\NEW} \parstepsto
      \config{\extS{(\lubstore{S}{S''})}{l}{\bot}{\frozenfalse}}{l}$.

      Note that:
      \begin{align*}
        \extS{(\lubstore{S}{S''})}{l}{\bot}{\frozenfalse} &=
        \lubstore{\extS{S}{l}{\bot}{\frozenfalse}}{\extS{S''}{l}{\bot}{\frozenfalse}} \\
        &= \lubstore{\lubstore{S}{\store{\storebinding{l}{\bot}{\frozenfalse}}}}{\lubstore{S''}{\store{\storebinding{l}{\bot}{\frozenfalse}}}} \\
        &= \lubstore{\lubstore{S}{\store{\storebinding{l}{\bot}{\frozenfalse}}}}{S''} \\
        &= \lubstore{\extS{S}{l}{\bot}{\frozenfalse}}{S''}.
      \end{align*}
      Therefore $\config{\lubstore{S}{S''}}{\NEW} \parstepsto
      \config{\lubstore{\extS{S}{l}{\bot}{\frozenfalse}}{S''}}{l}$, as we were
      required to show.

    \item Case {\sc E-Put}:

      Given: $\config{S}{\putiexp{l}} \parstepsto
      \config{\extSRaw{S}{l}{u_{p_i}(p_1)}}{\unit}$.

      To show: $\config{\lubstore{S}{S''}}{\putiexp{l}{d_2}}
      \parstepsto
      \config{\lubstore{\extSRaw{S}{l}{u_{p_i}(p_1)}}{S''}}{\unit}$.

      We will first show that

      $\config{\lubstore{S}{S''}}{\putiexp{l}{d_2}} \parstepsto
      \config{\extSRaw{(\lubstore{S}{S''})}{l}{u_{p_i}(p_1)}}{\unit}$

      and then show why this is sufficient.

      We proceed by cases on $l$:

      \begin{itemize}
        \item $l \notin \dom{S''}$:

          By assumption, $\lubstore{\extSRaw{S}{l}{u_{p_i}(p_1)}}{S''}
          \neq \topS$.

          By Lemma~\ref{lem:monotonicity},
          $\leqstore{S}{\extSRaw{S}{l}{u_{p_i}(p_1)}}$.

          Hence $\lubstore{S}{S''} \neq \topS$.

          Therefore, by Definition~\ref{def:lubstore},
          $(\lubstore{S}{S''})(l) = S(l)$.

          From the premises of {\sc E-Put}, $S(l) = p_1$.

          Hence $(\lubstore{S}{S''})(l) = p_1$.

          From the premises of {\sc E-Put}, $u_{p_i}(p_1) \neq \topp$.

          Therefore, by {\sc E-Put}, we have:
          $\config{\lubstore{S}{S''}}{\putiexp{l}} \parstepsto
          \config{\extSRaw{(\lubstore{S}{S''})}{l}{u_{p_i}(p_1)}}{\unit}$.

        \item $l \in \dom{S''}$:

          By assumption, $\lubstore{\extSRaw{S}{l}{u_{p_i}(p_1)}}{S''} \neq
          \topS$.

          By Lemma~\ref{lem:monotonicity},
          $\leqstore{S}{\extSRaw{S}{l}{u_{p_i}(p_1)}}$.

          Hence $\lubstore{S}{S''} \neq \topS$.

          Therefore $(\lubstore{S}{S''})(l) = \lubp{S(l)}{S''(l)}$.

          From the premises of {\sc E-Put}, $S(l) = p_1$.
          
          Hence $(\lubstore{S}{S''})(l) = p'_1$, where $p_1 \leqp
          p'_1$.

          \TODO{From here forward, this subcase still needs to be
            fixed.}

          By assumption, $\lubstore{\extSRaw{S}{l}{p_2}}{S''} \neq
          \topS$.

          Therefore, by Definition~\ref{def:lubstore},
          $\lubp{p_2}{S''(l)} \neq \topp$.

          Note that:
          \begin{align*}
            \topp &\neq \lubp{p_2}{S''(l)} \\
            &= \lubp{\lubp{p_1}{\state{d_2}{\frozenfalse}}}{S''(l)} \\
            &= \lubp{\lubp{S(l)}{\state{d_2}{\frozenfalse}}}{S''(l)} \\
            &= \lubp{\lubp{S(l)}{S''(l)}}{\state{d_2}{\frozenfalse}} \\
            &= \lubp{(\lubstore{S}{S''})(l)}{\state{d_2}{\frozenfalse}} \\
            &= \lubp{p'_1}{\state{d_2}{\frozenfalse}} \\
            &= p'_2. \\
          \end{align*}
          Hence $p'_2 \neq \topp$.

          Hence $(\lubstore{S}{S''})(l) = p'_1$ and $p'_2 =
          \lubp{p'_1}{\state{d_2}{\frozenfalse}}$ and $p'_2 \neq
          \topp$.

          Therefore, by {\sc E-Put} we have:
          $\config{\lubstore{S}{S''}}{\putiexp{l}{d_2}} \parstepsto
          \config{\extSRaw{(\lubstore{S}{S''})}{l}{p'_2}}{\unit}$.

          \lk{If we really wanted to be pedantic here, we'd actually
            prove that the stores are equal.  I'm assuming that if I
            can show that $\extSRaw{(\lubstore{S}{S''})}{l}{p'_2}$ and
            $\extSRaw{(\lubstore{S}{S''})}{l}{p_2}$ bind $l$ to the
            same value, then it will be obvious that they're equal.}

          Note that:
          \begin{align*}
            (\extSRaw{(\lubstore{S}{S''})}{l}{p'_2})(l) &= \lubp{(\lubstore{S}{S''})(l)}{(\store{\storebindingRaw{l}{p'_2}})(l)} \\
            &= \lubp{p'_1}{p'_2} \\
            &= \lubp{p'_1}{\lubp{p'_1}{\state{d_2}{\frozenfalse}}} \\
            &= \lubp{p'_1}{\state{d_2}{\frozenfalse}}
          \end{align*}
          and
          \begin{align*}
            (\extSRaw{(\lubstore{S}{S''})}{l}{p_2})(l) &= \lubp{(\lubstore{S}{S''})(l)}{(\store{\storebindingRaw{l}{p_2}})(l)} \\
            &= \lubp{p'_1}{p_2} \\
            &= \lubp{p'_1}{\lubp{p_1}{\state{d_2}{\frozenfalse}}} \\
            &= \lubp{p'_1}{\state{d_2}{\frozenfalse}} & \textrm{(since $p_1 \leqp p'_1$).}
          \end{align*}
          Therefore $\extSRaw{(\lubstore{S}{S''})}{l}{p'_2} =
          \extSRaw{(\lubstore{S}{S''})}{l}{p_2}$.

          Therefore, $\config{\lubstore{S}{S''}}{\putiexp{l}{d_2}}
          \parstepsto
          \config{\extSRaw{(\lubstore{S}{S''})}{l}{p_2}}{\unit}$.
      \end{itemize}

      Note that:
      \begin{align*}
        \extSRaw{(\lubstore{S}{S''})}{l}{p_2} &= \lubstore{\extSRaw{S}{l}{p_2}}{\extSRaw{S''}{l}{p_2}} \\
        &= \lubstore{\lubstore{S}{\store{\storebindingRaw{l}{p_2}}}}{\lubstore{S''}{\store{\storebindingRaw{l}{p_2}}}} \\
        &= \lubstore{\lubstore{S}{\store{\storebindingRaw{l}{p_2}}}}{S''} \\
        &= \lubstore{\extSRaw{S}{l}{p_2}}{S''}.
      \end{align*}
      Therefore $\config{\lubstore{S}{S''}}{\putiexp{l}{d_2}}
      \parstepsto \config{\lubstore{\extSRaw{S}{l}{p_2}}{S''}}{\unit}$,
      as we were required to show.

    \item Case {\sc E-Get}:

      Given: $\config{S}{\getexp{l}{P}} \parstepsto \config{S}{p_2}$.

      To show: $\config{\lubstore{S}{S''}}{\getexp{l}{P}} \parstepsto
      \config{\lubstore{S}{S''}}{p_2}$.

      From the premises of {\sc E-Get}, $S(l) = p_1$ and $\incomp{P}$
      and $p_2 \in P$ and $p_2 \leqp p_1$.

      By assumption, $\lubstore{S}{S''} \neq \topS$.

      Hence $(\lubstore{S}{S''}) = p'_1$, where $p_1 \leqp p'_1$.

      By the transitivity of $\leqp$, $p_2 \leqp p'_1$.

      Hence, $S(l) = p'_1$ and $\incomp{P}$ and $p_2 \in P$ and $p_2
      \leqp p'_1$.

      Therefore, by {\sc E-Get},

      $\config{\lubstore{S}{S''}}{\getexp{l}{P}} \parstepsto
      \config{\lubstore{S}{S''}}{p_2}$,

      as we were required to show.

    \item Case {\sc E-Freeze-Init}:

      Given: $\config{S}{\freezeafter{l}{Q}{\lam{x}{e}}} \parstepsto
      \config{S}{\freezeafterfull{l}{Q}{\lam{x}{e}}{\setof{}}{\setof{}}}$.

      To show:
      $\config{\lubstore{S}{S''}}{\freezeafter{l}{Q}{\lam{x}{e}}}
      \parstepsto
      \config{\lubstore{S}{S''}}{\freezeafterfull{l}{Q}{\lam{x}{e}}{\setof{}}{\setof{}}}$.

      Immediate by {\sc E-Freeze-Init}.

    \item Case {\sc E-Spawn-Handler}:

      Given:

      $\config{S}{\freezeafterfull{l}{Q}{\lam{x}{e_0}}{\setof{e,
            \dots}}{H}} \parstepsto
      \config{S}{\freezeafterfull{l}{Q}{\lam{x}{e_0}}{\setof{\subst{e_0}{x}{d_2},
            e, \dots}} {\{d_2\}\cup H}}$.

      To show:

      $\config{\lubstore{S}{S''}}{\freezeafterfull{l}{Q}{\lam{x}{e_0}}{\setof{e,
            \dots}}{H}} \parstepsto
      \config{\lubstore{S}{S''}}{\freezeafterfull{l}{Q}{\lam{x}{e_0}}{\setof{\subst{e_0}{x}{d_2},
            e, \dots}} {\{d_2\}\cup H}}$.

      From the premises of {\sc E-Spawn-Handler}, $S(l) =
      \state{d_1}{\status_1}$ and $d_2 \userleq d_1$ and $d_2 \notin
      H$ and $d_2 \in Q$.

      By assumption, $\lubstore{S}{S''} \neq \topS$.

      Hence $(\lubstore{S}{S''})(l) = \state{d'_1}{\status'_1}$ where
      $\state{d_1}{\status_1} \leqp \state{d'_1}{\status'_1}$.

      By Definition~\ref{def:lattice-with-status-bits}, $d_1 \userleq
      d'_1$.

      By the transitivity of $\userleq$, $d_2 \userleq d'_1$.

      Hence $(\lubstore{S}{S''})(l) =
      \state{d'_1}{\status'_1}$ and $d_2 \userleq d'_1$ and $d_2 \notin
      H$ and $d_2 \in Q$.

      Therefore, by {\sc E-Spawn-Handler},

      $\config{\lubstore{S}{S''}}{\freezeafterfull{l}{Q}{\lam{x}{e_0}}{\setof{e,
            \dots}}{H}} \parstepsto
      \config{\lubstore{S}{S''}}{\freezeafterfull{l}{Q}{\lam{x}{e_0}}{\setof{\subst{e_0}{x}{d_2},
            e, \dots}} {\{d_2\}\cup H}}$,

      as we were required to show.

    \item Case {\sc E-Freeze-Final}:

      \lk{This case wouldn't work but for the $\lubstore{S'}{S''}
        \statuseq S$ requirement, which makes it a no-op freeze.}

      Given:
      $\config{S}{\freezeafterfull{l}{Q}{\lam{x}{e_0}}{\setof{v,
            \dots}}{H}} \parstepsto
      \config{\extS{S}{l}{d_1}{\frozentrue}}{d_1}$.

      To show:
      $\config{\lubstore{S}{S''}}{\freezeafterfull{l}{Q}{\lam{x}{e_0}}{\setof{v,
            \dots}}{H}} \parstepsto
      \config{\lubstore{\extS{S}{l}{d_1}{\frozentrue}}{S''}}{d_1}$.

      We will first show that

      $\config{\lubstore{S}{S''}}{\freezeafterfull{l}{Q}{\lam{x}{e_0}}{\setof{v,
            \dots}}{H}} \parstepsto
      \config{\extS{(\lubstore{S}{S''})}{l}{d_1}{\frozentrue}}{d_1}$

      and then show why this is sufficient.

      We proceed by cases on $l$:
      \begin{itemize}
      \item $l \notin \dom{S''}$:

        By assumption, $\lubstore{\extS{S}{l}{d_1}{\frozentrue}}{S''}
        \neq \topS$.

        By Lemma~\ref{lem:monotonicity},
        $\leqstore{S}{\extS{S}{l}{d_1}{\frozentrue}}$.

        Therefore $\lubstore{S}{S''} \neq \topS$.

        Hence, by Definition~\ref{def:lubstore},
        $(\lubstore{S}{S''})(l) = S(l)$.

        From the premises of {\sc E-Freeze-Final}, we have that $S(l)
        = \state{d_1}{\status_1}$.

        Hence $(\lubstore{S}{S''})(l) = \state{d_1}{\status_1}$.

        From the premises of {\sc E-Freeze-Final}, we have that
        $\forall{d_2} ~.~ ( {d_2 \userleq d_1 \land d_2 \in Q} \Rightarrow d_2 \in
        H)$.

        Therefore, by {\sc E-Freeze-Final}, we have that

        $\config{\lubstore{S}{S''}}{\freezeafterfull{l}{Q}{\lam{x}{e_0}}{\setof{v,
              \dots}}{H}} \parstepsto
        \config{\extS{(\lubstore{S}{S''})}{l}{d_1}{\frozentrue}}{d_1}$.


      \item $l \in \dom{S''}$:

        By assumption, $\lubstore{\extS{S}{l}{d_1}{\frozentrue}}{S''}
        \neq \topS$.

        By Lemma~\ref{lem:monotonicity},
        $\leqstore{S}{\extS{S}{l}{d_1}{\frozentrue}}$.

        Therefore $\lubstore{S}{S''} \neq \topS$.

        Hence, by Definition~\ref{def:lubstore},
        $(\lubstore{S}{S''})(l) = \lubp{S(l)}{S''(l)}$.

        From the premises of {\sc E-Freeze-Final}, we have that
        $S(l) = \state{d_1}{\status_1}$.

        By assumption, $\lubstore{\extS{S}{l}{d_1}{\frozentrue}}{S''}
        \statuseq S$.

        Therefore $\status_1 = \frozentrue$.

        Therefore $S(l) = \state{d_1}{\frozentrue}$.

        Therefore $(\lubstore{S}{S''})(l) =
        \lubp{\state{d_1}{\frozentrue}}{S''(l)}$.

        We proceed by cases on $S''(l)$:
        \begin{itemize}
        \item $S''(l) = \state{d_3}{\frozenfalse}$, where $d_3 \userleq d_1$:

          By Definition~\ref{def:lubp},
          $\lubp{\state{d_1}{\frozentrue}}{\state{d_3}{\frozenfalse}}
          = \state{d_1}{\frozentrue}$.

          Therefore $(\lubstore{S}{S''})(l) =
          \state{d_1}{\frozentrue}$.

          From the premises of {\sc E-Freeze-Final}, we have that
          $\forall{d_2} ~.~ ( {d_2 \userleq d_1 \land d_2 \in Q} \Rightarrow d_2 \in
          H)$.

          Therefore, by {\sc E-Freeze-Final}, we have that

          $\config{\lubstore{S}{S''}}{\freezeafterfull{l}{Q}{\lam{x}{e_0}}{\setof{v,
                \dots}}{H}} \parstepsto
          \config{\extS{(\lubstore{S}{S''})}{l}{d_1}{\frozentrue}}{d_1}$.

        \item $S''(l) = \state{d_3}{\frozenfalse}$, where $d_3 \nuserleq d_1$:

          By Definition~\ref{def:lubp},
          $\lubp{\state{d_1}{\frozentrue}}{\state{d_3}{\frozenfalse}}
          = \state{\top}{\frozenfalse}$.

          Therefore $\lubp{S(l)}{S''(l)} =
          \state{\top}{\frozenfalse}$.

          By Definition~\ref{def:lattice-with-status-bits},
          $\state{\top}{\frozenfalse} = \topp$.

          Therefore $\lubp{S(l)}{S''(l)} = \topp$.

          Therefore, by Definition~\ref{def:lubstore},
          $\lubstore{S}{S''} = \topS$.

          This is a contradiction.

          Therefore,

          $\config{\lubstore{S}{S''}}{\freezeafterfull{l}{Q}{\lam{x}{e_0}}{\setof{v,
                \dots}}{H}} \parstepsto
          \config{\extS{(\lubstore{S}{S''})}{l}{d_1}{\frozentrue}}{d_1}$.

        \item $S''(l) = \state{d_3}{\frozentrue}$, where $d_3 = d_1$:

          By Definition~\ref{def:lubp},
          $\lubp{\state{d_1}{\frozentrue}}{\state{d_3}{\frozentrue}} =
          \state{d_1}{\frozentrue}$.

          Therefore $(\lubstore{S}{S''})(l) = \state{d_1}{\frozentrue}$.

          From the premises of {\sc E-Freeze-Final}, we have that
          $\forall{d_2} ~.~ ( {d_2 \userleq d_1 \land d_2 \in Q} \Rightarrow d_2 \in
          H)$.

          Therefore, by {\sc E-Freeze-Final}, we have that

          $\config{\lubstore{S}{S''}}{\freezeafterfull{l}{Q}{\lam{x}{e_0}}{\setof{v,
                \dots}}{H}} \parstepsto
          \config{\extS{(\lubstore{S}{S''})}{l}{d_1}{\frozentrue}}{d_1}$.

        \item $S''(l) = \state{d_3}{\frozentrue}$, where $d_3 \neq d_1$:

          By Definition~\ref{def:lubp},
          $\lubp{\state{d_1}{\frozentrue}}{\state{d_3}{\frozentrue}}
          = \state{\top}{\frozenfalse}$.

          Therefore $\lubp{S(l)}{S''(l)} = \state{\top}{\frozenfalse}$.

          By Definition~\ref{def:lattice-with-status-bits},
          $\state{\top}{\frozenfalse} = \topp$.

          Therefore $\lubp{S(l)}{S''(l)} = \topp$.

          Therefore, by Definition~\ref{def:lubstore},
          $\lubstore{S}{S''} = \topS$.

          This is a contradiction.

          Therefore,

          $\config{\lubstore{S}{S''}}{\freezeafterfull{l}{Q}{\lam{x}{e_0}}{\setof{v,
                \dots}}{H}} \parstepsto
          \config{\extS{(\lubstore{S}{S''})}{l}{d_1}{\frozentrue}}{d_1}$.
        \end{itemize}
      \end{itemize}

      In each case we have shown that

      $\config{\lubstore{S}{S''}}{\freezeafterfull{l}{Q}{\lam{x}{e_0}}{\setof{v,
            \dots}}{H}} \parstepsto
      \config{\extS{(\lubstore{S}{S''})}{l}{d_1}{\frozentrue}}{d_1}$.

      Note that:
      \begin{align*}
        \extS{(\lubstore{S}{S''})}{l}{d_1}{\frozentrue} &=
        \lubstore{\extS{S}{l}{d_1}{\frozentrue}}{\extS{S''}{l}{d_1}{\frozentrue}} \\
        &= \lubstore{\lubstore{S}{\store{\storebinding{l}{d_1}{\frozentrue}}}}{\lubstore{S''}{\store{\storebinding{l}{d_1}{\frozentrue}}}} \\
        &= \lubstore{\lubstore{S}{\store{\storebinding{l}{d_1}{\frozentrue}}}}{S''} \\
        &= \lubstore{\extS{S}{l}{d_1}{\frozentrue}}{S''}.
      \end{align*}
      Therefore

      $\config{\lubstore{S}{S''}}{\freezeafterfull{l}{Q}{\lam{x}{e_0}}{\setof{v,
            \dots}}{H}} \parstepsto
      \config{\lubstore{\extS{S}{l}{d_1}{\frozentrue}}{S''}}{d_1}$,

      as we were required to show.

    \item Case {\sc E-Freeze-Simple}:

      Given: $\config{S}{\freeze{l}} \parstepsto
      \config{\extS{S}{l}{d_1}{\frozentrue}}{d_1}$.

      To show: $\config{\lubstore{S}{S''}}{\freeze{l}}
      \parstepsto
      \config{\lubstore{\extS{S}{l}{d_1}{\frozentrue}}{S''}}{d_1}$.

      We will first show that

      $\config{\lubstore{S}{S''}}{\freeze{l}} \parstepsto
      \config{\extS{(\lubstore{S}{S''})}{l}{d_1}{\frozentrue}}{d_1}$

      and then show why this is sufficient.

      We proceed by cases on $l$:
      \begin{itemize}
      \item $l \notin \dom{S''}$:

        By assumption, $\lubstore{\extS{S}{l}{d_1}{\frozentrue}}{S''}
        \neq \topS$.

        By Lemma~\ref{lem:monotonicity},
        $\leqstore{S}{\extS{S}{l}{d_1}{\frozentrue}}$.

        Therefore $\lubstore{S}{S''} \neq \topS$.

        Hence, by Definition~\ref{def:lubstore},
        $(\lubstore{S}{S''})(l) = S(l)$.

        From the premise of {\sc E-Freeze-Simple}, we have that
        $S(l) = \state{d_1}{\status_1}$.

        Therefore, by {\sc E-Freeze-Simple}, we have that

        $\config{\lubstore{S}{S''}}{\freeze{l}}
        \parstepsto
        \config{\extS{(\lubstore{S}{S''})}{l}{d_1}{\frozentrue}}{d_1}$.

      \item $l \in \dom{S''}$:

        By assumption, $\lubstore{\extS{S}{l}{d_1}{\frozentrue}}{S''}
        \neq \topS$.

        By Lemma~\ref{lem:monotonicity},
        $\leqstore{S}{\extS{S}{l}{d_1}{\frozentrue}}$.

        Therefore $\lubstore{S}{S''} \neq \topS$.

        Hence, by Definition~\ref{def:lubstore},
        $(\lubstore{S}{S''})(l) = \lubp{S(l)}{S''(l)}$.

        From the premise of {\sc E-Freeze-Simple}, we have that
        $S(l) = \state{d_1}{\status_1}$.

        By assumption, $\lubstore{\extS{S}{l}{d_1}{\frozentrue}}{S''}
        \statuseq S$.

        Therefore $\status_1 = \frozentrue$.

        Therefore $(\lubstore{S}{S''})(l) =
        \lubp{\state{d_1}{\frozentrue}}{S''(l)}$.
        
        We proceed by cases on $S''(l)$:
        \begin{itemize}
        \item $S''(l) = \state{d_2}{\frozenfalse}$, where $d_2 \userleq d_1$:

          By Definition~\ref{def:lubp},
          $\lubp{\state{d_1}{\frozentrue}}{\state{d_2}{\frozenfalse}} =
          \state{d_1}{\frozentrue}$.

          Therefore $(\lubstore{S}{S''})(l) =
          \state{d_1}{\frozentrue}$.

          Therefore, by {\sc E-Freeze-Simple}, we have that

          $\config{\lubstore{S}{S''}}{\freeze{l}}
          \parstepsto
          \config{\extS{(\lubstore{S}{S''})}{l}{d_1}{\frozentrue}}{d_1}$.

        \item $S''(l) = \state{d_2}{\frozenfalse}$, where $d_2 \nuserleq d_1$:

          By Definition~\ref{def:lubp},
          $\lubp{\state{d_1}{\frozentrue}}{\state{d_2}{\frozenfalse}}
          = \state{\top}{\frozenfalse}$.

          By Definition~\ref{def:lattice-with-status-bits},
          $\state{\top}{\frozenfalse} = \topp$.

          Therefore $\lubp{S(l)}{S''(l)} = \topp$.

          Therefore, by Definition~\ref{def:lubstore},
          $\lubstore{S}{S''} = \topS$.

          This is a contradiction.

          Therefore,

          $\config{\lubstore{S}{S''}}{\freeze{l}}
          \parstepsto
          \config{\extS{(\lubstore{S}{S''})}{l}{d_1}{\frozentrue}}{d_1}$.

        \item $S''(l) = \state{d_2}{\frozentrue}$, where $d_2 = d_1$:

          Therefore $(\lubstore{S}{S''})(l) =
          \lubp{\state{d_1}{\frozentrue}}{\state{d_2}{\frozentrue}}$.

          By Definition~\ref{def:lubp},
          $\lubp{\state{d_1}{\frozentrue}}{\state{d_2}{\frozentrue}} =
          \state{d_1}{\frozentrue}$.

          Therefore $(\lubstore{S}{S''})(l) =
          \state{d_1}{\frozentrue}$.

          Therefore, by {\sc E-Freeze-Simple}, we have that

          $\config{\lubstore{S}{S''}}{\freeze{l}}
          \parstepsto
          \config{\extS{(\lubstore{S}{S''})}{l}{d_1}{\frozentrue}}{d_1}$.

        \item $S''(l) = \state{d_2}{\frozentrue}$, where $d_2 \neq d_1$:

          By Definition~\ref{def:lubp},
          $\lubp{\state{d_1}{\frozentrue}}{\state{d_2}{\frozentrue}}
          = \state{\top}{\frozenfalse}$.

          By Definition~\ref{def:lattice-with-status-bits},
          $\state{\top}{\frozenfalse} = \topp$.

          Therefore $\lubp{S(l)}{S''(l)} = \topp$.

          Therefore, by Definition~\ref{def:lubstore},
          $\lubstore{S}{S''} = \topS$.

          This is a contradiction.

          Therefore,

          $\config{\lubstore{S}{S''}}{\freeze{l}}
          \parstepsto
          \config{\extS{(\lubstore{S}{S''})}{l}{d_1}{\frozentrue}}{d_1}$.
        \end{itemize}
      \end{itemize}

      In each case we have shown that

      $\config{\lubstore{S}{S''}}{\freeze{l}} \parstepsto
      \config{\extS{(\lubstore{S}{S''})}{l}{d_1}{\frozentrue}}{d_1}$.

      Note that:
      \begin{align*}
        \extS{(\lubstore{S}{S''})}{l}{d_1}{\frozentrue} &=
        \lubstore{\extS{S}{l}{d_1}{\frozentrue}}{\extS{S''}{l}{d_1}{\frozentrue}} \\
        &= \lubstore{\lubstore{S}{\store{\storebinding{l}{d_1}{\frozentrue}}}}{\lubstore{S''}{\store{\storebinding{l}{d_1}{\frozentrue}}}} \\
        &= \lubstore{\lubstore{S}{\store{\storebinding{l}{d_1}{\frozentrue}}}}{S''} \\
        &= \lubstore{\extS{S}{l}{d_1}{\frozentrue}}{S''}.
      \end{align*}
      Therefore
      $\config{\lubstore{S}{S''}}{\freeze{l}}
      \parstepsto
      \config{\lubstore{\extS{S}{l}{d_1}{\frozentrue}}{S''}}{d_1}$,
      as we were required to show.
  \end{itemize}
\end{proof}


\section{Proof of Lemma~\ref{lem:strong-local-quasi-confluence}}\label{section:strong-local-quasi-confluence-proof}
\begin{proof}
  Suppose $\conf \ctxstepsto \conf_a$ and $\conf \ctxstepsto \conf_b$.

  We have to show that either there exist $\conf_c, i, j, \pi$ such
  that $\conf_a \ctxstepsto^i \conf_c$ and $\pi(\conf_b) \ctxstepsto^j
  \conf_c$ and $i \leq 1$ and $j \leq 1$, or that $\conf_a \ctxstepsto
  \error$ or $\conf_b \ctxstepsto \error$.

  By inspection of the operational semantics, it must be the case that
  $\conf$ steps to $\conf_a$ by the {\sc E-Eval-Ctxt} rule.

  Let $\conf = \config{S}{\evalctxt{E_a}{e_{a_1}}}$ and let $\conf_a =
  \config{S_a}{\evalctxt{E_a}{e_{a_2}}}$.

  Likewise, it must be the case that $\conf$ steps to $\conf_b$ by the
  {\sc E-Eval-Ctxt} rule.

  Let $\conf = \config{S}{\evalctxt{E_b}{e_{b_1}}}$ and let $\conf_b =
  \config{S_b}{\evalctxt{E_b}{e_{b_2}}}$.

  Note that $\conf = \config{S}{\evalctxt{E_a}{e_{a_1}}} =
  \config{S}{\evalctxt{E_b}{e_{b_1}}}$, and so
  $\evalctxt{E_a}{e_{a_1}} = \evalctxt{E_b}{e_{b_1}}$, but $E_a$ and
  $E_b$ may differ and $e_{a_1}$ and $e_{b_1}$ may differ.

  First, consider the possibility that $E_a = E_b$ (and $e_{a_1} =
  e_{b_1}$).

  Since $\config{S}{\evalctxt{E_a}{e_{a_1}}} \ctxstepsto
  \config{S_a}{\evalctxt{E_a}{e_{a_2}}}$ by {\sc E-Eval-Ctxt} and
  $\config{S}{\evalctxt{E_b}{e_{b_1}}} \ctxstepsto
  \config{S_b}{\evalctxt{E_b}{e_{b_2}}}$ by {\sc E-Eval-Ctxt}, we have
  from the premise of {\sc E-Eval-Ctxt} that $\config{S}{e_{a_1}}
  \parstepsto \config{S_a}{e_{a_2}}$ and $\config{S}{e_{b_1}}
  \parstepsto \config{S_b}{e_{b_2}}$.

  But then, since $e_{a_1} = e_{b_1}$, by Internal Determinism
  (Lemma~\ref{lem:internal-determinism}) there is a permutation $\pi'$
  such that $\config{S_a}{e_{a_2}} = \pi'(\config{S_b}{e_{b_2}})$,
  modulo choice of events.

  We have two cases:

  \begin{itemize}
  \item In the case where the steps $\conf \ctxstepsto \conf_a$ and
    $\conf \ctxstepsto \conf_b$ are both by {\sc E-Spawn-Handler} and
    they handle different events $d_2$ and $d'_2$, then we can satisfy
    the proof by choosing the final configuration $\conf_c$ as the
    configuration where both $d_2$ and $d'_2$ have been handled.

    Both $\conf_a$ and $\conf_b$ can step to this configuration by
    {\sc E-Spawn-Handler}: if the step from $\conf$ to $\conf_a$
    handles $d_2$ then the step from $\conf_a$ to $\conf_c$ handles
    $d'_2$, while if the step from $\conf$ to $\conf_b$ handles $d'_2$
    then the step from $\conf_b$ to $\conf_c$ handles $d_2$.

    The store in the final configuration is $S_a$ or $S_b$, which are
    equal because {\sc E-Spawn-Handler} does not affect the store, and
    we can satisfy the proof by choosing $i = 1$ and $j = 0$ and $\pi
    = \id$.

  \item Otherwise, we can satisfy the proof by choosing $\conf_c =
    \config{S_a}{e_{a_2}}$ and $i = 0$ and $j = 0$ and $\pi = \id$.
  \end{itemize}

  The rest of this proof deals with the more interesting case in which
  $E_a \neq E_b$ (and $e_{a_1} \neq e_{b_1}$).

  Since $\config{S}{\evalctxt{E_a}{e_{a_1}}} \ctxstepsto
  \config{S_a}{\evalctxt{E_a}{e_{a_2}}}$ and
  $\config{S}{\evalctxt{E_b}{e_{b_1}}} \ctxstepsto
  \config{S_b}{\evalctxt{E_b}{e_{b_2}}}$ and $\evalctxt{E_a}{e_{a_1}}
  = \evalctxt{E_b}{e_{b_1}}$, and since $E_a \neq E_b$, we have from
  Lemma~\ref{lem:locality} (Locality) that there exist evaluation
  contexts $E'_a$ and $E'_b$ such that:

  \begin{itemize}
  \item $\evalctxt{E'_a}{e_{a_1}} = \evalctxt{E_b}{e_{b_2}}$, and
  \item $\evalctxt{E'_b}{e_{b_1}} = \evalctxt{E_a}{e_{a_2}}$, and
  \item $\evalctxt{E'_a}{e_{a_2}} =
    \evalctxt{E'_b}{e_{b_2}}$.
  \end{itemize}

  In some of the cases that follow, we will choose $\conf_c = \error$,
  and in some we will prove that one of $\conf_a$ or $\conf_b$ steps
  to $\error$.

  In most cases, however, our approach will be to show that there
  exist $S', i, j, \pi$ such that:
  \begin{itemize}
  \item $\config{S_a}{\evalctxt{E_a}{e_{a_2}}} \ctxstepsto^i
    \config{S'}{\evalctxt{E'_a}{e_{a_2}}}$, and
  \item $\pi(\config{S_b}{\evalctxt{E_b}{e_{b_2}}}) \ctxstepsto^j
    \config{S'}{\evalctxt{E'_a}{e_{a_2}}}$.
  \end{itemize}
  Since $\evalctxt{E'_a}{e_{a_1}} = \evalctxt{E_b}{e_{b_2}}$,
  $\evalctxt{E'_b}{e_{b_1}} = \evalctxt{E_a}{e_{a_2}}$, and
  $\evalctxt{E'_a}{e_{a_2}} = \evalctxt{E'_b}{e_{b_2}}$, it suffices
  to show that:
  \begin{itemize}
  \item $\config{S_a}{\evalctxt{E'_b}{e_{b_1}}} \ctxstepsto^i
    \config{S'}{\evalctxt{E'_b}{e_{b_2}}}$, and
  \item $\pi(\config{S_b}{\evalctxt{E'_a}{e_{a_1}}}) \ctxstepsto^j
    \config{S'}{\evalctxt{E'_a}{e_{a_2}}}$.
  \end{itemize}
  From the premise of {\sc E-Eval-Ctxt}, we have that
  $\config{S}{e_{a_1}} \parstepsto \config{S_a}{e_{a_2}}$ and
  $\config{S}{e_{b_1}} \parstepsto \config{S_b}{e_{b_2}}$.

  We proceed by case analysis on the rule by which
  $\config{S}{e_{a_1}}$ steps to $\config{S_a}{e_{a_2}}$.

  Since the only way an $\error$ configuration can arise is by the
  {\sc E-Put-Err} rule, we can assume in all other cases that $\conf_a
  \neq \error$.

  \begin{enumerate}
  \item Case {\sc E-Beta}: We have $S_a = S$.

    We proceed by case analysis on the rule by which
    $\config{S}{e_{b_1}}$ steps to $\config{S_b}{e_{b_2}}$.

    Since the only way an $\error$ configuration can arise is by the
    {\sc E-Put-Err} rule, we can assume in all other cases that
    $\conf_b \neq \error$.
    \begin{enumerate}
    \item \label{slqc-beta-beta}Case {\sc E-Beta}: We have $S_a = S$
      and $S_b = S$.

      Choose $S' = S = S_a = S_b$, $i = 1$, $j = 1$, and $\pi = \id$.

      We have to show that:
      \begin{itemize}
      \item $\config{S}{\evalctxt{E'_b}{e_{b_1}}} \ctxstepsto
        \config{S_a}{\evalctxt{E'_b}{e_{b_2}}}$, and
      \item $\config{S}{\evalctxt{E'_a}{e_{a_1}}} \ctxstepsto
        \config{S_b}{\evalctxt{E'_a}{e_{a_2}}}$, 
      \end{itemize}

      both of which follow immediately from $\config{S}{e_{a_1}}
      \parstepsto \config{S_a}{e_{a_2}}$ and $\config{S}{e_{b_1}}
      \parstepsto \config{S_b}{e_{b_2}}$ and {\sc E-Eval-Ctxt}.

    \item \label{slqc-beta-new}Case {\sc E-New}: We have $S_a = S$ and
      $S_b = \extS{S}{l}{\bot}{\frozenfalse}$.

      Choose $S' = S_b$, $i = 1$, $j = 1$, and $\pi = \id$.

      We have to show that:
      \begin{itemize}
      \item $\config{S}{\evalctxt{E'_b}{e_{b_1}}} \ctxstepsto
        \config{S_b}{\evalctxt{E'_b}{e_{b_2}}}$, and
      \item $\config{S_b}{\evalctxt{E'_a}{e_{a_1}}} \ctxstepsto
        \config{S_b}{\evalctxt{E'_a}{e_{a_2}}}$.
      \end{itemize}

      The first of these follows immediately from $\config{S}{e_{b_1}}
      \parstepsto \config{S_b}{e_{b_2}}$ and {\sc E-Eval-Ctxt}.

      For the second, consider that $S_b =
      \extS{S}{l}{\bot}{\frozenfalse} = U_S(S)$, where $U_S$ is the
      store update operation that acts as the identity on the contents
      of all existing locations, and adds the binding
      $\storebinding{l}{\bot}{\frozenfalse}$ if no binding for $l$
      exists.

      Note that:
      \begin{itemize}
      \item $U_S$ is non-conflicting with $\config{S}{e_{a_1}}
        \parstepsto \config{S_a}{e_{a_2}}$, since no locations are
        allocated in the transition;
      \item $U_S(S_a) \neq \topS$, since $U_S(S_a) = U_S(S) = S_b$
        and we know that $\conf_b \neq \error$; and
      \item $U_S$ is freeze-safe with $\config{S}{e_{a_1}}
        \parstepsto \config{S_a}{e_{a_2}}$, since $S_a = S$, so
        there are no locations whose contents differ in status
        between them.
      \end{itemize}

      Therefore, by Lemma~\ref{lem:generalized-independence}
      (Generalized Independence), we have that

      $\config{U_S(S)}{e_{a_1}} \parstepsto
      \config{U_S(S_a)}{e_{a_2}}$.

      Hence $\config{S_b}{e_{a_1}} \parstepsto \config{S_b}{e_{a_2}}$.

      By {\sc E-Eval-Ctxt}, it follows that
      $\config{S_b}{\evalctxt{E'_a}{e_{a_1}}} \ctxstepsto
      \config{S_b}{\evalctxt{E'_a}{e_{a_2}}}$,
      as we were required to show.

    \item \label{slqc-beta-put}Case {\sc E-Put}: We have $S_a = S$ and
      $S_b = \extSRaw{S}{l}{u_{p_i}(p_1)}$.

      Choose $S' = S_b$, $i = 1$, $j = 1$, and $\pi = \id$.

      We have to show that:
      \begin{itemize}
      \item $\config{S}{\evalctxt{E'_b}{e_{b_1}}} \ctxstepsto
        \config{S_b}{\evalctxt{E'_b}{e_{b_2}}}$, and
      \item $\config{S_b}{\evalctxt{E'_a}{e_{a_1}}} \ctxstepsto
        \config{S_b}{\evalctxt{E'_a}{e_{a_2}}}$.
      \end{itemize}

      The first of these follows immediately from $\config{S}{e_{b_1}}
      \parstepsto \config{S_b}{e_{b_2}}$ and {\sc E-Eval-Ctxt}.

      For the second, consider that $S_b = U_S(S)$, where $U_S$ is the
      store update operation that applies $u_{p_i}$ to the contents of
      $l$ and acts as the identity on all other locations.

      Note that:
      \begin{itemize}
      \item $U_S$ is non-conflicting with $\config{S}{e_{a_1}}
        \parstepsto \config{S_a}{e_{a_2}}$, since no locations are
        allocated in the transition;
      \item $U_S(S_a) \neq \topS$, since $U_S(S_a) = U_S(S) = S_b$
        and we know that $\conf_b \neq \error$; and
      \item $U_S$ is freeze-safe with $\config{S}{e_{a_1}}
        \parstepsto \config{S_a}{e_{a_2}}$, since $S_a = S$, so
        there are no locations whose contents differ in status
        between them.
      \end{itemize}

      Therefore, by Lemma~\ref{lem:generalized-independence}
      (Generalized Independence), we have that

      $\config{U_S(S)}{e_{a_1}} \parstepsto
      \config{U_S(S_a)}{e_{a_2}}$.

      Hence $\config{S_b}{e_{a_1}} \parstepsto \config{S_b}{e_{a_2}}$.

      By {\sc E-Eval-Ctxt}, it follows that
      $\config{S_b}{\evalctxt{E'_a}{e_{a_1}}} \ctxstepsto
      \config{S_b}{\evalctxt{E'_a}{e_{a_2}}}$, as we were required to
      show.

    \item \label{slqc-beta-put-err}Case {\sc E-Put-Err}: We have $S_a
      = S$ and $\config{S_b}{e_{b_2}} = \error$, and so we choose
      $\conf_c = \error$, $i = 1$, $j = 0$, and $\pi = \id$.

      We have to show that:
      \begin{itemize}
      \item $\config{S}{\evalctxt{E'_b}{e_{b_1}}} \ctxstepsto \error$,
        and
      \item $\config{S_b}{\evalctxt{E'_a}{e_{a_1}}} = \error$.
      \end{itemize}

      The second of these is immediately true because since
      $\config{S_b}{e_{b_2}} = \error$, $S_b = \topS$, and so
      $\config{S_b}{\evalctxt{E'_a}{e_{a_1}}}$ is equal to $\error$ as
      well.

      For the first, observe that $\config{S}{e_{b_1}} \parstepsto
      \config{S_b}{e_{b_2}}$, hence by {\sc E-Eval-Ctxt},
      $\config{S}{\evalctxt{E'_b}{e_{b_1}}} \ctxstepsto
      \config{S_b}{\evalctxt{E'_b}{e_{b_2}}}$.

      But $S_b = \topS$, so $\config{S_b}{\evalctxt{E'_b}{e_{b_2}}}$
      is equal to $\error$, and so
      $\config{S}{\evalctxt{E'_b}{e_{b_1}}} \ctxstepsto \error$, as
      required.

    \item \label{slqc-beta-get}Case {\sc E-Get}: Similar to
      case~\ref{slqc-beta-beta}, since $S_a = S$ and $S_b = S$.
    \item \label{slqc-beta-freeze-init}Case {\sc E-Freeze-Init}:
      Similar to case~\ref{slqc-beta-beta}, since $S_a = S$ and $S_b =
      S$.
    \item \label{slqc-beta-spawn-handler}Case {\sc E-Spawn-Handler}:
      Similar to case~\ref{slqc-beta-beta}, since $S_a = S$ and $S_b =
      S$.
    \item \label{slqc-beta-freeze-final}Case {\sc E-Freeze-Final}: We
      have $S_a = S$ and $S_b = \extS{S}{l}{d_1}{\frozentrue}$.

      Choose $S' = S_b$, $i = 1$, $j = 1$, and $\pi = \id$.

      We have to show that:
      \begin{itemize}
      \item $\config{S}{\evalctxt{E'_b}{e_{b_1}}} \ctxstepsto
        \config{S_b}{\evalctxt{E'_b}{e_{b_2}}}$, and
      \item $\config{S_b}{\evalctxt{E'_a}{e_{a_1}}} \ctxstepsto
        \config{S_b}{\evalctxt{E'_a}{e_{a_2}}}$.
      \end{itemize}

      The first of these follows immediately from $\config{S}{e_{b_1}}
      \parstepsto \config{S_b}{e_{b_2}}$ and {\sc E-Eval-Ctxt}.

      For the second, note that $S_b = U_S(S)$, where $U_S$ is the
      store update operation that freezes the contents of $l$ and acts
      as the identity on the contents of all other locations.

      Note that:
      \begin{itemize}
      \item $U_S$ is non-conflicting with $\config{S}{e_{a_1}}
        \parstepsto \config{S_a}{e_{a_2}}$, since no locations are
        allocated in the transition;
      \item $U_S(S_a) \neq \topS$, since $U_S(S_a) = U_S(S) = S_b$
        and we know that $\conf_b \neq \error$; and
      \item $U_S$ is freeze-safe with $\config{S}{e_{a_1}}
        \parstepsto \config{S_a}{e_{a_2}}$, since $S_a = S$, so
        there are no locations whose contents differ in status
        between them.
      \end{itemize}

      Therefore, by Lemma~\ref{lem:generalized-independence}
      (Generalized Independence), we have that

      $\config{U_S(S)}{e_{a_1}} \parstepsto
      \config{U_S(S_a)}{e_{a_2}}$.

      Hence $\config{S_b}{e_{a_1}} \parstepsto \config{S_b}{e_{a_2}}$.

      By {\sc E-Eval-Ctxt}, it follows that
      $\config{S_b}{\evalctxt{E'_a}{e_{a_1}}} \ctxstepsto
      \config{S_b}{\evalctxt{E'_a}{e_{a_2}}}$, as we were required to
      show.

    \item \label{slqc-beta-freeze-simple}Case {\sc E-Freeze-Simple}:
      Similar to case~\ref{slqc-beta-freeze-final}, since $S_b =
      \extS{S}{l}{d_1}{\frozentrue}$.

    \end{enumerate}
  \item Case {\sc E-New}: We have $S_a = \extS{S}{l}{\bot}{\frozenfalse}$.

    We proceed by case analysis on the rule by which
    $\config{S}{e_{b_1}}$ steps to $\config{S_b}{e_{b_2}}$.

    Since the only way an $\error$ configuration can arise is by the
    {\sc E-Put-Err} rule, we can assume in all other cases that
    $\conf_b \neq \error$.
    \begin{enumerate}
    \item \label{slqc-new-beta}Case {\sc E-Beta}: By symmetry with case~\ref{slqc-beta-new}.
    \item \label{slqc-new-new}Case {\sc E-New}: We have $S_a =
      \extS{S}{l}{\bot}{\frozenfalse}$ and $S_b =
      \extS{S}{l'}{\bot}{\frozenfalse}$.

      Now consider whether $l = l'$:
      \begin{itemize}
      \item If $l \neq l'$:

        Choose $S' =
        \extS{\extS{S}{l'}{\bot}{\frozenfalse}}{l}{\bot}{\frozenfalse}$,
        $i = 1$, $j = 1$, and $\pi = \id$.

        We have to show that:
        \begin{itemize}
        \item $\config{S_a}{\evalctxt{E'_b}{e_{b_1}}} \ctxstepsto
          \config{\extS{\extS{S}{l'}{\bot}{\frozenfalse}}{l}{\bot}{\frozenfalse}}{\evalctxt{E'_b}{e_{b_2}}}$,
          and
        \item $\config{S_b}{\evalctxt{E'_a}{e_{a_1}}} \ctxstepsto
          \config{\extS{\extS{S}{l'}{\bot}{\frozenfalse}}{l}{\bot}{\frozenfalse}}{\evalctxt{E'_a}{e_{a_2}}}$.
        \end{itemize}

        For the first of these, consider that $S_a =
        \extS{S}{l}{\bot}{\frozenfalse} = U_S(S)$, where $U_S$ is
        the store update operation that acts as the identity on the
        contents of all existing locations, and adds the binding
        $\storebinding{l}{\bot}{\frozenfalse}$ if no binding for $l$
        exists.

        Note that:
        \begin{itemize}
        \item $U_S$ is non-conflicting with $\config{S}{e_{b_1}}
          \parstepsto \config{S_b}{e_{b_2}}$, since the only
          location allocated in the transition is $l'$, and $l
          \neq l'$ in this case;
        \item $U_S(S_b) \neq \topS$, since $U_S(S_b) =
          \extS{\extS{S}{l'}{\bot}{\frozenfalse}}{l}{\bot}{\frozenfalse}$
          and we know $S \neq \topS$ and the addition of new
          bindings $\storebinding{l}{\bot}{\frozenfalse}$ and
          $\storebinding{l'}{\bot}{\frozenfalse}$ cannot cause it to
          become $\topS$; and
        \item $U_S$ is freeze-safe with $\config{S}{e_{b_1}}
          \parstepsto \config{S_b}{e_{b_2}}$, since $S_b =
          \extS{S}{l'}{\bot}{\frozenfalse}$ and $l' \notin \dom{S}$,
          so there are no locations whose contents differ in status
          between $S$ and $S_b$.
        \end{itemize}

        Therefore, by Lemma~\ref{lem:generalized-independence}
        (Generalized Independence), we have that

        $\config{U_S(S)}{e_{b_1}} \parstepsto
        \config{U_S(S_b)}{e_{b_2}}$.

        Hence $\config{\extS{S}{l}{\bot}{\frozenfalse}}{e_{b_1}}
        \parstepsto
        \config{\extS{S_b}{l}{\bot}{\frozenfalse}}{e_{b_2}}$.

        By {\sc E-Eval-Ctxt} it follows that

        $\config{\extS{S}{l}{\bot}{\frozenfalse}}{\evalctxt{E'_b}{e_{b_1}}}
        \parstepsto
        \config{\extS{S_b}{l}{\bot}{\frozenfalse}}{\evalctxt{E'_b}{e_{b_2}}}$,
        which, since $S_b = \extS{S}{l'}{\bot}{\frozenfalse}$, is what
        we were required to show.

        The argument for the second is symmetrical.

      \item If $l = l'$:

        In this case, observe that we do \emph{not} want the
        expression in the final configuration to be
        $\evalctxt{E'_a}{e_{a_2}}$ (nor its equivalent,
        $\evalctxt{E'_b}{e_{b_2}}$).

        The reason for this is that $\evalctxt{E'_a}{e_{a_2}}$
        contains both occurrences of $l$.

        Rather, we want both configurations to step to a configuration
        in which exactly one occurrence of $l$ has been renamed to a
        fresh location $l''$.

        Let $l''$ be a location such that $l'' \notin \dom{S}$ and
        $l'' \neq l$ (and hence $l'' \neq l'$, as well).

        Then choose $S' =
        \extS{\extS{S}{l''}{\bot}{\frozenfalse}}{l}{\bot}{\frozenfalse}$,
        $i = 1$, $j = 1$, and $\pi = \setof{(l, l'')}$.

        Either
        $\config{\extS{\extS{S}{l''}{\bot}{\frozenfalse}}{l}{\bot}{\frozenfalse}}{\evalctxt{E'_a}{\pi(e_{a_2})}}$
        or
        $\config{\extS{\extS{S}{l''}{\bot}{\frozenfalse}}{l}{\bot}{\frozenfalse}}{\evalctxt{E'_b}{\pi(e_{b_2})}}$
        would work as a final configuration; we choose

        $\config{\extS{\extS{S}{l''}{\bot}{\frozenfalse}}{l}{\bot}{\frozenfalse}}{\evalctxt{E'_b}{\pi(e_{b_2})}}$.

        We have to show that:
        \begin{itemize}
        \item $\config{S_a}{\evalctxt{E'_b}{e_{b_1}}} \ctxstepsto
          \config{\extS{\extS{S}{l''}{\bot}{\frozenfalse}}{l}{\bot}{\frozenfalse}}{\evalctxt{E'_b}{\pi(e_{b_2})}}$,
          and
        \item $\pi(\config{S_b}{\evalctxt{E'_a}{e_{a_1}}})
          \ctxstepsto
          \config{\extS{\extS{S}{l''}{\bot}{\frozenfalse}}{l}{\bot}{\frozenfalse}}{\evalctxt{E'_b}{\pi(e_{b_2})}}$.
        \end{itemize}

        For the first of these, since $\config{S}{e_{b_1}}
        \parstepsto \config{S_b}{e_{b_2}}$, we have by
        Lemma~\ref{lem:permutability} (Permutability) that
        $\pi(\config{S}{e_{b_1}}) \parstepsto
        \pi(\config{S_b}{e_{b_2}})$.

        Since $\pi = \setof{(l, l'')}$, but $l \notin S$ (from the
        side condition on {\sc E-New}), we have that
        $\pi(\config{S}{e_{b_1}}) = \config{S}{e_{b_1}}$.

        Since $\config{S_b}{e_{b_2}} =
        \config{\extS{S}{l'}{\bot}{\frozenfalse}}{l'}$, and $l = l'$,
        we have that $\pi(\config{S_b}{e_{b_2}}) =
        \config{\extS{S}{l''}{\bot}{\frozenfalse}}{\pi(e_{b_2})}$.

        Hence $\config{S}{e_{b_1}} \parstepsto
        \config{\extS{S}{l''}{\bot}{\frozenfalse}}{\pi(e_{b_2})}$.

        Let $U_S$ be the store update operation that acts as the
        identity on the contents of all existing locations, and adds
        the binding $\storebinding{l}{\bot}{\frozenfalse}$ if no
        binding for $l$ exists.

        Note that:
        \begin{itemize}
        \item $U_S$ is non-conflicting with $\config{S}{e_{b_1}}
          \parstepsto
          \config{\extS{S}{l''}{\bot}{\frozenfalse}}{\pi(e_{b_2})}$,
          since the only location allocated in the transition is
          $l''$;
        \item $U_S(\extS{S}{l''}{\bot}{\frozenfalse}) \neq \topS$,
          since $U_S(\extS{S}{l''}{\bot}{\frozenfalse}) = \\
          \extS{\extS{S}{l''}{\bot}{\frozenfalse}}{l}{\bot}{\frozenfalse}$
          and we know $S \neq \topS$ and the addition of new
          bindings $\storebinding{l}{\bot}{\frozenfalse}$ and
          $\storebinding{l''}{\bot}{\frozenfalse}$ cannot cause it
          to become $\topS$; and
        \item $U_S$ is freeze-safe with $\config{S}{e_{b_1}}
          \parstepsto
          \config{\extS{S}{l''}{\bot}{\frozenfalse}}{\pi(e_{b_2})}$,
          since $l'' \notin \dom{S}$, so there are no locations
          whose contents differ in status between $S$ and
          $\extS{S}{l''}{\bot}{\frozenfalse}$.
        \end{itemize}

        Therefore, by Lemma~\ref{lem:generalized-independence}
        (Generalized Independence), we have that

        $\config{U_S(S)}{e_{b_1}} \parstepsto
        \config{U_S(\extS{S}{l''}{\bot}{\frozenfalse})}{\pi(e_{b_2})}$.

        Hence $\config{\extS{S}{l}{\bot}{\frozenfalse}}{e_{b_1}}
        \parstepsto
        \config{\extS{\extS{S}{l''}{\bot}{\frozenfalse}}{l}{\bot}{\frozenfalse}}{\pi(e_{b_2})}$.

        By {\sc E-Eval-Ctxt} it follows that

        $\config{\extS{S}{l}{\bot}{\frozenfalse}}{\evalctxt{E'_b}{e_{b_1}}}
        \parstepsto
        \config{\extS{\extS{S}{l''}{\bot}{\frozenfalse}}{l}{\bot}{\frozenfalse}}{\evalctxt{E'_b}{\pi(e_{b_2})}}$,

        which, since $\extS{S}{l}{\bot}{\frozenfalse} = S_a$, is what
        we were required to show.

        For the second, observe that since $S_b =
        \extS{S}{l}{\bot}{\frozenfalse}$, we have that $\pi(S_b) =
        \extS{S}{l''}{\bot}{\frozenfalse}$.

        Also, since $l$ does not occur in $e_{a_1}$, we have that
        $\pi(\evalctxt{E'_a}{e_{a_1}}) =
        \evalctxt{(\pi(E'_a))}{e_{a_1}}$.

        Hence we have to show that

        $\config{\extS{S}{l''}{\bot}{\frozenfalse}}{\evalctxt{(\pi(E'_a))}{e_{a_1}}}
        \ctxstepsto \\
        \config{\extS{\extS{S}{l''}{\bot}{\frozenfalse}}{l}{\bot}{\frozenfalse}}{\evalctxt{E'_b}{\pi(e_{b_2})}}$.

        Let $U_S$ be the store update operation that acts as the
        identity on the contents of all existing locations, and adds
        the binding $\storebinding{l''}{\bot}{\frozenfalse}$ if no
        binding for $l''$ exists.

        Note that:
        \begin{itemize}
        \item $U_S$ is non-conflicting with $\config{S}{e_{a_1}}
          \parstepsto \config{S_a}{e_{a_2}}$, since the only
          location allocated in the transition is $l$;
        \item $U_S(S_a) \neq \topS$, since $U_S(S_a) =
          \extS{\extS{S}{l''}{\bot}{\frozenfalse}}{l}{\bot}{\frozenfalse}$
          and we know $S \neq \topS$ and the addition of new
          bindings $\storebinding{l}{\bot}{\frozenfalse}$ and
          $\storebinding{l''}{\bot}{\frozenfalse}$ cannot cause it
          to become $\topS$; and
        \item $U_S$ is freeze-safe with $\config{S}{e_{a_1}}
          \parstepsto \config{S_a}{e_{a_2}}$, since $S_a =
          \extS{S}{l}{\bot}{\frozenfalse}$ and $l \notin \dom{S}$,
          so there are no locations whose contents differ in status
          between $S$ and $S_a$.
        \end{itemize}

        Therefore, by Lemma~\ref{lem:generalized-independence}
        (Generalized Independence), we have that

        $\config{U_S(S)}{e_{a_1}} \parstepsto
        \config{U_S(S_a)}{e_{a_2}}$.

        Hence $\config{\extS{S}{l''}{\bot}{\frozenfalse}}{e_{a_1}}
        \parstepsto
        \config{\extS{\extS{S}{l''}{\bot}{\frozenfalse}}{l}{\bot}{\frozenfalse}}{e_{a_2}}$.

        By {\sc E-Eval-Ctxt} it follows that
        
        $\config{\extS{S}{l''}{\bot}{\frozenfalse}}{\evalctxt{(\pi(E'_a))}{e_{a_1}}}
        \ctxstepsto \\
        \config{\extS{\extS{S}{l''}{\bot}{\frozenfalse}}{l}{\bot}{\frozenfalse}}{\evalctxt{(\pi(E'_a))}{e_{a_2}}}$,

        which completes the case since $\evalctxt{E'_b}{\pi(e_{b_2})}
        = \evalctxt{(\pi(E'_a))}{e_{a_2}}$.

        \lk{This assumes that you believe that
          $\evalctxt{E'_b}{\pi(e_{b_2})} =
          \evalctxt{(\pi(E'_a))}{e_{a_2}}$.}

      \end{itemize}

    \item \label{slqc-new-put}Case {\sc E-Put}: We have $S_a =
      \extS{S}{l}{\bot}{\frozenfalse}$ and $S_b =
      \extSRaw{S}{l'}{u_{p_i}(p_1)}$, where $l \neq l'$ (since $l
      \notin \dom{S}$, but $l' \in \dom{S}$).

      We have to show that:
      \begin{itemize}
      \item $\config{S_a}{\evalctxt{E'_b}{e_{b_1}}} \ctxstepsto
        \config{\extS{S_b}{l}{\bot}{\frozenfalse}}{\evalctxt{E'_b}{e_{b_2}}}$,
        and
      \item $\config{S_b}{\evalctxt{E'_a}{e_{a_1}}} \ctxstepsto
        \config{\extS{S_b}{l}{\bot}{\frozenfalse}}{\evalctxt{E'_a}{e_{a_2}}}$.
      \end{itemize}

      For the first of these, consider that $S_a =
      \extS{S}{l}{\bot}{\frozenfalse} = U_S(S)$, where $U_S$ is the
      store update operation that acts as the identity on the contents
      of all existing locations, and adds the binding
      $\storebinding{l}{\bot}{\frozenfalse}$ if no binding for $l$
      exists.

      Note that:
      \begin{itemize}
      \item $U_S$ is non-conflicting with $\config{S}{e_{b_1}}
        \parstepsto \config{S_b}{e_{b_2}}$, since no locations are
        allocated in the transition;
      \item $U_S(S_b) \neq \topS$, since $U_S(S_b) =
        \extS{S_b}{l}{\bot}{\frozenfalse}$, and we know $S_b \neq
        \topS$ and the addition of a new binding
        $\storebinding{l}{\bot}{\frozenfalse}$ cannot cause it to
        become $\topS$; and
      \item $U_S$ is freeze-safe with $\config{S}{e_{b_1}} \parstepsto
        \config{S_b}{e_{b_2}}$, since $S_b =
        \extSRaw{S}{l'}{u_{p_i}(p_1)}$ and $u_{p_i}$ does not alter
        the status of $p_1$.

        (By Definition~\ref{def:set-of-state-update-operations},
        $u_{p_i}$ can only change the status bit of a location if its
        contents are $\state{d}{\frozentrue}$ and $u_i(d) \neq d$, in
        which case $u_{p_i}$ changes the contents of the location to
        $\state{\top}{\frozenfalse}$; however, that cannot be the case
        here since then $u_{p_i}(p_1)$ would be $\topp$, contradicting
        the premise of {\sc E-Put}.)
      \end{itemize}

      Therefore, by Lemma~\ref{lem:generalized-independence}
      (Generalized Independence), we have that

      $\config{U_S(S)}{e_{b_1}} \parstepsto
      \config{U_S(S_b)}{e_{b_2}}$.

      Hence $\config{\extS{S}{l}{\bot}{\frozenfalse}}{e_{b_1}}
      \parstepsto
      \config{\extS{S_b}{l}{\bot}{\frozenfalse}}{e_{b_2}}$.

      By {\sc E-Eval-Ctxt}, it follows that

      $\config{\extS{S}{l}{\bot}{\frozenfalse}}{\evalctxt{E'_b}{e_{b_1}}}
      \ctxstepsto
      \config{\extS{S_b}{l}{\bot}{\frozenfalse}}{\evalctxt{E'_b}{e_{b_2}}}$,
      
      which, since $S_a = \extS{S}{l}{\bot}{\frozenfalse}$, is what we
      were required to show.

      For the second, let $U_S$ be the store update operation that
      applies $u_{p_i}$ to the contents of $l'$ if it exists, and adds
      a binding $\storebindingRaw{l'}{u_{p_i}(p_1)}$ if no binding for
      $l'$ exists.

      Consider that $S_b = U_S(S)$, and
      $\extS{S_b}{l}{\bot}{\frozenfalse} =
      \extSRaw{S_a}{l'}{u_{p_i}(p_1)} = U_S(S_a)$.

      Note that:
      \begin{itemize}
      \item $U_S$ is non-conflicting with $\config{S}{e_{a_1}}
        \parstepsto \config{S_a}{e_{a_2}}$, since the only location
        allocated in the transition is $l$;
      \item $U_S(S_a) \neq \topS$, since $U_S(S_a) =
        \extSRaw{\extS{S}{l}{\bot}{\frozenfalse}}{l'}{u_{p_i}(p_1)}$
        and we know $S \neq \topS$ and the addition of a new binding
        $\storebinding{l}{\bot}{\frozenfalse}$ and updating the
        contents of location $l'$ to $u_{p_i}(p_1)$ in $S$ cannot
        cause it to become $\topS$ (since if $u_{p_i}(p_1) = \topp$,
        $\config{S}{e_{b_1}}$ would not have been able to step by {\sc
          E-Put}); and
      \item $U_S$ is freeze-safe with $\config{S}{e_{a_1}} \parstepsto
        \config{S_a}{e_{a_2}}$, since $S_a =
        \extS{S}{l}{\bot}{\frozenfalse}$ and $l \notin \dom{S}$, so
        there are no locations whose contents differ in status between
        $S$ and $S_a$.
      \end{itemize}

      Therefore, by Lemma~\ref{lem:generalized-independence}
      (Generalized Independence), we have that

      $\config{U_S(S)}{e_{a_1}} \parstepsto
      \config{U_S(S_a)}{e_{a_2}}$.

      Hence $\config{S_b}{e_{a_1}}
      \parstepsto
      \config{\extS{S_b}{l}{\bot}{\frozenfalse}}{e_{a_2}}$.

      By {\sc E-Eval-Ctxt}, it follows that
      
      $\config{S_b}{\evalctxt{E'_a}{e_{a_1}}} \ctxstepsto
      \config{\extS{S_b}{l}{\bot}{\frozenfalse}}{\evalctxt{E'_a}{e_{a_2}}}$,
      
      as we were required to show.

    \item \label{slqc-new-put-err}Case {\sc E-Put-Err}: We have $S_a =
      \extS{S}{l}{\bot}{\frozenfalse}$ and $\config{S_b}{e_{b_2}} =
      \error$, and so we choose $\conf_c = \error$, $i = 1$, $j = 0$,
      and $\pi = \id$.

      We have to show that:
      \begin{itemize}
      \item $\config{S_a}{\evalctxt{E'_b}{e_{b_1}}} \ctxstepsto
        \error$, and
      \item $\config{S_b}{\evalctxt{E'_a}{e_{a_1}}} = \error$.
      \end{itemize}

      The second of these is immediately true because since
      $\config{S_b}{e_{b_2}} = \error$, $S_b = \topS$, and so
      $\config{S_b}{\evalctxt{E'_a}{e_{a_1}}}$ is equal to $\error$ as
      well.

      For the first, observe that since $\config{S}{e_{a_1}}
      \parstepsto \config{S_a}{e_{a_2}}$, we have by
      Lemma~\ref{lem:monotonicity} (Monotonicity) that
      $\leqstore{S}{S_a}$.

      Therefore, since $\config{S}{e_{b_1}} \parstepsto \error$,

      we have by Lemma~\ref{lem:error-preservation} (Error
      Preservation) that $\config{S_a}{e_{b_1}} \parstepsto \error$.

      Since $\error$ is equal to $\config{\topS}{e}$ for all
      expressions $e$, $\config{S_a}{e_{b_1}} \parstepsto
      \config{\topS}{e}$ for all $e$.

      Therefore, by {\sc E-Eval-Ctxt},
      $\config{S_a}{\evalctxt{E'_b}{e_{b_1}}} \ctxstepsto
      \config{\topS}{\evalctxt{E'_b}{e}}$ for all $e$.

      Since $\config{\topS}{\evalctxt{E'_b}{e}}$ is equal to $\error$,
      we have that $\config{S_a}{\evalctxt{E'_b}{e_{b_1}}} \ctxstepsto
      \error$, as we were required to show.

    \item \label{slqc-new-get}Case {\sc E-Get}: Similar to
      case~\ref{slqc-new-beta}, since $S_a =
      \extS{S}{l}{\bot}{\frozenfalse}$ and $S_b = S$.
    \item \label{slqc-new-freeze-init}Case {\sc E-Freeze-Init}:
      Similar to case~\ref{slqc-new-beta}, since $S_a =
      \extS{S}{l}{\bot}{\frozenfalse}$ and $S_b = S$.
    \item \label{slqc-new-spawn-handler}Case {\sc E-Spawn-Handler}:
      Similar to case~\ref{slqc-new-beta}, since $S_a =
      \extS{S}{l}{\bot}{\frozenfalse}$ and $S_b = S$.
    \item \label{slqc-new-freeze-final}Case {\sc E-Freeze-Final}: We
      have $S_a = \extS{S}{l}{\bot}{\frozenfalse}$ and $S_b =
      \extS{S}{l'}{d_1}{\frozentrue}$, where $l \neq l'$ (since $l
      \notin \dom{S}$, but $l' \in \dom{S}$).

      Choose $S' =
      \extS{\extS{S}{l}{\bot}{\frozenfalse}}{l'}{d_1}{\frozentrue}$,
      $i = i$, $j = 1$, and $\pi = \id$.

      We have to show that:
      \begin{itemize}
      \item
        $\config{\extS{S}{l}{\bot}{\frozenfalse}}{\evalctxt{E'_b}{e_{b_1}}}
        \ctxstepsto
        \config{\extS{\extS{S}{l}{\bot}{\frozenfalse}}{l'}{d_1}{\frozentrue}}{\evalctxt{E'_b}{e_{b_2}}}$,
        and
      \item
        $\config{\extS{S}{l'}{d_1}{\frozentrue}}{\evalctxt{E'_a}{e_{a_1}}}
        \ctxstepsto
        \config{\extS{\extS{S}{l}{\bot}{\frozenfalse}}{l'}{d_1}{\frozentrue}}{\evalctxt{E'_a}{e_{a_2}}}$.
      \end{itemize}

      For the first of these, consider that
      $\extS{S}{l}{\bot}{\frozenfalse} = U_S(S)$, where $U_S$ is the
      store update operation that acts as the identity on the contents
      of all existing locations, and adds the binding
      $\storebinding{l}{\bot}{\frozenfalse}$ if no binding for $l$
      exists.

      Note that:
      \begin{itemize}
      \item $U_S$ is non-conflicting with $\config{S}{e_{b_1}}
        \parstepsto \config{S_b}{e_{b_2}}$, since no locations are
        allocated in the transition;
      \item $U_S(S_b) \neq \topS$, since $U_S(S_b) =
        \extS{S_b}{l}{\bot}{\frozenfalse}$, and we know $S_b \neq
        \topS$ and the addition of a new binding
        $\storebinding{l}{\bot}{\frozenfalse}$ cannot cause it to
        become $\topS$; and
      \item $U_S$ is freeze-safe with $\config{S}{e_{b_1}}
        \parstepsto \config{S_b}{e_{b_2}}$, since $S_b =
        \extS{S}{l'}{d_1}{\frozentrue}$ and so the only location
        that can change in status between $S$ and $S_b$ is $l'$, and
        $U_S$ acts as the identity on $l'$.
      \end{itemize}
      Therefore, by Lemma~\ref{lem:generalized-independence}
      (Generalized Independence), we have that

      $\config{U_S(S)}{e_{b_1}} \parstepsto
      \config{U_S(S_b)}{e_{b_2}}$.

      Hence $\config{\extS{S}{l}{\bot}{\frozenfalse}}{e_{b_1}}
      \parstepsto
      \config{\extS{\extS{S}{l}{\bot}{\frozenfalse}}{l'}{d_1}{\frozentrue}}{e_{b_2}}$.

      By {\sc E-Eval-Ctxt}, it follows that

      $\config{\extS{S}{l}{\bot}{\frozenfalse}}{\evalctxt{E'_b}{e_{b_1}}}
      \ctxstepsto
      \config{\extS{\extS{S}{l}{\bot}{\frozenfalse}}{l'}{d_1}{\frozentrue}}{\evalctxt{E'_b}{e_{b_2}}}$,

      as we were required to show.

      For the second, consider that $\extS{S}{l'}{d_1}{\frozentrue} =
      U_S(S)$, where $U_S$ is the store update operation that freezes
      the contents of $l'$ and acts as the identity on the contents of
      all other locations.

      Note that:
      \begin{itemize}
      \item $U_S$ is non-conflicting with $\config{S}{e_{a_1}}
        \parstepsto \config{S_a}{e_{a_2}}$, since the only location
        allocated in the transition is $l$, and $l \neq l'$;
      \item $U_S(S_a) \neq \topS$, since $U_S(S_a) =
        \extS{S_a}{l'}{d_1}{\frozentrue} =
        \extS{S_b}{l}{\bot}{\frozenfalse}$, and we know $S_b \neq
        \topS$ and the addition of a new binding
        $\storebinding{l}{\bot}{\frozenfalse}$ cannot cause it to
        become $\topS$; and
      \item $U_S$ is freeze-safe with $\config{S}{e_{a_1}}
        \parstepsto \config{S_a}{e_{a_2}}$, since $S_a =
        \extS{S}{l}{\bot}{\frozenfalse}$ and $l \notin \dom{S}$, so
        there are no locations whose contents differ in status
        between $S$ and $S_a$.
      \end{itemize}

      Therefore, by Lemma~\ref{lem:generalized-independence}
      (Generalized Independence), we have that

      $\config{U_S(S)}{e_{a_1}} \parstepsto
      \config{U_S(S_a)}{e_{a_2}}$.

      Hence $\config{\extS{S}{l'}{d_1}{\frozentrue}}{e_{a_1}}
      \parstepsto
      \config{\extS{\extS{S}{l}{\bot}{\frozenfalse}}{l'}{d_1}{\frozentrue}}{e_{a_2}}$.

      By {\sc E-Eval-Ctxt} it follows that

      $\config{\extS{S}{l'}{d_1}{\frozentrue}}{\evalctxt{E'_a}{e_{a_1}}}
      \ctxstepsto
      \config{\extS{\extS{S}{l}{\bot}{\frozenfalse}}{l'}{d_1}{\frozentrue}}{\evalctxt{E'_a}{e_{a_2}}}$,

      as we were required to show.

    \item \label{slqc-new-freeze-simple}Case {\sc E-Freeze-Simple}:
      Similar to case~\ref{slqc-new-freeze-final}, since $S_a =
      \extS{S}{l}{\bot}{\frozenfalse}$ and $S_b =
      \extS{S}{l'}{d_1}{\frozentrue}$, where $l \neq l'$ (since $l
      \notin \dom{S}$, but $l' \in \dom{S}$).

    \end{enumerate}
  \item Case {\sc E-Put}: We have $S_a =
    \extSRaw{S}{l}{u_{p_i}(p_1)}$.

    We proceed by case analysis on the rule by which
    $\config{S}{e_{b_1}}$ steps to $\config{S_b}{e_{b_2}}$.

    Since the only way an $\error$ configuration can arise is by the
    {\sc E-Put-Err} rule, we can assume in all other cases that
    $\conf_b \neq \error$.
    \begin{enumerate}
    \item \label{slqc-put-beta}Case {\sc E-Beta}: By symmetry with case~\ref{slqc-beta-put}.
    \item \label{slqc-put-new}Case {\sc E-New}: By symmetry with case~\ref{slqc-new-put}.
    \item \label{slqc-put-put}Case {\sc E-Put}: We have $S_a =
      \extSRaw{S}{l}{u_{p_i}(p_1)}$ and $S_b =
      \extSRaw{S}{l'}{u_{p_j}(p'_1)}$, where $p'_1 = S(l')$.

      Now consider whether $l = l'$:
      \begin{itemize}
      \item If $l \neq l'$:

        Choose $S' =
        \extSRaw{\extSRaw{S}{l'}{u_{p_j}(p'_1)}}{l}{u_{p_i}(p_1)}$,
        $i = 1$, $j = 1$, and $\pi = \id$.

        We have to show that:
        \begin{itemize}
        \item
          $\config{\extSRaw{S}{l}{u_{p_i}(p_1)}}{\evalctxt{E'_b}{e_{b_1}}}
          \ctxstepsto
          \config{\extSRaw{\extSRaw{S}{l'}{u_{p_j}(p'_1)}}{l}{u_{p_i}(p_1)}}{\evalctxt{E'_b}{e_{b_2}}}$,
          and
        \item
          $\config{\extSRaw{S}{l'}{u_{p_j}(p'_1)}}{\evalctxt{E'_a}{e_{a_1}}}
          \ctxstepsto
          \config{\extSRaw{\extSRaw{S}{l'}{u_{p_j}(p'_1)}}{l}{u_{p_i}(p_1)}}{\evalctxt{E'_a}{e_{a_2}}}$.
        \end{itemize}

        For the first of these, consider that
        $\extSRaw{S}{l}{u_{p_i}(p_1)} = U_S(S)$, where $U_S$ is the
        store update operation that applies $u_{p_i}$ to the
        contents of $l$ if it exists, and adds a binding
        $\storebindingRaw{l}{u_{p_i}(p_1)}$ if no binding for $l$
        exists.

        Note that:
        \begin{itemize}
        \item $U_S$ is non-conflicting with $\config{S}{e_{b_1}}
          \parstepsto
          \config{\extSRaw{S}{l'}{u_{p_j}(p'_1)}}{e_{b_2}}$, since
          no locations are allocated in the transition;
        \item $U_S(\extSRaw{S}{l'}{u_{p_j}(p'_1)}) \neq \topS$,
          since $U_S(\extSRaw{S}{l'}{u_{p_j}(p'_1)}) =
          \extSRaw{\extSRaw{S}{l'}{u_{p_j}(p'_1)}}{l}{u_{p_i}(p_1)}$
          and we know $S \neq \topS$ and updating the contents of
          location $l$ to $u_{p_i}(p_1)$ and the contents of
          location $l'$ to $u_{p_j}(p'_1)$ in $S$ cannot cause it to
          become $\topS$ (because if so, then we would have $S_a =
          \topS$ or $S_b = \topS$, which we know are not the case);
          and
        \item $U_S$ is freeze-safe with $\config{S}{e_{b_1}}
          \parstepsto
          \config{\extSRaw{S}{l'}{u_{p_j}(p'_1)}}{e_{b_2}}$, since
          $u_{p_j}$ does not alter the status of $p'_1$.

          (By Definition~\ref{def:set-of-state-update-operations},
          $u_{p_j}$ can only change the status bit of a location if
          its contents are $\state{d}{\frozentrue}$ and $u_j(d) \neq
          d$, in which case $u_{p_j}$ changes the contents of the
          location to $\state{\top}{\frozenfalse}$; however, that
          cannot be the case here since then $u_{p_j}(p'_1)$ would be
          $\topp$, contradicting the premise of {\sc E-Put}.)
        \end{itemize}

        Therefore, by Lemma~\ref{lem:generalized-independence}
        (Generalized Independence), we have that

        $\config{U_S(S)}{e_{b_1}} \parstepsto
        \config{U_S(\extSRaw{S}{l'}{u_{p_j}(p'_1)})}{e_{b_2}}$.

        Hence $\config{\extSRaw{S}{l}{u_{p_i}(p_1)}}{e_{b_1}}
        \parstepsto
        \config{\extSRaw{\extSRaw{S}{l'}{u_{p_j}(p'_1)}}{l}{u_{p_i}(p_1)}}{e_{b_2}}$.

        By {\sc E-Eval-Ctxt}, it follows that

        $\config{\extSRaw{S}{l}{u_{p_i}(p_1)}}{\evalctxt{E'_b}{e_{b_1}}}
        \ctxstepsto
        \config{\extSRaw{\extSRaw{S}{l'}{u_{p_j}(p'_1)}}{l}{u_{p_i}(p_1)}}{\evalctxt{E'_b}{e_{b_2}}}$,

        as we were required to show.

        The argument for the second is symmetrical.

      \item If $l = l'$:
        Note that since $l = l'$, $p_1 = p'_1$ as well.

        Consider whether $u_{p_i}(u_{p_j}(p_1)) = \topp$:
        \begin{itemize}
        \item If $u_{p_i}(u_{p_j}(p_1)) = \topp$:

          Choose $\conf_c = \error$, $i = 1$, $j = 1$, and $\pi =
          \id$.

          We have to show that:

          \begin{itemize}
          \item
            $\config{\extSRaw{S}{l}{u_{p_i}(p_1)}}{\evalctxt{E'_b}{e_{b_1}}}
            \ctxstepsto \error$, and
          \item
            $\config{\extSRaw{S}{l}{u_{p_j}(p_1)}}{\evalctxt{E'_a}{e_{a_1}}}
            \ctxstepsto \error$.
          \end{itemize}

          For the first of these, consider that
          $\extSRaw{S}{l}{u_{p_i}(p_1)} = U_S(S)$, where $U_S$ is the
          store update operation that applies $u_{p_i}$ to the
          contents of $l$ if it exists.

          Note that:
          \begin{itemize}
          \item $U_S$ is non-conflicting with $\config{S}{e_{b_1}}
            \parstepsto
            \config{\extSRaw{S}{l}{u_{p_j}(p_1)}}{e_{b_2}}$, since
            no locations are allocated in the transition;
          \item $U_S(\extSRaw{S}{l}{u_{p_j}(p_1)}) = \topS$, since
            $U_S(\extSRaw{S}{l}{u_{p_j}(p_1)}) =
            \extSRaw{S}{l}{u_{p_i}(u_{p_j}(p_1))}$ and we know
            $u_{p_i}(u_{p_j}(p_1)) = \topp$ in this case;
          \item $U_S$ is freeze-safe with $\config{S}{e_{b_1}}
            \parstepsto
            \config{\extSRaw{S}{l}{u_{p_j}(p_1)}}{e_{b_2}}$, since
            $u_{p_j}$ does not alter the status of $p_1$.

            (By Definition~\ref{def:set-of-state-update-operations},
            $u_{p_j}$ can only change the status bit of a location if
            its contents are $\state{d}{\frozentrue}$ and $u_j(d) \neq
            d$, in which case $u_{p_j}$ changes the contents of the
            location to $\state{\top}{\frozenfalse}$; however, that
            cannot be the case here since then $u_{p_j}(p_1)$ would be
            $\topp$, contradicting the premise of {\sc E-Put}.)
          \end{itemize}

          Therefore, by Lemma~\ref{lem:generalized-clash}
          (Generalized Clash), we have that there exists $i' \leq 1$
          such that $\config{U_S(S)}{e_{b_1}} \parstepsto^{i'}
          \error$.

          Hence $\config{\extSRaw{S}{l}{u_{p_i}(p_1)}}{e_{b_1}}
          \parstepsto^{i'} \error$.

          If $i' = 0$, we would have
          $\config{\extSRaw{S}{l}{u_{p_i}(p_1)}}{e_{b_1}} =
          \config{S_a}{e_{b_1}} = \error$.

          So we would have $S_a = \topS$ by the definition of
          $\error$, but then we would have $\conf_a = \error$, a
          contradiction.

          Therefore $i' = 1$, and so we have
          $\config{\extSRaw{S}{l}{u_{p_i}(p_1)}}{e_{b_1}} \parstepsto
          \error$.

          Since $\error = \config{\topS}{e}$ for all $e$, we have
          $\config{\extSRaw{S}{l}{u_{p_i}(p_1)}}{e_{b_1}}
          \parstepsto \config{\topS}{e}$ for all $e$.

          So, by {\sc E-Eval-Ctxt}, we have that
          $\config{\extSRaw{S}{l}{u_{p_i}(p_1)}}{\evalctxt{E'_b}{e_{b_1}}}
          \parstepsto \config{\topS}{\evalctxt{E'_b}{e}}$ for all $e$.

          Hence
          $\config{\extSRaw{S}{l}{u_{p_i}(p_1)}}{\evalctxt{E'_b}{e_{b_1}}}
          \parstepsto \error$.

          The argument for the second is symmetrical.

        \item If $u_{p_i}(u_{p_j}(p_1)) \neq \topp$:

          Choose $S' = \extSRaw{S}{l}{u_{p_i}(u_{p_j}(p_1))}$, $i =
          1$, $j = 1$, and $\pi = \id$.

          We have to show that:
          \begin{itemize}
          \item
            $\config{\extSRaw{S}{l}{u_{p_i}(p_1)}}{\evalctxt{E'_b}{e_{b_1}}}
            \ctxstepsto
            \config{\extSRaw{S}{l}{u_{p_i}(u_{p_j}(p_1))}}{\evalctxt{E'_b}{e_{b_2}}}$,
            and
          \item
            $\config{\extSRaw{S}{l}{u_{p_j}(p_1)}}{\evalctxt{E'_a}{e_{a_1}}}
            \ctxstepsto
            \config{\extSRaw{S}{l}{u_{p_i}(u_{p_j}(p_1))}}{\evalctxt{E'_a}{e_{a_2}}}$.
          \end{itemize}

          For the first of these, consider that
          $\extSRaw{S}{l}{u_{p_i}(p_1)} = U_S(S)$, where $U_S$ is the
          store update operation that applies $u_{p_i}$ to the
          contents of $l$ if it exists.

          Note that:
          \begin{itemize}
          \item $U_S$ is non-conflicting with $\config{S}{e_{b_1}}
            \parstepsto
            \config{\extSRaw{S}{l}{u_{p_j}(p_1)}}{e_{b_2}}$, since no
            locations are allocated in the transition;
          \item $U_S(\extSRaw{S}{l}{u_{p_j}(p_1)}) \neq \topS$, since
            $U_S(\extSRaw{S}{l}{u_{p_j}(p_1)}) =
            \extSRaw{S}{l}{u_{p_i}(u_{p_j}(p_1))}$ and we know $S \neq
            \topS$ and $u_{p_i}(u_{p_j}(p_1)) \neq \topp$ in this
            case;
          \item $U_S$ is freeze-safe with $\config{S}{e_{b_1}}
            \parstepsto
            \config{\extSRaw{S}{l}{u_{p_j}(p_1)}}{e_{b_2}}$, since
            $u_{p_j}$ does not alter the status of $p_1$.

            (By Definition~\ref{def:set-of-state-update-operations},
            $u_{p_j}$ can only change the status bit of a location if
            its contents are $\state{d}{\frozentrue}$ and $u_j(d) \neq
            d$, in which case $u_{p_j}$ changes the contents of the
            location to $\state{\top}{\frozenfalse}$; however, that
            cannot be the case here since then $u_{p_j}(p_1)$ would be
            $\topp$, contradicting the premise of {\sc E-Put}.)
          \end{itemize}

          Therefore, by Lemma~\ref{lem:generalized-independence}
          (Generalized Independence), we have that

          $\config{U_S(S)}{e_{b_1}} \parstepsto
          \config{U_S(\extSRaw{S}{l}{u_{p_j}(p_1)})}{e_{b_2}}$.

          Hence $\config{\extSRaw{S}{l}{u_{p_i}(p_1)}}{e_{b_1}}
          \parstepsto
          \config{\extSRaw{S}{l}{u_{p_i}(u_{p_j}(p_1))}}{e_{b_2}}$.

          By {\sc E-Eval-Ctxt}, it follows that

          $\config{\extSRaw{S}{l}{u_{p_i}(p_1)}}{\evalctxt{E'_b}{e_{b_1}}}
          \ctxstepsto
          \config{\extSRaw{S}{l}{u_{p_i}(u_{p_j}(p_1))}}{\evalctxt{E'_b}{e_{b_2}}}$,

          as we were required to show.

          The argument for the second is symmetrical.

        \end{itemize}

      \end{itemize}

    \item \label{slqc-put-put-err}Case {\sc E-Put-Err}: We have $S_a =
      \extSRaw{S}{l}{u_{p_i}(p_1)}$ and $\config{S_b}{e_{b_2}} =
      \error$, and so we choose $\conf_c = \error$, $i = 1$, $j = 0$,
      and $\pi = \id$.

      We have to show that:
      \begin{itemize}
      \item $\config{S_a}{\evalctxt{E'_b}{e_{b_1}}} \ctxstepsto
        \error$, and
      \item $\config{S_b}{\evalctxt{E'_a}{e_{a_1}}} = \error$.
      \end{itemize}

      The second of these is immediately true because since
      $\config{S_b}{e_{b_2}} = \error$, $S_b = \topS$, and so
      $\config{S_b}{\evalctxt{E'_a}{e_{a_1}}}$ is equal to $\error$ as
      well.

      For the first, observe that since $\config{S}{e_{a_1}}
      \parstepsto \config{S_a}{e_{a_2}}$, we have by
      Lemma~\ref{lem:monotonicity} (Monotonicity) that
      $\leqstore{S}{S_a}$.

      Therefore, since $\config{S}{e_{b_1}} \parstepsto \error$,

      we have by Lemma~\ref{lem:error-preservation} (Error
      Preservation) that $\config{S_a}{e_{b_1}} \parstepsto \error$.
      
      Since $\error$ is equal to $\config{\topS}{e}$ for all
      expressions $e$, $\config{S_a}{e_{b_1}} \parstepsto
      \config{\topS}{e}$ for all $e$.

      Therefore, by {\sc E-Eval-Ctxt},
      $\config{S_a}{\evalctxt{E'_b}{e_{b_1}}} \ctxstepsto
      \config{\topS}{\evalctxt{E'_b}{e}}$ for all $e$.

      Since $\config{\topS}{\evalctxt{E'_b}{e}}$ is equal to $\error$,
      we have that $\config{S_a}{\evalctxt{E'_b}{e_{b_1}}} \ctxstepsto
      \error$, as we were required to show.

    \item \label{slqc-put-get}Case {\sc E-Get}: Similar to
      case~\ref{slqc-put-beta}, since $S_a =
      \extSRaw{S}{l}{u_{p_i}(p_1)}$ and $S_b = S$.
    \item \label{slqc-put-freeze-init}Case {\sc E-Freeze-Init}:
      Similar to case~\ref{slqc-put-beta}, since $S_a =
      \extSRaw{S}{l}{u_{p_i}(p_1)}$ and $S_b = S$.
    \item \label{slqc-put-spawn-handler}Case {\sc E-Spawn-Handler}:
      Similar to case~\ref{slqc-put-beta}, since $S_a =
      \extSRaw{S}{l}{u_{p_i}(p_1)}$ and $S_b = S$.
    \item \label{slqc-put-freeze-final}Case {\sc E-Freeze-Final}: We
      have $S_a = \extSRaw{S}{l}{u_{p_i}(p_1)}$ and $S_b =
      \extS{S}{l'}{d_1}{\frozentrue}$.

      Now consider whether $l = l'$:
      \begin{itemize}
      \item If $l \neq l'$:

        Choose $S' =
        \extS{\extSRaw{S}{l}{u_{p_i}(p_1)}}{l'}{d_1}{\frozentrue}$,
        $i = 1$, $j = 1$, and $\pi = \id$.

        We have to show that:
        \begin{itemize}
        \item
          $\config{\extSRaw{S}{l}{u_{p_i}(p_1)}}{\evalctxt{E'_b}{e_{b_1}}}
          \ctxstepsto
          \config{\extS{\extSRaw{S}{l}{u_{p_i}(p_1)}}{l'}{d_1}{\frozentrue}}{\evalctxt{E'_b}{e_{b_2}}}$,
          and
        \item
          $\config{\extS{S}{l'}{d_1}{\frozentrue}}{\evalctxt{E'_a}{e_{a_1}}}
          \ctxstepsto
          \config{\extS{\extSRaw{S}{l}{u_{p_i}(p_1)}}{l'}{d_1}{\frozentrue}}{\evalctxt{E'_a}{e_{a_2}}}$.
        \end{itemize}

        For the first of these, consider that
        $\extSRaw{S}{l}{u_{p_i}(p_1)} = U_S(S)$, where $U_S$ is the
        store update operation that applies $u_{p_i}$ to the
        contents of $l$ if it exists, and adds a binding
        $\storebindingRaw{l}{u_{p_i}(p_1)}$ if no binding for $l$
        exists, and acts as the identity on all other locations.

        Note that:
        \begin{itemize}
        \item $U_S$ is non-conflicting with $\config{S}{e_{b_1}}
          \parstepsto
          \config{\extS{S}{l'}{d_1}{\frozentrue}}{e_{b_2}}$, since
          no locations are allocated in the transition;
        \item $U_S(\extS{S}{l'}{d_1}{\frozentrue}) \neq \topS$,

          since $U_S(\extS{S}{l'}{d_1}{\frozentrue}) =
          \extSRaw{\extS{S}{l'}{d_1}{\frozentrue}}{l}{u_{p_i}(p_1)}$
          and we know $S \neq \topS$ and updating the contents of
          location $l$ to $u_{p_i}(p_1)$ and freezing the contents
          of location $l'$ in $S$ cannot cause it to become $\topS$
          (because if so, then we would have $S_a = \topS$ or $S_b =
          \topS$, which we know are not the case); and
        \item $U_S$ is freeze-safe with $\config{S}{e_{b_1}}
          \parstepsto
          \config{\extS{S}{l'}{d_1}{\frozentrue}}{e_{b_2}}$, since
          the only location that can change in status between $S$
          and $\extS{S}{l'}{d_1}{\frozentrue}$ is $l'$, and $U_S$
          acts as the identity on $l'$.
        \end{itemize}
        Therefore, by Lemma~\ref{lem:generalized-independence}
        (Generalized Independence), we have that

        $\config{U_S(S)}{e_{b_1}} \parstepsto
        \config{U_S(\extS{S}{l'}{d_1}{\frozentrue})}{e_{b_2}}$.

        Hence $\config{\extSRaw{S}{l}{u_{p_i}(p_1)}}{e_{b_1}}
        \parstepsto
        \config{\extSRaw{\extS{S}{l'}{d_1}{\frozentrue}}{l}{u_{p_i}(p_1)}}{e_{b_2}}$.

        By {\sc E-Eval-Ctxt}, it follows that

        $\config{\extSRaw{S}{l}{u_{p_i}(p_1)}}{\evalctxt{E'_b}{e_{b_1}}}
        \ctxstepsto
        \config{\extSRaw{\extS{S}{l'}{d_1}{\frozentrue}}{l}{u_{p_i}(p_1)}}{\evalctxt{E'_b}{e_{b_2}}}$,
        
        as we were required to show.

        For the second, consider that
        $\extS{S}{l'}{d_1}{\frozentrue} = U_S(S)$, where $U_S$ is
        the store update operation that freezes the contents of $l'$
        and acts as the identity on the contents of all other
        locations.

        Note that:
        \begin{itemize}
        \item $U_S$ is non-conflicting with $\config{S}{e_{a_1}}
          \parstepsto
          \config{\extSRaw{S}{l}{u_{p_i}(p_1)}}{e_{a_2}}$, since no
          locations are allocated in the transition;
        \item $U_S(\extSRaw{S}{l}{u_{p_i}(p_1)}) \neq \topS$, since
          $U_S(\extSRaw{S}{l}{u_{p_i}(p_1)}) =
          \extS{\extSRaw{S}{l}{u_{p_i}(p_1)}}{l'}{d_1}{\frozentrue}$,
          and we know $S \neq \topS$ and updating the contents of
          location $l$ to $u_{p_i}(p_1)$ and freezing the contents
          of location $l$ in $S$ cannot cause it to become $\topS$
          (because if so, then we would have $S_a = \topS$ or $S_b =
          \topS$, which we know are not the case); and
        \item $U_S$ is freeze-safe with $\config{S}{e_{a_1}}
          \parstepsto
          \config{\extSRaw{S}{l}{u_{p_i}(p_1)}}{e_{a_2}}$, since
          $u_{p_i}$ does not alter the status of $p_1$.

          (By Definition~\ref{def:set-of-state-update-operations},
          $u_{p_i}$ can only change the status bit of a location if
          its contents are $\state{d}{\frozentrue}$ and $u_i(d) \neq
          d$, in which case $u_{p_i}$ changes the contents of the
          location to $\state{\top}{\frozenfalse}$; however, that
          cannot be the case here since then $u_{p_i}(p_1)$ would be
          $\topp$, and we would have $S_a = \topS$, a contradiction.)
        \end{itemize}
        Therefore, by Lemma~\ref{lem:generalized-independence}
        (Generalized Independence), we have that

        $\config{U_S(S)}{e_{a_1}} \parstepsto
        \config{U_S(\extSRaw{S}{l}{u_{p_i}(p_1)})}{e_{a_2}}$.

        Hence $\config{\extS{S}{l'}{d_1}{\frozentrue}}{e_{a_1}}
        \parstepsto
        \config{\extS{\extSRaw{S}{l}{u_{p_i}(p_1)}}{l'}{d_1}{\frozentrue}}{e_{a_2}}$.

        By {\sc E-Eval-Ctxt}, it follows that
        $\config{\extS{S}{l'}{d_1}{\frozentrue}}{\evalctxt{E'_a}{e_{a_1}}}
        \ctxstepsto
        \config{\extS{\extSRaw{S}{l}{u_{p_i}(p_1)}}{l'}{d_1}{\frozentrue}}{\evalctxt{E'_a}{e_{a_2}}}$,
        
        as we were required to show.

      \item If $l = l'$:

        We have two cases to consider:

        \begin{itemize}
        \item $u_{p_i}(\state{d_1}{\frozentrue}) = \topp$:

          \lk{This is the interesting case: the potential
            put-after-freeze case.  It's important to note that this
            case doesn't necessarily end in a put-after-freeze (and
            hence an error); all we're required to show is that it
            \emph{can} end that way.}

          Since $(\extSRaw{S}{l}{\state{d_1}{\frozentrue}})(l) =
          \state{d_1}{\frozentrue}$ and
          $u_{p_i}(\state{d_1}{\frozentrue}) = \topp$, by {\sc
            E-Put-Err} we have that
          $\config{\extSRaw{S}{l}{\state{d_1}{\frozentrue}}}{\putiexp{l}}
          \parstepsto \error$.

          Since $S_b = \extSRaw{S}{l}{\state{d_1}{\frozentrue}}$,
          we have that $\config{S_b}{\putiexp{l}} \parstepsto
          \error$.

          Since $\config{S}{e_{a_1}} \parstepsto
          \config{S_a}{e_{a_2}}$ by {\sc E-Put}, it must be the
          case that $e_{a_1} = \putiexp{l}$.

          Hence $\config{S_b}{e_{a_1}} \parstepsto \error$.

          Since $\error$ is equal to $\config{\topS}{e}$ for all
          expressions $e$, $\config{S_b}{e_{a_1}} \parstepsto
          \config{\topS}{e}$ for all $e$.

          Therefore, by {\sc E-Eval-Ctxt},
          $\config{S_b}{\evalctxt{E'_a}{e_{a_1}}} \ctxstepsto
          \config{\topS}{\evalctxt{E'_a}{e}}$ for all $e$.

          Since $\config{\topS}{\evalctxt{E'_a}{e}}$ is equal to
          $\error$, we have that
          $\config{S_b}{\evalctxt{E'_a}{e_{a_1}}} \ctxstepsto \error$.

          Since $\evalctxt{E'_a}{e_{a_1}} =
          \evalctxt{E_b}{e_{b_2}}$, we have that
          $\config{S_b}{\evalctxt{E_b}{e_{b_2}}} \ctxstepsto
          \error$.

          Since $\conf_b = \config{S_b}{\evalctxt{E_b}{e_{b_2}}}$,
          we therefore have that $\conf_b \ctxstepsto \error$, and
          the case is satisfied.

        \item $u_{p_i}(\state{d_1}{\frozentrue}) \neq \topp$:

          \lk{This is the case where there's a conflicting put and
            freeze, but the put is a no-op, so it doesn't matter.}

          In this case, by the definition of $U_p$
          (Definition~\ref{def:set-of-state-update-operations}),
          
          it must be the case that $u_{p_i}(\state{d_1}{\frozentrue})
          = \state{d_1}{\frozentrue}$.

          Choose $S' = \extS{S}{l}{d_1}{\frozentrue}$, $i = 1$, $j
          = 1$, and $\pi = \id$.

          We have to show that:
          \begin{itemize}
          \item
            $\config{\extSRaw{S}{l}{u_{p_i}(p_1)}}{\evalctxt{E'_b}{e_{b_1}}}
            \ctxstepsto
            \config{\extS{S}{l}{d_1}{\frozentrue}}{\evalctxt{E'_b}{e_{b_2}}}$,
            and
          \item
            $\config{\extS{S}{l}{d_1}{\frozentrue}}{\evalctxt{E'_a}{e_{a_1}}}
            \ctxstepsto
            \config{\extS{S}{l}{d_1}{\frozentrue}}{\evalctxt{E'_a}{e_{a_2}}}$.
          \end{itemize}

          For the first of these, consider that
          $\extSRaw{S}{l}{u_{p_i}(p_1)} = U_S(S)$, where $U_S$ is
          the store update operation that applies $u_{p_i}$ to the
          contents of $l$ if it exists, and adds a binding
          $\storebindingRaw{l}{u_{p_i}(p_1)}$ if no binding for
          $l$ exists, and acts as the identity on all other
          locations.

          Note that:
          \begin{itemize}
          \item $U_S$ is non-conflicting with $\config{S}{e_{b_1}}
            \parstepsto
            \config{\extS{S}{l}{d_1}{\frozentrue}}{e_{b_2}}$,
            since no locations are allocated in the
            transition;
          \item $U_S(\extS{S}{l}{d_1}{\frozentrue}) \neq \topS$,
            
            since $U_S(\extS{S}{l}{d_1}{\frozentrue}) =
            \extSRaw{S}{l}{u_{p_i}(\state{d_1}{\frozentrue})}$ and
            we know $S \neq \topS$ and
            $u_{p_i}(\state{d_1}{\frozentrue}) \neq \topp$; and
          \item $U_S$ is freeze-safe with $\config{S}{e_{b_1}}
            \parstepsto
            \config{\extS{S}{l}{d_1}{\frozentrue}}{e_{b_2}}$, since
            the only location that can change in status between $S$
            and $\extS{S}{l}{d_1}{\frozentrue}$ is $l$, and $U_S$
            acts as the identity on $l$.
          \end{itemize}
          Therefore, by Lemma~\ref{lem:generalized-independence}
          (Generalized Independence), we have that

          $\config{U_S(S)}{e_{b_1}} \parstepsto
          \config{U_S(\extS{S}{l}{d_1}{\frozentrue})}{e_{b_2}}$.

          Hence $\config{\extSRaw{S}{l}{u_{p_i}(p_1)}}{e_{b_1}}
          \parstepsto
          \config{\extSRaw{S}{l}{u_{p_i}(\state{d_1}{\frozentrue})}}{e_{b_2}}$.

          Since $u_{p_i}(\state{d_1}{\frozentrue}) =
          \state{d_1}{\frozentrue}$,

          we have that
          $\config{\extSRaw{S}{l}{u_{p_i}(p_1)}}{e_{b_1}}
          \parstepsto
          \config{\extS{S}{l}{d_1}{\frozentrue}}{e_{b_2}}$.

          By {\sc E-Eval-Ctxt}, it follows that

          $\config{\extSRaw{S}{l}{u_{p_i}(p_1)}}{\evalctxt{E'_b}{e_{b_1}}}
          \ctxstepsto
          \config{\extS{S}{l}{d_1}{\frozentrue}}{\evalctxt{E'_b}{e_{b_2}}}$,

          as we were required to show.

          For the second, consider that
          $\extS{S}{l}{d_1}{\frozentrue} = U_S(S)$, where $U_S$ is
          the store update operation that freezes the contents of $l$
          and acts as the identity on the contents of all other
          locations.

          Note that:
          \begin{itemize}
          \item $U_S$ is non-conflicting with $\config{S}{e_{a_1}}
            \parstepsto
            \config{\extSRaw{S}{l}{u_{p_i}(p_1)}}{e_{a_2}}$, since no
            locations are allocated in the transition;
          \item $U_S(\extSRaw{S}{l}{u_{p_i}(p_1)}) \neq \topS$,
            since $U_S(\extSRaw{S}{l}{u_{p_i}(p_1)}) =
            \extS{S}{l}{d_1}{\frozentrue}$ (since, by
            Definition~\ref{def:set-of-state-update-operations},
            $u_i(d_1) = d_1$; otherwise we would have
            $u_{p_i}(\state{d_1}{\frozentrue}) = \topp$, a
            contradiction), and we know $S \neq \topS$ and
            freezing the contents of location $l$ in $S$ cannot
            cause it to become $\topS$; and
          \item $U_S$ is freeze-safe with $\config{S}{e_{a_1}}
            \parstepsto
            \config{\extSRaw{S}{l}{u_{p_i}(p_1)}}{e_{a_2}}$, since
            $u_{p_i}$ does not alter the status of $p_1$.

            (By Definition~\ref{def:set-of-state-update-operations},
            $u_{p_i}$ can only change the status bit of a location if
            its contents are $\state{d}{\frozentrue}$ and $u_i(d) \neq
            d$, in which case $u_{p_i}$ changes the contents of the
            location to $\state{\top}{\frozenfalse}$; however, that
            cannot be the case here since then $u_{p_i}(p_1)$ would be
            $\topp$, and we would have $S_a = \topS$, a
            contradiction.)
          \end{itemize}
          Therefore, by Lemma~\ref{lem:generalized-independence}
          (Generalized Independence), we have that

          $\config{U_S(S)}{e_{a_1}} \parstepsto
          \config{U_S(\extSRaw{S}{l}{u_{p_i}(p_1)})}{e_{a_2}}$.

          Hence $\config{\extS{S}{l}{d_1}{\frozentrue}}{e_{a_1}}
          \parstepsto
          \config{\extS{S}{l}{d_1}{\frozentrue}}{e_{a_2}}$.

          By {\sc E-Eval-Ctxt}, it follows that

          $\config{\extS{S}{l}{d_1}{\frozentrue}}{\evalctxt{E'_a}{e_{a_1}}}
          \ctxstepsto
          \config{\extS{S}{l}{d_1}{\frozentrue}}{\evalctxt{E'_a}{e_{a_2}}}$,

          as we were required to show.
        \end{itemize}

      \end{itemize}

    \item \label{slqc-put-freeze-simple}Case {\sc E-Freeze-Simple}:
      Similar to case~\ref{slqc-put-freeze-final}, since $S_a =
      \extSRaw{S}{l}{u_{p_i}(p_1)}$ and $S_b =
      \extS{S}{l'}{d_1}{\frozentrue}$.

    \end{enumerate}
  \item Case {\sc E-Put-Err}: We have $\config{S_a}{e_{a_2}} =
    \error$.

    We proceed by case analysis on the rule by which
    $\config{S}{e_{b_1}}$ steps to $\config{S_b}{e_{b_2}}$.

    Since the only way an $\error$ configuration can arise is by the
    {\sc E-Put-Err} rule, we can assume in all other cases that
    $\conf_b \neq \error$.
    \begin{enumerate}
    \item \label{slqc-put-err-beta}Case {\sc E-Beta}: By symmetry with case~\ref{slqc-beta-put-err}.
    \item \label{slqc-put-err-new}Case {\sc E-New}: By symmetry with case~\ref{slqc-new-put-err}.
    \item \label{slqc-put-err-put}Case {\sc E-Put}: By symmetry with case~\ref{slqc-put-put-err}.
    \item \label{slqc-put-err-put-err}Case {\sc E-Put-Err}: We have
      $\config{S_a}{e_{a_2}} = \error$ and $\config{S_b}{e_{b_2}} =
      \error$, and so we choose $\conf_c = \error$, $i = 0$, $j = 0$,
      and $\pi = \id$.

      We have to show that:
      \begin{itemize}
      \item $\config{S_a}{\evalctxt{E'_b}{e_{b_1}}} = \error$, and
      \item $\config{S_b}{\evalctxt{E'_a}{e_{a_1}}} = \error$.
      \end{itemize}

      Since $\config{S_a}{e_{a_2}} = \error$, $S_a = \topS$, and since
      $\config{S_b}{e_{b_2}} = \error$, $S_b = \topS$, so both of the
      above follow immediately.

    \item \label{slqc-put-err-get}Case {\sc E-Get}: Similar to
      case~\ref{slqc-put-err-beta}, since $\config{S_a}{e_{a_2}} =
      \error$ and $S_b = S$.
    \item \label{slqc-put-err-freeze-init}Case {\sc E-Freeze-Init}:
      Similar to case~\ref{slqc-put-err-beta}, since
      $\config{S_a}{e_{a_2}} = \error$ and $S_b = S$.
    \item \label{slqc-put-err-spawn-handler}Case {\sc
      E-Spawn-Handler}: Similar to case~\ref{slqc-put-err-beta}, since
      $\config{S_a}{e_{a_2}} = \error$ and $S_b = S$.
    \item \label{slqc-put-err-freeze-final}Case {\sc E-Freeze-Final}:
      We have $\config{S_a}{e_{a_2}} = \error$ and $S_b =
      \extS{S}{l}{d_1}{\frozentrue}$, and so we choose $\conf_c =
      \error$, $i = 0$, $j = 1$, and $\pi = \id$.

      We have to show that:
      \begin{itemize}
      \item $\config{S_a}{\evalctxt{E'_b}{e_{b_1}}} = \error$,
        and
      \item $\config{S_b}{\evalctxt{E'_a}{e_{a_1}}} \ctxstepsto
        \error$.
      \end{itemize}

      The first of these is immediately true because since
      $\config{S_a}{e_{a_2}} = \error$, $S_a = \topS$, and so
      $\config{S_a}{\evalctxt{E'_b}{e_{b_1}}}$ is equal to $\error$ as
      well.

      For the second, observe that since $\config{S}{e_{b_1}}
      \parstepsto \config{S_b}{e_{b_2}}$, we have by
      Lemma~\ref{lem:monotonicity} (Monotonicity) that
      $\leqstore{S}{S_b}$.

      Therefore, since $\config{S}{e_{a_1}} \parstepsto \error$, we
      have by Lemma~\ref{lem:error-preservation} that
      $\config{S_b}{e_{a_1}} \parstepsto \error$.

      Since $\error$ is equal to $\config{\topS}{e}$ for all
      expressions $e$, $\config{S_b}{e_{a_1}} \parstepsto
      \config{\topS}{e}$ for all $e$.

      Therefore, by {\sc E-Eval-Ctxt},
      $\config{S_b}{\evalctxt{E'_a}{e_{a_1}}} \ctxstepsto
      \config{\topS}{\evalctxt{E'_a}{e}}$ for all $e$.

      Since $\config{\topS}{\evalctxt{E'_a}{e}}$ is equal to $\error$,
      we have that $\config{S_b}{\evalctxt{E'_a}{e_{a_1}}} \ctxstepsto
      \error$, as we were required to show.

    \item \label{slqc-put-err-freeze-simple}Case {\sc
      E-Freeze-Simple}: Similar to
      case~\ref{slqc-put-err-freeze-final}, since $S_b =
      \extS{S}{l}{d_1}{\frozentrue}$.

    \end{enumerate}
  \item Case {\sc E-Get}: We have $S_a = S$.

    We proceed by case analysis on the rule by which
    $\config{S}{e_{b_1}}$ steps to $\config{S_b}{e_{b_2}}$.

    Since the only way an $\error$ configuration can arise is by the
    {\sc E-Put-Err} rule, we can assume in all other cases that
    $\conf_b \neq \error$.
    \begin{enumerate}
    \item \label{slqc-get-beta}Case {\sc E-Beta}: By symmetry with case~\ref{slqc-beta-get}.
    \item \label{slqc-get-new}Case {\sc E-New}: By symmetry with case~\ref{slqc-new-get}.
    \item \label{slqc-get-put}Case {\sc E-Put}: By symmetry with case~\ref{slqc-put-get}.
    \item \label{slqc-get-put-err}Case {\sc E-Put-Err}: By symmetry with case~\ref{slqc-put-err-get}.
    \item \label{slqc-get-get}Case {\sc E-Get}: Similar to
      case~\ref{slqc-get-beta}, since $S_a = S$ and $S_b = S$.
    \item \label{slqc-get-freeze-init}Case {\sc E-Freeze-Init}:
      Similar to case~\ref{slqc-get-beta}, since $S_a = S$ and $S_b = S$.
    \item \label{slqc-get-spawn-handler}Case {\sc E-Spawn-Handler}:
      Similar to case~\ref{slqc-get-beta}, since $S_a = S$ and $S_b = S$.
    \item \label{slqc-get-freeze-final}Case {\sc E-Freeze-Final}:
      Similar to case~\ref{slqc-beta-freeze-final}, since $S_a = S$
      and $S_b = \extS{S}{l}{d_1}{\frozentrue}$.
    \item \label{slqc-get-freeze-simple}Case {\sc E-Freeze-Simple}:
      Similar to case~\ref{slqc-beta-freeze-simple}, since $S_a = S$
      and $S_b = \extS{S}{l}{d_1}{\frozentrue}$.
    \end{enumerate}

  \item Case {\sc E-Freeze-Init}: We have $S_a = S$.

    We proceed by case analysis on the rule by which
    $\config{S}{e_{b_1}}$ steps to $\config{S_b}{e_{b_2}}$.

    Since the only way an $\error$ configuration can arise is by the
    {\sc E-Put-Err} rule, we can assume in all other cases that
    $\conf_b \neq \error$.
    \begin{enumerate}
    \item \label{slqc-freeze-init-beta}Case {\sc E-Beta}: By symmetry with case~\ref{slqc-beta-freeze-init}.
    \item \label{slqc-freeze-init-new}Case {\sc E-New}: By symmetry with case~\ref{slqc-new-freeze-init}.
    \item \label{slqc-freeze-init-put}Case {\sc E-Put}: By symmetry with case~\ref{slqc-put-freeze-init}.
    \item \label{slqc-freeze-init-put-err}Case {\sc E-Put-Err}: By symmetry with case~\ref{slqc-put-err-freeze-init}.
    \item \label{slqc-freeze-init-get}Case {\sc E-Get}: By symmetry with case~\ref{slqc-get-freeze-init}.
    \item \label{slqc-freeze-init-freeze-init}Case {\sc
      E-Freeze-Init}: Similar to case~\ref{slqc-freeze-init-beta},
      since $S_a = S$ and $S_b = S$.
    \item \label{slqc-freeze-init-spawn-handler}Case {\sc
      E-Spawn-Handler}: Similar to case~\ref{slqc-freeze-init-beta},
      since $S_a = S$ and $S_b = S$.
    \item \label{slqc-freeze-init-freeze-final}Case {\sc
      E-Freeze-Final}: Similar to case~\ref{slqc-beta-freeze-final},
      since $S_a = S$ and $S_b = \extS{S}{l}{d_1}{\frozentrue}$.
    \item \label{slqc-freeze-init-freeze-simple}Case {\sc
      E-Freeze-Simple}: Similar to case~\ref{slqc-beta-freeze-simple},
      since $S_a = S$ and $S_b = \extS{S}{l}{d_1}{\frozentrue}$.
    \end{enumerate}

  \item Case {\sc E-Spawn-Handler}: We have $S_a = S$.

    We proceed by case analysis on the rule by which
    $\config{S}{e_{b_1}}$ steps to $\config{S_b}{e_{b_2}}$.

    Since the only way an $\error$ configuration can arise is by the
    {\sc E-Put-Err} rule, we can assume in all other cases that
    $\conf_b \neq \error$.
    \begin{enumerate}
    \item \label{slqc-spawn-handler-beta}Case {\sc E-Beta}: By symmetry with case~\ref{slqc-beta-spawn-handler}.
    \item \label{slqc-spawn-handler-new}Case {\sc E-New}: By symmetry with case~\ref{slqc-new-spawn-handler}.
    \item \label{slqc-spawn-handler-put}Case {\sc E-Put}: By symmetry with case~\ref{slqc-put-spawn-handler}.
    \item \label{slqc-spawn-handler-put-err}Case {\sc E-Put-Err}: By symmetry with case~\ref{slqc-put-err-spawn-handler}.
    \item \label{slqc-spawn-handler-get}Case {\sc E-Get}: By symmetry with case~\ref{slqc-get-spawn-handler}.
    \item \label{slqc-spawn-handler-freeze-init}Case {\sc E-Freeze-Init}: By symmetry with case~\ref{slqc-freeze-init-spawn-handler}.
    \item \label{slqc-spawn-handler-spawn-handler}Case {\sc
      E-Spawn-Handler}: Similar to case~\ref{slqc-spawn-handler-beta},
      since $S_a = S$ and $S_b = S$.
    \item \label{slqc-spawn-handler-freeze-final}Case {\sc
      E-Freeze-Final}: Similar to case~\ref{slqc-beta-freeze-final},
      since $S_a = S$ and $S_b = \extS{S}{l}{d_1}{\frozentrue}$.
    \item \label{slqc-spawn-handler-freeze-simple}Case {\sc
      E-Freeze-Simple}: Similar to case~\ref{slqc-beta-freeze-simple},
      since $S_a = S$ and $S_b = \extS{S}{l}{d_1}{\frozentrue}$.
    \end{enumerate}

  \item Case {\sc E-Freeze-Final}: We have $S_a =
    \extS{S}{l}{d_1}{\frozentrue}$.

    We proceed by case analysis on the rule by which
    $\config{S}{e_{b_1}}$ steps to $\config{S_b}{e_{b_2}}$.

    Since the only way an $\error$ configuration can arise is by the
    {\sc E-Put-Err} rule, we can assume in all other cases that
    $\conf_b \neq \error$.
    \begin{enumerate}
    \item \label{slqc-freeze-final-beta}Case {\sc E-Beta}: By symmetry with case~\ref{slqc-beta-freeze-final}.
    \item \label{slqc-freeze-final-new}Case {\sc E-New}: By symmetry with case~\ref{slqc-new-freeze-final}.
    \item \label{slqc-freeze-final-put}Case {\sc E-Put}: By symmetry with case~\ref{slqc-put-freeze-final}.
    \item \label{slqc-freeze-final-put-err}Case {\sc E-Put-Err}: By symmetry with case~\ref{slqc-put-err-freeze-final}.
    \item \label{slqc-freeze-final-get}Case {\sc E-Get}: By symmetry with case~\ref{slqc-get-freeze-final}.
    \item \label{slqc-freeze-final-freeze-init}Case {\sc E-Freeze-Init}: By symmetry with case~\ref{slqc-freeze-init-freeze-final}.
    \item \label{slqc-freeze-final-spawn-handler}Case {\sc E-Spawn-Handler}: By symmetry with case~\ref{slqc-spawn-handler-freeze-final}.
    \item \label{slqc-freeze-final-freeze-final}Case {\sc
      E-Freeze-Final}: We have $S_a = \extS{S}{l}{d_1}{\frozentrue}$
      and $S_b = \extS{S}{l'}{d'_1}{\frozentrue}$.

      Now consider whether $l = l'$:
      \begin{itemize}
      \item If $l \neq l'$:

        Choose $S' =
        \extS{\extS{S}{l'}{d'_1}{\frozentrue}}{l}{d_1}{\frozentrue}$,
        $i = 1$, $j = 1$, and $\pi = \id$.

        We have to show that:
        \begin{itemize}
        \item
          $\config{\extS{S}{l}{d_1}{\frozentrue}}{\evalctxt{E'_b}{e_{b_1}}}
          \ctxstepsto
          \config{\extS{\extS{S}{l'}{d'_1}{\frozentrue}}{l}{d_1}{\frozentrue}}{\evalctxt{E'_b}{e_{b_2}}}$,
          and
        \item
          $\config{\extS{S}{l'}{d'_1}{\frozentrue}}{\evalctxt{E'_a}{e_{a_1}}}
          \ctxstepsto
          \config{\extS{\extS{S}{l'}{d'_1}{\frozentrue}}{l}{d_1}{\frozentrue}}{\evalctxt{E'_a}{e_{a_2}}}$.
        \end{itemize}

        For the first of these, consider that
        $\extS{S}{l}{d_1}{\frozentrue} = U_S(S)$, where $U_S$ is the
        store update operation that freezes the contents of $l$
        and acts as the identity on the contents of all other
        locations.

        Note that:
        \begin{itemize}
        \item $U_S$ is non-conflicting with $\config{S}{e_{b_1}}
          \parstepsto
          \config{\extS{S}{l'}{d'_1}{\frozentrue}}{e_{b_2}}$, since
          no locations are allocated in the transition;
        \item $U_S(\extS{S}{l'}{d'_1}{\frozentrue}) \neq \topS$,

          since $U_S(\extS{S}{l'}{d'_1}{\frozentrue}) =
          \extS{\extS{S}{l'}{d'_1}{\frozentrue}}{l}{d_1}{\frozentrue}$
          and we know $S \neq \topS$ and freezing the contents of
          locations $l$ and $l'$ in $S$ cannot cause it to become
          $\topS$ (because if so, then we would have $S_a = \topS$
          or $S_b = \topS$, which we know are not the case); and
        \item $U_S$ is freeze-safe with $\config{S}{e_{b_1}}
          \parstepsto
          \config{\extS{S}{l'}{d'_1}{\frozentrue}}{e_{b_2}}$, since
          the only location that can change in status between $S$
          and $\extS{S}{l'}{d'_1}{\frozentrue}$ is $l'$, and $U_S$
          acts as the identity on $l'$.
        \end{itemize}
        Therefore, by Lemma~\ref{lem:generalized-independence}
        (Generalized Independence), we have that

        $\config{U_S(S)}{e_{b_1}} \parstepsto
        \config{U_S(\extS{S}{l'}{d'_1}{\frozentrue})}{e_{b_2}}$.

        Hence $\config{\extS{S}{l}{d_1}{\frozentrue}}{e_{b_1}}
        \parstepsto
        \config{\extS{\extS{S}{l'}{d'_1}{\frozentrue}}{l}{d_1}{\frozentrue}}{e_{b_2}}$.

        By {\sc E-Eval-Ctxt}, it follows that

        $\config{\extS{S}{l}{d_1}{\frozentrue}}{\evalctxt{E'_b}{e_{b_1}}}
        \ctxstepsto
        \config{\extS{\extS{S}{l'}{d'_1}{\frozentrue}}{l}{d_1}{\frozentrue}}{\evalctxt{E'_b}{e_{b_2}}}$,
        
        as we were required to show.

        The argument for the second is symmetrical.

      \item If $l = l'$:

        \lk{This is the case where we freeze the same location twice,
          which is no problem; the second freeze is a no-op.}

        Note that since $l = l'$, $d_1 = d'_1$ as well.

        Choose $S' = \extS{S}{l}{d_1}{\frozentrue}$, $i = 1$, $j =
        1$, and $\pi = \id$.

        We have to show that:
        \begin{itemize}
        \item
          $\config{\extS{S}{l}{d_1}{\frozentrue}}{\evalctxt{E'_b}{e_{b_1}}}
          \ctxstepsto
          \config{\extS{S}{l}{d_1}{\frozentrue}}{\evalctxt{E'_b}{e_{b_2}}}$,
          and
        \item
          $\config{\extS{S}{l'}{d'_1}{\frozentrue}}{\evalctxt{E'_a}{e_{a_1}}}
          \ctxstepsto
          \config{\extS{S}{l}{d_1}{\frozentrue}}{\evalctxt{E'_a}{e_{a_2}}}$.
        \end{itemize}

        For the first of these, consider that
        $\extS{S}{l}{d_1}{\frozentrue} = U_S(S)$, where $U_S$ is the
        store update operation that freezes the contents of $l$ and
        acts as the identity on the contents of all other locations.

        Note that:
        \begin{itemize}
        \item $U_S$ is non-conflicting with $\config{S}{e_{b_1}}
          \parstepsto
          \config{\extS{S}{l}{d_1}{\frozentrue}}{e_{b_2}}$, since no
          locations are allocated in the transition;
        \item $U_S(\extS{S}{l}{d_1}{\frozentrue}) \neq \topS$, since
          $U_S(\extS{S}{l}{d_1}{\frozentrue}) =
          \extS{S}{l}{d_1}{\frozentrue}$, and we know $S \neq \topS$
          and freezing the contents of location $l$ in $S$ cannot
          cause it to become $\topS$; and
        \item $U_S$ is freeze-safe with $\config{S}{e_{b_1}}
          \parstepsto
          \config{\extS{S}{l}{d_1}{\frozentrue}}{e_{b_2}}$, since
          the only location that can change in status between $S$
          and $\extS{S}{l}{d_1}{\frozentrue}$ is $l$, and $U_S$
          freezes the contents of $l$ but has no other effect on
          them.
        \end{itemize}

        Therefore, by Lemma~\ref{lem:generalized-independence}
        (Generalized Independence), we have that

        $\config{U_S(S)}{e_{b_1}} \parstepsto
        \config{U_S(\extS{S}{l}{d_1}{\frozentrue})}{e_{b_2}}$.

        Hence $\config{\extS{S}{l}{d_1}{\frozentrue}}{e_{b_1}}
        \parstepsto
        \config{\extS{S}{l}{d_1}{\frozentrue}}{e_{b_2}}$.

        By {\sc E-Eval-Ctxt}, it follows that

        $\config{\extS{S}{l}{d_1}{\frozentrue}}{\evalctxt{E'_b}{e_{b_1}}}
        \ctxstepsto
        \config{\extS{S}{l}{d_1}{\frozentrue}}{\evalctxt{E'_b}{e_{b_2}}}$,

        as we were required to show.

        The argument for the second is symmetrical.

      \end{itemize}

    \item \label{slqc-freeze-final-freeze-simple}Case {\sc
      E-Freeze-Simple}: Similar to
      case~\ref{slqc-freeze-final-freeze-final}, since $S_a =
      \extS{S}{l}{d_1}{\frozentrue}$ and $S_b =
      \extS{S}{l'}{d'_1}{\frozentrue}$.
    \end{enumerate}

  \item Case {\sc E-Freeze-Simple}: We have $S_a =
    \extS{S}{l}{d_1}{\frozentrue}$.

    \begin{enumerate}
    \item \label{slqc-freeze-simple-beta}Case {\sc E-Beta}: By symmetry with case~\ref{slqc-beta-freeze-simple}.
    \item \label{slqc-freeze-simple-new}Case {\sc E-New}: By symmetry with case~\ref{slqc-new-freeze-simple}.
    \item \label{slqc-freeze-simple-put}Case {\sc E-Put}: By symmetry with case~\ref{slqc-put-freeze-simple}.
    \item \label{slqc-freeze-simple-put-err}Case {\sc E-Put-Err}: By symmetry with case~\ref{slqc-put-err-freeze-simple}.
    \item \label{slqc-freeze-simple-get}Case {\sc E-Get}: By symmetry with case~\ref{slqc-get-freeze-simple}.
    \item \label{slqc-freeze-simple-freeze-init}Case {\sc E-Freeze-Init}: By symmetry with case~\ref{slqc-freeze-init-freeze-simple}.
    \item \label{slqc-freeze-simple-spawn-handler}Case {\sc E-Spawn-Handler}: By symmetry with case~\ref{slqc-spawn-handler-freeze-simple}.
    \item \label{slqc-freeze-simple-freeze-final}Case {\sc E-Freeze-Final}: By symmetry with case~\ref{slqc-freeze-final-freeze-simple}.
    \item \label{slqc-freeze-simple-freeze-simple}Case {\sc
      E-Freeze-Simple}: Similar to
      case~\ref{slqc-freeze-simple-freeze-final}, since $S_a =
      \extS{S}{l}{d_1}{\frozentrue}$ and $S_b =
      \extS{S}{l'}{d'_1}{\frozentrue}$.
    \end{enumerate}

  \end{enumerate}
\end{proof}



\section{Proof of Lemma~\ref{lem:strong-one-sided-quasi-confluence}}\label{section:strong-one-sided-quasi-confluence-proof}
\begin{proof}
  Suppose $\conf \ctxstepsto \conf'$ and $\conf \ctxstepsto^m
  \conf''$, where $1 \leq m$.

  We are required to show that either:
  \begin{enumerate}
  \item there exist $\conf_c, i, j, \pi$ such that $\conf'
    \ctxstepsto^i \conf_c$ and $\pi(\conf'') \ctxstepsto^j \conf_c$
    and $i \leq m$ and $j \leq 1$, or
  \item there exists $k \leq m$ such that $\conf' \ctxstepsto^k
    \textup{\error}$, or there exists $k \leq 1$ such that $\conf''
    \ctxstepsto^k \textup{\error}$.
  \end{enumerate}

  We proceed by induction on $m$.

  In the base case of $m = 1$, the result is immediate from
  Lemma~\ref{lem:strong-local-quasi-confluence}, with $k = 1$.

  For the induction step, suppose $\conf \ctxstepsto^m \conf''
  \ctxstepsto \conf'''$ and suppose the lemma holds for $m$.

  We show that it holds for $m + 1$, as follows.

  From the induction hypothesis, we have that either:
  \begin{enumerate}
  \item there exist $\conf_c', i', j', \pi'$ such that $\conf'
    \ctxstepsto^{i'} \conf_c'$ and $\pi'(\conf'') \ctxstepsto^{j'}
    \conf_c'$ and $i' \leq m$ and $j' \leq 1$, or
  \item there exists $k' \leq m$ such that $\conf'
    \ctxstepsto^{k'} \error$, or there exists $k' \leq 1$ such that
    $\conf'' \ctxstepsto^{k'} \error$.
  \end{enumerate}

  We consider these two cases in turn:
  \begin{enumerate}
  \item There exist $\conf_c', i', j', \pi'$ such that $\conf'
    \ctxstepsto^{i'} \conf_c'$ and $\pi'(\conf'') \ctxstepsto^{j'}
    \conf_c'$ and $i' \leq m$ and $j' \leq 1$:

    We proceed by cases on $j'$:
    \begin{itemize}

    \item If $j' = 0$, then $\pi'(\conf'') = \conf_c'$.

      Since $\conf'' \ctxstepsto \conf'''$, we have that
      $\pi'(\conf'') \ctxstepsto \pi'(\conf''')$ by
      Lemma~\ref{lem:permutability} (Permutability).

      We can then choose $\conf_c = \pi'(\conf''')$ and $i = i' + 1$
      and $j = 0$ and $\pi = \pi'$.

      The key is that $\conf' \ctxstepsto^{i'} \conf'_c =
      \pi'(\conf'') \ctxstepsto \pi'(\conf''')$ for a total of $i' +
      1$ steps.
      
    \item If $j' = 1$:

      First, since $\pi'(\conf'') \ctxstepsto^{j'} \conf'_c$, then
      by Lemma~\ref{lem:permutability} (Permutability) we have that
      $\conf'' \ctxstepsto^{j'} \piprimeinv(\conf'_c)$.
      
      Then, by $\conf'' \ctxstepsto^{j'} \piprimeinv(\conf'_c)$ and
      $\conf'' \ctxstepsto \conf'''$ and
      Lemma~\ref{lem:strong-local-quasi-confluence}, one of the
      following two cases is true:
      \begin{enumerate}
      \item There exist $\conf_c''$ and $i''$ and $j''$ and $\pi''$
        such that $\piprimeinv(\conf'_c) \ctxstepsto^{i''}
        \conf_c''$ and $\pi''(\conf''') \ctxstepsto^{j''} \conf_c''$
        and $i'' \leq 1$ and $j'' \leq 1$.

        Since $\piprimeinv(\conf'_c) \ctxstepsto^{i''} \conf_c''$,
        by Lemma~\ref{lem:permutability} (Permutability) we have
        that $\conf'_c \ctxstepsto^{i''} \pi'(\conf_c'')$.

        So we also have $\conf' \ctxstepsto^{i'} \conf_c'
        \ctxstepsto^{i''} \pi'(\conf_c'')$.

        Since $\pi''(\conf''') \ctxstepsto^{j''} \conf_c''$, by
        Lemma~\ref{lem:permutability} (Permutability) we have that
        $\pi'(\pi''(\conf''')) \ctxstepsto^{j''} \pi'(\conf_c'')$.

        In summary, we pick $\conf_c = \pi'(\conf_c'')$ and $i = i' + i''$
        and $j = j''$ and $\pi = \pi'' \circ \pi'$, which is sufficient
        because $i = i' + i'' \leq m + 1$ and $j = j'' \leq 1$.

      \item $\piprimeinv(\conf'_c) \ctxstepsto \error$ or $\conf'''
        \ctxstepsto \error$.

        If $\conf''' \ctxstepsto \error$, then choosing $k = 1$
        satisfies the proof.

        Otherwise, $\piprimeinv(\conf'_c) \ctxstepsto \error$.

        Then, by Lemma~\ref{lem:permutability} we have that
        $\conf'_c \ctxstepsto \pi'(\error)$.

        By Definition~\ref{def:permutation-configuration},
        $\pi'(\error) = \error$, and so $\conf'_c \ctxstepsto
        \error$.

        Therefore $\conf' \ctxstepsto^{i'} \conf'_c \ctxstepsto
        \error$.

        Hence $\conf' \ctxstepsto^{i'+1} \error$.

        Since $i' \leq m$, we have that $i' + 1 \leq m + 1$, and
        so choosing $k = i' + 1$ satisfies the proof.
        
      \end{enumerate}

    \end{itemize}

  \item There exists $k' \leq m$ such that $\conf' \ctxstepsto^{k'}
    \error$, or there exists $k' \leq 1$ such that $\conf''
    \ctxstepsto^{k'} \error$:

    If there exists $k' \leq m$ such that $\conf' \ctxstepsto^{k'}
    \error$, then choosing $k = k'$ satisfies the proof.

    Otherwise, there exists $k' \leq 1$ such that $\conf''
    \ctxstepsto^{k'} \error$.

    We proceed by cases on $k'$:

    \begin{itemize}

    \item If $k' = 0$, then $\conf'' = \error$.

      Hence this case is not possible, since $\conf'' \ctxstepsto
      \conf'''$ and $\error$ cannot step.

    \item If $k' = 1$:

      From $\conf'' \ctxstepsto \conf'''$ and $\conf''
      \ctxstepsto^{k'} \error$ and
      Lemma~\ref{lem:strong-local-quasi-confluence}, one of the
      following two cases is true:

      \begin{enumerate}
      \item There exist $\conf_c''$ and $i''$ and $j''$ and $\pi''$
        such that $\error \ctxstepsto^{i''} \conf_c''$ and
        $\pi''(\conf''') \ctxstepsto^{j''} \conf_c''$ and $i'' \leq
        1$ and $j'' \leq 1$.

        Since $\error$ cannot step, $i'' = 0$ and $\conf''_c =
        \error$.

        By Definition~\ref{def:permutation-configuration},
        $\pi''(\conf''') = \conf'''$.

        Hence $\conf''' \ctxstepsto^{j''} \error$.

        \lk{This is the one place that we need to allow $k$ to be
          $\leq$ 1 instead of exactly 1.}

        Since $j'' \leq 1$, choosing $k = j''$ satisfies the proof.

      \item $\error \ctxstepsto \error$ or $\conf''' \ctxstepsto
        \error$.

        Since $\error$ cannot step, $\conf''' \ctxstepsto \error$.

        Hence choosing $k = 1$ satisfies the proof.

      \end{enumerate}

    \end{itemize}

  \end{enumerate}

\end{proof}


\section{Proof of Lemma~\ref{lem:strong-quasi-confluence}}\label{section:strong-quasi-confluence-proof}
\begin{proof}
  We proceed by induction on $n$.  In the base case of $n = 1$, the
  result is immediate from Lemma~\ref{lem:strong-one-sided-quasi-confluence}.

  For the induction step, suppose $\conf \parstepsto^n \conf'
  \parstepsto \conf'''$ and suppose the lemma holds for $n$.

  We show that it holds for $n + 1$, as follows.

  We are required to show that either:
  \begin{enumerate}
  \item there exist $\conf_c, i, j$ such that $\conf''' \parstepsto^i
    \conf_c$ and $\conf'' \parstepsto^j \conf_c$ and $i \leq m$ and $j
    \leq n + 1$, or
  \item there exists $k \leq m$ such that $\conf''' \parstepsto^k
    \error$, or there exists $k \leq n + 1$ such that $\conf''
    \parstepsto^k \error$.
  \end{enumerate}

  From the induction hypothesis, we have that either:
  \begin{enumerate}
  \item there exist $\conf'_c, i', j'$ such that $\conf'
    \parstepsto^{i'} \conf'_c$ and $\conf'' \parstepsto^{j'} \conf'_c$
    and $i' \leq m$ and $j' \leq n$, or
  \item there exists $k' \leq m$ such that $\conf' \parstepsto^{k'}
    \error$, or there exists $k' \leq n$ such that $\conf''
    \parstepsto^{k'} \error$.
  \end{enumerate}

  We consider these two cases in turn:

  \begin{enumerate}
  \item There exist $\conf'_c, i', j'$ such that $\conf'
    \parstepsto^{i'} \conf'_c$ and $\conf'' \parstepsto^{j'} \conf'_c$
    and $i' \leq m$ and $j' \leq n$:

    We proceed by cases on $i'$:
    \begin{itemize}

    \item If $i' = 0$, then $\conf' = \conf_c'$.  We can then choose
      $\conf_c = \conf'''$ and $i = 0$ and $j = j' + 1$.

    \item If $i' \geq 1$:

      From $\conf' \parstepsto \conf'''$ and $\conf' \parstepsto^{i'}
      \conf_c'$ and Lemma~\ref{lem:strong-one-sided-quasi-confluence},
      one of the following two cases is true:
      \begin{enumerate}
        \item There exist $\conf_c''$ and $i''$ and $j''$ such that
          $\conf''' \parstepsto^{i''} \conf_c''$ and $\conf_c'
          \parstepsto^{j''} \conf_c''$ and $i'' \leq i'$ and $j'' \leq
          1$.  So we also have $\conf'' \parstepsto^{j'} \conf_c'
          \parstepsto^{j''} \conf_c''$.  In summary, we pick $\conf_c
          = \conf_c''$ and $i = i''$ and $j = j' + j''$, which is
          sufficient because $i = i'' \leq i' \leq m$ and $j = j' +
          j'' \leq n + 1$.
        \item There exists $k'' \leq i'$ such that $\conf'''
          \parstepsto^{k''} \error$, or there exists $k'' \leq 1$ such
          that $\conf'_c \parstepsto^{k''} \error$.

          If there exists $k'' \leq i'$ such that $\conf'''
          \parstepsto^{k''} \error$, then choosing $k = k''$ satisfies
          the proof, since $k'' \leq i' \leq m$.

          Otherwise, there exists $k'' \leq 1$ such
          that $\conf'_c \parstepsto^{k''} \error$.

          Therefore, $\conf'' \parstepsto^{j'} \conf_c'
          \parstepsto^{k''} \error$.

          Hence $\conf'' \parstepsto^{j' + k''} \error$.

          Since $j' \leq n$ and $k'' \leq 1$, $j' + k'' \leq n + 1$.

          Hence choosing $k = j' + k''$ satisfies the proof.

      \end{enumerate}
    \end{itemize}

  \item There exists $k' \leq m$ such that $\conf' \parstepsto^{k'}
    \error$, or there exists $k' \leq n$ such that $\conf''
    \parstepsto^{k'} \error$:

    If there exists $k' \leq n$ such that $\conf'' \parstepsto^{k'}
    \error$, then choosing $k = k'$ satisfies the proof.

    Otherwise, there exists $k' \leq m$ such that $\conf'
    \parstepsto^{k'} \error$.  We proceed by cases on $k'$:

    \begin{itemize}

    \item If $k' = 0$, then $\conf' = \error$.

      Hence this case is not possible, since $\conf' \parstepsto
      \conf'''$ and $\error$ cannot step.

    \item If $k' \geq 1$:

      From $\conf' \parstepsto \conf'''$ and $\conf' \parstepsto^{k'}
      \error$ and Lemma~\ref{lem:strong-one-sided-quasi-confluence},
      one of the following two cases is true:

      \begin{enumerate}
        \item There exist $\conf''_c$ and $i''$ and $j''$ such that
          $\conf''' \parstepsto^{i''} \conf''_c$ and $\error
          \parstepsto^{j''} \conf''_c$ and $i'' \leq k'$ and $j'' \leq
          1$.

          Since $\error$ cannot step, $j'' = 0$ and $\conf''_c =
          \error$.

          Hence $\conf''' \parstepsto^{i''} \error$.

          Since $i'' \leq k' \leq m$, choosing $k = i''$ satisfies the
          proof.

        \item There exists $k'' \leq k'$ such that $\conf'''
          \parstepsto^{k''} \error$, or there exists $k'' \leq 1$ such
          that $\error \parstepsto^{k''} \error$.

          Since $\error$ cannot step, there exists $k'' \leq k'$ such
          that $\conf''' \parstepsto^{k''} \error$.

          Since $k'' \leq k' \leq m$, choosing $k = k''$ satisfies the
          proof.
      \end{enumerate}
    \end{itemize}
  \end{enumerate}

\end{proof}


\section{Proof of Theorem~\ref{thm:determinism-of-threshold-queries}}\label{section:determinism-of-threshold-queries-proof}
\begin{proof}
  Consider replica $i$ of a threshold CvRDT $(S, \leq, s^0, q, t, u,
  m)$.

  Let $\mathcal{S}$ be a threshold set with respect to
  $(S, \leq)$.

  Consider a method execution $t^{k+1}_i(\mathcal{S})$ (\ie, a
  threshold query that is the $k+1$th method execution on replica $i$,
  with threshold set $\mathcal{S}$ as its argument) that returns some
  set of activation states $S_a \in \mathcal{S}$.

  For part~\ref{thm:this-replica} of the theorem, we have to show that
  threshold queries with $\mathcal{S}$ as their argument will always
  return $S_a$ on subsequent executions at $i$.

  That is, we have to show that, for all $k' > (k+1)$, the threshold
  query $t^{k'}_i(\mathcal{S})$ on $i$ returns $S_a$.

  Since $t^{k+1}_i(\mathcal{S})$ returns $S_a$, from
  Definition~\ref{def:cvrdt-with-threshold-queries} we have that for
  some activation state $s_a \in S_a$, the condition $s_a \leq s^k_i$
  holds.

  Consider arbitrary $k' > (k+1)$.

  Since state is inflationary across updates, we know that the state
  $s^{k'}_i$ after method execution $k'$ is at least $s^k_i$.

  That is, $s^k_i \leq s^{k'}_i$.

  By transitivity of $\leq$, then, $s_a \leq s^{k'}_i$.

  Hence, by Definition~\ref{def:cvrdt-with-threshold-queries},
  $t^{k'}_i(\mathcal{S})$ returns $S_a$.

  For part~\ref{thm:any-replica} of the theorem, consider some replica
  $j$ of $(S, \leq, s^0, q, t, u, m)$, located at process $p_j$.

  We are required to show that, for all $x \geq 0$, the threshold
  query $t^{x+1}_j(\mathcal{S})$ returns $S_a$ eventually, and blocks
  until it does.\footnote{The occurrences of $k+1$ and $x+1$ in this
    proof are an artifact of how we index method executions starting
    from $1$, but states starting from $0$.  The initial state (of
    every replica) is $s^0$, and so $s^k_i$ is the state of replica
    $i$ after method execution $k$ has completed at $i$.}

  That is, we must show that, for all $x \geq 0$, there exists some
  finite $n \geq 0$ such that
  \begin{itemize}
  \item 
    for all $i$ in the range $0 \leq i \leq n-1$, the threshold query
    $t^{x+1+i}_j(\mathcal{S})$ returns $\block$, and
  \item
    for all $i \geq n$, the threshold query $t^{x+1+i}_j(\mathcal{S})$
    returns $S_a$.
  \end{itemize}
  Consider arbitrary $x \geq 0$.

  Recall that $s^x_j$ is the state of replica $j$ after the $x$th
  method execution, and therefore $s^x_j$ is also the state of $j$
  when $t^{x+1}_j(\mathcal{S})$ runs.
  %
  We have three cases to consider:
  \begin{itemize}
  \item $s^k_i \leq s^x_j$.

    (That is, replica $i$'s state after the $k$th method execution on $i$
    is \emph{at or below} replica $j$'s state after the $x$th method
    execution on $j$.)

    Choose $n = 0$.

    We have to show that, for all $i \geq n$, the threshold query
    $t^{x+1+i}_j(\mathcal{S})$ returns $S_a$.

    Since $t^{k+1}_i(\mathcal{S})$ returns $S_a$, we know that there
    exists an $s_a \in S_a$ such that $s_a \leq s^k_i$.

    Since $s^k_i \leq s^x_j$, we have by transitivity of $\leq$ that
    $s_a \leq s^x_j$.

    Therefore, by Definition~\ref{def:cvrdt-with-threshold-queries},
    $t^{x+1}_j(\mathcal{S})$ returns $S_a$.

    Then, by part~\ref{thm:this-replica} of the theorem, we have that
    subsequent executions $t^{x+1+i}_j(\mathcal{S})$ at replica $j$
    will also return $S_a$, and so the case holds.

    (Note that this case includes the possibility $s^k_i \equiv s^0$,
    in which no updates have executed at replica $i$.)

  \item $s^k_i > s^x_j$.

    (That is, replica $i$'s state after the $k$th method execution on $i$
    is \emph{above} replica $j$'s state after the $x$th method execution
    on $j$.)

    We have two subcases:

    \begin{itemize}
    \item
      There exists some activation state $s'_a \in S_a$ for which $s'_a \leq
      s^x_j$.

      In this case, we choose $n = 0$.

      We have to show that, for all $i \geq n$, the threshold query
      $t^{x+1+i}_j(\mathcal{S})$ returns $S_a$.

      Since $s'_a \leq s^x_j$, by
      Definition~\ref{def:cvrdt-with-threshold-queries},
      $t^{x+1}_j(\mathcal{S})$ returns $S_a$.

      Then, by part~\ref{thm:this-replica} of the theorem, we have
      that subsequent executions $t^{x+1+i}_j(\mathcal{S})$ at replica
      $j$ will also return $S_a$, and so the case holds.

    \item
      There is no activation state $s'_a \in S_a$ for which $s'_a \leq
      s^x_j$.

      Since $t^{k+1}_i(\mathcal{S})$ returns $S_a$, we know that there
      is some update $u^{k'}_i(a)$ in $i$'s causal history, for some
      $k' < (k+1)$, that updates $i$ from a state at or below $s^x_j$
      to $s^k_i$.\footnote{We know that $i$'s state was once at or
        below $s^x_j$, because $i$ and $j$ started at the same state
        $s^0$ and can both only grow.  Hence the least that $s^x_j$
        can be is $s^0$, and we know that $i$ was originally $s^0$ as
        well.}

      By eventual delivery, $u^{k'}_i(a)$ is eventually delivered at
      $j$.

      Hence some update or updates that will increase $j$'s state from
      $s^x_j$ to a state at or above some $s'_a$ must reach replica
      $j$.\footnote{We say ``some update or updates'' because the
        exact update $u^{k'}_i(a)$ may not be the update that causes
        the threshold query at $j$ to unblock; a different update or
        updates could do it.  Nevertheless, the existence of
        $u^{k'}_i(a)$ means that there is at least one update that
        will suffice to unblock the threshold query.}

      Let the $x+1+r$th method execution on $j$ be the first update on $j$
      that updates its state to some $s^{x+1+r}_j \geq s'_a$, for some
      activation state $s'_a \in S_a$.

      Choose $n = r+1$.

      We have to show that, for all $i$ in the range $0 \leq i \leq
      r$, the threshold query $t^{x+1+i}_j(\mathcal{S})$ returns
      $\block$, and that for all $i \geq r+1$, the threshold query
      $t^{x+1+i}_j(\mathcal{S})$ returns $S_a$.

      For the former, since the $x+1+r$th method execution on $j$ is the
      first one that updates its state to $s^{x+1+r}_j \geq s'_a$, we have
      by Definition~\ref{def:cvrdt-with-threshold-queries} that for all $i$
      in the range $0 \leq i \leq r$, the threshold query
      $t^{x+1+i}_j(\mathcal{S})$ returns $\block$.

      For the latter, since $s^{x+1+r}_j \geq s'_a$, by
      Definition~\ref{def:cvrdt-with-threshold-queries} we have that
      $t^{x+1+r+1}_j(\mathcal{S})$ returns $S_a$, and by
      part~\ref{thm:this-replica} of the theorem, we have that for $i \geq
      r+1$, subsequent executions $t^{x+1+i}_j(\mathcal{S})$ at replica $j$
      will also return $S_a$, and so the case holds.
    \end{itemize}

  \item $s^k_i \nleq s^x_j$ and $s^x_j \nleq s^k_i$.

    (That is, replica $i$'s state after the $k$th method execution on $i$
    is \emph{not comparable} to replica $j$'s state after the $x$th method
    execution on $j$.)

    Similar to the previous case.
  \end{itemize}
\end{proof}



\chapter{Proofs}\label{app:proofs}

\section{Proof of Lemma~\ref{lem:lvars-permutability}}\label{section:lvars-permutability-proof}
\begin{proof}
  Consider an arbitrary permutation $\pi$.  For
  part~\ref{thm:permutable-reduction-transitions}, we have to show
  that if $\conf \parstepsto \conf'$ then $\pi(\conf) \parstepsto
  \pi(\conf')$, and that if $\pi(\conf) \parstepsto \pi(\conf')$ then
  $\conf \parstepsto \conf'$.

  For the forward direction of
  part~\ref{thm:permutable-reduction-transitions}, suppose $\conf
  \parstepsto \conf'$.  We have to show that $\pi(\conf) \parstepsto
  \pi(\conf')$.  We proceed by cases on the rule by which $\conf$
  steps to $\conf'$.

  \begin{itemize}
    \item Case {\sc E-Beta}: $\conf =
      \config{S}{\app{(\lam{x}{e})}{v}}$, and $\conf' =
      \config{S}{\subst{e}{x}{v}}$.

      To show: $\pi(\config{S}{\app{(\lam{x}{e})}{v}}) \parstepsto
      \pi(\config{S}{\subst{e}{x}{v}})$.

      By Definitions~\ref{def:lvars-permutation-configuration}
      and~\ref{def:lvars-permutation-expression}, $\pi(\conf) =
      \config{\pi(S)}{\app{(\lam{x}{\pi(e)})}{\pi(v)}}$.

      By {\sc E-Beta},
      $\config{\pi(S)}{\app{(\lam{x}{\pi(e)})}{\pi(v)}}$ steps to
      $\config{\pi(S)}{\subst{\pi(e)}{x}{\pi(v)}}$.

      By Definition~\ref{def:lvars-permutation-expression},
      $\config{\pi(S)}{\subst{\pi(e)}{x}{\pi(v)}}$ is equal to
      $\config{\pi(S)}{\pi(\subst{e}{x}{v})}$.

      Hence $\config{\pi(S)}{\app{(\lam{x}{\pi(e)})}{\pi(v)}}$ steps
      to $\config{\pi(S)}{\pi(\subst{e}{x}{v})}$,

      which is equal to $\pi(\config{S}{\subst{e}{x}{v}})$ by
      Definition~\ref{def:lvars-permutation-configuration}.  Hence the
      case is satisfied.

    \item Case {\sc E-New}: $\conf = \config{S}{\NEW}$, and $\conf' =
      \config{\extSRaw{S}{l}{\bot}}{l}$.

      To show: $\pi(\config{S}{\NEW}) \parstepsto
      \pi(\config{\extSRaw{S}{l}{\bot}}{l})$.

      By Definitions~\ref{def:lvars-permutation-configuration}
      and~\ref{def:lvars-permutation-expression}, $\pi(\conf) =
      \config{\pi(S)}{\NEW}$.

      By {\sc E-New}, $\config{\pi(S)}{\NEW}$ steps to
      $\config{\extSRaw{(\pi(S))}{l'}{\bot}}{l'}$, where $l' \notin
      \dom{\pi(S)}$.
      
      It remains to show that
      $\config{\extSRaw{(\pi(S))}{l'}{\bot}}{l'}$ is equal to
      $\pi(\config{\extSRaw{S}{l}{\bot}}{l})$.

      By Definition~\ref{def:lvars-permutation-configuration},
      $\pi(\config{\extSRaw{S}{l}{\bot}}{l})$ is equal to
      $\config{\pi(\extSRaw{S}{l}{\bot})}{\pi(l)}$,

      which is equal to
      $\config{\extSRaw{(\pi(S))}{\pi(l)}{\bot}}{\pi(l)}$.

      So, we have to show that
      $\config{\extSRaw{(\pi(S))}{l'}{\bot}}{l'}$ is equal to
      $\config{\extSRaw{(\pi(S))}{\pi(l)}{\bot}}{\pi(l)}$.  Since we
      know (from the side condition of {\sc E-New}) that $l \notin
      \dom{S}$, it follows that $\pi(l) \notin \pi(\dom{S})$.
      Therefore, in $\config{\extSRaw{(\pi(S))}{l'}{\bot}}{l'}$, we
      can $\alpha$-rename $l'$ to $\pi(l)$, and so the two
      configurations are equal and the case is satisfied.

    \item Case {\sc E-Put}: $\conf = \config{S}{\putexp{l}{d_2}}$, and
      $\conf' = \config{\extSRaw{S}{l}{\userlub{d_1}{d_2}}}{\unit}$.

      To show: $\pi(\config{S}{\putexp{l}{d_2}}) \parstepsto
      \pi(\config{\extSRaw{S}{l}{\userlub{d_1}{d_2}}}{\unit})$.

      By Definitions~\ref{def:lvars-permutation-configuration}
      and~\ref{def:lvars-permutation-expression}, $\pi(\conf) =
      \config{\pi(S)}{\putexp{\pi(l)}{d_2}}$.

      By {\sc E-Put}, $\config{\pi(S)}{\putexp{\pi(l)}{d_2}}$ steps to
      $\config{\extSRaw{(\pi(S))}{\pi(l)}{\userlub{d_1}{d_2}}}{\unit}$,

      since $S(l) = (\pi(S))(\pi(l)) = d_1$.

      It remains to show that
      $\config{\extSRaw{(\pi(S))}{\pi(l)}{\userlub{d_1}{d_2}}}{\unit}$
      is equal to
      $\pi(\config{\extSRaw{S}{l}{\userlub{d_1}{d_2}}}{\unit})$.

      By Definitions~\ref{def:lvars-permutation-configuration}
      and~\ref{def:lvars-permutation-expression},
      $\pi(\config{\extSRaw{S}{l}{\userlub{d_1}{d_2}}}{\unit})$ is
      equal to
      $\config{\extSRaw{(\pi(S))}{\pi(l)}{\userlub{d_1}{d_2}}}{\unit}$,
      and so the two configurations are equal and the case is
      satisfied.

    \item Case {\sc E-Put-Err}: $\conf = \config{S}{\putexp{l}{d_2}}$,
      and $\conf' = \error$.

      To show: $\pi(\config{S}{\putexp{l}{d_2}}) \parstepsto
      \pi(\error)$.

      By Definitions~\ref{def:lvars-permutation-configuration}
      and~\ref{def:lvars-permutation-expression}, $\pi(\conf) =
      \config{\pi(S)}{\putexp{\pi(l)}{d_2}}$.

      By {\sc E-Put-Err}, $\config{\pi(S)}{\putexp{\pi(l)}{d_2}}$
      steps to $\error$,

      since $S(l) = (\pi(S))(\pi(l)) = d_1$.

      Since $\pi(\error) = \error$ by
      Definition~\ref{def:lvars-permutation-configuration}, the case
      is complete.

    \item Case {\sc E-Get}: $\conf = \config{S}{\getexp{l}{T}}$, and
      $\conf' = \config{S}{d_2}$.

      To show: $\pi(\config{S}{\getexp{l}{T}}) \parstepsto
      \pi(\config{S}{d_2})$.

      By Definitions~\ref{def:lvars-permutation-configuration}
      and~\ref{def:lvars-permutation-expression}, $\pi(\conf) =
      \config{\pi(S)}{\getexp{\pi(l)}{T}}$.

      By {\sc E-Get}, $\config{\pi(S)}{\getexp{\pi(l)}{T}}$ steps to
      $\config{\pi(S)}{d_2}$,

      since $S(l) = (\pi(S))(\pi(l)) = d_1$.

      By Definitions~\ref{def:lvars-permutation-configuration}
      and~\ref{def:lvars-permutation-expression},
      $\pi(\config{S}{d_2}) \config{\pi(S)}{d_2}$.  Therefore the case
      is complete.
  \end{itemize}

  For the reverse direction of
  part~\ref{thm:permutable-reduction-transitions}, suppose $\pi(\conf)
  \parstepsto \pi(\conf')$.  We have to show that $\conf \parstepsto
  \conf'$.

  We know from the forward direction of the proof that for all
  configurations $\conf$ and $\conf'$ and permutations $\pi$, if
  $\conf \parstepsto \conf'$ then $\pi(\conf) \parstepsto
  \pi(\conf')$.  Hence since $\pi(\conf) \parstepsto \pi(\conf')$, and
  since $\piinv$ is also a permutation, we have that
  $\piinv(\pi(\conf)) \parstepsto \piinv(\pi(\conf'))$.  Since
  $\piinv(\pi(l)) = l$ for every $l \in \Loc$, and that property lifts
  to configurations as well, we have that $\conf \parstepsto \conf'$.

  \lk{Is the above enough of a proof?}

  For the forward direction of
  part~\ref{thm:permutable-context-transitions}, suppose $\conf
  \ctxstepsto \conf'$.  We have to show that $\pi(\conf) \ctxstepsto
  \pi(\conf')$.

  By inspection of the operational semantics, $\conf$ must be of the
  form $\config{S}{\E{e}}$, and $\conf'$ must be of the form
  $\config{S'}{\E{e'}}$.  Hence we have to show that
  $\pi(\config{S}{\E{e}}) \ctxstepsto \pi(\config{S'}{\E{e'}})$.

  By Definition~\ref{def:lvars-permutation-configuration},
  $\pi(\config{S}{\E{e}})$ is equal to $\config{\pi(S)}{\pi(\E{e})}$.

  Also by Definition~\ref{def:lvars-permutation-configuration},
  $\pi(\config{S'}{\E{e'}})$ is equal to
  $\config{\pi(S')}{\pi(\E{e'})}$.

  Furthermore, $\config{\pi(S)}{\pi(\E{e})}$ is equal to
  $\config{\pi(S)}{\evalctxt{(\pi(E))}{\pi(e)}}$ and
  $\config{\pi(S')}{\pi(\E{e'})}$ is equal to
  $\config{\pi(S')}{\evalctxt{(\pi(E))}{\pi(e')}}$.

  So we have to show that
  $\config{\pi(S)}{\evalctxt{(\pi(E))}{\pi(e)}} \ctxstepsto
  \config{\pi(S')}{\evalctxt{(\pi(E))}{\pi(e')}}$.

  From the premise of {\sc E-Eval-Ctxt}, $\config{S}{e} \parstepsto
  \config{S'}{e'}$.  Hence, by
  part~\ref{thm:permutable-reduction-transitions}, $\pi(\config{S}{e})
  \parstepsto \pi(\config{S'}{e'})$.  By
  Definition~\ref{def:lvars-permutation-configuration},
  $\pi(\config{S}{e})$ is equal to $\config{\pi(S)}{\pi(e)}$ and
  $\pi(\config{S'}{e'})$ is equal to $\config{\pi(S')}{\pi(e')}$.

  Hence $\config{\pi(S)}{\pi(e)} \parstepsto
  \config{\pi(S')}{\pi(e')}$.

  Therefore, by {\sc E-Eval-Ctxt}, $\config{\pi(S)}{\E{\pi(e)}}
  \ctxstepsto \config{\pi(S')}{\E{\pi(e')}}$ for all evaluation
  contexts $E$.

  In particular, it is true that
  $\config{\pi(S)}{\evalctxt{(\pi(E))}{\pi(e)}} \ctxstepsto
  \config{\pi(S')}{\evalctxt{(\pi(E))}{\pi(e')}}$, as we were required
  to show.

  For the reverse direction of
  part~\ref{thm:permutable-context-transitions}, suppose $\pi(\conf)
  \ctxstepsto \pi(\conf')$.  We have to show that $\conf \ctxstepsto
  \conf'$.

  We know from the forward direction of the proof that for all
  configurations $\conf$ and $\conf'$ and permutations $\pi$, if
  $\conf \ctxstepsto \conf'$ then $\pi(\conf) \ctxstepsto
  \pi(\conf')$.  Hence since $\pi(\conf) \ctxstepsto \pi(\conf')$, and
  since $\piinv$ is also a permutation, we have that
  $\piinv(\pi(\conf)) \ctxstepsto \piinv(\pi(\conf'))$.  Since
  $\piinv(\pi(l)) = l$ for every $l \in \Loc$, and that property lifts
  to configurations as well, we have that $\conf \ctxstepsto \conf'$.

  \lk{Is the above enough of a proof?}
\end{proof}


\section{Proof of Lemma~\ref{lem:lvars-internal-determinism}}\label{section:lvars-internal-determinism-proof}
\begin{proof}
  Suppose $\conf \parstepsto \conf'$ and $\conf \parstepsto \conf''$.

  We have to show that there is a permutation $\pi$ such that $\conf'
  = \pi(\conf'')$.

  The proof is by cases on the rule by which $\conf$ steps to
  $\conf'$.

  \begin{itemize}

  \item Case {\sc E-Beta}:

    Given: $\config{S}{\app{(\lam{x}{e})}{v}} \parstepsto
    \config{S}{\subst{e}{x}{v}}$, and
    $\config{S}{\app{(\lam{x}{e})}{v}} \parstepsto \conf''$.

    To show: There exists a $\pi$ such that
    $\config{S}{\subst{e}{x}{v}} = \pi(\conf'')$.

    By inspection of the operational semantics, the only reduction
    rule by which $\config{S}{\app{(\lam{x}{e})}{v}}$ can step is {\sc
      E-Beta}.

    Hence $\conf'' = \config{S}{\subst{e}{x}{v}}$, and the case is
    satisfied by choosing $\pi$ to be the identity function.

  \item Case {\sc E-New}: 

    Given: $\config{S}{\NEW} \parstepsto
    \config{\extSRaw{S}{l}{\bot}}{l}$, and $\config{S}{\NEW}
    \parstepsto \conf''$.

    To show: There exists a $\pi$ such that
    $\config{\extSRaw{S}{l}{\bot}}{l} = \pi(\conf'')$.

    By inspection of the operational semantics, the only reduction
    rule by which $\config{S}{\NEW}$ can step is {\sc E-New}.

    Hence $\conf'' = \config{\extSRaw{S}{l'}{\bot}}{l'}$.

    Since, by the side condition of {\sc E-New}, neither $l$ nor $l'$
    occur in $\dom{S}$, the case is satisfied by choosing $\pi$ to be
    the permutation that maps $l'$ to $l$ and is the identity on every
    other element of $\Loc$.

  \item Case {\sc E-Put}:

    Given: $\config{S}{\putexp{l}{d_2}} \parstepsto
    \config{\extSRaw{S}{l}{\userlub{d_1}{d_2}}}{\unit}$, and
    $\config{S}{\putexp{l}{d_2}} \parstepsto \conf''$.

    To show: There exists a $\pi$ such that
    $\config{\extSRaw{S}{l}{\userlub{d_1}{d_2}}}{\unit} =
    \pi(\conf'')$.

    By inspection of the operational semantics, and since
    $\userlub{d_1}{d_2} \neq \top$ (from the premise of {\sc E-Put}),
    the only reduction rule by which $\config{S}{\putexp{l}{d_2}}$ can
    step is {\sc E-Put}.

    Hence $\conf'' =
    \config{\extSRaw{S}{l}{\userlub{d_1}{d_2}}}{\unit}$, and the case
    is satisfied by choosing $\pi$ to be the identity function.

  \item Case {\sc E-Put-Err}:

    Given: $\config{S}{\putexp{l}{d_2}} \parstepsto \error$, and
    $\config{S}{\putexp{l}{d_2}} \parstepsto \conf''$.

    To show: There exists a $\pi$ such that $\error = \pi(\conf'')$.

    By inspection of the operational semantics, and since
    $\userlub{d_1}{d_2} = \top$ (from the premise of {\sc E-Put-Err}),
    the only reduction rule by which $\config{S}{\putexp{l}{d_2}}$ can
    step is {\sc E-Put-Err}.

    Hence $\conf'' = \error$, and the case is satisfied by choosing
    $\pi$ to be the identity function.

  \item Case {\sc E-Get}:

    Given: $\config{S}{\getexp{l}{T}} \parstepsto \config{S}{d_2}$,
    and $\config{S}{\getexp{l}{T}} \parstepsto \conf''$.

    To show: There exists a $\pi$ such that $\config{S}{d_2} =
    \pi(\conf'')$.

    By inspection of the operational semantics, the only reduction
    rule by which $\config{S}{\getexp{l}{T}}$ can step is {\sc
      E-Get}.

    Hence $\conf'' = \config{S}{d_2}$, and the case is satisfied by
    choosing $\pi$ to be the identity function.

  \end{itemize}
\end{proof}



\section{Proof of Lemma~\ref{lem:lvars-monotonicity}}\label{section:lvars-monotonicity-proof}
\begin{proof}
  Suppose $\config{S}{e} \parstepsto \config{S'}{e'}$.  We are
  required to show that $\leqstore{S}{S'}$.  The proof is by cases on
  the rule by which $\config{S}{e}$ steps to $\config{S'}{e'}$.

  \begin{itemize}
    \item Case {\sc E-Beta}:

      Immediate by the definition of $\leqstore{}{}$, since $S$ does
      not change.

    \item Case {\sc E-New}:

      Given: $\config{S}{\NEW} \parstepsto
      \config{\extSRaw{S}{l}{\bot}}{l}$.

      To show: $\leqstore{S}{\extSRaw{S}{l}{\bot}}$.

      By Definition~\ref{def:lvars-leqstore}, we have to show that
      $\dom{S} \subseteq \dom{\extSRaw{S}{l}{\bot}}$ and
      that for all $l' \in \dom{S}$, $S(l') \userleq
      (\extSRaw{S}{l}{\bot})(l')$.

      By definition, a store update operation on $S$ can only either
      update an existing binding in $S$ or extend $S$ with a new
      binding.  Hence $\dom{S} \subseteq \dom{\extSRaw{S}{l}{\bot}}$.

      From the side condition of {\sc E-New}, $l \notin \dom{S}$.
      Hence $\extSRaw{S}{l}{\bot}$ adds a new binding for $l$ in $S$.

      Hence $\extSRaw{S}{l}{\bot}$ does not update any existing
      bindings in $S$.

      Hence, for all $l' \in \dom{S}, S(l') \userleq
      (\extSRaw{S}{l}{\bot})(l')$.

      Therefore $\leqstore{S}{\extSRaw{S}{l}{\bot}}$, as
      required.

    \item Case {\sc E-Put}:

      Given: $\config{S}{\putexp{l}{d_2}} \parstepsto
      \config{\extSRaw{S}{l}{\userlub{d_1}{d_2}}}{\unit}$.

      To show: $\leqstore{S}{\extSRaw{S}{l}{\userlub{d_1}{d_2}}}$.

      By Definition~\ref{def:lvars-leqstore}, we have to show that
      $\dom{S} \subseteq \dom{\extSRaw{S}{l}{\userlub{d_1}{d_2}}}$ and
      that for all $l' \in \dom{S}$, $S(l') \userleq
      (\extSRaw{S}{l}{\userlub{d_1}{d_2}})(l')$.

      By definition, a store update operation on $S$ can only either
      update an existing binding in $S$ or extend $S$ with a new
      binding.  Hence $\dom{S} \subseteq
      \dom{\extSRaw{S}{l}{\userlub{d_1}{d_2}}}$.

      From the premises of {\sc E-Put}, $S(l) = d_1$.  Therefore $l
      \in \dom{S}$.

      Hence $\extSRaw{S}{l}{\userlub{d_1}{d_2}}$ updates the existing
      binding for $l$ in $S$ from $d_1$ to $\userlub{d_1}{d_2}$.

      By the definition of $\userlub{}{}$, $d_1 \userleq
      (\userlub{d_1}{d_2})$.  $\extSRaw{S}{l}{\userlub{d_1}{d_2}}$
      does not update any other bindings in $S$, hence, for all $l'
      \in \dom{S}, S(l') \userleq
      (\extSRaw{S}{l}{\userlub{d_1}{d_2}})(l')$.

      Hence $\leqstore{S}{\extSRaw{S}{l}{\userlub{d_1}{d_2}}}$, as
      required.

    \item Case {\sc E-Put-Err}:

      Given: $\config{S}{\putexp{l}{d_2}} \parstepsto \error$.

      By the definition of $\error$, $\error$ is equal to
      $\config{\topS}{e}$ for all $e$.

      To show: $\leqstore{S}{\topS}$.

      Immediate by the definition of $\leqstore{}{}$.

    \item Case {\sc E-Get}:

      Immediate by the definition of $\leqstore{}{}$, since $S$ does
      not change.

  \end{itemize}

\end{proof}


\section{Proof of Lemma~\ref{lem:lvars-independence}}\label{section:lvars-independence-proof}
\begin{proof}
  Consider arbitrary $S''$ such that $S''$ is non-conflicting with
  $\config{S}{e} \parstepsto \config{S'}{e'}$ and $\lubstore{S'}{S''}
  \neq \topS$.

  To show: $\config{\lubstore{S}{S''}}{e} \parstepsto
  \config{\lubstore{S'}{S''}}{e'}$.

  The proof is by induction on the derivation of $\config{S}{e}
  \parstepsto \config{S'}{e'}$, by cases on the last rule in the
  derivation.  In every case we may assume that $\config{S'}{e'} \neq
  \error$.  Since $\config{S'}{e'} \neq \error$, we do not need to
  consider the {\sc E-Put-Err} rule.
  \begin{itemize}

    \item Case {\sc E-Eval-Ctxt}:

      Given: $\config{S}{\E{e}} \parstepsto \config{S'}{\E{e'}}$.

      To show: $\config{\lubstore{S}{S''}}{\E{e}} \parstepsto
      \config{\lubstore{S'}{S''}}{\E{e'}}$.

      From the premise of {\sc E-Eval-Ctxt}, we have that
      $\config{S}{e} \parstepsto \config{S'}{e'}$.

      Therefore, by IH, we have that $\config{\lubstore{S}{S''}}{e}
      \parstepsto \config{\lubstore{S'}{S''}}{e'}$.

      Therefore, by {\sc E-Eval-Ctxt}, we have that
      $\config{\lubstore{S}{S''}}{\E{e}} \parstepsto
      \config{\lubstore{S'}{S''}}{\E{e'}}$, as we were required to
      show.

    \item Case {\sc E-Beta}:

      Given: $\config{S}{\app{(\lam{x}{e})}{v}} \parstepsto
      \config{S}{\subst{e}{x}{v}}$.

      To show: $\config{\lubstore{S}{S''}}{\app{(\lam{x}{e})}{v}}
      \parstepsto \config{\lubstore{S}{S''}}{\subst{e}{x}{v}}$.

      Immediate by {\sc E-Beta}.

    \item Case {\sc E-New}:

      Given: $\config{S}{\NEW} \parstepsto
      \config{\extSRaw{S}{l}{\bot}}{l}$.

      To show: $\config{\lubstore{S}{S''}}{\NEW} \parstepsto
      \config{\lubstore{(\extSRaw{S}{l}{\bot})}{S''}}{l}$.

      By {\sc E-New}, we have that $\config{\lubstore{S}{S''}}{\NEW}
      \parstepsto \config{\extSRaw{(\lubstore{S}{S''})}{l'}{\bot}}{l'}$,
      where $l' \notin \dom{\lubstore{S}{S''}}$.

      By assumption, $S''$ is non-conflicting with $\config{S}{\NEW}
      \parstepsto \config{\extSRaw{S}{l}{\bot}}{l}$.
 
      Therefore $l \notin \dom{S''}$.

      From the side condition of {\sc E-New}, $l \notin \dom{S}$.

      Therefore $l \notin \dom{\lubstore{S}{S''}}$.

      Therefore, in
      $\config{\extSRaw{(\lubstore{S}{S''})}{l'}{\bot}}{l'}$, we can
      $\alpha$-rename $l'$ to $l$, \\ resulting in
      $\config{\extSRaw{(\lubstore{S}{S''})}{l}{\bot}}{l}$.

      Therefore $\config{\lubstore{S}{S''}}{\NEW} \parstepsto
      \config{\extSRaw{(\lubstore{S}{S''})}{l}{\bot}}{l}$.

      Note that:
      \begin{align*}
        \extSRaw{(\lubstore{S}{S''})}{l}{\bot} &=
        \lubstore{\extSRaw{S}{l}{\bot}}{\extSRaw{S''}{l}{\bot}} \\ &=
        \lubstore{\lubstore{S}{\store{\storebindingRaw{l}{\bot}}}}{\lubstore{S''}{\store{\storebindingRaw{l}{\bot}}}}
        \\ &=
        \lubstore{\lubstore{S}{\store{\storebindingRaw{l}{\bot}}}}{S''}
        \\ &= \lubstore{\extSRaw{S}{l}{\bot}}{S''}.
      \end{align*}
      Therefore $\config{\lubstore{S}{S''}}{\NEW} \parstepsto
      \config{\lubstore{\extSRaw{S}{l}{\bot}}{S''}}{l}$, as we were
      required to show.

    \item Case {\sc E-Put}:

      Given: $\config{S}{\putexp{l}{d_2}} \parstepsto
      \config{\extSRaw{S}{l}{d_2}}{\unit}$.

      To show: $\config{\lubstore{S}{S''}}{\putexp{l}{d_2}}
      \parstepsto
      \config{\lubstore{\extSRaw{S}{l}{d_2}}{S''}}{\unit}$.

      We will first show that

      $\config{\lubstore{S}{S''}}{\putexp{l}{d_2}} \parstepsto
      \config{\extSRaw{(\lubstore{S}{S''})}{l}{d_2}}{\unit}$

      and then show why this is sufficient.

      We proceed by cases on $l$:

      \begin{itemize}
        \item $l \notin \dom{S''}$:

          By assumption, $\lubstore{\extSRaw{S}{l}{d_2}}{S''} \neq
          \topS$.

          By Lemma~\ref{lem:lvars-monotonicity},
          $\leqstore{S}{\extSRaw{S}{l}{d_2}}$.

          Hence $\lubstore{S}{S''} \neq \topS$.

          Therefore, by Definition~\ref{def:lvars-lubstore},
          $(\lubstore{S}{S''})(l) = S(l)$.

          From the premises of {\sc E-Put}, $S(l) = d_1$.

          Hence $(\lubstore{S}{S''})(l) = d_1$.

          From the premises of {\sc E-Put}, $d_2 = \userlub{d_1}{d_2}$
          and $d_2 \neq \top$.

          Therefore, by {\sc E-Put}, we have:
          $\config{\lubstore{S}{S''}}{\putexp{l}{d_2}} \parstepsto
          \config{\extSRaw{(\lubstore{S}{S''})}{l}{d_2}}{\unit}$.

        \item $l \in \dom{S''}$:

          By assumption, $\lubstore{\extSRaw{S}{l}{d_2}}{S''} \neq
          \topS$.

          By Lemma~\ref{lem:lvars-monotonicity},
          $\leqstore{S}{\extSRaw{S}{l}{d_2}}$.

          Hence $\lubstore{S}{S''} \neq \topS$.

          Therefore $(\lubstore{S}{S''})(l) = \userlub{S(l)}{S''(l)}$.

          From the premises of {\sc E-Put}, $S(l) = d_1$.
          
          Hence $(\lubstore{S}{S''})(l) = d'_1$, where $d_1 \userleq
          d'_1$.

          From the premises of {\sc E-Put}, $d_2 =
          \userlub{d_1}{d_2}$.

          Let $d'_2 = \userlub{d'_1}{d_2}$.

          Hence $d_2 \userleq d'_2$.

          By assumption, $\lubstore{\extSRaw{S}{l}{d_2}}{S''} \neq
          \topS$.

          Therefore, by Definition~\ref{def:lvars-lubstore},
          $\lubstore{d_2}{S''(l)} \neq \top$.

          Note that:
          \begin{align*}
            \top &\neq \lubstore{d_2}{S''(l)} \\ &=
            \userlub{\userlub{d_1}{d_2}}{S''(l)} \\ &=
            \userlub{\userlub{S(l)}{d_2}}{S''(l)} \\ &=
            \userlub{\userlub{S(l)}{S''(l)}}{d_2} \\ &=
            \userlub{(\lubstore{S}{S''})(l)}{d_2} \\ &=
            \userlub{d'_1}{d_2} \\ &= d'_2. \\
          \end{align*}
          Hence $d'_2 \neq \top$.

          Hence $(\lubstore{S}{S''})(l) = d'_1$ and $d'_2 =
          \userlub{d'_1}{d_2}$ and $d'_2 \neq \top$.

          Therefore, by {\sc E-Put} we have:
          $\config{\lubstore{S}{S''}}{\putexp{l}{d_2}} \parstepsto
          \config{\extSRaw{(\lubstore{S}{S''})}{l}{d'_2}}{\unit}$.

          \lk{If we really wanted to be pedantic here, we'd actually
            prove that the stores are equal.  I'm assuming that if I
            can show that $\extSRaw{(\lubstore{S}{S''})}{l}{d'_2}$ and
            $\extSRaw{(\lubstore{S}{S''})}{l}{d_2}$ bind $l$ to the
            same value, then it will be obvious that they're equal.}

          Note that:
          \begin{align*}
            (\extSRaw{(\lubstore{S}{S''})}{l}{d'_2})(l) &=
            \userlub{(\lubstore{S}{S''})(l)}{(\store{\storebindingRaw{l}{d'_2}})(l)}
            \\ &= \userlub{d'_1}{d'_2} \\ &=
            \userlub{d'_1}{\userlub{d'_1}{d_2}} \\ &=
            \userlub{d'_1}{d_2}
          \end{align*}
          and
          \begin{align*}
            (\extSRaw{(\lubstore{S}{S''})}{l}{d_2})(l) &=
            \userlub{(\lubstore{S}{S''})(l)}{(\store{\storebindingRaw{l}{d_2}})(l)}
            \\ &= \userlub{d'_1}{d_2} \\ &=
            \userlub{d'_1}{\userlub{d_1}{d_2}} \\ &=
            \userlub{d'_1}{d_2} & \textrm{(since $d_1 \userleq
              d'_1$).}
          \end{align*}
          Therefore $\extSRaw{(\lubstore{S}{S''})}{l}{d'_2} =
          \extSRaw{(\lubstore{S}{S''})}{l}{d_2}$.

          Therefore, $\config{\lubstore{S}{S''}}{\putexp{l}{d_2}}
          \parstepsto
          \config{\extSRaw{(\lubstore{S}{S''})}{l}{d_2}}{\unit}$.
      \end{itemize}

      Note that:
      \begin{align*}
        \extSRaw{(\lubstore{S}{S''})}{l}{d_2} &=
        \lubstore{\extSRaw{S}{l}{d_2}}{\extSRaw{S''}{l}{d_2}} \\ &=
        \lubstore{\lubstore{S}{\store{\storebindingRaw{l}{d_2}}}}{\lubstore{S''}{\store{\storebindingRaw{l}{d_2}}}}
        \\ &=
        \lubstore{\lubstore{S}{\store{\storebindingRaw{l}{d_2}}}}{S''}
        \\ &= \lubstore{\extSRaw{S}{l}{d_2}}{S''}.
      \end{align*}
      Therefore $\config{\lubstore{S}{S''}}{\putexp{l}{d_2}}
      \parstepsto
      \config{\lubstore{\extSRaw{S}{l}{d_2}}{S''}}{\unit}$, as we were
      required to show.

    \item Case {\sc E-Get}:

      Given: $\config{S}{\getexp{l}{T}} \parstepsto \config{S}{d_2}$.

      To show: $\config{\lubstore{S}{S''}}{\getexp{l}{T}} \parstepsto
      \config{\lubstore{S}{S''}}{d_2}$.

      From the premises of {\sc E-Get}, $S(l) = d_1$ and $\incomp{T}$
      and $d_2 \in T$ and $d_2 \userleq d_1$.

      By assumption, $\lubstore{S}{S''} \neq \topS$.

      Hence $(\lubstore{S}{S''}) = d'_1$, where $d_1 \userleq d'_1$.

      By the transitivity of $\userleq$, $d_2 \userleq d'_1$.

      Hence, $S(l) = d'_1$ and $\incomp{T}$ and $d_2 \in T$ and $d_2
      \userleq d'_1$.

      Therefore, by {\sc E-Get},

      $\config{\lubstore{S}{S''}}{\getexp{l}{T}} \parstepsto
      \config{\lubstore{S}{S''}}{d_2}$,

      as we were required to show.
  \end{itemize}
\end{proof}


\section{Proof of Lemma~\ref{lem:lvars-clash}}\label{section:lvars-clash-proof}
\begin{proof}
  Consider arbitrary $S''$ such that $S''$ is non-conflicting with
  $\config{S}{e} \parstepsto \config{S'}{e'}$ and $\lubstore{S'}{S''}
  = \topS$.

  To show: $\config{\lubstore{S}{S''}}{e} \parstepsto \error$.

  The proof is by induction on the derivation of $\config{S}{e}
  \parstepsto \config{S'}{e'}$, by cases on the last rule in the
  derivation.  In every case we may assume that $\config{S'}{e'} \neq
  \error$.  Since $\config{S'}{e'} \neq \error$, we do not need to
  consider the {\sc E-Put-Err} rule.

  \begin{itemize}

    \item Case {\sc E-Eval-Ctxt}:

      Given: $\config{S}{\E{e}} \parstepsto \config{S'}{\E{e'}}$.

      To show: $\config{\lubstore{S}{S''}}{\E{e}} \parstepsto^i
      \error$, where $i \leq 1$.

      From the premise of {\sc E-Eval-Ctxt}, we have that
      $\config{S}{e} \parstepsto \config{S'}{e'}$.

      Therefore, by IH, we have that $\config{\lubstore{S}{S''}}{e}
      \parstepsto^{i'} \error$, where $i' \leq 1$.

      We proceed by cases on $i'$:

      \begin{itemize}
        \item $i' = 0$:

          In this case, $\config{\lubstore{S}{S''}}{e} = \error$.

          Hence, by the definition of $\error$, $\lubstore{S}{S''} =
          \topS$.

          Hence $\config{\lubstore{S}{S''}}{\E{e}} = \error$.

          Hence $\config{\lubstore{S}{S''}}{\E{e}} \parstepsto^i
          \error$, with $i = 0$.

        \item $i' = 1$:

          In this case, $\config{\lubstore{S}{S''}}{e} \parstepsto
          \error$.

          By the definition of $\error$, $\error =
          \config{\topS}{e''}$ for any $e''$.

          Hence $\config{\lubstore{S}{S''}}{e} \parstepsto
          \config{\topS}{e''}$.

          Hence, by {\sc E-Eval-Ctxt},
          $\config{\lubstore{S}{S''}}{\E{e}} \parstepsto
          \config{\topS}{\E{e''}}$.

          By the definition of $\error$, $\config{\topS}{\E{e''}} =
          \error$.

          Hence $\config{\lubstore{S}{S''}}{\E{e}} \parstepsto
          \error$.

          Hence $\config{\lubstore{S}{S''}}{\E{e}} \parstepsto^i
          \error$, with $i = 1$.

      \end{itemize}

    \item Case {\sc E-Beta}:

      Given: $\config{S}{\app{(\lam{x}{e})}{v}} \parstepsto
      \config{S}{\subst{e}{x}{v}}$.

      To show: $\config{\lubstore{S}{S''}}{\app{(\lam{x}{e})}{v}}
      \parstepsto^i \error$, where $i \leq 1$.

      By assumption, $\lubstore{S}{S''} = \topS$.

      Hence, by the definition of $\error$,
      $\config{\lubstore{S}{S''}}{\app{(\lam{x}{e})}{v}} = \error$.

      Hence $\config{\lubstore{S}{S''}}{\app{(\lam{x}{e})}{v}}
      \parstepsto^i \error$, with $i = 0$.

    \item Case {\sc E-New}:

      Given: $\config{S}{\NEW} \parstepsto
      \config{\extSRaw{S}{l}{\bot}}{l}$.

      To show: $\config{\lubstore{S}{S''}}{\NEW} \parstepsto^i
      \error$, where $i \leq 1$.

      By {\sc E-New}, $\config{\lubstore{S}{S''}}{\NEW} \parstepsto
      \config{\extSRaw{(\lubstore{S}{S''})}{l'}{\bot}}{l'}$, where $l'
      \notin \dom{\lubstore{S}{S''}}$.

      By assumption, $S''$ is non-conflicting with $\config{S}{\NEW}
      \parstepsto \config{\extSRaw{S}{l}{\bot}}{l}$.
 
      Therefore $l \notin \dom{S''}$.

      From the side condition of {\sc E-New}, $l \notin \dom{S}$.

      Therefore $l \notin \dom{\lubstore{S}{S''}}$.

      Therefore, in
      $\config{\extSRaw{(\lubstore{S}{S''})}{l'}{\bot}}{l'}$, we can
      $\alpha$-rename $l'$ to $l$, \\ resulting in
      $\config{\extSRaw{(\lubstore{S}{S''})}{l}{\bot}}{l}$.

      Therefore $\config{\lubstore{S}{S''}}{\NEW} \parstepsto
      \config{\extSRaw{(\lubstore{S}{S''})}{l}{\bot}}{l}$.

      By assumption, $\lubstore{\extSRaw{S}{l}{\bot}}{S''}
      = \topS$.

      Note that:
      \begin{align*}
        \topS &= \lubstore{\extSRaw{S}{l}{\bot}}{S''} \\ &=
        \lubstore{\lubstore{S}{\store{\storebindingRaw{l}{\bot}}}}{S''}
        \\ &=
        \lubstore{\lubstore{S}{S''}}{\store{\storebindingRaw{l}{\bot}}}
        \\ &=
        \lubstore{(\lubstore{S}{S''})}{\store{\storebindingRaw{l}{\bot}}}
        \\ &= \extSRaw{(\lubstore{S}{S''})}{l}{\bot} .
      \end{align*}

      Hence $\config{\lubstore{S}{S''}}{\NEW} \parstepsto
      \config{\topS}{l}$.

      Hence, by the definition of $\error$,
      $\config{\lubstore{S}{S''}}{\NEW} \parstepsto \error$.

      Hence $\config{\lubstore{S}{S''}}{\NEW} \parstepsto^i \error$,
      with $i = 1$.

    \item Case {\sc E-Put}:

      Given: $\config{S}{\putexp{l}{d_2}} \parstepsto
      \config{\extSRaw{S}{l}{d_2}}{\unit}$.

      To show: $\config{\lubstore{S}{S''}}{\putexp{l}{d_2}}
      \parstepsto^i \error$, where $i \leq 1$.

      We proceed by cases on $\lubstore{S}{S''}$:

      \begin{itemize}

        \item $\lubstore{S}{S''} = \topS$:

          In this case, by the definition of $\error$,
          $\config{\lubstore{S}{S''}}{\putexp{l}{d_2}} = \error$.

          Hence $\config{\lubstore{S}{S''}}{\putexp{l}{d_2}}
          \parstepsto^i \error$, with $i = 0$.

        \item $\lubstore{S}{S''} \neq \topS$:

          From the premises of {\sc E-Put}, we have that $S(l) = d_1$.

          Hence $(\lubstore{S}{S''})(l) = d'_1$, where $d_1 \userleq
          d'_1$.

          We show that $\userlub{d'_1}{d_2} =
          \top$, as follows:

          By assumption, $\lubstore{\extSRaw{S}{l}{d_2}}{S''} = \topS$.

          Hence, by Definition~\ref{def:lvars-lubstore}, there exists
          some $l' \in \dom{\extSRaw{S}{l}{d_2}} \cap \dom{S''}$ such
          that $\userlub{(\extSRaw{S}{l}{d_2})(l')}{S''(l')} = \top$.

          Now case on $l'$:

          \begin{itemize}
            \item $l' \neq l$:

              In this case, $(\extSRaw{S}{l}{d_2})(l') = S(l')$.

              Since $\userlub{(\extSRaw{S}{l}{d_2})(l')}{S''(l')} = \top$,
              we then have that $\userlub{S(l')}{S''(l')} = \top$.

              However, this is a contradiction since
              $\lubstore{S}{S''} \neq \topS$.

              Hence this case cannot occur.

            \item $l' = l$:

              Then $\userlub{(\extSRaw{S}{l}{d_2})(l)}{S''(l)} = \top$.

              Note that:
              \begin{align*}
                \top &= \userlub{(\extSRaw{S}{l}{d_2})(l)}{S''(l)} \\ &=
                \userlub{d_2}{S''(l)} \\ &=
                \userlub{\userlub{d_1}{d_2}}{S''(l)}
                \\ &=
                \userlub{\userlub{S(l)}{d_2}}{S''(l)}
                \\ &=
                \userlub{\userlub{S(l)}{S''(l)}}{d_2}
                \\ &=
                \userlub{(\lubstore{S}{S''})(l)}{d_2}
                \\ &= \userlub{d'_1}{d_2}.
              \end{align*}
              Hence $\userlub{d'_1}{d_2} = \top$.

              Hence, by {\sc E-Put-Err},
              $\config{\lubstore{S}{S''}}{\putexp{l}{d_2}} \parstepsto
              \error$.

              Hence $\config{\lubstore{S}{S''}}{\putexp{l}{d_2}}
              \parstepsto^i \error$, with $i = 1$.

          \end{itemize}

      \end{itemize}

    \item Case {\sc E-Get}:

      Given: $\config{S}{\getexp{l}{T}} \parstepsto \config{S}{d_2}$.

      To show: $\config{\lubstore{S}{S''}}{\getexp{l}{T}}
      \parstepsto^i \error$, where $i \leq 1$.

      By assumption, $\lubstore{S}{S''} = \topS$.

      Hence, by the definition of $\error$,
      $\config{\lubstore{S}{S''}}{\getexp{l}{T}} = \error$.

      Hence $\config{\lubstore{S}{S''}}{\getexp{l}{T}} \parstepsto^i
      \error$, with $i = 0$.
  \end{itemize}
\end{proof}


\section{Proof of Lemma~\ref{lem:lvars-error-preservation}}\label{section:lvars-error-preservation-proof}
\begin{proof}

  Given: $\config{S}{e} \parstepsto \error$ and $\leqstore{S}{S'}$.

  To show: $\config{S'}{e} \parstepsto \error$.

  \TODO{Figure out what to do here.  I think we need to handle both
    E-Eval-Ctxt and E-Put-Err.}
\end{proof}


\section{Proof of Lemma~\ref{lem:lvars-strong-local-confluence}}\label{section:lvars-strong-local-confluence-proof}
\begin{proof}
  Suppose $\conf \ctxstepsto \conf_a$ and $\conf \ctxstepsto \conf_b$.
  We have to show that there exist $\conf_c, i, j, \pi$ such that
  $\conf_a \ctxstepsto^i \conf_c$ and $\pi(\conf_b) \ctxstepsto^j
  \pi(\conf_c)$ and $i \leq 1$ and $j \leq 1$.

  By inspection of the operational semantics, it must be the case that
  $\conf$ steps to $\conf_a$ by the {\sc E-Eval-Ctxt} rule.  Let
  $\conf = \config{S}{\evalctxt{E_a}{e_{a_1}}}$ and let $\conf_a =
  \config{S_a}{\evalctxt{E_a}{e_{a_2}}}$.

  Likewise, it must be the case that $\conf$ steps to $\conf_b$ by the
  {\sc E-Eval-Ctxt} rule.  Let $\conf =
  \config{S}{\evalctxt{E_b}{e_{b_1}}}$ and let $\conf_b =
  \config{S_b}{\evalctxt{E_b}{e_{b_2}}}$.

  Note that $\conf = \config{S}{\evalctxt{E_a}{e_{a_1}}} =
  \config{S}{\evalctxt{E_b}{e_{b_1}}}$, and so
  $\evalctxt{E_a}{e_{a_1}} = \evalctxt{E_b}{e_{b_1}}$, but $E_a$ and
  $E_b$ may differ and $e_{a_1}$ and $e_{b_1}$ may differ.

  Since $\config{S}{\evalctxt{E_a}{e_{a_1}}} \ctxstepsto
  \config{S_a}{\evalctxt{E_a}{e_{a_2}}}$ and
  $\config{S}{\evalctxt{E_b}{e_{b_1}}} \ctxstepsto
  \config{S_b}{\evalctxt{E_b}{e_{b_2}}}$ and $\evalctxt{E_a}{e_{a_1}}
  = \evalctxt{E_b}{e_{b_1}}$, we have from
  Lemma~\ref{lem:lvars-locality} (Locality) that there exist
  evaluation contexts $E'_a$ and $E'_b$ such that:

  \begin{itemize}
  \item $\evalctxt{E'_a}{e_{a_1}} = \evalctxt{E_b}{e_{b_2}}$, and
  \item $\evalctxt{E'_b}{e_{b_1}} = \evalctxt{E_a}{e_{a_2}}$, and
  \item $\evalctxt{E'_a}{e_{a_2}} =
  \evalctxt{E'_b}{e_{b_2}}$.
  \end{itemize}

  Our approach will be to show that there exist $S', i, j, \pi$ such
  that:
  \begin{itemize}
  \item $\config{S_a}{\evalctxt{E_a}{e_{a_2}}} \ctxstepsto^i
    \config{S'}{\evalctxt{E'_a}{e_{a_2}}}$, and
  \item $\pi(\config{S_b}{\evalctxt{E_b}{e_{b_2}}}) \ctxstepsto^j
    \pi(\config{S'}{\evalctxt{E'_a}{e_{a_2}}})$.
  \end{itemize}
  Since $\evalctxt{E'_a}{e_{a_1}} = \evalctxt{E_b}{e_{b_2}}$,
  $\evalctxt{E'_b}{e_{b_1}} = \evalctxt{E_a}{e_{a_2}}$, and
  $\evalctxt{E'_a}{e_{a_2}} = \evalctxt{E'_b}{e_{b_2}}$, it suffices
  to show that:
  \begin{itemize}
  \item $\config{S_a}{\evalctxt{E'_b}{e_{b_1}}} \ctxstepsto^i
    \config{S'}{\evalctxt{E'_b}{e_{b_2}}}$, and
  \item $\pi(\config{S_b}{\evalctxt{E'_a}{e_{a_1}}}) \ctxstepsto^j
    \pi(\config{S'}{\evalctxt{E'_a}{e_{a_2}}})$.
  \end{itemize}

  From the premise of {\sc E-Eval-Ctxt}, we have that
  $\config{S}{e_{a_1}} \parstepsto \config{S_a}{e_{a_2}}$ and
  $\config{S}{e_{b_1}} \parstepsto \config{S_b}{e_{b_2}}$.  We proceed
  by case analysis on the rule by which $\config{S}{e_{a_1}}$ steps to
  $\config{S_a}{e_{a_2}}$.

  \begin{enumerate}
  \item Case {\sc E-Beta}:

    We have:
    \begin{itemize}
      \item $e_{a_1} = \app{\lam{x}{e'_a}}{v_a}$,
      \item $e_{a_2} = \subst{e'_a}{x}{v_a}$, and
      \item $S_a = S$.
    \end{itemize}

    Now, we proceed by case analysis on the rule by which
    $\config{S}{e_{b_1}}$ steps to $\config{S_b}{e_{b_2}}$:
    \begin{enumerate}
    \item Case {\sc E-Beta}:

      We have:
      \begin{itemize}
      \item $e_{b_1} = \app{\lam{x}{e'_b}}{v_b}$,
      \item $e_{b_2} = \subst{e'_b}{x}{v_b}$, and
      \item $S_b = S$.
      \end{itemize}

      Choose $S' = S$, $i = 1$, $j = 1$, and $\pi = \id$.

      We have to show that:

      \begin{itemize}
      \item $\config{S}{\evalctxt{E'_b}{e_{b_1}}} \ctxstepsto
        \config{S}{\evalctxt{E'_b}{e_{b_2}}}$, and
      \item $\config{S}{\evalctxt{E'_a}{e_{a_1}}} \ctxstepsto
        \config{S}{\evalctxt{E'_a}{e_{a_2}}}$, 
      \end{itemize}

      both of which follow immediately from $\config{S}{e_{a_1}}
      \parstepsto \config{S_a}{e_{a_2}}$ and $\config{S}{e_{b_1}}
      \parstepsto \config{S_b}{e_{b_2}}$ and {\sc E-Eval-Ctxt}.

    \item Case {\sc E-New}:

      We have:
      \begin{itemize}
      \item $e_{b_1} = \NEW$,
      \item $e_{b_2} = l$, and
      \item $S_b = \extSRaw{S}{l}{\bot}$.
      \end{itemize}

      Choose $S' = S_b$, $i = 1$, $j = 1$, and $\pi = \id$.

      We have to show that:

      \begin{itemize}
      \item $\config{S}{\evalctxt{E'_b}{e_{b_1}}} \ctxstepsto
        \config{S_b}{\evalctxt{E'_b}{e_{b_2}}}$, and
      \item
        $\config{S_b}{\evalctxt{E'_a}{e_{a_1}}} \ctxstepsto
        \config{S_b}{\evalctxt{E'_a}{e_{a_2}}}$.
      \end{itemize}

      The first of these follows immediately from $\config{S}{e_{b_1}}
      \parstepsto \config{S_b}{e_{b_2}}$ and {\sc E-Eval-Ctxt}.  For
      the second, consider that $S_b = \extSRaw{S}{l}{\bot} =
      \lubstore{S}{\store{\storebindingRaw{l}{\bot}}}$.  Furthermore, we
      know from the side condition of {\sc E-New} that $l \notin
      \dom{S}$, so $\store{\storebindingRaw{l}{\bot}}$ is non-conflicting
      with the transition $\config{S}{e_{a_1}} \parstepsto
      \config{S_a}{e_{a_2}}$, and we know that
      $\lubstore{S_a}{\store{\storebindingRaw{l}{\bot}}} \neq \topS$
      since $S_a$ is just $S$.  Therefore, by
      Lemma~\ref{lem:lvars-independence} (Independence), we have that
      $\config{\lubstore{S}{\store{\storebindingRaw{l}{\bot}}}}{e_{a_1}}
      \parstepsto
      \config{\lubstore{S_a}{\store{\storebindingRaw{l}{\bot}}}}{e_{a_2}}$.
      Hence $\config{S_b}{e_{a_1}} \parstepsto \config{S_b}{e_{a_2}}$.
      By {\sc E-Eval-Ctxt}, it follows that
      $\config{S_b}{\evalctxt{E'_a}{e_{a_1}}} \ctxstepsto
      \config{S_b}{\evalctxt{E'_a}{e_{a_2}}}$, as we were required to
      show.

    \item Case {\sc E-Put}: \TODO{}
    \item Case {\sc E-Put-Err}: \TODO{}
    \item Case {\sc E-Get}:\TODO{}
    \end{enumerate}
  \item Case {\sc E-New}:

    Now, we proceed by case analysis on the rule by which
    $\config{S}{e_{b_1}}$ steps to $\config{S_b}{e_{b_2}}$:
    \begin{enumerate}
    \item Case {\sc E-Beta}: \TODO{}
    \item Case {\sc E-New}: \TODO{}
    \item Case {\sc E-Put}: \TODO{}
    \item Case {\sc E-Put-Err}: \TODO{}
    \item Case {\sc E-Get}: \TODO{}
    \end{enumerate}
  \item Case {\sc E-Put}:

    Now, we proceed by case analysis on the rule by which
    $\config{S}{e_{b_1}}$ steps to $\config{S_b}{e_{b_2}}$:
    \begin{enumerate}
    \item Case {\sc E-Beta}: \TODO{}
    \item Case {\sc E-New}: \TODO{}
    \item Case {\sc E-Put}: \TODO{}
    \item Case {\sc E-Put-Err}: \TODO{}
    \item Case {\sc E-Get}: \TODO{}
    \end{enumerate}
  \item Case {\sc E-Put-Err}:

    Now, we proceed by case analysis on the rule by which
    $\config{S}{e_{b_1}}$ steps to $\config{S_b}{e_{b_2}}$:
    \begin{enumerate}
    \item Case {\sc E-Beta}: \TODO{}
    \item Case {\sc E-New}: \TODO{}
    \item Case {\sc E-Put}: \TODO{}
    \item Case {\sc E-Put-Err}: \TODO{}
    \item Case {\sc E-Get}: \TODO{}
    \end{enumerate}
  \item Case {\sc E-Get}:

    Now, we proceed by case analysis on the rule by which
    $\config{S}{e_{b_1}}$ steps to $\config{S_b}{e_{b_2}}$:
    \begin{enumerate}
    \item Case {\sc E-Beta}: \TODO{}
    \item Case {\sc E-New}: \TODO{}
    \item Case {\sc E-Put}: \TODO{}
    \item Case {\sc E-Put-Err}: \TODO{}
    \item Case {\sc E-Get}: \TODO{}
    \end{enumerate}
  \end{enumerate}

  \lk{I think we also still have to separately deal with cases where
    $\conf_a = \error$ or $\conf_b = \error$.}
\end{proof}


\section{Proof of Lemma~\ref{lem:lvars-strong-one-sided-confluence}}\label{section:lvars-strong-one-sided-confluence-proof}
\begin{proof}
  Suppose $\conf \ctxstepsto \conf'$ and $\conf \ctxstepsto^m
  \conf''$, where $1 \leq m$.  We have to show that there exist
  $\conf_c, i, j, \pi$ such that $\conf' \ctxstepsto^i \conf_c$ and
  $\pi(\conf'') \ctxstepsto^j \conf_c$ and $i \leq m$ and $j \leq 1$.

  We proceed by induction on $m$.  In the base case of $m = 1$, the
  result is immediate from
  Lemma~\ref{lem:lvars-strong-local-confluence}.

  For the induction step, suppose $\conf \ctxstepsto^m \conf''
  \ctxstepsto \conf'''$ and suppose the lemma holds for $m$.

  We show that it holds for $m + 1$, as follows.

  We are required to show that there exist $\conf_c, i, j, \pi$ such
  that $\conf' \ctxstepsto^{i} \conf_c$ and $\pi(\conf''')
  \ctxstepsto^{j} \conf_c$ and $i \leq m + 1$ and $j \leq 1$.

  From the induction hypothesis, there exist $\conf_c', i', j', \pi'$
  such that $\conf' \ctxstepsto^{i'} \conf_c'$ and $\pi'(\conf'')
  \ctxstepsto^{j'} \conf_c'$ and $i' \leq m$ and $j' \leq 1$.

  We proceed by cases on $j'$:
  \begin{itemize}

  \item If $j' = 0$, then $\pi'(\conf'') = \conf_c'$.

    Since $\conf'' \ctxstepsto \conf'''$, we have that $\pi'(\conf'')
    \ctxstepsto \pi'(\conf''')$ by
    Lemma~\ref{lem:lvars-permutability} (Permutability).

    We can then choose $\conf_c = \pi'(\conf''')$ and $i = i' + 1$ and
    $j = 0$ and $\pi = \pi'$.  The key is that $\conf'
    \ctxstepsto^{i'} \conf'_c = \pi'(\conf'') \ctxstepsto
    \pi'(\conf''')$ for a total of $i' + 1$ steps.
    
  \item If $j' = 1$:

    First, since $\pi'(\conf'') \ctxstepsto^{j'} \conf'_c$, then by
    Lemma~\ref{lem:lvars-permutability} (Permutability) we have that
    $\conf'' \ctxstepsto^{j'} \piprimeinv(\conf'_c)$.

    Then, by $\conf'' \ctxstepsto^{j'} \piprimeinv(\conf'_c)$ and
    $\conf'' \ctxstepsto \conf'''$ and
    Lemma~\ref{lem:lvars-strong-local-confluence} (Strong Local
    Confluence), we have that there exist $\conf_c''$ and $i''$ and
    $j''$ and $\pi''$ such that $\piprimeinv(\conf'_c)
    \ctxstepsto^{i''} \conf_c''$ and $\pi''(\conf''')
    \ctxstepsto^{j''} \conf_c''$ and $i'' \leq 1$ and $j'' \leq 1$.

    Since $\piprimeinv(\conf'_c) \ctxstepsto^{i''} \conf_c''$, by
    Lemma~\ref{lem:lvars-permutability} (Permutability) we have that
    $\conf'_c \ctxstepsto^{i''} \pi'(\conf_c'')$.

    So we also have $\conf' \ctxstepsto^{i'} \conf_c'
    \ctxstepsto^{i''} \pi'(\conf_c'')$.

    Since $\pi''(\conf''') \ctxstepsto^{j''} \conf_c''$, by
    Lemma~\ref{lem:lvars-permutability} (Permutability) we have that
    $\pi'(\pi''(\conf''')) \ctxstepsto^{j''} \pi'(\conf_c'')$.

    In summary, we pick $\conf_c = \pi'(\conf_c'')$ and $i = i' + i''$
    and $j = j''$ and $\pi = \pi'' \circ \pi'$, which is sufficient
    because $i = i' + i'' \leq m + 1$ and $j = j'' \leq 1$.
  \end{itemize}

 \end{proof}


\section{Proof of Lemma~\ref{lem:lvars-strong-confluence}}\label{section:lvars-strong-confluence-proof}
\begin{proof}
  We proceed by induction on $n$.  In the base case of $n = 1$, the
  result is immediate from
  Lemma~\ref{lem:lvars-strong-one-sided-confluence}.

  For the induction step, suppose $\conf \parstepsto^n \conf'
  \parstepsto \conf'''$ and suppose the lemma holds for $n$.

  We show that it holds for $n + 1$, as follows.

  We are required to show that there exist $\conf_c, i, j$ such that
  $\conf''' \parstepsto^i \conf_c$ and $\conf'' \parstepsto^j \conf_c$
  and $i \leq m$ and $j \leq n + 1$.

  From the induction hypothesis, we have that there exist $\conf'_c,
  i', j'$ such that $\conf' \parstepsto^{i'} \conf'_c$ and $\conf''
  \parstepsto^{j'} \conf'_c$ and $i' \leq m$ and $j' \leq n$.

  We proceed by cases on $i'$:
  \begin{itemize}

  \item If $i' = 0$, then $\conf' = \conf_c'$.  We can then choose
    $\conf_c = \conf'''$ and $i = 0$ and $j = j' + 1$.

  \item If $i' \geq 1$:

    From $\conf' \parstepsto \conf'''$ and $\conf' \parstepsto^{i'}
    \conf_c'$ and Lemma~\ref{lem:lvars-strong-one-sided-confluence},
    we have that there exist $\conf_c''$ and $i''$ and $j''$ such that
    $\conf''' \parstepsto^{i''} \conf_c''$ and $\conf_c'
    \parstepsto^{j''} \conf_c''$ and $i'' \leq i'$ and $j'' \leq 1$.
    So we also have $\conf'' \parstepsto^{j'} \conf_c'
    \parstepsto^{j''} \conf_c''$.  In summary, we pick $\conf_c =
    \conf_c''$ and $i = i''$ and $j = j' + j''$, which is sufficient
    because $i = i'' \leq i' \leq m$ and $j = j' + j'' \leq n + 1$.
  \end{itemize}

\end{proof}


\section{Proof of Lemma~\ref{lem:lattice-structure}}\label{section:lattice-structure-proof}
\begin{proof}
  Suppose that $(D, \userleq, \bot, \top)$ is a lattice and $(D_p,
  \leqp, \botp, \topp) = \Freeze{D, \userleq, \bot, \top}$.

  In order to show that $(D_p, \leqp, \botp, \topp)$ is a lattice, we
  have to show that:
  \begin{enumerate}
  \item $\leqp$ is a partial order over $D_p$.

  \item Every nonempty finite subset of $D_p$ has a lub.

  \item $\botp$ is the least element of $D_p$.

  \item $\topp$ is the greatest element of $D_p$.
  \end{enumerate}

  We prove each of these properties in turn:

  \begin{enumerate}
  \item $\leqp$ is a partial order over $D_p$.

    To show this, we need to show that $\leqp$ is reflexive, transitive,
    and antisymmetric. 
    \begin{enumerate}
    \item $\leqp$ is reflexive.

      Suppose $v \in D_p$.

      Then, by Lemma~\ref{lem:partition-of-Dp}, either $v =
      \state{d}{\frozenfalse}$ with $d \in D$, or $v =
      \state{x}{\frozentrue}$ with $x \in X$, where $X = D -
      \setof{\top}$.
      \begin{itemize}
      \item Suppose $v = \state{d}{\frozenfalse}$:

        By the reflexivity of $\userleq$, we know $d \userleq d$.

        By the definition of $\leqp$, we know $\state{d}{\frozenfalse}
        \leqp \state{d}{\frozenfalse}$.

      \item Suppose $v = \state{x}{\frozentrue}$: 
        
        By the reflexivity of equality, $x = x$.

        By the definition of $\leqp$, we know $\state{x}{\frozentrue}
        \leqp \state{x}{\frozentrue}$.
      \end{itemize}

    \item $\leqp$ is transitive. 

      Suppose $v_1 \leqp v_2$ and $v_2 \leqp v_3$.

      We want to show that $v_1 \leqp v_3$.

      We proceed by case analysis on $v_1, v_2$, and $v_3$.
      \begin{itemize}
      \item Case $v_1 = \state{d_1}{\frozenfalse}$ and $v_2 =
        \state{d_2}{\frozenfalse}$ and $v_3 =
        \state{d_3}{\frozenfalse}$:
        
        By inversion on $\leqp$, it follows that $d_1 \userleq d_2$.

        By inversion on $\leqp$, it follows that $d_2 \userleq d_3$.

        By the transitivity of $\userleq$, we know $d_1 \userleq d_3$.

        By the definition of $\leqp$, it follows that
        $\state{d_1}{\frozenfalse} \leqp \state{d_3}{\frozenfalse}$.

        Hence $v_1 \leqp v_3$.

      \item Case $v_1 = \state{d_1}{\frozenfalse}$ and $v_2 =
        \state{d_2}{\frozenfalse}$ and $v_3 =
        \state{x_3}{\frozentrue}$:

        By inversion on $\leqp$, it follows that $d_1 \userleq d_2$.

        By inversion on $\leqp$, it follows that $d_2 \userleq x_3$.

        By the transitivity of $\userleq$, we know $d_1 \userleq x_3$.

        By the definition of $\leqp$, it follows that
        $\state{d_1}{\frozenfalse} \leqp \state{x_3}{\frozentrue}$.

        Hence $v_1 \leqp v_3$.

      \item Case $v_1 = \state{d_1}{\frozenfalse}$ and $v_2 =
        \state{x_2}{\frozentrue}$ and $v_3 =
        \state{d_3}{\frozenfalse}$:

        By inversion on $\leqp$, it follows that $d_1 \userleq x_2$.

        By inversion on $\leqp$, it follows that $d_3 = \top$.

        Since $\top$ is the maximal element of $D$, we know $d_1
        \userleq \top \equiv d_3$.

        By the definition of $\leqp$, it follows that
        $\state{d_1}{\frozenfalse} \leqp \state{d_3}{\frozenfalse}$.

        Hence $v_1 \leqp v_3$.

      \item Case $v_1 = \state{d_1}{\frozenfalse}$ and $v_2 =
        \state{x_2}{\frozentrue}$ and $v_3 =
        \state{x_3}{\frozentrue}$:

        By inversion on $\leqp$, it follows that $d_1 \userleq x_2$.

        By inversion on $\leqp$, it follows that $x_2 = x_3$.

        Hence $d_1 \userleq x_3$.

        By the definition of $\leqp$, it follows that
        $\state{d_1}{\frozenfalse} \leqp \state{x_3}{\frozentrue}$.

        Hence $v_1 \leqp v_3$.

      \item Case $v_1 = \state{x_1}{\frozentrue}$ and $v_2 =
        \state{d_2}{\frozenfalse}$ and $v_3 =
        \state{d_3}{\frozenfalse}$:

        By inversion on $\leqp$, it follows that $d_2 = \top$.

        By inversion on $\leqp$, it follows that $d_2 \userleq d_3$.

        Since $\top$ is maximal, it follows that $d_3 = \top$.

        By the definition of $\leqp$, it follows that
        $\state{x_1}{\frozentrue} \leqp \state{d_3}{\frozenfalse}$.

        Hence $v_1 \leqp v_3$. 

      \item Case $v_1 = \state{x_1}{\frozentrue}$ and $v_2 =
        \state{d_2}{\frozenfalse}$ and $v_3 =
        \state{x_3}{\frozentrue}$:

        By inversion on $\leqp$, it follows that $d_2 = \top$.

        By inversion on $\leqp$, it follows that $d_2 \userleq x_3$.

        Since $\top$ is maximal, it follows that $x_3 = \top$.

        But since $x_3 \in X \subseteq D/\setof{\top}$, we know $x_3
        \not= \top$.

        This is a contradiction. \\

        Hence $v_1 \leqp v_3$. 

      \item Case $v_1 = \state{x_1}{\frozentrue}$ and $v_2 =
        \state{x_2}{\frozentrue}$ and $v_3 =
        \state{d_3}{\frozenfalse}$:

        By inversion on $\leqp$, it follows that $x_1 = x_2$.

        By inversion on $\leqp$, it follows that $d_3 = \top$.

        By the definition of $\leqp$, it follows that
        $\state{x_1}{\frozentrue} \leqp \state{d_3}{\frozenfalse}$.

        Hence $v_1 \leqp v_3$. 

      \item Case $v_1 = \state{x_1}{\frozentrue}$ and $v_2 =
        \state{x_2}{\frozentrue}$ and $v_3 =
        \state{x_3}{\frozentrue}$:

        By inversion on $\leqp$, it follows that $x_1 = x_2$.

        By inversion on $\leqp$, it follows that $x_2 = x_3$.

        By transitivity of $=$, $x_1 = x_3$.

        By the definition of $\leqp$, it follows that
        $\state{x_1}{\frozentrue} \leqp \state{x_3}{\frozentrue}$.

        Hence $v_1 \leqp v_3$. 
        
      \end{itemize}

    \item $\leqp$ is antisymmetric. 

      Suppose $v_1 \leqp v_2$ and $v_2 \leqp v_1$. Now, we proceed by
      cases on $v_1$ and $v_2$.
      \begin{itemize}
      \item Case $v_1 = \state{d_1}{\frozenfalse}$ and $v_2 =
        \state{d_2}{\frozenfalse}$:
        
        By inversion on $v_1 \leqp v_2$, we know that $d_1 \userleq
        d_2$.

        By inversion on $v_2 \leqp v_1$, we know that $d_2 \userleq
        d_1$.

        By the antisymmetry of $\leq$, we know $d_1 = d_2$.

        Hence $v_1 = v_2$. 

      \item Case $v_1 = \state{d_1}{\frozenfalse}$ and $v_2 =
        \state{x_2}{\frozentrue}$:

        By inversion on $v_1 \leqp v_2$, we know that $d_1 \userleq x_2$.

        By inversion on $v_2 \leqp v_1$, we know that $d_1 = \top$.

        Since $\top$ is maximal in $D$, we know $x_2 = \top$.

        But since $x_2 \in X \subseteq D/\setof{\top}$, we know $x_2 \not= \top$.

        This is a contradiction.

        Hence $v_1 = v_2$. 
        
      \item Case $v_1 = \state{x_1}{\frozentrue}$ and $v_2 =
        \state{d_2}{\frozenfalse}$:

        Similar to the previous case. 

      \item Case $v_1 = \state{x_1}{\frozentrue}$ and $v_2 =
        \state{x_2}{\frozentrue}$:

        By inversion on $v_1 \leqp v_2$, we know that $x_1 = x_2$.

        Hence $v_1 = v_2$. 
      \end{itemize}
    \end{enumerate}

  \item Every nonempty finite subset of $D_p$ has a lub.

    To show this, it is sufficient to show that every two elements of
    $D_p$ have a lub, since a binary lub operation can be repeatedly
    applied to compute the lub of any finite set.

    We will show that every two elements of $D_p$ have a lub by
    showing that the $\lubp{}{}$ operation defined by
    Definition~\ref{def:lubp} computes their lub.

    It suffices to show the following two properties:
    \begin{enumerate}
    \item For all $v_1, v_2, v \in D_p$, if $v_1 \leqp v$ and $v_2
      \leqp v$, then $(\lubp{v_1}{v_2}) \leqp v$.
    \item For all $v_1, v_2 \in D_p$, $v_1 \leqp (\lubp{v_1}{v_2})$
      and $v_2 \leqp (\lubp{v_1}{v_2})$.
    \end{enumerate}
    \begin{enumerate}
    \item For all $v_1, v_2, v \in D_p$, if $v_1 \leqp v$ and $v_2
      \leqp v$, then $\lubp{v_1}{v_2} \leqp v$.
      
      Assume $v_1, v_2, v \in D_p$, and $v_1 \leqp v$ and $v_2 \leqp
      v$.

      Now we do a case analysis on $v_1$ and $v_2$.
      \begin{itemize}
      \item Case $v_1 = \state{d_1}{\frozenfalse}$ and $v_2 =
        \state{d_2}{\frozenfalse}$.
        
        Now case on $v$: 
        \begin{itemize}
        \item Case $v = \state{d}{\frozenfalse}$: 

          By the definition of $\lubp{}{}$,
          $\lubp{\state{d_1}{\frozenfalse}}{\state{d_2}{\frozenfalse}}
          = \state{\userlub{d_1}{d_2}}{\frozenfalse}$.

          By inversion on $\state{d_1}{\frozenfalse} \leqp
          \state{d}{\frozenfalse}$, $d_1 \userleq l$.

          By inversion on $\state{d_2}{\frozenfalse} \leqp
          \state{d}{\frozenfalse}$, $d_2 \userleq l$.

          Hence $l$ is an upper bound for $d_1$ and $d_2$.

          Hence $\userlub{d_1}{d_2} \userleq l$.

          Hence $\state{\userlub{d_1}{d_2}}{\frozenfalse} \leqp
          \state{d}{\frozenfalse}$.

          Hence $\lubp{v_1}{v_2} \leqp v$.
          
        \item Case $v = \state{x}{\frozentrue}$: 
          
          By the definition of $\lubp{}{}$, $\state{d_1}{\frozenfalse}
          \lubp{}{} \state{d_2}{\frozenfalse} =
          \state{\userlub{d_1}{d_2}}{\frozenfalse}$.

          By inversion on $\state{d_1}{\frozenfalse} \leqp
          \state{x}{\frozentrue}$, $d_1 \userleq x$.

          By inversion on $\state{d_2}{\frozenfalse} \leqp
          \state{x}{\frozentrue}$, $d_2 \userleq x$.
     
          Hence $x$ is an upper bound for $d_1$ and $d_2$.

          Hence $\userlub{d_1}{d_2} \userleq x$.

          Hence $\state{\userlub{d_1}{d_2}}{\frozenfalse} \leqp
          \state{x}{\frozentrue}$.

          Hence $\lubp{v_1}{v_2} \leqp v$.
        \end{itemize}
        
      \item Case $v_1 = \state{x_1}{\frozentrue}$ and $v_2 =
        \state{x_2}{\frozentrue}$:
        
        Now case on $v$: 
        \begin{itemize}
        \item Case $v = \state{d}{\frozenfalse}$: 
          
          By inversion on $\state{x_1}{\frozentrue} \leqp
          \state{d}{\frozenfalse}$, we know $l = \top$.

          By inversion on $\state{x_2}{\frozentrue} \leqp
          \state{d}{\frozenfalse}$, we know $l = \top$.

          Now consider whether $x_1 = x_2$ or not.
        
          If it does, then by the definition of $\lubp{}{}$,
          $\state{x_1}{\frozentrue} \lubp{}{} \state{x_2}{\frozentrue}
          = \state{x_1}{\frozentrue}$.

          By definition of $\leqp$, we have $\state{x_1}{\frozentrue}
          \leqp \state{\top}{\frozenfalse}$.

          So $\lubp{v_1}{v_2} \leqp v$.

          If it does not, then $\lubp{v_1}{v_2} =
          \state{\top}{\frozenfalse}$.

          By the definition of $\leqp$, we have
          $\state{\top}{\frozenfalse} \leqp
          \state{\top}{\frozenfalse}$.

          So $\lubp{v_1}{v_2} \leqp v$.
          
        \item Case $v = \state{x}{\frozentrue}$: 
          
          By inversion on $\state{x_1}{\frozentrue} \leqp
          \state{x}{\frozentrue}$, we know $x = x_1$.

          By inversion on $\state{x_2}{\frozentrue} \leqp
          \state{x}{\frozentrue}$, we know $x = x_2$.

          Hence $x_1 = x_2$.

          By the definition of $\lubp{}{}$, $\state{x_1}{\frozentrue}
          \lubp{}{} \state{x_2}{\frozentrue} =
          \state{x_1}{\frozentrue}$.

          Hence $\lubp{v_1}{v_2} \leqp v$.
        \end{itemize}
        
      \item Case $v_1 = \state{x_1}{\frozentrue}$ and $v_2 =
        \state{d_2}{\frozenfalse}$:
        
        Now case on $v$:
        \begin{itemize}
        \item Case $v = \state{d}{\frozenfalse}$:
          
          Now consider whether $d_2 \userleq x_1$.

          If it is, then $\state{x_1}{\frozentrue} \lubp{}{}
          \state{d_2}{\frozenfalse} = \state{x_1}{\frozentrue} = v_1$.

          Hence $\lubp{v_1}{v_2} \leqp v$.

          Otherwise, $\state{x_1}{\frozentrue} \lubp{}{}
          \state{d_2}{\frozenfalse} = \state{\top}{\frozenfalse}$.

          By inversion on $\state{x_1}{\frozentrue} \leqp
          \state{d}{\frozenfalse}$, we know $l = \top$.

          By reflexivity, $\state{\top}{\frozenfalse} \leqp
          \state{\top}{\frozenfalse}$.

          Hence $\lubp{v_1}{v_2} \leqp v$. 
          
        \item Case $v = \state{x}{\frozentrue}$:  
          
          By inversion on $\state{x_1}{\frozentrue} \leqp
          \state{x}{\frozentrue}$, we know that $x_1 = x$.

          By inversion on $\state{d_2}{\frozenfalse} \leqp
          \state{x}{\frozentrue}$, we know that $d_2 \userleq x$.

          By transitivity, $d_2 \userleq x_1$.

          By the definition of $\lubp{}{}$, it follows that
          $\state{x_1}{\frozentrue} \lubp{}{}
          \state{d_2}{\frozenfalse} = \state{x_1}{\frozentrue}$.

          By definition of $\leqp$, $\state{x_1}{\frozentrue} \leqp
          \state{x_1}{\frozentrue}$.

          Hence $\lubp{v_1}{v_2} \leqp v$. 
        \end{itemize}
        
      \item Case $v_1 = \state{d_1}{\frozenfalse}$ and $v_2 =
        \state{x_2}{\frozentrue}$:
        
        Symmetric with the previous case. 
      \end{itemize}
    \item For all $v_1, v_2 \in D_p$, $v_1 \leqp \lubp{v_1}{v_2}$ and
      $v_2 \leqp \lubp{v_1}{v_2}$.
      
      Assume $v_1, v_2 \in D_p$, and proceed by case analysis. 
      \begin{itemize}
      \item Case $v_1 = \state{d_1}{\frozenfalse}$ and $v_2 =
        \state{d_2}{\frozenfalse}$:

        Since $\userlub{}{}$ is a join operator, we know $d_1 \userleq
        \userlub{d_1}{d_2}$.

        By the definition of $\leqp$, $\state{d_1}{\frozenfalse}
        \userleq \state{\userlub{d_1}{d_2}}{\frozenfalse}$.

        By the definition of $\lubp{}{}$, $\lubp{v_1}{v_2} =
        \state{\userlub{d_1}{d_2}}{\frozenfalse}$.

        Hence $v_1 \leqp \lubp{v_1}{v_2}$.

        Since $\userlub{}{}$ is a join operator, we know $d_1 \userleq
        \userlub{d_1}{d_2}$.

        By the definition of $\leqp$, $\state{d_2}{\frozenfalse}
        \userleq \state{\userlub{d_1}{d_2}}{\frozenfalse}$.

        By the definition of $\lubp{}{}$, $\lubp{v_1}{v_2} =
        \state{\userlub{d_1}{d_2}}{\frozenfalse}$.

        Hence $v_2 \leqp \lubp{v_1}{v_2}$. 

        Therefore $v_1 \leqp v_1 \userlub{}{} v_2$ and $v_2 \leqp v_1
        \userlub{}{} v_2$.
 
      \item Case $v_1 = \state{d_1}{\frozenfalse}$ and $v_2 = \state{x_2}{\frozentrue}$:

        Consider whether $d_1 \userleq x_2$. 
        \begin{itemize}
        \item Case  $d_1 \userleq x_2$:

          By the definition of $\lubp{}{}$, we know
          $\state{d_1}{\frozenfalse} \lubp{}{}
          \state{x_2}{\frozentrue} = \state{x_2}{\frozentrue}$.

          By the definition of $\lubp{}{}$, we know
          $\state{d_1}{\frozenfalse} \leqp \state{x_2}{\frozentrue}$.

          Hence $v_1 \leqp \lubp{v_1}{v_2}$.

          By reflexivity, $\state{x_2}{\frozentrue} \leqp
          \state{x_2}{\frozentrue}$.

          Hence $v_2 \leqp \lubp{v_1}{v_2}$.

          Therefore $v_1 \leqp v_1 \userlub{}{} v_2$ and $v_2 \leqp
          v_1 \userlub{}{} v_2$.

        \item Case $d_1 \not\userleq x_2$:

          By the definition of $\lubp{}{}$, we know
          $\state{d_1}{\frozenfalse} \lubp{}{}
          \state{x_2}{\frozentrue} = \state{\top}{\frozenfalse}$.

          Since $d_1 \userleq \top$, by the definition of $\leqp$ we
          know $\state{d_1}{\frozenfalse} \userleq
          \state{\top}{\frozenfalse}$.

          Hence $v_1 \leqp \lubp{v_1}{v_2}$.

          By the definition of $\leqp$, we know
          $\state{x_2}{\frozentrue} \userleq
          \state{\top}{\frozenfalse}$.

          Hence $v_2 \leqp \lubp{v_1}{v_2}$.

          Therefore $v_1 \leqp v_1 \userlub{}{} v_2$ and $v_2 \leqp
          v_1 \userlub{}{} v_2$.
        \end{itemize}
      \item Case $v_1 = \state{x_1}{\frozentrue}$ and $v_2 =
        \state{d_2}{\frozenfalse}$:

        Symmetric with the previous case. 
      \item Case $v_1 = \state{x_1}{\frozentrue}$ and $v_2 =
        \state{x_2}{\frozentrue}$:

        Consider whether $x_1$ equals $x_2$. 
        \begin{itemize}
        \item Case $x_1 = x_2$:
          
          By the definition $\lubp{}{}$, $\state{x_1}{\frozentrue}
          \lubp{}{} \state{x_2}{\frozentrue} =
          \state{x_1}{\frozentrue}$.
 
          By reflexivity, $\state{x_1}{\frozentrue} \leqp
          \state{x_1}{\frozentrue}$.

          Hence $v_1 \leqp \lubp{v_1}{v_2}$.

          By reflexivity, $\state{x_2}{\frozentrue} \leqp
          \state{x_1}{\frozentrue}$.

          Hence $v_2 \leqp \lubp{v_1}{v_2}$.

          Therefore $v_1 \leqp v_1 \userlub{}{} v_2$ and $v_2 \leqp
          v_1 \userlub{}{} v_2$.

        \item Case $x_1 \not= x_2$: 

          By the definition $\lubp{}{}$, $\state{x_1}{\frozentrue}
          \lubp{}{} \state{x_2}{\frozentrue} =
          \state{\top}{\frozenfalse}$.

          By the definition of $\leqp$, $\state{x_1}{\frozentrue}
          \leqp \state{\top}{\frozenfalse}$.

          Hence $v_1 \leqp \lubp{v_1}{v_2}$.

          By the definition of $\leqp$, $\state{x_2}{\frozentrue}
          \leqp \state{\top}{\frozenfalse}$.

          Hence $v_2 \leqp \lubp{v_1}{v_2}$.

          Therefore $v_1 \leqp v_1 \userlub{}{} v_2$ and $v_2 \leqp
          v_1 \userlub{}{} v_2$.
        \end{itemize}
      \end{itemize}
    \end{enumerate}

  \item $\botp$ is the least element of $D_p$. 

    $\botp$ is defined to be $\state{\bot}{\frozenfalse}$.

    In order to be the least element of $D_p$, it must be less than or
    equal to every element of $D_p$.

    By Lemma~\ref{lem:partition-of-Dp}, the elements of $D_p$
    partition into $\state{d}{\frozenfalse}$ for all $d \in D$, and
    $\state{x}{\frozentrue}$ for all $x \in X$, where $X = D -
    \setof{\top}$.

    We consider both cases:

    \begin{itemize}
    \item $\state{d}{\frozenfalse}$ for all $d \in D$:

      By the definition of $\leqp$, $\state{\bot}{\frozenfalse} \leqp
      \state{d}{\frozenfalse}$ iff $\bot \userleq d$.

      Since $\bot$ is the least element of $D$, $\bot \userleq d$.

      Therefore $\botp = \state{\bot}{\frozenfalse} \leqp
      \state{d}{\frozenfalse}$.

    \item $\state{x}{\frozentrue}$ for all $x \in X$:

      By the definition of $\leqp$, $\state{\bot}{\frozenfalse} \leqp
      \state{x}{\frozentrue}$ iff $\bot \userleq x$.

      Since $\bot$ is the least element of $D$, $\bot \userleq x$.

      Therefore $\botp = \state{\bot}{\frozenfalse} \leqp
      \state{x}{\frozentrue}$.

    \end{itemize}

    Therefore $\botp$ is less than or equal to all elements of $D_p$.

  \item $\topp$ is the greatest element of $D_p$.

    $\topp$ is defined to be $\state{\top}{\frozenfalse}$.

    In order to be the greatest element of $D_p$, every element of
    $D_p$ must be less than or equal to it.

    By Lemma~\ref{lem:partition-of-Dp}, the elements of $D_p$
    partition into $\state{d}{\frozenfalse}$ for all $d \in D$, and
    $\state{x}{\frozentrue}$ for all $x \in X$, where $X = D -
    \setof{\top}$.

    We consider both cases:

    \begin{itemize}
    \item $\state{d}{\frozenfalse}$ for all $d \in D$:

      By the definition of $\leqp$, $\state{d}{\frozenfalse} \leqp
      \state{\top}{\frozenfalse}$ iff $d \userleq \top$.

      Since $\top$ is the greatest element of $D$, $d \userleq \top$.

      Therefore $\state{d}{\frozenfalse} \leqp
      \state{\top}{\frozenfalse} = \topp$.

    \item $\state{x}{\frozentrue}$ for all $x \in X$:

      By the definition of $\leqp$, $\state{x}{\frozentrue} \leqp
      \state{\top}{\frozenfalse}$ iff $\top \userleq \top$.

      Therefore $\state{x}{\frozentrue} \leqp
      \state{\top}{\frozenfalse} = \topp$.

    \end{itemize}

    Therefore all elements of $D_p$ are less than or equal to $\topp$.
  \end{enumerate}
\end{proof}


\section{Proof of Lemma~\ref{lem:monotonicity}}\label{section:monotonicity-proof}
\begin{proof}
  \TODO{Fix the typos I found in this.}

  \begin{itemize}

    \item Case {\sc E-Eval-Ctxt}:

      Given: $\config{S}{\E{e}} \parstepsto \config{S'}{\E{e'}}$.

      To show: $\leqstore{S}{S'}$.

      From the premise of {\sc E-Eval-Ctxt}, $\config{S}{e}
      \parstepsto \config{S'}{e'}$.

      Hence by IH, $\leqstore{S}{S'}$, as we were required to show.

    \item Case {\sc E-Beta}:

      Immediate by the definition of $\leqstore{}{}$, since $S$ does
      not change.

    \item Case {\sc E-New}:

      Given: $\config{S}{\NEW} \parstepsto
      \config{\extS{S}{l}{\bot}{\frozenfalse}}{l}$.

      To show: $\leqstore{S}{\extS{S}{l}{\bot}{\frozenfalse}}$.

      By Definition~\ref{def:leqstore}, we have to show that $\dom{S}
      \subseteq \dom{\extS{S}{l}{\bot}{\frozenfalse}}$ and that for
      all $l' \in \dom{S}, \\
      S(l') \leqp (\extS{S}{l}{\bot}{\frozenfalse})(l')$.

      By the definition of store update,
      $\extS{S}{l}{d_1}{\frozentrue}$ can only either update an
      existing binding in $S$ or extend $S$ with a new binding.

      Hence $\dom{S} \subseteq \dom{\extS{S}{l}{\bot}{\frozenfalse}}$.

      From the side condition of {\sc E-New}, $l \notin \dom{S}$.

      Hence $\extS{S}{l}{\bot}{\frozenfalse}$ adds a new binding for
      $l$ in $S$.

      Hence $\extS{S}{l}{d_1}{\frozentrue}$ does not update any
      existing bindings in $S$.

      Hence, for all $l' \in \dom{S}, S(l') \leqp
      (\extS{S}{l}{d_1}{\frozentrue})(l')$.

      Therefore $\leqstore{S}{\extS{S}{l}{\bot}{\frozenfalse}}$, as
      required.

    \item Case {\sc E-Put}:

      Given: $\config{S}{\putexp{l}{d_2}} \parstepsto
      \config{\extSRaw{S}{l}{p_2}}{\unit}$.

      To show: $\leqstore{S}{\extSRaw{S}{l}{p_2}}$.

      By Definition~\ref{def:leqstore}, we have to show that $\dom{S}
      \subseteq \dom{\extSRaw{S}{l}{p_2}}$ and that for all $l' \in
      \dom{S}, \\
      S(l') \leqp (\extSRaw{S}{l}{p_2})(l')$.

      By the definition of store update, $\extSRaw{S}{l}{p_2}$ can only
      either update an existing binding in $S$ or extend $S$ with a
      new binding.

      Hence $\dom{S} \subseteq \dom{\extSRaw{S}{l}{p_2}}$.

      From the premises of {\sc E-Put}, $S(l) = p_1$.  Therefore $l
      \in \dom{S}$.

      Hence $\extSRaw{S}{l}{p_2}$ updates the existing binding for $l$
      in $S$ from $p_1$ to $p_2$.

      From the premises of {\sc E-Put}, $p_2 =
      \lubp{p_1}{\state{d_2}{\frozenfalse}}$.

      Hence, by the definition of $\lubp{}{}$, $p_1 \leqp p_2$.

      $\extSRaw{S}{l}{p_2}$ does not update any other bindings in $S$,
      hence, for all $l' \in \dom{S}, S(l') \leqp
      (\extSRaw{S}{l}{p_2})(l')$.

      Hence $\leqstore{S}{\extSRaw{S}{l}{p_2}}$, as required.

    \item Case {\sc E-Put-Err}:

      Given: $\config{S}{\putexp{l}{d_2}} \parstepsto \error$.

      By the definition of $\error$, $\error = \config{\topS}{e}$ for
      any $e$.

      To show: $\leqstore{S}{\topS}$.

      Immediate by the definition of $\leqstore{}{}$.

    \item Case {\sc E-Get}:

      Immediate by the definition of $\leqstore{}{}$, since $S$ does
      not change.

    \item Case {\sc E-Freeze-Init}:

      Immediate by the definition of $\leqstore{}{}$, since $S$ does
      not change.

    \item Case {\sc E-Spawn-Handler}:

      Immediate by the definition of $\leqstore{}{}$, since $S$ does
      not change.

    \item Case {\sc E-Freeze-Final}:

      Given: $\config{S}{\freezeafterfull{l}{Q}{v}{\setof{v\dots}}{H}}
      \parstepsto \config{\extS{S}{l}{d_1}{\frozentrue}}{d_1}$.

      To show: $\leqstore{S}{\extS{S}{l}{d_1}{\frozentrue}}$.

      By Definition~\ref{def:leqstore}, we have to show that $\dom{S}
      \subseteq \dom{\extS{S}{l}{d_1}{\frozentrue}}$ and that for all
      $l' \in \dom{S}, \\
      S(l') \leqp (\extS{S}{l}{d_1}{\frozentrue})(l')$.

      \lk{We could spell this out in even more excruciating detail,
        but I think it's obvious enough.}

      By the definition of store update,
      $\extS{S}{l}{d_1}{\frozentrue}$ can only either update an
      existing binding in $S$ or extend $S$ with a new binding.

      Hence $\dom{S} \subseteq \dom{\extS{S}{l}{d_1}{\frozentrue}}$.

      From the premises of {\sc E-Freeze-Final}, $S(l) =
      \state{d_1}{\status_1}$.  Therefore $l \in \dom{S}$.

      Hence $\extS{S}{l}{d_1}{\frozentrue}$ updates the existing
      binding for $l$ in $S$ from $\state{d_1}{\status_1}$ to
      $\state{d_1}{\frozentrue}$.

      By the definition of $\leqp$, $\state{d_1}{\status_1} \leqp
      \state{d_1}{\frozentrue}$.

      $\extS{S}{l}{d_1}{\frozentrue}$ does not update any other
      bindings in $S$, hence, for all $l' \in \dom{S}, \\
      S(l') \leqp (\extS{S}{l}{d_1}{\frozentrue})(l')$.

      Hence $\leqstore{S}{\extS{S}{l}{d_1}{\frozentrue}}$, as
      required.

    \item Case {\sc E-Freeze-Simple}:

      Given: $\config{S}{\freeze{l}} \parstepsto
      \config{\extS{S}{l}{d_1}{\frozentrue}}{d_1}$.

      To show: $\leqstore{S}{\extS{S}{l}{d_1}{\frozentrue}}$.

      Similar to the previous case.

  \end{itemize}

\end{proof}


\section{Proof of Lemma~\ref{lem:independence}}\label{section:independence-proof}
\begin{proof}
  Consider arbitrary $S''$ such that $S''$ is non-conflicting with
  $\config{S}{e} \parstepsto \config{S'}{e'}$ and $\lubstore{S'}{S''}
  \statuseq S$ and $\lubstore{S'}{S''} \neq \topS$.

  To show: $\config{\lubstore{S}{S''}}{e} \parstepsto
  \config{\lubstore{S'}{S''}}{e'}$.

  The proof is by cases on the rule of the reduction semantics by
  which $\config{S}{e}$ steps to $\config{S'}{e'}$.  Since
  $\config{S'}{e'} \neq \error$, we do not need to consider the {\sc
    E-Put-Err} rule.

  The assumption that $\lubstore{S'}{S''} \statuseq S$ is only needed
  in the {\sc E-Freeze-Final} and {\sc E-Freeze-Simple} cases.

  \begin{itemize}

    \item Case {\sc E-Beta}:

      Given: $\config{S}{\app{(\lam{x}{e})}{v}} \parstepsto
      \config{S}{\subst{e}{x}{v}}$.

      To show: $\config{\lubstore{S}{S''}}{\app{(\lam{x}{e})}{v}} \parstepsto
      \config{\lubstore{S}{S''}}{\subst{e}{x}{v}}$.

      Immediate by {\sc E-Beta}.

    \item Case {\sc E-New}:

      Given: $\config{S}{\NEW} \parstepsto
      \config{\extS{S}{l}{\bot}{\frozenfalse}}{l}$.

      To show: $\config{\lubstore{S}{S''}}{\NEW} \parstepsto
      \config{\lubstore{(\extS{S}{l}{\bot}{\frozenfalse})}{S''}}{l}$.

      By {\sc E-New}, we have that $\config{\lubstore{S}{S''}}{\NEW}
      \parstepsto
      \config{\extS{(\lubstore{S}{S''})}{l'}{\bot}{\frozenfalse}}{l'}$,
      where $l' \notin \dom{\lubstore{S}{S''}}$.

      By assumption, $S''$ is non-conflicting with $\config{S}{\NEW}
      \parstepsto \config{\extS{S}{l}{\bot}{\frozenfalse}}{l}$.
 
      Therefore $l \notin \dom{S''}$.

      From the side condition of {\sc E-New}, $l \notin \dom{S}$.

      Therefore $l \notin \dom{\lubstore{S}{S''}}$.

      Therefore, in
      $\config{\extS{(\lubstore{S}{S''})}{l'}{\bot}{\frozenfalse}}{l'}$,
      we can $\alpha$-rename $l'$ to $l$, resulting in
      $\config{\extS{(\lubstore{S}{S''})}{l}{\bot}{\frozenfalse}}{l}$.

      Therefore $\config{\lubstore{S}{S''}}{\NEW} \parstepsto
      \config{\extS{(\lubstore{S}{S''})}{l}{\bot}{\frozenfalse}}{l}$.

      Note that:
      \begin{align*}
        \extS{(\lubstore{S}{S''})}{l}{\bot}{\frozenfalse} &=
        \lubstore{\extS{S}{l}{\bot}{\frozenfalse}}{\extS{S''}{l}{\bot}{\frozenfalse}} \\
        &= \lubstore{\lubstore{S}{\store{\storebinding{l}{\bot}{\frozenfalse}}}}{\lubstore{S''}{\store{\storebinding{l}{\bot}{\frozenfalse}}}} \\
        &= \lubstore{\lubstore{S}{\store{\storebinding{l}{\bot}{\frozenfalse}}}}{S''} \\
        &= \lubstore{\extS{S}{l}{\bot}{\frozenfalse}}{S''}.
      \end{align*}
      Therefore $\config{\lubstore{S}{S''}}{\NEW} \parstepsto
      \config{\lubstore{\extS{S}{l}{\bot}{\frozenfalse}}{S''}}{l}$, as we were
      required to show.

    \item Case {\sc E-Put}:

      Given: $\config{S}{\putiexp{l}} \parstepsto
      \config{\extSRaw{S}{l}{u_{p_i}(p_1)}}{\unit}$.

      To show: $\config{\lubstore{S}{S''}}{\putiexp{l}{d_2}}
      \parstepsto
      \config{\lubstore{\extSRaw{S}{l}{u_{p_i}(p_1)}}{S''}}{\unit}$.

      We will first show that

      $\config{\lubstore{S}{S''}}{\putiexp{l}{d_2}} \parstepsto
      \config{\extSRaw{(\lubstore{S}{S''})}{l}{u_{p_i}(p_1)}}{\unit}$

      and then show why this is sufficient.

      We proceed by cases on $l$:

      \begin{itemize}
        \item $l \notin \dom{S''}$:

          By assumption, $\lubstore{\extSRaw{S}{l}{u_{p_i}(p_1)}}{S''}
          \neq \topS$.

          By Lemma~\ref{lem:monotonicity},
          $\leqstore{S}{\extSRaw{S}{l}{u_{p_i}(p_1)}}$.

          Hence $\lubstore{S}{S''} \neq \topS$.

          Therefore, by Definition~\ref{def:lubstore},
          $(\lubstore{S}{S''})(l) = S(l)$.

          From the premises of {\sc E-Put}, $S(l) = p_1$.

          Hence $(\lubstore{S}{S''})(l) = p_1$.

          From the premises of {\sc E-Put}, $u_{p_i}(p_1) \neq \topp$.

          Therefore, by {\sc E-Put}, we have:
          $\config{\lubstore{S}{S''}}{\putiexp{l}} \parstepsto
          \config{\extSRaw{(\lubstore{S}{S''})}{l}{u_{p_i}(p_1)}}{\unit}$.

        \item $l \in \dom{S''}$:

          By assumption, $\lubstore{\extSRaw{S}{l}{u_{p_i}(p_1)}}{S''} \neq
          \topS$.

          By Lemma~\ref{lem:monotonicity},
          $\leqstore{S}{\extSRaw{S}{l}{u_{p_i}(p_1)}}$.

          Hence $\lubstore{S}{S''} \neq \topS$.

          Therefore $(\lubstore{S}{S''})(l) = \lubp{S(l)}{S''(l)}$.

          From the premises of {\sc E-Put}, $S(l) = p_1$.
          
          Hence $(\lubstore{S}{S''})(l) = p'_1$, where $p_1 \leqp
          p'_1$.

          \TODO{From here forward, this subcase still needs to be
            fixed.}

          By assumption, $\lubstore{\extSRaw{S}{l}{p_2}}{S''} \neq
          \topS$.

          Therefore, by Definition~\ref{def:lubstore},
          $\lubp{p_2}{S''(l)} \neq \topp$.

          Note that:
          \begin{align*}
            \topp &\neq \lubp{p_2}{S''(l)} \\
            &= \lubp{\lubp{p_1}{\state{d_2}{\frozenfalse}}}{S''(l)} \\
            &= \lubp{\lubp{S(l)}{\state{d_2}{\frozenfalse}}}{S''(l)} \\
            &= \lubp{\lubp{S(l)}{S''(l)}}{\state{d_2}{\frozenfalse}} \\
            &= \lubp{(\lubstore{S}{S''})(l)}{\state{d_2}{\frozenfalse}} \\
            &= \lubp{p'_1}{\state{d_2}{\frozenfalse}} \\
            &= p'_2. \\
          \end{align*}
          Hence $p'_2 \neq \topp$.

          Hence $(\lubstore{S}{S''})(l) = p'_1$ and $p'_2 =
          \lubp{p'_1}{\state{d_2}{\frozenfalse}}$ and $p'_2 \neq
          \topp$.

          Therefore, by {\sc E-Put} we have:
          $\config{\lubstore{S}{S''}}{\putiexp{l}{d_2}} \parstepsto
          \config{\extSRaw{(\lubstore{S}{S''})}{l}{p'_2}}{\unit}$.

          \lk{If we really wanted to be pedantic here, we'd actually
            prove that the stores are equal.  I'm assuming that if I
            can show that $\extSRaw{(\lubstore{S}{S''})}{l}{p'_2}$ and
            $\extSRaw{(\lubstore{S}{S''})}{l}{p_2}$ bind $l$ to the
            same value, then it will be obvious that they're equal.}

          Note that:
          \begin{align*}
            (\extSRaw{(\lubstore{S}{S''})}{l}{p'_2})(l) &= \lubp{(\lubstore{S}{S''})(l)}{(\store{\storebindingRaw{l}{p'_2}})(l)} \\
            &= \lubp{p'_1}{p'_2} \\
            &= \lubp{p'_1}{\lubp{p'_1}{\state{d_2}{\frozenfalse}}} \\
            &= \lubp{p'_1}{\state{d_2}{\frozenfalse}}
          \end{align*}
          and
          \begin{align*}
            (\extSRaw{(\lubstore{S}{S''})}{l}{p_2})(l) &= \lubp{(\lubstore{S}{S''})(l)}{(\store{\storebindingRaw{l}{p_2}})(l)} \\
            &= \lubp{p'_1}{p_2} \\
            &= \lubp{p'_1}{\lubp{p_1}{\state{d_2}{\frozenfalse}}} \\
            &= \lubp{p'_1}{\state{d_2}{\frozenfalse}} & \textrm{(since $p_1 \leqp p'_1$).}
          \end{align*}
          Therefore $\extSRaw{(\lubstore{S}{S''})}{l}{p'_2} =
          \extSRaw{(\lubstore{S}{S''})}{l}{p_2}$.

          Therefore, $\config{\lubstore{S}{S''}}{\putiexp{l}{d_2}}
          \parstepsto
          \config{\extSRaw{(\lubstore{S}{S''})}{l}{p_2}}{\unit}$.
      \end{itemize}

      Note that:
      \begin{align*}
        \extSRaw{(\lubstore{S}{S''})}{l}{p_2} &= \lubstore{\extSRaw{S}{l}{p_2}}{\extSRaw{S''}{l}{p_2}} \\
        &= \lubstore{\lubstore{S}{\store{\storebindingRaw{l}{p_2}}}}{\lubstore{S''}{\store{\storebindingRaw{l}{p_2}}}} \\
        &= \lubstore{\lubstore{S}{\store{\storebindingRaw{l}{p_2}}}}{S''} \\
        &= \lubstore{\extSRaw{S}{l}{p_2}}{S''}.
      \end{align*}
      Therefore $\config{\lubstore{S}{S''}}{\putiexp{l}{d_2}}
      \parstepsto \config{\lubstore{\extSRaw{S}{l}{p_2}}{S''}}{\unit}$,
      as we were required to show.

    \item Case {\sc E-Get}:

      Given: $\config{S}{\getexp{l}{P}} \parstepsto \config{S}{p_2}$.

      To show: $\config{\lubstore{S}{S''}}{\getexp{l}{P}} \parstepsto
      \config{\lubstore{S}{S''}}{p_2}$.

      From the premises of {\sc E-Get}, $S(l) = p_1$ and $\incomp{P}$
      and $p_2 \in P$ and $p_2 \leqp p_1$.

      By assumption, $\lubstore{S}{S''} \neq \topS$.

      Hence $(\lubstore{S}{S''}) = p'_1$, where $p_1 \leqp p'_1$.

      By the transitivity of $\leqp$, $p_2 \leqp p'_1$.

      Hence, $S(l) = p'_1$ and $\incomp{P}$ and $p_2 \in P$ and $p_2
      \leqp p'_1$.

      Therefore, by {\sc E-Get},

      $\config{\lubstore{S}{S''}}{\getexp{l}{P}} \parstepsto
      \config{\lubstore{S}{S''}}{p_2}$,

      as we were required to show.

    \item Case {\sc E-Freeze-Init}:

      Given: $\config{S}{\freezeafter{l}{Q}{\lam{x}{e}}} \parstepsto
      \config{S}{\freezeafterfull{l}{Q}{\lam{x}{e}}{\setof{}}{\setof{}}}$.

      To show:
      $\config{\lubstore{S}{S''}}{\freezeafter{l}{Q}{\lam{x}{e}}}
      \parstepsto
      \config{\lubstore{S}{S''}}{\freezeafterfull{l}{Q}{\lam{x}{e}}{\setof{}}{\setof{}}}$.

      Immediate by {\sc E-Freeze-Init}.

    \item Case {\sc E-Spawn-Handler}:

      Given:

      $\config{S}{\freezeafterfull{l}{Q}{\lam{x}{e_0}}{\setof{e,
            \dots}}{H}} \parstepsto
      \config{S}{\freezeafterfull{l}{Q}{\lam{x}{e_0}}{\setof{\subst{e_0}{x}{d_2},
            e, \dots}} {\{d_2\}\cup H}}$.

      To show:

      $\config{\lubstore{S}{S''}}{\freezeafterfull{l}{Q}{\lam{x}{e_0}}{\setof{e,
            \dots}}{H}} \parstepsto
      \config{\lubstore{S}{S''}}{\freezeafterfull{l}{Q}{\lam{x}{e_0}}{\setof{\subst{e_0}{x}{d_2},
            e, \dots}} {\{d_2\}\cup H}}$.

      From the premises of {\sc E-Spawn-Handler}, $S(l) =
      \state{d_1}{\status_1}$ and $d_2 \userleq d_1$ and $d_2 \notin
      H$ and $d_2 \in Q$.

      By assumption, $\lubstore{S}{S''} \neq \topS$.

      Hence $(\lubstore{S}{S''})(l) = \state{d'_1}{\status'_1}$ where
      $\state{d_1}{\status_1} \leqp \state{d'_1}{\status'_1}$.

      By Definition~\ref{def:lattice-with-status-bits}, $d_1 \userleq
      d'_1$.

      By the transitivity of $\userleq$, $d_2 \userleq d'_1$.

      Hence $(\lubstore{S}{S''})(l) =
      \state{d'_1}{\status'_1}$ and $d_2 \userleq d'_1$ and $d_2 \notin
      H$ and $d_2 \in Q$.

      Therefore, by {\sc E-Spawn-Handler},

      $\config{\lubstore{S}{S''}}{\freezeafterfull{l}{Q}{\lam{x}{e_0}}{\setof{e,
            \dots}}{H}} \parstepsto
      \config{\lubstore{S}{S''}}{\freezeafterfull{l}{Q}{\lam{x}{e_0}}{\setof{\subst{e_0}{x}{d_2},
            e, \dots}} {\{d_2\}\cup H}}$,

      as we were required to show.

    \item Case {\sc E-Freeze-Final}:

      \lk{This case wouldn't work but for the $\lubstore{S'}{S''}
        \statuseq S$ requirement, which makes it a no-op freeze.}

      Given:
      $\config{S}{\freezeafterfull{l}{Q}{\lam{x}{e_0}}{\setof{v,
            \dots}}{H}} \parstepsto
      \config{\extS{S}{l}{d_1}{\frozentrue}}{d_1}$.

      To show:
      $\config{\lubstore{S}{S''}}{\freezeafterfull{l}{Q}{\lam{x}{e_0}}{\setof{v,
            \dots}}{H}} \parstepsto
      \config{\lubstore{\extS{S}{l}{d_1}{\frozentrue}}{S''}}{d_1}$.

      We will first show that

      $\config{\lubstore{S}{S''}}{\freezeafterfull{l}{Q}{\lam{x}{e_0}}{\setof{v,
            \dots}}{H}} \parstepsto
      \config{\extS{(\lubstore{S}{S''})}{l}{d_1}{\frozentrue}}{d_1}$

      and then show why this is sufficient.

      We proceed by cases on $l$:
      \begin{itemize}
      \item $l \notin \dom{S''}$:

        By assumption, $\lubstore{\extS{S}{l}{d_1}{\frozentrue}}{S''}
        \neq \topS$.

        By Lemma~\ref{lem:monotonicity},
        $\leqstore{S}{\extS{S}{l}{d_1}{\frozentrue}}$.

        Therefore $\lubstore{S}{S''} \neq \topS$.

        Hence, by Definition~\ref{def:lubstore},
        $(\lubstore{S}{S''})(l) = S(l)$.

        From the premises of {\sc E-Freeze-Final}, we have that $S(l)
        = \state{d_1}{\status_1}$.

        Hence $(\lubstore{S}{S''})(l) = \state{d_1}{\status_1}$.

        From the premises of {\sc E-Freeze-Final}, we have that
        $\forall{d_2} ~.~ ( {d_2 \userleq d_1 \land d_2 \in Q} \Rightarrow d_2 \in
        H)$.

        Therefore, by {\sc E-Freeze-Final}, we have that

        $\config{\lubstore{S}{S''}}{\freezeafterfull{l}{Q}{\lam{x}{e_0}}{\setof{v,
              \dots}}{H}} \parstepsto
        \config{\extS{(\lubstore{S}{S''})}{l}{d_1}{\frozentrue}}{d_1}$.


      \item $l \in \dom{S''}$:

        By assumption, $\lubstore{\extS{S}{l}{d_1}{\frozentrue}}{S''}
        \neq \topS$.

        By Lemma~\ref{lem:monotonicity},
        $\leqstore{S}{\extS{S}{l}{d_1}{\frozentrue}}$.

        Therefore $\lubstore{S}{S''} \neq \topS$.

        Hence, by Definition~\ref{def:lubstore},
        $(\lubstore{S}{S''})(l) = \lubp{S(l)}{S''(l)}$.

        From the premises of {\sc E-Freeze-Final}, we have that
        $S(l) = \state{d_1}{\status_1}$.

        By assumption, $\lubstore{\extS{S}{l}{d_1}{\frozentrue}}{S''}
        \statuseq S$.

        Therefore $\status_1 = \frozentrue$.

        Therefore $S(l) = \state{d_1}{\frozentrue}$.

        Therefore $(\lubstore{S}{S''})(l) =
        \lubp{\state{d_1}{\frozentrue}}{S''(l)}$.

        We proceed by cases on $S''(l)$:
        \begin{itemize}
        \item $S''(l) = \state{d_3}{\frozenfalse}$, where $d_3 \userleq d_1$:

          By Definition~\ref{def:lubp},
          $\lubp{\state{d_1}{\frozentrue}}{\state{d_3}{\frozenfalse}}
          = \state{d_1}{\frozentrue}$.

          Therefore $(\lubstore{S}{S''})(l) =
          \state{d_1}{\frozentrue}$.

          From the premises of {\sc E-Freeze-Final}, we have that
          $\forall{d_2} ~.~ ( {d_2 \userleq d_1 \land d_2 \in Q} \Rightarrow d_2 \in
          H)$.

          Therefore, by {\sc E-Freeze-Final}, we have that

          $\config{\lubstore{S}{S''}}{\freezeafterfull{l}{Q}{\lam{x}{e_0}}{\setof{v,
                \dots}}{H}} \parstepsto
          \config{\extS{(\lubstore{S}{S''})}{l}{d_1}{\frozentrue}}{d_1}$.

        \item $S''(l) = \state{d_3}{\frozenfalse}$, where $d_3 \nuserleq d_1$:

          By Definition~\ref{def:lubp},
          $\lubp{\state{d_1}{\frozentrue}}{\state{d_3}{\frozenfalse}}
          = \state{\top}{\frozenfalse}$.

          Therefore $\lubp{S(l)}{S''(l)} =
          \state{\top}{\frozenfalse}$.

          By Definition~\ref{def:lattice-with-status-bits},
          $\state{\top}{\frozenfalse} = \topp$.

          Therefore $\lubp{S(l)}{S''(l)} = \topp$.

          Therefore, by Definition~\ref{def:lubstore},
          $\lubstore{S}{S''} = \topS$.

          This is a contradiction.

          Therefore,

          $\config{\lubstore{S}{S''}}{\freezeafterfull{l}{Q}{\lam{x}{e_0}}{\setof{v,
                \dots}}{H}} \parstepsto
          \config{\extS{(\lubstore{S}{S''})}{l}{d_1}{\frozentrue}}{d_1}$.

        \item $S''(l) = \state{d_3}{\frozentrue}$, where $d_3 = d_1$:

          By Definition~\ref{def:lubp},
          $\lubp{\state{d_1}{\frozentrue}}{\state{d_3}{\frozentrue}} =
          \state{d_1}{\frozentrue}$.

          Therefore $(\lubstore{S}{S''})(l) = \state{d_1}{\frozentrue}$.

          From the premises of {\sc E-Freeze-Final}, we have that
          $\forall{d_2} ~.~ ( {d_2 \userleq d_1 \land d_2 \in Q} \Rightarrow d_2 \in
          H)$.

          Therefore, by {\sc E-Freeze-Final}, we have that

          $\config{\lubstore{S}{S''}}{\freezeafterfull{l}{Q}{\lam{x}{e_0}}{\setof{v,
                \dots}}{H}} \parstepsto
          \config{\extS{(\lubstore{S}{S''})}{l}{d_1}{\frozentrue}}{d_1}$.

        \item $S''(l) = \state{d_3}{\frozentrue}$, where $d_3 \neq d_1$:

          By Definition~\ref{def:lubp},
          $\lubp{\state{d_1}{\frozentrue}}{\state{d_3}{\frozentrue}}
          = \state{\top}{\frozenfalse}$.

          Therefore $\lubp{S(l)}{S''(l)} = \state{\top}{\frozenfalse}$.

          By Definition~\ref{def:lattice-with-status-bits},
          $\state{\top}{\frozenfalse} = \topp$.

          Therefore $\lubp{S(l)}{S''(l)} = \topp$.

          Therefore, by Definition~\ref{def:lubstore},
          $\lubstore{S}{S''} = \topS$.

          This is a contradiction.

          Therefore,

          $\config{\lubstore{S}{S''}}{\freezeafterfull{l}{Q}{\lam{x}{e_0}}{\setof{v,
                \dots}}{H}} \parstepsto
          \config{\extS{(\lubstore{S}{S''})}{l}{d_1}{\frozentrue}}{d_1}$.
        \end{itemize}
      \end{itemize}

      In each case we have shown that

      $\config{\lubstore{S}{S''}}{\freezeafterfull{l}{Q}{\lam{x}{e_0}}{\setof{v,
            \dots}}{H}} \parstepsto
      \config{\extS{(\lubstore{S}{S''})}{l}{d_1}{\frozentrue}}{d_1}$.

      Note that:
      \begin{align*}
        \extS{(\lubstore{S}{S''})}{l}{d_1}{\frozentrue} &=
        \lubstore{\extS{S}{l}{d_1}{\frozentrue}}{\extS{S''}{l}{d_1}{\frozentrue}} \\
        &= \lubstore{\lubstore{S}{\store{\storebinding{l}{d_1}{\frozentrue}}}}{\lubstore{S''}{\store{\storebinding{l}{d_1}{\frozentrue}}}} \\
        &= \lubstore{\lubstore{S}{\store{\storebinding{l}{d_1}{\frozentrue}}}}{S''} \\
        &= \lubstore{\extS{S}{l}{d_1}{\frozentrue}}{S''}.
      \end{align*}
      Therefore

      $\config{\lubstore{S}{S''}}{\freezeafterfull{l}{Q}{\lam{x}{e_0}}{\setof{v,
            \dots}}{H}} \parstepsto
      \config{\lubstore{\extS{S}{l}{d_1}{\frozentrue}}{S''}}{d_1}$,

      as we were required to show.

    \item Case {\sc E-Freeze-Simple}:

      Given: $\config{S}{\freeze{l}} \parstepsto
      \config{\extS{S}{l}{d_1}{\frozentrue}}{d_1}$.

      To show: $\config{\lubstore{S}{S''}}{\freeze{l}}
      \parstepsto
      \config{\lubstore{\extS{S}{l}{d_1}{\frozentrue}}{S''}}{d_1}$.

      We will first show that

      $\config{\lubstore{S}{S''}}{\freeze{l}} \parstepsto
      \config{\extS{(\lubstore{S}{S''})}{l}{d_1}{\frozentrue}}{d_1}$

      and then show why this is sufficient.

      We proceed by cases on $l$:
      \begin{itemize}
      \item $l \notin \dom{S''}$:

        By assumption, $\lubstore{\extS{S}{l}{d_1}{\frozentrue}}{S''}
        \neq \topS$.

        By Lemma~\ref{lem:monotonicity},
        $\leqstore{S}{\extS{S}{l}{d_1}{\frozentrue}}$.

        Therefore $\lubstore{S}{S''} \neq \topS$.

        Hence, by Definition~\ref{def:lubstore},
        $(\lubstore{S}{S''})(l) = S(l)$.

        From the premise of {\sc E-Freeze-Simple}, we have that
        $S(l) = \state{d_1}{\status_1}$.

        Therefore, by {\sc E-Freeze-Simple}, we have that

        $\config{\lubstore{S}{S''}}{\freeze{l}}
        \parstepsto
        \config{\extS{(\lubstore{S}{S''})}{l}{d_1}{\frozentrue}}{d_1}$.

      \item $l \in \dom{S''}$:

        By assumption, $\lubstore{\extS{S}{l}{d_1}{\frozentrue}}{S''}
        \neq \topS$.

        By Lemma~\ref{lem:monotonicity},
        $\leqstore{S}{\extS{S}{l}{d_1}{\frozentrue}}$.

        Therefore $\lubstore{S}{S''} \neq \topS$.

        Hence, by Definition~\ref{def:lubstore},
        $(\lubstore{S}{S''})(l) = \lubp{S(l)}{S''(l)}$.

        From the premise of {\sc E-Freeze-Simple}, we have that
        $S(l) = \state{d_1}{\status_1}$.

        By assumption, $\lubstore{\extS{S}{l}{d_1}{\frozentrue}}{S''}
        \statuseq S$.

        Therefore $\status_1 = \frozentrue$.

        Therefore $(\lubstore{S}{S''})(l) =
        \lubp{\state{d_1}{\frozentrue}}{S''(l)}$.
        
        We proceed by cases on $S''(l)$:
        \begin{itemize}
        \item $S''(l) = \state{d_2}{\frozenfalse}$, where $d_2 \userleq d_1$:

          By Definition~\ref{def:lubp},
          $\lubp{\state{d_1}{\frozentrue}}{\state{d_2}{\frozenfalse}} =
          \state{d_1}{\frozentrue}$.

          Therefore $(\lubstore{S}{S''})(l) =
          \state{d_1}{\frozentrue}$.

          Therefore, by {\sc E-Freeze-Simple}, we have that

          $\config{\lubstore{S}{S''}}{\freeze{l}}
          \parstepsto
          \config{\extS{(\lubstore{S}{S''})}{l}{d_1}{\frozentrue}}{d_1}$.

        \item $S''(l) = \state{d_2}{\frozenfalse}$, where $d_2 \nuserleq d_1$:

          By Definition~\ref{def:lubp},
          $\lubp{\state{d_1}{\frozentrue}}{\state{d_2}{\frozenfalse}}
          = \state{\top}{\frozenfalse}$.

          By Definition~\ref{def:lattice-with-status-bits},
          $\state{\top}{\frozenfalse} = \topp$.

          Therefore $\lubp{S(l)}{S''(l)} = \topp$.

          Therefore, by Definition~\ref{def:lubstore},
          $\lubstore{S}{S''} = \topS$.

          This is a contradiction.

          Therefore,

          $\config{\lubstore{S}{S''}}{\freeze{l}}
          \parstepsto
          \config{\extS{(\lubstore{S}{S''})}{l}{d_1}{\frozentrue}}{d_1}$.

        \item $S''(l) = \state{d_2}{\frozentrue}$, where $d_2 = d_1$:

          Therefore $(\lubstore{S}{S''})(l) =
          \lubp{\state{d_1}{\frozentrue}}{\state{d_2}{\frozentrue}}$.

          By Definition~\ref{def:lubp},
          $\lubp{\state{d_1}{\frozentrue}}{\state{d_2}{\frozentrue}} =
          \state{d_1}{\frozentrue}$.

          Therefore $(\lubstore{S}{S''})(l) =
          \state{d_1}{\frozentrue}$.

          Therefore, by {\sc E-Freeze-Simple}, we have that

          $\config{\lubstore{S}{S''}}{\freeze{l}}
          \parstepsto
          \config{\extS{(\lubstore{S}{S''})}{l}{d_1}{\frozentrue}}{d_1}$.

        \item $S''(l) = \state{d_2}{\frozentrue}$, where $d_2 \neq d_1$:

          By Definition~\ref{def:lubp},
          $\lubp{\state{d_1}{\frozentrue}}{\state{d_2}{\frozentrue}}
          = \state{\top}{\frozenfalse}$.

          By Definition~\ref{def:lattice-with-status-bits},
          $\state{\top}{\frozenfalse} = \topp$.

          Therefore $\lubp{S(l)}{S''(l)} = \topp$.

          Therefore, by Definition~\ref{def:lubstore},
          $\lubstore{S}{S''} = \topS$.

          This is a contradiction.

          Therefore,

          $\config{\lubstore{S}{S''}}{\freeze{l}}
          \parstepsto
          \config{\extS{(\lubstore{S}{S''})}{l}{d_1}{\frozentrue}}{d_1}$.
        \end{itemize}
      \end{itemize}

      In each case we have shown that

      $\config{\lubstore{S}{S''}}{\freeze{l}} \parstepsto
      \config{\extS{(\lubstore{S}{S''})}{l}{d_1}{\frozentrue}}{d_1}$.

      Note that:
      \begin{align*}
        \extS{(\lubstore{S}{S''})}{l}{d_1}{\frozentrue} &=
        \lubstore{\extS{S}{l}{d_1}{\frozentrue}}{\extS{S''}{l}{d_1}{\frozentrue}} \\
        &= \lubstore{\lubstore{S}{\store{\storebinding{l}{d_1}{\frozentrue}}}}{\lubstore{S''}{\store{\storebinding{l}{d_1}{\frozentrue}}}} \\
        &= \lubstore{\lubstore{S}{\store{\storebinding{l}{d_1}{\frozentrue}}}}{S''} \\
        &= \lubstore{\extS{S}{l}{d_1}{\frozentrue}}{S''}.
      \end{align*}
      Therefore
      $\config{\lubstore{S}{S''}}{\freeze{l}}
      \parstepsto
      \config{\lubstore{\extS{S}{l}{d_1}{\frozentrue}}{S''}}{d_1}$,
      as we were required to show.
  \end{itemize}
\end{proof}


\section{Proof of Lemma~\ref{lem:strong-local-quasi-confluence}}\label{section:strong-local-quasi-confluence-proof}
\begin{proof}
  Suppose $\conf \ctxstepsto \conf_a$ and $\conf \ctxstepsto \conf_b$.

  We have to show that either there exist $\conf_c, i, j, \pi$ such
  that $\conf_a \ctxstepsto^i \conf_c$ and $\pi(\conf_b) \ctxstepsto^j
  \conf_c$ and $i \leq 1$ and $j \leq 1$, or that $\conf_a \ctxstepsto
  \error$ or $\conf_b \ctxstepsto \error$.

  By inspection of the operational semantics, it must be the case that
  $\conf$ steps to $\conf_a$ by the {\sc E-Eval-Ctxt} rule.

  Let $\conf = \config{S}{\evalctxt{E_a}{e_{a_1}}}$ and let $\conf_a =
  \config{S_a}{\evalctxt{E_a}{e_{a_2}}}$.

  Likewise, it must be the case that $\conf$ steps to $\conf_b$ by the
  {\sc E-Eval-Ctxt} rule.

  Let $\conf = \config{S}{\evalctxt{E_b}{e_{b_1}}}$ and let $\conf_b =
  \config{S_b}{\evalctxt{E_b}{e_{b_2}}}$.

  Note that $\conf = \config{S}{\evalctxt{E_a}{e_{a_1}}} =
  \config{S}{\evalctxt{E_b}{e_{b_1}}}$, and so
  $\evalctxt{E_a}{e_{a_1}} = \evalctxt{E_b}{e_{b_1}}$, but $E_a$ and
  $E_b$ may differ and $e_{a_1}$ and $e_{b_1}$ may differ.

  First, consider the possibility that $E_a = E_b$ (and $e_{a_1} =
  e_{b_1}$).

  Since $\config{S}{\evalctxt{E_a}{e_{a_1}}} \ctxstepsto
  \config{S_a}{\evalctxt{E_a}{e_{a_2}}}$ by {\sc E-Eval-Ctxt} and
  $\config{S}{\evalctxt{E_b}{e_{b_1}}} \ctxstepsto
  \config{S_b}{\evalctxt{E_b}{e_{b_2}}}$ by {\sc E-Eval-Ctxt}, we have
  from the premise of {\sc E-Eval-Ctxt} that $\config{S}{e_{a_1}}
  \parstepsto \config{S_a}{e_{a_2}}$ and $\config{S}{e_{b_1}}
  \parstepsto \config{S_b}{e_{b_2}}$.

  But then, since $e_{a_1} = e_{b_1}$, by Internal Determinism
  (Lemma~\ref{lem:internal-determinism}) there is a permutation $\pi'$
  such that $\config{S_a}{e_{a_2}} = \pi'(\config{S_b}{e_{b_2}})$,
  modulo choice of events.

  We have two cases:

  \begin{itemize}
  \item In the case where the steps $\conf \ctxstepsto \conf_a$ and
    $\conf \ctxstepsto \conf_b$ are both by {\sc E-Spawn-Handler} and
    they handle different events $d_2$ and $d'_2$, then we can satisfy
    the proof by choosing the final configuration $\conf_c$ as the
    configuration where both $d_2$ and $d'_2$ have been handled.

    Both $\conf_a$ and $\conf_b$ can step to this configuration by
    {\sc E-Spawn-Handler}: if the step from $\conf$ to $\conf_a$
    handles $d_2$ then the step from $\conf_a$ to $\conf_c$ handles
    $d'_2$, while if the step from $\conf$ to $\conf_b$ handles $d'_2$
    then the step from $\conf_b$ to $\conf_c$ handles $d_2$.

    The store in the final configuration is $S_a$ or $S_b$, which are
    equal because {\sc E-Spawn-Handler} does not affect the store, and
    we can satisfy the proof by choosing $i = 1$ and $j = 0$ and $\pi
    = \id$.

  \item Otherwise, we can satisfy the proof by choosing $\conf_c =
    \config{S_a}{e_{a_2}}$ and $i = 0$ and $j = 0$ and $\pi = \id$.
  \end{itemize}

  The rest of this proof deals with the more interesting case in which
  $E_a \neq E_b$ (and $e_{a_1} \neq e_{b_1}$).

  Since $\config{S}{\evalctxt{E_a}{e_{a_1}}} \ctxstepsto
  \config{S_a}{\evalctxt{E_a}{e_{a_2}}}$ and
  $\config{S}{\evalctxt{E_b}{e_{b_1}}} \ctxstepsto
  \config{S_b}{\evalctxt{E_b}{e_{b_2}}}$ and $\evalctxt{E_a}{e_{a_1}}
  = \evalctxt{E_b}{e_{b_1}}$, and since $E_a \neq E_b$, we have from
  Lemma~\ref{lem:locality} (Locality) that there exist evaluation
  contexts $E'_a$ and $E'_b$ such that:

  \begin{itemize}
  \item $\evalctxt{E'_a}{e_{a_1}} = \evalctxt{E_b}{e_{b_2}}$, and
  \item $\evalctxt{E'_b}{e_{b_1}} = \evalctxt{E_a}{e_{a_2}}$, and
  \item $\evalctxt{E'_a}{e_{a_2}} =
    \evalctxt{E'_b}{e_{b_2}}$.
  \end{itemize}

  In some of the cases that follow, we will choose $\conf_c = \error$,
  and in some we will prove that one of $\conf_a$ or $\conf_b$ steps
  to $\error$.

  In most cases, however, our approach will be to show that there
  exist $S', i, j, \pi$ such that:
  \begin{itemize}
  \item $\config{S_a}{\evalctxt{E_a}{e_{a_2}}} \ctxstepsto^i
    \config{S'}{\evalctxt{E'_a}{e_{a_2}}}$, and
  \item $\pi(\config{S_b}{\evalctxt{E_b}{e_{b_2}}}) \ctxstepsto^j
    \config{S'}{\evalctxt{E'_a}{e_{a_2}}}$.
  \end{itemize}
  Since $\evalctxt{E'_a}{e_{a_1}} = \evalctxt{E_b}{e_{b_2}}$,
  $\evalctxt{E'_b}{e_{b_1}} = \evalctxt{E_a}{e_{a_2}}$, and
  $\evalctxt{E'_a}{e_{a_2}} = \evalctxt{E'_b}{e_{b_2}}$, it suffices
  to show that:
  \begin{itemize}
  \item $\config{S_a}{\evalctxt{E'_b}{e_{b_1}}} \ctxstepsto^i
    \config{S'}{\evalctxt{E'_b}{e_{b_2}}}$, and
  \item $\pi(\config{S_b}{\evalctxt{E'_a}{e_{a_1}}}) \ctxstepsto^j
    \config{S'}{\evalctxt{E'_a}{e_{a_2}}}$.
  \end{itemize}
  From the premise of {\sc E-Eval-Ctxt}, we have that
  $\config{S}{e_{a_1}} \parstepsto \config{S_a}{e_{a_2}}$ and
  $\config{S}{e_{b_1}} \parstepsto \config{S_b}{e_{b_2}}$.

  We proceed by case analysis on the rule by which
  $\config{S}{e_{a_1}}$ steps to $\config{S_a}{e_{a_2}}$.

  Since the only way an $\error$ configuration can arise is by the
  {\sc E-Put-Err} rule, we can assume in all other cases that $\conf_a
  \neq \error$.

  \begin{enumerate}
  \item Case {\sc E-Beta}: We have $S_a = S$.

    We proceed by case analysis on the rule by which
    $\config{S}{e_{b_1}}$ steps to $\config{S_b}{e_{b_2}}$.

    Since the only way an $\error$ configuration can arise is by the
    {\sc E-Put-Err} rule, we can assume in all other cases that
    $\conf_b \neq \error$.
    \begin{enumerate}
    \item \label{slqc-beta-beta}Case {\sc E-Beta}: We have $S_a = S$
      and $S_b = S$.

      Choose $S' = S = S_a = S_b$, $i = 1$, $j = 1$, and $\pi = \id$.

      We have to show that:
      \begin{itemize}
      \item $\config{S}{\evalctxt{E'_b}{e_{b_1}}} \ctxstepsto
        \config{S_a}{\evalctxt{E'_b}{e_{b_2}}}$, and
      \item $\config{S}{\evalctxt{E'_a}{e_{a_1}}} \ctxstepsto
        \config{S_b}{\evalctxt{E'_a}{e_{a_2}}}$, 
      \end{itemize}

      both of which follow immediately from $\config{S}{e_{a_1}}
      \parstepsto \config{S_a}{e_{a_2}}$ and $\config{S}{e_{b_1}}
      \parstepsto \config{S_b}{e_{b_2}}$ and {\sc E-Eval-Ctxt}.

    \item \label{slqc-beta-new}Case {\sc E-New}: We have $S_a = S$ and
      $S_b = \extS{S}{l}{\bot}{\frozenfalse}$.

      Choose $S' = S_b$, $i = 1$, $j = 1$, and $\pi = \id$.

      We have to show that:
      \begin{itemize}
      \item $\config{S}{\evalctxt{E'_b}{e_{b_1}}} \ctxstepsto
        \config{S_b}{\evalctxt{E'_b}{e_{b_2}}}$, and
      \item $\config{S_b}{\evalctxt{E'_a}{e_{a_1}}} \ctxstepsto
        \config{S_b}{\evalctxt{E'_a}{e_{a_2}}}$.
      \end{itemize}

      The first of these follows immediately from $\config{S}{e_{b_1}}
      \parstepsto \config{S_b}{e_{b_2}}$ and {\sc E-Eval-Ctxt}.

      For the second, consider that $S_b =
      \extS{S}{l}{\bot}{\frozenfalse} = U_S(S)$, where $U_S$ is the
      store update operation that acts as the identity on the contents
      of all existing locations, and adds the binding
      $\storebinding{l}{\bot}{\frozenfalse}$ if no binding for $l$
      exists.

      Note that:
      \begin{itemize}
      \item $U_S$ is non-conflicting with $\config{S}{e_{a_1}}
        \parstepsto \config{S_a}{e_{a_2}}$, since no locations are
        allocated in the transition;
      \item $U_S(S_a) \neq \topS$, since $U_S(S_a) = U_S(S) = S_b$
        and we know that $\conf_b \neq \error$; and
      \item $U_S$ is freeze-safe with $\config{S}{e_{a_1}}
        \parstepsto \config{S_a}{e_{a_2}}$, since $S_a = S$, so
        there are no locations whose contents differ in status
        between them.
      \end{itemize}

      Therefore, by Lemma~\ref{lem:generalized-independence}
      (Generalized Independence), we have that

      $\config{U_S(S)}{e_{a_1}} \parstepsto
      \config{U_S(S_a)}{e_{a_2}}$.

      Hence $\config{S_b}{e_{a_1}} \parstepsto \config{S_b}{e_{a_2}}$.

      By {\sc E-Eval-Ctxt}, it follows that
      $\config{S_b}{\evalctxt{E'_a}{e_{a_1}}} \ctxstepsto
      \config{S_b}{\evalctxt{E'_a}{e_{a_2}}}$,
      as we were required to show.

    \item \label{slqc-beta-put}Case {\sc E-Put}: We have $S_a = S$ and
      $S_b = \extSRaw{S}{l}{u_{p_i}(p_1)}$.

      Choose $S' = S_b$, $i = 1$, $j = 1$, and $\pi = \id$.

      We have to show that:
      \begin{itemize}
      \item $\config{S}{\evalctxt{E'_b}{e_{b_1}}} \ctxstepsto
        \config{S_b}{\evalctxt{E'_b}{e_{b_2}}}$, and
      \item $\config{S_b}{\evalctxt{E'_a}{e_{a_1}}} \ctxstepsto
        \config{S_b}{\evalctxt{E'_a}{e_{a_2}}}$.
      \end{itemize}

      The first of these follows immediately from $\config{S}{e_{b_1}}
      \parstepsto \config{S_b}{e_{b_2}}$ and {\sc E-Eval-Ctxt}.

      For the second, consider that $S_b = U_S(S)$, where $U_S$ is the
      store update operation that applies $u_{p_i}$ to the contents of
      $l$ and acts as the identity on all other locations.

      Note that:
      \begin{itemize}
      \item $U_S$ is non-conflicting with $\config{S}{e_{a_1}}
        \parstepsto \config{S_a}{e_{a_2}}$, since no locations are
        allocated in the transition;
      \item $U_S(S_a) \neq \topS$, since $U_S(S_a) = U_S(S) = S_b$
        and we know that $\conf_b \neq \error$; and
      \item $U_S$ is freeze-safe with $\config{S}{e_{a_1}}
        \parstepsto \config{S_a}{e_{a_2}}$, since $S_a = S$, so
        there are no locations whose contents differ in status
        between them.
      \end{itemize}

      Therefore, by Lemma~\ref{lem:generalized-independence}
      (Generalized Independence), we have that

      $\config{U_S(S)}{e_{a_1}} \parstepsto
      \config{U_S(S_a)}{e_{a_2}}$.

      Hence $\config{S_b}{e_{a_1}} \parstepsto \config{S_b}{e_{a_2}}$.

      By {\sc E-Eval-Ctxt}, it follows that
      $\config{S_b}{\evalctxt{E'_a}{e_{a_1}}} \ctxstepsto
      \config{S_b}{\evalctxt{E'_a}{e_{a_2}}}$, as we were required to
      show.

    \item \label{slqc-beta-put-err}Case {\sc E-Put-Err}: We have $S_a
      = S$ and $\config{S_b}{e_{b_2}} = \error$, and so we choose
      $\conf_c = \error$, $i = 1$, $j = 0$, and $\pi = \id$.

      We have to show that:
      \begin{itemize}
      \item $\config{S}{\evalctxt{E'_b}{e_{b_1}}} \ctxstepsto \error$,
        and
      \item $\config{S_b}{\evalctxt{E'_a}{e_{a_1}}} = \error$.
      \end{itemize}

      The second of these is immediately true because since
      $\config{S_b}{e_{b_2}} = \error$, $S_b = \topS$, and so
      $\config{S_b}{\evalctxt{E'_a}{e_{a_1}}}$ is equal to $\error$ as
      well.

      For the first, observe that $\config{S}{e_{b_1}} \parstepsto
      \config{S_b}{e_{b_2}}$, hence by {\sc E-Eval-Ctxt},
      $\config{S}{\evalctxt{E'_b}{e_{b_1}}} \ctxstepsto
      \config{S_b}{\evalctxt{E'_b}{e_{b_2}}}$.

      But $S_b = \topS$, so $\config{S_b}{\evalctxt{E'_b}{e_{b_2}}}$
      is equal to $\error$, and so
      $\config{S}{\evalctxt{E'_b}{e_{b_1}}} \ctxstepsto \error$, as
      required.

    \item \label{slqc-beta-get}Case {\sc E-Get}: Similar to
      case~\ref{slqc-beta-beta}, since $S_a = S$ and $S_b = S$.
    \item \label{slqc-beta-freeze-init}Case {\sc E-Freeze-Init}:
      Similar to case~\ref{slqc-beta-beta}, since $S_a = S$ and $S_b =
      S$.
    \item \label{slqc-beta-spawn-handler}Case {\sc E-Spawn-Handler}:
      Similar to case~\ref{slqc-beta-beta}, since $S_a = S$ and $S_b =
      S$.
    \item \label{slqc-beta-freeze-final}Case {\sc E-Freeze-Final}: We
      have $S_a = S$ and $S_b = \extS{S}{l}{d_1}{\frozentrue}$.

      Choose $S' = S_b$, $i = 1$, $j = 1$, and $\pi = \id$.

      We have to show that:
      \begin{itemize}
      \item $\config{S}{\evalctxt{E'_b}{e_{b_1}}} \ctxstepsto
        \config{S_b}{\evalctxt{E'_b}{e_{b_2}}}$, and
      \item $\config{S_b}{\evalctxt{E'_a}{e_{a_1}}} \ctxstepsto
        \config{S_b}{\evalctxt{E'_a}{e_{a_2}}}$.
      \end{itemize}

      The first of these follows immediately from $\config{S}{e_{b_1}}
      \parstepsto \config{S_b}{e_{b_2}}$ and {\sc E-Eval-Ctxt}.

      For the second, note that $S_b = U_S(S)$, where $U_S$ is the
      store update operation that freezes the contents of $l$ and acts
      as the identity on the contents of all other locations.

      Note that:
      \begin{itemize}
      \item $U_S$ is non-conflicting with $\config{S}{e_{a_1}}
        \parstepsto \config{S_a}{e_{a_2}}$, since no locations are
        allocated in the transition;
      \item $U_S(S_a) \neq \topS$, since $U_S(S_a) = U_S(S) = S_b$
        and we know that $\conf_b \neq \error$; and
      \item $U_S$ is freeze-safe with $\config{S}{e_{a_1}}
        \parstepsto \config{S_a}{e_{a_2}}$, since $S_a = S$, so
        there are no locations whose contents differ in status
        between them.
      \end{itemize}

      Therefore, by Lemma~\ref{lem:generalized-independence}
      (Generalized Independence), we have that

      $\config{U_S(S)}{e_{a_1}} \parstepsto
      \config{U_S(S_a)}{e_{a_2}}$.

      Hence $\config{S_b}{e_{a_1}} \parstepsto \config{S_b}{e_{a_2}}$.

      By {\sc E-Eval-Ctxt}, it follows that
      $\config{S_b}{\evalctxt{E'_a}{e_{a_1}}} \ctxstepsto
      \config{S_b}{\evalctxt{E'_a}{e_{a_2}}}$, as we were required to
      show.

    \item \label{slqc-beta-freeze-simple}Case {\sc E-Freeze-Simple}:
      Similar to case~\ref{slqc-beta-freeze-final}, since $S_b =
      \extS{S}{l}{d_1}{\frozentrue}$.

    \end{enumerate}
  \item Case {\sc E-New}: We have $S_a = \extS{S}{l}{\bot}{\frozenfalse}$.

    We proceed by case analysis on the rule by which
    $\config{S}{e_{b_1}}$ steps to $\config{S_b}{e_{b_2}}$.

    Since the only way an $\error$ configuration can arise is by the
    {\sc E-Put-Err} rule, we can assume in all other cases that
    $\conf_b \neq \error$.
    \begin{enumerate}
    \item \label{slqc-new-beta}Case {\sc E-Beta}: By symmetry with case~\ref{slqc-beta-new}.
    \item \label{slqc-new-new}Case {\sc E-New}: We have $S_a =
      \extS{S}{l}{\bot}{\frozenfalse}$ and $S_b =
      \extS{S}{l'}{\bot}{\frozenfalse}$.

      Now consider whether $l = l'$:
      \begin{itemize}
      \item If $l \neq l'$:

        Choose $S' =
        \extS{\extS{S}{l'}{\bot}{\frozenfalse}}{l}{\bot}{\frozenfalse}$,
        $i = 1$, $j = 1$, and $\pi = \id$.

        We have to show that:
        \begin{itemize}
        \item $\config{S_a}{\evalctxt{E'_b}{e_{b_1}}} \ctxstepsto
          \config{\extS{\extS{S}{l'}{\bot}{\frozenfalse}}{l}{\bot}{\frozenfalse}}{\evalctxt{E'_b}{e_{b_2}}}$,
          and
        \item $\config{S_b}{\evalctxt{E'_a}{e_{a_1}}} \ctxstepsto
          \config{\extS{\extS{S}{l'}{\bot}{\frozenfalse}}{l}{\bot}{\frozenfalse}}{\evalctxt{E'_a}{e_{a_2}}}$.
        \end{itemize}

        For the first of these, consider that $S_a =
        \extS{S}{l}{\bot}{\frozenfalse} = U_S(S)$, where $U_S$ is
        the store update operation that acts as the identity on the
        contents of all existing locations, and adds the binding
        $\storebinding{l}{\bot}{\frozenfalse}$ if no binding for $l$
        exists.

        Note that:
        \begin{itemize}
        \item $U_S$ is non-conflicting with $\config{S}{e_{b_1}}
          \parstepsto \config{S_b}{e_{b_2}}$, since the only
          location allocated in the transition is $l'$, and $l
          \neq l'$ in this case;
        \item $U_S(S_b) \neq \topS$, since $U_S(S_b) =
          \extS{\extS{S}{l'}{\bot}{\frozenfalse}}{l}{\bot}{\frozenfalse}$
          and we know $S \neq \topS$ and the addition of new
          bindings $\storebinding{l}{\bot}{\frozenfalse}$ and
          $\storebinding{l'}{\bot}{\frozenfalse}$ cannot cause it to
          become $\topS$; and
        \item $U_S$ is freeze-safe with $\config{S}{e_{b_1}}
          \parstepsto \config{S_b}{e_{b_2}}$, since $S_b =
          \extS{S}{l'}{\bot}{\frozenfalse}$ and $l' \notin \dom{S}$,
          so there are no locations whose contents differ in status
          between $S$ and $S_b$.
        \end{itemize}

        Therefore, by Lemma~\ref{lem:generalized-independence}
        (Generalized Independence), we have that

        $\config{U_S(S)}{e_{b_1}} \parstepsto
        \config{U_S(S_b)}{e_{b_2}}$.

        Hence $\config{\extS{S}{l}{\bot}{\frozenfalse}}{e_{b_1}}
        \parstepsto
        \config{\extS{S_b}{l}{\bot}{\frozenfalse}}{e_{b_2}}$.

        By {\sc E-Eval-Ctxt} it follows that

        $\config{\extS{S}{l}{\bot}{\frozenfalse}}{\evalctxt{E'_b}{e_{b_1}}}
        \parstepsto
        \config{\extS{S_b}{l}{\bot}{\frozenfalse}}{\evalctxt{E'_b}{e_{b_2}}}$,
        which, since $S_b = \extS{S}{l'}{\bot}{\frozenfalse}$, is what
        we were required to show.

        The argument for the second is symmetrical.

      \item If $l = l'$:

        In this case, observe that we do \emph{not} want the
        expression in the final configuration to be
        $\evalctxt{E'_a}{e_{a_2}}$ (nor its equivalent,
        $\evalctxt{E'_b}{e_{b_2}}$).

        The reason for this is that $\evalctxt{E'_a}{e_{a_2}}$
        contains both occurrences of $l$.

        Rather, we want both configurations to step to a configuration
        in which exactly one occurrence of $l$ has been renamed to a
        fresh location $l''$.

        Let $l''$ be a location such that $l'' \notin \dom{S}$ and
        $l'' \neq l$ (and hence $l'' \neq l'$, as well).

        Then choose $S' =
        \extS{\extS{S}{l''}{\bot}{\frozenfalse}}{l}{\bot}{\frozenfalse}$,
        $i = 1$, $j = 1$, and $\pi = \setof{(l, l'')}$.

        Either
        $\config{\extS{\extS{S}{l''}{\bot}{\frozenfalse}}{l}{\bot}{\frozenfalse}}{\evalctxt{E'_a}{\pi(e_{a_2})}}$
        or
        $\config{\extS{\extS{S}{l''}{\bot}{\frozenfalse}}{l}{\bot}{\frozenfalse}}{\evalctxt{E'_b}{\pi(e_{b_2})}}$
        would work as a final configuration; we choose

        $\config{\extS{\extS{S}{l''}{\bot}{\frozenfalse}}{l}{\bot}{\frozenfalse}}{\evalctxt{E'_b}{\pi(e_{b_2})}}$.

        We have to show that:
        \begin{itemize}
        \item $\config{S_a}{\evalctxt{E'_b}{e_{b_1}}} \ctxstepsto
          \config{\extS{\extS{S}{l''}{\bot}{\frozenfalse}}{l}{\bot}{\frozenfalse}}{\evalctxt{E'_b}{\pi(e_{b_2})}}$,
          and
        \item $\pi(\config{S_b}{\evalctxt{E'_a}{e_{a_1}}})
          \ctxstepsto
          \config{\extS{\extS{S}{l''}{\bot}{\frozenfalse}}{l}{\bot}{\frozenfalse}}{\evalctxt{E'_b}{\pi(e_{b_2})}}$.
        \end{itemize}

        For the first of these, since $\config{S}{e_{b_1}}
        \parstepsto \config{S_b}{e_{b_2}}$, we have by
        Lemma~\ref{lem:permutability} (Permutability) that
        $\pi(\config{S}{e_{b_1}}) \parstepsto
        \pi(\config{S_b}{e_{b_2}})$.

        Since $\pi = \setof{(l, l'')}$, but $l \notin S$ (from the
        side condition on {\sc E-New}), we have that
        $\pi(\config{S}{e_{b_1}}) = \config{S}{e_{b_1}}$.

        Since $\config{S_b}{e_{b_2}} =
        \config{\extS{S}{l'}{\bot}{\frozenfalse}}{l'}$, and $l = l'$,
        we have that $\pi(\config{S_b}{e_{b_2}}) =
        \config{\extS{S}{l''}{\bot}{\frozenfalse}}{\pi(e_{b_2})}$.

        Hence $\config{S}{e_{b_1}} \parstepsto
        \config{\extS{S}{l''}{\bot}{\frozenfalse}}{\pi(e_{b_2})}$.

        Let $U_S$ be the store update operation that acts as the
        identity on the contents of all existing locations, and adds
        the binding $\storebinding{l}{\bot}{\frozenfalse}$ if no
        binding for $l$ exists.

        Note that:
        \begin{itemize}
        \item $U_S$ is non-conflicting with $\config{S}{e_{b_1}}
          \parstepsto
          \config{\extS{S}{l''}{\bot}{\frozenfalse}}{\pi(e_{b_2})}$,
          since the only location allocated in the transition is
          $l''$;
        \item $U_S(\extS{S}{l''}{\bot}{\frozenfalse}) \neq \topS$,
          since $U_S(\extS{S}{l''}{\bot}{\frozenfalse}) = \\
          \extS{\extS{S}{l''}{\bot}{\frozenfalse}}{l}{\bot}{\frozenfalse}$
          and we know $S \neq \topS$ and the addition of new
          bindings $\storebinding{l}{\bot}{\frozenfalse}$ and
          $\storebinding{l''}{\bot}{\frozenfalse}$ cannot cause it
          to become $\topS$; and
        \item $U_S$ is freeze-safe with $\config{S}{e_{b_1}}
          \parstepsto
          \config{\extS{S}{l''}{\bot}{\frozenfalse}}{\pi(e_{b_2})}$,
          since $l'' \notin \dom{S}$, so there are no locations
          whose contents differ in status between $S$ and
          $\extS{S}{l''}{\bot}{\frozenfalse}$.
        \end{itemize}

        Therefore, by Lemma~\ref{lem:generalized-independence}
        (Generalized Independence), we have that

        $\config{U_S(S)}{e_{b_1}} \parstepsto
        \config{U_S(\extS{S}{l''}{\bot}{\frozenfalse})}{\pi(e_{b_2})}$.

        Hence $\config{\extS{S}{l}{\bot}{\frozenfalse}}{e_{b_1}}
        \parstepsto
        \config{\extS{\extS{S}{l''}{\bot}{\frozenfalse}}{l}{\bot}{\frozenfalse}}{\pi(e_{b_2})}$.

        By {\sc E-Eval-Ctxt} it follows that

        $\config{\extS{S}{l}{\bot}{\frozenfalse}}{\evalctxt{E'_b}{e_{b_1}}}
        \parstepsto
        \config{\extS{\extS{S}{l''}{\bot}{\frozenfalse}}{l}{\bot}{\frozenfalse}}{\evalctxt{E'_b}{\pi(e_{b_2})}}$,

        which, since $\extS{S}{l}{\bot}{\frozenfalse} = S_a$, is what
        we were required to show.

        For the second, observe that since $S_b =
        \extS{S}{l}{\bot}{\frozenfalse}$, we have that $\pi(S_b) =
        \extS{S}{l''}{\bot}{\frozenfalse}$.

        Also, since $l$ does not occur in $e_{a_1}$, we have that
        $\pi(\evalctxt{E'_a}{e_{a_1}}) =
        \evalctxt{(\pi(E'_a))}{e_{a_1}}$.

        Hence we have to show that

        $\config{\extS{S}{l''}{\bot}{\frozenfalse}}{\evalctxt{(\pi(E'_a))}{e_{a_1}}}
        \ctxstepsto \\
        \config{\extS{\extS{S}{l''}{\bot}{\frozenfalse}}{l}{\bot}{\frozenfalse}}{\evalctxt{E'_b}{\pi(e_{b_2})}}$.

        Let $U_S$ be the store update operation that acts as the
        identity on the contents of all existing locations, and adds
        the binding $\storebinding{l''}{\bot}{\frozenfalse}$ if no
        binding for $l''$ exists.

        Note that:
        \begin{itemize}
        \item $U_S$ is non-conflicting with $\config{S}{e_{a_1}}
          \parstepsto \config{S_a}{e_{a_2}}$, since the only
          location allocated in the transition is $l$;
        \item $U_S(S_a) \neq \topS$, since $U_S(S_a) =
          \extS{\extS{S}{l''}{\bot}{\frozenfalse}}{l}{\bot}{\frozenfalse}$
          and we know $S \neq \topS$ and the addition of new
          bindings $\storebinding{l}{\bot}{\frozenfalse}$ and
          $\storebinding{l''}{\bot}{\frozenfalse}$ cannot cause it
          to become $\topS$; and
        \item $U_S$ is freeze-safe with $\config{S}{e_{a_1}}
          \parstepsto \config{S_a}{e_{a_2}}$, since $S_a =
          \extS{S}{l}{\bot}{\frozenfalse}$ and $l \notin \dom{S}$,
          so there are no locations whose contents differ in status
          between $S$ and $S_a$.
        \end{itemize}

        Therefore, by Lemma~\ref{lem:generalized-independence}
        (Generalized Independence), we have that

        $\config{U_S(S)}{e_{a_1}} \parstepsto
        \config{U_S(S_a)}{e_{a_2}}$.

        Hence $\config{\extS{S}{l''}{\bot}{\frozenfalse}}{e_{a_1}}
        \parstepsto
        \config{\extS{\extS{S}{l''}{\bot}{\frozenfalse}}{l}{\bot}{\frozenfalse}}{e_{a_2}}$.

        By {\sc E-Eval-Ctxt} it follows that
        
        $\config{\extS{S}{l''}{\bot}{\frozenfalse}}{\evalctxt{(\pi(E'_a))}{e_{a_1}}}
        \ctxstepsto \\
        \config{\extS{\extS{S}{l''}{\bot}{\frozenfalse}}{l}{\bot}{\frozenfalse}}{\evalctxt{(\pi(E'_a))}{e_{a_2}}}$,

        which completes the case since $\evalctxt{E'_b}{\pi(e_{b_2})}
        = \evalctxt{(\pi(E'_a))}{e_{a_2}}$.

        \lk{This assumes that you believe that
          $\evalctxt{E'_b}{\pi(e_{b_2})} =
          \evalctxt{(\pi(E'_a))}{e_{a_2}}$.}

      \end{itemize}

    \item \label{slqc-new-put}Case {\sc E-Put}: We have $S_a =
      \extS{S}{l}{\bot}{\frozenfalse}$ and $S_b =
      \extSRaw{S}{l'}{u_{p_i}(p_1)}$, where $l \neq l'$ (since $l
      \notin \dom{S}$, but $l' \in \dom{S}$).

      We have to show that:
      \begin{itemize}
      \item $\config{S_a}{\evalctxt{E'_b}{e_{b_1}}} \ctxstepsto
        \config{\extS{S_b}{l}{\bot}{\frozenfalse}}{\evalctxt{E'_b}{e_{b_2}}}$,
        and
      \item $\config{S_b}{\evalctxt{E'_a}{e_{a_1}}} \ctxstepsto
        \config{\extS{S_b}{l}{\bot}{\frozenfalse}}{\evalctxt{E'_a}{e_{a_2}}}$.
      \end{itemize}

      For the first of these, consider that $S_a =
      \extS{S}{l}{\bot}{\frozenfalse} = U_S(S)$, where $U_S$ is the
      store update operation that acts as the identity on the contents
      of all existing locations, and adds the binding
      $\storebinding{l}{\bot}{\frozenfalse}$ if no binding for $l$
      exists.

      Note that:
      \begin{itemize}
      \item $U_S$ is non-conflicting with $\config{S}{e_{b_1}}
        \parstepsto \config{S_b}{e_{b_2}}$, since no locations are
        allocated in the transition;
      \item $U_S(S_b) \neq \topS$, since $U_S(S_b) =
        \extS{S_b}{l}{\bot}{\frozenfalse}$, and we know $S_b \neq
        \topS$ and the addition of a new binding
        $\storebinding{l}{\bot}{\frozenfalse}$ cannot cause it to
        become $\topS$; and
      \item $U_S$ is freeze-safe with $\config{S}{e_{b_1}} \parstepsto
        \config{S_b}{e_{b_2}}$, since $S_b =
        \extSRaw{S}{l'}{u_{p_i}(p_1)}$ and $u_{p_i}$ does not alter
        the status of $p_1$.

        (By Definition~\ref{def:set-of-state-update-operations},
        $u_{p_i}$ can only change the status bit of a location if its
        contents are $\state{d}{\frozentrue}$ and $u_i(d) \neq d$, in
        which case $u_{p_i}$ changes the contents of the location to
        $\state{\top}{\frozenfalse}$; however, that cannot be the case
        here since then $u_{p_i}(p_1)$ would be $\topp$, contradicting
        the premise of {\sc E-Put}.)
      \end{itemize}

      Therefore, by Lemma~\ref{lem:generalized-independence}
      (Generalized Independence), we have that

      $\config{U_S(S)}{e_{b_1}} \parstepsto
      \config{U_S(S_b)}{e_{b_2}}$.

      Hence $\config{\extS{S}{l}{\bot}{\frozenfalse}}{e_{b_1}}
      \parstepsto
      \config{\extS{S_b}{l}{\bot}{\frozenfalse}}{e_{b_2}}$.

      By {\sc E-Eval-Ctxt}, it follows that

      $\config{\extS{S}{l}{\bot}{\frozenfalse}}{\evalctxt{E'_b}{e_{b_1}}}
      \ctxstepsto
      \config{\extS{S_b}{l}{\bot}{\frozenfalse}}{\evalctxt{E'_b}{e_{b_2}}}$,
      
      which, since $S_a = \extS{S}{l}{\bot}{\frozenfalse}$, is what we
      were required to show.

      For the second, let $U_S$ be the store update operation that
      applies $u_{p_i}$ to the contents of $l'$ if it exists, and adds
      a binding $\storebindingRaw{l'}{u_{p_i}(p_1)}$ if no binding for
      $l'$ exists.

      Consider that $S_b = U_S(S)$, and
      $\extS{S_b}{l}{\bot}{\frozenfalse} =
      \extSRaw{S_a}{l'}{u_{p_i}(p_1)} = U_S(S_a)$.

      Note that:
      \begin{itemize}
      \item $U_S$ is non-conflicting with $\config{S}{e_{a_1}}
        \parstepsto \config{S_a}{e_{a_2}}$, since the only location
        allocated in the transition is $l$;
      \item $U_S(S_a) \neq \topS$, since $U_S(S_a) =
        \extSRaw{\extS{S}{l}{\bot}{\frozenfalse}}{l'}{u_{p_i}(p_1)}$
        and we know $S \neq \topS$ and the addition of a new binding
        $\storebinding{l}{\bot}{\frozenfalse}$ and updating the
        contents of location $l'$ to $u_{p_i}(p_1)$ in $S$ cannot
        cause it to become $\topS$ (since if $u_{p_i}(p_1) = \topp$,
        $\config{S}{e_{b_1}}$ would not have been able to step by {\sc
          E-Put}); and
      \item $U_S$ is freeze-safe with $\config{S}{e_{a_1}} \parstepsto
        \config{S_a}{e_{a_2}}$, since $S_a =
        \extS{S}{l}{\bot}{\frozenfalse}$ and $l \notin \dom{S}$, so
        there are no locations whose contents differ in status between
        $S$ and $S_a$.
      \end{itemize}

      Therefore, by Lemma~\ref{lem:generalized-independence}
      (Generalized Independence), we have that

      $\config{U_S(S)}{e_{a_1}} \parstepsto
      \config{U_S(S_a)}{e_{a_2}}$.

      Hence $\config{S_b}{e_{a_1}}
      \parstepsto
      \config{\extS{S_b}{l}{\bot}{\frozenfalse}}{e_{a_2}}$.

      By {\sc E-Eval-Ctxt}, it follows that
      
      $\config{S_b}{\evalctxt{E'_a}{e_{a_1}}} \ctxstepsto
      \config{\extS{S_b}{l}{\bot}{\frozenfalse}}{\evalctxt{E'_a}{e_{a_2}}}$,
      
      as we were required to show.

    \item \label{slqc-new-put-err}Case {\sc E-Put-Err}: We have $S_a =
      \extS{S}{l}{\bot}{\frozenfalse}$ and $\config{S_b}{e_{b_2}} =
      \error$, and so we choose $\conf_c = \error$, $i = 1$, $j = 0$,
      and $\pi = \id$.

      We have to show that:
      \begin{itemize}
      \item $\config{S_a}{\evalctxt{E'_b}{e_{b_1}}} \ctxstepsto
        \error$, and
      \item $\config{S_b}{\evalctxt{E'_a}{e_{a_1}}} = \error$.
      \end{itemize}

      The second of these is immediately true because since
      $\config{S_b}{e_{b_2}} = \error$, $S_b = \topS$, and so
      $\config{S_b}{\evalctxt{E'_a}{e_{a_1}}}$ is equal to $\error$ as
      well.

      For the first, observe that since $\config{S}{e_{a_1}}
      \parstepsto \config{S_a}{e_{a_2}}$, we have by
      Lemma~\ref{lem:monotonicity} (Monotonicity) that
      $\leqstore{S}{S_a}$.

      Therefore, since $\config{S}{e_{b_1}} \parstepsto \error$,

      we have by Lemma~\ref{lem:error-preservation} (Error
      Preservation) that $\config{S_a}{e_{b_1}} \parstepsto \error$.

      Since $\error$ is equal to $\config{\topS}{e}$ for all
      expressions $e$, $\config{S_a}{e_{b_1}} \parstepsto
      \config{\topS}{e}$ for all $e$.

      Therefore, by {\sc E-Eval-Ctxt},
      $\config{S_a}{\evalctxt{E'_b}{e_{b_1}}} \ctxstepsto
      \config{\topS}{\evalctxt{E'_b}{e}}$ for all $e$.

      Since $\config{\topS}{\evalctxt{E'_b}{e}}$ is equal to $\error$,
      we have that $\config{S_a}{\evalctxt{E'_b}{e_{b_1}}} \ctxstepsto
      \error$, as we were required to show.

    \item \label{slqc-new-get}Case {\sc E-Get}: Similar to
      case~\ref{slqc-new-beta}, since $S_a =
      \extS{S}{l}{\bot}{\frozenfalse}$ and $S_b = S$.
    \item \label{slqc-new-freeze-init}Case {\sc E-Freeze-Init}:
      Similar to case~\ref{slqc-new-beta}, since $S_a =
      \extS{S}{l}{\bot}{\frozenfalse}$ and $S_b = S$.
    \item \label{slqc-new-spawn-handler}Case {\sc E-Spawn-Handler}:
      Similar to case~\ref{slqc-new-beta}, since $S_a =
      \extS{S}{l}{\bot}{\frozenfalse}$ and $S_b = S$.
    \item \label{slqc-new-freeze-final}Case {\sc E-Freeze-Final}: We
      have $S_a = \extS{S}{l}{\bot}{\frozenfalse}$ and $S_b =
      \extS{S}{l'}{d_1}{\frozentrue}$, where $l \neq l'$ (since $l
      \notin \dom{S}$, but $l' \in \dom{S}$).

      Choose $S' =
      \extS{\extS{S}{l}{\bot}{\frozenfalse}}{l'}{d_1}{\frozentrue}$,
      $i = i$, $j = 1$, and $\pi = \id$.

      We have to show that:
      \begin{itemize}
      \item
        $\config{\extS{S}{l}{\bot}{\frozenfalse}}{\evalctxt{E'_b}{e_{b_1}}}
        \ctxstepsto
        \config{\extS{\extS{S}{l}{\bot}{\frozenfalse}}{l'}{d_1}{\frozentrue}}{\evalctxt{E'_b}{e_{b_2}}}$,
        and
      \item
        $\config{\extS{S}{l'}{d_1}{\frozentrue}}{\evalctxt{E'_a}{e_{a_1}}}
        \ctxstepsto
        \config{\extS{\extS{S}{l}{\bot}{\frozenfalse}}{l'}{d_1}{\frozentrue}}{\evalctxt{E'_a}{e_{a_2}}}$.
      \end{itemize}

      For the first of these, consider that
      $\extS{S}{l}{\bot}{\frozenfalse} = U_S(S)$, where $U_S$ is the
      store update operation that acts as the identity on the contents
      of all existing locations, and adds the binding
      $\storebinding{l}{\bot}{\frozenfalse}$ if no binding for $l$
      exists.

      Note that:
      \begin{itemize}
      \item $U_S$ is non-conflicting with $\config{S}{e_{b_1}}
        \parstepsto \config{S_b}{e_{b_2}}$, since no locations are
        allocated in the transition;
      \item $U_S(S_b) \neq \topS$, since $U_S(S_b) =
        \extS{S_b}{l}{\bot}{\frozenfalse}$, and we know $S_b \neq
        \topS$ and the addition of a new binding
        $\storebinding{l}{\bot}{\frozenfalse}$ cannot cause it to
        become $\topS$; and
      \item $U_S$ is freeze-safe with $\config{S}{e_{b_1}}
        \parstepsto \config{S_b}{e_{b_2}}$, since $S_b =
        \extS{S}{l'}{d_1}{\frozentrue}$ and so the only location
        that can change in status between $S$ and $S_b$ is $l'$, and
        $U_S$ acts as the identity on $l'$.
      \end{itemize}
      Therefore, by Lemma~\ref{lem:generalized-independence}
      (Generalized Independence), we have that

      $\config{U_S(S)}{e_{b_1}} \parstepsto
      \config{U_S(S_b)}{e_{b_2}}$.

      Hence $\config{\extS{S}{l}{\bot}{\frozenfalse}}{e_{b_1}}
      \parstepsto
      \config{\extS{\extS{S}{l}{\bot}{\frozenfalse}}{l'}{d_1}{\frozentrue}}{e_{b_2}}$.

      By {\sc E-Eval-Ctxt}, it follows that

      $\config{\extS{S}{l}{\bot}{\frozenfalse}}{\evalctxt{E'_b}{e_{b_1}}}
      \ctxstepsto
      \config{\extS{\extS{S}{l}{\bot}{\frozenfalse}}{l'}{d_1}{\frozentrue}}{\evalctxt{E'_b}{e_{b_2}}}$,

      as we were required to show.

      For the second, consider that $\extS{S}{l'}{d_1}{\frozentrue} =
      U_S(S)$, where $U_S$ is the store update operation that freezes
      the contents of $l'$ and acts as the identity on the contents of
      all other locations.

      Note that:
      \begin{itemize}
      \item $U_S$ is non-conflicting with $\config{S}{e_{a_1}}
        \parstepsto \config{S_a}{e_{a_2}}$, since the only location
        allocated in the transition is $l$, and $l \neq l'$;
      \item $U_S(S_a) \neq \topS$, since $U_S(S_a) =
        \extS{S_a}{l'}{d_1}{\frozentrue} =
        \extS{S_b}{l}{\bot}{\frozenfalse}$, and we know $S_b \neq
        \topS$ and the addition of a new binding
        $\storebinding{l}{\bot}{\frozenfalse}$ cannot cause it to
        become $\topS$; and
      \item $U_S$ is freeze-safe with $\config{S}{e_{a_1}}
        \parstepsto \config{S_a}{e_{a_2}}$, since $S_a =
        \extS{S}{l}{\bot}{\frozenfalse}$ and $l \notin \dom{S}$, so
        there are no locations whose contents differ in status
        between $S$ and $S_a$.
      \end{itemize}

      Therefore, by Lemma~\ref{lem:generalized-independence}
      (Generalized Independence), we have that

      $\config{U_S(S)}{e_{a_1}} \parstepsto
      \config{U_S(S_a)}{e_{a_2}}$.

      Hence $\config{\extS{S}{l'}{d_1}{\frozentrue}}{e_{a_1}}
      \parstepsto
      \config{\extS{\extS{S}{l}{\bot}{\frozenfalse}}{l'}{d_1}{\frozentrue}}{e_{a_2}}$.

      By {\sc E-Eval-Ctxt} it follows that

      $\config{\extS{S}{l'}{d_1}{\frozentrue}}{\evalctxt{E'_a}{e_{a_1}}}
      \ctxstepsto
      \config{\extS{\extS{S}{l}{\bot}{\frozenfalse}}{l'}{d_1}{\frozentrue}}{\evalctxt{E'_a}{e_{a_2}}}$,

      as we were required to show.

    \item \label{slqc-new-freeze-simple}Case {\sc E-Freeze-Simple}:
      Similar to case~\ref{slqc-new-freeze-final}, since $S_a =
      \extS{S}{l}{\bot}{\frozenfalse}$ and $S_b =
      \extS{S}{l'}{d_1}{\frozentrue}$, where $l \neq l'$ (since $l
      \notin \dom{S}$, but $l' \in \dom{S}$).

    \end{enumerate}
  \item Case {\sc E-Put}: We have $S_a =
    \extSRaw{S}{l}{u_{p_i}(p_1)}$.

    We proceed by case analysis on the rule by which
    $\config{S}{e_{b_1}}$ steps to $\config{S_b}{e_{b_2}}$.

    Since the only way an $\error$ configuration can arise is by the
    {\sc E-Put-Err} rule, we can assume in all other cases that
    $\conf_b \neq \error$.
    \begin{enumerate}
    \item \label{slqc-put-beta}Case {\sc E-Beta}: By symmetry with case~\ref{slqc-beta-put}.
    \item \label{slqc-put-new}Case {\sc E-New}: By symmetry with case~\ref{slqc-new-put}.
    \item \label{slqc-put-put}Case {\sc E-Put}: We have $S_a =
      \extSRaw{S}{l}{u_{p_i}(p_1)}$ and $S_b =
      \extSRaw{S}{l'}{u_{p_j}(p'_1)}$, where $p'_1 = S(l')$.

      Now consider whether $l = l'$:
      \begin{itemize}
      \item If $l \neq l'$:

        Choose $S' =
        \extSRaw{\extSRaw{S}{l'}{u_{p_j}(p'_1)}}{l}{u_{p_i}(p_1)}$,
        $i = 1$, $j = 1$, and $\pi = \id$.

        We have to show that:
        \begin{itemize}
        \item
          $\config{\extSRaw{S}{l}{u_{p_i}(p_1)}}{\evalctxt{E'_b}{e_{b_1}}}
          \ctxstepsto
          \config{\extSRaw{\extSRaw{S}{l'}{u_{p_j}(p'_1)}}{l}{u_{p_i}(p_1)}}{\evalctxt{E'_b}{e_{b_2}}}$,
          and
        \item
          $\config{\extSRaw{S}{l'}{u_{p_j}(p'_1)}}{\evalctxt{E'_a}{e_{a_1}}}
          \ctxstepsto
          \config{\extSRaw{\extSRaw{S}{l'}{u_{p_j}(p'_1)}}{l}{u_{p_i}(p_1)}}{\evalctxt{E'_a}{e_{a_2}}}$.
        \end{itemize}

        For the first of these, consider that
        $\extSRaw{S}{l}{u_{p_i}(p_1)} = U_S(S)$, where $U_S$ is the
        store update operation that applies $u_{p_i}$ to the
        contents of $l$ if it exists, and adds a binding
        $\storebindingRaw{l}{u_{p_i}(p_1)}$ if no binding for $l$
        exists.

        Note that:
        \begin{itemize}
        \item $U_S$ is non-conflicting with $\config{S}{e_{b_1}}
          \parstepsto
          \config{\extSRaw{S}{l'}{u_{p_j}(p'_1)}}{e_{b_2}}$, since
          no locations are allocated in the transition;
        \item $U_S(\extSRaw{S}{l'}{u_{p_j}(p'_1)}) \neq \topS$,
          since $U_S(\extSRaw{S}{l'}{u_{p_j}(p'_1)}) =
          \extSRaw{\extSRaw{S}{l'}{u_{p_j}(p'_1)}}{l}{u_{p_i}(p_1)}$
          and we know $S \neq \topS$ and updating the contents of
          location $l$ to $u_{p_i}(p_1)$ and the contents of
          location $l'$ to $u_{p_j}(p'_1)$ in $S$ cannot cause it to
          become $\topS$ (because if so, then we would have $S_a =
          \topS$ or $S_b = \topS$, which we know are not the case);
          and
        \item $U_S$ is freeze-safe with $\config{S}{e_{b_1}}
          \parstepsto
          \config{\extSRaw{S}{l'}{u_{p_j}(p'_1)}}{e_{b_2}}$, since
          $u_{p_j}$ does not alter the status of $p'_1$.

          (By Definition~\ref{def:set-of-state-update-operations},
          $u_{p_j}$ can only change the status bit of a location if
          its contents are $\state{d}{\frozentrue}$ and $u_j(d) \neq
          d$, in which case $u_{p_j}$ changes the contents of the
          location to $\state{\top}{\frozenfalse}$; however, that
          cannot be the case here since then $u_{p_j}(p'_1)$ would be
          $\topp$, contradicting the premise of {\sc E-Put}.)
        \end{itemize}

        Therefore, by Lemma~\ref{lem:generalized-independence}
        (Generalized Independence), we have that

        $\config{U_S(S)}{e_{b_1}} \parstepsto
        \config{U_S(\extSRaw{S}{l'}{u_{p_j}(p'_1)})}{e_{b_2}}$.

        Hence $\config{\extSRaw{S}{l}{u_{p_i}(p_1)}}{e_{b_1}}
        \parstepsto
        \config{\extSRaw{\extSRaw{S}{l'}{u_{p_j}(p'_1)}}{l}{u_{p_i}(p_1)}}{e_{b_2}}$.

        By {\sc E-Eval-Ctxt}, it follows that

        $\config{\extSRaw{S}{l}{u_{p_i}(p_1)}}{\evalctxt{E'_b}{e_{b_1}}}
        \ctxstepsto
        \config{\extSRaw{\extSRaw{S}{l'}{u_{p_j}(p'_1)}}{l}{u_{p_i}(p_1)}}{\evalctxt{E'_b}{e_{b_2}}}$,

        as we were required to show.

        The argument for the second is symmetrical.

      \item If $l = l'$:
        Note that since $l = l'$, $p_1 = p'_1$ as well.

        Consider whether $u_{p_i}(u_{p_j}(p_1)) = \topp$:
        \begin{itemize}
        \item If $u_{p_i}(u_{p_j}(p_1)) = \topp$:

          Choose $\conf_c = \error$, $i = 1$, $j = 1$, and $\pi =
          \id$.

          We have to show that:

          \begin{itemize}
          \item
            $\config{\extSRaw{S}{l}{u_{p_i}(p_1)}}{\evalctxt{E'_b}{e_{b_1}}}
            \ctxstepsto \error$, and
          \item
            $\config{\extSRaw{S}{l}{u_{p_j}(p_1)}}{\evalctxt{E'_a}{e_{a_1}}}
            \ctxstepsto \error$.
          \end{itemize}

          For the first of these, consider that
          $\extSRaw{S}{l}{u_{p_i}(p_1)} = U_S(S)$, where $U_S$ is the
          store update operation that applies $u_{p_i}$ to the
          contents of $l$ if it exists.

          Note that:
          \begin{itemize}
          \item $U_S$ is non-conflicting with $\config{S}{e_{b_1}}
            \parstepsto
            \config{\extSRaw{S}{l}{u_{p_j}(p_1)}}{e_{b_2}}$, since
            no locations are allocated in the transition;
          \item $U_S(\extSRaw{S}{l}{u_{p_j}(p_1)}) = \topS$, since
            $U_S(\extSRaw{S}{l}{u_{p_j}(p_1)}) =
            \extSRaw{S}{l}{u_{p_i}(u_{p_j}(p_1))}$ and we know
            $u_{p_i}(u_{p_j}(p_1)) = \topp$ in this case;
          \item $U_S$ is freeze-safe with $\config{S}{e_{b_1}}
            \parstepsto
            \config{\extSRaw{S}{l}{u_{p_j}(p_1)}}{e_{b_2}}$, since
            $u_{p_j}$ does not alter the status of $p_1$.

            (By Definition~\ref{def:set-of-state-update-operations},
            $u_{p_j}$ can only change the status bit of a location if
            its contents are $\state{d}{\frozentrue}$ and $u_j(d) \neq
            d$, in which case $u_{p_j}$ changes the contents of the
            location to $\state{\top}{\frozenfalse}$; however, that
            cannot be the case here since then $u_{p_j}(p_1)$ would be
            $\topp$, contradicting the premise of {\sc E-Put}.)
          \end{itemize}

          Therefore, by Lemma~\ref{lem:generalized-clash}
          (Generalized Clash), we have that there exists $i' \leq 1$
          such that $\config{U_S(S)}{e_{b_1}} \parstepsto^{i'}
          \error$.

          Hence $\config{\extSRaw{S}{l}{u_{p_i}(p_1)}}{e_{b_1}}
          \parstepsto^{i'} \error$.

          If $i' = 0$, we would have
          $\config{\extSRaw{S}{l}{u_{p_i}(p_1)}}{e_{b_1}} =
          \config{S_a}{e_{b_1}} = \error$.

          So we would have $S_a = \topS$ by the definition of
          $\error$, but then we would have $\conf_a = \error$, a
          contradiction.

          Therefore $i' = 1$, and so we have
          $\config{\extSRaw{S}{l}{u_{p_i}(p_1)}}{e_{b_1}} \parstepsto
          \error$.

          Since $\error = \config{\topS}{e}$ for all $e$, we have
          $\config{\extSRaw{S}{l}{u_{p_i}(p_1)}}{e_{b_1}}
          \parstepsto \config{\topS}{e}$ for all $e$.

          So, by {\sc E-Eval-Ctxt}, we have that
          $\config{\extSRaw{S}{l}{u_{p_i}(p_1)}}{\evalctxt{E'_b}{e_{b_1}}}
          \parstepsto \config{\topS}{\evalctxt{E'_b}{e}}$ for all $e$.

          Hence
          $\config{\extSRaw{S}{l}{u_{p_i}(p_1)}}{\evalctxt{E'_b}{e_{b_1}}}
          \parstepsto \error$.

          The argument for the second is symmetrical.

        \item If $u_{p_i}(u_{p_j}(p_1)) \neq \topp$:

          Choose $S' = \extSRaw{S}{l}{u_{p_i}(u_{p_j}(p_1))}$, $i =
          1$, $j = 1$, and $\pi = \id$.

          We have to show that:
          \begin{itemize}
          \item
            $\config{\extSRaw{S}{l}{u_{p_i}(p_1)}}{\evalctxt{E'_b}{e_{b_1}}}
            \ctxstepsto
            \config{\extSRaw{S}{l}{u_{p_i}(u_{p_j}(p_1))}}{\evalctxt{E'_b}{e_{b_2}}}$,
            and
          \item
            $\config{\extSRaw{S}{l}{u_{p_j}(p_1)}}{\evalctxt{E'_a}{e_{a_1}}}
            \ctxstepsto
            \config{\extSRaw{S}{l}{u_{p_i}(u_{p_j}(p_1))}}{\evalctxt{E'_a}{e_{a_2}}}$.
          \end{itemize}

          For the first of these, consider that
          $\extSRaw{S}{l}{u_{p_i}(p_1)} = U_S(S)$, where $U_S$ is the
          store update operation that applies $u_{p_i}$ to the
          contents of $l$ if it exists.

          Note that:
          \begin{itemize}
          \item $U_S$ is non-conflicting with $\config{S}{e_{b_1}}
            \parstepsto
            \config{\extSRaw{S}{l}{u_{p_j}(p_1)}}{e_{b_2}}$, since no
            locations are allocated in the transition;
          \item $U_S(\extSRaw{S}{l}{u_{p_j}(p_1)}) \neq \topS$, since
            $U_S(\extSRaw{S}{l}{u_{p_j}(p_1)}) =
            \extSRaw{S}{l}{u_{p_i}(u_{p_j}(p_1))}$ and we know $S \neq
            \topS$ and $u_{p_i}(u_{p_j}(p_1)) \neq \topp$ in this
            case;
          \item $U_S$ is freeze-safe with $\config{S}{e_{b_1}}
            \parstepsto
            \config{\extSRaw{S}{l}{u_{p_j}(p_1)}}{e_{b_2}}$, since
            $u_{p_j}$ does not alter the status of $p_1$.

            (By Definition~\ref{def:set-of-state-update-operations},
            $u_{p_j}$ can only change the status bit of a location if
            its contents are $\state{d}{\frozentrue}$ and $u_j(d) \neq
            d$, in which case $u_{p_j}$ changes the contents of the
            location to $\state{\top}{\frozenfalse}$; however, that
            cannot be the case here since then $u_{p_j}(p_1)$ would be
            $\topp$, contradicting the premise of {\sc E-Put}.)
          \end{itemize}

          Therefore, by Lemma~\ref{lem:generalized-independence}
          (Generalized Independence), we have that

          $\config{U_S(S)}{e_{b_1}} \parstepsto
          \config{U_S(\extSRaw{S}{l}{u_{p_j}(p_1)})}{e_{b_2}}$.

          Hence $\config{\extSRaw{S}{l}{u_{p_i}(p_1)}}{e_{b_1}}
          \parstepsto
          \config{\extSRaw{S}{l}{u_{p_i}(u_{p_j}(p_1))}}{e_{b_2}}$.

          By {\sc E-Eval-Ctxt}, it follows that

          $\config{\extSRaw{S}{l}{u_{p_i}(p_1)}}{\evalctxt{E'_b}{e_{b_1}}}
          \ctxstepsto
          \config{\extSRaw{S}{l}{u_{p_i}(u_{p_j}(p_1))}}{\evalctxt{E'_b}{e_{b_2}}}$,

          as we were required to show.

          The argument for the second is symmetrical.

        \end{itemize}

      \end{itemize}

    \item \label{slqc-put-put-err}Case {\sc E-Put-Err}: We have $S_a =
      \extSRaw{S}{l}{u_{p_i}(p_1)}$ and $\config{S_b}{e_{b_2}} =
      \error$, and so we choose $\conf_c = \error$, $i = 1$, $j = 0$,
      and $\pi = \id$.

      We have to show that:
      \begin{itemize}
      \item $\config{S_a}{\evalctxt{E'_b}{e_{b_1}}} \ctxstepsto
        \error$, and
      \item $\config{S_b}{\evalctxt{E'_a}{e_{a_1}}} = \error$.
      \end{itemize}

      The second of these is immediately true because since
      $\config{S_b}{e_{b_2}} = \error$, $S_b = \topS$, and so
      $\config{S_b}{\evalctxt{E'_a}{e_{a_1}}}$ is equal to $\error$ as
      well.

      For the first, observe that since $\config{S}{e_{a_1}}
      \parstepsto \config{S_a}{e_{a_2}}$, we have by
      Lemma~\ref{lem:monotonicity} (Monotonicity) that
      $\leqstore{S}{S_a}$.

      Therefore, since $\config{S}{e_{b_1}} \parstepsto \error$,

      we have by Lemma~\ref{lem:error-preservation} (Error
      Preservation) that $\config{S_a}{e_{b_1}} \parstepsto \error$.
      
      Since $\error$ is equal to $\config{\topS}{e}$ for all
      expressions $e$, $\config{S_a}{e_{b_1}} \parstepsto
      \config{\topS}{e}$ for all $e$.

      Therefore, by {\sc E-Eval-Ctxt},
      $\config{S_a}{\evalctxt{E'_b}{e_{b_1}}} \ctxstepsto
      \config{\topS}{\evalctxt{E'_b}{e}}$ for all $e$.

      Since $\config{\topS}{\evalctxt{E'_b}{e}}$ is equal to $\error$,
      we have that $\config{S_a}{\evalctxt{E'_b}{e_{b_1}}} \ctxstepsto
      \error$, as we were required to show.

    \item \label{slqc-put-get}Case {\sc E-Get}: Similar to
      case~\ref{slqc-put-beta}, since $S_a =
      \extSRaw{S}{l}{u_{p_i}(p_1)}$ and $S_b = S$.
    \item \label{slqc-put-freeze-init}Case {\sc E-Freeze-Init}:
      Similar to case~\ref{slqc-put-beta}, since $S_a =
      \extSRaw{S}{l}{u_{p_i}(p_1)}$ and $S_b = S$.
    \item \label{slqc-put-spawn-handler}Case {\sc E-Spawn-Handler}:
      Similar to case~\ref{slqc-put-beta}, since $S_a =
      \extSRaw{S}{l}{u_{p_i}(p_1)}$ and $S_b = S$.
    \item \label{slqc-put-freeze-final}Case {\sc E-Freeze-Final}: We
      have $S_a = \extSRaw{S}{l}{u_{p_i}(p_1)}$ and $S_b =
      \extS{S}{l'}{d_1}{\frozentrue}$.

      Now consider whether $l = l'$:
      \begin{itemize}
      \item If $l \neq l'$:

        Choose $S' =
        \extS{\extSRaw{S}{l}{u_{p_i}(p_1)}}{l'}{d_1}{\frozentrue}$,
        $i = 1$, $j = 1$, and $\pi = \id$.

        We have to show that:
        \begin{itemize}
        \item
          $\config{\extSRaw{S}{l}{u_{p_i}(p_1)}}{\evalctxt{E'_b}{e_{b_1}}}
          \ctxstepsto
          \config{\extS{\extSRaw{S}{l}{u_{p_i}(p_1)}}{l'}{d_1}{\frozentrue}}{\evalctxt{E'_b}{e_{b_2}}}$,
          and
        \item
          $\config{\extS{S}{l'}{d_1}{\frozentrue}}{\evalctxt{E'_a}{e_{a_1}}}
          \ctxstepsto
          \config{\extS{\extSRaw{S}{l}{u_{p_i}(p_1)}}{l'}{d_1}{\frozentrue}}{\evalctxt{E'_a}{e_{a_2}}}$.
        \end{itemize}

        For the first of these, consider that
        $\extSRaw{S}{l}{u_{p_i}(p_1)} = U_S(S)$, where $U_S$ is the
        store update operation that applies $u_{p_i}$ to the
        contents of $l$ if it exists, and adds a binding
        $\storebindingRaw{l}{u_{p_i}(p_1)}$ if no binding for $l$
        exists, and acts as the identity on all other locations.

        Note that:
        \begin{itemize}
        \item $U_S$ is non-conflicting with $\config{S}{e_{b_1}}
          \parstepsto
          \config{\extS{S}{l'}{d_1}{\frozentrue}}{e_{b_2}}$, since
          no locations are allocated in the transition;
        \item $U_S(\extS{S}{l'}{d_1}{\frozentrue}) \neq \topS$,

          since $U_S(\extS{S}{l'}{d_1}{\frozentrue}) =
          \extSRaw{\extS{S}{l'}{d_1}{\frozentrue}}{l}{u_{p_i}(p_1)}$
          and we know $S \neq \topS$ and updating the contents of
          location $l$ to $u_{p_i}(p_1)$ and freezing the contents
          of location $l'$ in $S$ cannot cause it to become $\topS$
          (because if so, then we would have $S_a = \topS$ or $S_b =
          \topS$, which we know are not the case); and
        \item $U_S$ is freeze-safe with $\config{S}{e_{b_1}}
          \parstepsto
          \config{\extS{S}{l'}{d_1}{\frozentrue}}{e_{b_2}}$, since
          the only location that can change in status between $S$
          and $\extS{S}{l'}{d_1}{\frozentrue}$ is $l'$, and $U_S$
          acts as the identity on $l'$.
        \end{itemize}
        Therefore, by Lemma~\ref{lem:generalized-independence}
        (Generalized Independence), we have that

        $\config{U_S(S)}{e_{b_1}} \parstepsto
        \config{U_S(\extS{S}{l'}{d_1}{\frozentrue})}{e_{b_2}}$.

        Hence $\config{\extSRaw{S}{l}{u_{p_i}(p_1)}}{e_{b_1}}
        \parstepsto
        \config{\extSRaw{\extS{S}{l'}{d_1}{\frozentrue}}{l}{u_{p_i}(p_1)}}{e_{b_2}}$.

        By {\sc E-Eval-Ctxt}, it follows that

        $\config{\extSRaw{S}{l}{u_{p_i}(p_1)}}{\evalctxt{E'_b}{e_{b_1}}}
        \ctxstepsto
        \config{\extSRaw{\extS{S}{l'}{d_1}{\frozentrue}}{l}{u_{p_i}(p_1)}}{\evalctxt{E'_b}{e_{b_2}}}$,
        
        as we were required to show.

        For the second, consider that
        $\extS{S}{l'}{d_1}{\frozentrue} = U_S(S)$, where $U_S$ is
        the store update operation that freezes the contents of $l'$
        and acts as the identity on the contents of all other
        locations.

        Note that:
        \begin{itemize}
        \item $U_S$ is non-conflicting with $\config{S}{e_{a_1}}
          \parstepsto
          \config{\extSRaw{S}{l}{u_{p_i}(p_1)}}{e_{a_2}}$, since no
          locations are allocated in the transition;
        \item $U_S(\extSRaw{S}{l}{u_{p_i}(p_1)}) \neq \topS$, since
          $U_S(\extSRaw{S}{l}{u_{p_i}(p_1)}) =
          \extS{\extSRaw{S}{l}{u_{p_i}(p_1)}}{l'}{d_1}{\frozentrue}$,
          and we know $S \neq \topS$ and updating the contents of
          location $l$ to $u_{p_i}(p_1)$ and freezing the contents
          of location $l$ in $S$ cannot cause it to become $\topS$
          (because if so, then we would have $S_a = \topS$ or $S_b =
          \topS$, which we know are not the case); and
        \item $U_S$ is freeze-safe with $\config{S}{e_{a_1}}
          \parstepsto
          \config{\extSRaw{S}{l}{u_{p_i}(p_1)}}{e_{a_2}}$, since
          $u_{p_i}$ does not alter the status of $p_1$.

          (By Definition~\ref{def:set-of-state-update-operations},
          $u_{p_i}$ can only change the status bit of a location if
          its contents are $\state{d}{\frozentrue}$ and $u_i(d) \neq
          d$, in which case $u_{p_i}$ changes the contents of the
          location to $\state{\top}{\frozenfalse}$; however, that
          cannot be the case here since then $u_{p_i}(p_1)$ would be
          $\topp$, and we would have $S_a = \topS$, a contradiction.)
        \end{itemize}
        Therefore, by Lemma~\ref{lem:generalized-independence}
        (Generalized Independence), we have that

        $\config{U_S(S)}{e_{a_1}} \parstepsto
        \config{U_S(\extSRaw{S}{l}{u_{p_i}(p_1)})}{e_{a_2}}$.

        Hence $\config{\extS{S}{l'}{d_1}{\frozentrue}}{e_{a_1}}
        \parstepsto
        \config{\extS{\extSRaw{S}{l}{u_{p_i}(p_1)}}{l'}{d_1}{\frozentrue}}{e_{a_2}}$.

        By {\sc E-Eval-Ctxt}, it follows that
        $\config{\extS{S}{l'}{d_1}{\frozentrue}}{\evalctxt{E'_a}{e_{a_1}}}
        \ctxstepsto
        \config{\extS{\extSRaw{S}{l}{u_{p_i}(p_1)}}{l'}{d_1}{\frozentrue}}{\evalctxt{E'_a}{e_{a_2}}}$,
        
        as we were required to show.

      \item If $l = l'$:

        We have two cases to consider:

        \begin{itemize}
        \item $u_{p_i}(\state{d_1}{\frozentrue}) = \topp$:

          \lk{This is the interesting case: the potential
            put-after-freeze case.  It's important to note that this
            case doesn't necessarily end in a put-after-freeze (and
            hence an error); all we're required to show is that it
            \emph{can} end that way.}

          Since $(\extSRaw{S}{l}{\state{d_1}{\frozentrue}})(l) =
          \state{d_1}{\frozentrue}$ and
          $u_{p_i}(\state{d_1}{\frozentrue}) = \topp$, by {\sc
            E-Put-Err} we have that
          $\config{\extSRaw{S}{l}{\state{d_1}{\frozentrue}}}{\putiexp{l}}
          \parstepsto \error$.

          Since $S_b = \extSRaw{S}{l}{\state{d_1}{\frozentrue}}$,
          we have that $\config{S_b}{\putiexp{l}} \parstepsto
          \error$.

          Since $\config{S}{e_{a_1}} \parstepsto
          \config{S_a}{e_{a_2}}$ by {\sc E-Put}, it must be the
          case that $e_{a_1} = \putiexp{l}$.

          Hence $\config{S_b}{e_{a_1}} \parstepsto \error$.

          Since $\error$ is equal to $\config{\topS}{e}$ for all
          expressions $e$, $\config{S_b}{e_{a_1}} \parstepsto
          \config{\topS}{e}$ for all $e$.

          Therefore, by {\sc E-Eval-Ctxt},
          $\config{S_b}{\evalctxt{E'_a}{e_{a_1}}} \ctxstepsto
          \config{\topS}{\evalctxt{E'_a}{e}}$ for all $e$.

          Since $\config{\topS}{\evalctxt{E'_a}{e}}$ is equal to
          $\error$, we have that
          $\config{S_b}{\evalctxt{E'_a}{e_{a_1}}} \ctxstepsto \error$.

          Since $\evalctxt{E'_a}{e_{a_1}} =
          \evalctxt{E_b}{e_{b_2}}$, we have that
          $\config{S_b}{\evalctxt{E_b}{e_{b_2}}} \ctxstepsto
          \error$.

          Since $\conf_b = \config{S_b}{\evalctxt{E_b}{e_{b_2}}}$,
          we therefore have that $\conf_b \ctxstepsto \error$, and
          the case is satisfied.

        \item $u_{p_i}(\state{d_1}{\frozentrue}) \neq \topp$:

          \lk{This is the case where there's a conflicting put and
            freeze, but the put is a no-op, so it doesn't matter.}

          In this case, by the definition of $U_p$
          (Definition~\ref{def:set-of-state-update-operations}),
          
          it must be the case that $u_{p_i}(\state{d_1}{\frozentrue})
          = \state{d_1}{\frozentrue}$.

          Choose $S' = \extS{S}{l}{d_1}{\frozentrue}$, $i = 1$, $j
          = 1$, and $\pi = \id$.

          We have to show that:
          \begin{itemize}
          \item
            $\config{\extSRaw{S}{l}{u_{p_i}(p_1)}}{\evalctxt{E'_b}{e_{b_1}}}
            \ctxstepsto
            \config{\extS{S}{l}{d_1}{\frozentrue}}{\evalctxt{E'_b}{e_{b_2}}}$,
            and
          \item
            $\config{\extS{S}{l}{d_1}{\frozentrue}}{\evalctxt{E'_a}{e_{a_1}}}
            \ctxstepsto
            \config{\extS{S}{l}{d_1}{\frozentrue}}{\evalctxt{E'_a}{e_{a_2}}}$.
          \end{itemize}

          For the first of these, consider that
          $\extSRaw{S}{l}{u_{p_i}(p_1)} = U_S(S)$, where $U_S$ is
          the store update operation that applies $u_{p_i}$ to the
          contents of $l$ if it exists, and adds a binding
          $\storebindingRaw{l}{u_{p_i}(p_1)}$ if no binding for
          $l$ exists, and acts as the identity on all other
          locations.

          Note that:
          \begin{itemize}
          \item $U_S$ is non-conflicting with $\config{S}{e_{b_1}}
            \parstepsto
            \config{\extS{S}{l}{d_1}{\frozentrue}}{e_{b_2}}$,
            since no locations are allocated in the
            transition;
          \item $U_S(\extS{S}{l}{d_1}{\frozentrue}) \neq \topS$,
            
            since $U_S(\extS{S}{l}{d_1}{\frozentrue}) =
            \extSRaw{S}{l}{u_{p_i}(\state{d_1}{\frozentrue})}$ and
            we know $S \neq \topS$ and
            $u_{p_i}(\state{d_1}{\frozentrue}) \neq \topp$; and
          \item $U_S$ is freeze-safe with $\config{S}{e_{b_1}}
            \parstepsto
            \config{\extS{S}{l}{d_1}{\frozentrue}}{e_{b_2}}$, since
            the only location that can change in status between $S$
            and $\extS{S}{l}{d_1}{\frozentrue}$ is $l$, and $U_S$
            acts as the identity on $l$.
          \end{itemize}
          Therefore, by Lemma~\ref{lem:generalized-independence}
          (Generalized Independence), we have that

          $\config{U_S(S)}{e_{b_1}} \parstepsto
          \config{U_S(\extS{S}{l}{d_1}{\frozentrue})}{e_{b_2}}$.

          Hence $\config{\extSRaw{S}{l}{u_{p_i}(p_1)}}{e_{b_1}}
          \parstepsto
          \config{\extSRaw{S}{l}{u_{p_i}(\state{d_1}{\frozentrue})}}{e_{b_2}}$.

          Since $u_{p_i}(\state{d_1}{\frozentrue}) =
          \state{d_1}{\frozentrue}$,

          we have that
          $\config{\extSRaw{S}{l}{u_{p_i}(p_1)}}{e_{b_1}}
          \parstepsto
          \config{\extS{S}{l}{d_1}{\frozentrue}}{e_{b_2}}$.

          By {\sc E-Eval-Ctxt}, it follows that

          $\config{\extSRaw{S}{l}{u_{p_i}(p_1)}}{\evalctxt{E'_b}{e_{b_1}}}
          \ctxstepsto
          \config{\extS{S}{l}{d_1}{\frozentrue}}{\evalctxt{E'_b}{e_{b_2}}}$,

          as we were required to show.

          For the second, consider that
          $\extS{S}{l}{d_1}{\frozentrue} = U_S(S)$, where $U_S$ is
          the store update operation that freezes the contents of $l$
          and acts as the identity on the contents of all other
          locations.

          Note that:
          \begin{itemize}
          \item $U_S$ is non-conflicting with $\config{S}{e_{a_1}}
            \parstepsto
            \config{\extSRaw{S}{l}{u_{p_i}(p_1)}}{e_{a_2}}$, since no
            locations are allocated in the transition;
          \item $U_S(\extSRaw{S}{l}{u_{p_i}(p_1)}) \neq \topS$,
            since $U_S(\extSRaw{S}{l}{u_{p_i}(p_1)}) =
            \extS{S}{l}{d_1}{\frozentrue}$ (since, by
            Definition~\ref{def:set-of-state-update-operations},
            $u_i(d_1) = d_1$; otherwise we would have
            $u_{p_i}(\state{d_1}{\frozentrue}) = \topp$, a
            contradiction), and we know $S \neq \topS$ and
            freezing the contents of location $l$ in $S$ cannot
            cause it to become $\topS$; and
          \item $U_S$ is freeze-safe with $\config{S}{e_{a_1}}
            \parstepsto
            \config{\extSRaw{S}{l}{u_{p_i}(p_1)}}{e_{a_2}}$, since
            $u_{p_i}$ does not alter the status of $p_1$.

            (By Definition~\ref{def:set-of-state-update-operations},
            $u_{p_i}$ can only change the status bit of a location if
            its contents are $\state{d}{\frozentrue}$ and $u_i(d) \neq
            d$, in which case $u_{p_i}$ changes the contents of the
            location to $\state{\top}{\frozenfalse}$; however, that
            cannot be the case here since then $u_{p_i}(p_1)$ would be
            $\topp$, and we would have $S_a = \topS$, a
            contradiction.)
          \end{itemize}
          Therefore, by Lemma~\ref{lem:generalized-independence}
          (Generalized Independence), we have that

          $\config{U_S(S)}{e_{a_1}} \parstepsto
          \config{U_S(\extSRaw{S}{l}{u_{p_i}(p_1)})}{e_{a_2}}$.

          Hence $\config{\extS{S}{l}{d_1}{\frozentrue}}{e_{a_1}}
          \parstepsto
          \config{\extS{S}{l}{d_1}{\frozentrue}}{e_{a_2}}$.

          By {\sc E-Eval-Ctxt}, it follows that

          $\config{\extS{S}{l}{d_1}{\frozentrue}}{\evalctxt{E'_a}{e_{a_1}}}
          \ctxstepsto
          \config{\extS{S}{l}{d_1}{\frozentrue}}{\evalctxt{E'_a}{e_{a_2}}}$,

          as we were required to show.
        \end{itemize}

      \end{itemize}

    \item \label{slqc-put-freeze-simple}Case {\sc E-Freeze-Simple}:
      Similar to case~\ref{slqc-put-freeze-final}, since $S_a =
      \extSRaw{S}{l}{u_{p_i}(p_1)}$ and $S_b =
      \extS{S}{l'}{d_1}{\frozentrue}$.

    \end{enumerate}
  \item Case {\sc E-Put-Err}: We have $\config{S_a}{e_{a_2}} =
    \error$.

    We proceed by case analysis on the rule by which
    $\config{S}{e_{b_1}}$ steps to $\config{S_b}{e_{b_2}}$.

    Since the only way an $\error$ configuration can arise is by the
    {\sc E-Put-Err} rule, we can assume in all other cases that
    $\conf_b \neq \error$.
    \begin{enumerate}
    \item \label{slqc-put-err-beta}Case {\sc E-Beta}: By symmetry with case~\ref{slqc-beta-put-err}.
    \item \label{slqc-put-err-new}Case {\sc E-New}: By symmetry with case~\ref{slqc-new-put-err}.
    \item \label{slqc-put-err-put}Case {\sc E-Put}: By symmetry with case~\ref{slqc-put-put-err}.
    \item \label{slqc-put-err-put-err}Case {\sc E-Put-Err}: We have
      $\config{S_a}{e_{a_2}} = \error$ and $\config{S_b}{e_{b_2}} =
      \error$, and so we choose $\conf_c = \error$, $i = 0$, $j = 0$,
      and $\pi = \id$.

      We have to show that:
      \begin{itemize}
      \item $\config{S_a}{\evalctxt{E'_b}{e_{b_1}}} = \error$, and
      \item $\config{S_b}{\evalctxt{E'_a}{e_{a_1}}} = \error$.
      \end{itemize}

      Since $\config{S_a}{e_{a_2}} = \error$, $S_a = \topS$, and since
      $\config{S_b}{e_{b_2}} = \error$, $S_b = \topS$, so both of the
      above follow immediately.

    \item \label{slqc-put-err-get}Case {\sc E-Get}: Similar to
      case~\ref{slqc-put-err-beta}, since $\config{S_a}{e_{a_2}} =
      \error$ and $S_b = S$.
    \item \label{slqc-put-err-freeze-init}Case {\sc E-Freeze-Init}:
      Similar to case~\ref{slqc-put-err-beta}, since
      $\config{S_a}{e_{a_2}} = \error$ and $S_b = S$.
    \item \label{slqc-put-err-spawn-handler}Case {\sc
      E-Spawn-Handler}: Similar to case~\ref{slqc-put-err-beta}, since
      $\config{S_a}{e_{a_2}} = \error$ and $S_b = S$.
    \item \label{slqc-put-err-freeze-final}Case {\sc E-Freeze-Final}:
      We have $\config{S_a}{e_{a_2}} = \error$ and $S_b =
      \extS{S}{l}{d_1}{\frozentrue}$, and so we choose $\conf_c =
      \error$, $i = 0$, $j = 1$, and $\pi = \id$.

      We have to show that:
      \begin{itemize}
      \item $\config{S_a}{\evalctxt{E'_b}{e_{b_1}}} = \error$,
        and
      \item $\config{S_b}{\evalctxt{E'_a}{e_{a_1}}} \ctxstepsto
        \error$.
      \end{itemize}

      The first of these is immediately true because since
      $\config{S_a}{e_{a_2}} = \error$, $S_a = \topS$, and so
      $\config{S_a}{\evalctxt{E'_b}{e_{b_1}}}$ is equal to $\error$ as
      well.

      For the second, observe that since $\config{S}{e_{b_1}}
      \parstepsto \config{S_b}{e_{b_2}}$, we have by
      Lemma~\ref{lem:monotonicity} (Monotonicity) that
      $\leqstore{S}{S_b}$.

      Therefore, since $\config{S}{e_{a_1}} \parstepsto \error$, we
      have by Lemma~\ref{lem:error-preservation} that
      $\config{S_b}{e_{a_1}} \parstepsto \error$.

      Since $\error$ is equal to $\config{\topS}{e}$ for all
      expressions $e$, $\config{S_b}{e_{a_1}} \parstepsto
      \config{\topS}{e}$ for all $e$.

      Therefore, by {\sc E-Eval-Ctxt},
      $\config{S_b}{\evalctxt{E'_a}{e_{a_1}}} \ctxstepsto
      \config{\topS}{\evalctxt{E'_a}{e}}$ for all $e$.

      Since $\config{\topS}{\evalctxt{E'_a}{e}}$ is equal to $\error$,
      we have that $\config{S_b}{\evalctxt{E'_a}{e_{a_1}}} \ctxstepsto
      \error$, as we were required to show.

    \item \label{slqc-put-err-freeze-simple}Case {\sc
      E-Freeze-Simple}: Similar to
      case~\ref{slqc-put-err-freeze-final}, since $S_b =
      \extS{S}{l}{d_1}{\frozentrue}$.

    \end{enumerate}
  \item Case {\sc E-Get}: We have $S_a = S$.

    We proceed by case analysis on the rule by which
    $\config{S}{e_{b_1}}$ steps to $\config{S_b}{e_{b_2}}$.

    Since the only way an $\error$ configuration can arise is by the
    {\sc E-Put-Err} rule, we can assume in all other cases that
    $\conf_b \neq \error$.
    \begin{enumerate}
    \item \label{slqc-get-beta}Case {\sc E-Beta}: By symmetry with case~\ref{slqc-beta-get}.
    \item \label{slqc-get-new}Case {\sc E-New}: By symmetry with case~\ref{slqc-new-get}.
    \item \label{slqc-get-put}Case {\sc E-Put}: By symmetry with case~\ref{slqc-put-get}.
    \item \label{slqc-get-put-err}Case {\sc E-Put-Err}: By symmetry with case~\ref{slqc-put-err-get}.
    \item \label{slqc-get-get}Case {\sc E-Get}: Similar to
      case~\ref{slqc-get-beta}, since $S_a = S$ and $S_b = S$.
    \item \label{slqc-get-freeze-init}Case {\sc E-Freeze-Init}:
      Similar to case~\ref{slqc-get-beta}, since $S_a = S$ and $S_b = S$.
    \item \label{slqc-get-spawn-handler}Case {\sc E-Spawn-Handler}:
      Similar to case~\ref{slqc-get-beta}, since $S_a = S$ and $S_b = S$.
    \item \label{slqc-get-freeze-final}Case {\sc E-Freeze-Final}:
      Similar to case~\ref{slqc-beta-freeze-final}, since $S_a = S$
      and $S_b = \extS{S}{l}{d_1}{\frozentrue}$.
    \item \label{slqc-get-freeze-simple}Case {\sc E-Freeze-Simple}:
      Similar to case~\ref{slqc-beta-freeze-simple}, since $S_a = S$
      and $S_b = \extS{S}{l}{d_1}{\frozentrue}$.
    \end{enumerate}

  \item Case {\sc E-Freeze-Init}: We have $S_a = S$.

    We proceed by case analysis on the rule by which
    $\config{S}{e_{b_1}}$ steps to $\config{S_b}{e_{b_2}}$.

    Since the only way an $\error$ configuration can arise is by the
    {\sc E-Put-Err} rule, we can assume in all other cases that
    $\conf_b \neq \error$.
    \begin{enumerate}
    \item \label{slqc-freeze-init-beta}Case {\sc E-Beta}: By symmetry with case~\ref{slqc-beta-freeze-init}.
    \item \label{slqc-freeze-init-new}Case {\sc E-New}: By symmetry with case~\ref{slqc-new-freeze-init}.
    \item \label{slqc-freeze-init-put}Case {\sc E-Put}: By symmetry with case~\ref{slqc-put-freeze-init}.
    \item \label{slqc-freeze-init-put-err}Case {\sc E-Put-Err}: By symmetry with case~\ref{slqc-put-err-freeze-init}.
    \item \label{slqc-freeze-init-get}Case {\sc E-Get}: By symmetry with case~\ref{slqc-get-freeze-init}.
    \item \label{slqc-freeze-init-freeze-init}Case {\sc
      E-Freeze-Init}: Similar to case~\ref{slqc-freeze-init-beta},
      since $S_a = S$ and $S_b = S$.
    \item \label{slqc-freeze-init-spawn-handler}Case {\sc
      E-Spawn-Handler}: Similar to case~\ref{slqc-freeze-init-beta},
      since $S_a = S$ and $S_b = S$.
    \item \label{slqc-freeze-init-freeze-final}Case {\sc
      E-Freeze-Final}: Similar to case~\ref{slqc-beta-freeze-final},
      since $S_a = S$ and $S_b = \extS{S}{l}{d_1}{\frozentrue}$.
    \item \label{slqc-freeze-init-freeze-simple}Case {\sc
      E-Freeze-Simple}: Similar to case~\ref{slqc-beta-freeze-simple},
      since $S_a = S$ and $S_b = \extS{S}{l}{d_1}{\frozentrue}$.
    \end{enumerate}

  \item Case {\sc E-Spawn-Handler}: We have $S_a = S$.

    We proceed by case analysis on the rule by which
    $\config{S}{e_{b_1}}$ steps to $\config{S_b}{e_{b_2}}$.

    Since the only way an $\error$ configuration can arise is by the
    {\sc E-Put-Err} rule, we can assume in all other cases that
    $\conf_b \neq \error$.
    \begin{enumerate}
    \item \label{slqc-spawn-handler-beta}Case {\sc E-Beta}: By symmetry with case~\ref{slqc-beta-spawn-handler}.
    \item \label{slqc-spawn-handler-new}Case {\sc E-New}: By symmetry with case~\ref{slqc-new-spawn-handler}.
    \item \label{slqc-spawn-handler-put}Case {\sc E-Put}: By symmetry with case~\ref{slqc-put-spawn-handler}.
    \item \label{slqc-spawn-handler-put-err}Case {\sc E-Put-Err}: By symmetry with case~\ref{slqc-put-err-spawn-handler}.
    \item \label{slqc-spawn-handler-get}Case {\sc E-Get}: By symmetry with case~\ref{slqc-get-spawn-handler}.
    \item \label{slqc-spawn-handler-freeze-init}Case {\sc E-Freeze-Init}: By symmetry with case~\ref{slqc-freeze-init-spawn-handler}.
    \item \label{slqc-spawn-handler-spawn-handler}Case {\sc
      E-Spawn-Handler}: Similar to case~\ref{slqc-spawn-handler-beta},
      since $S_a = S$ and $S_b = S$.
    \item \label{slqc-spawn-handler-freeze-final}Case {\sc
      E-Freeze-Final}: Similar to case~\ref{slqc-beta-freeze-final},
      since $S_a = S$ and $S_b = \extS{S}{l}{d_1}{\frozentrue}$.
    \item \label{slqc-spawn-handler-freeze-simple}Case {\sc
      E-Freeze-Simple}: Similar to case~\ref{slqc-beta-freeze-simple},
      since $S_a = S$ and $S_b = \extS{S}{l}{d_1}{\frozentrue}$.
    \end{enumerate}

  \item Case {\sc E-Freeze-Final}: We have $S_a =
    \extS{S}{l}{d_1}{\frozentrue}$.

    We proceed by case analysis on the rule by which
    $\config{S}{e_{b_1}}$ steps to $\config{S_b}{e_{b_2}}$.

    Since the only way an $\error$ configuration can arise is by the
    {\sc E-Put-Err} rule, we can assume in all other cases that
    $\conf_b \neq \error$.
    \begin{enumerate}
    \item \label{slqc-freeze-final-beta}Case {\sc E-Beta}: By symmetry with case~\ref{slqc-beta-freeze-final}.
    \item \label{slqc-freeze-final-new}Case {\sc E-New}: By symmetry with case~\ref{slqc-new-freeze-final}.
    \item \label{slqc-freeze-final-put}Case {\sc E-Put}: By symmetry with case~\ref{slqc-put-freeze-final}.
    \item \label{slqc-freeze-final-put-err}Case {\sc E-Put-Err}: By symmetry with case~\ref{slqc-put-err-freeze-final}.
    \item \label{slqc-freeze-final-get}Case {\sc E-Get}: By symmetry with case~\ref{slqc-get-freeze-final}.
    \item \label{slqc-freeze-final-freeze-init}Case {\sc E-Freeze-Init}: By symmetry with case~\ref{slqc-freeze-init-freeze-final}.
    \item \label{slqc-freeze-final-spawn-handler}Case {\sc E-Spawn-Handler}: By symmetry with case~\ref{slqc-spawn-handler-freeze-final}.
    \item \label{slqc-freeze-final-freeze-final}Case {\sc
      E-Freeze-Final}: We have $S_a = \extS{S}{l}{d_1}{\frozentrue}$
      and $S_b = \extS{S}{l'}{d'_1}{\frozentrue}$.

      Now consider whether $l = l'$:
      \begin{itemize}
      \item If $l \neq l'$:

        Choose $S' =
        \extS{\extS{S}{l'}{d'_1}{\frozentrue}}{l}{d_1}{\frozentrue}$,
        $i = 1$, $j = 1$, and $\pi = \id$.

        We have to show that:
        \begin{itemize}
        \item
          $\config{\extS{S}{l}{d_1}{\frozentrue}}{\evalctxt{E'_b}{e_{b_1}}}
          \ctxstepsto
          \config{\extS{\extS{S}{l'}{d'_1}{\frozentrue}}{l}{d_1}{\frozentrue}}{\evalctxt{E'_b}{e_{b_2}}}$,
          and
        \item
          $\config{\extS{S}{l'}{d'_1}{\frozentrue}}{\evalctxt{E'_a}{e_{a_1}}}
          \ctxstepsto
          \config{\extS{\extS{S}{l'}{d'_1}{\frozentrue}}{l}{d_1}{\frozentrue}}{\evalctxt{E'_a}{e_{a_2}}}$.
        \end{itemize}

        For the first of these, consider that
        $\extS{S}{l}{d_1}{\frozentrue} = U_S(S)$, where $U_S$ is the
        store update operation that freezes the contents of $l$
        and acts as the identity on the contents of all other
        locations.

        Note that:
        \begin{itemize}
        \item $U_S$ is non-conflicting with $\config{S}{e_{b_1}}
          \parstepsto
          \config{\extS{S}{l'}{d'_1}{\frozentrue}}{e_{b_2}}$, since
          no locations are allocated in the transition;
        \item $U_S(\extS{S}{l'}{d'_1}{\frozentrue}) \neq \topS$,

          since $U_S(\extS{S}{l'}{d'_1}{\frozentrue}) =
          \extS{\extS{S}{l'}{d'_1}{\frozentrue}}{l}{d_1}{\frozentrue}$
          and we know $S \neq \topS$ and freezing the contents of
          locations $l$ and $l'$ in $S$ cannot cause it to become
          $\topS$ (because if so, then we would have $S_a = \topS$
          or $S_b = \topS$, which we know are not the case); and
        \item $U_S$ is freeze-safe with $\config{S}{e_{b_1}}
          \parstepsto
          \config{\extS{S}{l'}{d'_1}{\frozentrue}}{e_{b_2}}$, since
          the only location that can change in status between $S$
          and $\extS{S}{l'}{d'_1}{\frozentrue}$ is $l'$, and $U_S$
          acts as the identity on $l'$.
        \end{itemize}
        Therefore, by Lemma~\ref{lem:generalized-independence}
        (Generalized Independence), we have that

        $\config{U_S(S)}{e_{b_1}} \parstepsto
        \config{U_S(\extS{S}{l'}{d'_1}{\frozentrue})}{e_{b_2}}$.

        Hence $\config{\extS{S}{l}{d_1}{\frozentrue}}{e_{b_1}}
        \parstepsto
        \config{\extS{\extS{S}{l'}{d'_1}{\frozentrue}}{l}{d_1}{\frozentrue}}{e_{b_2}}$.

        By {\sc E-Eval-Ctxt}, it follows that

        $\config{\extS{S}{l}{d_1}{\frozentrue}}{\evalctxt{E'_b}{e_{b_1}}}
        \ctxstepsto
        \config{\extS{\extS{S}{l'}{d'_1}{\frozentrue}}{l}{d_1}{\frozentrue}}{\evalctxt{E'_b}{e_{b_2}}}$,
        
        as we were required to show.

        The argument for the second is symmetrical.

      \item If $l = l'$:

        \lk{This is the case where we freeze the same location twice,
          which is no problem; the second freeze is a no-op.}

        Note that since $l = l'$, $d_1 = d'_1$ as well.

        Choose $S' = \extS{S}{l}{d_1}{\frozentrue}$, $i = 1$, $j =
        1$, and $\pi = \id$.

        We have to show that:
        \begin{itemize}
        \item
          $\config{\extS{S}{l}{d_1}{\frozentrue}}{\evalctxt{E'_b}{e_{b_1}}}
          \ctxstepsto
          \config{\extS{S}{l}{d_1}{\frozentrue}}{\evalctxt{E'_b}{e_{b_2}}}$,
          and
        \item
          $\config{\extS{S}{l'}{d'_1}{\frozentrue}}{\evalctxt{E'_a}{e_{a_1}}}
          \ctxstepsto
          \config{\extS{S}{l}{d_1}{\frozentrue}}{\evalctxt{E'_a}{e_{a_2}}}$.
        \end{itemize}

        For the first of these, consider that
        $\extS{S}{l}{d_1}{\frozentrue} = U_S(S)$, where $U_S$ is the
        store update operation that freezes the contents of $l$ and
        acts as the identity on the contents of all other locations.

        Note that:
        \begin{itemize}
        \item $U_S$ is non-conflicting with $\config{S}{e_{b_1}}
          \parstepsto
          \config{\extS{S}{l}{d_1}{\frozentrue}}{e_{b_2}}$, since no
          locations are allocated in the transition;
        \item $U_S(\extS{S}{l}{d_1}{\frozentrue}) \neq \topS$, since
          $U_S(\extS{S}{l}{d_1}{\frozentrue}) =
          \extS{S}{l}{d_1}{\frozentrue}$, and we know $S \neq \topS$
          and freezing the contents of location $l$ in $S$ cannot
          cause it to become $\topS$; and
        \item $U_S$ is freeze-safe with $\config{S}{e_{b_1}}
          \parstepsto
          \config{\extS{S}{l}{d_1}{\frozentrue}}{e_{b_2}}$, since
          the only location that can change in status between $S$
          and $\extS{S}{l}{d_1}{\frozentrue}$ is $l$, and $U_S$
          freezes the contents of $l$ but has no other effect on
          them.
        \end{itemize}

        Therefore, by Lemma~\ref{lem:generalized-independence}
        (Generalized Independence), we have that

        $\config{U_S(S)}{e_{b_1}} \parstepsto
        \config{U_S(\extS{S}{l}{d_1}{\frozentrue})}{e_{b_2}}$.

        Hence $\config{\extS{S}{l}{d_1}{\frozentrue}}{e_{b_1}}
        \parstepsto
        \config{\extS{S}{l}{d_1}{\frozentrue}}{e_{b_2}}$.

        By {\sc E-Eval-Ctxt}, it follows that

        $\config{\extS{S}{l}{d_1}{\frozentrue}}{\evalctxt{E'_b}{e_{b_1}}}
        \ctxstepsto
        \config{\extS{S}{l}{d_1}{\frozentrue}}{\evalctxt{E'_b}{e_{b_2}}}$,

        as we were required to show.

        The argument for the second is symmetrical.

      \end{itemize}

    \item \label{slqc-freeze-final-freeze-simple}Case {\sc
      E-Freeze-Simple}: Similar to
      case~\ref{slqc-freeze-final-freeze-final}, since $S_a =
      \extS{S}{l}{d_1}{\frozentrue}$ and $S_b =
      \extS{S}{l'}{d'_1}{\frozentrue}$.
    \end{enumerate}

  \item Case {\sc E-Freeze-Simple}: We have $S_a =
    \extS{S}{l}{d_1}{\frozentrue}$.

    \begin{enumerate}
    \item \label{slqc-freeze-simple-beta}Case {\sc E-Beta}: By symmetry with case~\ref{slqc-beta-freeze-simple}.
    \item \label{slqc-freeze-simple-new}Case {\sc E-New}: By symmetry with case~\ref{slqc-new-freeze-simple}.
    \item \label{slqc-freeze-simple-put}Case {\sc E-Put}: By symmetry with case~\ref{slqc-put-freeze-simple}.
    \item \label{slqc-freeze-simple-put-err}Case {\sc E-Put-Err}: By symmetry with case~\ref{slqc-put-err-freeze-simple}.
    \item \label{slqc-freeze-simple-get}Case {\sc E-Get}: By symmetry with case~\ref{slqc-get-freeze-simple}.
    \item \label{slqc-freeze-simple-freeze-init}Case {\sc E-Freeze-Init}: By symmetry with case~\ref{slqc-freeze-init-freeze-simple}.
    \item \label{slqc-freeze-simple-spawn-handler}Case {\sc E-Spawn-Handler}: By symmetry with case~\ref{slqc-spawn-handler-freeze-simple}.
    \item \label{slqc-freeze-simple-freeze-final}Case {\sc E-Freeze-Final}: By symmetry with case~\ref{slqc-freeze-final-freeze-simple}.
    \item \label{slqc-freeze-simple-freeze-simple}Case {\sc
      E-Freeze-Simple}: Similar to
      case~\ref{slqc-freeze-simple-freeze-final}, since $S_a =
      \extS{S}{l}{d_1}{\frozentrue}$ and $S_b =
      \extS{S}{l'}{d'_1}{\frozentrue}$.
    \end{enumerate}

  \end{enumerate}
\end{proof}



\section{Proof of Lemma~\ref{lem:strong-one-sided-quasi-confluence}}\label{section:strong-one-sided-quasi-confluence-proof}
\begin{proof}
  Suppose $\conf \ctxstepsto \conf'$ and $\conf \ctxstepsto^m
  \conf''$, where $1 \leq m$.

  We are required to show that either:
  \begin{enumerate}
  \item there exist $\conf_c, i, j, \pi$ such that $\conf'
    \ctxstepsto^i \conf_c$ and $\pi(\conf'') \ctxstepsto^j \conf_c$
    and $i \leq m$ and $j \leq 1$, or
  \item there exists $k \leq m$ such that $\conf' \ctxstepsto^k
    \textup{\error}$, or there exists $k \leq 1$ such that $\conf''
    \ctxstepsto^k \textup{\error}$.
  \end{enumerate}

  We proceed by induction on $m$.

  In the base case of $m = 1$, the result is immediate from
  Lemma~\ref{lem:strong-local-quasi-confluence}, with $k = 1$.

  For the induction step, suppose $\conf \ctxstepsto^m \conf''
  \ctxstepsto \conf'''$ and suppose the lemma holds for $m$.

  We show that it holds for $m + 1$, as follows.

  From the induction hypothesis, we have that either:
  \begin{enumerate}
  \item there exist $\conf_c', i', j', \pi'$ such that $\conf'
    \ctxstepsto^{i'} \conf_c'$ and $\pi'(\conf'') \ctxstepsto^{j'}
    \conf_c'$ and $i' \leq m$ and $j' \leq 1$, or
  \item there exists $k' \leq m$ such that $\conf'
    \ctxstepsto^{k'} \error$, or there exists $k' \leq 1$ such that
    $\conf'' \ctxstepsto^{k'} \error$.
  \end{enumerate}

  We consider these two cases in turn:
  \begin{enumerate}
  \item There exist $\conf_c', i', j', \pi'$ such that $\conf'
    \ctxstepsto^{i'} \conf_c'$ and $\pi'(\conf'') \ctxstepsto^{j'}
    \conf_c'$ and $i' \leq m$ and $j' \leq 1$:

    We proceed by cases on $j'$:
    \begin{itemize}

    \item If $j' = 0$, then $\pi'(\conf'') = \conf_c'$.

      Since $\conf'' \ctxstepsto \conf'''$, we have that
      $\pi'(\conf'') \ctxstepsto \pi'(\conf''')$ by
      Lemma~\ref{lem:permutability} (Permutability).

      We can then choose $\conf_c = \pi'(\conf''')$ and $i = i' + 1$
      and $j = 0$ and $\pi = \pi'$.

      The key is that $\conf' \ctxstepsto^{i'} \conf'_c =
      \pi'(\conf'') \ctxstepsto \pi'(\conf''')$ for a total of $i' +
      1$ steps.
      
    \item If $j' = 1$:

      First, since $\pi'(\conf'') \ctxstepsto^{j'} \conf'_c$, then
      by Lemma~\ref{lem:permutability} (Permutability) we have that
      $\conf'' \ctxstepsto^{j'} \piprimeinv(\conf'_c)$.
      
      Then, by $\conf'' \ctxstepsto^{j'} \piprimeinv(\conf'_c)$ and
      $\conf'' \ctxstepsto \conf'''$ and
      Lemma~\ref{lem:strong-local-quasi-confluence}, one of the
      following two cases is true:
      \begin{enumerate}
      \item There exist $\conf_c''$ and $i''$ and $j''$ and $\pi''$
        such that $\piprimeinv(\conf'_c) \ctxstepsto^{i''}
        \conf_c''$ and $\pi''(\conf''') \ctxstepsto^{j''} \conf_c''$
        and $i'' \leq 1$ and $j'' \leq 1$.

        Since $\piprimeinv(\conf'_c) \ctxstepsto^{i''} \conf_c''$,
        by Lemma~\ref{lem:permutability} (Permutability) we have
        that $\conf'_c \ctxstepsto^{i''} \pi'(\conf_c'')$.

        So we also have $\conf' \ctxstepsto^{i'} \conf_c'
        \ctxstepsto^{i''} \pi'(\conf_c'')$.

        Since $\pi''(\conf''') \ctxstepsto^{j''} \conf_c''$, by
        Lemma~\ref{lem:permutability} (Permutability) we have that
        $\pi'(\pi''(\conf''')) \ctxstepsto^{j''} \pi'(\conf_c'')$.

        In summary, we pick $\conf_c = \pi'(\conf_c'')$ and $i = i' + i''$
        and $j = j''$ and $\pi = \pi'' \circ \pi'$, which is sufficient
        because $i = i' + i'' \leq m + 1$ and $j = j'' \leq 1$.

      \item $\piprimeinv(\conf'_c) \ctxstepsto \error$ or $\conf'''
        \ctxstepsto \error$.

        If $\conf''' \ctxstepsto \error$, then choosing $k = 1$
        satisfies the proof.

        Otherwise, $\piprimeinv(\conf'_c) \ctxstepsto \error$.

        Then, by Lemma~\ref{lem:permutability} we have that
        $\conf'_c \ctxstepsto \pi'(\error)$.

        By Definition~\ref{def:permutation-configuration},
        $\pi'(\error) = \error$, and so $\conf'_c \ctxstepsto
        \error$.

        Therefore $\conf' \ctxstepsto^{i'} \conf'_c \ctxstepsto
        \error$.

        Hence $\conf' \ctxstepsto^{i'+1} \error$.

        Since $i' \leq m$, we have that $i' + 1 \leq m + 1$, and
        so choosing $k = i' + 1$ satisfies the proof.
        
      \end{enumerate}

    \end{itemize}

  \item There exists $k' \leq m$ such that $\conf' \ctxstepsto^{k'}
    \error$, or there exists $k' \leq 1$ such that $\conf''
    \ctxstepsto^{k'} \error$:

    If there exists $k' \leq m$ such that $\conf' \ctxstepsto^{k'}
    \error$, then choosing $k = k'$ satisfies the proof.

    Otherwise, there exists $k' \leq 1$ such that $\conf''
    \ctxstepsto^{k'} \error$.

    We proceed by cases on $k'$:

    \begin{itemize}

    \item If $k' = 0$, then $\conf'' = \error$.

      Hence this case is not possible, since $\conf'' \ctxstepsto
      \conf'''$ and $\error$ cannot step.

    \item If $k' = 1$:

      From $\conf'' \ctxstepsto \conf'''$ and $\conf''
      \ctxstepsto^{k'} \error$ and
      Lemma~\ref{lem:strong-local-quasi-confluence}, one of the
      following two cases is true:

      \begin{enumerate}
      \item There exist $\conf_c''$ and $i''$ and $j''$ and $\pi''$
        such that $\error \ctxstepsto^{i''} \conf_c''$ and
        $\pi''(\conf''') \ctxstepsto^{j''} \conf_c''$ and $i'' \leq
        1$ and $j'' \leq 1$.

        Since $\error$ cannot step, $i'' = 0$ and $\conf''_c =
        \error$.

        By Definition~\ref{def:permutation-configuration},
        $\pi''(\conf''') = \conf'''$.

        Hence $\conf''' \ctxstepsto^{j''} \error$.

        \lk{This is the one place that we need to allow $k$ to be
          $\leq$ 1 instead of exactly 1.}

        Since $j'' \leq 1$, choosing $k = j''$ satisfies the proof.

      \item $\error \ctxstepsto \error$ or $\conf''' \ctxstepsto
        \error$.

        Since $\error$ cannot step, $\conf''' \ctxstepsto \error$.

        Hence choosing $k = 1$ satisfies the proof.

      \end{enumerate}

    \end{itemize}

  \end{enumerate}

\end{proof}


\section{Proof of Lemma~\ref{lem:strong-quasi-confluence}}\label{section:strong-quasi-confluence-proof}
\begin{proof}
  We proceed by induction on $n$.  In the base case of $n = 1$, the
  result is immediate from Lemma~\ref{lem:strong-one-sided-quasi-confluence}.

  For the induction step, suppose $\conf \parstepsto^n \conf'
  \parstepsto \conf'''$ and suppose the lemma holds for $n$.

  We show that it holds for $n + 1$, as follows.

  We are required to show that either:
  \begin{enumerate}
  \item there exist $\conf_c, i, j$ such that $\conf''' \parstepsto^i
    \conf_c$ and $\conf'' \parstepsto^j \conf_c$ and $i \leq m$ and $j
    \leq n + 1$, or
  \item there exists $k \leq m$ such that $\conf''' \parstepsto^k
    \error$, or there exists $k \leq n + 1$ such that $\conf''
    \parstepsto^k \error$.
  \end{enumerate}

  From the induction hypothesis, we have that either:
  \begin{enumerate}
  \item there exist $\conf'_c, i', j'$ such that $\conf'
    \parstepsto^{i'} \conf'_c$ and $\conf'' \parstepsto^{j'} \conf'_c$
    and $i' \leq m$ and $j' \leq n$, or
  \item there exists $k' \leq m$ such that $\conf' \parstepsto^{k'}
    \error$, or there exists $k' \leq n$ such that $\conf''
    \parstepsto^{k'} \error$.
  \end{enumerate}

  We consider these two cases in turn:

  \begin{enumerate}
  \item There exist $\conf'_c, i', j'$ such that $\conf'
    \parstepsto^{i'} \conf'_c$ and $\conf'' \parstepsto^{j'} \conf'_c$
    and $i' \leq m$ and $j' \leq n$:

    We proceed by cases on $i'$:
    \begin{itemize}

    \item If $i' = 0$, then $\conf' = \conf_c'$.  We can then choose
      $\conf_c = \conf'''$ and $i = 0$ and $j = j' + 1$.

    \item If $i' \geq 1$:

      From $\conf' \parstepsto \conf'''$ and $\conf' \parstepsto^{i'}
      \conf_c'$ and Lemma~\ref{lem:strong-one-sided-quasi-confluence},
      one of the following two cases is true:
      \begin{enumerate}
        \item There exist $\conf_c''$ and $i''$ and $j''$ such that
          $\conf''' \parstepsto^{i''} \conf_c''$ and $\conf_c'
          \parstepsto^{j''} \conf_c''$ and $i'' \leq i'$ and $j'' \leq
          1$.  So we also have $\conf'' \parstepsto^{j'} \conf_c'
          \parstepsto^{j''} \conf_c''$.  In summary, we pick $\conf_c
          = \conf_c''$ and $i = i''$ and $j = j' + j''$, which is
          sufficient because $i = i'' \leq i' \leq m$ and $j = j' +
          j'' \leq n + 1$.
        \item There exists $k'' \leq i'$ such that $\conf'''
          \parstepsto^{k''} \error$, or there exists $k'' \leq 1$ such
          that $\conf'_c \parstepsto^{k''} \error$.

          If there exists $k'' \leq i'$ such that $\conf'''
          \parstepsto^{k''} \error$, then choosing $k = k''$ satisfies
          the proof, since $k'' \leq i' \leq m$.

          Otherwise, there exists $k'' \leq 1$ such
          that $\conf'_c \parstepsto^{k''} \error$.

          Therefore, $\conf'' \parstepsto^{j'} \conf_c'
          \parstepsto^{k''} \error$.

          Hence $\conf'' \parstepsto^{j' + k''} \error$.

          Since $j' \leq n$ and $k'' \leq 1$, $j' + k'' \leq n + 1$.

          Hence choosing $k = j' + k''$ satisfies the proof.

      \end{enumerate}
    \end{itemize}

  \item There exists $k' \leq m$ such that $\conf' \parstepsto^{k'}
    \error$, or there exists $k' \leq n$ such that $\conf''
    \parstepsto^{k'} \error$:

    If there exists $k' \leq n$ such that $\conf'' \parstepsto^{k'}
    \error$, then choosing $k = k'$ satisfies the proof.

    Otherwise, there exists $k' \leq m$ such that $\conf'
    \parstepsto^{k'} \error$.  We proceed by cases on $k'$:

    \begin{itemize}

    \item If $k' = 0$, then $\conf' = \error$.

      Hence this case is not possible, since $\conf' \parstepsto
      \conf'''$ and $\error$ cannot step.

    \item If $k' \geq 1$:

      From $\conf' \parstepsto \conf'''$ and $\conf' \parstepsto^{k'}
      \error$ and Lemma~\ref{lem:strong-one-sided-quasi-confluence},
      one of the following two cases is true:

      \begin{enumerate}
        \item There exist $\conf''_c$ and $i''$ and $j''$ such that
          $\conf''' \parstepsto^{i''} \conf''_c$ and $\error
          \parstepsto^{j''} \conf''_c$ and $i'' \leq k'$ and $j'' \leq
          1$.

          Since $\error$ cannot step, $j'' = 0$ and $\conf''_c =
          \error$.

          Hence $\conf''' \parstepsto^{i''} \error$.

          Since $i'' \leq k' \leq m$, choosing $k = i''$ satisfies the
          proof.

        \item There exists $k'' \leq k'$ such that $\conf'''
          \parstepsto^{k''} \error$, or there exists $k'' \leq 1$ such
          that $\error \parstepsto^{k''} \error$.

          Since $\error$ cannot step, there exists $k'' \leq k'$ such
          that $\conf''' \parstepsto^{k''} \error$.

          Since $k'' \leq k' \leq m$, choosing $k = k''$ satisfies the
          proof.
      \end{enumerate}
    \end{itemize}
  \end{enumerate}

\end{proof}


\section{Proof of Theorem~\ref{thm:determinism-of-threshold-queries}}\label{section:determinism-of-threshold-queries-proof}
\begin{proof}
  Consider replica $i$ of a threshold CvRDT $(S, \leq, s^0, q, t, u,
  m)$.

  Let $\mathcal{S}$ be a threshold set with respect to
  $(S, \leq)$.

  Consider a method execution $t^{k+1}_i(\mathcal{S})$ (\ie, a
  threshold query that is the $k+1$th method execution on replica $i$,
  with threshold set $\mathcal{S}$ as its argument) that returns some
  set of activation states $S_a \in \mathcal{S}$.

  For part~\ref{thm:this-replica} of the theorem, we have to show that
  threshold queries with $\mathcal{S}$ as their argument will always
  return $S_a$ on subsequent executions at $i$.

  That is, we have to show that, for all $k' > (k+1)$, the threshold
  query $t^{k'}_i(\mathcal{S})$ on $i$ returns $S_a$.

  Since $t^{k+1}_i(\mathcal{S})$ returns $S_a$, from
  Definition~\ref{def:cvrdt-with-threshold-queries} we have that for
  some activation state $s_a \in S_a$, the condition $s_a \leq s^k_i$
  holds.

  Consider arbitrary $k' > (k+1)$.

  Since state is inflationary across updates, we know that the state
  $s^{k'}_i$ after method execution $k'$ is at least $s^k_i$.

  That is, $s^k_i \leq s^{k'}_i$.

  By transitivity of $\leq$, then, $s_a \leq s^{k'}_i$.

  Hence, by Definition~\ref{def:cvrdt-with-threshold-queries},
  $t^{k'}_i(\mathcal{S})$ returns $S_a$.

  For part~\ref{thm:any-replica} of the theorem, consider some replica
  $j$ of $(S, \leq, s^0, q, t, u, m)$, located at process $p_j$.

  We are required to show that, for all $x \geq 0$, the threshold
  query $t^{x+1}_j(\mathcal{S})$ returns $S_a$ eventually, and blocks
  until it does.\footnote{The occurrences of $k+1$ and $x+1$ in this
    proof are an artifact of how we index method executions starting
    from $1$, but states starting from $0$.  The initial state (of
    every replica) is $s^0$, and so $s^k_i$ is the state of replica
    $i$ after method execution $k$ has completed at $i$.}

  That is, we must show that, for all $x \geq 0$, there exists some
  finite $n \geq 0$ such that
  \begin{itemize}
  \item 
    for all $i$ in the range $0 \leq i \leq n-1$, the threshold query
    $t^{x+1+i}_j(\mathcal{S})$ returns $\block$, and
  \item
    for all $i \geq n$, the threshold query $t^{x+1+i}_j(\mathcal{S})$
    returns $S_a$.
  \end{itemize}
  Consider arbitrary $x \geq 0$.

  Recall that $s^x_j$ is the state of replica $j$ after the $x$th
  method execution, and therefore $s^x_j$ is also the state of $j$
  when $t^{x+1}_j(\mathcal{S})$ runs.
  %
  We have three cases to consider:
  \begin{itemize}
  \item $s^k_i \leq s^x_j$.

    (That is, replica $i$'s state after the $k$th method execution on $i$
    is \emph{at or below} replica $j$'s state after the $x$th method
    execution on $j$.)

    Choose $n = 0$.

    We have to show that, for all $i \geq n$, the threshold query
    $t^{x+1+i}_j(\mathcal{S})$ returns $S_a$.

    Since $t^{k+1}_i(\mathcal{S})$ returns $S_a$, we know that there
    exists an $s_a \in S_a$ such that $s_a \leq s^k_i$.

    Since $s^k_i \leq s^x_j$, we have by transitivity of $\leq$ that
    $s_a \leq s^x_j$.

    Therefore, by Definition~\ref{def:cvrdt-with-threshold-queries},
    $t^{x+1}_j(\mathcal{S})$ returns $S_a$.

    Then, by part~\ref{thm:this-replica} of the theorem, we have that
    subsequent executions $t^{x+1+i}_j(\mathcal{S})$ at replica $j$
    will also return $S_a$, and so the case holds.

    (Note that this case includes the possibility $s^k_i \equiv s^0$,
    in which no updates have executed at replica $i$.)

  \item $s^k_i > s^x_j$.

    (That is, replica $i$'s state after the $k$th method execution on $i$
    is \emph{above} replica $j$'s state after the $x$th method execution
    on $j$.)

    We have two subcases:

    \begin{itemize}
    \item
      There exists some activation state $s'_a \in S_a$ for which $s'_a \leq
      s^x_j$.

      In this case, we choose $n = 0$.

      We have to show that, for all $i \geq n$, the threshold query
      $t^{x+1+i}_j(\mathcal{S})$ returns $S_a$.

      Since $s'_a \leq s^x_j$, by
      Definition~\ref{def:cvrdt-with-threshold-queries},
      $t^{x+1}_j(\mathcal{S})$ returns $S_a$.

      Then, by part~\ref{thm:this-replica} of the theorem, we have
      that subsequent executions $t^{x+1+i}_j(\mathcal{S})$ at replica
      $j$ will also return $S_a$, and so the case holds.

    \item
      There is no activation state $s'_a \in S_a$ for which $s'_a \leq
      s^x_j$.

      Since $t^{k+1}_i(\mathcal{S})$ returns $S_a$, we know that there
      is some update $u^{k'}_i(a)$ in $i$'s causal history, for some
      $k' < (k+1)$, that updates $i$ from a state at or below $s^x_j$
      to $s^k_i$.\footnote{We know that $i$'s state was once at or
        below $s^x_j$, because $i$ and $j$ started at the same state
        $s^0$ and can both only grow.  Hence the least that $s^x_j$
        can be is $s^0$, and we know that $i$ was originally $s^0$ as
        well.}

      By eventual delivery, $u^{k'}_i(a)$ is eventually delivered at
      $j$.

      Hence some update or updates that will increase $j$'s state from
      $s^x_j$ to a state at or above some $s'_a$ must reach replica
      $j$.\footnote{We say ``some update or updates'' because the
        exact update $u^{k'}_i(a)$ may not be the update that causes
        the threshold query at $j$ to unblock; a different update or
        updates could do it.  Nevertheless, the existence of
        $u^{k'}_i(a)$ means that there is at least one update that
        will suffice to unblock the threshold query.}

      Let the $x+1+r$th method execution on $j$ be the first update on $j$
      that updates its state to some $s^{x+1+r}_j \geq s'_a$, for some
      activation state $s'_a \in S_a$.

      Choose $n = r+1$.

      We have to show that, for all $i$ in the range $0 \leq i \leq
      r$, the threshold query $t^{x+1+i}_j(\mathcal{S})$ returns
      $\block$, and that for all $i \geq r+1$, the threshold query
      $t^{x+1+i}_j(\mathcal{S})$ returns $S_a$.

      For the former, since the $x+1+r$th method execution on $j$ is the
      first one that updates its state to $s^{x+1+r}_j \geq s'_a$, we have
      by Definition~\ref{def:cvrdt-with-threshold-queries} that for all $i$
      in the range $0 \leq i \leq r$, the threshold query
      $t^{x+1+i}_j(\mathcal{S})$ returns $\block$.

      For the latter, since $s^{x+1+r}_j \geq s'_a$, by
      Definition~\ref{def:cvrdt-with-threshold-queries} we have that
      $t^{x+1+r+1}_j(\mathcal{S})$ returns $S_a$, and by
      part~\ref{thm:this-replica} of the theorem, we have that for $i \geq
      r+1$, subsequent executions $t^{x+1+i}_j(\mathcal{S})$ at replica $j$
      will also return $S_a$, and so the case holds.
    \end{itemize}

  \item $s^k_i \nleq s^x_j$ and $s^x_j \nleq s^k_i$.

    (That is, replica $i$'s state after the $k$th method execution on $i$
    is \emph{not comparable} to replica $j$'s state after the $x$th method
    execution on $j$.)

    Similar to the previous case.
  \end{itemize}
\end{proof}



\chapter{Proofs}\label{app:proofs}

\section{Proof of Lemma~\ref{lem:lvars-permutability}}\label{section:lvars-permutability-proof}
\begin{proof}
  Consider an arbitrary permutation $\pi$.  For
  part~\ref{thm:permutable-reduction-transitions}, we have to show
  that if $\conf \parstepsto \conf'$ then $\pi(\conf) \parstepsto
  \pi(\conf')$, and that if $\pi(\conf) \parstepsto \pi(\conf')$ then
  $\conf \parstepsto \conf'$.

  For the forward direction of
  part~\ref{thm:permutable-reduction-transitions}, suppose $\conf
  \parstepsto \conf'$.  We have to show that $\pi(\conf) \parstepsto
  \pi(\conf')$.  We proceed by cases on the rule by which $\conf$
  steps to $\conf'$.

  \begin{itemize}
    \item Case {\sc E-Beta}: $\conf =
      \config{S}{\app{(\lam{x}{e})}{v}}$, and $\conf' =
      \config{S}{\subst{e}{x}{v}}$.

      To show: $\pi(\config{S}{\app{(\lam{x}{e})}{v}}) \parstepsto
      \pi(\config{S}{\subst{e}{x}{v}})$.

      By Definitions~\ref{def:lvars-permutation-configuration}
      and~\ref{def:lvars-permutation-expression}, $\pi(\conf) =
      \config{\pi(S)}{\app{(\lam{x}{\pi(e)})}{\pi(v)}}$.

      By {\sc E-Beta},
      $\config{\pi(S)}{\app{(\lam{x}{\pi(e)})}{\pi(v)}}$ steps to
      $\config{\pi(S)}{\subst{\pi(e)}{x}{\pi(v)}}$.

      By Definition~\ref{def:lvars-permutation-expression},
      $\config{\pi(S)}{\subst{\pi(e)}{x}{\pi(v)}}$ is equal to
      $\config{\pi(S)}{\pi(\subst{e}{x}{v})}$.

      Hence $\config{\pi(S)}{\app{(\lam{x}{\pi(e)})}{\pi(v)}}$ steps
      to $\config{\pi(S)}{\pi(\subst{e}{x}{v})}$,

      which is equal to $\pi(\config{S}{\subst{e}{x}{v}})$ by
      Definition~\ref{def:lvars-permutation-configuration}.  Hence the
      case is satisfied.

    \item Case {\sc E-New}: $\conf = \config{S}{\NEW}$, and $\conf' =
      \config{\extSRaw{S}{l}{\bot}}{l}$.

      To show: $\pi(\config{S}{\NEW}) \parstepsto
      \pi(\config{\extSRaw{S}{l}{\bot}}{l})$.

      By Definitions~\ref{def:lvars-permutation-configuration}
      and~\ref{def:lvars-permutation-expression}, $\pi(\conf) =
      \config{\pi(S)}{\NEW}$.

      By {\sc E-New}, $\config{\pi(S)}{\NEW}$ steps to
      $\config{\extSRaw{(\pi(S))}{l'}{\bot}}{l'}$, where $l' \notin
      \dom{\pi(S)}$.
      
      It remains to show that
      $\config{\extSRaw{(\pi(S))}{l'}{\bot}}{l'}$ is equal to
      $\pi(\config{\extSRaw{S}{l}{\bot}}{l})$.

      By Definition~\ref{def:lvars-permutation-configuration},
      $\pi(\config{\extSRaw{S}{l}{\bot}}{l})$ is equal to
      $\config{\pi(\extSRaw{S}{l}{\bot})}{\pi(l)}$,

      which is equal to
      $\config{\extSRaw{(\pi(S))}{\pi(l)}{\bot}}{\pi(l)}$.

      So, we have to show that
      $\config{\extSRaw{(\pi(S))}{l'}{\bot}}{l'}$ is equal to
      $\config{\extSRaw{(\pi(S))}{\pi(l)}{\bot}}{\pi(l)}$.  Since we
      know (from the side condition of {\sc E-New}) that $l \notin
      \dom{S}$, it follows that $\pi(l) \notin \pi(\dom{S})$.
      Therefore, in $\config{\extSRaw{(\pi(S))}{l'}{\bot}}{l'}$, we
      can $\alpha$-rename $l'$ to $\pi(l)$, and so the two
      configurations are equal and the case is satisfied.

    \item Case {\sc E-Put}: $\conf = \config{S}{\putexp{l}{d_2}}$, and
      $\conf' = \config{\extSRaw{S}{l}{\userlub{d_1}{d_2}}}{\unit}$.

      To show: $\pi(\config{S}{\putexp{l}{d_2}}) \parstepsto
      \pi(\config{\extSRaw{S}{l}{\userlub{d_1}{d_2}}}{\unit})$.

      By Definitions~\ref{def:lvars-permutation-configuration}
      and~\ref{def:lvars-permutation-expression}, $\pi(\conf) =
      \config{\pi(S)}{\putexp{\pi(l)}{d_2}}$.

      By {\sc E-Put}, $\config{\pi(S)}{\putexp{\pi(l)}{d_2}}$ steps to
      $\config{\extSRaw{(\pi(S))}{\pi(l)}{\userlub{d_1}{d_2}}}{\unit}$,

      since $S(l) = (\pi(S))(\pi(l)) = d_1$.

      It remains to show that
      $\config{\extSRaw{(\pi(S))}{\pi(l)}{\userlub{d_1}{d_2}}}{\unit}$
      is equal to
      $\pi(\config{\extSRaw{S}{l}{\userlub{d_1}{d_2}}}{\unit})$.

      By Definitions~\ref{def:lvars-permutation-configuration}
      and~\ref{def:lvars-permutation-expression},
      $\pi(\config{\extSRaw{S}{l}{\userlub{d_1}{d_2}}}{\unit})$ is
      equal to
      $\config{\extSRaw{(\pi(S))}{\pi(l)}{\userlub{d_1}{d_2}}}{\unit}$,
      and so the two configurations are equal and the case is
      satisfied.

    \item Case {\sc E-Put-Err}: $\conf = \config{S}{\putexp{l}{d_2}}$,
      and $\conf' = \error$.

      To show: $\pi(\config{S}{\putexp{l}{d_2}}) \parstepsto
      \pi(\error)$.

      By Definitions~\ref{def:lvars-permutation-configuration}
      and~\ref{def:lvars-permutation-expression}, $\pi(\conf) =
      \config{\pi(S)}{\putexp{\pi(l)}{d_2}}$.

      By {\sc E-Put-Err}, $\config{\pi(S)}{\putexp{\pi(l)}{d_2}}$
      steps to $\error$,

      since $S(l) = (\pi(S))(\pi(l)) = d_1$.

      Since $\pi(\error) = \error$ by
      Definition~\ref{def:lvars-permutation-configuration}, the case
      is complete.

    \item Case {\sc E-Get}: $\conf = \config{S}{\getexp{l}{T}}$, and
      $\conf' = \config{S}{d_2}$.

      To show: $\pi(\config{S}{\getexp{l}{T}}) \parstepsto
      \pi(\config{S}{d_2})$.

      By Definitions~\ref{def:lvars-permutation-configuration}
      and~\ref{def:lvars-permutation-expression}, $\pi(\conf) =
      \config{\pi(S)}{\getexp{\pi(l)}{T}}$.

      By {\sc E-Get}, $\config{\pi(S)}{\getexp{\pi(l)}{T}}$ steps to
      $\config{\pi(S)}{d_2}$,

      since $S(l) = (\pi(S))(\pi(l)) = d_1$.

      By Definitions~\ref{def:lvars-permutation-configuration}
      and~\ref{def:lvars-permutation-expression},
      $\pi(\config{S}{d_2}) \config{\pi(S)}{d_2}$.  Therefore the case
      is complete.
  \end{itemize}

  For the reverse direction of
  part~\ref{thm:permutable-reduction-transitions}, suppose $\pi(\conf)
  \parstepsto \pi(\conf')$.  We have to show that $\conf \parstepsto
  \conf'$.

  We know from the forward direction of the proof that for all
  configurations $\conf$ and $\conf'$ and permutations $\pi$, if
  $\conf \parstepsto \conf'$ then $\pi(\conf) \parstepsto
  \pi(\conf')$.  Hence since $\pi(\conf) \parstepsto \pi(\conf')$, and
  since $\piinv$ is also a permutation, we have that
  $\piinv(\pi(\conf)) \parstepsto \piinv(\pi(\conf'))$.  Since
  $\piinv(\pi(l)) = l$ for every $l \in \Loc$, and that property lifts
  to configurations as well, we have that $\conf \parstepsto \conf'$.

  \lk{Is the above enough of a proof?}

  For the forward direction of
  part~\ref{thm:permutable-context-transitions}, suppose $\conf
  \ctxstepsto \conf'$.  We have to show that $\pi(\conf) \ctxstepsto
  \pi(\conf')$.

  By inspection of the operational semantics, $\conf$ must be of the
  form $\config{S}{\E{e}}$, and $\conf'$ must be of the form
  $\config{S'}{\E{e'}}$.  Hence we have to show that
  $\pi(\config{S}{\E{e}}) \ctxstepsto \pi(\config{S'}{\E{e'}})$.

  By Definition~\ref{def:lvars-permutation-configuration},
  $\pi(\config{S}{\E{e}})$ is equal to $\config{\pi(S)}{\pi(\E{e})}$.

  Also by Definition~\ref{def:lvars-permutation-configuration},
  $\pi(\config{S'}{\E{e'}})$ is equal to
  $\config{\pi(S')}{\pi(\E{e'})}$.

  Furthermore, $\config{\pi(S)}{\pi(\E{e})}$ is equal to
  $\config{\pi(S)}{\evalctxt{(\pi(E))}{\pi(e)}}$ and
  $\config{\pi(S')}{\pi(\E{e'})}$ is equal to
  $\config{\pi(S')}{\evalctxt{(\pi(E))}{\pi(e')}}$.

  So we have to show that
  $\config{\pi(S)}{\evalctxt{(\pi(E))}{\pi(e)}} \ctxstepsto
  \config{\pi(S')}{\evalctxt{(\pi(E))}{\pi(e')}}$.

  From the premise of {\sc E-Eval-Ctxt}, $\config{S}{e} \parstepsto
  \config{S'}{e'}$.  Hence, by
  part~\ref{thm:permutable-reduction-transitions}, $\pi(\config{S}{e})
  \parstepsto \pi(\config{S'}{e'})$.  By
  Definition~\ref{def:lvars-permutation-configuration},
  $\pi(\config{S}{e})$ is equal to $\config{\pi(S)}{\pi(e)}$ and
  $\pi(\config{S'}{e'})$ is equal to $\config{\pi(S')}{\pi(e')}$.

  Hence $\config{\pi(S)}{\pi(e)} \parstepsto
  \config{\pi(S')}{\pi(e')}$.

  Therefore, by {\sc E-Eval-Ctxt}, $\config{\pi(S)}{\E{\pi(e)}}
  \ctxstepsto \config{\pi(S')}{\E{\pi(e')}}$ for all evaluation
  contexts $E$.

  In particular, it is true that
  $\config{\pi(S)}{\evalctxt{(\pi(E))}{\pi(e)}} \ctxstepsto
  \config{\pi(S')}{\evalctxt{(\pi(E))}{\pi(e')}}$, as we were required
  to show.

  For the reverse direction of
  part~\ref{thm:permutable-context-transitions}, suppose $\pi(\conf)
  \ctxstepsto \pi(\conf')$.  We have to show that $\conf \ctxstepsto
  \conf'$.

  We know from the forward direction of the proof that for all
  configurations $\conf$ and $\conf'$ and permutations $\pi$, if
  $\conf \ctxstepsto \conf'$ then $\pi(\conf) \ctxstepsto
  \pi(\conf')$.  Hence since $\pi(\conf) \ctxstepsto \pi(\conf')$, and
  since $\piinv$ is also a permutation, we have that
  $\piinv(\pi(\conf)) \ctxstepsto \piinv(\pi(\conf'))$.  Since
  $\piinv(\pi(l)) = l$ for every $l \in \Loc$, and that property lifts
  to configurations as well, we have that $\conf \ctxstepsto \conf'$.

  \lk{Is the above enough of a proof?}
\end{proof}


\section{Proof of Lemma~\ref{lem:lvars-internal-determinism}}\label{section:lvars-internal-determinism-proof}
\begin{proof}
  Suppose $\conf \parstepsto \conf'$ and $\conf \parstepsto \conf''$.

  We have to show that there is a permutation $\pi$ such that $\conf'
  = \pi(\conf'')$.

  The proof is by cases on the rule by which $\conf$ steps to
  $\conf'$.

  \begin{itemize}

  \item Case {\sc E-Beta}:

    Given: $\config{S}{\app{(\lam{x}{e})}{v}} \parstepsto
    \config{S}{\subst{e}{x}{v}}$, and
    $\config{S}{\app{(\lam{x}{e})}{v}} \parstepsto \conf''$.

    To show: There exists a $\pi$ such that
    $\config{S}{\subst{e}{x}{v}} = \pi(\conf'')$.

    By inspection of the operational semantics, the only reduction
    rule by which $\config{S}{\app{(\lam{x}{e})}{v}}$ can step is {\sc
      E-Beta}.

    Hence $\conf'' = \config{S}{\subst{e}{x}{v}}$, and the case is
    satisfied by choosing $\pi$ to be the identity function.

  \item Case {\sc E-New}: 

    Given: $\config{S}{\NEW} \parstepsto
    \config{\extSRaw{S}{l}{\bot}}{l}$, and $\config{S}{\NEW}
    \parstepsto \conf''$.

    To show: There exists a $\pi$ such that
    $\config{\extSRaw{S}{l}{\bot}}{l} = \pi(\conf'')$.

    By inspection of the operational semantics, the only reduction
    rule by which $\config{S}{\NEW}$ can step is {\sc E-New}.

    Hence $\conf'' = \config{\extSRaw{S}{l'}{\bot}}{l'}$.

    Since, by the side condition of {\sc E-New}, neither $l$ nor $l'$
    occur in $\dom{S}$, the case is satisfied by choosing $\pi$ to be
    the permutation that maps $l'$ to $l$ and is the identity on every
    other element of $\Loc$.

  \item Case {\sc E-Put}:

    Given: $\config{S}{\putexp{l}{d_2}} \parstepsto
    \config{\extSRaw{S}{l}{\userlub{d_1}{d_2}}}{\unit}$, and
    $\config{S}{\putexp{l}{d_2}} \parstepsto \conf''$.

    To show: There exists a $\pi$ such that
    $\config{\extSRaw{S}{l}{\userlub{d_1}{d_2}}}{\unit} =
    \pi(\conf'')$.

    By inspection of the operational semantics, and since
    $\userlub{d_1}{d_2} \neq \top$ (from the premise of {\sc E-Put}),
    the only reduction rule by which $\config{S}{\putexp{l}{d_2}}$ can
    step is {\sc E-Put}.

    Hence $\conf'' =
    \config{\extSRaw{S}{l}{\userlub{d_1}{d_2}}}{\unit}$, and the case
    is satisfied by choosing $\pi$ to be the identity function.

  \item Case {\sc E-Put-Err}:

    Given: $\config{S}{\putexp{l}{d_2}} \parstepsto \error$, and
    $\config{S}{\putexp{l}{d_2}} \parstepsto \conf''$.

    To show: There exists a $\pi$ such that $\error = \pi(\conf'')$.

    By inspection of the operational semantics, and since
    $\userlub{d_1}{d_2} = \top$ (from the premise of {\sc E-Put-Err}),
    the only reduction rule by which $\config{S}{\putexp{l}{d_2}}$ can
    step is {\sc E-Put-Err}.

    Hence $\conf'' = \error$, and the case is satisfied by choosing
    $\pi$ to be the identity function.

  \item Case {\sc E-Get}:

    Given: $\config{S}{\getexp{l}{T}} \parstepsto \config{S}{d_2}$,
    and $\config{S}{\getexp{l}{T}} \parstepsto \conf''$.

    To show: There exists a $\pi$ such that $\config{S}{d_2} =
    \pi(\conf'')$.

    By inspection of the operational semantics, the only reduction
    rule by which $\config{S}{\getexp{l}{T}}$ can step is {\sc
      E-Get}.

    Hence $\conf'' = \config{S}{d_2}$, and the case is satisfied by
    choosing $\pi$ to be the identity function.

  \end{itemize}
\end{proof}



\section{Proof of Lemma~\ref{lem:lvars-monotonicity}}\label{section:lvars-monotonicity-proof}
\begin{proof}
  Suppose $\config{S}{e} \parstepsto \config{S'}{e'}$.  We are
  required to show that $\leqstore{S}{S'}$.  The proof is by cases on
  the rule by which $\config{S}{e}$ steps to $\config{S'}{e'}$.

  \begin{itemize}
    \item Case {\sc E-Beta}:

      Immediate by the definition of $\leqstore{}{}$, since $S$ does
      not change.

    \item Case {\sc E-New}:

      Given: $\config{S}{\NEW} \parstepsto
      \config{\extSRaw{S}{l}{\bot}}{l}$.

      To show: $\leqstore{S}{\extSRaw{S}{l}{\bot}}$.

      By Definition~\ref{def:lvars-leqstore}, we have to show that
      $\dom{S} \subseteq \dom{\extSRaw{S}{l}{\bot}}$ and
      that for all $l' \in \dom{S}$, $S(l') \userleq
      (\extSRaw{S}{l}{\bot})(l')$.

      By definition, a store update operation on $S$ can only either
      update an existing binding in $S$ or extend $S$ with a new
      binding.  Hence $\dom{S} \subseteq \dom{\extSRaw{S}{l}{\bot}}$.

      From the side condition of {\sc E-New}, $l \notin \dom{S}$.
      Hence $\extSRaw{S}{l}{\bot}$ adds a new binding for $l$ in $S$.

      Hence $\extSRaw{S}{l}{\bot}$ does not update any existing
      bindings in $S$.

      Hence, for all $l' \in \dom{S}, S(l') \userleq
      (\extSRaw{S}{l}{\bot})(l')$.

      Therefore $\leqstore{S}{\extSRaw{S}{l}{\bot}}$, as
      required.

    \item Case {\sc E-Put}:

      Given: $\config{S}{\putexp{l}{d_2}} \parstepsto
      \config{\extSRaw{S}{l}{\userlub{d_1}{d_2}}}{\unit}$.

      To show: $\leqstore{S}{\extSRaw{S}{l}{\userlub{d_1}{d_2}}}$.

      By Definition~\ref{def:lvars-leqstore}, we have to show that
      $\dom{S} \subseteq \dom{\extSRaw{S}{l}{\userlub{d_1}{d_2}}}$ and
      that for all $l' \in \dom{S}$, $S(l') \userleq
      (\extSRaw{S}{l}{\userlub{d_1}{d_2}})(l')$.

      By definition, a store update operation on $S$ can only either
      update an existing binding in $S$ or extend $S$ with a new
      binding.  Hence $\dom{S} \subseteq
      \dom{\extSRaw{S}{l}{\userlub{d_1}{d_2}}}$.

      From the premises of {\sc E-Put}, $S(l) = d_1$.  Therefore $l
      \in \dom{S}$.

      Hence $\extSRaw{S}{l}{\userlub{d_1}{d_2}}$ updates the existing
      binding for $l$ in $S$ from $d_1$ to $\userlub{d_1}{d_2}$.

      By the definition of $\userlub{}{}$, $d_1 \userleq
      (\userlub{d_1}{d_2})$.  $\extSRaw{S}{l}{\userlub{d_1}{d_2}}$
      does not update any other bindings in $S$, hence, for all $l'
      \in \dom{S}, S(l') \userleq
      (\extSRaw{S}{l}{\userlub{d_1}{d_2}})(l')$.

      Hence $\leqstore{S}{\extSRaw{S}{l}{\userlub{d_1}{d_2}}}$, as
      required.

    \item Case {\sc E-Put-Err}:

      Given: $\config{S}{\putexp{l}{d_2}} \parstepsto \error$.

      By the definition of $\error$, $\error$ is equal to
      $\config{\topS}{e}$ for all $e$.

      To show: $\leqstore{S}{\topS}$.

      Immediate by the definition of $\leqstore{}{}$.

    \item Case {\sc E-Get}:

      Immediate by the definition of $\leqstore{}{}$, since $S$ does
      not change.

  \end{itemize}

\end{proof}


\section{Proof of Lemma~\ref{lem:lvars-independence}}\label{section:lvars-independence-proof}
\begin{proof}
  Consider arbitrary $S''$ such that $S''$ is non-conflicting with
  $\config{S}{e} \parstepsto \config{S'}{e'}$ and $\lubstore{S'}{S''}
  \neq \topS$.

  To show: $\config{\lubstore{S}{S''}}{e} \parstepsto
  \config{\lubstore{S'}{S''}}{e'}$.

  The proof is by induction on the derivation of $\config{S}{e}
  \parstepsto \config{S'}{e'}$, by cases on the last rule in the
  derivation.  In every case we may assume that $\config{S'}{e'} \neq
  \error$.  Since $\config{S'}{e'} \neq \error$, we do not need to
  consider the {\sc E-Put-Err} rule.
  \begin{itemize}

    \item Case {\sc E-Eval-Ctxt}:

      Given: $\config{S}{\E{e}} \parstepsto \config{S'}{\E{e'}}$.

      To show: $\config{\lubstore{S}{S''}}{\E{e}} \parstepsto
      \config{\lubstore{S'}{S''}}{\E{e'}}$.

      From the premise of {\sc E-Eval-Ctxt}, we have that
      $\config{S}{e} \parstepsto \config{S'}{e'}$.

      Therefore, by IH, we have that $\config{\lubstore{S}{S''}}{e}
      \parstepsto \config{\lubstore{S'}{S''}}{e'}$.

      Therefore, by {\sc E-Eval-Ctxt}, we have that
      $\config{\lubstore{S}{S''}}{\E{e}} \parstepsto
      \config{\lubstore{S'}{S''}}{\E{e'}}$, as we were required to
      show.

    \item Case {\sc E-Beta}:

      Given: $\config{S}{\app{(\lam{x}{e})}{v}} \parstepsto
      \config{S}{\subst{e}{x}{v}}$.

      To show: $\config{\lubstore{S}{S''}}{\app{(\lam{x}{e})}{v}}
      \parstepsto \config{\lubstore{S}{S''}}{\subst{e}{x}{v}}$.

      Immediate by {\sc E-Beta}.

    \item Case {\sc E-New}:

      Given: $\config{S}{\NEW} \parstepsto
      \config{\extSRaw{S}{l}{\bot}}{l}$.

      To show: $\config{\lubstore{S}{S''}}{\NEW} \parstepsto
      \config{\lubstore{(\extSRaw{S}{l}{\bot})}{S''}}{l}$.

      By {\sc E-New}, we have that $\config{\lubstore{S}{S''}}{\NEW}
      \parstepsto \config{\extSRaw{(\lubstore{S}{S''})}{l'}{\bot}}{l'}$,
      where $l' \notin \dom{\lubstore{S}{S''}}$.

      By assumption, $S''$ is non-conflicting with $\config{S}{\NEW}
      \parstepsto \config{\extSRaw{S}{l}{\bot}}{l}$.
 
      Therefore $l \notin \dom{S''}$.

      From the side condition of {\sc E-New}, $l \notin \dom{S}$.

      Therefore $l \notin \dom{\lubstore{S}{S''}}$.

      Therefore, in
      $\config{\extSRaw{(\lubstore{S}{S''})}{l'}{\bot}}{l'}$, we can
      $\alpha$-rename $l'$ to $l$, \\ resulting in
      $\config{\extSRaw{(\lubstore{S}{S''})}{l}{\bot}}{l}$.

      Therefore $\config{\lubstore{S}{S''}}{\NEW} \parstepsto
      \config{\extSRaw{(\lubstore{S}{S''})}{l}{\bot}}{l}$.

      Note that:
      \begin{align*}
        \extSRaw{(\lubstore{S}{S''})}{l}{\bot} &=
        \lubstore{\extSRaw{S}{l}{\bot}}{\extSRaw{S''}{l}{\bot}} \\ &=
        \lubstore{\lubstore{S}{\store{\storebindingRaw{l}{\bot}}}}{\lubstore{S''}{\store{\storebindingRaw{l}{\bot}}}}
        \\ &=
        \lubstore{\lubstore{S}{\store{\storebindingRaw{l}{\bot}}}}{S''}
        \\ &= \lubstore{\extSRaw{S}{l}{\bot}}{S''}.
      \end{align*}
      Therefore $\config{\lubstore{S}{S''}}{\NEW} \parstepsto
      \config{\lubstore{\extSRaw{S}{l}{\bot}}{S''}}{l}$, as we were
      required to show.

    \item Case {\sc E-Put}:

      Given: $\config{S}{\putexp{l}{d_2}} \parstepsto
      \config{\extSRaw{S}{l}{d_2}}{\unit}$.

      To show: $\config{\lubstore{S}{S''}}{\putexp{l}{d_2}}
      \parstepsto
      \config{\lubstore{\extSRaw{S}{l}{d_2}}{S''}}{\unit}$.

      We will first show that

      $\config{\lubstore{S}{S''}}{\putexp{l}{d_2}} \parstepsto
      \config{\extSRaw{(\lubstore{S}{S''})}{l}{d_2}}{\unit}$

      and then show why this is sufficient.

      We proceed by cases on $l$:

      \begin{itemize}
        \item $l \notin \dom{S''}$:

          By assumption, $\lubstore{\extSRaw{S}{l}{d_2}}{S''} \neq
          \topS$.

          By Lemma~\ref{lem:lvars-monotonicity},
          $\leqstore{S}{\extSRaw{S}{l}{d_2}}$.

          Hence $\lubstore{S}{S''} \neq \topS$.

          Therefore, by Definition~\ref{def:lvars-lubstore},
          $(\lubstore{S}{S''})(l) = S(l)$.

          From the premises of {\sc E-Put}, $S(l) = d_1$.

          Hence $(\lubstore{S}{S''})(l) = d_1$.

          From the premises of {\sc E-Put}, $d_2 = \userlub{d_1}{d_2}$
          and $d_2 \neq \top$.

          Therefore, by {\sc E-Put}, we have:
          $\config{\lubstore{S}{S''}}{\putexp{l}{d_2}} \parstepsto
          \config{\extSRaw{(\lubstore{S}{S''})}{l}{d_2}}{\unit}$.

        \item $l \in \dom{S''}$:

          By assumption, $\lubstore{\extSRaw{S}{l}{d_2}}{S''} \neq
          \topS$.

          By Lemma~\ref{lem:lvars-monotonicity},
          $\leqstore{S}{\extSRaw{S}{l}{d_2}}$.

          Hence $\lubstore{S}{S''} \neq \topS$.

          Therefore $(\lubstore{S}{S''})(l) = \userlub{S(l)}{S''(l)}$.

          From the premises of {\sc E-Put}, $S(l) = d_1$.
          
          Hence $(\lubstore{S}{S''})(l) = d'_1$, where $d_1 \userleq
          d'_1$.

          From the premises of {\sc E-Put}, $d_2 =
          \userlub{d_1}{d_2}$.

          Let $d'_2 = \userlub{d'_1}{d_2}$.

          Hence $d_2 \userleq d'_2$.

          By assumption, $\lubstore{\extSRaw{S}{l}{d_2}}{S''} \neq
          \topS$.

          Therefore, by Definition~\ref{def:lvars-lubstore},
          $\lubstore{d_2}{S''(l)} \neq \top$.

          Note that:
          \begin{align*}
            \top &\neq \lubstore{d_2}{S''(l)} \\ &=
            \userlub{\userlub{d_1}{d_2}}{S''(l)} \\ &=
            \userlub{\userlub{S(l)}{d_2}}{S''(l)} \\ &=
            \userlub{\userlub{S(l)}{S''(l)}}{d_2} \\ &=
            \userlub{(\lubstore{S}{S''})(l)}{d_2} \\ &=
            \userlub{d'_1}{d_2} \\ &= d'_2. \\
          \end{align*}
          Hence $d'_2 \neq \top$.

          Hence $(\lubstore{S}{S''})(l) = d'_1$ and $d'_2 =
          \userlub{d'_1}{d_2}$ and $d'_2 \neq \top$.

          Therefore, by {\sc E-Put} we have:
          $\config{\lubstore{S}{S''}}{\putexp{l}{d_2}} \parstepsto
          \config{\extSRaw{(\lubstore{S}{S''})}{l}{d'_2}}{\unit}$.

          \lk{If we really wanted to be pedantic here, we'd actually
            prove that the stores are equal.  I'm assuming that if I
            can show that $\extSRaw{(\lubstore{S}{S''})}{l}{d'_2}$ and
            $\extSRaw{(\lubstore{S}{S''})}{l}{d_2}$ bind $l$ to the
            same value, then it will be obvious that they're equal.}

          Note that:
          \begin{align*}
            (\extSRaw{(\lubstore{S}{S''})}{l}{d'_2})(l) &=
            \userlub{(\lubstore{S}{S''})(l)}{(\store{\storebindingRaw{l}{d'_2}})(l)}
            \\ &= \userlub{d'_1}{d'_2} \\ &=
            \userlub{d'_1}{\userlub{d'_1}{d_2}} \\ &=
            \userlub{d'_1}{d_2}
          \end{align*}
          and
          \begin{align*}
            (\extSRaw{(\lubstore{S}{S''})}{l}{d_2})(l) &=
            \userlub{(\lubstore{S}{S''})(l)}{(\store{\storebindingRaw{l}{d_2}})(l)}
            \\ &= \userlub{d'_1}{d_2} \\ &=
            \userlub{d'_1}{\userlub{d_1}{d_2}} \\ &=
            \userlub{d'_1}{d_2} & \textrm{(since $d_1 \userleq
              d'_1$).}
          \end{align*}
          Therefore $\extSRaw{(\lubstore{S}{S''})}{l}{d'_2} =
          \extSRaw{(\lubstore{S}{S''})}{l}{d_2}$.

          Therefore, $\config{\lubstore{S}{S''}}{\putexp{l}{d_2}}
          \parstepsto
          \config{\extSRaw{(\lubstore{S}{S''})}{l}{d_2}}{\unit}$.
      \end{itemize}

      Note that:
      \begin{align*}
        \extSRaw{(\lubstore{S}{S''})}{l}{d_2} &=
        \lubstore{\extSRaw{S}{l}{d_2}}{\extSRaw{S''}{l}{d_2}} \\ &=
        \lubstore{\lubstore{S}{\store{\storebindingRaw{l}{d_2}}}}{\lubstore{S''}{\store{\storebindingRaw{l}{d_2}}}}
        \\ &=
        \lubstore{\lubstore{S}{\store{\storebindingRaw{l}{d_2}}}}{S''}
        \\ &= \lubstore{\extSRaw{S}{l}{d_2}}{S''}.
      \end{align*}
      Therefore $\config{\lubstore{S}{S''}}{\putexp{l}{d_2}}
      \parstepsto
      \config{\lubstore{\extSRaw{S}{l}{d_2}}{S''}}{\unit}$, as we were
      required to show.

    \item Case {\sc E-Get}:

      Given: $\config{S}{\getexp{l}{T}} \parstepsto \config{S}{d_2}$.

      To show: $\config{\lubstore{S}{S''}}{\getexp{l}{T}} \parstepsto
      \config{\lubstore{S}{S''}}{d_2}$.

      From the premises of {\sc E-Get}, $S(l) = d_1$ and $\incomp{T}$
      and $d_2 \in T$ and $d_2 \userleq d_1$.

      By assumption, $\lubstore{S}{S''} \neq \topS$.

      Hence $(\lubstore{S}{S''}) = d'_1$, where $d_1 \userleq d'_1$.

      By the transitivity of $\userleq$, $d_2 \userleq d'_1$.

      Hence, $S(l) = d'_1$ and $\incomp{T}$ and $d_2 \in T$ and $d_2
      \userleq d'_1$.

      Therefore, by {\sc E-Get},

      $\config{\lubstore{S}{S''}}{\getexp{l}{T}} \parstepsto
      \config{\lubstore{S}{S''}}{d_2}$,

      as we were required to show.
  \end{itemize}
\end{proof}


\section{Proof of Lemma~\ref{lem:lvars-clash}}\label{section:lvars-clash-proof}
\begin{proof}
  Consider arbitrary $S''$ such that $S''$ is non-conflicting with
  $\config{S}{e} \parstepsto \config{S'}{e'}$ and $\lubstore{S'}{S''}
  = \topS$.

  To show: $\config{\lubstore{S}{S''}}{e} \parstepsto \error$.

  The proof is by induction on the derivation of $\config{S}{e}
  \parstepsto \config{S'}{e'}$, by cases on the last rule in the
  derivation.  In every case we may assume that $\config{S'}{e'} \neq
  \error$.  Since $\config{S'}{e'} \neq \error$, we do not need to
  consider the {\sc E-Put-Err} rule.

  \begin{itemize}

    \item Case {\sc E-Eval-Ctxt}:

      Given: $\config{S}{\E{e}} \parstepsto \config{S'}{\E{e'}}$.

      To show: $\config{\lubstore{S}{S''}}{\E{e}} \parstepsto^i
      \error$, where $i \leq 1$.

      From the premise of {\sc E-Eval-Ctxt}, we have that
      $\config{S}{e} \parstepsto \config{S'}{e'}$.

      Therefore, by IH, we have that $\config{\lubstore{S}{S''}}{e}
      \parstepsto^{i'} \error$, where $i' \leq 1$.

      We proceed by cases on $i'$:

      \begin{itemize}
        \item $i' = 0$:

          In this case, $\config{\lubstore{S}{S''}}{e} = \error$.

          Hence, by the definition of $\error$, $\lubstore{S}{S''} =
          \topS$.

          Hence $\config{\lubstore{S}{S''}}{\E{e}} = \error$.

          Hence $\config{\lubstore{S}{S''}}{\E{e}} \parstepsto^i
          \error$, with $i = 0$.

        \item $i' = 1$:

          In this case, $\config{\lubstore{S}{S''}}{e} \parstepsto
          \error$.

          By the definition of $\error$, $\error =
          \config{\topS}{e''}$ for any $e''$.

          Hence $\config{\lubstore{S}{S''}}{e} \parstepsto
          \config{\topS}{e''}$.

          Hence, by {\sc E-Eval-Ctxt},
          $\config{\lubstore{S}{S''}}{\E{e}} \parstepsto
          \config{\topS}{\E{e''}}$.

          By the definition of $\error$, $\config{\topS}{\E{e''}} =
          \error$.

          Hence $\config{\lubstore{S}{S''}}{\E{e}} \parstepsto
          \error$.

          Hence $\config{\lubstore{S}{S''}}{\E{e}} \parstepsto^i
          \error$, with $i = 1$.

      \end{itemize}

    \item Case {\sc E-Beta}:

      Given: $\config{S}{\app{(\lam{x}{e})}{v}} \parstepsto
      \config{S}{\subst{e}{x}{v}}$.

      To show: $\config{\lubstore{S}{S''}}{\app{(\lam{x}{e})}{v}}
      \parstepsto^i \error$, where $i \leq 1$.

      By assumption, $\lubstore{S}{S''} = \topS$.

      Hence, by the definition of $\error$,
      $\config{\lubstore{S}{S''}}{\app{(\lam{x}{e})}{v}} = \error$.

      Hence $\config{\lubstore{S}{S''}}{\app{(\lam{x}{e})}{v}}
      \parstepsto^i \error$, with $i = 0$.

    \item Case {\sc E-New}:

      Given: $\config{S}{\NEW} \parstepsto
      \config{\extSRaw{S}{l}{\bot}}{l}$.

      To show: $\config{\lubstore{S}{S''}}{\NEW} \parstepsto^i
      \error$, where $i \leq 1$.

      By {\sc E-New}, $\config{\lubstore{S}{S''}}{\NEW} \parstepsto
      \config{\extSRaw{(\lubstore{S}{S''})}{l'}{\bot}}{l'}$, where $l'
      \notin \dom{\lubstore{S}{S''}}$.

      By assumption, $S''$ is non-conflicting with $\config{S}{\NEW}
      \parstepsto \config{\extSRaw{S}{l}{\bot}}{l}$.
 
      Therefore $l \notin \dom{S''}$.

      From the side condition of {\sc E-New}, $l \notin \dom{S}$.

      Therefore $l \notin \dom{\lubstore{S}{S''}}$.

      Therefore, in
      $\config{\extSRaw{(\lubstore{S}{S''})}{l'}{\bot}}{l'}$, we can
      $\alpha$-rename $l'$ to $l$, \\ resulting in
      $\config{\extSRaw{(\lubstore{S}{S''})}{l}{\bot}}{l}$.

      Therefore $\config{\lubstore{S}{S''}}{\NEW} \parstepsto
      \config{\extSRaw{(\lubstore{S}{S''})}{l}{\bot}}{l}$.

      By assumption, $\lubstore{\extSRaw{S}{l}{\bot}}{S''}
      = \topS$.

      Note that:
      \begin{align*}
        \topS &= \lubstore{\extSRaw{S}{l}{\bot}}{S''} \\ &=
        \lubstore{\lubstore{S}{\store{\storebindingRaw{l}{\bot}}}}{S''}
        \\ &=
        \lubstore{\lubstore{S}{S''}}{\store{\storebindingRaw{l}{\bot}}}
        \\ &=
        \lubstore{(\lubstore{S}{S''})}{\store{\storebindingRaw{l}{\bot}}}
        \\ &= \extSRaw{(\lubstore{S}{S''})}{l}{\bot} .
      \end{align*}

      Hence $\config{\lubstore{S}{S''}}{\NEW} \parstepsto
      \config{\topS}{l}$.

      Hence, by the definition of $\error$,
      $\config{\lubstore{S}{S''}}{\NEW} \parstepsto \error$.

      Hence $\config{\lubstore{S}{S''}}{\NEW} \parstepsto^i \error$,
      with $i = 1$.

    \item Case {\sc E-Put}:

      Given: $\config{S}{\putexp{l}{d_2}} \parstepsto
      \config{\extSRaw{S}{l}{d_2}}{\unit}$.

      To show: $\config{\lubstore{S}{S''}}{\putexp{l}{d_2}}
      \parstepsto^i \error$, where $i \leq 1$.

      We proceed by cases on $\lubstore{S}{S''}$:

      \begin{itemize}

        \item $\lubstore{S}{S''} = \topS$:

          In this case, by the definition of $\error$,
          $\config{\lubstore{S}{S''}}{\putexp{l}{d_2}} = \error$.

          Hence $\config{\lubstore{S}{S''}}{\putexp{l}{d_2}}
          \parstepsto^i \error$, with $i = 0$.

        \item $\lubstore{S}{S''} \neq \topS$:

          From the premises of {\sc E-Put}, we have that $S(l) = d_1$.

          Hence $(\lubstore{S}{S''})(l) = d'_1$, where $d_1 \userleq
          d'_1$.

          We show that $\userlub{d'_1}{d_2} =
          \top$, as follows:

          By assumption, $\lubstore{\extSRaw{S}{l}{d_2}}{S''} = \topS$.

          Hence, by Definition~\ref{def:lvars-lubstore}, there exists
          some $l' \in \dom{\extSRaw{S}{l}{d_2}} \cap \dom{S''}$ such
          that $\userlub{(\extSRaw{S}{l}{d_2})(l')}{S''(l')} = \top$.

          Now case on $l'$:

          \begin{itemize}
            \item $l' \neq l$:

              In this case, $(\extSRaw{S}{l}{d_2})(l') = S(l')$.

              Since $\userlub{(\extSRaw{S}{l}{d_2})(l')}{S''(l')} = \top$,
              we then have that $\userlub{S(l')}{S''(l')} = \top$.

              However, this is a contradiction since
              $\lubstore{S}{S''} \neq \topS$.

              Hence this case cannot occur.

            \item $l' = l$:

              Then $\userlub{(\extSRaw{S}{l}{d_2})(l)}{S''(l)} = \top$.

              Note that:
              \begin{align*}
                \top &= \userlub{(\extSRaw{S}{l}{d_2})(l)}{S''(l)} \\ &=
                \userlub{d_2}{S''(l)} \\ &=
                \userlub{\userlub{d_1}{d_2}}{S''(l)}
                \\ &=
                \userlub{\userlub{S(l)}{d_2}}{S''(l)}
                \\ &=
                \userlub{\userlub{S(l)}{S''(l)}}{d_2}
                \\ &=
                \userlub{(\lubstore{S}{S''})(l)}{d_2}
                \\ &= \userlub{d'_1}{d_2}.
              \end{align*}
              Hence $\userlub{d'_1}{d_2} = \top$.

              Hence, by {\sc E-Put-Err},
              $\config{\lubstore{S}{S''}}{\putexp{l}{d_2}} \parstepsto
              \error$.

              Hence $\config{\lubstore{S}{S''}}{\putexp{l}{d_2}}
              \parstepsto^i \error$, with $i = 1$.

          \end{itemize}

      \end{itemize}

    \item Case {\sc E-Get}:

      Given: $\config{S}{\getexp{l}{T}} \parstepsto \config{S}{d_2}$.

      To show: $\config{\lubstore{S}{S''}}{\getexp{l}{T}}
      \parstepsto^i \error$, where $i \leq 1$.

      By assumption, $\lubstore{S}{S''} = \topS$.

      Hence, by the definition of $\error$,
      $\config{\lubstore{S}{S''}}{\getexp{l}{T}} = \error$.

      Hence $\config{\lubstore{S}{S''}}{\getexp{l}{T}} \parstepsto^i
      \error$, with $i = 0$.
  \end{itemize}
\end{proof}


\section{Proof of Lemma~\ref{lem:lvars-error-preservation}}\label{section:lvars-error-preservation-proof}
\begin{proof}

  Given: $\config{S}{e} \parstepsto \error$ and $\leqstore{S}{S'}$.

  To show: $\config{S'}{e} \parstepsto \error$.

  \TODO{Figure out what to do here.  I think we need to handle both
    E-Eval-Ctxt and E-Put-Err.}
\end{proof}


\section{Proof of Lemma~\ref{lem:lvars-strong-local-confluence}}\label{section:lvars-strong-local-confluence-proof}
\begin{proof}
  Suppose $\conf \ctxstepsto \conf_a$ and $\conf \ctxstepsto \conf_b$.
  We have to show that there exist $\conf_c, i, j, \pi$ such that
  $\conf_a \ctxstepsto^i \conf_c$ and $\pi(\conf_b) \ctxstepsto^j
  \pi(\conf_c)$ and $i \leq 1$ and $j \leq 1$.

  By inspection of the operational semantics, it must be the case that
  $\conf$ steps to $\conf_a$ by the {\sc E-Eval-Ctxt} rule.  Let
  $\conf = \config{S}{\evalctxt{E_a}{e_{a_1}}}$ and let $\conf_a =
  \config{S_a}{\evalctxt{E_a}{e_{a_2}}}$.

  Likewise, it must be the case that $\conf$ steps to $\conf_b$ by the
  {\sc E-Eval-Ctxt} rule.  Let $\conf =
  \config{S}{\evalctxt{E_b}{e_{b_1}}}$ and let $\conf_b =
  \config{S_b}{\evalctxt{E_b}{e_{b_2}}}$.

  Note that $\conf = \config{S}{\evalctxt{E_a}{e_{a_1}}} =
  \config{S}{\evalctxt{E_b}{e_{b_1}}}$, and so
  $\evalctxt{E_a}{e_{a_1}} = \evalctxt{E_b}{e_{b_1}}$, but $E_a$ and
  $E_b$ may differ and $e_{a_1}$ and $e_{b_1}$ may differ.

  Since $\config{S}{\evalctxt{E_a}{e_{a_1}}} \ctxstepsto
  \config{S_a}{\evalctxt{E_a}{e_{a_2}}}$ and
  $\config{S}{\evalctxt{E_b}{e_{b_1}}} \ctxstepsto
  \config{S_b}{\evalctxt{E_b}{e_{b_2}}}$ and $\evalctxt{E_a}{e_{a_1}}
  = \evalctxt{E_b}{e_{b_1}}$, we have from
  Lemma~\ref{lem:lvars-locality} (Locality) that there exist
  evaluation contexts $E'_a$ and $E'_b$ such that:

  \begin{itemize}
  \item $\evalctxt{E'_a}{e_{a_1}} = \evalctxt{E_b}{e_{b_2}}$, and
  \item $\evalctxt{E'_b}{e_{b_1}} = \evalctxt{E_a}{e_{a_2}}$, and
  \item $\evalctxt{E'_a}{e_{a_2}} =
  \evalctxt{E'_b}{e_{b_2}}$.
  \end{itemize}

  Our approach will be to show that there exist $S', i, j, \pi$ such
  that:
  \begin{itemize}
  \item $\config{S_a}{\evalctxt{E_a}{e_{a_2}}} \ctxstepsto^i
    \config{S'}{\evalctxt{E'_a}{e_{a_2}}}$, and
  \item $\pi(\config{S_b}{\evalctxt{E_b}{e_{b_2}}}) \ctxstepsto^j
    \pi(\config{S'}{\evalctxt{E'_a}{e_{a_2}}})$.
  \end{itemize}
  Since $\evalctxt{E'_a}{e_{a_1}} = \evalctxt{E_b}{e_{b_2}}$,
  $\evalctxt{E'_b}{e_{b_1}} = \evalctxt{E_a}{e_{a_2}}$, and
  $\evalctxt{E'_a}{e_{a_2}} = \evalctxt{E'_b}{e_{b_2}}$, it suffices
  to show that:
  \begin{itemize}
  \item $\config{S_a}{\evalctxt{E'_b}{e_{b_1}}} \ctxstepsto^i
    \config{S'}{\evalctxt{E'_b}{e_{b_2}}}$, and
  \item $\pi(\config{S_b}{\evalctxt{E'_a}{e_{a_1}}}) \ctxstepsto^j
    \pi(\config{S'}{\evalctxt{E'_a}{e_{a_2}}})$.
  \end{itemize}

  From the premise of {\sc E-Eval-Ctxt}, we have that
  $\config{S}{e_{a_1}} \parstepsto \config{S_a}{e_{a_2}}$ and
  $\config{S}{e_{b_1}} \parstepsto \config{S_b}{e_{b_2}}$.  We proceed
  by case analysis on the rule by which $\config{S}{e_{a_1}}$ steps to
  $\config{S_a}{e_{a_2}}$.

  \begin{enumerate}
  \item Case {\sc E-Beta}:

    We have:
    \begin{itemize}
      \item $e_{a_1} = \app{\lam{x}{e'_a}}{v_a}$,
      \item $e_{a_2} = \subst{e'_a}{x}{v_a}$, and
      \item $S_a = S$.
    \end{itemize}

    Now, we proceed by case analysis on the rule by which
    $\config{S}{e_{b_1}}$ steps to $\config{S_b}{e_{b_2}}$:
    \begin{enumerate}
    \item Case {\sc E-Beta}:

      We have:
      \begin{itemize}
      \item $e_{b_1} = \app{\lam{x}{e'_b}}{v_b}$,
      \item $e_{b_2} = \subst{e'_b}{x}{v_b}$, and
      \item $S_b = S$.
      \end{itemize}

      Choose $S' = S$, $i = 1$, $j = 1$, and $\pi = \id$.

      We have to show that:

      \begin{itemize}
      \item $\config{S}{\evalctxt{E'_b}{e_{b_1}}} \ctxstepsto
        \config{S}{\evalctxt{E'_b}{e_{b_2}}}$, and
      \item $\config{S}{\evalctxt{E'_a}{e_{a_1}}} \ctxstepsto
        \config{S}{\evalctxt{E'_a}{e_{a_2}}}$, 
      \end{itemize}

      both of which follow immediately from $\config{S}{e_{a_1}}
      \parstepsto \config{S_a}{e_{a_2}}$ and $\config{S}{e_{b_1}}
      \parstepsto \config{S_b}{e_{b_2}}$ and {\sc E-Eval-Ctxt}.

    \item Case {\sc E-New}:

      We have:
      \begin{itemize}
      \item $e_{b_1} = \NEW$,
      \item $e_{b_2} = l$, and
      \item $S_b = \extSRaw{S}{l}{\bot}$.
      \end{itemize}

      Choose $S' = S_b$, $i = 1$, $j = 1$, and $\pi = \id$.

      We have to show that:

      \begin{itemize}
      \item $\config{S}{\evalctxt{E'_b}{e_{b_1}}} \ctxstepsto
        \config{S_b}{\evalctxt{E'_b}{e_{b_2}}}$, and
      \item
        $\config{S_b}{\evalctxt{E'_a}{e_{a_1}}} \ctxstepsto
        \config{S_b}{\evalctxt{E'_a}{e_{a_2}}}$.
      \end{itemize}

      The first of these follows immediately from $\config{S}{e_{b_1}}
      \parstepsto \config{S_b}{e_{b_2}}$ and {\sc E-Eval-Ctxt}.  For
      the second, consider that $S_b = \extSRaw{S}{l}{\bot} =
      \lubstore{S}{\store{\storebindingRaw{l}{\bot}}}$.  Furthermore, we
      know from the side condition of {\sc E-New} that $l \notin
      \dom{S}$, so $\store{\storebindingRaw{l}{\bot}}$ is non-conflicting
      with the transition $\config{S}{e_{a_1}} \parstepsto
      \config{S_a}{e_{a_2}}$, and we know that
      $\lubstore{S_a}{\store{\storebindingRaw{l}{\bot}}} \neq \topS$
      since $S_a$ is just $S$.  Therefore, by
      Lemma~\ref{lem:lvars-independence} (Independence), we have that
      $\config{\lubstore{S}{\store{\storebindingRaw{l}{\bot}}}}{e_{a_1}}
      \parstepsto
      \config{\lubstore{S_a}{\store{\storebindingRaw{l}{\bot}}}}{e_{a_2}}$.
      Hence $\config{S_b}{e_{a_1}} \parstepsto \config{S_b}{e_{a_2}}$.
      By {\sc E-Eval-Ctxt}, it follows that
      $\config{S_b}{\evalctxt{E'_a}{e_{a_1}}} \ctxstepsto
      \config{S_b}{\evalctxt{E'_a}{e_{a_2}}}$, as we were required to
      show.

    \item Case {\sc E-Put}: \TODO{}
    \item Case {\sc E-Put-Err}: \TODO{}
    \item Case {\sc E-Get}:\TODO{}
    \end{enumerate}
  \item Case {\sc E-New}:

    Now, we proceed by case analysis on the rule by which
    $\config{S}{e_{b_1}}$ steps to $\config{S_b}{e_{b_2}}$:
    \begin{enumerate}
    \item Case {\sc E-Beta}: \TODO{}
    \item Case {\sc E-New}: \TODO{}
    \item Case {\sc E-Put}: \TODO{}
    \item Case {\sc E-Put-Err}: \TODO{}
    \item Case {\sc E-Get}: \TODO{}
    \end{enumerate}
  \item Case {\sc E-Put}:

    Now, we proceed by case analysis on the rule by which
    $\config{S}{e_{b_1}}$ steps to $\config{S_b}{e_{b_2}}$:
    \begin{enumerate}
    \item Case {\sc E-Beta}: \TODO{}
    \item Case {\sc E-New}: \TODO{}
    \item Case {\sc E-Put}: \TODO{}
    \item Case {\sc E-Put-Err}: \TODO{}
    \item Case {\sc E-Get}: \TODO{}
    \end{enumerate}
  \item Case {\sc E-Put-Err}:

    Now, we proceed by case analysis on the rule by which
    $\config{S}{e_{b_1}}$ steps to $\config{S_b}{e_{b_2}}$:
    \begin{enumerate}
    \item Case {\sc E-Beta}: \TODO{}
    \item Case {\sc E-New}: \TODO{}
    \item Case {\sc E-Put}: \TODO{}
    \item Case {\sc E-Put-Err}: \TODO{}
    \item Case {\sc E-Get}: \TODO{}
    \end{enumerate}
  \item Case {\sc E-Get}:

    Now, we proceed by case analysis on the rule by which
    $\config{S}{e_{b_1}}$ steps to $\config{S_b}{e_{b_2}}$:
    \begin{enumerate}
    \item Case {\sc E-Beta}: \TODO{}
    \item Case {\sc E-New}: \TODO{}
    \item Case {\sc E-Put}: \TODO{}
    \item Case {\sc E-Put-Err}: \TODO{}
    \item Case {\sc E-Get}: \TODO{}
    \end{enumerate}
  \end{enumerate}

  \lk{I think we also still have to separately deal with cases where
    $\conf_a = \error$ or $\conf_b = \error$.}
\end{proof}


\section{Proof of Lemma~\ref{lem:lvars-strong-one-sided-confluence}}\label{section:lvars-strong-one-sided-confluence-proof}
\begin{proof}
  Suppose $\conf \ctxstepsto \conf'$ and $\conf \ctxstepsto^m
  \conf''$, where $1 \leq m$.  We have to show that there exist
  $\conf_c, i, j, \pi$ such that $\conf' \ctxstepsto^i \conf_c$ and
  $\pi(\conf'') \ctxstepsto^j \conf_c$ and $i \leq m$ and $j \leq 1$.

  We proceed by induction on $m$.  In the base case of $m = 1$, the
  result is immediate from
  Lemma~\ref{lem:lvars-strong-local-confluence}.

  For the induction step, suppose $\conf \ctxstepsto^m \conf''
  \ctxstepsto \conf'''$ and suppose the lemma holds for $m$.

  We show that it holds for $m + 1$, as follows.

  We are required to show that there exist $\conf_c, i, j, \pi$ such
  that $\conf' \ctxstepsto^{i} \conf_c$ and $\pi(\conf''')
  \ctxstepsto^{j} \conf_c$ and $i \leq m + 1$ and $j \leq 1$.

  From the induction hypothesis, there exist $\conf_c', i', j', \pi'$
  such that $\conf' \ctxstepsto^{i'} \conf_c'$ and $\pi'(\conf'')
  \ctxstepsto^{j'} \conf_c'$ and $i' \leq m$ and $j' \leq 1$.

  We proceed by cases on $j'$:
  \begin{itemize}

  \item If $j' = 0$, then $\pi'(\conf'') = \conf_c'$.

    Since $\conf'' \ctxstepsto \conf'''$, we have that $\pi'(\conf'')
    \ctxstepsto \pi'(\conf''')$ by
    Lemma~\ref{lem:lvars-permutability} (Permutability).

    We can then choose $\conf_c = \pi'(\conf''')$ and $i = i' + 1$ and
    $j = 0$ and $\pi = \pi'$.  The key is that $\conf'
    \ctxstepsto^{i'} \conf'_c = \pi'(\conf'') \ctxstepsto
    \pi'(\conf''')$ for a total of $i' + 1$ steps.
    
  \item If $j' = 1$:

    First, since $\pi'(\conf'') \ctxstepsto^{j'} \conf'_c$, then by
    Lemma~\ref{lem:lvars-permutability} (Permutability) we have that
    $\conf'' \ctxstepsto^{j'} \piprimeinv(\conf'_c)$.

    Then, by $\conf'' \ctxstepsto^{j'} \piprimeinv(\conf'_c)$ and
    $\conf'' \ctxstepsto \conf'''$ and
    Lemma~\ref{lem:lvars-strong-local-confluence} (Strong Local
    Confluence), we have that there exist $\conf_c''$ and $i''$ and
    $j''$ and $\pi''$ such that $\piprimeinv(\conf'_c)
    \ctxstepsto^{i''} \conf_c''$ and $\pi''(\conf''')
    \ctxstepsto^{j''} \conf_c''$ and $i'' \leq 1$ and $j'' \leq 1$.

    Since $\piprimeinv(\conf'_c) \ctxstepsto^{i''} \conf_c''$, by
    Lemma~\ref{lem:lvars-permutability} (Permutability) we have that
    $\conf'_c \ctxstepsto^{i''} \pi'(\conf_c'')$.

    So we also have $\conf' \ctxstepsto^{i'} \conf_c'
    \ctxstepsto^{i''} \pi'(\conf_c'')$.

    Since $\pi''(\conf''') \ctxstepsto^{j''} \conf_c''$, by
    Lemma~\ref{lem:lvars-permutability} (Permutability) we have that
    $\pi'(\pi''(\conf''')) \ctxstepsto^{j''} \pi'(\conf_c'')$.

    In summary, we pick $\conf_c = \pi'(\conf_c'')$ and $i = i' + i''$
    and $j = j''$ and $\pi = \pi'' \circ \pi'$, which is sufficient
    because $i = i' + i'' \leq m + 1$ and $j = j'' \leq 1$.
  \end{itemize}

 \end{proof}


\section{Proof of Lemma~\ref{lem:lvars-strong-confluence}}\label{section:lvars-strong-confluence-proof}
\begin{proof}
  We proceed by induction on $n$.  In the base case of $n = 1$, the
  result is immediate from
  Lemma~\ref{lem:lvars-strong-one-sided-confluence}.

  For the induction step, suppose $\conf \parstepsto^n \conf'
  \parstepsto \conf'''$ and suppose the lemma holds for $n$.

  We show that it holds for $n + 1$, as follows.

  We are required to show that there exist $\conf_c, i, j$ such that
  $\conf''' \parstepsto^i \conf_c$ and $\conf'' \parstepsto^j \conf_c$
  and $i \leq m$ and $j \leq n + 1$.

  From the induction hypothesis, we have that there exist $\conf'_c,
  i', j'$ such that $\conf' \parstepsto^{i'} \conf'_c$ and $\conf''
  \parstepsto^{j'} \conf'_c$ and $i' \leq m$ and $j' \leq n$.

  We proceed by cases on $i'$:
  \begin{itemize}

  \item If $i' = 0$, then $\conf' = \conf_c'$.  We can then choose
    $\conf_c = \conf'''$ and $i = 0$ and $j = j' + 1$.

  \item If $i' \geq 1$:

    From $\conf' \parstepsto \conf'''$ and $\conf' \parstepsto^{i'}
    \conf_c'$ and Lemma~\ref{lem:lvars-strong-one-sided-confluence},
    we have that there exist $\conf_c''$ and $i''$ and $j''$ such that
    $\conf''' \parstepsto^{i''} \conf_c''$ and $\conf_c'
    \parstepsto^{j''} \conf_c''$ and $i'' \leq i'$ and $j'' \leq 1$.
    So we also have $\conf'' \parstepsto^{j'} \conf_c'
    \parstepsto^{j''} \conf_c''$.  In summary, we pick $\conf_c =
    \conf_c''$ and $i = i''$ and $j = j' + j''$, which is sufficient
    because $i = i'' \leq i' \leq m$ and $j = j' + j'' \leq n + 1$.
  \end{itemize}

\end{proof}


\section{Proof of Lemma~\ref{lem:lattice-structure}}\label{section:lattice-structure-proof}
\begin{proof}
  Suppose that $(D, \userleq, \bot, \top)$ is a lattice and $(D_p,
  \leqp, \botp, \topp) = \Freeze{D, \userleq, \bot, \top}$.

  In order to show that $(D_p, \leqp, \botp, \topp)$ is a lattice, we
  have to show that:
  \begin{enumerate}
  \item $\leqp$ is a partial order over $D_p$.

  \item Every nonempty finite subset of $D_p$ has a lub.

  \item $\botp$ is the least element of $D_p$.

  \item $\topp$ is the greatest element of $D_p$.
  \end{enumerate}

  We prove each of these properties in turn:

  \begin{enumerate}
  \item $\leqp$ is a partial order over $D_p$.

    To show this, we need to show that $\leqp$ is reflexive, transitive,
    and antisymmetric. 
    \begin{enumerate}
    \item $\leqp$ is reflexive.

      Suppose $v \in D_p$.

      Then, by Lemma~\ref{lem:partition-of-Dp}, either $v =
      \state{d}{\frozenfalse}$ with $d \in D$, or $v =
      \state{x}{\frozentrue}$ with $x \in X$, where $X = D -
      \setof{\top}$.
      \begin{itemize}
      \item Suppose $v = \state{d}{\frozenfalse}$:

        By the reflexivity of $\userleq$, we know $d \userleq d$.

        By the definition of $\leqp$, we know $\state{d}{\frozenfalse}
        \leqp \state{d}{\frozenfalse}$.

      \item Suppose $v = \state{x}{\frozentrue}$: 
        
        By the reflexivity of equality, $x = x$.

        By the definition of $\leqp$, we know $\state{x}{\frozentrue}
        \leqp \state{x}{\frozentrue}$.
      \end{itemize}

    \item $\leqp$ is transitive. 

      Suppose $v_1 \leqp v_2$ and $v_2 \leqp v_3$.

      We want to show that $v_1 \leqp v_3$.

      We proceed by case analysis on $v_1, v_2$, and $v_3$.
      \begin{itemize}
      \item Case $v_1 = \state{d_1}{\frozenfalse}$ and $v_2 =
        \state{d_2}{\frozenfalse}$ and $v_3 =
        \state{d_3}{\frozenfalse}$:
        
        By inversion on $\leqp$, it follows that $d_1 \userleq d_2$.

        By inversion on $\leqp$, it follows that $d_2 \userleq d_3$.

        By the transitivity of $\userleq$, we know $d_1 \userleq d_3$.

        By the definition of $\leqp$, it follows that
        $\state{d_1}{\frozenfalse} \leqp \state{d_3}{\frozenfalse}$.

        Hence $v_1 \leqp v_3$.

      \item Case $v_1 = \state{d_1}{\frozenfalse}$ and $v_2 =
        \state{d_2}{\frozenfalse}$ and $v_3 =
        \state{x_3}{\frozentrue}$:

        By inversion on $\leqp$, it follows that $d_1 \userleq d_2$.

        By inversion on $\leqp$, it follows that $d_2 \userleq x_3$.

        By the transitivity of $\userleq$, we know $d_1 \userleq x_3$.

        By the definition of $\leqp$, it follows that
        $\state{d_1}{\frozenfalse} \leqp \state{x_3}{\frozentrue}$.

        Hence $v_1 \leqp v_3$.

      \item Case $v_1 = \state{d_1}{\frozenfalse}$ and $v_2 =
        \state{x_2}{\frozentrue}$ and $v_3 =
        \state{d_3}{\frozenfalse}$:

        By inversion on $\leqp$, it follows that $d_1 \userleq x_2$.

        By inversion on $\leqp$, it follows that $d_3 = \top$.

        Since $\top$ is the maximal element of $D$, we know $d_1
        \userleq \top \equiv d_3$.

        By the definition of $\leqp$, it follows that
        $\state{d_1}{\frozenfalse} \leqp \state{d_3}{\frozenfalse}$.

        Hence $v_1 \leqp v_3$.

      \item Case $v_1 = \state{d_1}{\frozenfalse}$ and $v_2 =
        \state{x_2}{\frozentrue}$ and $v_3 =
        \state{x_3}{\frozentrue}$:

        By inversion on $\leqp$, it follows that $d_1 \userleq x_2$.

        By inversion on $\leqp$, it follows that $x_2 = x_3$.

        Hence $d_1 \userleq x_3$.

        By the definition of $\leqp$, it follows that
        $\state{d_1}{\frozenfalse} \leqp \state{x_3}{\frozentrue}$.

        Hence $v_1 \leqp v_3$.

      \item Case $v_1 = \state{x_1}{\frozentrue}$ and $v_2 =
        \state{d_2}{\frozenfalse}$ and $v_3 =
        \state{d_3}{\frozenfalse}$:

        By inversion on $\leqp$, it follows that $d_2 = \top$.

        By inversion on $\leqp$, it follows that $d_2 \userleq d_3$.

        Since $\top$ is maximal, it follows that $d_3 = \top$.

        By the definition of $\leqp$, it follows that
        $\state{x_1}{\frozentrue} \leqp \state{d_3}{\frozenfalse}$.

        Hence $v_1 \leqp v_3$. 

      \item Case $v_1 = \state{x_1}{\frozentrue}$ and $v_2 =
        \state{d_2}{\frozenfalse}$ and $v_3 =
        \state{x_3}{\frozentrue}$:

        By inversion on $\leqp$, it follows that $d_2 = \top$.

        By inversion on $\leqp$, it follows that $d_2 \userleq x_3$.

        Since $\top$ is maximal, it follows that $x_3 = \top$.

        But since $x_3 \in X \subseteq D/\setof{\top}$, we know $x_3
        \not= \top$.

        This is a contradiction. \\

        Hence $v_1 \leqp v_3$. 

      \item Case $v_1 = \state{x_1}{\frozentrue}$ and $v_2 =
        \state{x_2}{\frozentrue}$ and $v_3 =
        \state{d_3}{\frozenfalse}$:

        By inversion on $\leqp$, it follows that $x_1 = x_2$.

        By inversion on $\leqp$, it follows that $d_3 = \top$.

        By the definition of $\leqp$, it follows that
        $\state{x_1}{\frozentrue} \leqp \state{d_3}{\frozenfalse}$.

        Hence $v_1 \leqp v_3$. 

      \item Case $v_1 = \state{x_1}{\frozentrue}$ and $v_2 =
        \state{x_2}{\frozentrue}$ and $v_3 =
        \state{x_3}{\frozentrue}$:

        By inversion on $\leqp$, it follows that $x_1 = x_2$.

        By inversion on $\leqp$, it follows that $x_2 = x_3$.

        By transitivity of $=$, $x_1 = x_3$.

        By the definition of $\leqp$, it follows that
        $\state{x_1}{\frozentrue} \leqp \state{x_3}{\frozentrue}$.

        Hence $v_1 \leqp v_3$. 
        
      \end{itemize}

    \item $\leqp$ is antisymmetric. 

      Suppose $v_1 \leqp v_2$ and $v_2 \leqp v_1$. Now, we proceed by
      cases on $v_1$ and $v_2$.
      \begin{itemize}
      \item Case $v_1 = \state{d_1}{\frozenfalse}$ and $v_2 =
        \state{d_2}{\frozenfalse}$:
        
        By inversion on $v_1 \leqp v_2$, we know that $d_1 \userleq
        d_2$.

        By inversion on $v_2 \leqp v_1$, we know that $d_2 \userleq
        d_1$.

        By the antisymmetry of $\leq$, we know $d_1 = d_2$.

        Hence $v_1 = v_2$. 

      \item Case $v_1 = \state{d_1}{\frozenfalse}$ and $v_2 =
        \state{x_2}{\frozentrue}$:

        By inversion on $v_1 \leqp v_2$, we know that $d_1 \userleq x_2$.

        By inversion on $v_2 \leqp v_1$, we know that $d_1 = \top$.

        Since $\top$ is maximal in $D$, we know $x_2 = \top$.

        But since $x_2 \in X \subseteq D/\setof{\top}$, we know $x_2 \not= \top$.

        This is a contradiction.

        Hence $v_1 = v_2$. 
        
      \item Case $v_1 = \state{x_1}{\frozentrue}$ and $v_2 =
        \state{d_2}{\frozenfalse}$:

        Similar to the previous case. 

      \item Case $v_1 = \state{x_1}{\frozentrue}$ and $v_2 =
        \state{x_2}{\frozentrue}$:

        By inversion on $v_1 \leqp v_2$, we know that $x_1 = x_2$.

        Hence $v_1 = v_2$. 
      \end{itemize}
    \end{enumerate}

  \item Every nonempty finite subset of $D_p$ has a lub.

    To show this, it is sufficient to show that every two elements of
    $D_p$ have a lub, since a binary lub operation can be repeatedly
    applied to compute the lub of any finite set.

    We will show that every two elements of $D_p$ have a lub by
    showing that the $\lubp{}{}$ operation defined by
    Definition~\ref{def:lubp} computes their lub.

    It suffices to show the following two properties:
    \begin{enumerate}
    \item For all $v_1, v_2, v \in D_p$, if $v_1 \leqp v$ and $v_2
      \leqp v$, then $(\lubp{v_1}{v_2}) \leqp v$.
    \item For all $v_1, v_2 \in D_p$, $v_1 \leqp (\lubp{v_1}{v_2})$
      and $v_2 \leqp (\lubp{v_1}{v_2})$.
    \end{enumerate}
    \begin{enumerate}
    \item For all $v_1, v_2, v \in D_p$, if $v_1 \leqp v$ and $v_2
      \leqp v$, then $\lubp{v_1}{v_2} \leqp v$.
      
      Assume $v_1, v_2, v \in D_p$, and $v_1 \leqp v$ and $v_2 \leqp
      v$.

      Now we do a case analysis on $v_1$ and $v_2$.
      \begin{itemize}
      \item Case $v_1 = \state{d_1}{\frozenfalse}$ and $v_2 =
        \state{d_2}{\frozenfalse}$.
        
        Now case on $v$: 
        \begin{itemize}
        \item Case $v = \state{d}{\frozenfalse}$: 

          By the definition of $\lubp{}{}$,
          $\lubp{\state{d_1}{\frozenfalse}}{\state{d_2}{\frozenfalse}}
          = \state{\userlub{d_1}{d_2}}{\frozenfalse}$.

          By inversion on $\state{d_1}{\frozenfalse} \leqp
          \state{d}{\frozenfalse}$, $d_1 \userleq l$.

          By inversion on $\state{d_2}{\frozenfalse} \leqp
          \state{d}{\frozenfalse}$, $d_2 \userleq l$.

          Hence $l$ is an upper bound for $d_1$ and $d_2$.

          Hence $\userlub{d_1}{d_2} \userleq l$.

          Hence $\state{\userlub{d_1}{d_2}}{\frozenfalse} \leqp
          \state{d}{\frozenfalse}$.

          Hence $\lubp{v_1}{v_2} \leqp v$.
          
        \item Case $v = \state{x}{\frozentrue}$: 
          
          By the definition of $\lubp{}{}$, $\state{d_1}{\frozenfalse}
          \lubp{}{} \state{d_2}{\frozenfalse} =
          \state{\userlub{d_1}{d_2}}{\frozenfalse}$.

          By inversion on $\state{d_1}{\frozenfalse} \leqp
          \state{x}{\frozentrue}$, $d_1 \userleq x$.

          By inversion on $\state{d_2}{\frozenfalse} \leqp
          \state{x}{\frozentrue}$, $d_2 \userleq x$.
     
          Hence $x$ is an upper bound for $d_1$ and $d_2$.

          Hence $\userlub{d_1}{d_2} \userleq x$.

          Hence $\state{\userlub{d_1}{d_2}}{\frozenfalse} \leqp
          \state{x}{\frozentrue}$.

          Hence $\lubp{v_1}{v_2} \leqp v$.
        \end{itemize}
        
      \item Case $v_1 = \state{x_1}{\frozentrue}$ and $v_2 =
        \state{x_2}{\frozentrue}$:
        
        Now case on $v$: 
        \begin{itemize}
        \item Case $v = \state{d}{\frozenfalse}$: 
          
          By inversion on $\state{x_1}{\frozentrue} \leqp
          \state{d}{\frozenfalse}$, we know $l = \top$.

          By inversion on $\state{x_2}{\frozentrue} \leqp
          \state{d}{\frozenfalse}$, we know $l = \top$.

          Now consider whether $x_1 = x_2$ or not.
        
          If it does, then by the definition of $\lubp{}{}$,
          $\state{x_1}{\frozentrue} \lubp{}{} \state{x_2}{\frozentrue}
          = \state{x_1}{\frozentrue}$.

          By definition of $\leqp$, we have $\state{x_1}{\frozentrue}
          \leqp \state{\top}{\frozenfalse}$.

          So $\lubp{v_1}{v_2} \leqp v$.

          If it does not, then $\lubp{v_1}{v_2} =
          \state{\top}{\frozenfalse}$.

          By the definition of $\leqp$, we have
          $\state{\top}{\frozenfalse} \leqp
          \state{\top}{\frozenfalse}$.

          So $\lubp{v_1}{v_2} \leqp v$.
          
        \item Case $v = \state{x}{\frozentrue}$: 
          
          By inversion on $\state{x_1}{\frozentrue} \leqp
          \state{x}{\frozentrue}$, we know $x = x_1$.

          By inversion on $\state{x_2}{\frozentrue} \leqp
          \state{x}{\frozentrue}$, we know $x = x_2$.

          Hence $x_1 = x_2$.

          By the definition of $\lubp{}{}$, $\state{x_1}{\frozentrue}
          \lubp{}{} \state{x_2}{\frozentrue} =
          \state{x_1}{\frozentrue}$.

          Hence $\lubp{v_1}{v_2} \leqp v$.
        \end{itemize}
        
      \item Case $v_1 = \state{x_1}{\frozentrue}$ and $v_2 =
        \state{d_2}{\frozenfalse}$:
        
        Now case on $v$:
        \begin{itemize}
        \item Case $v = \state{d}{\frozenfalse}$:
          
          Now consider whether $d_2 \userleq x_1$.

          If it is, then $\state{x_1}{\frozentrue} \lubp{}{}
          \state{d_2}{\frozenfalse} = \state{x_1}{\frozentrue} = v_1$.

          Hence $\lubp{v_1}{v_2} \leqp v$.

          Otherwise, $\state{x_1}{\frozentrue} \lubp{}{}
          \state{d_2}{\frozenfalse} = \state{\top}{\frozenfalse}$.

          By inversion on $\state{x_1}{\frozentrue} \leqp
          \state{d}{\frozenfalse}$, we know $l = \top$.

          By reflexivity, $\state{\top}{\frozenfalse} \leqp
          \state{\top}{\frozenfalse}$.

          Hence $\lubp{v_1}{v_2} \leqp v$. 
          
        \item Case $v = \state{x}{\frozentrue}$:  
          
          By inversion on $\state{x_1}{\frozentrue} \leqp
          \state{x}{\frozentrue}$, we know that $x_1 = x$.

          By inversion on $\state{d_2}{\frozenfalse} \leqp
          \state{x}{\frozentrue}$, we know that $d_2 \userleq x$.

          By transitivity, $d_2 \userleq x_1$.

          By the definition of $\lubp{}{}$, it follows that
          $\state{x_1}{\frozentrue} \lubp{}{}
          \state{d_2}{\frozenfalse} = \state{x_1}{\frozentrue}$.

          By definition of $\leqp$, $\state{x_1}{\frozentrue} \leqp
          \state{x_1}{\frozentrue}$.

          Hence $\lubp{v_1}{v_2} \leqp v$. 
        \end{itemize}
        
      \item Case $v_1 = \state{d_1}{\frozenfalse}$ and $v_2 =
        \state{x_2}{\frozentrue}$:
        
        Symmetric with the previous case. 
      \end{itemize}
    \item For all $v_1, v_2 \in D_p$, $v_1 \leqp \lubp{v_1}{v_2}$ and
      $v_2 \leqp \lubp{v_1}{v_2}$.
      
      Assume $v_1, v_2 \in D_p$, and proceed by case analysis. 
      \begin{itemize}
      \item Case $v_1 = \state{d_1}{\frozenfalse}$ and $v_2 =
        \state{d_2}{\frozenfalse}$:

        Since $\userlub{}{}$ is a join operator, we know $d_1 \userleq
        \userlub{d_1}{d_2}$.

        By the definition of $\leqp$, $\state{d_1}{\frozenfalse}
        \userleq \state{\userlub{d_1}{d_2}}{\frozenfalse}$.

        By the definition of $\lubp{}{}$, $\lubp{v_1}{v_2} =
        \state{\userlub{d_1}{d_2}}{\frozenfalse}$.

        Hence $v_1 \leqp \lubp{v_1}{v_2}$.

        Since $\userlub{}{}$ is a join operator, we know $d_1 \userleq
        \userlub{d_1}{d_2}$.

        By the definition of $\leqp$, $\state{d_2}{\frozenfalse}
        \userleq \state{\userlub{d_1}{d_2}}{\frozenfalse}$.

        By the definition of $\lubp{}{}$, $\lubp{v_1}{v_2} =
        \state{\userlub{d_1}{d_2}}{\frozenfalse}$.

        Hence $v_2 \leqp \lubp{v_1}{v_2}$. 

        Therefore $v_1 \leqp v_1 \userlub{}{} v_2$ and $v_2 \leqp v_1
        \userlub{}{} v_2$.
 
      \item Case $v_1 = \state{d_1}{\frozenfalse}$ and $v_2 = \state{x_2}{\frozentrue}$:

        Consider whether $d_1 \userleq x_2$. 
        \begin{itemize}
        \item Case  $d_1 \userleq x_2$:

          By the definition of $\lubp{}{}$, we know
          $\state{d_1}{\frozenfalse} \lubp{}{}
          \state{x_2}{\frozentrue} = \state{x_2}{\frozentrue}$.

          By the definition of $\lubp{}{}$, we know
          $\state{d_1}{\frozenfalse} \leqp \state{x_2}{\frozentrue}$.

          Hence $v_1 \leqp \lubp{v_1}{v_2}$.

          By reflexivity, $\state{x_2}{\frozentrue} \leqp
          \state{x_2}{\frozentrue}$.

          Hence $v_2 \leqp \lubp{v_1}{v_2}$.

          Therefore $v_1 \leqp v_1 \userlub{}{} v_2$ and $v_2 \leqp
          v_1 \userlub{}{} v_2$.

        \item Case $d_1 \not\userleq x_2$:

          By the definition of $\lubp{}{}$, we know
          $\state{d_1}{\frozenfalse} \lubp{}{}
          \state{x_2}{\frozentrue} = \state{\top}{\frozenfalse}$.

          Since $d_1 \userleq \top$, by the definition of $\leqp$ we
          know $\state{d_1}{\frozenfalse} \userleq
          \state{\top}{\frozenfalse}$.

          Hence $v_1 \leqp \lubp{v_1}{v_2}$.

          By the definition of $\leqp$, we know
          $\state{x_2}{\frozentrue} \userleq
          \state{\top}{\frozenfalse}$.

          Hence $v_2 \leqp \lubp{v_1}{v_2}$.

          Therefore $v_1 \leqp v_1 \userlub{}{} v_2$ and $v_2 \leqp
          v_1 \userlub{}{} v_2$.
        \end{itemize}
      \item Case $v_1 = \state{x_1}{\frozentrue}$ and $v_2 =
        \state{d_2}{\frozenfalse}$:

        Symmetric with the previous case. 
      \item Case $v_1 = \state{x_1}{\frozentrue}$ and $v_2 =
        \state{x_2}{\frozentrue}$:

        Consider whether $x_1$ equals $x_2$. 
        \begin{itemize}
        \item Case $x_1 = x_2$:
          
          By the definition $\lubp{}{}$, $\state{x_1}{\frozentrue}
          \lubp{}{} \state{x_2}{\frozentrue} =
          \state{x_1}{\frozentrue}$.
 
          By reflexivity, $\state{x_1}{\frozentrue} \leqp
          \state{x_1}{\frozentrue}$.

          Hence $v_1 \leqp \lubp{v_1}{v_2}$.

          By reflexivity, $\state{x_2}{\frozentrue} \leqp
          \state{x_1}{\frozentrue}$.

          Hence $v_2 \leqp \lubp{v_1}{v_2}$.

          Therefore $v_1 \leqp v_1 \userlub{}{} v_2$ and $v_2 \leqp
          v_1 \userlub{}{} v_2$.

        \item Case $x_1 \not= x_2$: 

          By the definition $\lubp{}{}$, $\state{x_1}{\frozentrue}
          \lubp{}{} \state{x_2}{\frozentrue} =
          \state{\top}{\frozenfalse}$.

          By the definition of $\leqp$, $\state{x_1}{\frozentrue}
          \leqp \state{\top}{\frozenfalse}$.

          Hence $v_1 \leqp \lubp{v_1}{v_2}$.

          By the definition of $\leqp$, $\state{x_2}{\frozentrue}
          \leqp \state{\top}{\frozenfalse}$.

          Hence $v_2 \leqp \lubp{v_1}{v_2}$.

          Therefore $v_1 \leqp v_1 \userlub{}{} v_2$ and $v_2 \leqp
          v_1 \userlub{}{} v_2$.
        \end{itemize}
      \end{itemize}
    \end{enumerate}

  \item $\botp$ is the least element of $D_p$. 

    $\botp$ is defined to be $\state{\bot}{\frozenfalse}$.

    In order to be the least element of $D_p$, it must be less than or
    equal to every element of $D_p$.

    By Lemma~\ref{lem:partition-of-Dp}, the elements of $D_p$
    partition into $\state{d}{\frozenfalse}$ for all $d \in D$, and
    $\state{x}{\frozentrue}$ for all $x \in X$, where $X = D -
    \setof{\top}$.

    We consider both cases:

    \begin{itemize}
    \item $\state{d}{\frozenfalse}$ for all $d \in D$:

      By the definition of $\leqp$, $\state{\bot}{\frozenfalse} \leqp
      \state{d}{\frozenfalse}$ iff $\bot \userleq d$.

      Since $\bot$ is the least element of $D$, $\bot \userleq d$.

      Therefore $\botp = \state{\bot}{\frozenfalse} \leqp
      \state{d}{\frozenfalse}$.

    \item $\state{x}{\frozentrue}$ for all $x \in X$:

      By the definition of $\leqp$, $\state{\bot}{\frozenfalse} \leqp
      \state{x}{\frozentrue}$ iff $\bot \userleq x$.

      Since $\bot$ is the least element of $D$, $\bot \userleq x$.

      Therefore $\botp = \state{\bot}{\frozenfalse} \leqp
      \state{x}{\frozentrue}$.

    \end{itemize}

    Therefore $\botp$ is less than or equal to all elements of $D_p$.

  \item $\topp$ is the greatest element of $D_p$.

    $\topp$ is defined to be $\state{\top}{\frozenfalse}$.

    In order to be the greatest element of $D_p$, every element of
    $D_p$ must be less than or equal to it.

    By Lemma~\ref{lem:partition-of-Dp}, the elements of $D_p$
    partition into $\state{d}{\frozenfalse}$ for all $d \in D$, and
    $\state{x}{\frozentrue}$ for all $x \in X$, where $X = D -
    \setof{\top}$.

    We consider both cases:

    \begin{itemize}
    \item $\state{d}{\frozenfalse}$ for all $d \in D$:

      By the definition of $\leqp$, $\state{d}{\frozenfalse} \leqp
      \state{\top}{\frozenfalse}$ iff $d \userleq \top$.

      Since $\top$ is the greatest element of $D$, $d \userleq \top$.

      Therefore $\state{d}{\frozenfalse} \leqp
      \state{\top}{\frozenfalse} = \topp$.

    \item $\state{x}{\frozentrue}$ for all $x \in X$:

      By the definition of $\leqp$, $\state{x}{\frozentrue} \leqp
      \state{\top}{\frozenfalse}$ iff $\top \userleq \top$.

      Therefore $\state{x}{\frozentrue} \leqp
      \state{\top}{\frozenfalse} = \topp$.

    \end{itemize}

    Therefore all elements of $D_p$ are less than or equal to $\topp$.
  \end{enumerate}
\end{proof}


\section{Proof of Lemma~\ref{lem:monotonicity}}\label{section:monotonicity-proof}
\begin{proof}
  \TODO{Fix the typos I found in this.}

  \begin{itemize}

    \item Case {\sc E-Eval-Ctxt}:

      Given: $\config{S}{\E{e}} \parstepsto \config{S'}{\E{e'}}$.

      To show: $\leqstore{S}{S'}$.

      From the premise of {\sc E-Eval-Ctxt}, $\config{S}{e}
      \parstepsto \config{S'}{e'}$.

      Hence by IH, $\leqstore{S}{S'}$, as we were required to show.

    \item Case {\sc E-Beta}:

      Immediate by the definition of $\leqstore{}{}$, since $S$ does
      not change.

    \item Case {\sc E-New}:

      Given: $\config{S}{\NEW} \parstepsto
      \config{\extS{S}{l}{\bot}{\frozenfalse}}{l}$.

      To show: $\leqstore{S}{\extS{S}{l}{\bot}{\frozenfalse}}$.

      By Definition~\ref{def:leqstore}, we have to show that $\dom{S}
      \subseteq \dom{\extS{S}{l}{\bot}{\frozenfalse}}$ and that for
      all $l' \in \dom{S}, \\
      S(l') \leqp (\extS{S}{l}{\bot}{\frozenfalse})(l')$.

      By the definition of store update,
      $\extS{S}{l}{d_1}{\frozentrue}$ can only either update an
      existing binding in $S$ or extend $S$ with a new binding.

      Hence $\dom{S} \subseteq \dom{\extS{S}{l}{\bot}{\frozenfalse}}$.

      From the side condition of {\sc E-New}, $l \notin \dom{S}$.

      Hence $\extS{S}{l}{\bot}{\frozenfalse}$ adds a new binding for
      $l$ in $S$.

      Hence $\extS{S}{l}{d_1}{\frozentrue}$ does not update any
      existing bindings in $S$.

      Hence, for all $l' \in \dom{S}, S(l') \leqp
      (\extS{S}{l}{d_1}{\frozentrue})(l')$.

      Therefore $\leqstore{S}{\extS{S}{l}{\bot}{\frozenfalse}}$, as
      required.

    \item Case {\sc E-Put}:

      Given: $\config{S}{\putexp{l}{d_2}} \parstepsto
      \config{\extSRaw{S}{l}{p_2}}{\unit}$.

      To show: $\leqstore{S}{\extSRaw{S}{l}{p_2}}$.

      By Definition~\ref{def:leqstore}, we have to show that $\dom{S}
      \subseteq \dom{\extSRaw{S}{l}{p_2}}$ and that for all $l' \in
      \dom{S}, \\
      S(l') \leqp (\extSRaw{S}{l}{p_2})(l')$.

      By the definition of store update, $\extSRaw{S}{l}{p_2}$ can only
      either update an existing binding in $S$ or extend $S$ with a
      new binding.

      Hence $\dom{S} \subseteq \dom{\extSRaw{S}{l}{p_2}}$.

      From the premises of {\sc E-Put}, $S(l) = p_1$.  Therefore $l
      \in \dom{S}$.

      Hence $\extSRaw{S}{l}{p_2}$ updates the existing binding for $l$
      in $S$ from $p_1$ to $p_2$.

      From the premises of {\sc E-Put}, $p_2 =
      \lubp{p_1}{\state{d_2}{\frozenfalse}}$.

      Hence, by the definition of $\lubp{}{}$, $p_1 \leqp p_2$.

      $\extSRaw{S}{l}{p_2}$ does not update any other bindings in $S$,
      hence, for all $l' \in \dom{S}, S(l') \leqp
      (\extSRaw{S}{l}{p_2})(l')$.

      Hence $\leqstore{S}{\extSRaw{S}{l}{p_2}}$, as required.

    \item Case {\sc E-Put-Err}:

      Given: $\config{S}{\putexp{l}{d_2}} \parstepsto \error$.

      By the definition of $\error$, $\error = \config{\topS}{e}$ for
      any $e$.

      To show: $\leqstore{S}{\topS}$.

      Immediate by the definition of $\leqstore{}{}$.

    \item Case {\sc E-Get}:

      Immediate by the definition of $\leqstore{}{}$, since $S$ does
      not change.

    \item Case {\sc E-Freeze-Init}:

      Immediate by the definition of $\leqstore{}{}$, since $S$ does
      not change.

    \item Case {\sc E-Spawn-Handler}:

      Immediate by the definition of $\leqstore{}{}$, since $S$ does
      not change.

    \item Case {\sc E-Freeze-Final}:

      Given: $\config{S}{\freezeafterfull{l}{Q}{v}{\setof{v\dots}}{H}}
      \parstepsto \config{\extS{S}{l}{d_1}{\frozentrue}}{d_1}$.

      To show: $\leqstore{S}{\extS{S}{l}{d_1}{\frozentrue}}$.

      By Definition~\ref{def:leqstore}, we have to show that $\dom{S}
      \subseteq \dom{\extS{S}{l}{d_1}{\frozentrue}}$ and that for all
      $l' \in \dom{S}, \\
      S(l') \leqp (\extS{S}{l}{d_1}{\frozentrue})(l')$.

      \lk{We could spell this out in even more excruciating detail,
        but I think it's obvious enough.}

      By the definition of store update,
      $\extS{S}{l}{d_1}{\frozentrue}$ can only either update an
      existing binding in $S$ or extend $S$ with a new binding.

      Hence $\dom{S} \subseteq \dom{\extS{S}{l}{d_1}{\frozentrue}}$.

      From the premises of {\sc E-Freeze-Final}, $S(l) =
      \state{d_1}{\status_1}$.  Therefore $l \in \dom{S}$.

      Hence $\extS{S}{l}{d_1}{\frozentrue}$ updates the existing
      binding for $l$ in $S$ from $\state{d_1}{\status_1}$ to
      $\state{d_1}{\frozentrue}$.

      By the definition of $\leqp$, $\state{d_1}{\status_1} \leqp
      \state{d_1}{\frozentrue}$.

      $\extS{S}{l}{d_1}{\frozentrue}$ does not update any other
      bindings in $S$, hence, for all $l' \in \dom{S}, \\
      S(l') \leqp (\extS{S}{l}{d_1}{\frozentrue})(l')$.

      Hence $\leqstore{S}{\extS{S}{l}{d_1}{\frozentrue}}$, as
      required.

    \item Case {\sc E-Freeze-Simple}:

      Given: $\config{S}{\freeze{l}} \parstepsto
      \config{\extS{S}{l}{d_1}{\frozentrue}}{d_1}$.

      To show: $\leqstore{S}{\extS{S}{l}{d_1}{\frozentrue}}$.

      Similar to the previous case.

  \end{itemize}

\end{proof}


\section{Proof of Lemma~\ref{lem:independence}}\label{section:independence-proof}
\begin{proof}
  Consider arbitrary $S''$ such that $S''$ is non-conflicting with
  $\config{S}{e} \parstepsto \config{S'}{e'}$ and $\lubstore{S'}{S''}
  \statuseq S$ and $\lubstore{S'}{S''} \neq \topS$.

  To show: $\config{\lubstore{S}{S''}}{e} \parstepsto
  \config{\lubstore{S'}{S''}}{e'}$.

  The proof is by cases on the rule of the reduction semantics by
  which $\config{S}{e}$ steps to $\config{S'}{e'}$.  Since
  $\config{S'}{e'} \neq \error$, we do not need to consider the {\sc
    E-Put-Err} rule.

  The assumption that $\lubstore{S'}{S''} \statuseq S$ is only needed
  in the {\sc E-Freeze-Final} and {\sc E-Freeze-Simple} cases.

  \begin{itemize}

    \item Case {\sc E-Beta}:

      Given: $\config{S}{\app{(\lam{x}{e})}{v}} \parstepsto
      \config{S}{\subst{e}{x}{v}}$.

      To show: $\config{\lubstore{S}{S''}}{\app{(\lam{x}{e})}{v}} \parstepsto
      \config{\lubstore{S}{S''}}{\subst{e}{x}{v}}$.

      Immediate by {\sc E-Beta}.

    \item Case {\sc E-New}:

      Given: $\config{S}{\NEW} \parstepsto
      \config{\extS{S}{l}{\bot}{\frozenfalse}}{l}$.

      To show: $\config{\lubstore{S}{S''}}{\NEW} \parstepsto
      \config{\lubstore{(\extS{S}{l}{\bot}{\frozenfalse})}{S''}}{l}$.

      By {\sc E-New}, we have that $\config{\lubstore{S}{S''}}{\NEW}
      \parstepsto
      \config{\extS{(\lubstore{S}{S''})}{l'}{\bot}{\frozenfalse}}{l'}$,
      where $l' \notin \dom{\lubstore{S}{S''}}$.

      By assumption, $S''$ is non-conflicting with $\config{S}{\NEW}
      \parstepsto \config{\extS{S}{l}{\bot}{\frozenfalse}}{l}$.
 
      Therefore $l \notin \dom{S''}$.

      From the side condition of {\sc E-New}, $l \notin \dom{S}$.

      Therefore $l \notin \dom{\lubstore{S}{S''}}$.

      Therefore, in
      $\config{\extS{(\lubstore{S}{S''})}{l'}{\bot}{\frozenfalse}}{l'}$,
      we can $\alpha$-rename $l'$ to $l$, resulting in
      $\config{\extS{(\lubstore{S}{S''})}{l}{\bot}{\frozenfalse}}{l}$.

      Therefore $\config{\lubstore{S}{S''}}{\NEW} \parstepsto
      \config{\extS{(\lubstore{S}{S''})}{l}{\bot}{\frozenfalse}}{l}$.

      Note that:
      \begin{align*}
        \extS{(\lubstore{S}{S''})}{l}{\bot}{\frozenfalse} &=
        \lubstore{\extS{S}{l}{\bot}{\frozenfalse}}{\extS{S''}{l}{\bot}{\frozenfalse}} \\
        &= \lubstore{\lubstore{S}{\store{\storebinding{l}{\bot}{\frozenfalse}}}}{\lubstore{S''}{\store{\storebinding{l}{\bot}{\frozenfalse}}}} \\
        &= \lubstore{\lubstore{S}{\store{\storebinding{l}{\bot}{\frozenfalse}}}}{S''} \\
        &= \lubstore{\extS{S}{l}{\bot}{\frozenfalse}}{S''}.
      \end{align*}
      Therefore $\config{\lubstore{S}{S''}}{\NEW} \parstepsto
      \config{\lubstore{\extS{S}{l}{\bot}{\frozenfalse}}{S''}}{l}$, as we were
      required to show.

    \item Case {\sc E-Put}:

      Given: $\config{S}{\putiexp{l}} \parstepsto
      \config{\extSRaw{S}{l}{u_{p_i}(p_1)}}{\unit}$.

      To show: $\config{\lubstore{S}{S''}}{\putiexp{l}{d_2}}
      \parstepsto
      \config{\lubstore{\extSRaw{S}{l}{u_{p_i}(p_1)}}{S''}}{\unit}$.

      We will first show that

      $\config{\lubstore{S}{S''}}{\putiexp{l}{d_2}} \parstepsto
      \config{\extSRaw{(\lubstore{S}{S''})}{l}{u_{p_i}(p_1)}}{\unit}$

      and then show why this is sufficient.

      We proceed by cases on $l$:

      \begin{itemize}
        \item $l \notin \dom{S''}$:

          By assumption, $\lubstore{\extSRaw{S}{l}{u_{p_i}(p_1)}}{S''}
          \neq \topS$.

          By Lemma~\ref{lem:monotonicity},
          $\leqstore{S}{\extSRaw{S}{l}{u_{p_i}(p_1)}}$.

          Hence $\lubstore{S}{S''} \neq \topS$.

          Therefore, by Definition~\ref{def:lubstore},
          $(\lubstore{S}{S''})(l) = S(l)$.

          From the premises of {\sc E-Put}, $S(l) = p_1$.

          Hence $(\lubstore{S}{S''})(l) = p_1$.

          From the premises of {\sc E-Put}, $u_{p_i}(p_1) \neq \topp$.

          Therefore, by {\sc E-Put}, we have:
          $\config{\lubstore{S}{S''}}{\putiexp{l}} \parstepsto
          \config{\extSRaw{(\lubstore{S}{S''})}{l}{u_{p_i}(p_1)}}{\unit}$.

        \item $l \in \dom{S''}$:

          By assumption, $\lubstore{\extSRaw{S}{l}{u_{p_i}(p_1)}}{S''} \neq
          \topS$.

          By Lemma~\ref{lem:monotonicity},
          $\leqstore{S}{\extSRaw{S}{l}{u_{p_i}(p_1)}}$.

          Hence $\lubstore{S}{S''} \neq \topS$.

          Therefore $(\lubstore{S}{S''})(l) = \lubp{S(l)}{S''(l)}$.

          From the premises of {\sc E-Put}, $S(l) = p_1$.
          
          Hence $(\lubstore{S}{S''})(l) = p'_1$, where $p_1 \leqp
          p'_1$.

          \TODO{From here forward, this subcase still needs to be
            fixed.}

          By assumption, $\lubstore{\extSRaw{S}{l}{p_2}}{S''} \neq
          \topS$.

          Therefore, by Definition~\ref{def:lubstore},
          $\lubp{p_2}{S''(l)} \neq \topp$.

          Note that:
          \begin{align*}
            \topp &\neq \lubp{p_2}{S''(l)} \\
            &= \lubp{\lubp{p_1}{\state{d_2}{\frozenfalse}}}{S''(l)} \\
            &= \lubp{\lubp{S(l)}{\state{d_2}{\frozenfalse}}}{S''(l)} \\
            &= \lubp{\lubp{S(l)}{S''(l)}}{\state{d_2}{\frozenfalse}} \\
            &= \lubp{(\lubstore{S}{S''})(l)}{\state{d_2}{\frozenfalse}} \\
            &= \lubp{p'_1}{\state{d_2}{\frozenfalse}} \\
            &= p'_2. \\
          \end{align*}
          Hence $p'_2 \neq \topp$.

          Hence $(\lubstore{S}{S''})(l) = p'_1$ and $p'_2 =
          \lubp{p'_1}{\state{d_2}{\frozenfalse}}$ and $p'_2 \neq
          \topp$.

          Therefore, by {\sc E-Put} we have:
          $\config{\lubstore{S}{S''}}{\putiexp{l}{d_2}} \parstepsto
          \config{\extSRaw{(\lubstore{S}{S''})}{l}{p'_2}}{\unit}$.

          \lk{If we really wanted to be pedantic here, we'd actually
            prove that the stores are equal.  I'm assuming that if I
            can show that $\extSRaw{(\lubstore{S}{S''})}{l}{p'_2}$ and
            $\extSRaw{(\lubstore{S}{S''})}{l}{p_2}$ bind $l$ to the
            same value, then it will be obvious that they're equal.}

          Note that:
          \begin{align*}
            (\extSRaw{(\lubstore{S}{S''})}{l}{p'_2})(l) &= \lubp{(\lubstore{S}{S''})(l)}{(\store{\storebindingRaw{l}{p'_2}})(l)} \\
            &= \lubp{p'_1}{p'_2} \\
            &= \lubp{p'_1}{\lubp{p'_1}{\state{d_2}{\frozenfalse}}} \\
            &= \lubp{p'_1}{\state{d_2}{\frozenfalse}}
          \end{align*}
          and
          \begin{align*}
            (\extSRaw{(\lubstore{S}{S''})}{l}{p_2})(l) &= \lubp{(\lubstore{S}{S''})(l)}{(\store{\storebindingRaw{l}{p_2}})(l)} \\
            &= \lubp{p'_1}{p_2} \\
            &= \lubp{p'_1}{\lubp{p_1}{\state{d_2}{\frozenfalse}}} \\
            &= \lubp{p'_1}{\state{d_2}{\frozenfalse}} & \textrm{(since $p_1 \leqp p'_1$).}
          \end{align*}
          Therefore $\extSRaw{(\lubstore{S}{S''})}{l}{p'_2} =
          \extSRaw{(\lubstore{S}{S''})}{l}{p_2}$.

          Therefore, $\config{\lubstore{S}{S''}}{\putiexp{l}{d_2}}
          \parstepsto
          \config{\extSRaw{(\lubstore{S}{S''})}{l}{p_2}}{\unit}$.
      \end{itemize}

      Note that:
      \begin{align*}
        \extSRaw{(\lubstore{S}{S''})}{l}{p_2} &= \lubstore{\extSRaw{S}{l}{p_2}}{\extSRaw{S''}{l}{p_2}} \\
        &= \lubstore{\lubstore{S}{\store{\storebindingRaw{l}{p_2}}}}{\lubstore{S''}{\store{\storebindingRaw{l}{p_2}}}} \\
        &= \lubstore{\lubstore{S}{\store{\storebindingRaw{l}{p_2}}}}{S''} \\
        &= \lubstore{\extSRaw{S}{l}{p_2}}{S''}.
      \end{align*}
      Therefore $\config{\lubstore{S}{S''}}{\putiexp{l}{d_2}}
      \parstepsto \config{\lubstore{\extSRaw{S}{l}{p_2}}{S''}}{\unit}$,
      as we were required to show.

    \item Case {\sc E-Get}:

      Given: $\config{S}{\getexp{l}{P}} \parstepsto \config{S}{p_2}$.

      To show: $\config{\lubstore{S}{S''}}{\getexp{l}{P}} \parstepsto
      \config{\lubstore{S}{S''}}{p_2}$.

      From the premises of {\sc E-Get}, $S(l) = p_1$ and $\incomp{P}$
      and $p_2 \in P$ and $p_2 \leqp p_1$.

      By assumption, $\lubstore{S}{S''} \neq \topS$.

      Hence $(\lubstore{S}{S''}) = p'_1$, where $p_1 \leqp p'_1$.

      By the transitivity of $\leqp$, $p_2 \leqp p'_1$.

      Hence, $S(l) = p'_1$ and $\incomp{P}$ and $p_2 \in P$ and $p_2
      \leqp p'_1$.

      Therefore, by {\sc E-Get},

      $\config{\lubstore{S}{S''}}{\getexp{l}{P}} \parstepsto
      \config{\lubstore{S}{S''}}{p_2}$,

      as we were required to show.

    \item Case {\sc E-Freeze-Init}:

      Given: $\config{S}{\freezeafter{l}{Q}{\lam{x}{e}}} \parstepsto
      \config{S}{\freezeafterfull{l}{Q}{\lam{x}{e}}{\setof{}}{\setof{}}}$.

      To show:
      $\config{\lubstore{S}{S''}}{\freezeafter{l}{Q}{\lam{x}{e}}}
      \parstepsto
      \config{\lubstore{S}{S''}}{\freezeafterfull{l}{Q}{\lam{x}{e}}{\setof{}}{\setof{}}}$.

      Immediate by {\sc E-Freeze-Init}.

    \item Case {\sc E-Spawn-Handler}:

      Given:

      $\config{S}{\freezeafterfull{l}{Q}{\lam{x}{e_0}}{\setof{e,
            \dots}}{H}} \parstepsto
      \config{S}{\freezeafterfull{l}{Q}{\lam{x}{e_0}}{\setof{\subst{e_0}{x}{d_2},
            e, \dots}} {\{d_2\}\cup H}}$.

      To show:

      $\config{\lubstore{S}{S''}}{\freezeafterfull{l}{Q}{\lam{x}{e_0}}{\setof{e,
            \dots}}{H}} \parstepsto
      \config{\lubstore{S}{S''}}{\freezeafterfull{l}{Q}{\lam{x}{e_0}}{\setof{\subst{e_0}{x}{d_2},
            e, \dots}} {\{d_2\}\cup H}}$.

      From the premises of {\sc E-Spawn-Handler}, $S(l) =
      \state{d_1}{\status_1}$ and $d_2 \userleq d_1$ and $d_2 \notin
      H$ and $d_2 \in Q$.

      By assumption, $\lubstore{S}{S''} \neq \topS$.

      Hence $(\lubstore{S}{S''})(l) = \state{d'_1}{\status'_1}$ where
      $\state{d_1}{\status_1} \leqp \state{d'_1}{\status'_1}$.

      By Definition~\ref{def:lattice-with-status-bits}, $d_1 \userleq
      d'_1$.

      By the transitivity of $\userleq$, $d_2 \userleq d'_1$.

      Hence $(\lubstore{S}{S''})(l) =
      \state{d'_1}{\status'_1}$ and $d_2 \userleq d'_1$ and $d_2 \notin
      H$ and $d_2 \in Q$.

      Therefore, by {\sc E-Spawn-Handler},

      $\config{\lubstore{S}{S''}}{\freezeafterfull{l}{Q}{\lam{x}{e_0}}{\setof{e,
            \dots}}{H}} \parstepsto
      \config{\lubstore{S}{S''}}{\freezeafterfull{l}{Q}{\lam{x}{e_0}}{\setof{\subst{e_0}{x}{d_2},
            e, \dots}} {\{d_2\}\cup H}}$,

      as we were required to show.

    \item Case {\sc E-Freeze-Final}:

      \lk{This case wouldn't work but for the $\lubstore{S'}{S''}
        \statuseq S$ requirement, which makes it a no-op freeze.}

      Given:
      $\config{S}{\freezeafterfull{l}{Q}{\lam{x}{e_0}}{\setof{v,
            \dots}}{H}} \parstepsto
      \config{\extS{S}{l}{d_1}{\frozentrue}}{d_1}$.

      To show:
      $\config{\lubstore{S}{S''}}{\freezeafterfull{l}{Q}{\lam{x}{e_0}}{\setof{v,
            \dots}}{H}} \parstepsto
      \config{\lubstore{\extS{S}{l}{d_1}{\frozentrue}}{S''}}{d_1}$.

      We will first show that

      $\config{\lubstore{S}{S''}}{\freezeafterfull{l}{Q}{\lam{x}{e_0}}{\setof{v,
            \dots}}{H}} \parstepsto
      \config{\extS{(\lubstore{S}{S''})}{l}{d_1}{\frozentrue}}{d_1}$

      and then show why this is sufficient.

      We proceed by cases on $l$:
      \begin{itemize}
      \item $l \notin \dom{S''}$:

        By assumption, $\lubstore{\extS{S}{l}{d_1}{\frozentrue}}{S''}
        \neq \topS$.

        By Lemma~\ref{lem:monotonicity},
        $\leqstore{S}{\extS{S}{l}{d_1}{\frozentrue}}$.

        Therefore $\lubstore{S}{S''} \neq \topS$.

        Hence, by Definition~\ref{def:lubstore},
        $(\lubstore{S}{S''})(l) = S(l)$.

        From the premises of {\sc E-Freeze-Final}, we have that $S(l)
        = \state{d_1}{\status_1}$.

        Hence $(\lubstore{S}{S''})(l) = \state{d_1}{\status_1}$.

        From the premises of {\sc E-Freeze-Final}, we have that
        $\forall{d_2} ~.~ ( {d_2 \userleq d_1 \land d_2 \in Q} \Rightarrow d_2 \in
        H)$.

        Therefore, by {\sc E-Freeze-Final}, we have that

        $\config{\lubstore{S}{S''}}{\freezeafterfull{l}{Q}{\lam{x}{e_0}}{\setof{v,
              \dots}}{H}} \parstepsto
        \config{\extS{(\lubstore{S}{S''})}{l}{d_1}{\frozentrue}}{d_1}$.


      \item $l \in \dom{S''}$:

        By assumption, $\lubstore{\extS{S}{l}{d_1}{\frozentrue}}{S''}
        \neq \topS$.

        By Lemma~\ref{lem:monotonicity},
        $\leqstore{S}{\extS{S}{l}{d_1}{\frozentrue}}$.

        Therefore $\lubstore{S}{S''} \neq \topS$.

        Hence, by Definition~\ref{def:lubstore},
        $(\lubstore{S}{S''})(l) = \lubp{S(l)}{S''(l)}$.

        From the premises of {\sc E-Freeze-Final}, we have that
        $S(l) = \state{d_1}{\status_1}$.

        By assumption, $\lubstore{\extS{S}{l}{d_1}{\frozentrue}}{S''}
        \statuseq S$.

        Therefore $\status_1 = \frozentrue$.

        Therefore $S(l) = \state{d_1}{\frozentrue}$.

        Therefore $(\lubstore{S}{S''})(l) =
        \lubp{\state{d_1}{\frozentrue}}{S''(l)}$.

        We proceed by cases on $S''(l)$:
        \begin{itemize}
        \item $S''(l) = \state{d_3}{\frozenfalse}$, where $d_3 \userleq d_1$:

          By Definition~\ref{def:lubp},
          $\lubp{\state{d_1}{\frozentrue}}{\state{d_3}{\frozenfalse}}
          = \state{d_1}{\frozentrue}$.

          Therefore $(\lubstore{S}{S''})(l) =
          \state{d_1}{\frozentrue}$.

          From the premises of {\sc E-Freeze-Final}, we have that
          $\forall{d_2} ~.~ ( {d_2 \userleq d_1 \land d_2 \in Q} \Rightarrow d_2 \in
          H)$.

          Therefore, by {\sc E-Freeze-Final}, we have that

          $\config{\lubstore{S}{S''}}{\freezeafterfull{l}{Q}{\lam{x}{e_0}}{\setof{v,
                \dots}}{H}} \parstepsto
          \config{\extS{(\lubstore{S}{S''})}{l}{d_1}{\frozentrue}}{d_1}$.

        \item $S''(l) = \state{d_3}{\frozenfalse}$, where $d_3 \nuserleq d_1$:

          By Definition~\ref{def:lubp},
          $\lubp{\state{d_1}{\frozentrue}}{\state{d_3}{\frozenfalse}}
          = \state{\top}{\frozenfalse}$.

          Therefore $\lubp{S(l)}{S''(l)} =
          \state{\top}{\frozenfalse}$.

          By Definition~\ref{def:lattice-with-status-bits},
          $\state{\top}{\frozenfalse} = \topp$.

          Therefore $\lubp{S(l)}{S''(l)} = \topp$.

          Therefore, by Definition~\ref{def:lubstore},
          $\lubstore{S}{S''} = \topS$.

          This is a contradiction.

          Therefore,

          $\config{\lubstore{S}{S''}}{\freezeafterfull{l}{Q}{\lam{x}{e_0}}{\setof{v,
                \dots}}{H}} \parstepsto
          \config{\extS{(\lubstore{S}{S''})}{l}{d_1}{\frozentrue}}{d_1}$.

        \item $S''(l) = \state{d_3}{\frozentrue}$, where $d_3 = d_1$:

          By Definition~\ref{def:lubp},
          $\lubp{\state{d_1}{\frozentrue}}{\state{d_3}{\frozentrue}} =
          \state{d_1}{\frozentrue}$.

          Therefore $(\lubstore{S}{S''})(l) = \state{d_1}{\frozentrue}$.

          From the premises of {\sc E-Freeze-Final}, we have that
          $\forall{d_2} ~.~ ( {d_2 \userleq d_1 \land d_2 \in Q} \Rightarrow d_2 \in
          H)$.

          Therefore, by {\sc E-Freeze-Final}, we have that

          $\config{\lubstore{S}{S''}}{\freezeafterfull{l}{Q}{\lam{x}{e_0}}{\setof{v,
                \dots}}{H}} \parstepsto
          \config{\extS{(\lubstore{S}{S''})}{l}{d_1}{\frozentrue}}{d_1}$.

        \item $S''(l) = \state{d_3}{\frozentrue}$, where $d_3 \neq d_1$:

          By Definition~\ref{def:lubp},
          $\lubp{\state{d_1}{\frozentrue}}{\state{d_3}{\frozentrue}}
          = \state{\top}{\frozenfalse}$.

          Therefore $\lubp{S(l)}{S''(l)} = \state{\top}{\frozenfalse}$.

          By Definition~\ref{def:lattice-with-status-bits},
          $\state{\top}{\frozenfalse} = \topp$.

          Therefore $\lubp{S(l)}{S''(l)} = \topp$.

          Therefore, by Definition~\ref{def:lubstore},
          $\lubstore{S}{S''} = \topS$.

          This is a contradiction.

          Therefore,

          $\config{\lubstore{S}{S''}}{\freezeafterfull{l}{Q}{\lam{x}{e_0}}{\setof{v,
                \dots}}{H}} \parstepsto
          \config{\extS{(\lubstore{S}{S''})}{l}{d_1}{\frozentrue}}{d_1}$.
        \end{itemize}
      \end{itemize}

      In each case we have shown that

      $\config{\lubstore{S}{S''}}{\freezeafterfull{l}{Q}{\lam{x}{e_0}}{\setof{v,
            \dots}}{H}} \parstepsto
      \config{\extS{(\lubstore{S}{S''})}{l}{d_1}{\frozentrue}}{d_1}$.

      Note that:
      \begin{align*}
        \extS{(\lubstore{S}{S''})}{l}{d_1}{\frozentrue} &=
        \lubstore{\extS{S}{l}{d_1}{\frozentrue}}{\extS{S''}{l}{d_1}{\frozentrue}} \\
        &= \lubstore{\lubstore{S}{\store{\storebinding{l}{d_1}{\frozentrue}}}}{\lubstore{S''}{\store{\storebinding{l}{d_1}{\frozentrue}}}} \\
        &= \lubstore{\lubstore{S}{\store{\storebinding{l}{d_1}{\frozentrue}}}}{S''} \\
        &= \lubstore{\extS{S}{l}{d_1}{\frozentrue}}{S''}.
      \end{align*}
      Therefore

      $\config{\lubstore{S}{S''}}{\freezeafterfull{l}{Q}{\lam{x}{e_0}}{\setof{v,
            \dots}}{H}} \parstepsto
      \config{\lubstore{\extS{S}{l}{d_1}{\frozentrue}}{S''}}{d_1}$,

      as we were required to show.

    \item Case {\sc E-Freeze-Simple}:

      Given: $\config{S}{\freeze{l}} \parstepsto
      \config{\extS{S}{l}{d_1}{\frozentrue}}{d_1}$.

      To show: $\config{\lubstore{S}{S''}}{\freeze{l}}
      \parstepsto
      \config{\lubstore{\extS{S}{l}{d_1}{\frozentrue}}{S''}}{d_1}$.

      We will first show that

      $\config{\lubstore{S}{S''}}{\freeze{l}} \parstepsto
      \config{\extS{(\lubstore{S}{S''})}{l}{d_1}{\frozentrue}}{d_1}$

      and then show why this is sufficient.

      We proceed by cases on $l$:
      \begin{itemize}
      \item $l \notin \dom{S''}$:

        By assumption, $\lubstore{\extS{S}{l}{d_1}{\frozentrue}}{S''}
        \neq \topS$.

        By Lemma~\ref{lem:monotonicity},
        $\leqstore{S}{\extS{S}{l}{d_1}{\frozentrue}}$.

        Therefore $\lubstore{S}{S''} \neq \topS$.

        Hence, by Definition~\ref{def:lubstore},
        $(\lubstore{S}{S''})(l) = S(l)$.

        From the premise of {\sc E-Freeze-Simple}, we have that
        $S(l) = \state{d_1}{\status_1}$.

        Therefore, by {\sc E-Freeze-Simple}, we have that

        $\config{\lubstore{S}{S''}}{\freeze{l}}
        \parstepsto
        \config{\extS{(\lubstore{S}{S''})}{l}{d_1}{\frozentrue}}{d_1}$.

      \item $l \in \dom{S''}$:

        By assumption, $\lubstore{\extS{S}{l}{d_1}{\frozentrue}}{S''}
        \neq \topS$.

        By Lemma~\ref{lem:monotonicity},
        $\leqstore{S}{\extS{S}{l}{d_1}{\frozentrue}}$.

        Therefore $\lubstore{S}{S''} \neq \topS$.

        Hence, by Definition~\ref{def:lubstore},
        $(\lubstore{S}{S''})(l) = \lubp{S(l)}{S''(l)}$.

        From the premise of {\sc E-Freeze-Simple}, we have that
        $S(l) = \state{d_1}{\status_1}$.

        By assumption, $\lubstore{\extS{S}{l}{d_1}{\frozentrue}}{S''}
        \statuseq S$.

        Therefore $\status_1 = \frozentrue$.

        Therefore $(\lubstore{S}{S''})(l) =
        \lubp{\state{d_1}{\frozentrue}}{S''(l)}$.
        
        We proceed by cases on $S''(l)$:
        \begin{itemize}
        \item $S''(l) = \state{d_2}{\frozenfalse}$, where $d_2 \userleq d_1$:

          By Definition~\ref{def:lubp},
          $\lubp{\state{d_1}{\frozentrue}}{\state{d_2}{\frozenfalse}} =
          \state{d_1}{\frozentrue}$.

          Therefore $(\lubstore{S}{S''})(l) =
          \state{d_1}{\frozentrue}$.

          Therefore, by {\sc E-Freeze-Simple}, we have that

          $\config{\lubstore{S}{S''}}{\freeze{l}}
          \parstepsto
          \config{\extS{(\lubstore{S}{S''})}{l}{d_1}{\frozentrue}}{d_1}$.

        \item $S''(l) = \state{d_2}{\frozenfalse}$, where $d_2 \nuserleq d_1$:

          By Definition~\ref{def:lubp},
          $\lubp{\state{d_1}{\frozentrue}}{\state{d_2}{\frozenfalse}}
          = \state{\top}{\frozenfalse}$.

          By Definition~\ref{def:lattice-with-status-bits},
          $\state{\top}{\frozenfalse} = \topp$.

          Therefore $\lubp{S(l)}{S''(l)} = \topp$.

          Therefore, by Definition~\ref{def:lubstore},
          $\lubstore{S}{S''} = \topS$.

          This is a contradiction.

          Therefore,

          $\config{\lubstore{S}{S''}}{\freeze{l}}
          \parstepsto
          \config{\extS{(\lubstore{S}{S''})}{l}{d_1}{\frozentrue}}{d_1}$.

        \item $S''(l) = \state{d_2}{\frozentrue}$, where $d_2 = d_1$:

          Therefore $(\lubstore{S}{S''})(l) =
          \lubp{\state{d_1}{\frozentrue}}{\state{d_2}{\frozentrue}}$.

          By Definition~\ref{def:lubp},
          $\lubp{\state{d_1}{\frozentrue}}{\state{d_2}{\frozentrue}} =
          \state{d_1}{\frozentrue}$.

          Therefore $(\lubstore{S}{S''})(l) =
          \state{d_1}{\frozentrue}$.

          Therefore, by {\sc E-Freeze-Simple}, we have that

          $\config{\lubstore{S}{S''}}{\freeze{l}}
          \parstepsto
          \config{\extS{(\lubstore{S}{S''})}{l}{d_1}{\frozentrue}}{d_1}$.

        \item $S''(l) = \state{d_2}{\frozentrue}$, where $d_2 \neq d_1$:

          By Definition~\ref{def:lubp},
          $\lubp{\state{d_1}{\frozentrue}}{\state{d_2}{\frozentrue}}
          = \state{\top}{\frozenfalse}$.

          By Definition~\ref{def:lattice-with-status-bits},
          $\state{\top}{\frozenfalse} = \topp$.

          Therefore $\lubp{S(l)}{S''(l)} = \topp$.

          Therefore, by Definition~\ref{def:lubstore},
          $\lubstore{S}{S''} = \topS$.

          This is a contradiction.

          Therefore,

          $\config{\lubstore{S}{S''}}{\freeze{l}}
          \parstepsto
          \config{\extS{(\lubstore{S}{S''})}{l}{d_1}{\frozentrue}}{d_1}$.
        \end{itemize}
      \end{itemize}

      In each case we have shown that

      $\config{\lubstore{S}{S''}}{\freeze{l}} \parstepsto
      \config{\extS{(\lubstore{S}{S''})}{l}{d_1}{\frozentrue}}{d_1}$.

      Note that:
      \begin{align*}
        \extS{(\lubstore{S}{S''})}{l}{d_1}{\frozentrue} &=
        \lubstore{\extS{S}{l}{d_1}{\frozentrue}}{\extS{S''}{l}{d_1}{\frozentrue}} \\
        &= \lubstore{\lubstore{S}{\store{\storebinding{l}{d_1}{\frozentrue}}}}{\lubstore{S''}{\store{\storebinding{l}{d_1}{\frozentrue}}}} \\
        &= \lubstore{\lubstore{S}{\store{\storebinding{l}{d_1}{\frozentrue}}}}{S''} \\
        &= \lubstore{\extS{S}{l}{d_1}{\frozentrue}}{S''}.
      \end{align*}
      Therefore
      $\config{\lubstore{S}{S''}}{\freeze{l}}
      \parstepsto
      \config{\lubstore{\extS{S}{l}{d_1}{\frozentrue}}{S''}}{d_1}$,
      as we were required to show.
  \end{itemize}
\end{proof}


\section{Proof of Lemma~\ref{lem:strong-local-quasi-confluence}}\label{section:strong-local-quasi-confluence-proof}
\begin{proof}
  Suppose $\conf \ctxstepsto \conf_a$ and $\conf \ctxstepsto \conf_b$.

  We have to show that either there exist $\conf_c, i, j, \pi$ such
  that $\conf_a \ctxstepsto^i \conf_c$ and $\pi(\conf_b) \ctxstepsto^j
  \conf_c$ and $i \leq 1$ and $j \leq 1$, or that $\conf_a \ctxstepsto
  \error$ or $\conf_b \ctxstepsto \error$.

  By inspection of the operational semantics, it must be the case that
  $\conf$ steps to $\conf_a$ by the {\sc E-Eval-Ctxt} rule.

  Let $\conf = \config{S}{\evalctxt{E_a}{e_{a_1}}}$ and let $\conf_a =
  \config{S_a}{\evalctxt{E_a}{e_{a_2}}}$.

  Likewise, it must be the case that $\conf$ steps to $\conf_b$ by the
  {\sc E-Eval-Ctxt} rule.

  Let $\conf = \config{S}{\evalctxt{E_b}{e_{b_1}}}$ and let $\conf_b =
  \config{S_b}{\evalctxt{E_b}{e_{b_2}}}$.

  Note that $\conf = \config{S}{\evalctxt{E_a}{e_{a_1}}} =
  \config{S}{\evalctxt{E_b}{e_{b_1}}}$, and so
  $\evalctxt{E_a}{e_{a_1}} = \evalctxt{E_b}{e_{b_1}}$, but $E_a$ and
  $E_b$ may differ and $e_{a_1}$ and $e_{b_1}$ may differ.

  First, consider the possibility that $E_a = E_b$ (and $e_{a_1} =
  e_{b_1}$).

  Since $\config{S}{\evalctxt{E_a}{e_{a_1}}} \ctxstepsto
  \config{S_a}{\evalctxt{E_a}{e_{a_2}}}$ by {\sc E-Eval-Ctxt} and
  $\config{S}{\evalctxt{E_b}{e_{b_1}}} \ctxstepsto
  \config{S_b}{\evalctxt{E_b}{e_{b_2}}}$ by {\sc E-Eval-Ctxt}, we have
  from the premise of {\sc E-Eval-Ctxt} that $\config{S}{e_{a_1}}
  \parstepsto \config{S_a}{e_{a_2}}$ and $\config{S}{e_{b_1}}
  \parstepsto \config{S_b}{e_{b_2}}$.

  But then, since $e_{a_1} = e_{b_1}$, by Internal Determinism
  (Lemma~\ref{lem:internal-determinism}) there is a permutation $\pi'$
  such that $\config{S_a}{e_{a_2}} = \pi'(\config{S_b}{e_{b_2}})$,
  modulo choice of events.

  We have two cases:

  \begin{itemize}
  \item In the case where the steps $\conf \ctxstepsto \conf_a$ and
    $\conf \ctxstepsto \conf_b$ are both by {\sc E-Spawn-Handler} and
    they handle different events $d_2$ and $d'_2$, then we can satisfy
    the proof by choosing the final configuration $\conf_c$ as the
    configuration where both $d_2$ and $d'_2$ have been handled.

    Both $\conf_a$ and $\conf_b$ can step to this configuration by
    {\sc E-Spawn-Handler}: if the step from $\conf$ to $\conf_a$
    handles $d_2$ then the step from $\conf_a$ to $\conf_c$ handles
    $d'_2$, while if the step from $\conf$ to $\conf_b$ handles $d'_2$
    then the step from $\conf_b$ to $\conf_c$ handles $d_2$.

    The store in the final configuration is $S_a$ or $S_b$, which are
    equal because {\sc E-Spawn-Handler} does not affect the store, and
    we can satisfy the proof by choosing $i = 1$ and $j = 0$ and $\pi
    = \id$.

  \item Otherwise, we can satisfy the proof by choosing $\conf_c =
    \config{S_a}{e_{a_2}}$ and $i = 0$ and $j = 0$ and $\pi = \id$.
  \end{itemize}

  The rest of this proof deals with the more interesting case in which
  $E_a \neq E_b$ (and $e_{a_1} \neq e_{b_1}$).

  Since $\config{S}{\evalctxt{E_a}{e_{a_1}}} \ctxstepsto
  \config{S_a}{\evalctxt{E_a}{e_{a_2}}}$ and
  $\config{S}{\evalctxt{E_b}{e_{b_1}}} \ctxstepsto
  \config{S_b}{\evalctxt{E_b}{e_{b_2}}}$ and $\evalctxt{E_a}{e_{a_1}}
  = \evalctxt{E_b}{e_{b_1}}$, and since $E_a \neq E_b$, we have from
  Lemma~\ref{lem:locality} (Locality) that there exist evaluation
  contexts $E'_a$ and $E'_b$ such that:

  \begin{itemize}
  \item $\evalctxt{E'_a}{e_{a_1}} = \evalctxt{E_b}{e_{b_2}}$, and
  \item $\evalctxt{E'_b}{e_{b_1}} = \evalctxt{E_a}{e_{a_2}}$, and
  \item $\evalctxt{E'_a}{e_{a_2}} =
    \evalctxt{E'_b}{e_{b_2}}$.
  \end{itemize}

  In some of the cases that follow, we will choose $\conf_c = \error$,
  and in some we will prove that one of $\conf_a$ or $\conf_b$ steps
  to $\error$.

  In most cases, however, our approach will be to show that there
  exist $S', i, j, \pi$ such that:
  \begin{itemize}
  \item $\config{S_a}{\evalctxt{E_a}{e_{a_2}}} \ctxstepsto^i
    \config{S'}{\evalctxt{E'_a}{e_{a_2}}}$, and
  \item $\pi(\config{S_b}{\evalctxt{E_b}{e_{b_2}}}) \ctxstepsto^j
    \config{S'}{\evalctxt{E'_a}{e_{a_2}}}$.
  \end{itemize}
  Since $\evalctxt{E'_a}{e_{a_1}} = \evalctxt{E_b}{e_{b_2}}$,
  $\evalctxt{E'_b}{e_{b_1}} = \evalctxt{E_a}{e_{a_2}}$, and
  $\evalctxt{E'_a}{e_{a_2}} = \evalctxt{E'_b}{e_{b_2}}$, it suffices
  to show that:
  \begin{itemize}
  \item $\config{S_a}{\evalctxt{E'_b}{e_{b_1}}} \ctxstepsto^i
    \config{S'}{\evalctxt{E'_b}{e_{b_2}}}$, and
  \item $\pi(\config{S_b}{\evalctxt{E'_a}{e_{a_1}}}) \ctxstepsto^j
    \config{S'}{\evalctxt{E'_a}{e_{a_2}}}$.
  \end{itemize}
  From the premise of {\sc E-Eval-Ctxt}, we have that
  $\config{S}{e_{a_1}} \parstepsto \config{S_a}{e_{a_2}}$ and
  $\config{S}{e_{b_1}} \parstepsto \config{S_b}{e_{b_2}}$.

  We proceed by case analysis on the rule by which
  $\config{S}{e_{a_1}}$ steps to $\config{S_a}{e_{a_2}}$.

  Since the only way an $\error$ configuration can arise is by the
  {\sc E-Put-Err} rule, we can assume in all other cases that $\conf_a
  \neq \error$.

  \begin{enumerate}
  \item Case {\sc E-Beta}: We have $S_a = S$.

    We proceed by case analysis on the rule by which
    $\config{S}{e_{b_1}}$ steps to $\config{S_b}{e_{b_2}}$.

    Since the only way an $\error$ configuration can arise is by the
    {\sc E-Put-Err} rule, we can assume in all other cases that
    $\conf_b \neq \error$.
    \begin{enumerate}
    \item \label{slqc-beta-beta}Case {\sc E-Beta}: We have $S_a = S$
      and $S_b = S$.

      Choose $S' = S = S_a = S_b$, $i = 1$, $j = 1$, and $\pi = \id$.

      We have to show that:
      \begin{itemize}
      \item $\config{S}{\evalctxt{E'_b}{e_{b_1}}} \ctxstepsto
        \config{S_a}{\evalctxt{E'_b}{e_{b_2}}}$, and
      \item $\config{S}{\evalctxt{E'_a}{e_{a_1}}} \ctxstepsto
        \config{S_b}{\evalctxt{E'_a}{e_{a_2}}}$, 
      \end{itemize}

      both of which follow immediately from $\config{S}{e_{a_1}}
      \parstepsto \config{S_a}{e_{a_2}}$ and $\config{S}{e_{b_1}}
      \parstepsto \config{S_b}{e_{b_2}}$ and {\sc E-Eval-Ctxt}.

    \item \label{slqc-beta-new}Case {\sc E-New}: We have $S_a = S$ and
      $S_b = \extS{S}{l}{\bot}{\frozenfalse}$.

      Choose $S' = S_b$, $i = 1$, $j = 1$, and $\pi = \id$.

      We have to show that:
      \begin{itemize}
      \item $\config{S}{\evalctxt{E'_b}{e_{b_1}}} \ctxstepsto
        \config{S_b}{\evalctxt{E'_b}{e_{b_2}}}$, and
      \item $\config{S_b}{\evalctxt{E'_a}{e_{a_1}}} \ctxstepsto
        \config{S_b}{\evalctxt{E'_a}{e_{a_2}}}$.
      \end{itemize}

      The first of these follows immediately from $\config{S}{e_{b_1}}
      \parstepsto \config{S_b}{e_{b_2}}$ and {\sc E-Eval-Ctxt}.

      For the second, consider that $S_b =
      \extS{S}{l}{\bot}{\frozenfalse} = U_S(S)$, where $U_S$ is the
      store update operation that acts as the identity on the contents
      of all existing locations, and adds the binding
      $\storebinding{l}{\bot}{\frozenfalse}$ if no binding for $l$
      exists.

      Note that:
      \begin{itemize}
      \item $U_S$ is non-conflicting with $\config{S}{e_{a_1}}
        \parstepsto \config{S_a}{e_{a_2}}$, since no locations are
        allocated in the transition;
      \item $U_S(S_a) \neq \topS$, since $U_S(S_a) = U_S(S) = S_b$
        and we know that $\conf_b \neq \error$; and
      \item $U_S$ is freeze-safe with $\config{S}{e_{a_1}}
        \parstepsto \config{S_a}{e_{a_2}}$, since $S_a = S$, so
        there are no locations whose contents differ in status
        between them.
      \end{itemize}

      Therefore, by Lemma~\ref{lem:generalized-independence}
      (Generalized Independence), we have that

      $\config{U_S(S)}{e_{a_1}} \parstepsto
      \config{U_S(S_a)}{e_{a_2}}$.

      Hence $\config{S_b}{e_{a_1}} \parstepsto \config{S_b}{e_{a_2}}$.

      By {\sc E-Eval-Ctxt}, it follows that
      $\config{S_b}{\evalctxt{E'_a}{e_{a_1}}} \ctxstepsto
      \config{S_b}{\evalctxt{E'_a}{e_{a_2}}}$,
      as we were required to show.

    \item \label{slqc-beta-put}Case {\sc E-Put}: We have $S_a = S$ and
      $S_b = \extSRaw{S}{l}{u_{p_i}(p_1)}$.

      Choose $S' = S_b$, $i = 1$, $j = 1$, and $\pi = \id$.

      We have to show that:
      \begin{itemize}
      \item $\config{S}{\evalctxt{E'_b}{e_{b_1}}} \ctxstepsto
        \config{S_b}{\evalctxt{E'_b}{e_{b_2}}}$, and
      \item $\config{S_b}{\evalctxt{E'_a}{e_{a_1}}} \ctxstepsto
        \config{S_b}{\evalctxt{E'_a}{e_{a_2}}}$.
      \end{itemize}

      The first of these follows immediately from $\config{S}{e_{b_1}}
      \parstepsto \config{S_b}{e_{b_2}}$ and {\sc E-Eval-Ctxt}.

      For the second, consider that $S_b = U_S(S)$, where $U_S$ is the
      store update operation that applies $u_{p_i}$ to the contents of
      $l$ and acts as the identity on all other locations.

      Note that:
      \begin{itemize}
      \item $U_S$ is non-conflicting with $\config{S}{e_{a_1}}
        \parstepsto \config{S_a}{e_{a_2}}$, since no locations are
        allocated in the transition;
      \item $U_S(S_a) \neq \topS$, since $U_S(S_a) = U_S(S) = S_b$
        and we know that $\conf_b \neq \error$; and
      \item $U_S$ is freeze-safe with $\config{S}{e_{a_1}}
        \parstepsto \config{S_a}{e_{a_2}}$, since $S_a = S$, so
        there are no locations whose contents differ in status
        between them.
      \end{itemize}

      Therefore, by Lemma~\ref{lem:generalized-independence}
      (Generalized Independence), we have that

      $\config{U_S(S)}{e_{a_1}} \parstepsto
      \config{U_S(S_a)}{e_{a_2}}$.

      Hence $\config{S_b}{e_{a_1}} \parstepsto \config{S_b}{e_{a_2}}$.

      By {\sc E-Eval-Ctxt}, it follows that
      $\config{S_b}{\evalctxt{E'_a}{e_{a_1}}} \ctxstepsto
      \config{S_b}{\evalctxt{E'_a}{e_{a_2}}}$, as we were required to
      show.

    \item \label{slqc-beta-put-err}Case {\sc E-Put-Err}: We have $S_a
      = S$ and $\config{S_b}{e_{b_2}} = \error$, and so we choose
      $\conf_c = \error$, $i = 1$, $j = 0$, and $\pi = \id$.

      We have to show that:
      \begin{itemize}
      \item $\config{S}{\evalctxt{E'_b}{e_{b_1}}} \ctxstepsto \error$,
        and
      \item $\config{S_b}{\evalctxt{E'_a}{e_{a_1}}} = \error$.
      \end{itemize}

      The second of these is immediately true because since
      $\config{S_b}{e_{b_2}} = \error$, $S_b = \topS$, and so
      $\config{S_b}{\evalctxt{E'_a}{e_{a_1}}}$ is equal to $\error$ as
      well.

      For the first, observe that $\config{S}{e_{b_1}} \parstepsto
      \config{S_b}{e_{b_2}}$, hence by {\sc E-Eval-Ctxt},
      $\config{S}{\evalctxt{E'_b}{e_{b_1}}} \ctxstepsto
      \config{S_b}{\evalctxt{E'_b}{e_{b_2}}}$.

      But $S_b = \topS$, so $\config{S_b}{\evalctxt{E'_b}{e_{b_2}}}$
      is equal to $\error$, and so
      $\config{S}{\evalctxt{E'_b}{e_{b_1}}} \ctxstepsto \error$, as
      required.

    \item \label{slqc-beta-get}Case {\sc E-Get}: Similar to
      case~\ref{slqc-beta-beta}, since $S_a = S$ and $S_b = S$.
    \item \label{slqc-beta-freeze-init}Case {\sc E-Freeze-Init}:
      Similar to case~\ref{slqc-beta-beta}, since $S_a = S$ and $S_b =
      S$.
    \item \label{slqc-beta-spawn-handler}Case {\sc E-Spawn-Handler}:
      Similar to case~\ref{slqc-beta-beta}, since $S_a = S$ and $S_b =
      S$.
    \item \label{slqc-beta-freeze-final}Case {\sc E-Freeze-Final}: We
      have $S_a = S$ and $S_b = \extS{S}{l}{d_1}{\frozentrue}$.

      Choose $S' = S_b$, $i = 1$, $j = 1$, and $\pi = \id$.

      We have to show that:
      \begin{itemize}
      \item $\config{S}{\evalctxt{E'_b}{e_{b_1}}} \ctxstepsto
        \config{S_b}{\evalctxt{E'_b}{e_{b_2}}}$, and
      \item $\config{S_b}{\evalctxt{E'_a}{e_{a_1}}} \ctxstepsto
        \config{S_b}{\evalctxt{E'_a}{e_{a_2}}}$.
      \end{itemize}

      The first of these follows immediately from $\config{S}{e_{b_1}}
      \parstepsto \config{S_b}{e_{b_2}}$ and {\sc E-Eval-Ctxt}.

      For the second, note that $S_b = U_S(S)$, where $U_S$ is the
      store update operation that freezes the contents of $l$ and acts
      as the identity on the contents of all other locations.

      Note that:
      \begin{itemize}
      \item $U_S$ is non-conflicting with $\config{S}{e_{a_1}}
        \parstepsto \config{S_a}{e_{a_2}}$, since no locations are
        allocated in the transition;
      \item $U_S(S_a) \neq \topS$, since $U_S(S_a) = U_S(S) = S_b$
        and we know that $\conf_b \neq \error$; and
      \item $U_S$ is freeze-safe with $\config{S}{e_{a_1}}
        \parstepsto \config{S_a}{e_{a_2}}$, since $S_a = S$, so
        there are no locations whose contents differ in status
        between them.
      \end{itemize}

      Therefore, by Lemma~\ref{lem:generalized-independence}
      (Generalized Independence), we have that

      $\config{U_S(S)}{e_{a_1}} \parstepsto
      \config{U_S(S_a)}{e_{a_2}}$.

      Hence $\config{S_b}{e_{a_1}} \parstepsto \config{S_b}{e_{a_2}}$.

      By {\sc E-Eval-Ctxt}, it follows that
      $\config{S_b}{\evalctxt{E'_a}{e_{a_1}}} \ctxstepsto
      \config{S_b}{\evalctxt{E'_a}{e_{a_2}}}$, as we were required to
      show.

    \item \label{slqc-beta-freeze-simple}Case {\sc E-Freeze-Simple}:
      Similar to case~\ref{slqc-beta-freeze-final}, since $S_b =
      \extS{S}{l}{d_1}{\frozentrue}$.

    \end{enumerate}
  \item Case {\sc E-New}: We have $S_a = \extS{S}{l}{\bot}{\frozenfalse}$.

    We proceed by case analysis on the rule by which
    $\config{S}{e_{b_1}}$ steps to $\config{S_b}{e_{b_2}}$.

    Since the only way an $\error$ configuration can arise is by the
    {\sc E-Put-Err} rule, we can assume in all other cases that
    $\conf_b \neq \error$.
    \begin{enumerate}
    \item \label{slqc-new-beta}Case {\sc E-Beta}: By symmetry with case~\ref{slqc-beta-new}.
    \item \label{slqc-new-new}Case {\sc E-New}: We have $S_a =
      \extS{S}{l}{\bot}{\frozenfalse}$ and $S_b =
      \extS{S}{l'}{\bot}{\frozenfalse}$.

      Now consider whether $l = l'$:
      \begin{itemize}
      \item If $l \neq l'$:

        Choose $S' =
        \extS{\extS{S}{l'}{\bot}{\frozenfalse}}{l}{\bot}{\frozenfalse}$,
        $i = 1$, $j = 1$, and $\pi = \id$.

        We have to show that:
        \begin{itemize}
        \item $\config{S_a}{\evalctxt{E'_b}{e_{b_1}}} \ctxstepsto
          \config{\extS{\extS{S}{l'}{\bot}{\frozenfalse}}{l}{\bot}{\frozenfalse}}{\evalctxt{E'_b}{e_{b_2}}}$,
          and
        \item $\config{S_b}{\evalctxt{E'_a}{e_{a_1}}} \ctxstepsto
          \config{\extS{\extS{S}{l'}{\bot}{\frozenfalse}}{l}{\bot}{\frozenfalse}}{\evalctxt{E'_a}{e_{a_2}}}$.
        \end{itemize}

        For the first of these, consider that $S_a =
        \extS{S}{l}{\bot}{\frozenfalse} = U_S(S)$, where $U_S$ is
        the store update operation that acts as the identity on the
        contents of all existing locations, and adds the binding
        $\storebinding{l}{\bot}{\frozenfalse}$ if no binding for $l$
        exists.

        Note that:
        \begin{itemize}
        \item $U_S$ is non-conflicting with $\config{S}{e_{b_1}}
          \parstepsto \config{S_b}{e_{b_2}}$, since the only
          location allocated in the transition is $l'$, and $l
          \neq l'$ in this case;
        \item $U_S(S_b) \neq \topS$, since $U_S(S_b) =
          \extS{\extS{S}{l'}{\bot}{\frozenfalse}}{l}{\bot}{\frozenfalse}$
          and we know $S \neq \topS$ and the addition of new
          bindings $\storebinding{l}{\bot}{\frozenfalse}$ and
          $\storebinding{l'}{\bot}{\frozenfalse}$ cannot cause it to
          become $\topS$; and
        \item $U_S$ is freeze-safe with $\config{S}{e_{b_1}}
          \parstepsto \config{S_b}{e_{b_2}}$, since $S_b =
          \extS{S}{l'}{\bot}{\frozenfalse}$ and $l' \notin \dom{S}$,
          so there are no locations whose contents differ in status
          between $S$ and $S_b$.
        \end{itemize}

        Therefore, by Lemma~\ref{lem:generalized-independence}
        (Generalized Independence), we have that

        $\config{U_S(S)}{e_{b_1}} \parstepsto
        \config{U_S(S_b)}{e_{b_2}}$.

        Hence $\config{\extS{S}{l}{\bot}{\frozenfalse}}{e_{b_1}}
        \parstepsto
        \config{\extS{S_b}{l}{\bot}{\frozenfalse}}{e_{b_2}}$.

        By {\sc E-Eval-Ctxt} it follows that

        $\config{\extS{S}{l}{\bot}{\frozenfalse}}{\evalctxt{E'_b}{e_{b_1}}}
        \parstepsto
        \config{\extS{S_b}{l}{\bot}{\frozenfalse}}{\evalctxt{E'_b}{e_{b_2}}}$,
        which, since $S_b = \extS{S}{l'}{\bot}{\frozenfalse}$, is what
        we were required to show.

        The argument for the second is symmetrical.

      \item If $l = l'$:

        In this case, observe that we do \emph{not} want the
        expression in the final configuration to be
        $\evalctxt{E'_a}{e_{a_2}}$ (nor its equivalent,
        $\evalctxt{E'_b}{e_{b_2}}$).

        The reason for this is that $\evalctxt{E'_a}{e_{a_2}}$
        contains both occurrences of $l$.

        Rather, we want both configurations to step to a configuration
        in which exactly one occurrence of $l$ has been renamed to a
        fresh location $l''$.

        Let $l''$ be a location such that $l'' \notin \dom{S}$ and
        $l'' \neq l$ (and hence $l'' \neq l'$, as well).

        Then choose $S' =
        \extS{\extS{S}{l''}{\bot}{\frozenfalse}}{l}{\bot}{\frozenfalse}$,
        $i = 1$, $j = 1$, and $\pi = \setof{(l, l'')}$.

        Either
        $\config{\extS{\extS{S}{l''}{\bot}{\frozenfalse}}{l}{\bot}{\frozenfalse}}{\evalctxt{E'_a}{\pi(e_{a_2})}}$
        or
        $\config{\extS{\extS{S}{l''}{\bot}{\frozenfalse}}{l}{\bot}{\frozenfalse}}{\evalctxt{E'_b}{\pi(e_{b_2})}}$
        would work as a final configuration; we choose

        $\config{\extS{\extS{S}{l''}{\bot}{\frozenfalse}}{l}{\bot}{\frozenfalse}}{\evalctxt{E'_b}{\pi(e_{b_2})}}$.

        We have to show that:
        \begin{itemize}
        \item $\config{S_a}{\evalctxt{E'_b}{e_{b_1}}} \ctxstepsto
          \config{\extS{\extS{S}{l''}{\bot}{\frozenfalse}}{l}{\bot}{\frozenfalse}}{\evalctxt{E'_b}{\pi(e_{b_2})}}$,
          and
        \item $\pi(\config{S_b}{\evalctxt{E'_a}{e_{a_1}}})
          \ctxstepsto
          \config{\extS{\extS{S}{l''}{\bot}{\frozenfalse}}{l}{\bot}{\frozenfalse}}{\evalctxt{E'_b}{\pi(e_{b_2})}}$.
        \end{itemize}

        For the first of these, since $\config{S}{e_{b_1}}
        \parstepsto \config{S_b}{e_{b_2}}$, we have by
        Lemma~\ref{lem:permutability} (Permutability) that
        $\pi(\config{S}{e_{b_1}}) \parstepsto
        \pi(\config{S_b}{e_{b_2}})$.

        Since $\pi = \setof{(l, l'')}$, but $l \notin S$ (from the
        side condition on {\sc E-New}), we have that
        $\pi(\config{S}{e_{b_1}}) = \config{S}{e_{b_1}}$.

        Since $\config{S_b}{e_{b_2}} =
        \config{\extS{S}{l'}{\bot}{\frozenfalse}}{l'}$, and $l = l'$,
        we have that $\pi(\config{S_b}{e_{b_2}}) =
        \config{\extS{S}{l''}{\bot}{\frozenfalse}}{\pi(e_{b_2})}$.

        Hence $\config{S}{e_{b_1}} \parstepsto
        \config{\extS{S}{l''}{\bot}{\frozenfalse}}{\pi(e_{b_2})}$.

        Let $U_S$ be the store update operation that acts as the
        identity on the contents of all existing locations, and adds
        the binding $\storebinding{l}{\bot}{\frozenfalse}$ if no
        binding for $l$ exists.

        Note that:
        \begin{itemize}
        \item $U_S$ is non-conflicting with $\config{S}{e_{b_1}}
          \parstepsto
          \config{\extS{S}{l''}{\bot}{\frozenfalse}}{\pi(e_{b_2})}$,
          since the only location allocated in the transition is
          $l''$;
        \item $U_S(\extS{S}{l''}{\bot}{\frozenfalse}) \neq \topS$,
          since $U_S(\extS{S}{l''}{\bot}{\frozenfalse}) = \\
          \extS{\extS{S}{l''}{\bot}{\frozenfalse}}{l}{\bot}{\frozenfalse}$
          and we know $S \neq \topS$ and the addition of new
          bindings $\storebinding{l}{\bot}{\frozenfalse}$ and
          $\storebinding{l''}{\bot}{\frozenfalse}$ cannot cause it
          to become $\topS$; and
        \item $U_S$ is freeze-safe with $\config{S}{e_{b_1}}
          \parstepsto
          \config{\extS{S}{l''}{\bot}{\frozenfalse}}{\pi(e_{b_2})}$,
          since $l'' \notin \dom{S}$, so there are no locations
          whose contents differ in status between $S$ and
          $\extS{S}{l''}{\bot}{\frozenfalse}$.
        \end{itemize}

        Therefore, by Lemma~\ref{lem:generalized-independence}
        (Generalized Independence), we have that

        $\config{U_S(S)}{e_{b_1}} \parstepsto
        \config{U_S(\extS{S}{l''}{\bot}{\frozenfalse})}{\pi(e_{b_2})}$.

        Hence $\config{\extS{S}{l}{\bot}{\frozenfalse}}{e_{b_1}}
        \parstepsto
        \config{\extS{\extS{S}{l''}{\bot}{\frozenfalse}}{l}{\bot}{\frozenfalse}}{\pi(e_{b_2})}$.

        By {\sc E-Eval-Ctxt} it follows that

        $\config{\extS{S}{l}{\bot}{\frozenfalse}}{\evalctxt{E'_b}{e_{b_1}}}
        \parstepsto
        \config{\extS{\extS{S}{l''}{\bot}{\frozenfalse}}{l}{\bot}{\frozenfalse}}{\evalctxt{E'_b}{\pi(e_{b_2})}}$,

        which, since $\extS{S}{l}{\bot}{\frozenfalse} = S_a$, is what
        we were required to show.

        For the second, observe that since $S_b =
        \extS{S}{l}{\bot}{\frozenfalse}$, we have that $\pi(S_b) =
        \extS{S}{l''}{\bot}{\frozenfalse}$.

        Also, since $l$ does not occur in $e_{a_1}$, we have that
        $\pi(\evalctxt{E'_a}{e_{a_1}}) =
        \evalctxt{(\pi(E'_a))}{e_{a_1}}$.

        Hence we have to show that

        $\config{\extS{S}{l''}{\bot}{\frozenfalse}}{\evalctxt{(\pi(E'_a))}{e_{a_1}}}
        \ctxstepsto \\
        \config{\extS{\extS{S}{l''}{\bot}{\frozenfalse}}{l}{\bot}{\frozenfalse}}{\evalctxt{E'_b}{\pi(e_{b_2})}}$.

        Let $U_S$ be the store update operation that acts as the
        identity on the contents of all existing locations, and adds
        the binding $\storebinding{l''}{\bot}{\frozenfalse}$ if no
        binding for $l''$ exists.

        Note that:
        \begin{itemize}
        \item $U_S$ is non-conflicting with $\config{S}{e_{a_1}}
          \parstepsto \config{S_a}{e_{a_2}}$, since the only
          location allocated in the transition is $l$;
        \item $U_S(S_a) \neq \topS$, since $U_S(S_a) =
          \extS{\extS{S}{l''}{\bot}{\frozenfalse}}{l}{\bot}{\frozenfalse}$
          and we know $S \neq \topS$ and the addition of new
          bindings $\storebinding{l}{\bot}{\frozenfalse}$ and
          $\storebinding{l''}{\bot}{\frozenfalse}$ cannot cause it
          to become $\topS$; and
        \item $U_S$ is freeze-safe with $\config{S}{e_{a_1}}
          \parstepsto \config{S_a}{e_{a_2}}$, since $S_a =
          \extS{S}{l}{\bot}{\frozenfalse}$ and $l \notin \dom{S}$,
          so there are no locations whose contents differ in status
          between $S$ and $S_a$.
        \end{itemize}

        Therefore, by Lemma~\ref{lem:generalized-independence}
        (Generalized Independence), we have that

        $\config{U_S(S)}{e_{a_1}} \parstepsto
        \config{U_S(S_a)}{e_{a_2}}$.

        Hence $\config{\extS{S}{l''}{\bot}{\frozenfalse}}{e_{a_1}}
        \parstepsto
        \config{\extS{\extS{S}{l''}{\bot}{\frozenfalse}}{l}{\bot}{\frozenfalse}}{e_{a_2}}$.

        By {\sc E-Eval-Ctxt} it follows that
        
        $\config{\extS{S}{l''}{\bot}{\frozenfalse}}{\evalctxt{(\pi(E'_a))}{e_{a_1}}}
        \ctxstepsto \\
        \config{\extS{\extS{S}{l''}{\bot}{\frozenfalse}}{l}{\bot}{\frozenfalse}}{\evalctxt{(\pi(E'_a))}{e_{a_2}}}$,

        which completes the case since $\evalctxt{E'_b}{\pi(e_{b_2})}
        = \evalctxt{(\pi(E'_a))}{e_{a_2}}$.

        \lk{This assumes that you believe that
          $\evalctxt{E'_b}{\pi(e_{b_2})} =
          \evalctxt{(\pi(E'_a))}{e_{a_2}}$.}

      \end{itemize}

    \item \label{slqc-new-put}Case {\sc E-Put}: We have $S_a =
      \extS{S}{l}{\bot}{\frozenfalse}$ and $S_b =
      \extSRaw{S}{l'}{u_{p_i}(p_1)}$, where $l \neq l'$ (since $l
      \notin \dom{S}$, but $l' \in \dom{S}$).

      We have to show that:
      \begin{itemize}
      \item $\config{S_a}{\evalctxt{E'_b}{e_{b_1}}} \ctxstepsto
        \config{\extS{S_b}{l}{\bot}{\frozenfalse}}{\evalctxt{E'_b}{e_{b_2}}}$,
        and
      \item $\config{S_b}{\evalctxt{E'_a}{e_{a_1}}} \ctxstepsto
        \config{\extS{S_b}{l}{\bot}{\frozenfalse}}{\evalctxt{E'_a}{e_{a_2}}}$.
      \end{itemize}

      For the first of these, consider that $S_a =
      \extS{S}{l}{\bot}{\frozenfalse} = U_S(S)$, where $U_S$ is the
      store update operation that acts as the identity on the contents
      of all existing locations, and adds the binding
      $\storebinding{l}{\bot}{\frozenfalse}$ if no binding for $l$
      exists.

      Note that:
      \begin{itemize}
      \item $U_S$ is non-conflicting with $\config{S}{e_{b_1}}
        \parstepsto \config{S_b}{e_{b_2}}$, since no locations are
        allocated in the transition;
      \item $U_S(S_b) \neq \topS$, since $U_S(S_b) =
        \extS{S_b}{l}{\bot}{\frozenfalse}$, and we know $S_b \neq
        \topS$ and the addition of a new binding
        $\storebinding{l}{\bot}{\frozenfalse}$ cannot cause it to
        become $\topS$; and
      \item $U_S$ is freeze-safe with $\config{S}{e_{b_1}} \parstepsto
        \config{S_b}{e_{b_2}}$, since $S_b =
        \extSRaw{S}{l'}{u_{p_i}(p_1)}$ and $u_{p_i}$ does not alter
        the status of $p_1$.

        (By Definition~\ref{def:set-of-state-update-operations},
        $u_{p_i}$ can only change the status bit of a location if its
        contents are $\state{d}{\frozentrue}$ and $u_i(d) \neq d$, in
        which case $u_{p_i}$ changes the contents of the location to
        $\state{\top}{\frozenfalse}$; however, that cannot be the case
        here since then $u_{p_i}(p_1)$ would be $\topp$, contradicting
        the premise of {\sc E-Put}.)
      \end{itemize}

      Therefore, by Lemma~\ref{lem:generalized-independence}
      (Generalized Independence), we have that

      $\config{U_S(S)}{e_{b_1}} \parstepsto
      \config{U_S(S_b)}{e_{b_2}}$.

      Hence $\config{\extS{S}{l}{\bot}{\frozenfalse}}{e_{b_1}}
      \parstepsto
      \config{\extS{S_b}{l}{\bot}{\frozenfalse}}{e_{b_2}}$.

      By {\sc E-Eval-Ctxt}, it follows that

      $\config{\extS{S}{l}{\bot}{\frozenfalse}}{\evalctxt{E'_b}{e_{b_1}}}
      \ctxstepsto
      \config{\extS{S_b}{l}{\bot}{\frozenfalse}}{\evalctxt{E'_b}{e_{b_2}}}$,
      
      which, since $S_a = \extS{S}{l}{\bot}{\frozenfalse}$, is what we
      were required to show.

      For the second, let $U_S$ be the store update operation that
      applies $u_{p_i}$ to the contents of $l'$ if it exists, and adds
      a binding $\storebindingRaw{l'}{u_{p_i}(p_1)}$ if no binding for
      $l'$ exists.

      Consider that $S_b = U_S(S)$, and
      $\extS{S_b}{l}{\bot}{\frozenfalse} =
      \extSRaw{S_a}{l'}{u_{p_i}(p_1)} = U_S(S_a)$.

      Note that:
      \begin{itemize}
      \item $U_S$ is non-conflicting with $\config{S}{e_{a_1}}
        \parstepsto \config{S_a}{e_{a_2}}$, since the only location
        allocated in the transition is $l$;
      \item $U_S(S_a) \neq \topS$, since $U_S(S_a) =
        \extSRaw{\extS{S}{l}{\bot}{\frozenfalse}}{l'}{u_{p_i}(p_1)}$
        and we know $S \neq \topS$ and the addition of a new binding
        $\storebinding{l}{\bot}{\frozenfalse}$ and updating the
        contents of location $l'$ to $u_{p_i}(p_1)$ in $S$ cannot
        cause it to become $\topS$ (since if $u_{p_i}(p_1) = \topp$,
        $\config{S}{e_{b_1}}$ would not have been able to step by {\sc
          E-Put}); and
      \item $U_S$ is freeze-safe with $\config{S}{e_{a_1}} \parstepsto
        \config{S_a}{e_{a_2}}$, since $S_a =
        \extS{S}{l}{\bot}{\frozenfalse}$ and $l \notin \dom{S}$, so
        there are no locations whose contents differ in status between
        $S$ and $S_a$.
      \end{itemize}

      Therefore, by Lemma~\ref{lem:generalized-independence}
      (Generalized Independence), we have that

      $\config{U_S(S)}{e_{a_1}} \parstepsto
      \config{U_S(S_a)}{e_{a_2}}$.

      Hence $\config{S_b}{e_{a_1}}
      \parstepsto
      \config{\extS{S_b}{l}{\bot}{\frozenfalse}}{e_{a_2}}$.

      By {\sc E-Eval-Ctxt}, it follows that
      
      $\config{S_b}{\evalctxt{E'_a}{e_{a_1}}} \ctxstepsto
      \config{\extS{S_b}{l}{\bot}{\frozenfalse}}{\evalctxt{E'_a}{e_{a_2}}}$,
      
      as we were required to show.

    \item \label{slqc-new-put-err}Case {\sc E-Put-Err}: We have $S_a =
      \extS{S}{l}{\bot}{\frozenfalse}$ and $\config{S_b}{e_{b_2}} =
      \error$, and so we choose $\conf_c = \error$, $i = 1$, $j = 0$,
      and $\pi = \id$.

      We have to show that:
      \begin{itemize}
      \item $\config{S_a}{\evalctxt{E'_b}{e_{b_1}}} \ctxstepsto
        \error$, and
      \item $\config{S_b}{\evalctxt{E'_a}{e_{a_1}}} = \error$.
      \end{itemize}

      The second of these is immediately true because since
      $\config{S_b}{e_{b_2}} = \error$, $S_b = \topS$, and so
      $\config{S_b}{\evalctxt{E'_a}{e_{a_1}}}$ is equal to $\error$ as
      well.

      For the first, observe that since $\config{S}{e_{a_1}}
      \parstepsto \config{S_a}{e_{a_2}}$, we have by
      Lemma~\ref{lem:monotonicity} (Monotonicity) that
      $\leqstore{S}{S_a}$.

      Therefore, since $\config{S}{e_{b_1}} \parstepsto \error$,

      we have by Lemma~\ref{lem:error-preservation} (Error
      Preservation) that $\config{S_a}{e_{b_1}} \parstepsto \error$.

      Since $\error$ is equal to $\config{\topS}{e}$ for all
      expressions $e$, $\config{S_a}{e_{b_1}} \parstepsto
      \config{\topS}{e}$ for all $e$.

      Therefore, by {\sc E-Eval-Ctxt},
      $\config{S_a}{\evalctxt{E'_b}{e_{b_1}}} \ctxstepsto
      \config{\topS}{\evalctxt{E'_b}{e}}$ for all $e$.

      Since $\config{\topS}{\evalctxt{E'_b}{e}}$ is equal to $\error$,
      we have that $\config{S_a}{\evalctxt{E'_b}{e_{b_1}}} \ctxstepsto
      \error$, as we were required to show.

    \item \label{slqc-new-get}Case {\sc E-Get}: Similar to
      case~\ref{slqc-new-beta}, since $S_a =
      \extS{S}{l}{\bot}{\frozenfalse}$ and $S_b = S$.
    \item \label{slqc-new-freeze-init}Case {\sc E-Freeze-Init}:
      Similar to case~\ref{slqc-new-beta}, since $S_a =
      \extS{S}{l}{\bot}{\frozenfalse}$ and $S_b = S$.
    \item \label{slqc-new-spawn-handler}Case {\sc E-Spawn-Handler}:
      Similar to case~\ref{slqc-new-beta}, since $S_a =
      \extS{S}{l}{\bot}{\frozenfalse}$ and $S_b = S$.
    \item \label{slqc-new-freeze-final}Case {\sc E-Freeze-Final}: We
      have $S_a = \extS{S}{l}{\bot}{\frozenfalse}$ and $S_b =
      \extS{S}{l'}{d_1}{\frozentrue}$, where $l \neq l'$ (since $l
      \notin \dom{S}$, but $l' \in \dom{S}$).

      Choose $S' =
      \extS{\extS{S}{l}{\bot}{\frozenfalse}}{l'}{d_1}{\frozentrue}$,
      $i = i$, $j = 1$, and $\pi = \id$.

      We have to show that:
      \begin{itemize}
      \item
        $\config{\extS{S}{l}{\bot}{\frozenfalse}}{\evalctxt{E'_b}{e_{b_1}}}
        \ctxstepsto
        \config{\extS{\extS{S}{l}{\bot}{\frozenfalse}}{l'}{d_1}{\frozentrue}}{\evalctxt{E'_b}{e_{b_2}}}$,
        and
      \item
        $\config{\extS{S}{l'}{d_1}{\frozentrue}}{\evalctxt{E'_a}{e_{a_1}}}
        \ctxstepsto
        \config{\extS{\extS{S}{l}{\bot}{\frozenfalse}}{l'}{d_1}{\frozentrue}}{\evalctxt{E'_a}{e_{a_2}}}$.
      \end{itemize}

      For the first of these, consider that
      $\extS{S}{l}{\bot}{\frozenfalse} = U_S(S)$, where $U_S$ is the
      store update operation that acts as the identity on the contents
      of all existing locations, and adds the binding
      $\storebinding{l}{\bot}{\frozenfalse}$ if no binding for $l$
      exists.

      Note that:
      \begin{itemize}
      \item $U_S$ is non-conflicting with $\config{S}{e_{b_1}}
        \parstepsto \config{S_b}{e_{b_2}}$, since no locations are
        allocated in the transition;
      \item $U_S(S_b) \neq \topS$, since $U_S(S_b) =
        \extS{S_b}{l}{\bot}{\frozenfalse}$, and we know $S_b \neq
        \topS$ and the addition of a new binding
        $\storebinding{l}{\bot}{\frozenfalse}$ cannot cause it to
        become $\topS$; and
      \item $U_S$ is freeze-safe with $\config{S}{e_{b_1}}
        \parstepsto \config{S_b}{e_{b_2}}$, since $S_b =
        \extS{S}{l'}{d_1}{\frozentrue}$ and so the only location
        that can change in status between $S$ and $S_b$ is $l'$, and
        $U_S$ acts as the identity on $l'$.
      \end{itemize}
      Therefore, by Lemma~\ref{lem:generalized-independence}
      (Generalized Independence), we have that

      $\config{U_S(S)}{e_{b_1}} \parstepsto
      \config{U_S(S_b)}{e_{b_2}}$.

      Hence $\config{\extS{S}{l}{\bot}{\frozenfalse}}{e_{b_1}}
      \parstepsto
      \config{\extS{\extS{S}{l}{\bot}{\frozenfalse}}{l'}{d_1}{\frozentrue}}{e_{b_2}}$.

      By {\sc E-Eval-Ctxt}, it follows that

      $\config{\extS{S}{l}{\bot}{\frozenfalse}}{\evalctxt{E'_b}{e_{b_1}}}
      \ctxstepsto
      \config{\extS{\extS{S}{l}{\bot}{\frozenfalse}}{l'}{d_1}{\frozentrue}}{\evalctxt{E'_b}{e_{b_2}}}$,

      as we were required to show.

      For the second, consider that $\extS{S}{l'}{d_1}{\frozentrue} =
      U_S(S)$, where $U_S$ is the store update operation that freezes
      the contents of $l'$ and acts as the identity on the contents of
      all other locations.

      Note that:
      \begin{itemize}
      \item $U_S$ is non-conflicting with $\config{S}{e_{a_1}}
        \parstepsto \config{S_a}{e_{a_2}}$, since the only location
        allocated in the transition is $l$, and $l \neq l'$;
      \item $U_S(S_a) \neq \topS$, since $U_S(S_a) =
        \extS{S_a}{l'}{d_1}{\frozentrue} =
        \extS{S_b}{l}{\bot}{\frozenfalse}$, and we know $S_b \neq
        \topS$ and the addition of a new binding
        $\storebinding{l}{\bot}{\frozenfalse}$ cannot cause it to
        become $\topS$; and
      \item $U_S$ is freeze-safe with $\config{S}{e_{a_1}}
        \parstepsto \config{S_a}{e_{a_2}}$, since $S_a =
        \extS{S}{l}{\bot}{\frozenfalse}$ and $l \notin \dom{S}$, so
        there are no locations whose contents differ in status
        between $S$ and $S_a$.
      \end{itemize}

      Therefore, by Lemma~\ref{lem:generalized-independence}
      (Generalized Independence), we have that

      $\config{U_S(S)}{e_{a_1}} \parstepsto
      \config{U_S(S_a)}{e_{a_2}}$.

      Hence $\config{\extS{S}{l'}{d_1}{\frozentrue}}{e_{a_1}}
      \parstepsto
      \config{\extS{\extS{S}{l}{\bot}{\frozenfalse}}{l'}{d_1}{\frozentrue}}{e_{a_2}}$.

      By {\sc E-Eval-Ctxt} it follows that

      $\config{\extS{S}{l'}{d_1}{\frozentrue}}{\evalctxt{E'_a}{e_{a_1}}}
      \ctxstepsto
      \config{\extS{\extS{S}{l}{\bot}{\frozenfalse}}{l'}{d_1}{\frozentrue}}{\evalctxt{E'_a}{e_{a_2}}}$,

      as we were required to show.

    \item \label{slqc-new-freeze-simple}Case {\sc E-Freeze-Simple}:
      Similar to case~\ref{slqc-new-freeze-final}, since $S_a =
      \extS{S}{l}{\bot}{\frozenfalse}$ and $S_b =
      \extS{S}{l'}{d_1}{\frozentrue}$, where $l \neq l'$ (since $l
      \notin \dom{S}$, but $l' \in \dom{S}$).

    \end{enumerate}
  \item Case {\sc E-Put}: We have $S_a =
    \extSRaw{S}{l}{u_{p_i}(p_1)}$.

    We proceed by case analysis on the rule by which
    $\config{S}{e_{b_1}}$ steps to $\config{S_b}{e_{b_2}}$.

    Since the only way an $\error$ configuration can arise is by the
    {\sc E-Put-Err} rule, we can assume in all other cases that
    $\conf_b \neq \error$.
    \begin{enumerate}
    \item \label{slqc-put-beta}Case {\sc E-Beta}: By symmetry with case~\ref{slqc-beta-put}.
    \item \label{slqc-put-new}Case {\sc E-New}: By symmetry with case~\ref{slqc-new-put}.
    \item \label{slqc-put-put}Case {\sc E-Put}: We have $S_a =
      \extSRaw{S}{l}{u_{p_i}(p_1)}$ and $S_b =
      \extSRaw{S}{l'}{u_{p_j}(p'_1)}$, where $p'_1 = S(l')$.

      Now consider whether $l = l'$:
      \begin{itemize}
      \item If $l \neq l'$:

        Choose $S' =
        \extSRaw{\extSRaw{S}{l'}{u_{p_j}(p'_1)}}{l}{u_{p_i}(p_1)}$,
        $i = 1$, $j = 1$, and $\pi = \id$.

        We have to show that:
        \begin{itemize}
        \item
          $\config{\extSRaw{S}{l}{u_{p_i}(p_1)}}{\evalctxt{E'_b}{e_{b_1}}}
          \ctxstepsto
          \config{\extSRaw{\extSRaw{S}{l'}{u_{p_j}(p'_1)}}{l}{u_{p_i}(p_1)}}{\evalctxt{E'_b}{e_{b_2}}}$,
          and
        \item
          $\config{\extSRaw{S}{l'}{u_{p_j}(p'_1)}}{\evalctxt{E'_a}{e_{a_1}}}
          \ctxstepsto
          \config{\extSRaw{\extSRaw{S}{l'}{u_{p_j}(p'_1)}}{l}{u_{p_i}(p_1)}}{\evalctxt{E'_a}{e_{a_2}}}$.
        \end{itemize}

        For the first of these, consider that
        $\extSRaw{S}{l}{u_{p_i}(p_1)} = U_S(S)$, where $U_S$ is the
        store update operation that applies $u_{p_i}$ to the
        contents of $l$ if it exists, and adds a binding
        $\storebindingRaw{l}{u_{p_i}(p_1)}$ if no binding for $l$
        exists.

        Note that:
        \begin{itemize}
        \item $U_S$ is non-conflicting with $\config{S}{e_{b_1}}
          \parstepsto
          \config{\extSRaw{S}{l'}{u_{p_j}(p'_1)}}{e_{b_2}}$, since
          no locations are allocated in the transition;
        \item $U_S(\extSRaw{S}{l'}{u_{p_j}(p'_1)}) \neq \topS$,
          since $U_S(\extSRaw{S}{l'}{u_{p_j}(p'_1)}) =
          \extSRaw{\extSRaw{S}{l'}{u_{p_j}(p'_1)}}{l}{u_{p_i}(p_1)}$
          and we know $S \neq \topS$ and updating the contents of
          location $l$ to $u_{p_i}(p_1)$ and the contents of
          location $l'$ to $u_{p_j}(p'_1)$ in $S$ cannot cause it to
          become $\topS$ (because if so, then we would have $S_a =
          \topS$ or $S_b = \topS$, which we know are not the case);
          and
        \item $U_S$ is freeze-safe with $\config{S}{e_{b_1}}
          \parstepsto
          \config{\extSRaw{S}{l'}{u_{p_j}(p'_1)}}{e_{b_2}}$, since
          $u_{p_j}$ does not alter the status of $p'_1$.

          (By Definition~\ref{def:set-of-state-update-operations},
          $u_{p_j}$ can only change the status bit of a location if
          its contents are $\state{d}{\frozentrue}$ and $u_j(d) \neq
          d$, in which case $u_{p_j}$ changes the contents of the
          location to $\state{\top}{\frozenfalse}$; however, that
          cannot be the case here since then $u_{p_j}(p'_1)$ would be
          $\topp$, contradicting the premise of {\sc E-Put}.)
        \end{itemize}

        Therefore, by Lemma~\ref{lem:generalized-independence}
        (Generalized Independence), we have that

        $\config{U_S(S)}{e_{b_1}} \parstepsto
        \config{U_S(\extSRaw{S}{l'}{u_{p_j}(p'_1)})}{e_{b_2}}$.

        Hence $\config{\extSRaw{S}{l}{u_{p_i}(p_1)}}{e_{b_1}}
        \parstepsto
        \config{\extSRaw{\extSRaw{S}{l'}{u_{p_j}(p'_1)}}{l}{u_{p_i}(p_1)}}{e_{b_2}}$.

        By {\sc E-Eval-Ctxt}, it follows that

        $\config{\extSRaw{S}{l}{u_{p_i}(p_1)}}{\evalctxt{E'_b}{e_{b_1}}}
        \ctxstepsto
        \config{\extSRaw{\extSRaw{S}{l'}{u_{p_j}(p'_1)}}{l}{u_{p_i}(p_1)}}{\evalctxt{E'_b}{e_{b_2}}}$,

        as we were required to show.

        The argument for the second is symmetrical.

      \item If $l = l'$:
        Note that since $l = l'$, $p_1 = p'_1$ as well.

        Consider whether $u_{p_i}(u_{p_j}(p_1)) = \topp$:
        \begin{itemize}
        \item If $u_{p_i}(u_{p_j}(p_1)) = \topp$:

          Choose $\conf_c = \error$, $i = 1$, $j = 1$, and $\pi =
          \id$.

          We have to show that:

          \begin{itemize}
          \item
            $\config{\extSRaw{S}{l}{u_{p_i}(p_1)}}{\evalctxt{E'_b}{e_{b_1}}}
            \ctxstepsto \error$, and
          \item
            $\config{\extSRaw{S}{l}{u_{p_j}(p_1)}}{\evalctxt{E'_a}{e_{a_1}}}
            \ctxstepsto \error$.
          \end{itemize}

          For the first of these, consider that
          $\extSRaw{S}{l}{u_{p_i}(p_1)} = U_S(S)$, where $U_S$ is the
          store update operation that applies $u_{p_i}$ to the
          contents of $l$ if it exists.

          Note that:
          \begin{itemize}
          \item $U_S$ is non-conflicting with $\config{S}{e_{b_1}}
            \parstepsto
            \config{\extSRaw{S}{l}{u_{p_j}(p_1)}}{e_{b_2}}$, since
            no locations are allocated in the transition;
          \item $U_S(\extSRaw{S}{l}{u_{p_j}(p_1)}) = \topS$, since
            $U_S(\extSRaw{S}{l}{u_{p_j}(p_1)}) =
            \extSRaw{S}{l}{u_{p_i}(u_{p_j}(p_1))}$ and we know
            $u_{p_i}(u_{p_j}(p_1)) = \topp$ in this case;
          \item $U_S$ is freeze-safe with $\config{S}{e_{b_1}}
            \parstepsto
            \config{\extSRaw{S}{l}{u_{p_j}(p_1)}}{e_{b_2}}$, since
            $u_{p_j}$ does not alter the status of $p_1$.

            (By Definition~\ref{def:set-of-state-update-operations},
            $u_{p_j}$ can only change the status bit of a location if
            its contents are $\state{d}{\frozentrue}$ and $u_j(d) \neq
            d$, in which case $u_{p_j}$ changes the contents of the
            location to $\state{\top}{\frozenfalse}$; however, that
            cannot be the case here since then $u_{p_j}(p_1)$ would be
            $\topp$, contradicting the premise of {\sc E-Put}.)
          \end{itemize}

          Therefore, by Lemma~\ref{lem:generalized-clash}
          (Generalized Clash), we have that there exists $i' \leq 1$
          such that $\config{U_S(S)}{e_{b_1}} \parstepsto^{i'}
          \error$.

          Hence $\config{\extSRaw{S}{l}{u_{p_i}(p_1)}}{e_{b_1}}
          \parstepsto^{i'} \error$.

          If $i' = 0$, we would have
          $\config{\extSRaw{S}{l}{u_{p_i}(p_1)}}{e_{b_1}} =
          \config{S_a}{e_{b_1}} = \error$.

          So we would have $S_a = \topS$ by the definition of
          $\error$, but then we would have $\conf_a = \error$, a
          contradiction.

          Therefore $i' = 1$, and so we have
          $\config{\extSRaw{S}{l}{u_{p_i}(p_1)}}{e_{b_1}} \parstepsto
          \error$.

          Since $\error = \config{\topS}{e}$ for all $e$, we have
          $\config{\extSRaw{S}{l}{u_{p_i}(p_1)}}{e_{b_1}}
          \parstepsto \config{\topS}{e}$ for all $e$.

          So, by {\sc E-Eval-Ctxt}, we have that
          $\config{\extSRaw{S}{l}{u_{p_i}(p_1)}}{\evalctxt{E'_b}{e_{b_1}}}
          \parstepsto \config{\topS}{\evalctxt{E'_b}{e}}$ for all $e$.

          Hence
          $\config{\extSRaw{S}{l}{u_{p_i}(p_1)}}{\evalctxt{E'_b}{e_{b_1}}}
          \parstepsto \error$.

          The argument for the second is symmetrical.

        \item If $u_{p_i}(u_{p_j}(p_1)) \neq \topp$:

          Choose $S' = \extSRaw{S}{l}{u_{p_i}(u_{p_j}(p_1))}$, $i =
          1$, $j = 1$, and $\pi = \id$.

          We have to show that:
          \begin{itemize}
          \item
            $\config{\extSRaw{S}{l}{u_{p_i}(p_1)}}{\evalctxt{E'_b}{e_{b_1}}}
            \ctxstepsto
            \config{\extSRaw{S}{l}{u_{p_i}(u_{p_j}(p_1))}}{\evalctxt{E'_b}{e_{b_2}}}$,
            and
          \item
            $\config{\extSRaw{S}{l}{u_{p_j}(p_1)}}{\evalctxt{E'_a}{e_{a_1}}}
            \ctxstepsto
            \config{\extSRaw{S}{l}{u_{p_i}(u_{p_j}(p_1))}}{\evalctxt{E'_a}{e_{a_2}}}$.
          \end{itemize}

          For the first of these, consider that
          $\extSRaw{S}{l}{u_{p_i}(p_1)} = U_S(S)$, where $U_S$ is the
          store update operation that applies $u_{p_i}$ to the
          contents of $l$ if it exists.

          Note that:
          \begin{itemize}
          \item $U_S$ is non-conflicting with $\config{S}{e_{b_1}}
            \parstepsto
            \config{\extSRaw{S}{l}{u_{p_j}(p_1)}}{e_{b_2}}$, since no
            locations are allocated in the transition;
          \item $U_S(\extSRaw{S}{l}{u_{p_j}(p_1)}) \neq \topS$, since
            $U_S(\extSRaw{S}{l}{u_{p_j}(p_1)}) =
            \extSRaw{S}{l}{u_{p_i}(u_{p_j}(p_1))}$ and we know $S \neq
            \topS$ and $u_{p_i}(u_{p_j}(p_1)) \neq \topp$ in this
            case;
          \item $U_S$ is freeze-safe with $\config{S}{e_{b_1}}
            \parstepsto
            \config{\extSRaw{S}{l}{u_{p_j}(p_1)}}{e_{b_2}}$, since
            $u_{p_j}$ does not alter the status of $p_1$.

            (By Definition~\ref{def:set-of-state-update-operations},
            $u_{p_j}$ can only change the status bit of a location if
            its contents are $\state{d}{\frozentrue}$ and $u_j(d) \neq
            d$, in which case $u_{p_j}$ changes the contents of the
            location to $\state{\top}{\frozenfalse}$; however, that
            cannot be the case here since then $u_{p_j}(p_1)$ would be
            $\topp$, contradicting the premise of {\sc E-Put}.)
          \end{itemize}

          Therefore, by Lemma~\ref{lem:generalized-independence}
          (Generalized Independence), we have that

          $\config{U_S(S)}{e_{b_1}} \parstepsto
          \config{U_S(\extSRaw{S}{l}{u_{p_j}(p_1)})}{e_{b_2}}$.

          Hence $\config{\extSRaw{S}{l}{u_{p_i}(p_1)}}{e_{b_1}}
          \parstepsto
          \config{\extSRaw{S}{l}{u_{p_i}(u_{p_j}(p_1))}}{e_{b_2}}$.

          By {\sc E-Eval-Ctxt}, it follows that

          $\config{\extSRaw{S}{l}{u_{p_i}(p_1)}}{\evalctxt{E'_b}{e_{b_1}}}
          \ctxstepsto
          \config{\extSRaw{S}{l}{u_{p_i}(u_{p_j}(p_1))}}{\evalctxt{E'_b}{e_{b_2}}}$,

          as we were required to show.

          The argument for the second is symmetrical.

        \end{itemize}

      \end{itemize}

    \item \label{slqc-put-put-err}Case {\sc E-Put-Err}: We have $S_a =
      \extSRaw{S}{l}{u_{p_i}(p_1)}$ and $\config{S_b}{e_{b_2}} =
      \error$, and so we choose $\conf_c = \error$, $i = 1$, $j = 0$,
      and $\pi = \id$.

      We have to show that:
      \begin{itemize}
      \item $\config{S_a}{\evalctxt{E'_b}{e_{b_1}}} \ctxstepsto
        \error$, and
      \item $\config{S_b}{\evalctxt{E'_a}{e_{a_1}}} = \error$.
      \end{itemize}

      The second of these is immediately true because since
      $\config{S_b}{e_{b_2}} = \error$, $S_b = \topS$, and so
      $\config{S_b}{\evalctxt{E'_a}{e_{a_1}}}$ is equal to $\error$ as
      well.

      For the first, observe that since $\config{S}{e_{a_1}}
      \parstepsto \config{S_a}{e_{a_2}}$, we have by
      Lemma~\ref{lem:monotonicity} (Monotonicity) that
      $\leqstore{S}{S_a}$.

      Therefore, since $\config{S}{e_{b_1}} \parstepsto \error$,

      we have by Lemma~\ref{lem:error-preservation} (Error
      Preservation) that $\config{S_a}{e_{b_1}} \parstepsto \error$.
      
      Since $\error$ is equal to $\config{\topS}{e}$ for all
      expressions $e$, $\config{S_a}{e_{b_1}} \parstepsto
      \config{\topS}{e}$ for all $e$.

      Therefore, by {\sc E-Eval-Ctxt},
      $\config{S_a}{\evalctxt{E'_b}{e_{b_1}}} \ctxstepsto
      \config{\topS}{\evalctxt{E'_b}{e}}$ for all $e$.

      Since $\config{\topS}{\evalctxt{E'_b}{e}}$ is equal to $\error$,
      we have that $\config{S_a}{\evalctxt{E'_b}{e_{b_1}}} \ctxstepsto
      \error$, as we were required to show.

    \item \label{slqc-put-get}Case {\sc E-Get}: Similar to
      case~\ref{slqc-put-beta}, since $S_a =
      \extSRaw{S}{l}{u_{p_i}(p_1)}$ and $S_b = S$.
    \item \label{slqc-put-freeze-init}Case {\sc E-Freeze-Init}:
      Similar to case~\ref{slqc-put-beta}, since $S_a =
      \extSRaw{S}{l}{u_{p_i}(p_1)}$ and $S_b = S$.
    \item \label{slqc-put-spawn-handler}Case {\sc E-Spawn-Handler}:
      Similar to case~\ref{slqc-put-beta}, since $S_a =
      \extSRaw{S}{l}{u_{p_i}(p_1)}$ and $S_b = S$.
    \item \label{slqc-put-freeze-final}Case {\sc E-Freeze-Final}: We
      have $S_a = \extSRaw{S}{l}{u_{p_i}(p_1)}$ and $S_b =
      \extS{S}{l'}{d_1}{\frozentrue}$.

      Now consider whether $l = l'$:
      \begin{itemize}
      \item If $l \neq l'$:

        Choose $S' =
        \extS{\extSRaw{S}{l}{u_{p_i}(p_1)}}{l'}{d_1}{\frozentrue}$,
        $i = 1$, $j = 1$, and $\pi = \id$.

        We have to show that:
        \begin{itemize}
        \item
          $\config{\extSRaw{S}{l}{u_{p_i}(p_1)}}{\evalctxt{E'_b}{e_{b_1}}}
          \ctxstepsto
          \config{\extS{\extSRaw{S}{l}{u_{p_i}(p_1)}}{l'}{d_1}{\frozentrue}}{\evalctxt{E'_b}{e_{b_2}}}$,
          and
        \item
          $\config{\extS{S}{l'}{d_1}{\frozentrue}}{\evalctxt{E'_a}{e_{a_1}}}
          \ctxstepsto
          \config{\extS{\extSRaw{S}{l}{u_{p_i}(p_1)}}{l'}{d_1}{\frozentrue}}{\evalctxt{E'_a}{e_{a_2}}}$.
        \end{itemize}

        For the first of these, consider that
        $\extSRaw{S}{l}{u_{p_i}(p_1)} = U_S(S)$, where $U_S$ is the
        store update operation that applies $u_{p_i}$ to the
        contents of $l$ if it exists, and adds a binding
        $\storebindingRaw{l}{u_{p_i}(p_1)}$ if no binding for $l$
        exists, and acts as the identity on all other locations.

        Note that:
        \begin{itemize}
        \item $U_S$ is non-conflicting with $\config{S}{e_{b_1}}
          \parstepsto
          \config{\extS{S}{l'}{d_1}{\frozentrue}}{e_{b_2}}$, since
          no locations are allocated in the transition;
        \item $U_S(\extS{S}{l'}{d_1}{\frozentrue}) \neq \topS$,

          since $U_S(\extS{S}{l'}{d_1}{\frozentrue}) =
          \extSRaw{\extS{S}{l'}{d_1}{\frozentrue}}{l}{u_{p_i}(p_1)}$
          and we know $S \neq \topS$ and updating the contents of
          location $l$ to $u_{p_i}(p_1)$ and freezing the contents
          of location $l'$ in $S$ cannot cause it to become $\topS$
          (because if so, then we would have $S_a = \topS$ or $S_b =
          \topS$, which we know are not the case); and
        \item $U_S$ is freeze-safe with $\config{S}{e_{b_1}}
          \parstepsto
          \config{\extS{S}{l'}{d_1}{\frozentrue}}{e_{b_2}}$, since
          the only location that can change in status between $S$
          and $\extS{S}{l'}{d_1}{\frozentrue}$ is $l'$, and $U_S$
          acts as the identity on $l'$.
        \end{itemize}
        Therefore, by Lemma~\ref{lem:generalized-independence}
        (Generalized Independence), we have that

        $\config{U_S(S)}{e_{b_1}} \parstepsto
        \config{U_S(\extS{S}{l'}{d_1}{\frozentrue})}{e_{b_2}}$.

        Hence $\config{\extSRaw{S}{l}{u_{p_i}(p_1)}}{e_{b_1}}
        \parstepsto
        \config{\extSRaw{\extS{S}{l'}{d_1}{\frozentrue}}{l}{u_{p_i}(p_1)}}{e_{b_2}}$.

        By {\sc E-Eval-Ctxt}, it follows that

        $\config{\extSRaw{S}{l}{u_{p_i}(p_1)}}{\evalctxt{E'_b}{e_{b_1}}}
        \ctxstepsto
        \config{\extSRaw{\extS{S}{l'}{d_1}{\frozentrue}}{l}{u_{p_i}(p_1)}}{\evalctxt{E'_b}{e_{b_2}}}$,
        
        as we were required to show.

        For the second, consider that
        $\extS{S}{l'}{d_1}{\frozentrue} = U_S(S)$, where $U_S$ is
        the store update operation that freezes the contents of $l'$
        and acts as the identity on the contents of all other
        locations.

        Note that:
        \begin{itemize}
        \item $U_S$ is non-conflicting with $\config{S}{e_{a_1}}
          \parstepsto
          \config{\extSRaw{S}{l}{u_{p_i}(p_1)}}{e_{a_2}}$, since no
          locations are allocated in the transition;
        \item $U_S(\extSRaw{S}{l}{u_{p_i}(p_1)}) \neq \topS$, since
          $U_S(\extSRaw{S}{l}{u_{p_i}(p_1)}) =
          \extS{\extSRaw{S}{l}{u_{p_i}(p_1)}}{l'}{d_1}{\frozentrue}$,
          and we know $S \neq \topS$ and updating the contents of
          location $l$ to $u_{p_i}(p_1)$ and freezing the contents
          of location $l$ in $S$ cannot cause it to become $\topS$
          (because if so, then we would have $S_a = \topS$ or $S_b =
          \topS$, which we know are not the case); and
        \item $U_S$ is freeze-safe with $\config{S}{e_{a_1}}
          \parstepsto
          \config{\extSRaw{S}{l}{u_{p_i}(p_1)}}{e_{a_2}}$, since
          $u_{p_i}$ does not alter the status of $p_1$.

          (By Definition~\ref{def:set-of-state-update-operations},
          $u_{p_i}$ can only change the status bit of a location if
          its contents are $\state{d}{\frozentrue}$ and $u_i(d) \neq
          d$, in which case $u_{p_i}$ changes the contents of the
          location to $\state{\top}{\frozenfalse}$; however, that
          cannot be the case here since then $u_{p_i}(p_1)$ would be
          $\topp$, and we would have $S_a = \topS$, a contradiction.)
        \end{itemize}
        Therefore, by Lemma~\ref{lem:generalized-independence}
        (Generalized Independence), we have that

        $\config{U_S(S)}{e_{a_1}} \parstepsto
        \config{U_S(\extSRaw{S}{l}{u_{p_i}(p_1)})}{e_{a_2}}$.

        Hence $\config{\extS{S}{l'}{d_1}{\frozentrue}}{e_{a_1}}
        \parstepsto
        \config{\extS{\extSRaw{S}{l}{u_{p_i}(p_1)}}{l'}{d_1}{\frozentrue}}{e_{a_2}}$.

        By {\sc E-Eval-Ctxt}, it follows that
        $\config{\extS{S}{l'}{d_1}{\frozentrue}}{\evalctxt{E'_a}{e_{a_1}}}
        \ctxstepsto
        \config{\extS{\extSRaw{S}{l}{u_{p_i}(p_1)}}{l'}{d_1}{\frozentrue}}{\evalctxt{E'_a}{e_{a_2}}}$,
        
        as we were required to show.

      \item If $l = l'$:

        We have two cases to consider:

        \begin{itemize}
        \item $u_{p_i}(\state{d_1}{\frozentrue}) = \topp$:

          \lk{This is the interesting case: the potential
            put-after-freeze case.  It's important to note that this
            case doesn't necessarily end in a put-after-freeze (and
            hence an error); all we're required to show is that it
            \emph{can} end that way.}

          Since $(\extSRaw{S}{l}{\state{d_1}{\frozentrue}})(l) =
          \state{d_1}{\frozentrue}$ and
          $u_{p_i}(\state{d_1}{\frozentrue}) = \topp$, by {\sc
            E-Put-Err} we have that
          $\config{\extSRaw{S}{l}{\state{d_1}{\frozentrue}}}{\putiexp{l}}
          \parstepsto \error$.

          Since $S_b = \extSRaw{S}{l}{\state{d_1}{\frozentrue}}$,
          we have that $\config{S_b}{\putiexp{l}} \parstepsto
          \error$.

          Since $\config{S}{e_{a_1}} \parstepsto
          \config{S_a}{e_{a_2}}$ by {\sc E-Put}, it must be the
          case that $e_{a_1} = \putiexp{l}$.

          Hence $\config{S_b}{e_{a_1}} \parstepsto \error$.

          Since $\error$ is equal to $\config{\topS}{e}$ for all
          expressions $e$, $\config{S_b}{e_{a_1}} \parstepsto
          \config{\topS}{e}$ for all $e$.

          Therefore, by {\sc E-Eval-Ctxt},
          $\config{S_b}{\evalctxt{E'_a}{e_{a_1}}} \ctxstepsto
          \config{\topS}{\evalctxt{E'_a}{e}}$ for all $e$.

          Since $\config{\topS}{\evalctxt{E'_a}{e}}$ is equal to
          $\error$, we have that
          $\config{S_b}{\evalctxt{E'_a}{e_{a_1}}} \ctxstepsto \error$.

          Since $\evalctxt{E'_a}{e_{a_1}} =
          \evalctxt{E_b}{e_{b_2}}$, we have that
          $\config{S_b}{\evalctxt{E_b}{e_{b_2}}} \ctxstepsto
          \error$.

          Since $\conf_b = \config{S_b}{\evalctxt{E_b}{e_{b_2}}}$,
          we therefore have that $\conf_b \ctxstepsto \error$, and
          the case is satisfied.

        \item $u_{p_i}(\state{d_1}{\frozentrue}) \neq \topp$:

          \lk{This is the case where there's a conflicting put and
            freeze, but the put is a no-op, so it doesn't matter.}

          In this case, by the definition of $U_p$
          (Definition~\ref{def:set-of-state-update-operations}),
          
          it must be the case that $u_{p_i}(\state{d_1}{\frozentrue})
          = \state{d_1}{\frozentrue}$.

          Choose $S' = \extS{S}{l}{d_1}{\frozentrue}$, $i = 1$, $j
          = 1$, and $\pi = \id$.

          We have to show that:
          \begin{itemize}
          \item
            $\config{\extSRaw{S}{l}{u_{p_i}(p_1)}}{\evalctxt{E'_b}{e_{b_1}}}
            \ctxstepsto
            \config{\extS{S}{l}{d_1}{\frozentrue}}{\evalctxt{E'_b}{e_{b_2}}}$,
            and
          \item
            $\config{\extS{S}{l}{d_1}{\frozentrue}}{\evalctxt{E'_a}{e_{a_1}}}
            \ctxstepsto
            \config{\extS{S}{l}{d_1}{\frozentrue}}{\evalctxt{E'_a}{e_{a_2}}}$.
          \end{itemize}

          For the first of these, consider that
          $\extSRaw{S}{l}{u_{p_i}(p_1)} = U_S(S)$, where $U_S$ is
          the store update operation that applies $u_{p_i}$ to the
          contents of $l$ if it exists, and adds a binding
          $\storebindingRaw{l}{u_{p_i}(p_1)}$ if no binding for
          $l$ exists, and acts as the identity on all other
          locations.

          Note that:
          \begin{itemize}
          \item $U_S$ is non-conflicting with $\config{S}{e_{b_1}}
            \parstepsto
            \config{\extS{S}{l}{d_1}{\frozentrue}}{e_{b_2}}$,
            since no locations are allocated in the
            transition;
          \item $U_S(\extS{S}{l}{d_1}{\frozentrue}) \neq \topS$,
            
            since $U_S(\extS{S}{l}{d_1}{\frozentrue}) =
            \extSRaw{S}{l}{u_{p_i}(\state{d_1}{\frozentrue})}$ and
            we know $S \neq \topS$ and
            $u_{p_i}(\state{d_1}{\frozentrue}) \neq \topp$; and
          \item $U_S$ is freeze-safe with $\config{S}{e_{b_1}}
            \parstepsto
            \config{\extS{S}{l}{d_1}{\frozentrue}}{e_{b_2}}$, since
            the only location that can change in status between $S$
            and $\extS{S}{l}{d_1}{\frozentrue}$ is $l$, and $U_S$
            acts as the identity on $l$.
          \end{itemize}
          Therefore, by Lemma~\ref{lem:generalized-independence}
          (Generalized Independence), we have that

          $\config{U_S(S)}{e_{b_1}} \parstepsto
          \config{U_S(\extS{S}{l}{d_1}{\frozentrue})}{e_{b_2}}$.

          Hence $\config{\extSRaw{S}{l}{u_{p_i}(p_1)}}{e_{b_1}}
          \parstepsto
          \config{\extSRaw{S}{l}{u_{p_i}(\state{d_1}{\frozentrue})}}{e_{b_2}}$.

          Since $u_{p_i}(\state{d_1}{\frozentrue}) =
          \state{d_1}{\frozentrue}$,

          we have that
          $\config{\extSRaw{S}{l}{u_{p_i}(p_1)}}{e_{b_1}}
          \parstepsto
          \config{\extS{S}{l}{d_1}{\frozentrue}}{e_{b_2}}$.

          By {\sc E-Eval-Ctxt}, it follows that

          $\config{\extSRaw{S}{l}{u_{p_i}(p_1)}}{\evalctxt{E'_b}{e_{b_1}}}
          \ctxstepsto
          \config{\extS{S}{l}{d_1}{\frozentrue}}{\evalctxt{E'_b}{e_{b_2}}}$,

          as we were required to show.

          For the second, consider that
          $\extS{S}{l}{d_1}{\frozentrue} = U_S(S)$, where $U_S$ is
          the store update operation that freezes the contents of $l$
          and acts as the identity on the contents of all other
          locations.

          Note that:
          \begin{itemize}
          \item $U_S$ is non-conflicting with $\config{S}{e_{a_1}}
            \parstepsto
            \config{\extSRaw{S}{l}{u_{p_i}(p_1)}}{e_{a_2}}$, since no
            locations are allocated in the transition;
          \item $U_S(\extSRaw{S}{l}{u_{p_i}(p_1)}) \neq \topS$,
            since $U_S(\extSRaw{S}{l}{u_{p_i}(p_1)}) =
            \extS{S}{l}{d_1}{\frozentrue}$ (since, by
            Definition~\ref{def:set-of-state-update-operations},
            $u_i(d_1) = d_1$; otherwise we would have
            $u_{p_i}(\state{d_1}{\frozentrue}) = \topp$, a
            contradiction), and we know $S \neq \topS$ and
            freezing the contents of location $l$ in $S$ cannot
            cause it to become $\topS$; and
          \item $U_S$ is freeze-safe with $\config{S}{e_{a_1}}
            \parstepsto
            \config{\extSRaw{S}{l}{u_{p_i}(p_1)}}{e_{a_2}}$, since
            $u_{p_i}$ does not alter the status of $p_1$.

            (By Definition~\ref{def:set-of-state-update-operations},
            $u_{p_i}$ can only change the status bit of a location if
            its contents are $\state{d}{\frozentrue}$ and $u_i(d) \neq
            d$, in which case $u_{p_i}$ changes the contents of the
            location to $\state{\top}{\frozenfalse}$; however, that
            cannot be the case here since then $u_{p_i}(p_1)$ would be
            $\topp$, and we would have $S_a = \topS$, a
            contradiction.)
          \end{itemize}
          Therefore, by Lemma~\ref{lem:generalized-independence}
          (Generalized Independence), we have that

          $\config{U_S(S)}{e_{a_1}} \parstepsto
          \config{U_S(\extSRaw{S}{l}{u_{p_i}(p_1)})}{e_{a_2}}$.

          Hence $\config{\extS{S}{l}{d_1}{\frozentrue}}{e_{a_1}}
          \parstepsto
          \config{\extS{S}{l}{d_1}{\frozentrue}}{e_{a_2}}$.

          By {\sc E-Eval-Ctxt}, it follows that

          $\config{\extS{S}{l}{d_1}{\frozentrue}}{\evalctxt{E'_a}{e_{a_1}}}
          \ctxstepsto
          \config{\extS{S}{l}{d_1}{\frozentrue}}{\evalctxt{E'_a}{e_{a_2}}}$,

          as we were required to show.
        \end{itemize}

      \end{itemize}

    \item \label{slqc-put-freeze-simple}Case {\sc E-Freeze-Simple}:
      Similar to case~\ref{slqc-put-freeze-final}, since $S_a =
      \extSRaw{S}{l}{u_{p_i}(p_1)}$ and $S_b =
      \extS{S}{l'}{d_1}{\frozentrue}$.

    \end{enumerate}
  \item Case {\sc E-Put-Err}: We have $\config{S_a}{e_{a_2}} =
    \error$.

    We proceed by case analysis on the rule by which
    $\config{S}{e_{b_1}}$ steps to $\config{S_b}{e_{b_2}}$.

    Since the only way an $\error$ configuration can arise is by the
    {\sc E-Put-Err} rule, we can assume in all other cases that
    $\conf_b \neq \error$.
    \begin{enumerate}
    \item \label{slqc-put-err-beta}Case {\sc E-Beta}: By symmetry with case~\ref{slqc-beta-put-err}.
    \item \label{slqc-put-err-new}Case {\sc E-New}: By symmetry with case~\ref{slqc-new-put-err}.
    \item \label{slqc-put-err-put}Case {\sc E-Put}: By symmetry with case~\ref{slqc-put-put-err}.
    \item \label{slqc-put-err-put-err}Case {\sc E-Put-Err}: We have
      $\config{S_a}{e_{a_2}} = \error$ and $\config{S_b}{e_{b_2}} =
      \error$, and so we choose $\conf_c = \error$, $i = 0$, $j = 0$,
      and $\pi = \id$.

      We have to show that:
      \begin{itemize}
      \item $\config{S_a}{\evalctxt{E'_b}{e_{b_1}}} = \error$, and
      \item $\config{S_b}{\evalctxt{E'_a}{e_{a_1}}} = \error$.
      \end{itemize}

      Since $\config{S_a}{e_{a_2}} = \error$, $S_a = \topS$, and since
      $\config{S_b}{e_{b_2}} = \error$, $S_b = \topS$, so both of the
      above follow immediately.

    \item \label{slqc-put-err-get}Case {\sc E-Get}: Similar to
      case~\ref{slqc-put-err-beta}, since $\config{S_a}{e_{a_2}} =
      \error$ and $S_b = S$.
    \item \label{slqc-put-err-freeze-init}Case {\sc E-Freeze-Init}:
      Similar to case~\ref{slqc-put-err-beta}, since
      $\config{S_a}{e_{a_2}} = \error$ and $S_b = S$.
    \item \label{slqc-put-err-spawn-handler}Case {\sc
      E-Spawn-Handler}: Similar to case~\ref{slqc-put-err-beta}, since
      $\config{S_a}{e_{a_2}} = \error$ and $S_b = S$.
    \item \label{slqc-put-err-freeze-final}Case {\sc E-Freeze-Final}:
      We have $\config{S_a}{e_{a_2}} = \error$ and $S_b =
      \extS{S}{l}{d_1}{\frozentrue}$, and so we choose $\conf_c =
      \error$, $i = 0$, $j = 1$, and $\pi = \id$.

      We have to show that:
      \begin{itemize}
      \item $\config{S_a}{\evalctxt{E'_b}{e_{b_1}}} = \error$,
        and
      \item $\config{S_b}{\evalctxt{E'_a}{e_{a_1}}} \ctxstepsto
        \error$.
      \end{itemize}

      The first of these is immediately true because since
      $\config{S_a}{e_{a_2}} = \error$, $S_a = \topS$, and so
      $\config{S_a}{\evalctxt{E'_b}{e_{b_1}}}$ is equal to $\error$ as
      well.

      For the second, observe that since $\config{S}{e_{b_1}}
      \parstepsto \config{S_b}{e_{b_2}}$, we have by
      Lemma~\ref{lem:monotonicity} (Monotonicity) that
      $\leqstore{S}{S_b}$.

      Therefore, since $\config{S}{e_{a_1}} \parstepsto \error$, we
      have by Lemma~\ref{lem:error-preservation} that
      $\config{S_b}{e_{a_1}} \parstepsto \error$.

      Since $\error$ is equal to $\config{\topS}{e}$ for all
      expressions $e$, $\config{S_b}{e_{a_1}} \parstepsto
      \config{\topS}{e}$ for all $e$.

      Therefore, by {\sc E-Eval-Ctxt},
      $\config{S_b}{\evalctxt{E'_a}{e_{a_1}}} \ctxstepsto
      \config{\topS}{\evalctxt{E'_a}{e}}$ for all $e$.

      Since $\config{\topS}{\evalctxt{E'_a}{e}}$ is equal to $\error$,
      we have that $\config{S_b}{\evalctxt{E'_a}{e_{a_1}}} \ctxstepsto
      \error$, as we were required to show.

    \item \label{slqc-put-err-freeze-simple}Case {\sc
      E-Freeze-Simple}: Similar to
      case~\ref{slqc-put-err-freeze-final}, since $S_b =
      \extS{S}{l}{d_1}{\frozentrue}$.

    \end{enumerate}
  \item Case {\sc E-Get}: We have $S_a = S$.

    We proceed by case analysis on the rule by which
    $\config{S}{e_{b_1}}$ steps to $\config{S_b}{e_{b_2}}$.

    Since the only way an $\error$ configuration can arise is by the
    {\sc E-Put-Err} rule, we can assume in all other cases that
    $\conf_b \neq \error$.
    \begin{enumerate}
    \item \label{slqc-get-beta}Case {\sc E-Beta}: By symmetry with case~\ref{slqc-beta-get}.
    \item \label{slqc-get-new}Case {\sc E-New}: By symmetry with case~\ref{slqc-new-get}.
    \item \label{slqc-get-put}Case {\sc E-Put}: By symmetry with case~\ref{slqc-put-get}.
    \item \label{slqc-get-put-err}Case {\sc E-Put-Err}: By symmetry with case~\ref{slqc-put-err-get}.
    \item \label{slqc-get-get}Case {\sc E-Get}: Similar to
      case~\ref{slqc-get-beta}, since $S_a = S$ and $S_b = S$.
    \item \label{slqc-get-freeze-init}Case {\sc E-Freeze-Init}:
      Similar to case~\ref{slqc-get-beta}, since $S_a = S$ and $S_b = S$.
    \item \label{slqc-get-spawn-handler}Case {\sc E-Spawn-Handler}:
      Similar to case~\ref{slqc-get-beta}, since $S_a = S$ and $S_b = S$.
    \item \label{slqc-get-freeze-final}Case {\sc E-Freeze-Final}:
      Similar to case~\ref{slqc-beta-freeze-final}, since $S_a = S$
      and $S_b = \extS{S}{l}{d_1}{\frozentrue}$.
    \item \label{slqc-get-freeze-simple}Case {\sc E-Freeze-Simple}:
      Similar to case~\ref{slqc-beta-freeze-simple}, since $S_a = S$
      and $S_b = \extS{S}{l}{d_1}{\frozentrue}$.
    \end{enumerate}

  \item Case {\sc E-Freeze-Init}: We have $S_a = S$.

    We proceed by case analysis on the rule by which
    $\config{S}{e_{b_1}}$ steps to $\config{S_b}{e_{b_2}}$.

    Since the only way an $\error$ configuration can arise is by the
    {\sc E-Put-Err} rule, we can assume in all other cases that
    $\conf_b \neq \error$.
    \begin{enumerate}
    \item \label{slqc-freeze-init-beta}Case {\sc E-Beta}: By symmetry with case~\ref{slqc-beta-freeze-init}.
    \item \label{slqc-freeze-init-new}Case {\sc E-New}: By symmetry with case~\ref{slqc-new-freeze-init}.
    \item \label{slqc-freeze-init-put}Case {\sc E-Put}: By symmetry with case~\ref{slqc-put-freeze-init}.
    \item \label{slqc-freeze-init-put-err}Case {\sc E-Put-Err}: By symmetry with case~\ref{slqc-put-err-freeze-init}.
    \item \label{slqc-freeze-init-get}Case {\sc E-Get}: By symmetry with case~\ref{slqc-get-freeze-init}.
    \item \label{slqc-freeze-init-freeze-init}Case {\sc
      E-Freeze-Init}: Similar to case~\ref{slqc-freeze-init-beta},
      since $S_a = S$ and $S_b = S$.
    \item \label{slqc-freeze-init-spawn-handler}Case {\sc
      E-Spawn-Handler}: Similar to case~\ref{slqc-freeze-init-beta},
      since $S_a = S$ and $S_b = S$.
    \item \label{slqc-freeze-init-freeze-final}Case {\sc
      E-Freeze-Final}: Similar to case~\ref{slqc-beta-freeze-final},
      since $S_a = S$ and $S_b = \extS{S}{l}{d_1}{\frozentrue}$.
    \item \label{slqc-freeze-init-freeze-simple}Case {\sc
      E-Freeze-Simple}: Similar to case~\ref{slqc-beta-freeze-simple},
      since $S_a = S$ and $S_b = \extS{S}{l}{d_1}{\frozentrue}$.
    \end{enumerate}

  \item Case {\sc E-Spawn-Handler}: We have $S_a = S$.

    We proceed by case analysis on the rule by which
    $\config{S}{e_{b_1}}$ steps to $\config{S_b}{e_{b_2}}$.

    Since the only way an $\error$ configuration can arise is by the
    {\sc E-Put-Err} rule, we can assume in all other cases that
    $\conf_b \neq \error$.
    \begin{enumerate}
    \item \label{slqc-spawn-handler-beta}Case {\sc E-Beta}: By symmetry with case~\ref{slqc-beta-spawn-handler}.
    \item \label{slqc-spawn-handler-new}Case {\sc E-New}: By symmetry with case~\ref{slqc-new-spawn-handler}.
    \item \label{slqc-spawn-handler-put}Case {\sc E-Put}: By symmetry with case~\ref{slqc-put-spawn-handler}.
    \item \label{slqc-spawn-handler-put-err}Case {\sc E-Put-Err}: By symmetry with case~\ref{slqc-put-err-spawn-handler}.
    \item \label{slqc-spawn-handler-get}Case {\sc E-Get}: By symmetry with case~\ref{slqc-get-spawn-handler}.
    \item \label{slqc-spawn-handler-freeze-init}Case {\sc E-Freeze-Init}: By symmetry with case~\ref{slqc-freeze-init-spawn-handler}.
    \item \label{slqc-spawn-handler-spawn-handler}Case {\sc
      E-Spawn-Handler}: Similar to case~\ref{slqc-spawn-handler-beta},
      since $S_a = S$ and $S_b = S$.
    \item \label{slqc-spawn-handler-freeze-final}Case {\sc
      E-Freeze-Final}: Similar to case~\ref{slqc-beta-freeze-final},
      since $S_a = S$ and $S_b = \extS{S}{l}{d_1}{\frozentrue}$.
    \item \label{slqc-spawn-handler-freeze-simple}Case {\sc
      E-Freeze-Simple}: Similar to case~\ref{slqc-beta-freeze-simple},
      since $S_a = S$ and $S_b = \extS{S}{l}{d_1}{\frozentrue}$.
    \end{enumerate}

  \item Case {\sc E-Freeze-Final}: We have $S_a =
    \extS{S}{l}{d_1}{\frozentrue}$.

    We proceed by case analysis on the rule by which
    $\config{S}{e_{b_1}}$ steps to $\config{S_b}{e_{b_2}}$.

    Since the only way an $\error$ configuration can arise is by the
    {\sc E-Put-Err} rule, we can assume in all other cases that
    $\conf_b \neq \error$.
    \begin{enumerate}
    \item \label{slqc-freeze-final-beta}Case {\sc E-Beta}: By symmetry with case~\ref{slqc-beta-freeze-final}.
    \item \label{slqc-freeze-final-new}Case {\sc E-New}: By symmetry with case~\ref{slqc-new-freeze-final}.
    \item \label{slqc-freeze-final-put}Case {\sc E-Put}: By symmetry with case~\ref{slqc-put-freeze-final}.
    \item \label{slqc-freeze-final-put-err}Case {\sc E-Put-Err}: By symmetry with case~\ref{slqc-put-err-freeze-final}.
    \item \label{slqc-freeze-final-get}Case {\sc E-Get}: By symmetry with case~\ref{slqc-get-freeze-final}.
    \item \label{slqc-freeze-final-freeze-init}Case {\sc E-Freeze-Init}: By symmetry with case~\ref{slqc-freeze-init-freeze-final}.
    \item \label{slqc-freeze-final-spawn-handler}Case {\sc E-Spawn-Handler}: By symmetry with case~\ref{slqc-spawn-handler-freeze-final}.
    \item \label{slqc-freeze-final-freeze-final}Case {\sc
      E-Freeze-Final}: We have $S_a = \extS{S}{l}{d_1}{\frozentrue}$
      and $S_b = \extS{S}{l'}{d'_1}{\frozentrue}$.

      Now consider whether $l = l'$:
      \begin{itemize}
      \item If $l \neq l'$:

        Choose $S' =
        \extS{\extS{S}{l'}{d'_1}{\frozentrue}}{l}{d_1}{\frozentrue}$,
        $i = 1$, $j = 1$, and $\pi = \id$.

        We have to show that:
        \begin{itemize}
        \item
          $\config{\extS{S}{l}{d_1}{\frozentrue}}{\evalctxt{E'_b}{e_{b_1}}}
          \ctxstepsto
          \config{\extS{\extS{S}{l'}{d'_1}{\frozentrue}}{l}{d_1}{\frozentrue}}{\evalctxt{E'_b}{e_{b_2}}}$,
          and
        \item
          $\config{\extS{S}{l'}{d'_1}{\frozentrue}}{\evalctxt{E'_a}{e_{a_1}}}
          \ctxstepsto
          \config{\extS{\extS{S}{l'}{d'_1}{\frozentrue}}{l}{d_1}{\frozentrue}}{\evalctxt{E'_a}{e_{a_2}}}$.
        \end{itemize}

        For the first of these, consider that
        $\extS{S}{l}{d_1}{\frozentrue} = U_S(S)$, where $U_S$ is the
        store update operation that freezes the contents of $l$
        and acts as the identity on the contents of all other
        locations.

        Note that:
        \begin{itemize}
        \item $U_S$ is non-conflicting with $\config{S}{e_{b_1}}
          \parstepsto
          \config{\extS{S}{l'}{d'_1}{\frozentrue}}{e_{b_2}}$, since
          no locations are allocated in the transition;
        \item $U_S(\extS{S}{l'}{d'_1}{\frozentrue}) \neq \topS$,

          since $U_S(\extS{S}{l'}{d'_1}{\frozentrue}) =
          \extS{\extS{S}{l'}{d'_1}{\frozentrue}}{l}{d_1}{\frozentrue}$
          and we know $S \neq \topS$ and freezing the contents of
          locations $l$ and $l'$ in $S$ cannot cause it to become
          $\topS$ (because if so, then we would have $S_a = \topS$
          or $S_b = \topS$, which we know are not the case); and
        \item $U_S$ is freeze-safe with $\config{S}{e_{b_1}}
          \parstepsto
          \config{\extS{S}{l'}{d'_1}{\frozentrue}}{e_{b_2}}$, since
          the only location that can change in status between $S$
          and $\extS{S}{l'}{d'_1}{\frozentrue}$ is $l'$, and $U_S$
          acts as the identity on $l'$.
        \end{itemize}
        Therefore, by Lemma~\ref{lem:generalized-independence}
        (Generalized Independence), we have that

        $\config{U_S(S)}{e_{b_1}} \parstepsto
        \config{U_S(\extS{S}{l'}{d'_1}{\frozentrue})}{e_{b_2}}$.

        Hence $\config{\extS{S}{l}{d_1}{\frozentrue}}{e_{b_1}}
        \parstepsto
        \config{\extS{\extS{S}{l'}{d'_1}{\frozentrue}}{l}{d_1}{\frozentrue}}{e_{b_2}}$.

        By {\sc E-Eval-Ctxt}, it follows that

        $\config{\extS{S}{l}{d_1}{\frozentrue}}{\evalctxt{E'_b}{e_{b_1}}}
        \ctxstepsto
        \config{\extS{\extS{S}{l'}{d'_1}{\frozentrue}}{l}{d_1}{\frozentrue}}{\evalctxt{E'_b}{e_{b_2}}}$,
        
        as we were required to show.

        The argument for the second is symmetrical.

      \item If $l = l'$:

        \lk{This is the case where we freeze the same location twice,
          which is no problem; the second freeze is a no-op.}

        Note that since $l = l'$, $d_1 = d'_1$ as well.

        Choose $S' = \extS{S}{l}{d_1}{\frozentrue}$, $i = 1$, $j =
        1$, and $\pi = \id$.

        We have to show that:
        \begin{itemize}
        \item
          $\config{\extS{S}{l}{d_1}{\frozentrue}}{\evalctxt{E'_b}{e_{b_1}}}
          \ctxstepsto
          \config{\extS{S}{l}{d_1}{\frozentrue}}{\evalctxt{E'_b}{e_{b_2}}}$,
          and
        \item
          $\config{\extS{S}{l'}{d'_1}{\frozentrue}}{\evalctxt{E'_a}{e_{a_1}}}
          \ctxstepsto
          \config{\extS{S}{l}{d_1}{\frozentrue}}{\evalctxt{E'_a}{e_{a_2}}}$.
        \end{itemize}

        For the first of these, consider that
        $\extS{S}{l}{d_1}{\frozentrue} = U_S(S)$, where $U_S$ is the
        store update operation that freezes the contents of $l$ and
        acts as the identity on the contents of all other locations.

        Note that:
        \begin{itemize}
        \item $U_S$ is non-conflicting with $\config{S}{e_{b_1}}
          \parstepsto
          \config{\extS{S}{l}{d_1}{\frozentrue}}{e_{b_2}}$, since no
          locations are allocated in the transition;
        \item $U_S(\extS{S}{l}{d_1}{\frozentrue}) \neq \topS$, since
          $U_S(\extS{S}{l}{d_1}{\frozentrue}) =
          \extS{S}{l}{d_1}{\frozentrue}$, and we know $S \neq \topS$
          and freezing the contents of location $l$ in $S$ cannot
          cause it to become $\topS$; and
        \item $U_S$ is freeze-safe with $\config{S}{e_{b_1}}
          \parstepsto
          \config{\extS{S}{l}{d_1}{\frozentrue}}{e_{b_2}}$, since
          the only location that can change in status between $S$
          and $\extS{S}{l}{d_1}{\frozentrue}$ is $l$, and $U_S$
          freezes the contents of $l$ but has no other effect on
          them.
        \end{itemize}

        Therefore, by Lemma~\ref{lem:generalized-independence}
        (Generalized Independence), we have that

        $\config{U_S(S)}{e_{b_1}} \parstepsto
        \config{U_S(\extS{S}{l}{d_1}{\frozentrue})}{e_{b_2}}$.

        Hence $\config{\extS{S}{l}{d_1}{\frozentrue}}{e_{b_1}}
        \parstepsto
        \config{\extS{S}{l}{d_1}{\frozentrue}}{e_{b_2}}$.

        By {\sc E-Eval-Ctxt}, it follows that

        $\config{\extS{S}{l}{d_1}{\frozentrue}}{\evalctxt{E'_b}{e_{b_1}}}
        \ctxstepsto
        \config{\extS{S}{l}{d_1}{\frozentrue}}{\evalctxt{E'_b}{e_{b_2}}}$,

        as we were required to show.

        The argument for the second is symmetrical.

      \end{itemize}

    \item \label{slqc-freeze-final-freeze-simple}Case {\sc
      E-Freeze-Simple}: Similar to
      case~\ref{slqc-freeze-final-freeze-final}, since $S_a =
      \extS{S}{l}{d_1}{\frozentrue}$ and $S_b =
      \extS{S}{l'}{d'_1}{\frozentrue}$.
    \end{enumerate}

  \item Case {\sc E-Freeze-Simple}: We have $S_a =
    \extS{S}{l}{d_1}{\frozentrue}$.

    \begin{enumerate}
    \item \label{slqc-freeze-simple-beta}Case {\sc E-Beta}: By symmetry with case~\ref{slqc-beta-freeze-simple}.
    \item \label{slqc-freeze-simple-new}Case {\sc E-New}: By symmetry with case~\ref{slqc-new-freeze-simple}.
    \item \label{slqc-freeze-simple-put}Case {\sc E-Put}: By symmetry with case~\ref{slqc-put-freeze-simple}.
    \item \label{slqc-freeze-simple-put-err}Case {\sc E-Put-Err}: By symmetry with case~\ref{slqc-put-err-freeze-simple}.
    \item \label{slqc-freeze-simple-get}Case {\sc E-Get}: By symmetry with case~\ref{slqc-get-freeze-simple}.
    \item \label{slqc-freeze-simple-freeze-init}Case {\sc E-Freeze-Init}: By symmetry with case~\ref{slqc-freeze-init-freeze-simple}.
    \item \label{slqc-freeze-simple-spawn-handler}Case {\sc E-Spawn-Handler}: By symmetry with case~\ref{slqc-spawn-handler-freeze-simple}.
    \item \label{slqc-freeze-simple-freeze-final}Case {\sc E-Freeze-Final}: By symmetry with case~\ref{slqc-freeze-final-freeze-simple}.
    \item \label{slqc-freeze-simple-freeze-simple}Case {\sc
      E-Freeze-Simple}: Similar to
      case~\ref{slqc-freeze-simple-freeze-final}, since $S_a =
      \extS{S}{l}{d_1}{\frozentrue}$ and $S_b =
      \extS{S}{l'}{d'_1}{\frozentrue}$.
    \end{enumerate}

  \end{enumerate}
\end{proof}



\section{Proof of Lemma~\ref{lem:strong-one-sided-quasi-confluence}}\label{section:strong-one-sided-quasi-confluence-proof}
\begin{proof}
  Suppose $\conf \ctxstepsto \conf'$ and $\conf \ctxstepsto^m
  \conf''$, where $1 \leq m$.

  We are required to show that either:
  \begin{enumerate}
  \item there exist $\conf_c, i, j, \pi$ such that $\conf'
    \ctxstepsto^i \conf_c$ and $\pi(\conf'') \ctxstepsto^j \conf_c$
    and $i \leq m$ and $j \leq 1$, or
  \item there exists $k \leq m$ such that $\conf' \ctxstepsto^k
    \textup{\error}$, or there exists $k \leq 1$ such that $\conf''
    \ctxstepsto^k \textup{\error}$.
  \end{enumerate}

  We proceed by induction on $m$.

  In the base case of $m = 1$, the result is immediate from
  Lemma~\ref{lem:strong-local-quasi-confluence}, with $k = 1$.

  For the induction step, suppose $\conf \ctxstepsto^m \conf''
  \ctxstepsto \conf'''$ and suppose the lemma holds for $m$.

  We show that it holds for $m + 1$, as follows.

  From the induction hypothesis, we have that either:
  \begin{enumerate}
  \item there exist $\conf_c', i', j', \pi'$ such that $\conf'
    \ctxstepsto^{i'} \conf_c'$ and $\pi'(\conf'') \ctxstepsto^{j'}
    \conf_c'$ and $i' \leq m$ and $j' \leq 1$, or
  \item there exists $k' \leq m$ such that $\conf'
    \ctxstepsto^{k'} \error$, or there exists $k' \leq 1$ such that
    $\conf'' \ctxstepsto^{k'} \error$.
  \end{enumerate}

  We consider these two cases in turn:
  \begin{enumerate}
  \item There exist $\conf_c', i', j', \pi'$ such that $\conf'
    \ctxstepsto^{i'} \conf_c'$ and $\pi'(\conf'') \ctxstepsto^{j'}
    \conf_c'$ and $i' \leq m$ and $j' \leq 1$:

    We proceed by cases on $j'$:
    \begin{itemize}

    \item If $j' = 0$, then $\pi'(\conf'') = \conf_c'$.

      Since $\conf'' \ctxstepsto \conf'''$, we have that
      $\pi'(\conf'') \ctxstepsto \pi'(\conf''')$ by
      Lemma~\ref{lem:permutability} (Permutability).

      We can then choose $\conf_c = \pi'(\conf''')$ and $i = i' + 1$
      and $j = 0$ and $\pi = \pi'$.

      The key is that $\conf' \ctxstepsto^{i'} \conf'_c =
      \pi'(\conf'') \ctxstepsto \pi'(\conf''')$ for a total of $i' +
      1$ steps.
      
    \item If $j' = 1$:

      First, since $\pi'(\conf'') \ctxstepsto^{j'} \conf'_c$, then
      by Lemma~\ref{lem:permutability} (Permutability) we have that
      $\conf'' \ctxstepsto^{j'} \piprimeinv(\conf'_c)$.
      
      Then, by $\conf'' \ctxstepsto^{j'} \piprimeinv(\conf'_c)$ and
      $\conf'' \ctxstepsto \conf'''$ and
      Lemma~\ref{lem:strong-local-quasi-confluence}, one of the
      following two cases is true:
      \begin{enumerate}
      \item There exist $\conf_c''$ and $i''$ and $j''$ and $\pi''$
        such that $\piprimeinv(\conf'_c) \ctxstepsto^{i''}
        \conf_c''$ and $\pi''(\conf''') \ctxstepsto^{j''} \conf_c''$
        and $i'' \leq 1$ and $j'' \leq 1$.

        Since $\piprimeinv(\conf'_c) \ctxstepsto^{i''} \conf_c''$,
        by Lemma~\ref{lem:permutability} (Permutability) we have
        that $\conf'_c \ctxstepsto^{i''} \pi'(\conf_c'')$.

        So we also have $\conf' \ctxstepsto^{i'} \conf_c'
        \ctxstepsto^{i''} \pi'(\conf_c'')$.

        Since $\pi''(\conf''') \ctxstepsto^{j''} \conf_c''$, by
        Lemma~\ref{lem:permutability} (Permutability) we have that
        $\pi'(\pi''(\conf''')) \ctxstepsto^{j''} \pi'(\conf_c'')$.

        In summary, we pick $\conf_c = \pi'(\conf_c'')$ and $i = i' + i''$
        and $j = j''$ and $\pi = \pi'' \circ \pi'$, which is sufficient
        because $i = i' + i'' \leq m + 1$ and $j = j'' \leq 1$.

      \item $\piprimeinv(\conf'_c) \ctxstepsto \error$ or $\conf'''
        \ctxstepsto \error$.

        If $\conf''' \ctxstepsto \error$, then choosing $k = 1$
        satisfies the proof.

        Otherwise, $\piprimeinv(\conf'_c) \ctxstepsto \error$.

        Then, by Lemma~\ref{lem:permutability} we have that
        $\conf'_c \ctxstepsto \pi'(\error)$.

        By Definition~\ref{def:permutation-configuration},
        $\pi'(\error) = \error$, and so $\conf'_c \ctxstepsto
        \error$.

        Therefore $\conf' \ctxstepsto^{i'} \conf'_c \ctxstepsto
        \error$.

        Hence $\conf' \ctxstepsto^{i'+1} \error$.

        Since $i' \leq m$, we have that $i' + 1 \leq m + 1$, and
        so choosing $k = i' + 1$ satisfies the proof.
        
      \end{enumerate}

    \end{itemize}

  \item There exists $k' \leq m$ such that $\conf' \ctxstepsto^{k'}
    \error$, or there exists $k' \leq 1$ such that $\conf''
    \ctxstepsto^{k'} \error$:

    If there exists $k' \leq m$ such that $\conf' \ctxstepsto^{k'}
    \error$, then choosing $k = k'$ satisfies the proof.

    Otherwise, there exists $k' \leq 1$ such that $\conf''
    \ctxstepsto^{k'} \error$.

    We proceed by cases on $k'$:

    \begin{itemize}

    \item If $k' = 0$, then $\conf'' = \error$.

      Hence this case is not possible, since $\conf'' \ctxstepsto
      \conf'''$ and $\error$ cannot step.

    \item If $k' = 1$:

      From $\conf'' \ctxstepsto \conf'''$ and $\conf''
      \ctxstepsto^{k'} \error$ and
      Lemma~\ref{lem:strong-local-quasi-confluence}, one of the
      following two cases is true:

      \begin{enumerate}
      \item There exist $\conf_c''$ and $i''$ and $j''$ and $\pi''$
        such that $\error \ctxstepsto^{i''} \conf_c''$ and
        $\pi''(\conf''') \ctxstepsto^{j''} \conf_c''$ and $i'' \leq
        1$ and $j'' \leq 1$.

        Since $\error$ cannot step, $i'' = 0$ and $\conf''_c =
        \error$.

        By Definition~\ref{def:permutation-configuration},
        $\pi''(\conf''') = \conf'''$.

        Hence $\conf''' \ctxstepsto^{j''} \error$.

        \lk{This is the one place that we need to allow $k$ to be
          $\leq$ 1 instead of exactly 1.}

        Since $j'' \leq 1$, choosing $k = j''$ satisfies the proof.

      \item $\error \ctxstepsto \error$ or $\conf''' \ctxstepsto
        \error$.

        Since $\error$ cannot step, $\conf''' \ctxstepsto \error$.

        Hence choosing $k = 1$ satisfies the proof.

      \end{enumerate}

    \end{itemize}

  \end{enumerate}

\end{proof}


\section{Proof of Lemma~\ref{lem:strong-quasi-confluence}}\label{section:strong-quasi-confluence-proof}
\begin{proof}
  We proceed by induction on $n$.  In the base case of $n = 1$, the
  result is immediate from Lemma~\ref{lem:strong-one-sided-quasi-confluence}.

  For the induction step, suppose $\conf \parstepsto^n \conf'
  \parstepsto \conf'''$ and suppose the lemma holds for $n$.

  We show that it holds for $n + 1$, as follows.

  We are required to show that either:
  \begin{enumerate}
  \item there exist $\conf_c, i, j$ such that $\conf''' \parstepsto^i
    \conf_c$ and $\conf'' \parstepsto^j \conf_c$ and $i \leq m$ and $j
    \leq n + 1$, or
  \item there exists $k \leq m$ such that $\conf''' \parstepsto^k
    \error$, or there exists $k \leq n + 1$ such that $\conf''
    \parstepsto^k \error$.
  \end{enumerate}

  From the induction hypothesis, we have that either:
  \begin{enumerate}
  \item there exist $\conf'_c, i', j'$ such that $\conf'
    \parstepsto^{i'} \conf'_c$ and $\conf'' \parstepsto^{j'} \conf'_c$
    and $i' \leq m$ and $j' \leq n$, or
  \item there exists $k' \leq m$ such that $\conf' \parstepsto^{k'}
    \error$, or there exists $k' \leq n$ such that $\conf''
    \parstepsto^{k'} \error$.
  \end{enumerate}

  We consider these two cases in turn:

  \begin{enumerate}
  \item There exist $\conf'_c, i', j'$ such that $\conf'
    \parstepsto^{i'} \conf'_c$ and $\conf'' \parstepsto^{j'} \conf'_c$
    and $i' \leq m$ and $j' \leq n$:

    We proceed by cases on $i'$:
    \begin{itemize}

    \item If $i' = 0$, then $\conf' = \conf_c'$.  We can then choose
      $\conf_c = \conf'''$ and $i = 0$ and $j = j' + 1$.

    \item If $i' \geq 1$:

      From $\conf' \parstepsto \conf'''$ and $\conf' \parstepsto^{i'}
      \conf_c'$ and Lemma~\ref{lem:strong-one-sided-quasi-confluence},
      one of the following two cases is true:
      \begin{enumerate}
        \item There exist $\conf_c''$ and $i''$ and $j''$ such that
          $\conf''' \parstepsto^{i''} \conf_c''$ and $\conf_c'
          \parstepsto^{j''} \conf_c''$ and $i'' \leq i'$ and $j'' \leq
          1$.  So we also have $\conf'' \parstepsto^{j'} \conf_c'
          \parstepsto^{j''} \conf_c''$.  In summary, we pick $\conf_c
          = \conf_c''$ and $i = i''$ and $j = j' + j''$, which is
          sufficient because $i = i'' \leq i' \leq m$ and $j = j' +
          j'' \leq n + 1$.
        \item There exists $k'' \leq i'$ such that $\conf'''
          \parstepsto^{k''} \error$, or there exists $k'' \leq 1$ such
          that $\conf'_c \parstepsto^{k''} \error$.

          If there exists $k'' \leq i'$ such that $\conf'''
          \parstepsto^{k''} \error$, then choosing $k = k''$ satisfies
          the proof, since $k'' \leq i' \leq m$.

          Otherwise, there exists $k'' \leq 1$ such
          that $\conf'_c \parstepsto^{k''} \error$.

          Therefore, $\conf'' \parstepsto^{j'} \conf_c'
          \parstepsto^{k''} \error$.

          Hence $\conf'' \parstepsto^{j' + k''} \error$.

          Since $j' \leq n$ and $k'' \leq 1$, $j' + k'' \leq n + 1$.

          Hence choosing $k = j' + k''$ satisfies the proof.

      \end{enumerate}
    \end{itemize}

  \item There exists $k' \leq m$ such that $\conf' \parstepsto^{k'}
    \error$, or there exists $k' \leq n$ such that $\conf''
    \parstepsto^{k'} \error$:

    If there exists $k' \leq n$ such that $\conf'' \parstepsto^{k'}
    \error$, then choosing $k = k'$ satisfies the proof.

    Otherwise, there exists $k' \leq m$ such that $\conf'
    \parstepsto^{k'} \error$.  We proceed by cases on $k'$:

    \begin{itemize}

    \item If $k' = 0$, then $\conf' = \error$.

      Hence this case is not possible, since $\conf' \parstepsto
      \conf'''$ and $\error$ cannot step.

    \item If $k' \geq 1$:

      From $\conf' \parstepsto \conf'''$ and $\conf' \parstepsto^{k'}
      \error$ and Lemma~\ref{lem:strong-one-sided-quasi-confluence},
      one of the following two cases is true:

      \begin{enumerate}
        \item There exist $\conf''_c$ and $i''$ and $j''$ such that
          $\conf''' \parstepsto^{i''} \conf''_c$ and $\error
          \parstepsto^{j''} \conf''_c$ and $i'' \leq k'$ and $j'' \leq
          1$.

          Since $\error$ cannot step, $j'' = 0$ and $\conf''_c =
          \error$.

          Hence $\conf''' \parstepsto^{i''} \error$.

          Since $i'' \leq k' \leq m$, choosing $k = i''$ satisfies the
          proof.

        \item There exists $k'' \leq k'$ such that $\conf'''
          \parstepsto^{k''} \error$, or there exists $k'' \leq 1$ such
          that $\error \parstepsto^{k''} \error$.

          Since $\error$ cannot step, there exists $k'' \leq k'$ such
          that $\conf''' \parstepsto^{k''} \error$.

          Since $k'' \leq k' \leq m$, choosing $k = k''$ satisfies the
          proof.
      \end{enumerate}
    \end{itemize}
  \end{enumerate}

\end{proof}


\section{Proof of Theorem~\ref{thm:determinism-of-threshold-queries}}\label{section:determinism-of-threshold-queries-proof}
\begin{proof}
  Consider replica $i$ of a threshold CvRDT $(S, \leq, s^0, q, t, u,
  m)$.

  Let $\mathcal{S}$ be a threshold set with respect to
  $(S, \leq)$.

  Consider a method execution $t^{k+1}_i(\mathcal{S})$ (\ie, a
  threshold query that is the $k+1$th method execution on replica $i$,
  with threshold set $\mathcal{S}$ as its argument) that returns some
  set of activation states $S_a \in \mathcal{S}$.

  For part~\ref{thm:this-replica} of the theorem, we have to show that
  threshold queries with $\mathcal{S}$ as their argument will always
  return $S_a$ on subsequent executions at $i$.

  That is, we have to show that, for all $k' > (k+1)$, the threshold
  query $t^{k'}_i(\mathcal{S})$ on $i$ returns $S_a$.

  Since $t^{k+1}_i(\mathcal{S})$ returns $S_a$, from
  Definition~\ref{def:cvrdt-with-threshold-queries} we have that for
  some activation state $s_a \in S_a$, the condition $s_a \leq s^k_i$
  holds.

  Consider arbitrary $k' > (k+1)$.

  Since state is inflationary across updates, we know that the state
  $s^{k'}_i$ after method execution $k'$ is at least $s^k_i$.

  That is, $s^k_i \leq s^{k'}_i$.

  By transitivity of $\leq$, then, $s_a \leq s^{k'}_i$.

  Hence, by Definition~\ref{def:cvrdt-with-threshold-queries},
  $t^{k'}_i(\mathcal{S})$ returns $S_a$.

  For part~\ref{thm:any-replica} of the theorem, consider some replica
  $j$ of $(S, \leq, s^0, q, t, u, m)$, located at process $p_j$.

  We are required to show that, for all $x \geq 0$, the threshold
  query $t^{x+1}_j(\mathcal{S})$ returns $S_a$ eventually, and blocks
  until it does.\footnote{The occurrences of $k+1$ and $x+1$ in this
    proof are an artifact of how we index method executions starting
    from $1$, but states starting from $0$.  The initial state (of
    every replica) is $s^0$, and so $s^k_i$ is the state of replica
    $i$ after method execution $k$ has completed at $i$.}

  That is, we must show that, for all $x \geq 0$, there exists some
  finite $n \geq 0$ such that
  \begin{itemize}
  \item 
    for all $i$ in the range $0 \leq i \leq n-1$, the threshold query
    $t^{x+1+i}_j(\mathcal{S})$ returns $\block$, and
  \item
    for all $i \geq n$, the threshold query $t^{x+1+i}_j(\mathcal{S})$
    returns $S_a$.
  \end{itemize}
  Consider arbitrary $x \geq 0$.

  Recall that $s^x_j$ is the state of replica $j$ after the $x$th
  method execution, and therefore $s^x_j$ is also the state of $j$
  when $t^{x+1}_j(\mathcal{S})$ runs.
  %
  We have three cases to consider:
  \begin{itemize}
  \item $s^k_i \leq s^x_j$.

    (That is, replica $i$'s state after the $k$th method execution on $i$
    is \emph{at or below} replica $j$'s state after the $x$th method
    execution on $j$.)

    Choose $n = 0$.

    We have to show that, for all $i \geq n$, the threshold query
    $t^{x+1+i}_j(\mathcal{S})$ returns $S_a$.

    Since $t^{k+1}_i(\mathcal{S})$ returns $S_a$, we know that there
    exists an $s_a \in S_a$ such that $s_a \leq s^k_i$.

    Since $s^k_i \leq s^x_j$, we have by transitivity of $\leq$ that
    $s_a \leq s^x_j$.

    Therefore, by Definition~\ref{def:cvrdt-with-threshold-queries},
    $t^{x+1}_j(\mathcal{S})$ returns $S_a$.

    Then, by part~\ref{thm:this-replica} of the theorem, we have that
    subsequent executions $t^{x+1+i}_j(\mathcal{S})$ at replica $j$
    will also return $S_a$, and so the case holds.

    (Note that this case includes the possibility $s^k_i \equiv s^0$,
    in which no updates have executed at replica $i$.)

  \item $s^k_i > s^x_j$.

    (That is, replica $i$'s state after the $k$th method execution on $i$
    is \emph{above} replica $j$'s state after the $x$th method execution
    on $j$.)

    We have two subcases:

    \begin{itemize}
    \item
      There exists some activation state $s'_a \in S_a$ for which $s'_a \leq
      s^x_j$.

      In this case, we choose $n = 0$.

      We have to show that, for all $i \geq n$, the threshold query
      $t^{x+1+i}_j(\mathcal{S})$ returns $S_a$.

      Since $s'_a \leq s^x_j$, by
      Definition~\ref{def:cvrdt-with-threshold-queries},
      $t^{x+1}_j(\mathcal{S})$ returns $S_a$.

      Then, by part~\ref{thm:this-replica} of the theorem, we have
      that subsequent executions $t^{x+1+i}_j(\mathcal{S})$ at replica
      $j$ will also return $S_a$, and so the case holds.

    \item
      There is no activation state $s'_a \in S_a$ for which $s'_a \leq
      s^x_j$.

      Since $t^{k+1}_i(\mathcal{S})$ returns $S_a$, we know that there
      is some update $u^{k'}_i(a)$ in $i$'s causal history, for some
      $k' < (k+1)$, that updates $i$ from a state at or below $s^x_j$
      to $s^k_i$.\footnote{We know that $i$'s state was once at or
        below $s^x_j$, because $i$ and $j$ started at the same state
        $s^0$ and can both only grow.  Hence the least that $s^x_j$
        can be is $s^0$, and we know that $i$ was originally $s^0$ as
        well.}

      By eventual delivery, $u^{k'}_i(a)$ is eventually delivered at
      $j$.

      Hence some update or updates that will increase $j$'s state from
      $s^x_j$ to a state at or above some $s'_a$ must reach replica
      $j$.\footnote{We say ``some update or updates'' because the
        exact update $u^{k'}_i(a)$ may not be the update that causes
        the threshold query at $j$ to unblock; a different update or
        updates could do it.  Nevertheless, the existence of
        $u^{k'}_i(a)$ means that there is at least one update that
        will suffice to unblock the threshold query.}

      Let the $x+1+r$th method execution on $j$ be the first update on $j$
      that updates its state to some $s^{x+1+r}_j \geq s'_a$, for some
      activation state $s'_a \in S_a$.

      Choose $n = r+1$.

      We have to show that, for all $i$ in the range $0 \leq i \leq
      r$, the threshold query $t^{x+1+i}_j(\mathcal{S})$ returns
      $\block$, and that for all $i \geq r+1$, the threshold query
      $t^{x+1+i}_j(\mathcal{S})$ returns $S_a$.

      For the former, since the $x+1+r$th method execution on $j$ is the
      first one that updates its state to $s^{x+1+r}_j \geq s'_a$, we have
      by Definition~\ref{def:cvrdt-with-threshold-queries} that for all $i$
      in the range $0 \leq i \leq r$, the threshold query
      $t^{x+1+i}_j(\mathcal{S})$ returns $\block$.

      For the latter, since $s^{x+1+r}_j \geq s'_a$, by
      Definition~\ref{def:cvrdt-with-threshold-queries} we have that
      $t^{x+1+r+1}_j(\mathcal{S})$ returns $S_a$, and by
      part~\ref{thm:this-replica} of the theorem, we have that for $i \geq
      r+1$, subsequent executions $t^{x+1+i}_j(\mathcal{S})$ at replica $j$
      will also return $S_a$, and so the case holds.
    \end{itemize}

  \item $s^k_i \nleq s^x_j$ and $s^x_j \nleq s^k_i$.

    (That is, replica $i$'s state after the $k$th method execution on $i$
    is \emph{not comparable} to replica $j$'s state after the $x$th method
    execution on $j$.)

    Similar to the previous case.
  \end{itemize}
\end{proof}



\chapter{Proofs}\label{app:proofs}

\section{Proof of Lemma~\ref{lem:lvars-permutability}}\label{section:lvars-permutability-proof}
\begin{proof}
  Consider an arbitrary permutation $\pi$.  For
  part~\ref{thm:permutable-reduction-transitions}, we have to show
  that if $\conf \parstepsto \conf'$ then $\pi(\conf) \parstepsto
  \pi(\conf')$, and that if $\pi(\conf) \parstepsto \pi(\conf')$ then
  $\conf \parstepsto \conf'$.

  For the forward direction of
  part~\ref{thm:permutable-reduction-transitions}, suppose $\conf
  \parstepsto \conf'$.  We have to show that $\pi(\conf) \parstepsto
  \pi(\conf')$.  We proceed by cases on the rule by which $\conf$
  steps to $\conf'$.

  \begin{itemize}
    \item Case {\sc E-Beta}: $\conf =
      \config{S}{\app{(\lam{x}{e})}{v}}$, and $\conf' =
      \config{S}{\subst{e}{x}{v}}$.

      To show: $\pi(\config{S}{\app{(\lam{x}{e})}{v}}) \parstepsto
      \pi(\config{S}{\subst{e}{x}{v}})$.

      By Definitions~\ref{def:lvars-permutation-configuration}
      and~\ref{def:lvars-permutation-expression}, $\pi(\conf) =
      \config{\pi(S)}{\app{(\lam{x}{\pi(e)})}{\pi(v)}}$.

      By {\sc E-Beta},
      $\config{\pi(S)}{\app{(\lam{x}{\pi(e)})}{\pi(v)}}$ steps to
      $\config{\pi(S)}{\subst{\pi(e)}{x}{\pi(v)}}$.

      By Definition~\ref{def:lvars-permutation-expression},
      $\config{\pi(S)}{\subst{\pi(e)}{x}{\pi(v)}}$ is equal to
      $\config{\pi(S)}{\pi(\subst{e}{x}{v})}$.

      Hence $\config{\pi(S)}{\app{(\lam{x}{\pi(e)})}{\pi(v)}}$ steps
      to $\config{\pi(S)}{\pi(\subst{e}{x}{v})}$,

      which is equal to $\pi(\config{S}{\subst{e}{x}{v}})$ by
      Definition~\ref{def:lvars-permutation-configuration}.  Hence the
      case is satisfied.

    \item Case {\sc E-New}: $\conf = \config{S}{\NEW}$, and $\conf' =
      \config{\extSRaw{S}{l}{\bot}}{l}$.

      To show: $\pi(\config{S}{\NEW}) \parstepsto
      \pi(\config{\extSRaw{S}{l}{\bot}}{l})$.

      By Definitions~\ref{def:lvars-permutation-configuration}
      and~\ref{def:lvars-permutation-expression}, $\pi(\conf) =
      \config{\pi(S)}{\NEW}$.

      By {\sc E-New}, $\config{\pi(S)}{\NEW}$ steps to
      $\config{\extSRaw{(\pi(S))}{l'}{\bot}}{l'}$, where $l' \notin
      \dom{\pi(S)}$.
      
      It remains to show that
      $\config{\extSRaw{(\pi(S))}{l'}{\bot}}{l'}$ is equal to
      $\pi(\config{\extSRaw{S}{l}{\bot}}{l})$.

      By Definition~\ref{def:lvars-permutation-configuration},
      $\pi(\config{\extSRaw{S}{l}{\bot}}{l})$ is equal to
      $\config{\pi(\extSRaw{S}{l}{\bot})}{\pi(l)}$,

      which is equal to
      $\config{\extSRaw{(\pi(S))}{\pi(l)}{\bot}}{\pi(l)}$.

      So, we have to show that
      $\config{\extSRaw{(\pi(S))}{l'}{\bot}}{l'}$ is equal to
      $\config{\extSRaw{(\pi(S))}{\pi(l)}{\bot}}{\pi(l)}$.  Since we
      know (from the side condition of {\sc E-New}) that $l \notin
      \dom{S}$, it follows that $\pi(l) \notin \pi(\dom{S})$.
      Therefore, in $\config{\extSRaw{(\pi(S))}{l'}{\bot}}{l'}$, we
      can $\alpha$-rename $l'$ to $\pi(l)$, and so the two
      configurations are equal and the case is satisfied.

    \item Case {\sc E-Put}: $\conf = \config{S}{\putexp{l}{d_2}}$, and
      $\conf' = \config{\extSRaw{S}{l}{\userlub{d_1}{d_2}}}{\unit}$.

      To show: $\pi(\config{S}{\putexp{l}{d_2}}) \parstepsto
      \pi(\config{\extSRaw{S}{l}{\userlub{d_1}{d_2}}}{\unit})$.

      By Definitions~\ref{def:lvars-permutation-configuration}
      and~\ref{def:lvars-permutation-expression}, $\pi(\conf) =
      \config{\pi(S)}{\putexp{\pi(l)}{d_2}}$.

      By {\sc E-Put}, $\config{\pi(S)}{\putexp{\pi(l)}{d_2}}$ steps to
      $\config{\extSRaw{(\pi(S))}{\pi(l)}{\userlub{d_1}{d_2}}}{\unit}$,

      since $S(l) = (\pi(S))(\pi(l)) = d_1$.

      It remains to show that
      $\config{\extSRaw{(\pi(S))}{\pi(l)}{\userlub{d_1}{d_2}}}{\unit}$
      is equal to
      $\pi(\config{\extSRaw{S}{l}{\userlub{d_1}{d_2}}}{\unit})$.

      By Definitions~\ref{def:lvars-permutation-configuration}
      and~\ref{def:lvars-permutation-expression},
      $\pi(\config{\extSRaw{S}{l}{\userlub{d_1}{d_2}}}{\unit})$ is
      equal to
      $\config{\extSRaw{(\pi(S))}{\pi(l)}{\userlub{d_1}{d_2}}}{\unit}$,
      and so the two configurations are equal and the case is
      satisfied.

    \item Case {\sc E-Put-Err}: $\conf = \config{S}{\putexp{l}{d_2}}$,
      and $\conf' = \error$.

      To show: $\pi(\config{S}{\putexp{l}{d_2}}) \parstepsto
      \pi(\error)$.

      By Definitions~\ref{def:lvars-permutation-configuration}
      and~\ref{def:lvars-permutation-expression}, $\pi(\conf) =
      \config{\pi(S)}{\putexp{\pi(l)}{d_2}}$.

      By {\sc E-Put-Err}, $\config{\pi(S)}{\putexp{\pi(l)}{d_2}}$
      steps to $\error$,

      since $S(l) = (\pi(S))(\pi(l)) = d_1$.

      Since $\pi(\error) = \error$ by
      Definition~\ref{def:lvars-permutation-configuration}, the case
      is complete.

    \item Case {\sc E-Get}: $\conf = \config{S}{\getexp{l}{T}}$, and
      $\conf' = \config{S}{d_2}$.

      To show: $\pi(\config{S}{\getexp{l}{T}}) \parstepsto
      \pi(\config{S}{d_2})$.

      By Definitions~\ref{def:lvars-permutation-configuration}
      and~\ref{def:lvars-permutation-expression}, $\pi(\conf) =
      \config{\pi(S)}{\getexp{\pi(l)}{T}}$.

      By {\sc E-Get}, $\config{\pi(S)}{\getexp{\pi(l)}{T}}$ steps to
      $\config{\pi(S)}{d_2}$,

      since $S(l) = (\pi(S))(\pi(l)) = d_1$.

      By Definitions~\ref{def:lvars-permutation-configuration}
      and~\ref{def:lvars-permutation-expression},
      $\pi(\config{S}{d_2}) \config{\pi(S)}{d_2}$.  Therefore the case
      is complete.
  \end{itemize}

  For the reverse direction of
  part~\ref{thm:permutable-reduction-transitions}, suppose $\pi(\conf)
  \parstepsto \pi(\conf')$.  We have to show that $\conf \parstepsto
  \conf'$.

  We know from the forward direction of the proof that for all
  configurations $\conf$ and $\conf'$ and permutations $\pi$, if
  $\conf \parstepsto \conf'$ then $\pi(\conf) \parstepsto
  \pi(\conf')$.  Hence since $\pi(\conf) \parstepsto \pi(\conf')$, and
  since $\piinv$ is also a permutation, we have that
  $\piinv(\pi(\conf)) \parstepsto \piinv(\pi(\conf'))$.  Since
  $\piinv(\pi(l)) = l$ for every $l \in \Loc$, and that property lifts
  to configurations as well, we have that $\conf \parstepsto \conf'$.

  \lk{Is the above enough of a proof?}

  For the forward direction of
  part~\ref{thm:permutable-context-transitions}, suppose $\conf
  \ctxstepsto \conf'$.  We have to show that $\pi(\conf) \ctxstepsto
  \pi(\conf')$.

  By inspection of the operational semantics, $\conf$ must be of the
  form $\config{S}{\E{e}}$, and $\conf'$ must be of the form
  $\config{S'}{\E{e'}}$.  Hence we have to show that
  $\pi(\config{S}{\E{e}}) \ctxstepsto \pi(\config{S'}{\E{e'}})$.

  By Definition~\ref{def:lvars-permutation-configuration},
  $\pi(\config{S}{\E{e}})$ is equal to $\config{\pi(S)}{\pi(\E{e})}$.

  Also by Definition~\ref{def:lvars-permutation-configuration},
  $\pi(\config{S'}{\E{e'}})$ is equal to
  $\config{\pi(S')}{\pi(\E{e'})}$.

  Furthermore, $\config{\pi(S)}{\pi(\E{e})}$ is equal to
  $\config{\pi(S)}{\evalctxt{(\pi(E))}{\pi(e)}}$ and
  $\config{\pi(S')}{\pi(\E{e'})}$ is equal to
  $\config{\pi(S')}{\evalctxt{(\pi(E))}{\pi(e')}}$.

  So we have to show that
  $\config{\pi(S)}{\evalctxt{(\pi(E))}{\pi(e)}} \ctxstepsto
  \config{\pi(S')}{\evalctxt{(\pi(E))}{\pi(e')}}$.

  From the premise of {\sc E-Eval-Ctxt}, $\config{S}{e} \parstepsto
  \config{S'}{e'}$.  Hence, by
  part~\ref{thm:permutable-reduction-transitions}, $\pi(\config{S}{e})
  \parstepsto \pi(\config{S'}{e'})$.  By
  Definition~\ref{def:lvars-permutation-configuration},
  $\pi(\config{S}{e})$ is equal to $\config{\pi(S)}{\pi(e)}$ and
  $\pi(\config{S'}{e'})$ is equal to $\config{\pi(S')}{\pi(e')}$.

  Hence $\config{\pi(S)}{\pi(e)} \parstepsto
  \config{\pi(S')}{\pi(e')}$.

  Therefore, by {\sc E-Eval-Ctxt}, $\config{\pi(S)}{\E{\pi(e)}}
  \ctxstepsto \config{\pi(S')}{\E{\pi(e')}}$ for all evaluation
  contexts $E$.

  In particular, it is true that
  $\config{\pi(S)}{\evalctxt{(\pi(E))}{\pi(e)}} \ctxstepsto
  \config{\pi(S')}{\evalctxt{(\pi(E))}{\pi(e')}}$, as we were required
  to show.

  For the reverse direction of
  part~\ref{thm:permutable-context-transitions}, suppose $\pi(\conf)
  \ctxstepsto \pi(\conf')$.  We have to show that $\conf \ctxstepsto
  \conf'$.

  We know from the forward direction of the proof that for all
  configurations $\conf$ and $\conf'$ and permutations $\pi$, if
  $\conf \ctxstepsto \conf'$ then $\pi(\conf) \ctxstepsto
  \pi(\conf')$.  Hence since $\pi(\conf) \ctxstepsto \pi(\conf')$, and
  since $\piinv$ is also a permutation, we have that
  $\piinv(\pi(\conf)) \ctxstepsto \piinv(\pi(\conf'))$.  Since
  $\piinv(\pi(l)) = l$ for every $l \in \Loc$, and that property lifts
  to configurations as well, we have that $\conf \ctxstepsto \conf'$.

  \lk{Is the above enough of a proof?}
\end{proof}


\section{Proof of Lemma~\ref{lem:lvars-internal-determinism}}\label{section:lvars-internal-determinism-proof}
\begin{proof}
  Suppose $\conf \parstepsto \conf'$ and $\conf \parstepsto \conf''$.

  We have to show that there is a permutation $\pi$ such that $\conf'
  = \pi(\conf'')$.

  The proof is by cases on the rule by which $\conf$ steps to
  $\conf'$.

  \begin{itemize}

  \item Case {\sc E-Beta}:

    Given: $\config{S}{\app{(\lam{x}{e})}{v}} \parstepsto
    \config{S}{\subst{e}{x}{v}}$, and
    $\config{S}{\app{(\lam{x}{e})}{v}} \parstepsto \conf''$.

    To show: There exists a $\pi$ such that
    $\config{S}{\subst{e}{x}{v}} = \pi(\conf'')$.

    By inspection of the operational semantics, the only reduction
    rule by which $\config{S}{\app{(\lam{x}{e})}{v}}$ can step is {\sc
      E-Beta}.

    Hence $\conf'' = \config{S}{\subst{e}{x}{v}}$, and the case is
    satisfied by choosing $\pi$ to be the identity function.

  \item Case {\sc E-New}: 

    Given: $\config{S}{\NEW} \parstepsto
    \config{\extSRaw{S}{l}{\bot}}{l}$, and $\config{S}{\NEW}
    \parstepsto \conf''$.

    To show: There exists a $\pi$ such that
    $\config{\extSRaw{S}{l}{\bot}}{l} = \pi(\conf'')$.

    By inspection of the operational semantics, the only reduction
    rule by which $\config{S}{\NEW}$ can step is {\sc E-New}.

    Hence $\conf'' = \config{\extSRaw{S}{l'}{\bot}}{l'}$.

    Since, by the side condition of {\sc E-New}, neither $l$ nor $l'$
    occur in $\dom{S}$, the case is satisfied by choosing $\pi$ to be
    the permutation that maps $l'$ to $l$ and is the identity on every
    other element of $\Loc$.

  \item Case {\sc E-Put}:

    Given: $\config{S}{\putexp{l}{d_2}} \parstepsto
    \config{\extSRaw{S}{l}{\userlub{d_1}{d_2}}}{\unit}$, and
    $\config{S}{\putexp{l}{d_2}} \parstepsto \conf''$.

    To show: There exists a $\pi$ such that
    $\config{\extSRaw{S}{l}{\userlub{d_1}{d_2}}}{\unit} =
    \pi(\conf'')$.

    By inspection of the operational semantics, and since
    $\userlub{d_1}{d_2} \neq \top$ (from the premise of {\sc E-Put}),
    the only reduction rule by which $\config{S}{\putexp{l}{d_2}}$ can
    step is {\sc E-Put}.

    Hence $\conf'' =
    \config{\extSRaw{S}{l}{\userlub{d_1}{d_2}}}{\unit}$, and the case
    is satisfied by choosing $\pi$ to be the identity function.

  \item Case {\sc E-Put-Err}:

    Given: $\config{S}{\putexp{l}{d_2}} \parstepsto \error$, and
    $\config{S}{\putexp{l}{d_2}} \parstepsto \conf''$.

    To show: There exists a $\pi$ such that $\error = \pi(\conf'')$.

    By inspection of the operational semantics, and since
    $\userlub{d_1}{d_2} = \top$ (from the premise of {\sc E-Put-Err}),
    the only reduction rule by which $\config{S}{\putexp{l}{d_2}}$ can
    step is {\sc E-Put-Err}.

    Hence $\conf'' = \error$, and the case is satisfied by choosing
    $\pi$ to be the identity function.

  \item Case {\sc E-Get}:

    Given: $\config{S}{\getexp{l}{T}} \parstepsto \config{S}{d_2}$,
    and $\config{S}{\getexp{l}{T}} \parstepsto \conf''$.

    To show: There exists a $\pi$ such that $\config{S}{d_2} =
    \pi(\conf'')$.

    By inspection of the operational semantics, the only reduction
    rule by which $\config{S}{\getexp{l}{T}}$ can step is {\sc
      E-Get}.

    Hence $\conf'' = \config{S}{d_2}$, and the case is satisfied by
    choosing $\pi$ to be the identity function.

  \end{itemize}
\end{proof}



\section{Proof of Lemma~\ref{lem:lvars-monotonicity}}\label{section:lvars-monotonicity-proof}
\begin{proof}
  Suppose $\config{S}{e} \parstepsto \config{S'}{e'}$.  We are
  required to show that $\leqstore{S}{S'}$.  The proof is by cases on
  the rule by which $\config{S}{e}$ steps to $\config{S'}{e'}$.

  \begin{itemize}
    \item Case {\sc E-Beta}:

      Immediate by the definition of $\leqstore{}{}$, since $S$ does
      not change.

    \item Case {\sc E-New}:

      Given: $\config{S}{\NEW} \parstepsto
      \config{\extSRaw{S}{l}{\bot}}{l}$.

      To show: $\leqstore{S}{\extSRaw{S}{l}{\bot}}$.

      By Definition~\ref{def:lvars-leqstore}, we have to show that
      $\dom{S} \subseteq \dom{\extSRaw{S}{l}{\bot}}$ and
      that for all $l' \in \dom{S}$, $S(l') \userleq
      (\extSRaw{S}{l}{\bot})(l')$.

      By definition, a store update operation on $S$ can only either
      update an existing binding in $S$ or extend $S$ with a new
      binding.  Hence $\dom{S} \subseteq \dom{\extSRaw{S}{l}{\bot}}$.

      From the side condition of {\sc E-New}, $l \notin \dom{S}$.
      Hence $\extSRaw{S}{l}{\bot}$ adds a new binding for $l$ in $S$.

      Hence $\extSRaw{S}{l}{\bot}$ does not update any existing
      bindings in $S$.

      Hence, for all $l' \in \dom{S}, S(l') \userleq
      (\extSRaw{S}{l}{\bot})(l')$.

      Therefore $\leqstore{S}{\extSRaw{S}{l}{\bot}}$, as
      required.

    \item Case {\sc E-Put}:

      Given: $\config{S}{\putexp{l}{d_2}} \parstepsto
      \config{\extSRaw{S}{l}{\userlub{d_1}{d_2}}}{\unit}$.

      To show: $\leqstore{S}{\extSRaw{S}{l}{\userlub{d_1}{d_2}}}$.

      By Definition~\ref{def:lvars-leqstore}, we have to show that
      $\dom{S} \subseteq \dom{\extSRaw{S}{l}{\userlub{d_1}{d_2}}}$ and
      that for all $l' \in \dom{S}$, $S(l') \userleq
      (\extSRaw{S}{l}{\userlub{d_1}{d_2}})(l')$.

      By definition, a store update operation on $S$ can only either
      update an existing binding in $S$ or extend $S$ with a new
      binding.  Hence $\dom{S} \subseteq
      \dom{\extSRaw{S}{l}{\userlub{d_1}{d_2}}}$.

      From the premises of {\sc E-Put}, $S(l) = d_1$.  Therefore $l
      \in \dom{S}$.

      Hence $\extSRaw{S}{l}{\userlub{d_1}{d_2}}$ updates the existing
      binding for $l$ in $S$ from $d_1$ to $\userlub{d_1}{d_2}$.

      By the definition of $\userlub{}{}$, $d_1 \userleq
      (\userlub{d_1}{d_2})$.  $\extSRaw{S}{l}{\userlub{d_1}{d_2}}$
      does not update any other bindings in $S$, hence, for all $l'
      \in \dom{S}, S(l') \userleq
      (\extSRaw{S}{l}{\userlub{d_1}{d_2}})(l')$.

      Hence $\leqstore{S}{\extSRaw{S}{l}{\userlub{d_1}{d_2}}}$, as
      required.

    \item Case {\sc E-Put-Err}:

      Given: $\config{S}{\putexp{l}{d_2}} \parstepsto \error$.

      By the definition of $\error$, $\error$ is equal to
      $\config{\topS}{e}$ for all $e$.

      To show: $\leqstore{S}{\topS}$.

      Immediate by the definition of $\leqstore{}{}$.

    \item Case {\sc E-Get}:

      Immediate by the definition of $\leqstore{}{}$, since $S$ does
      not change.

  \end{itemize}

\end{proof}


\section{Proof of Lemma~\ref{lem:lvars-independence}}\label{section:lvars-independence-proof}
\begin{proof}
  Consider arbitrary $S''$ such that $S''$ is non-conflicting with
  $\config{S}{e} \parstepsto \config{S'}{e'}$ and $\lubstore{S'}{S''}
  \neq \topS$.

  To show: $\config{\lubstore{S}{S''}}{e} \parstepsto
  \config{\lubstore{S'}{S''}}{e'}$.

  The proof is by induction on the derivation of $\config{S}{e}
  \parstepsto \config{S'}{e'}$, by cases on the last rule in the
  derivation.  In every case we may assume that $\config{S'}{e'} \neq
  \error$.  Since $\config{S'}{e'} \neq \error$, we do not need to
  consider the {\sc E-Put-Err} rule.
  \begin{itemize}

    \item Case {\sc E-Eval-Ctxt}:

      Given: $\config{S}{\E{e}} \parstepsto \config{S'}{\E{e'}}$.

      To show: $\config{\lubstore{S}{S''}}{\E{e}} \parstepsto
      \config{\lubstore{S'}{S''}}{\E{e'}}$.

      From the premise of {\sc E-Eval-Ctxt}, we have that
      $\config{S}{e} \parstepsto \config{S'}{e'}$.

      Therefore, by IH, we have that $\config{\lubstore{S}{S''}}{e}
      \parstepsto \config{\lubstore{S'}{S''}}{e'}$.

      Therefore, by {\sc E-Eval-Ctxt}, we have that
      $\config{\lubstore{S}{S''}}{\E{e}} \parstepsto
      \config{\lubstore{S'}{S''}}{\E{e'}}$, as we were required to
      show.

    \item Case {\sc E-Beta}:

      Given: $\config{S}{\app{(\lam{x}{e})}{v}} \parstepsto
      \config{S}{\subst{e}{x}{v}}$.

      To show: $\config{\lubstore{S}{S''}}{\app{(\lam{x}{e})}{v}}
      \parstepsto \config{\lubstore{S}{S''}}{\subst{e}{x}{v}}$.

      Immediate by {\sc E-Beta}.

    \item Case {\sc E-New}:

      Given: $\config{S}{\NEW} \parstepsto
      \config{\extSRaw{S}{l}{\bot}}{l}$.

      To show: $\config{\lubstore{S}{S''}}{\NEW} \parstepsto
      \config{\lubstore{(\extSRaw{S}{l}{\bot})}{S''}}{l}$.

      By {\sc E-New}, we have that $\config{\lubstore{S}{S''}}{\NEW}
      \parstepsto \config{\extSRaw{(\lubstore{S}{S''})}{l'}{\bot}}{l'}$,
      where $l' \notin \dom{\lubstore{S}{S''}}$.

      By assumption, $S''$ is non-conflicting with $\config{S}{\NEW}
      \parstepsto \config{\extSRaw{S}{l}{\bot}}{l}$.
 
      Therefore $l \notin \dom{S''}$.

      From the side condition of {\sc E-New}, $l \notin \dom{S}$.

      Therefore $l \notin \dom{\lubstore{S}{S''}}$.

      Therefore, in
      $\config{\extSRaw{(\lubstore{S}{S''})}{l'}{\bot}}{l'}$, we can
      $\alpha$-rename $l'$ to $l$, \\ resulting in
      $\config{\extSRaw{(\lubstore{S}{S''})}{l}{\bot}}{l}$.

      Therefore $\config{\lubstore{S}{S''}}{\NEW} \parstepsto
      \config{\extSRaw{(\lubstore{S}{S''})}{l}{\bot}}{l}$.

      Note that:
      \begin{align*}
        \extSRaw{(\lubstore{S}{S''})}{l}{\bot} &=
        \lubstore{\extSRaw{S}{l}{\bot}}{\extSRaw{S''}{l}{\bot}} \\ &=
        \lubstore{\lubstore{S}{\store{\storebindingRaw{l}{\bot}}}}{\lubstore{S''}{\store{\storebindingRaw{l}{\bot}}}}
        \\ &=
        \lubstore{\lubstore{S}{\store{\storebindingRaw{l}{\bot}}}}{S''}
        \\ &= \lubstore{\extSRaw{S}{l}{\bot}}{S''}.
      \end{align*}
      Therefore $\config{\lubstore{S}{S''}}{\NEW} \parstepsto
      \config{\lubstore{\extSRaw{S}{l}{\bot}}{S''}}{l}$, as we were
      required to show.

    \item Case {\sc E-Put}:

      Given: $\config{S}{\putexp{l}{d_2}} \parstepsto
      \config{\extSRaw{S}{l}{d_2}}{\unit}$.

      To show: $\config{\lubstore{S}{S''}}{\putexp{l}{d_2}}
      \parstepsto
      \config{\lubstore{\extSRaw{S}{l}{d_2}}{S''}}{\unit}$.

      We will first show that

      $\config{\lubstore{S}{S''}}{\putexp{l}{d_2}} \parstepsto
      \config{\extSRaw{(\lubstore{S}{S''})}{l}{d_2}}{\unit}$

      and then show why this is sufficient.

      We proceed by cases on $l$:

      \begin{itemize}
        \item $l \notin \dom{S''}$:

          By assumption, $\lubstore{\extSRaw{S}{l}{d_2}}{S''} \neq
          \topS$.

          By Lemma~\ref{lem:lvars-monotonicity},
          $\leqstore{S}{\extSRaw{S}{l}{d_2}}$.

          Hence $\lubstore{S}{S''} \neq \topS$.

          Therefore, by Definition~\ref{def:lvars-lubstore},
          $(\lubstore{S}{S''})(l) = S(l)$.

          From the premises of {\sc E-Put}, $S(l) = d_1$.

          Hence $(\lubstore{S}{S''})(l) = d_1$.

          From the premises of {\sc E-Put}, $d_2 = \userlub{d_1}{d_2}$
          and $d_2 \neq \top$.

          Therefore, by {\sc E-Put}, we have:
          $\config{\lubstore{S}{S''}}{\putexp{l}{d_2}} \parstepsto
          \config{\extSRaw{(\lubstore{S}{S''})}{l}{d_2}}{\unit}$.

        \item $l \in \dom{S''}$:

          By assumption, $\lubstore{\extSRaw{S}{l}{d_2}}{S''} \neq
          \topS$.

          By Lemma~\ref{lem:lvars-monotonicity},
          $\leqstore{S}{\extSRaw{S}{l}{d_2}}$.

          Hence $\lubstore{S}{S''} \neq \topS$.

          Therefore $(\lubstore{S}{S''})(l) = \userlub{S(l)}{S''(l)}$.

          From the premises of {\sc E-Put}, $S(l) = d_1$.
          
          Hence $(\lubstore{S}{S''})(l) = d'_1$, where $d_1 \userleq
          d'_1$.

          From the premises of {\sc E-Put}, $d_2 =
          \userlub{d_1}{d_2}$.

          Let $d'_2 = \userlub{d'_1}{d_2}$.

          Hence $d_2 \userleq d'_2$.

          By assumption, $\lubstore{\extSRaw{S}{l}{d_2}}{S''} \neq
          \topS$.

          Therefore, by Definition~\ref{def:lvars-lubstore},
          $\lubstore{d_2}{S''(l)} \neq \top$.

          Note that:
          \begin{align*}
            \top &\neq \lubstore{d_2}{S''(l)} \\ &=
            \userlub{\userlub{d_1}{d_2}}{S''(l)} \\ &=
            \userlub{\userlub{S(l)}{d_2}}{S''(l)} \\ &=
            \userlub{\userlub{S(l)}{S''(l)}}{d_2} \\ &=
            \userlub{(\lubstore{S}{S''})(l)}{d_2} \\ &=
            \userlub{d'_1}{d_2} \\ &= d'_2. \\
          \end{align*}
          Hence $d'_2 \neq \top$.

          Hence $(\lubstore{S}{S''})(l) = d'_1$ and $d'_2 =
          \userlub{d'_1}{d_2}$ and $d'_2 \neq \top$.

          Therefore, by {\sc E-Put} we have:
          $\config{\lubstore{S}{S''}}{\putexp{l}{d_2}} \parstepsto
          \config{\extSRaw{(\lubstore{S}{S''})}{l}{d'_2}}{\unit}$.

          \lk{If we really wanted to be pedantic here, we'd actually
            prove that the stores are equal.  I'm assuming that if I
            can show that $\extSRaw{(\lubstore{S}{S''})}{l}{d'_2}$ and
            $\extSRaw{(\lubstore{S}{S''})}{l}{d_2}$ bind $l$ to the
            same value, then it will be obvious that they're equal.}

          Note that:
          \begin{align*}
            (\extSRaw{(\lubstore{S}{S''})}{l}{d'_2})(l) &=
            \userlub{(\lubstore{S}{S''})(l)}{(\store{\storebindingRaw{l}{d'_2}})(l)}
            \\ &= \userlub{d'_1}{d'_2} \\ &=
            \userlub{d'_1}{\userlub{d'_1}{d_2}} \\ &=
            \userlub{d'_1}{d_2}
          \end{align*}
          and
          \begin{align*}
            (\extSRaw{(\lubstore{S}{S''})}{l}{d_2})(l) &=
            \userlub{(\lubstore{S}{S''})(l)}{(\store{\storebindingRaw{l}{d_2}})(l)}
            \\ &= \userlub{d'_1}{d_2} \\ &=
            \userlub{d'_1}{\userlub{d_1}{d_2}} \\ &=
            \userlub{d'_1}{d_2} & \textrm{(since $d_1 \userleq
              d'_1$).}
          \end{align*}
          Therefore $\extSRaw{(\lubstore{S}{S''})}{l}{d'_2} =
          \extSRaw{(\lubstore{S}{S''})}{l}{d_2}$.

          Therefore, $\config{\lubstore{S}{S''}}{\putexp{l}{d_2}}
          \parstepsto
          \config{\extSRaw{(\lubstore{S}{S''})}{l}{d_2}}{\unit}$.
      \end{itemize}

      Note that:
      \begin{align*}
        \extSRaw{(\lubstore{S}{S''})}{l}{d_2} &=
        \lubstore{\extSRaw{S}{l}{d_2}}{\extSRaw{S''}{l}{d_2}} \\ &=
        \lubstore{\lubstore{S}{\store{\storebindingRaw{l}{d_2}}}}{\lubstore{S''}{\store{\storebindingRaw{l}{d_2}}}}
        \\ &=
        \lubstore{\lubstore{S}{\store{\storebindingRaw{l}{d_2}}}}{S''}
        \\ &= \lubstore{\extSRaw{S}{l}{d_2}}{S''}.
      \end{align*}
      Therefore $\config{\lubstore{S}{S''}}{\putexp{l}{d_2}}
      \parstepsto
      \config{\lubstore{\extSRaw{S}{l}{d_2}}{S''}}{\unit}$, as we were
      required to show.

    \item Case {\sc E-Get}:

      Given: $\config{S}{\getexp{l}{T}} \parstepsto \config{S}{d_2}$.

      To show: $\config{\lubstore{S}{S''}}{\getexp{l}{T}} \parstepsto
      \config{\lubstore{S}{S''}}{d_2}$.

      From the premises of {\sc E-Get}, $S(l) = d_1$ and $\incomp{T}$
      and $d_2 \in T$ and $d_2 \userleq d_1$.

      By assumption, $\lubstore{S}{S''} \neq \topS$.

      Hence $(\lubstore{S}{S''}) = d'_1$, where $d_1 \userleq d'_1$.

      By the transitivity of $\userleq$, $d_2 \userleq d'_1$.

      Hence, $S(l) = d'_1$ and $\incomp{T}$ and $d_2 \in T$ and $d_2
      \userleq d'_1$.

      Therefore, by {\sc E-Get},

      $\config{\lubstore{S}{S''}}{\getexp{l}{T}} \parstepsto
      \config{\lubstore{S}{S''}}{d_2}$,

      as we were required to show.
  \end{itemize}
\end{proof}


\section{Proof of Lemma~\ref{lem:lvars-clash}}\label{section:lvars-clash-proof}
\begin{proof}
  Consider arbitrary $S''$ such that $S''$ is non-conflicting with
  $\config{S}{e} \parstepsto \config{S'}{e'}$ and $\lubstore{S'}{S''}
  = \topS$.

  To show: $\config{\lubstore{S}{S''}}{e} \parstepsto \error$.

  The proof is by induction on the derivation of $\config{S}{e}
  \parstepsto \config{S'}{e'}$, by cases on the last rule in the
  derivation.  In every case we may assume that $\config{S'}{e'} \neq
  \error$.  Since $\config{S'}{e'} \neq \error$, we do not need to
  consider the {\sc E-Put-Err} rule.

  \begin{itemize}

    \item Case {\sc E-Eval-Ctxt}:

      Given: $\config{S}{\E{e}} \parstepsto \config{S'}{\E{e'}}$.

      To show: $\config{\lubstore{S}{S''}}{\E{e}} \parstepsto^i
      \error$, where $i \leq 1$.

      From the premise of {\sc E-Eval-Ctxt}, we have that
      $\config{S}{e} \parstepsto \config{S'}{e'}$.

      Therefore, by IH, we have that $\config{\lubstore{S}{S''}}{e}
      \parstepsto^{i'} \error$, where $i' \leq 1$.

      We proceed by cases on $i'$:

      \begin{itemize}
        \item $i' = 0$:

          In this case, $\config{\lubstore{S}{S''}}{e} = \error$.

          Hence, by the definition of $\error$, $\lubstore{S}{S''} =
          \topS$.

          Hence $\config{\lubstore{S}{S''}}{\E{e}} = \error$.

          Hence $\config{\lubstore{S}{S''}}{\E{e}} \parstepsto^i
          \error$, with $i = 0$.

        \item $i' = 1$:

          In this case, $\config{\lubstore{S}{S''}}{e} \parstepsto
          \error$.

          By the definition of $\error$, $\error =
          \config{\topS}{e''}$ for any $e''$.

          Hence $\config{\lubstore{S}{S''}}{e} \parstepsto
          \config{\topS}{e''}$.

          Hence, by {\sc E-Eval-Ctxt},
          $\config{\lubstore{S}{S''}}{\E{e}} \parstepsto
          \config{\topS}{\E{e''}}$.

          By the definition of $\error$, $\config{\topS}{\E{e''}} =
          \error$.

          Hence $\config{\lubstore{S}{S''}}{\E{e}} \parstepsto
          \error$.

          Hence $\config{\lubstore{S}{S''}}{\E{e}} \parstepsto^i
          \error$, with $i = 1$.

      \end{itemize}

    \item Case {\sc E-Beta}:

      Given: $\config{S}{\app{(\lam{x}{e})}{v}} \parstepsto
      \config{S}{\subst{e}{x}{v}}$.

      To show: $\config{\lubstore{S}{S''}}{\app{(\lam{x}{e})}{v}}
      \parstepsto^i \error$, where $i \leq 1$.

      By assumption, $\lubstore{S}{S''} = \topS$.

      Hence, by the definition of $\error$,
      $\config{\lubstore{S}{S''}}{\app{(\lam{x}{e})}{v}} = \error$.

      Hence $\config{\lubstore{S}{S''}}{\app{(\lam{x}{e})}{v}}
      \parstepsto^i \error$, with $i = 0$.

    \item Case {\sc E-New}:

      Given: $\config{S}{\NEW} \parstepsto
      \config{\extSRaw{S}{l}{\bot}}{l}$.

      To show: $\config{\lubstore{S}{S''}}{\NEW} \parstepsto^i
      \error$, where $i \leq 1$.

      By {\sc E-New}, $\config{\lubstore{S}{S''}}{\NEW} \parstepsto
      \config{\extSRaw{(\lubstore{S}{S''})}{l'}{\bot}}{l'}$, where $l'
      \notin \dom{\lubstore{S}{S''}}$.

      By assumption, $S''$ is non-conflicting with $\config{S}{\NEW}
      \parstepsto \config{\extSRaw{S}{l}{\bot}}{l}$.
 
      Therefore $l \notin \dom{S''}$.

      From the side condition of {\sc E-New}, $l \notin \dom{S}$.

      Therefore $l \notin \dom{\lubstore{S}{S''}}$.

      Therefore, in
      $\config{\extSRaw{(\lubstore{S}{S''})}{l'}{\bot}}{l'}$, we can
      $\alpha$-rename $l'$ to $l$, \\ resulting in
      $\config{\extSRaw{(\lubstore{S}{S''})}{l}{\bot}}{l}$.

      Therefore $\config{\lubstore{S}{S''}}{\NEW} \parstepsto
      \config{\extSRaw{(\lubstore{S}{S''})}{l}{\bot}}{l}$.

      By assumption, $\lubstore{\extSRaw{S}{l}{\bot}}{S''}
      = \topS$.

      Note that:
      \begin{align*}
        \topS &= \lubstore{\extSRaw{S}{l}{\bot}}{S''} \\ &=
        \lubstore{\lubstore{S}{\store{\storebindingRaw{l}{\bot}}}}{S''}
        \\ &=
        \lubstore{\lubstore{S}{S''}}{\store{\storebindingRaw{l}{\bot}}}
        \\ &=
        \lubstore{(\lubstore{S}{S''})}{\store{\storebindingRaw{l}{\bot}}}
        \\ &= \extSRaw{(\lubstore{S}{S''})}{l}{\bot} .
      \end{align*}

      Hence $\config{\lubstore{S}{S''}}{\NEW} \parstepsto
      \config{\topS}{l}$.

      Hence, by the definition of $\error$,
      $\config{\lubstore{S}{S''}}{\NEW} \parstepsto \error$.

      Hence $\config{\lubstore{S}{S''}}{\NEW} \parstepsto^i \error$,
      with $i = 1$.

    \item Case {\sc E-Put}:

      Given: $\config{S}{\putexp{l}{d_2}} \parstepsto
      \config{\extSRaw{S}{l}{d_2}}{\unit}$.

      To show: $\config{\lubstore{S}{S''}}{\putexp{l}{d_2}}
      \parstepsto^i \error$, where $i \leq 1$.

      We proceed by cases on $\lubstore{S}{S''}$:

      \begin{itemize}

        \item $\lubstore{S}{S''} = \topS$:

          In this case, by the definition of $\error$,
          $\config{\lubstore{S}{S''}}{\putexp{l}{d_2}} = \error$.

          Hence $\config{\lubstore{S}{S''}}{\putexp{l}{d_2}}
          \parstepsto^i \error$, with $i = 0$.

        \item $\lubstore{S}{S''} \neq \topS$:

          From the premises of {\sc E-Put}, we have that $S(l) = d_1$.

          Hence $(\lubstore{S}{S''})(l) = d'_1$, where $d_1 \userleq
          d'_1$.

          We show that $\userlub{d'_1}{d_2} =
          \top$, as follows:

          By assumption, $\lubstore{\extSRaw{S}{l}{d_2}}{S''} = \topS$.

          Hence, by Definition~\ref{def:lvars-lubstore}, there exists
          some $l' \in \dom{\extSRaw{S}{l}{d_2}} \cap \dom{S''}$ such
          that $\userlub{(\extSRaw{S}{l}{d_2})(l')}{S''(l')} = \top$.

          Now case on $l'$:

          \begin{itemize}
            \item $l' \neq l$:

              In this case, $(\extSRaw{S}{l}{d_2})(l') = S(l')$.

              Since $\userlub{(\extSRaw{S}{l}{d_2})(l')}{S''(l')} = \top$,
              we then have that $\userlub{S(l')}{S''(l')} = \top$.

              However, this is a contradiction since
              $\lubstore{S}{S''} \neq \topS$.

              Hence this case cannot occur.

            \item $l' = l$:

              Then $\userlub{(\extSRaw{S}{l}{d_2})(l)}{S''(l)} = \top$.

              Note that:
              \begin{align*}
                \top &= \userlub{(\extSRaw{S}{l}{d_2})(l)}{S''(l)} \\ &=
                \userlub{d_2}{S''(l)} \\ &=
                \userlub{\userlub{d_1}{d_2}}{S''(l)}
                \\ &=
                \userlub{\userlub{S(l)}{d_2}}{S''(l)}
                \\ &=
                \userlub{\userlub{S(l)}{S''(l)}}{d_2}
                \\ &=
                \userlub{(\lubstore{S}{S''})(l)}{d_2}
                \\ &= \userlub{d'_1}{d_2}.
              \end{align*}
              Hence $\userlub{d'_1}{d_2} = \top$.

              Hence, by {\sc E-Put-Err},
              $\config{\lubstore{S}{S''}}{\putexp{l}{d_2}} \parstepsto
              \error$.

              Hence $\config{\lubstore{S}{S''}}{\putexp{l}{d_2}}
              \parstepsto^i \error$, with $i = 1$.

          \end{itemize}

      \end{itemize}

    \item Case {\sc E-Get}:

      Given: $\config{S}{\getexp{l}{T}} \parstepsto \config{S}{d_2}$.

      To show: $\config{\lubstore{S}{S''}}{\getexp{l}{T}}
      \parstepsto^i \error$, where $i \leq 1$.

      By assumption, $\lubstore{S}{S''} = \topS$.

      Hence, by the definition of $\error$,
      $\config{\lubstore{S}{S''}}{\getexp{l}{T}} = \error$.

      Hence $\config{\lubstore{S}{S''}}{\getexp{l}{T}} \parstepsto^i
      \error$, with $i = 0$.
  \end{itemize}
\end{proof}


\section{Proof of Lemma~\ref{lem:lvars-error-preservation}}\label{section:lvars-error-preservation-proof}
\begin{proof}

  Given: $\config{S}{e} \parstepsto \error$ and $\leqstore{S}{S'}$.

  To show: $\config{S'}{e} \parstepsto \error$.

  \TODO{Figure out what to do here.  I think we need to handle both
    E-Eval-Ctxt and E-Put-Err.}
\end{proof}


\section{Proof of Lemma~\ref{lem:lvars-strong-local-confluence}}\label{section:lvars-strong-local-confluence-proof}
\begin{proof}
  Suppose $\conf \ctxstepsto \conf_a$ and $\conf \ctxstepsto \conf_b$.
  We have to show that there exist $\conf_c, i, j, \pi$ such that
  $\conf_a \ctxstepsto^i \conf_c$ and $\pi(\conf_b) \ctxstepsto^j
  \pi(\conf_c)$ and $i \leq 1$ and $j \leq 1$.

  By inspection of the operational semantics, it must be the case that
  $\conf$ steps to $\conf_a$ by the {\sc E-Eval-Ctxt} rule.  Let
  $\conf = \config{S}{\evalctxt{E_a}{e_{a_1}}}$ and let $\conf_a =
  \config{S_a}{\evalctxt{E_a}{e_{a_2}}}$.

  Likewise, it must be the case that $\conf$ steps to $\conf_b$ by the
  {\sc E-Eval-Ctxt} rule.  Let $\conf =
  \config{S}{\evalctxt{E_b}{e_{b_1}}}$ and let $\conf_b =
  \config{S_b}{\evalctxt{E_b}{e_{b_2}}}$.

  Note that $\conf = \config{S}{\evalctxt{E_a}{e_{a_1}}} =
  \config{S}{\evalctxt{E_b}{e_{b_1}}}$, and so
  $\evalctxt{E_a}{e_{a_1}} = \evalctxt{E_b}{e_{b_1}}$, but $E_a$ and
  $E_b$ may differ and $e_{a_1}$ and $e_{b_1}$ may differ.

  Since $\config{S}{\evalctxt{E_a}{e_{a_1}}} \ctxstepsto
  \config{S_a}{\evalctxt{E_a}{e_{a_2}}}$ and
  $\config{S}{\evalctxt{E_b}{e_{b_1}}} \ctxstepsto
  \config{S_b}{\evalctxt{E_b}{e_{b_2}}}$ and $\evalctxt{E_a}{e_{a_1}}
  = \evalctxt{E_b}{e_{b_1}}$, we have from
  Lemma~\ref{lem:lvars-locality} (Locality) that there exist
  evaluation contexts $E'_a$ and $E'_b$ such that:

  \begin{itemize}
  \item $\evalctxt{E'_a}{e_{a_1}} = \evalctxt{E_b}{e_{b_2}}$, and
  \item $\evalctxt{E'_b}{e_{b_1}} = \evalctxt{E_a}{e_{a_2}}$, and
  \item $\evalctxt{E'_a}{e_{a_2}} =
  \evalctxt{E'_b}{e_{b_2}}$.
  \end{itemize}

  Our approach will be to show that there exist $S', i, j, \pi$ such
  that:
  \begin{itemize}
  \item $\config{S_a}{\evalctxt{E_a}{e_{a_2}}} \ctxstepsto^i
    \config{S'}{\evalctxt{E'_a}{e_{a_2}}}$, and
  \item $\pi(\config{S_b}{\evalctxt{E_b}{e_{b_2}}}) \ctxstepsto^j
    \pi(\config{S'}{\evalctxt{E'_a}{e_{a_2}}})$.
  \end{itemize}
  Since $\evalctxt{E'_a}{e_{a_1}} = \evalctxt{E_b}{e_{b_2}}$,
  $\evalctxt{E'_b}{e_{b_1}} = \evalctxt{E_a}{e_{a_2}}$, and
  $\evalctxt{E'_a}{e_{a_2}} = \evalctxt{E'_b}{e_{b_2}}$, it suffices
  to show that:
  \begin{itemize}
  \item $\config{S_a}{\evalctxt{E'_b}{e_{b_1}}} \ctxstepsto^i
    \config{S'}{\evalctxt{E'_b}{e_{b_2}}}$, and
  \item $\pi(\config{S_b}{\evalctxt{E'_a}{e_{a_1}}}) \ctxstepsto^j
    \pi(\config{S'}{\evalctxt{E'_a}{e_{a_2}}})$.
  \end{itemize}

  From the premise of {\sc E-Eval-Ctxt}, we have that
  $\config{S}{e_{a_1}} \parstepsto \config{S_a}{e_{a_2}}$ and
  $\config{S}{e_{b_1}} \parstepsto \config{S_b}{e_{b_2}}$.  We proceed
  by case analysis on the rule by which $\config{S}{e_{a_1}}$ steps to
  $\config{S_a}{e_{a_2}}$.

  \begin{enumerate}
  \item Case {\sc E-Beta}:

    We have:
    \begin{itemize}
      \item $e_{a_1} = \app{\lam{x}{e'_a}}{v_a}$,
      \item $e_{a_2} = \subst{e'_a}{x}{v_a}$, and
      \item $S_a = S$.
    \end{itemize}

    Now, we proceed by case analysis on the rule by which
    $\config{S}{e_{b_1}}$ steps to $\config{S_b}{e_{b_2}}$:
    \begin{enumerate}
    \item Case {\sc E-Beta}:

      We have:
      \begin{itemize}
      \item $e_{b_1} = \app{\lam{x}{e'_b}}{v_b}$,
      \item $e_{b_2} = \subst{e'_b}{x}{v_b}$, and
      \item $S_b = S$.
      \end{itemize}

      Choose $S' = S$, $i = 1$, $j = 1$, and $\pi = \id$.

      We have to show that:

      \begin{itemize}
      \item $\config{S}{\evalctxt{E'_b}{e_{b_1}}} \ctxstepsto
        \config{S}{\evalctxt{E'_b}{e_{b_2}}}$, and
      \item $\config{S}{\evalctxt{E'_a}{e_{a_1}}} \ctxstepsto
        \config{S}{\evalctxt{E'_a}{e_{a_2}}}$, 
      \end{itemize}

      both of which follow immediately from $\config{S}{e_{a_1}}
      \parstepsto \config{S_a}{e_{a_2}}$ and $\config{S}{e_{b_1}}
      \parstepsto \config{S_b}{e_{b_2}}$ and {\sc E-Eval-Ctxt}.

    \item Case {\sc E-New}:

      We have:
      \begin{itemize}
      \item $e_{b_1} = \NEW$,
      \item $e_{b_2} = l$, and
      \item $S_b = \extSRaw{S}{l}{\bot}$.
      \end{itemize}

      Choose $S' = S_b$, $i = 1$, $j = 1$, and $\pi = \id$.

      We have to show that:

      \begin{itemize}
      \item $\config{S}{\evalctxt{E'_b}{e_{b_1}}} \ctxstepsto
        \config{S_b}{\evalctxt{E'_b}{e_{b_2}}}$, and
      \item
        $\config{S_b}{\evalctxt{E'_a}{e_{a_1}}} \ctxstepsto
        \config{S_b}{\evalctxt{E'_a}{e_{a_2}}}$.
      \end{itemize}

      The first of these follows immediately from $\config{S}{e_{b_1}}
      \parstepsto \config{S_b}{e_{b_2}}$ and {\sc E-Eval-Ctxt}.  For
      the second, consider that $S_b = \extSRaw{S}{l}{\bot} =
      \lubstore{S}{\store{\storebindingRaw{l}{\bot}}}$.  Furthermore, we
      know from the side condition of {\sc E-New} that $l \notin
      \dom{S}$, so $\store{\storebindingRaw{l}{\bot}}$ is non-conflicting
      with the transition $\config{S}{e_{a_1}} \parstepsto
      \config{S_a}{e_{a_2}}$, and we know that
      $\lubstore{S_a}{\store{\storebindingRaw{l}{\bot}}} \neq \topS$
      since $S_a$ is just $S$.  Therefore, by
      Lemma~\ref{lem:lvars-independence} (Independence), we have that
      $\config{\lubstore{S}{\store{\storebindingRaw{l}{\bot}}}}{e_{a_1}}
      \parstepsto
      \config{\lubstore{S_a}{\store{\storebindingRaw{l}{\bot}}}}{e_{a_2}}$.
      Hence $\config{S_b}{e_{a_1}} \parstepsto \config{S_b}{e_{a_2}}$.
      By {\sc E-Eval-Ctxt}, it follows that
      $\config{S_b}{\evalctxt{E'_a}{e_{a_1}}} \ctxstepsto
      \config{S_b}{\evalctxt{E'_a}{e_{a_2}}}$, as we were required to
      show.

    \item Case {\sc E-Put}: \TODO{}
    \item Case {\sc E-Put-Err}: \TODO{}
    \item Case {\sc E-Get}:\TODO{}
    \end{enumerate}
  \item Case {\sc E-New}:

    Now, we proceed by case analysis on the rule by which
    $\config{S}{e_{b_1}}$ steps to $\config{S_b}{e_{b_2}}$:
    \begin{enumerate}
    \item Case {\sc E-Beta}: \TODO{}
    \item Case {\sc E-New}: \TODO{}
    \item Case {\sc E-Put}: \TODO{}
    \item Case {\sc E-Put-Err}: \TODO{}
    \item Case {\sc E-Get}: \TODO{}
    \end{enumerate}
  \item Case {\sc E-Put}:

    Now, we proceed by case analysis on the rule by which
    $\config{S}{e_{b_1}}$ steps to $\config{S_b}{e_{b_2}}$:
    \begin{enumerate}
    \item Case {\sc E-Beta}: \TODO{}
    \item Case {\sc E-New}: \TODO{}
    \item Case {\sc E-Put}: \TODO{}
    \item Case {\sc E-Put-Err}: \TODO{}
    \item Case {\sc E-Get}: \TODO{}
    \end{enumerate}
  \item Case {\sc E-Put-Err}:

    Now, we proceed by case analysis on the rule by which
    $\config{S}{e_{b_1}}$ steps to $\config{S_b}{e_{b_2}}$:
    \begin{enumerate}
    \item Case {\sc E-Beta}: \TODO{}
    \item Case {\sc E-New}: \TODO{}
    \item Case {\sc E-Put}: \TODO{}
    \item Case {\sc E-Put-Err}: \TODO{}
    \item Case {\sc E-Get}: \TODO{}
    \end{enumerate}
  \item Case {\sc E-Get}:

    Now, we proceed by case analysis on the rule by which
    $\config{S}{e_{b_1}}$ steps to $\config{S_b}{e_{b_2}}$:
    \begin{enumerate}
    \item Case {\sc E-Beta}: \TODO{}
    \item Case {\sc E-New}: \TODO{}
    \item Case {\sc E-Put}: \TODO{}
    \item Case {\sc E-Put-Err}: \TODO{}
    \item Case {\sc E-Get}: \TODO{}
    \end{enumerate}
  \end{enumerate}

  \lk{I think we also still have to separately deal with cases where
    $\conf_a = \error$ or $\conf_b = \error$.}
\end{proof}


\section{Proof of Lemma~\ref{lem:lvars-strong-one-sided-confluence}}\label{section:lvars-strong-one-sided-confluence-proof}
\begin{proof}
  Suppose $\conf \ctxstepsto \conf'$ and $\conf \ctxstepsto^m
  \conf''$, where $1 \leq m$.  We have to show that there exist
  $\conf_c, i, j, \pi$ such that $\conf' \ctxstepsto^i \conf_c$ and
  $\pi(\conf'') \ctxstepsto^j \conf_c$ and $i \leq m$ and $j \leq 1$.

  We proceed by induction on $m$.  In the base case of $m = 1$, the
  result is immediate from
  Lemma~\ref{lem:lvars-strong-local-confluence}.

  For the induction step, suppose $\conf \ctxstepsto^m \conf''
  \ctxstepsto \conf'''$ and suppose the lemma holds for $m$.

  We show that it holds for $m + 1$, as follows.

  We are required to show that there exist $\conf_c, i, j, \pi$ such
  that $\conf' \ctxstepsto^{i} \conf_c$ and $\pi(\conf''')
  \ctxstepsto^{j} \conf_c$ and $i \leq m + 1$ and $j \leq 1$.

  From the induction hypothesis, there exist $\conf_c', i', j', \pi'$
  such that $\conf' \ctxstepsto^{i'} \conf_c'$ and $\pi'(\conf'')
  \ctxstepsto^{j'} \conf_c'$ and $i' \leq m$ and $j' \leq 1$.

  We proceed by cases on $j'$:
  \begin{itemize}

  \item If $j' = 0$, then $\pi'(\conf'') = \conf_c'$.

    Since $\conf'' \ctxstepsto \conf'''$, we have that $\pi'(\conf'')
    \ctxstepsto \pi'(\conf''')$ by
    Lemma~\ref{lem:lvars-permutability} (Permutability).

    We can then choose $\conf_c = \pi'(\conf''')$ and $i = i' + 1$ and
    $j = 0$ and $\pi = \pi'$.  The key is that $\conf'
    \ctxstepsto^{i'} \conf'_c = \pi'(\conf'') \ctxstepsto
    \pi'(\conf''')$ for a total of $i' + 1$ steps.
    
  \item If $j' = 1$:

    First, since $\pi'(\conf'') \ctxstepsto^{j'} \conf'_c$, then by
    Lemma~\ref{lem:lvars-permutability} (Permutability) we have that
    $\conf'' \ctxstepsto^{j'} \piprimeinv(\conf'_c)$.

    Then, by $\conf'' \ctxstepsto^{j'} \piprimeinv(\conf'_c)$ and
    $\conf'' \ctxstepsto \conf'''$ and
    Lemma~\ref{lem:lvars-strong-local-confluence} (Strong Local
    Confluence), we have that there exist $\conf_c''$ and $i''$ and
    $j''$ and $\pi''$ such that $\piprimeinv(\conf'_c)
    \ctxstepsto^{i''} \conf_c''$ and $\pi''(\conf''')
    \ctxstepsto^{j''} \conf_c''$ and $i'' \leq 1$ and $j'' \leq 1$.

    Since $\piprimeinv(\conf'_c) \ctxstepsto^{i''} \conf_c''$, by
    Lemma~\ref{lem:lvars-permutability} (Permutability) we have that
    $\conf'_c \ctxstepsto^{i''} \pi'(\conf_c'')$.

    So we also have $\conf' \ctxstepsto^{i'} \conf_c'
    \ctxstepsto^{i''} \pi'(\conf_c'')$.

    Since $\pi''(\conf''') \ctxstepsto^{j''} \conf_c''$, by
    Lemma~\ref{lem:lvars-permutability} (Permutability) we have that
    $\pi'(\pi''(\conf''')) \ctxstepsto^{j''} \pi'(\conf_c'')$.

    In summary, we pick $\conf_c = \pi'(\conf_c'')$ and $i = i' + i''$
    and $j = j''$ and $\pi = \pi'' \circ \pi'$, which is sufficient
    because $i = i' + i'' \leq m + 1$ and $j = j'' \leq 1$.
  \end{itemize}

 \end{proof}


\section{Proof of Lemma~\ref{lem:lvars-strong-confluence}}\label{section:lvars-strong-confluence-proof}
\begin{proof}
  We proceed by induction on $n$.  In the base case of $n = 1$, the
  result is immediate from
  Lemma~\ref{lem:lvars-strong-one-sided-confluence}.

  For the induction step, suppose $\conf \parstepsto^n \conf'
  \parstepsto \conf'''$ and suppose the lemma holds for $n$.

  We show that it holds for $n + 1$, as follows.

  We are required to show that there exist $\conf_c, i, j$ such that
  $\conf''' \parstepsto^i \conf_c$ and $\conf'' \parstepsto^j \conf_c$
  and $i \leq m$ and $j \leq n + 1$.

  From the induction hypothesis, we have that there exist $\conf'_c,
  i', j'$ such that $\conf' \parstepsto^{i'} \conf'_c$ and $\conf''
  \parstepsto^{j'} \conf'_c$ and $i' \leq m$ and $j' \leq n$.

  We proceed by cases on $i'$:
  \begin{itemize}

  \item If $i' = 0$, then $\conf' = \conf_c'$.  We can then choose
    $\conf_c = \conf'''$ and $i = 0$ and $j = j' + 1$.

  \item If $i' \geq 1$:

    From $\conf' \parstepsto \conf'''$ and $\conf' \parstepsto^{i'}
    \conf_c'$ and Lemma~\ref{lem:lvars-strong-one-sided-confluence},
    we have that there exist $\conf_c''$ and $i''$ and $j''$ such that
    $\conf''' \parstepsto^{i''} \conf_c''$ and $\conf_c'
    \parstepsto^{j''} \conf_c''$ and $i'' \leq i'$ and $j'' \leq 1$.
    So we also have $\conf'' \parstepsto^{j'} \conf_c'
    \parstepsto^{j''} \conf_c''$.  In summary, we pick $\conf_c =
    \conf_c''$ and $i = i''$ and $j = j' + j''$, which is sufficient
    because $i = i'' \leq i' \leq m$ and $j = j' + j'' \leq n + 1$.
  \end{itemize}

\end{proof}


\section{Proof of Lemma~\ref{lem:lattice-structure}}\label{section:lattice-structure-proof}
\begin{proof}
  Suppose that $(D, \userleq, \bot, \top)$ is a lattice and $(D_p,
  \leqp, \botp, \topp) = \Freeze{D, \userleq, \bot, \top}$.

  In order to show that $(D_p, \leqp, \botp, \topp)$ is a lattice, we
  have to show that:
  \begin{enumerate}
  \item $\leqp$ is a partial order over $D_p$.

  \item Every nonempty finite subset of $D_p$ has a lub.

  \item $\botp$ is the least element of $D_p$.

  \item $\topp$ is the greatest element of $D_p$.
  \end{enumerate}

  We prove each of these properties in turn:

  \begin{enumerate}
  \item $\leqp$ is a partial order over $D_p$.

    To show this, we need to show that $\leqp$ is reflexive, transitive,
    and antisymmetric. 
    \begin{enumerate}
    \item $\leqp$ is reflexive.

      Suppose $v \in D_p$.

      Then, by Lemma~\ref{lem:partition-of-Dp}, either $v =
      \state{d}{\frozenfalse}$ with $d \in D$, or $v =
      \state{x}{\frozentrue}$ with $x \in X$, where $X = D -
      \setof{\top}$.
      \begin{itemize}
      \item Suppose $v = \state{d}{\frozenfalse}$:

        By the reflexivity of $\userleq$, we know $d \userleq d$.

        By the definition of $\leqp$, we know $\state{d}{\frozenfalse}
        \leqp \state{d}{\frozenfalse}$.

      \item Suppose $v = \state{x}{\frozentrue}$: 
        
        By the reflexivity of equality, $x = x$.

        By the definition of $\leqp$, we know $\state{x}{\frozentrue}
        \leqp \state{x}{\frozentrue}$.
      \end{itemize}

    \item $\leqp$ is transitive. 

      Suppose $v_1 \leqp v_2$ and $v_2 \leqp v_3$.

      We want to show that $v_1 \leqp v_3$.

      We proceed by case analysis on $v_1, v_2$, and $v_3$.
      \begin{itemize}
      \item Case $v_1 = \state{d_1}{\frozenfalse}$ and $v_2 =
        \state{d_2}{\frozenfalse}$ and $v_3 =
        \state{d_3}{\frozenfalse}$:
        
        By inversion on $\leqp$, it follows that $d_1 \userleq d_2$.

        By inversion on $\leqp$, it follows that $d_2 \userleq d_3$.

        By the transitivity of $\userleq$, we know $d_1 \userleq d_3$.

        By the definition of $\leqp$, it follows that
        $\state{d_1}{\frozenfalse} \leqp \state{d_3}{\frozenfalse}$.

        Hence $v_1 \leqp v_3$.

      \item Case $v_1 = \state{d_1}{\frozenfalse}$ and $v_2 =
        \state{d_2}{\frozenfalse}$ and $v_3 =
        \state{x_3}{\frozentrue}$:

        By inversion on $\leqp$, it follows that $d_1 \userleq d_2$.

        By inversion on $\leqp$, it follows that $d_2 \userleq x_3$.

        By the transitivity of $\userleq$, we know $d_1 \userleq x_3$.

        By the definition of $\leqp$, it follows that
        $\state{d_1}{\frozenfalse} \leqp \state{x_3}{\frozentrue}$.

        Hence $v_1 \leqp v_3$.

      \item Case $v_1 = \state{d_1}{\frozenfalse}$ and $v_2 =
        \state{x_2}{\frozentrue}$ and $v_3 =
        \state{d_3}{\frozenfalse}$:

        By inversion on $\leqp$, it follows that $d_1 \userleq x_2$.

        By inversion on $\leqp$, it follows that $d_3 = \top$.

        Since $\top$ is the maximal element of $D$, we know $d_1
        \userleq \top \equiv d_3$.

        By the definition of $\leqp$, it follows that
        $\state{d_1}{\frozenfalse} \leqp \state{d_3}{\frozenfalse}$.

        Hence $v_1 \leqp v_3$.

      \item Case $v_1 = \state{d_1}{\frozenfalse}$ and $v_2 =
        \state{x_2}{\frozentrue}$ and $v_3 =
        \state{x_3}{\frozentrue}$:

        By inversion on $\leqp$, it follows that $d_1 \userleq x_2$.

        By inversion on $\leqp$, it follows that $x_2 = x_3$.

        Hence $d_1 \userleq x_3$.

        By the definition of $\leqp$, it follows that
        $\state{d_1}{\frozenfalse} \leqp \state{x_3}{\frozentrue}$.

        Hence $v_1 \leqp v_3$.

      \item Case $v_1 = \state{x_1}{\frozentrue}$ and $v_2 =
        \state{d_2}{\frozenfalse}$ and $v_3 =
        \state{d_3}{\frozenfalse}$:

        By inversion on $\leqp$, it follows that $d_2 = \top$.

        By inversion on $\leqp$, it follows that $d_2 \userleq d_3$.

        Since $\top$ is maximal, it follows that $d_3 = \top$.

        By the definition of $\leqp$, it follows that
        $\state{x_1}{\frozentrue} \leqp \state{d_3}{\frozenfalse}$.

        Hence $v_1 \leqp v_3$. 

      \item Case $v_1 = \state{x_1}{\frozentrue}$ and $v_2 =
        \state{d_2}{\frozenfalse}$ and $v_3 =
        \state{x_3}{\frozentrue}$:

        By inversion on $\leqp$, it follows that $d_2 = \top$.

        By inversion on $\leqp$, it follows that $d_2 \userleq x_3$.

        Since $\top$ is maximal, it follows that $x_3 = \top$.

        But since $x_3 \in X \subseteq D/\setof{\top}$, we know $x_3
        \not= \top$.

        This is a contradiction. \\

        Hence $v_1 \leqp v_3$. 

      \item Case $v_1 = \state{x_1}{\frozentrue}$ and $v_2 =
        \state{x_2}{\frozentrue}$ and $v_3 =
        \state{d_3}{\frozenfalse}$:

        By inversion on $\leqp$, it follows that $x_1 = x_2$.

        By inversion on $\leqp$, it follows that $d_3 = \top$.

        By the definition of $\leqp$, it follows that
        $\state{x_1}{\frozentrue} \leqp \state{d_3}{\frozenfalse}$.

        Hence $v_1 \leqp v_3$. 

      \item Case $v_1 = \state{x_1}{\frozentrue}$ and $v_2 =
        \state{x_2}{\frozentrue}$ and $v_3 =
        \state{x_3}{\frozentrue}$:

        By inversion on $\leqp$, it follows that $x_1 = x_2$.

        By inversion on $\leqp$, it follows that $x_2 = x_3$.

        By transitivity of $=$, $x_1 = x_3$.

        By the definition of $\leqp$, it follows that
        $\state{x_1}{\frozentrue} \leqp \state{x_3}{\frozentrue}$.

        Hence $v_1 \leqp v_3$. 
        
      \end{itemize}

    \item $\leqp$ is antisymmetric. 

      Suppose $v_1 \leqp v_2$ and $v_2 \leqp v_1$. Now, we proceed by
      cases on $v_1$ and $v_2$.
      \begin{itemize}
      \item Case $v_1 = \state{d_1}{\frozenfalse}$ and $v_2 =
        \state{d_2}{\frozenfalse}$:
        
        By inversion on $v_1 \leqp v_2$, we know that $d_1 \userleq
        d_2$.

        By inversion on $v_2 \leqp v_1$, we know that $d_2 \userleq
        d_1$.

        By the antisymmetry of $\leq$, we know $d_1 = d_2$.

        Hence $v_1 = v_2$. 

      \item Case $v_1 = \state{d_1}{\frozenfalse}$ and $v_2 =
        \state{x_2}{\frozentrue}$:

        By inversion on $v_1 \leqp v_2$, we know that $d_1 \userleq x_2$.

        By inversion on $v_2 \leqp v_1$, we know that $d_1 = \top$.

        Since $\top$ is maximal in $D$, we know $x_2 = \top$.

        But since $x_2 \in X \subseteq D/\setof{\top}$, we know $x_2 \not= \top$.

        This is a contradiction.

        Hence $v_1 = v_2$. 
        
      \item Case $v_1 = \state{x_1}{\frozentrue}$ and $v_2 =
        \state{d_2}{\frozenfalse}$:

        Similar to the previous case. 

      \item Case $v_1 = \state{x_1}{\frozentrue}$ and $v_2 =
        \state{x_2}{\frozentrue}$:

        By inversion on $v_1 \leqp v_2$, we know that $x_1 = x_2$.

        Hence $v_1 = v_2$. 
      \end{itemize}
    \end{enumerate}

  \item Every nonempty finite subset of $D_p$ has a lub.

    To show this, it is sufficient to show that every two elements of
    $D_p$ have a lub, since a binary lub operation can be repeatedly
    applied to compute the lub of any finite set.

    We will show that every two elements of $D_p$ have a lub by
    showing that the $\lubp{}{}$ operation defined by
    Definition~\ref{def:lubp} computes their lub.

    It suffices to show the following two properties:
    \begin{enumerate}
    \item For all $v_1, v_2, v \in D_p$, if $v_1 \leqp v$ and $v_2
      \leqp v$, then $(\lubp{v_1}{v_2}) \leqp v$.
    \item For all $v_1, v_2 \in D_p$, $v_1 \leqp (\lubp{v_1}{v_2})$
      and $v_2 \leqp (\lubp{v_1}{v_2})$.
    \end{enumerate}
    \begin{enumerate}
    \item For all $v_1, v_2, v \in D_p$, if $v_1 \leqp v$ and $v_2
      \leqp v$, then $\lubp{v_1}{v_2} \leqp v$.
      
      Assume $v_1, v_2, v \in D_p$, and $v_1 \leqp v$ and $v_2 \leqp
      v$.

      Now we do a case analysis on $v_1$ and $v_2$.
      \begin{itemize}
      \item Case $v_1 = \state{d_1}{\frozenfalse}$ and $v_2 =
        \state{d_2}{\frozenfalse}$.
        
        Now case on $v$: 
        \begin{itemize}
        \item Case $v = \state{d}{\frozenfalse}$: 

          By the definition of $\lubp{}{}$,
          $\lubp{\state{d_1}{\frozenfalse}}{\state{d_2}{\frozenfalse}}
          = \state{\userlub{d_1}{d_2}}{\frozenfalse}$.

          By inversion on $\state{d_1}{\frozenfalse} \leqp
          \state{d}{\frozenfalse}$, $d_1 \userleq l$.

          By inversion on $\state{d_2}{\frozenfalse} \leqp
          \state{d}{\frozenfalse}$, $d_2 \userleq l$.

          Hence $l$ is an upper bound for $d_1$ and $d_2$.

          Hence $\userlub{d_1}{d_2} \userleq l$.

          Hence $\state{\userlub{d_1}{d_2}}{\frozenfalse} \leqp
          \state{d}{\frozenfalse}$.

          Hence $\lubp{v_1}{v_2} \leqp v$.
          
        \item Case $v = \state{x}{\frozentrue}$: 
          
          By the definition of $\lubp{}{}$, $\state{d_1}{\frozenfalse}
          \lubp{}{} \state{d_2}{\frozenfalse} =
          \state{\userlub{d_1}{d_2}}{\frozenfalse}$.

          By inversion on $\state{d_1}{\frozenfalse} \leqp
          \state{x}{\frozentrue}$, $d_1 \userleq x$.

          By inversion on $\state{d_2}{\frozenfalse} \leqp
          \state{x}{\frozentrue}$, $d_2 \userleq x$.
     
          Hence $x$ is an upper bound for $d_1$ and $d_2$.

          Hence $\userlub{d_1}{d_2} \userleq x$.

          Hence $\state{\userlub{d_1}{d_2}}{\frozenfalse} \leqp
          \state{x}{\frozentrue}$.

          Hence $\lubp{v_1}{v_2} \leqp v$.
        \end{itemize}
        
      \item Case $v_1 = \state{x_1}{\frozentrue}$ and $v_2 =
        \state{x_2}{\frozentrue}$:
        
        Now case on $v$: 
        \begin{itemize}
        \item Case $v = \state{d}{\frozenfalse}$: 
          
          By inversion on $\state{x_1}{\frozentrue} \leqp
          \state{d}{\frozenfalse}$, we know $l = \top$.

          By inversion on $\state{x_2}{\frozentrue} \leqp
          \state{d}{\frozenfalse}$, we know $l = \top$.

          Now consider whether $x_1 = x_2$ or not.
        
          If it does, then by the definition of $\lubp{}{}$,
          $\state{x_1}{\frozentrue} \lubp{}{} \state{x_2}{\frozentrue}
          = \state{x_1}{\frozentrue}$.

          By definition of $\leqp$, we have $\state{x_1}{\frozentrue}
          \leqp \state{\top}{\frozenfalse}$.

          So $\lubp{v_1}{v_2} \leqp v$.

          If it does not, then $\lubp{v_1}{v_2} =
          \state{\top}{\frozenfalse}$.

          By the definition of $\leqp$, we have
          $\state{\top}{\frozenfalse} \leqp
          \state{\top}{\frozenfalse}$.

          So $\lubp{v_1}{v_2} \leqp v$.
          
        \item Case $v = \state{x}{\frozentrue}$: 
          
          By inversion on $\state{x_1}{\frozentrue} \leqp
          \state{x}{\frozentrue}$, we know $x = x_1$.

          By inversion on $\state{x_2}{\frozentrue} \leqp
          \state{x}{\frozentrue}$, we know $x = x_2$.

          Hence $x_1 = x_2$.

          By the definition of $\lubp{}{}$, $\state{x_1}{\frozentrue}
          \lubp{}{} \state{x_2}{\frozentrue} =
          \state{x_1}{\frozentrue}$.

          Hence $\lubp{v_1}{v_2} \leqp v$.
        \end{itemize}
        
      \item Case $v_1 = \state{x_1}{\frozentrue}$ and $v_2 =
        \state{d_2}{\frozenfalse}$:
        
        Now case on $v$:
        \begin{itemize}
        \item Case $v = \state{d}{\frozenfalse}$:
          
          Now consider whether $d_2 \userleq x_1$.

          If it is, then $\state{x_1}{\frozentrue} \lubp{}{}
          \state{d_2}{\frozenfalse} = \state{x_1}{\frozentrue} = v_1$.

          Hence $\lubp{v_1}{v_2} \leqp v$.

          Otherwise, $\state{x_1}{\frozentrue} \lubp{}{}
          \state{d_2}{\frozenfalse} = \state{\top}{\frozenfalse}$.

          By inversion on $\state{x_1}{\frozentrue} \leqp
          \state{d}{\frozenfalse}$, we know $l = \top$.

          By reflexivity, $\state{\top}{\frozenfalse} \leqp
          \state{\top}{\frozenfalse}$.

          Hence $\lubp{v_1}{v_2} \leqp v$. 
          
        \item Case $v = \state{x}{\frozentrue}$:  
          
          By inversion on $\state{x_1}{\frozentrue} \leqp
          \state{x}{\frozentrue}$, we know that $x_1 = x$.

          By inversion on $\state{d_2}{\frozenfalse} \leqp
          \state{x}{\frozentrue}$, we know that $d_2 \userleq x$.

          By transitivity, $d_2 \userleq x_1$.

          By the definition of $\lubp{}{}$, it follows that
          $\state{x_1}{\frozentrue} \lubp{}{}
          \state{d_2}{\frozenfalse} = \state{x_1}{\frozentrue}$.

          By definition of $\leqp$, $\state{x_1}{\frozentrue} \leqp
          \state{x_1}{\frozentrue}$.

          Hence $\lubp{v_1}{v_2} \leqp v$. 
        \end{itemize}
        
      \item Case $v_1 = \state{d_1}{\frozenfalse}$ and $v_2 =
        \state{x_2}{\frozentrue}$:
        
        Symmetric with the previous case. 
      \end{itemize}
    \item For all $v_1, v_2 \in D_p$, $v_1 \leqp \lubp{v_1}{v_2}$ and
      $v_2 \leqp \lubp{v_1}{v_2}$.
      
      Assume $v_1, v_2 \in D_p$, and proceed by case analysis. 
      \begin{itemize}
      \item Case $v_1 = \state{d_1}{\frozenfalse}$ and $v_2 =
        \state{d_2}{\frozenfalse}$:

        Since $\userlub{}{}$ is a join operator, we know $d_1 \userleq
        \userlub{d_1}{d_2}$.

        By the definition of $\leqp$, $\state{d_1}{\frozenfalse}
        \userleq \state{\userlub{d_1}{d_2}}{\frozenfalse}$.

        By the definition of $\lubp{}{}$, $\lubp{v_1}{v_2} =
        \state{\userlub{d_1}{d_2}}{\frozenfalse}$.

        Hence $v_1 \leqp \lubp{v_1}{v_2}$.

        Since $\userlub{}{}$ is a join operator, we know $d_1 \userleq
        \userlub{d_1}{d_2}$.

        By the definition of $\leqp$, $\state{d_2}{\frozenfalse}
        \userleq \state{\userlub{d_1}{d_2}}{\frozenfalse}$.

        By the definition of $\lubp{}{}$, $\lubp{v_1}{v_2} =
        \state{\userlub{d_1}{d_2}}{\frozenfalse}$.

        Hence $v_2 \leqp \lubp{v_1}{v_2}$. 

        Therefore $v_1 \leqp v_1 \userlub{}{} v_2$ and $v_2 \leqp v_1
        \userlub{}{} v_2$.
 
      \item Case $v_1 = \state{d_1}{\frozenfalse}$ and $v_2 = \state{x_2}{\frozentrue}$:

        Consider whether $d_1 \userleq x_2$. 
        \begin{itemize}
        \item Case  $d_1 \userleq x_2$:

          By the definition of $\lubp{}{}$, we know
          $\state{d_1}{\frozenfalse} \lubp{}{}
          \state{x_2}{\frozentrue} = \state{x_2}{\frozentrue}$.

          By the definition of $\lubp{}{}$, we know
          $\state{d_1}{\frozenfalse} \leqp \state{x_2}{\frozentrue}$.

          Hence $v_1 \leqp \lubp{v_1}{v_2}$.

          By reflexivity, $\state{x_2}{\frozentrue} \leqp
          \state{x_2}{\frozentrue}$.

          Hence $v_2 \leqp \lubp{v_1}{v_2}$.

          Therefore $v_1 \leqp v_1 \userlub{}{} v_2$ and $v_2 \leqp
          v_1 \userlub{}{} v_2$.

        \item Case $d_1 \not\userleq x_2$:

          By the definition of $\lubp{}{}$, we know
          $\state{d_1}{\frozenfalse} \lubp{}{}
          \state{x_2}{\frozentrue} = \state{\top}{\frozenfalse}$.

          Since $d_1 \userleq \top$, by the definition of $\leqp$ we
          know $\state{d_1}{\frozenfalse} \userleq
          \state{\top}{\frozenfalse}$.

          Hence $v_1 \leqp \lubp{v_1}{v_2}$.

          By the definition of $\leqp$, we know
          $\state{x_2}{\frozentrue} \userleq
          \state{\top}{\frozenfalse}$.

          Hence $v_2 \leqp \lubp{v_1}{v_2}$.

          Therefore $v_1 \leqp v_1 \userlub{}{} v_2$ and $v_2 \leqp
          v_1 \userlub{}{} v_2$.
        \end{itemize}
      \item Case $v_1 = \state{x_1}{\frozentrue}$ and $v_2 =
        \state{d_2}{\frozenfalse}$:

        Symmetric with the previous case. 
      \item Case $v_1 = \state{x_1}{\frozentrue}$ and $v_2 =
        \state{x_2}{\frozentrue}$:

        Consider whether $x_1$ equals $x_2$. 
        \begin{itemize}
        \item Case $x_1 = x_2$:
          
          By the definition $\lubp{}{}$, $\state{x_1}{\frozentrue}
          \lubp{}{} \state{x_2}{\frozentrue} =
          \state{x_1}{\frozentrue}$.
 
          By reflexivity, $\state{x_1}{\frozentrue} \leqp
          \state{x_1}{\frozentrue}$.

          Hence $v_1 \leqp \lubp{v_1}{v_2}$.

          By reflexivity, $\state{x_2}{\frozentrue} \leqp
          \state{x_1}{\frozentrue}$.

          Hence $v_2 \leqp \lubp{v_1}{v_2}$.

          Therefore $v_1 \leqp v_1 \userlub{}{} v_2$ and $v_2 \leqp
          v_1 \userlub{}{} v_2$.

        \item Case $x_1 \not= x_2$: 

          By the definition $\lubp{}{}$, $\state{x_1}{\frozentrue}
          \lubp{}{} \state{x_2}{\frozentrue} =
          \state{\top}{\frozenfalse}$.

          By the definition of $\leqp$, $\state{x_1}{\frozentrue}
          \leqp \state{\top}{\frozenfalse}$.

          Hence $v_1 \leqp \lubp{v_1}{v_2}$.

          By the definition of $\leqp$, $\state{x_2}{\frozentrue}
          \leqp \state{\top}{\frozenfalse}$.

          Hence $v_2 \leqp \lubp{v_1}{v_2}$.

          Therefore $v_1 \leqp v_1 \userlub{}{} v_2$ and $v_2 \leqp
          v_1 \userlub{}{} v_2$.
        \end{itemize}
      \end{itemize}
    \end{enumerate}

  \item $\botp$ is the least element of $D_p$. 

    $\botp$ is defined to be $\state{\bot}{\frozenfalse}$.

    In order to be the least element of $D_p$, it must be less than or
    equal to every element of $D_p$.

    By Lemma~\ref{lem:partition-of-Dp}, the elements of $D_p$
    partition into $\state{d}{\frozenfalse}$ for all $d \in D$, and
    $\state{x}{\frozentrue}$ for all $x \in X$, where $X = D -
    \setof{\top}$.

    We consider both cases:

    \begin{itemize}
    \item $\state{d}{\frozenfalse}$ for all $d \in D$:

      By the definition of $\leqp$, $\state{\bot}{\frozenfalse} \leqp
      \state{d}{\frozenfalse}$ iff $\bot \userleq d$.

      Since $\bot$ is the least element of $D$, $\bot \userleq d$.

      Therefore $\botp = \state{\bot}{\frozenfalse} \leqp
      \state{d}{\frozenfalse}$.

    \item $\state{x}{\frozentrue}$ for all $x \in X$:

      By the definition of $\leqp$, $\state{\bot}{\frozenfalse} \leqp
      \state{x}{\frozentrue}$ iff $\bot \userleq x$.

      Since $\bot$ is the least element of $D$, $\bot \userleq x$.

      Therefore $\botp = \state{\bot}{\frozenfalse} \leqp
      \state{x}{\frozentrue}$.

    \end{itemize}

    Therefore $\botp$ is less than or equal to all elements of $D_p$.

  \item $\topp$ is the greatest element of $D_p$.

    $\topp$ is defined to be $\state{\top}{\frozenfalse}$.

    In order to be the greatest element of $D_p$, every element of
    $D_p$ must be less than or equal to it.

    By Lemma~\ref{lem:partition-of-Dp}, the elements of $D_p$
    partition into $\state{d}{\frozenfalse}$ for all $d \in D$, and
    $\state{x}{\frozentrue}$ for all $x \in X$, where $X = D -
    \setof{\top}$.

    We consider both cases:

    \begin{itemize}
    \item $\state{d}{\frozenfalse}$ for all $d \in D$:

      By the definition of $\leqp$, $\state{d}{\frozenfalse} \leqp
      \state{\top}{\frozenfalse}$ iff $d \userleq \top$.

      Since $\top$ is the greatest element of $D$, $d \userleq \top$.

      Therefore $\state{d}{\frozenfalse} \leqp
      \state{\top}{\frozenfalse} = \topp$.

    \item $\state{x}{\frozentrue}$ for all $x \in X$:

      By the definition of $\leqp$, $\state{x}{\frozentrue} \leqp
      \state{\top}{\frozenfalse}$ iff $\top \userleq \top$.

      Therefore $\state{x}{\frozentrue} \leqp
      \state{\top}{\frozenfalse} = \topp$.

    \end{itemize}

    Therefore all elements of $D_p$ are less than or equal to $\topp$.
  \end{enumerate}
\end{proof}


\section{Proof of Lemma~\ref{lem:monotonicity}}\label{section:monotonicity-proof}
\begin{proof}
  \TODO{Fix the typos I found in this.}

  \begin{itemize}

    \item Case {\sc E-Eval-Ctxt}:

      Given: $\config{S}{\E{e}} \parstepsto \config{S'}{\E{e'}}$.

      To show: $\leqstore{S}{S'}$.

      From the premise of {\sc E-Eval-Ctxt}, $\config{S}{e}
      \parstepsto \config{S'}{e'}$.

      Hence by IH, $\leqstore{S}{S'}$, as we were required to show.

    \item Case {\sc E-Beta}:

      Immediate by the definition of $\leqstore{}{}$, since $S$ does
      not change.

    \item Case {\sc E-New}:

      Given: $\config{S}{\NEW} \parstepsto
      \config{\extS{S}{l}{\bot}{\frozenfalse}}{l}$.

      To show: $\leqstore{S}{\extS{S}{l}{\bot}{\frozenfalse}}$.

      By Definition~\ref{def:leqstore}, we have to show that $\dom{S}
      \subseteq \dom{\extS{S}{l}{\bot}{\frozenfalse}}$ and that for
      all $l' \in \dom{S}, \\
      S(l') \leqp (\extS{S}{l}{\bot}{\frozenfalse})(l')$.

      By the definition of store update,
      $\extS{S}{l}{d_1}{\frozentrue}$ can only either update an
      existing binding in $S$ or extend $S$ with a new binding.

      Hence $\dom{S} \subseteq \dom{\extS{S}{l}{\bot}{\frozenfalse}}$.

      From the side condition of {\sc E-New}, $l \notin \dom{S}$.

      Hence $\extS{S}{l}{\bot}{\frozenfalse}$ adds a new binding for
      $l$ in $S$.

      Hence $\extS{S}{l}{d_1}{\frozentrue}$ does not update any
      existing bindings in $S$.

      Hence, for all $l' \in \dom{S}, S(l') \leqp
      (\extS{S}{l}{d_1}{\frozentrue})(l')$.

      Therefore $\leqstore{S}{\extS{S}{l}{\bot}{\frozenfalse}}$, as
      required.

    \item Case {\sc E-Put}:

      Given: $\config{S}{\putexp{l}{d_2}} \parstepsto
      \config{\extSRaw{S}{l}{p_2}}{\unit}$.

      To show: $\leqstore{S}{\extSRaw{S}{l}{p_2}}$.

      By Definition~\ref{def:leqstore}, we have to show that $\dom{S}
      \subseteq \dom{\extSRaw{S}{l}{p_2}}$ and that for all $l' \in
      \dom{S}, \\
      S(l') \leqp (\extSRaw{S}{l}{p_2})(l')$.

      By the definition of store update, $\extSRaw{S}{l}{p_2}$ can only
      either update an existing binding in $S$ or extend $S$ with a
      new binding.

      Hence $\dom{S} \subseteq \dom{\extSRaw{S}{l}{p_2}}$.

      From the premises of {\sc E-Put}, $S(l) = p_1$.  Therefore $l
      \in \dom{S}$.

      Hence $\extSRaw{S}{l}{p_2}$ updates the existing binding for $l$
      in $S$ from $p_1$ to $p_2$.

      From the premises of {\sc E-Put}, $p_2 =
      \lubp{p_1}{\state{d_2}{\frozenfalse}}$.

      Hence, by the definition of $\lubp{}{}$, $p_1 \leqp p_2$.

      $\extSRaw{S}{l}{p_2}$ does not update any other bindings in $S$,
      hence, for all $l' \in \dom{S}, S(l') \leqp
      (\extSRaw{S}{l}{p_2})(l')$.

      Hence $\leqstore{S}{\extSRaw{S}{l}{p_2}}$, as required.

    \item Case {\sc E-Put-Err}:

      Given: $\config{S}{\putexp{l}{d_2}} \parstepsto \error$.

      By the definition of $\error$, $\error = \config{\topS}{e}$ for
      any $e$.

      To show: $\leqstore{S}{\topS}$.

      Immediate by the definition of $\leqstore{}{}$.

    \item Case {\sc E-Get}:

      Immediate by the definition of $\leqstore{}{}$, since $S$ does
      not change.

    \item Case {\sc E-Freeze-Init}:

      Immediate by the definition of $\leqstore{}{}$, since $S$ does
      not change.

    \item Case {\sc E-Spawn-Handler}:

      Immediate by the definition of $\leqstore{}{}$, since $S$ does
      not change.

    \item Case {\sc E-Freeze-Final}:

      Given: $\config{S}{\freezeafterfull{l}{Q}{v}{\setof{v\dots}}{H}}
      \parstepsto \config{\extS{S}{l}{d_1}{\frozentrue}}{d_1}$.

      To show: $\leqstore{S}{\extS{S}{l}{d_1}{\frozentrue}}$.

      By Definition~\ref{def:leqstore}, we have to show that $\dom{S}
      \subseteq \dom{\extS{S}{l}{d_1}{\frozentrue}}$ and that for all
      $l' \in \dom{S}, \\
      S(l') \leqp (\extS{S}{l}{d_1}{\frozentrue})(l')$.

      \lk{We could spell this out in even more excruciating detail,
        but I think it's obvious enough.}

      By the definition of store update,
      $\extS{S}{l}{d_1}{\frozentrue}$ can only either update an
      existing binding in $S$ or extend $S$ with a new binding.

      Hence $\dom{S} \subseteq \dom{\extS{S}{l}{d_1}{\frozentrue}}$.

      From the premises of {\sc E-Freeze-Final}, $S(l) =
      \state{d_1}{\status_1}$.  Therefore $l \in \dom{S}$.

      Hence $\extS{S}{l}{d_1}{\frozentrue}$ updates the existing
      binding for $l$ in $S$ from $\state{d_1}{\status_1}$ to
      $\state{d_1}{\frozentrue}$.

      By the definition of $\leqp$, $\state{d_1}{\status_1} \leqp
      \state{d_1}{\frozentrue}$.

      $\extS{S}{l}{d_1}{\frozentrue}$ does not update any other
      bindings in $S$, hence, for all $l' \in \dom{S}, \\
      S(l') \leqp (\extS{S}{l}{d_1}{\frozentrue})(l')$.

      Hence $\leqstore{S}{\extS{S}{l}{d_1}{\frozentrue}}$, as
      required.

    \item Case {\sc E-Freeze-Simple}:

      Given: $\config{S}{\freeze{l}} \parstepsto
      \config{\extS{S}{l}{d_1}{\frozentrue}}{d_1}$.

      To show: $\leqstore{S}{\extS{S}{l}{d_1}{\frozentrue}}$.

      Similar to the previous case.

  \end{itemize}

\end{proof}


\section{Proof of Lemma~\ref{lem:independence}}\label{section:independence-proof}
\begin{proof}
  Consider arbitrary $S''$ such that $S''$ is non-conflicting with
  $\config{S}{e} \parstepsto \config{S'}{e'}$ and $\lubstore{S'}{S''}
  \statuseq S$ and $\lubstore{S'}{S''} \neq \topS$.

  To show: $\config{\lubstore{S}{S''}}{e} \parstepsto
  \config{\lubstore{S'}{S''}}{e'}$.

  The proof is by cases on the rule of the reduction semantics by
  which $\config{S}{e}$ steps to $\config{S'}{e'}$.  Since
  $\config{S'}{e'} \neq \error$, we do not need to consider the {\sc
    E-Put-Err} rule.

  The assumption that $\lubstore{S'}{S''} \statuseq S$ is only needed
  in the {\sc E-Freeze-Final} and {\sc E-Freeze-Simple} cases.

  \begin{itemize}

    \item Case {\sc E-Beta}:

      Given: $\config{S}{\app{(\lam{x}{e})}{v}} \parstepsto
      \config{S}{\subst{e}{x}{v}}$.

      To show: $\config{\lubstore{S}{S''}}{\app{(\lam{x}{e})}{v}} \parstepsto
      \config{\lubstore{S}{S''}}{\subst{e}{x}{v}}$.

      Immediate by {\sc E-Beta}.

    \item Case {\sc E-New}:

      Given: $\config{S}{\NEW} \parstepsto
      \config{\extS{S}{l}{\bot}{\frozenfalse}}{l}$.

      To show: $\config{\lubstore{S}{S''}}{\NEW} \parstepsto
      \config{\lubstore{(\extS{S}{l}{\bot}{\frozenfalse})}{S''}}{l}$.

      By {\sc E-New}, we have that $\config{\lubstore{S}{S''}}{\NEW}
      \parstepsto
      \config{\extS{(\lubstore{S}{S''})}{l'}{\bot}{\frozenfalse}}{l'}$,
      where $l' \notin \dom{\lubstore{S}{S''}}$.

      By assumption, $S''$ is non-conflicting with $\config{S}{\NEW}
      \parstepsto \config{\extS{S}{l}{\bot}{\frozenfalse}}{l}$.
 
      Therefore $l \notin \dom{S''}$.

      From the side condition of {\sc E-New}, $l \notin \dom{S}$.

      Therefore $l \notin \dom{\lubstore{S}{S''}}$.

      Therefore, in
      $\config{\extS{(\lubstore{S}{S''})}{l'}{\bot}{\frozenfalse}}{l'}$,
      we can $\alpha$-rename $l'$ to $l$, resulting in
      $\config{\extS{(\lubstore{S}{S''})}{l}{\bot}{\frozenfalse}}{l}$.

      Therefore $\config{\lubstore{S}{S''}}{\NEW} \parstepsto
      \config{\extS{(\lubstore{S}{S''})}{l}{\bot}{\frozenfalse}}{l}$.

      Note that:
      \begin{align*}
        \extS{(\lubstore{S}{S''})}{l}{\bot}{\frozenfalse} &=
        \lubstore{\extS{S}{l}{\bot}{\frozenfalse}}{\extS{S''}{l}{\bot}{\frozenfalse}} \\
        &= \lubstore{\lubstore{S}{\store{\storebinding{l}{\bot}{\frozenfalse}}}}{\lubstore{S''}{\store{\storebinding{l}{\bot}{\frozenfalse}}}} \\
        &= \lubstore{\lubstore{S}{\store{\storebinding{l}{\bot}{\frozenfalse}}}}{S''} \\
        &= \lubstore{\extS{S}{l}{\bot}{\frozenfalse}}{S''}.
      \end{align*}
      Therefore $\config{\lubstore{S}{S''}}{\NEW} \parstepsto
      \config{\lubstore{\extS{S}{l}{\bot}{\frozenfalse}}{S''}}{l}$, as we were
      required to show.

    \item Case {\sc E-Put}:

      Given: $\config{S}{\putiexp{l}} \parstepsto
      \config{\extSRaw{S}{l}{u_{p_i}(p_1)}}{\unit}$.

      To show: $\config{\lubstore{S}{S''}}{\putiexp{l}{d_2}}
      \parstepsto
      \config{\lubstore{\extSRaw{S}{l}{u_{p_i}(p_1)}}{S''}}{\unit}$.

      We will first show that

      $\config{\lubstore{S}{S''}}{\putiexp{l}{d_2}} \parstepsto
      \config{\extSRaw{(\lubstore{S}{S''})}{l}{u_{p_i}(p_1)}}{\unit}$

      and then show why this is sufficient.

      We proceed by cases on $l$:

      \begin{itemize}
        \item $l \notin \dom{S''}$:

          By assumption, $\lubstore{\extSRaw{S}{l}{u_{p_i}(p_1)}}{S''}
          \neq \topS$.

          By Lemma~\ref{lem:monotonicity},
          $\leqstore{S}{\extSRaw{S}{l}{u_{p_i}(p_1)}}$.

          Hence $\lubstore{S}{S''} \neq \topS$.

          Therefore, by Definition~\ref{def:lubstore},
          $(\lubstore{S}{S''})(l) = S(l)$.

          From the premises of {\sc E-Put}, $S(l) = p_1$.

          Hence $(\lubstore{S}{S''})(l) = p_1$.

          From the premises of {\sc E-Put}, $u_{p_i}(p_1) \neq \topp$.

          Therefore, by {\sc E-Put}, we have:
          $\config{\lubstore{S}{S''}}{\putiexp{l}} \parstepsto
          \config{\extSRaw{(\lubstore{S}{S''})}{l}{u_{p_i}(p_1)}}{\unit}$.

        \item $l \in \dom{S''}$:

          By assumption, $\lubstore{\extSRaw{S}{l}{u_{p_i}(p_1)}}{S''} \neq
          \topS$.

          By Lemma~\ref{lem:monotonicity},
          $\leqstore{S}{\extSRaw{S}{l}{u_{p_i}(p_1)}}$.

          Hence $\lubstore{S}{S''} \neq \topS$.

          Therefore $(\lubstore{S}{S''})(l) = \lubp{S(l)}{S''(l)}$.

          From the premises of {\sc E-Put}, $S(l) = p_1$.
          
          Hence $(\lubstore{S}{S''})(l) = p'_1$, where $p_1 \leqp
          p'_1$.

          \TODO{From here forward, this subcase still needs to be
            fixed.}

          By assumption, $\lubstore{\extSRaw{S}{l}{p_2}}{S''} \neq
          \topS$.

          Therefore, by Definition~\ref{def:lubstore},
          $\lubp{p_2}{S''(l)} \neq \topp$.

          Note that:
          \begin{align*}
            \topp &\neq \lubp{p_2}{S''(l)} \\
            &= \lubp{\lubp{p_1}{\state{d_2}{\frozenfalse}}}{S''(l)} \\
            &= \lubp{\lubp{S(l)}{\state{d_2}{\frozenfalse}}}{S''(l)} \\
            &= \lubp{\lubp{S(l)}{S''(l)}}{\state{d_2}{\frozenfalse}} \\
            &= \lubp{(\lubstore{S}{S''})(l)}{\state{d_2}{\frozenfalse}} \\
            &= \lubp{p'_1}{\state{d_2}{\frozenfalse}} \\
            &= p'_2. \\
          \end{align*}
          Hence $p'_2 \neq \topp$.

          Hence $(\lubstore{S}{S''})(l) = p'_1$ and $p'_2 =
          \lubp{p'_1}{\state{d_2}{\frozenfalse}}$ and $p'_2 \neq
          \topp$.

          Therefore, by {\sc E-Put} we have:
          $\config{\lubstore{S}{S''}}{\putiexp{l}{d_2}} \parstepsto
          \config{\extSRaw{(\lubstore{S}{S''})}{l}{p'_2}}{\unit}$.

          \lk{If we really wanted to be pedantic here, we'd actually
            prove that the stores are equal.  I'm assuming that if I
            can show that $\extSRaw{(\lubstore{S}{S''})}{l}{p'_2}$ and
            $\extSRaw{(\lubstore{S}{S''})}{l}{p_2}$ bind $l$ to the
            same value, then it will be obvious that they're equal.}

          Note that:
          \begin{align*}
            (\extSRaw{(\lubstore{S}{S''})}{l}{p'_2})(l) &= \lubp{(\lubstore{S}{S''})(l)}{(\store{\storebindingRaw{l}{p'_2}})(l)} \\
            &= \lubp{p'_1}{p'_2} \\
            &= \lubp{p'_1}{\lubp{p'_1}{\state{d_2}{\frozenfalse}}} \\
            &= \lubp{p'_1}{\state{d_2}{\frozenfalse}}
          \end{align*}
          and
          \begin{align*}
            (\extSRaw{(\lubstore{S}{S''})}{l}{p_2})(l) &= \lubp{(\lubstore{S}{S''})(l)}{(\store{\storebindingRaw{l}{p_2}})(l)} \\
            &= \lubp{p'_1}{p_2} \\
            &= \lubp{p'_1}{\lubp{p_1}{\state{d_2}{\frozenfalse}}} \\
            &= \lubp{p'_1}{\state{d_2}{\frozenfalse}} & \textrm{(since $p_1 \leqp p'_1$).}
          \end{align*}
          Therefore $\extSRaw{(\lubstore{S}{S''})}{l}{p'_2} =
          \extSRaw{(\lubstore{S}{S''})}{l}{p_2}$.

          Therefore, $\config{\lubstore{S}{S''}}{\putiexp{l}{d_2}}
          \parstepsto
          \config{\extSRaw{(\lubstore{S}{S''})}{l}{p_2}}{\unit}$.
      \end{itemize}

      Note that:
      \begin{align*}
        \extSRaw{(\lubstore{S}{S''})}{l}{p_2} &= \lubstore{\extSRaw{S}{l}{p_2}}{\extSRaw{S''}{l}{p_2}} \\
        &= \lubstore{\lubstore{S}{\store{\storebindingRaw{l}{p_2}}}}{\lubstore{S''}{\store{\storebindingRaw{l}{p_2}}}} \\
        &= \lubstore{\lubstore{S}{\store{\storebindingRaw{l}{p_2}}}}{S''} \\
        &= \lubstore{\extSRaw{S}{l}{p_2}}{S''}.
      \end{align*}
      Therefore $\config{\lubstore{S}{S''}}{\putiexp{l}{d_2}}
      \parstepsto \config{\lubstore{\extSRaw{S}{l}{p_2}}{S''}}{\unit}$,
      as we were required to show.

    \item Case {\sc E-Get}:

      Given: $\config{S}{\getexp{l}{P}} \parstepsto \config{S}{p_2}$.

      To show: $\config{\lubstore{S}{S''}}{\getexp{l}{P}} \parstepsto
      \config{\lubstore{S}{S''}}{p_2}$.

      From the premises of {\sc E-Get}, $S(l) = p_1$ and $\incomp{P}$
      and $p_2 \in P$ and $p_2 \leqp p_1$.

      By assumption, $\lubstore{S}{S''} \neq \topS$.

      Hence $(\lubstore{S}{S''}) = p'_1$, where $p_1 \leqp p'_1$.

      By the transitivity of $\leqp$, $p_2 \leqp p'_1$.

      Hence, $S(l) = p'_1$ and $\incomp{P}$ and $p_2 \in P$ and $p_2
      \leqp p'_1$.

      Therefore, by {\sc E-Get},

      $\config{\lubstore{S}{S''}}{\getexp{l}{P}} \parstepsto
      \config{\lubstore{S}{S''}}{p_2}$,

      as we were required to show.

    \item Case {\sc E-Freeze-Init}:

      Given: $\config{S}{\freezeafter{l}{Q}{\lam{x}{e}}} \parstepsto
      \config{S}{\freezeafterfull{l}{Q}{\lam{x}{e}}{\setof{}}{\setof{}}}$.

      To show:
      $\config{\lubstore{S}{S''}}{\freezeafter{l}{Q}{\lam{x}{e}}}
      \parstepsto
      \config{\lubstore{S}{S''}}{\freezeafterfull{l}{Q}{\lam{x}{e}}{\setof{}}{\setof{}}}$.

      Immediate by {\sc E-Freeze-Init}.

    \item Case {\sc E-Spawn-Handler}:

      Given:

      $\config{S}{\freezeafterfull{l}{Q}{\lam{x}{e_0}}{\setof{e,
            \dots}}{H}} \parstepsto
      \config{S}{\freezeafterfull{l}{Q}{\lam{x}{e_0}}{\setof{\subst{e_0}{x}{d_2},
            e, \dots}} {\{d_2\}\cup H}}$.

      To show:

      $\config{\lubstore{S}{S''}}{\freezeafterfull{l}{Q}{\lam{x}{e_0}}{\setof{e,
            \dots}}{H}} \parstepsto
      \config{\lubstore{S}{S''}}{\freezeafterfull{l}{Q}{\lam{x}{e_0}}{\setof{\subst{e_0}{x}{d_2},
            e, \dots}} {\{d_2\}\cup H}}$.

      From the premises of {\sc E-Spawn-Handler}, $S(l) =
      \state{d_1}{\status_1}$ and $d_2 \userleq d_1$ and $d_2 \notin
      H$ and $d_2 \in Q$.

      By assumption, $\lubstore{S}{S''} \neq \topS$.

      Hence $(\lubstore{S}{S''})(l) = \state{d'_1}{\status'_1}$ where
      $\state{d_1}{\status_1} \leqp \state{d'_1}{\status'_1}$.

      By Definition~\ref{def:lattice-with-status-bits}, $d_1 \userleq
      d'_1$.

      By the transitivity of $\userleq$, $d_2 \userleq d'_1$.

      Hence $(\lubstore{S}{S''})(l) =
      \state{d'_1}{\status'_1}$ and $d_2 \userleq d'_1$ and $d_2 \notin
      H$ and $d_2 \in Q$.

      Therefore, by {\sc E-Spawn-Handler},

      $\config{\lubstore{S}{S''}}{\freezeafterfull{l}{Q}{\lam{x}{e_0}}{\setof{e,
            \dots}}{H}} \parstepsto
      \config{\lubstore{S}{S''}}{\freezeafterfull{l}{Q}{\lam{x}{e_0}}{\setof{\subst{e_0}{x}{d_2},
            e, \dots}} {\{d_2\}\cup H}}$,

      as we were required to show.

    \item Case {\sc E-Freeze-Final}:

      \lk{This case wouldn't work but for the $\lubstore{S'}{S''}
        \statuseq S$ requirement, which makes it a no-op freeze.}

      Given:
      $\config{S}{\freezeafterfull{l}{Q}{\lam{x}{e_0}}{\setof{v,
            \dots}}{H}} \parstepsto
      \config{\extS{S}{l}{d_1}{\frozentrue}}{d_1}$.

      To show:
      $\config{\lubstore{S}{S''}}{\freezeafterfull{l}{Q}{\lam{x}{e_0}}{\setof{v,
            \dots}}{H}} \parstepsto
      \config{\lubstore{\extS{S}{l}{d_1}{\frozentrue}}{S''}}{d_1}$.

      We will first show that

      $\config{\lubstore{S}{S''}}{\freezeafterfull{l}{Q}{\lam{x}{e_0}}{\setof{v,
            \dots}}{H}} \parstepsto
      \config{\extS{(\lubstore{S}{S''})}{l}{d_1}{\frozentrue}}{d_1}$

      and then show why this is sufficient.

      We proceed by cases on $l$:
      \begin{itemize}
      \item $l \notin \dom{S''}$:

        By assumption, $\lubstore{\extS{S}{l}{d_1}{\frozentrue}}{S''}
        \neq \topS$.

        By Lemma~\ref{lem:monotonicity},
        $\leqstore{S}{\extS{S}{l}{d_1}{\frozentrue}}$.

        Therefore $\lubstore{S}{S''} \neq \topS$.

        Hence, by Definition~\ref{def:lubstore},
        $(\lubstore{S}{S''})(l) = S(l)$.

        From the premises of {\sc E-Freeze-Final}, we have that $S(l)
        = \state{d_1}{\status_1}$.

        Hence $(\lubstore{S}{S''})(l) = \state{d_1}{\status_1}$.

        From the premises of {\sc E-Freeze-Final}, we have that
        $\forall{d_2} ~.~ ( {d_2 \userleq d_1 \land d_2 \in Q} \Rightarrow d_2 \in
        H)$.

        Therefore, by {\sc E-Freeze-Final}, we have that

        $\config{\lubstore{S}{S''}}{\freezeafterfull{l}{Q}{\lam{x}{e_0}}{\setof{v,
              \dots}}{H}} \parstepsto
        \config{\extS{(\lubstore{S}{S''})}{l}{d_1}{\frozentrue}}{d_1}$.


      \item $l \in \dom{S''}$:

        By assumption, $\lubstore{\extS{S}{l}{d_1}{\frozentrue}}{S''}
        \neq \topS$.

        By Lemma~\ref{lem:monotonicity},
        $\leqstore{S}{\extS{S}{l}{d_1}{\frozentrue}}$.

        Therefore $\lubstore{S}{S''} \neq \topS$.

        Hence, by Definition~\ref{def:lubstore},
        $(\lubstore{S}{S''})(l) = \lubp{S(l)}{S''(l)}$.

        From the premises of {\sc E-Freeze-Final}, we have that
        $S(l) = \state{d_1}{\status_1}$.

        By assumption, $\lubstore{\extS{S}{l}{d_1}{\frozentrue}}{S''}
        \statuseq S$.

        Therefore $\status_1 = \frozentrue$.

        Therefore $S(l) = \state{d_1}{\frozentrue}$.

        Therefore $(\lubstore{S}{S''})(l) =
        \lubp{\state{d_1}{\frozentrue}}{S''(l)}$.

        We proceed by cases on $S''(l)$:
        \begin{itemize}
        \item $S''(l) = \state{d_3}{\frozenfalse}$, where $d_3 \userleq d_1$:

          By Definition~\ref{def:lubp},
          $\lubp{\state{d_1}{\frozentrue}}{\state{d_3}{\frozenfalse}}
          = \state{d_1}{\frozentrue}$.

          Therefore $(\lubstore{S}{S''})(l) =
          \state{d_1}{\frozentrue}$.

          From the premises of {\sc E-Freeze-Final}, we have that
          $\forall{d_2} ~.~ ( {d_2 \userleq d_1 \land d_2 \in Q} \Rightarrow d_2 \in
          H)$.

          Therefore, by {\sc E-Freeze-Final}, we have that

          $\config{\lubstore{S}{S''}}{\freezeafterfull{l}{Q}{\lam{x}{e_0}}{\setof{v,
                \dots}}{H}} \parstepsto
          \config{\extS{(\lubstore{S}{S''})}{l}{d_1}{\frozentrue}}{d_1}$.

        \item $S''(l) = \state{d_3}{\frozenfalse}$, where $d_3 \nuserleq d_1$:

          By Definition~\ref{def:lubp},
          $\lubp{\state{d_1}{\frozentrue}}{\state{d_3}{\frozenfalse}}
          = \state{\top}{\frozenfalse}$.

          Therefore $\lubp{S(l)}{S''(l)} =
          \state{\top}{\frozenfalse}$.

          By Definition~\ref{def:lattice-with-status-bits},
          $\state{\top}{\frozenfalse} = \topp$.

          Therefore $\lubp{S(l)}{S''(l)} = \topp$.

          Therefore, by Definition~\ref{def:lubstore},
          $\lubstore{S}{S''} = \topS$.

          This is a contradiction.

          Therefore,

          $\config{\lubstore{S}{S''}}{\freezeafterfull{l}{Q}{\lam{x}{e_0}}{\setof{v,
                \dots}}{H}} \parstepsto
          \config{\extS{(\lubstore{S}{S''})}{l}{d_1}{\frozentrue}}{d_1}$.

        \item $S''(l) = \state{d_3}{\frozentrue}$, where $d_3 = d_1$:

          By Definition~\ref{def:lubp},
          $\lubp{\state{d_1}{\frozentrue}}{\state{d_3}{\frozentrue}} =
          \state{d_1}{\frozentrue}$.

          Therefore $(\lubstore{S}{S''})(l) = \state{d_1}{\frozentrue}$.

          From the premises of {\sc E-Freeze-Final}, we have that
          $\forall{d_2} ~.~ ( {d_2 \userleq d_1 \land d_2 \in Q} \Rightarrow d_2 \in
          H)$.

          Therefore, by {\sc E-Freeze-Final}, we have that

          $\config{\lubstore{S}{S''}}{\freezeafterfull{l}{Q}{\lam{x}{e_0}}{\setof{v,
                \dots}}{H}} \parstepsto
          \config{\extS{(\lubstore{S}{S''})}{l}{d_1}{\frozentrue}}{d_1}$.

        \item $S''(l) = \state{d_3}{\frozentrue}$, where $d_3 \neq d_1$:

          By Definition~\ref{def:lubp},
          $\lubp{\state{d_1}{\frozentrue}}{\state{d_3}{\frozentrue}}
          = \state{\top}{\frozenfalse}$.

          Therefore $\lubp{S(l)}{S''(l)} = \state{\top}{\frozenfalse}$.

          By Definition~\ref{def:lattice-with-status-bits},
          $\state{\top}{\frozenfalse} = \topp$.

          Therefore $\lubp{S(l)}{S''(l)} = \topp$.

          Therefore, by Definition~\ref{def:lubstore},
          $\lubstore{S}{S''} = \topS$.

          This is a contradiction.

          Therefore,

          $\config{\lubstore{S}{S''}}{\freezeafterfull{l}{Q}{\lam{x}{e_0}}{\setof{v,
                \dots}}{H}} \parstepsto
          \config{\extS{(\lubstore{S}{S''})}{l}{d_1}{\frozentrue}}{d_1}$.
        \end{itemize}
      \end{itemize}

      In each case we have shown that

      $\config{\lubstore{S}{S''}}{\freezeafterfull{l}{Q}{\lam{x}{e_0}}{\setof{v,
            \dots}}{H}} \parstepsto
      \config{\extS{(\lubstore{S}{S''})}{l}{d_1}{\frozentrue}}{d_1}$.

      Note that:
      \begin{align*}
        \extS{(\lubstore{S}{S''})}{l}{d_1}{\frozentrue} &=
        \lubstore{\extS{S}{l}{d_1}{\frozentrue}}{\extS{S''}{l}{d_1}{\frozentrue}} \\
        &= \lubstore{\lubstore{S}{\store{\storebinding{l}{d_1}{\frozentrue}}}}{\lubstore{S''}{\store{\storebinding{l}{d_1}{\frozentrue}}}} \\
        &= \lubstore{\lubstore{S}{\store{\storebinding{l}{d_1}{\frozentrue}}}}{S''} \\
        &= \lubstore{\extS{S}{l}{d_1}{\frozentrue}}{S''}.
      \end{align*}
      Therefore

      $\config{\lubstore{S}{S''}}{\freezeafterfull{l}{Q}{\lam{x}{e_0}}{\setof{v,
            \dots}}{H}} \parstepsto
      \config{\lubstore{\extS{S}{l}{d_1}{\frozentrue}}{S''}}{d_1}$,

      as we were required to show.

    \item Case {\sc E-Freeze-Simple}:

      Given: $\config{S}{\freeze{l}} \parstepsto
      \config{\extS{S}{l}{d_1}{\frozentrue}}{d_1}$.

      To show: $\config{\lubstore{S}{S''}}{\freeze{l}}
      \parstepsto
      \config{\lubstore{\extS{S}{l}{d_1}{\frozentrue}}{S''}}{d_1}$.

      We will first show that

      $\config{\lubstore{S}{S''}}{\freeze{l}} \parstepsto
      \config{\extS{(\lubstore{S}{S''})}{l}{d_1}{\frozentrue}}{d_1}$

      and then show why this is sufficient.

      We proceed by cases on $l$:
      \begin{itemize}
      \item $l \notin \dom{S''}$:

        By assumption, $\lubstore{\extS{S}{l}{d_1}{\frozentrue}}{S''}
        \neq \topS$.

        By Lemma~\ref{lem:monotonicity},
        $\leqstore{S}{\extS{S}{l}{d_1}{\frozentrue}}$.

        Therefore $\lubstore{S}{S''} \neq \topS$.

        Hence, by Definition~\ref{def:lubstore},
        $(\lubstore{S}{S''})(l) = S(l)$.

        From the premise of {\sc E-Freeze-Simple}, we have that
        $S(l) = \state{d_1}{\status_1}$.

        Therefore, by {\sc E-Freeze-Simple}, we have that

        $\config{\lubstore{S}{S''}}{\freeze{l}}
        \parstepsto
        \config{\extS{(\lubstore{S}{S''})}{l}{d_1}{\frozentrue}}{d_1}$.

      \item $l \in \dom{S''}$:

        By assumption, $\lubstore{\extS{S}{l}{d_1}{\frozentrue}}{S''}
        \neq \topS$.

        By Lemma~\ref{lem:monotonicity},
        $\leqstore{S}{\extS{S}{l}{d_1}{\frozentrue}}$.

        Therefore $\lubstore{S}{S''} \neq \topS$.

        Hence, by Definition~\ref{def:lubstore},
        $(\lubstore{S}{S''})(l) = \lubp{S(l)}{S''(l)}$.

        From the premise of {\sc E-Freeze-Simple}, we have that
        $S(l) = \state{d_1}{\status_1}$.

        By assumption, $\lubstore{\extS{S}{l}{d_1}{\frozentrue}}{S''}
        \statuseq S$.

        Therefore $\status_1 = \frozentrue$.

        Therefore $(\lubstore{S}{S''})(l) =
        \lubp{\state{d_1}{\frozentrue}}{S''(l)}$.
        
        We proceed by cases on $S''(l)$:
        \begin{itemize}
        \item $S''(l) = \state{d_2}{\frozenfalse}$, where $d_2 \userleq d_1$:

          By Definition~\ref{def:lubp},
          $\lubp{\state{d_1}{\frozentrue}}{\state{d_2}{\frozenfalse}} =
          \state{d_1}{\frozentrue}$.

          Therefore $(\lubstore{S}{S''})(l) =
          \state{d_1}{\frozentrue}$.

          Therefore, by {\sc E-Freeze-Simple}, we have that

          $\config{\lubstore{S}{S''}}{\freeze{l}}
          \parstepsto
          \config{\extS{(\lubstore{S}{S''})}{l}{d_1}{\frozentrue}}{d_1}$.

        \item $S''(l) = \state{d_2}{\frozenfalse}$, where $d_2 \nuserleq d_1$:

          By Definition~\ref{def:lubp},
          $\lubp{\state{d_1}{\frozentrue}}{\state{d_2}{\frozenfalse}}
          = \state{\top}{\frozenfalse}$.

          By Definition~\ref{def:lattice-with-status-bits},
          $\state{\top}{\frozenfalse} = \topp$.

          Therefore $\lubp{S(l)}{S''(l)} = \topp$.

          Therefore, by Definition~\ref{def:lubstore},
          $\lubstore{S}{S''} = \topS$.

          This is a contradiction.

          Therefore,

          $\config{\lubstore{S}{S''}}{\freeze{l}}
          \parstepsto
          \config{\extS{(\lubstore{S}{S''})}{l}{d_1}{\frozentrue}}{d_1}$.

        \item $S''(l) = \state{d_2}{\frozentrue}$, where $d_2 = d_1$:

          Therefore $(\lubstore{S}{S''})(l) =
          \lubp{\state{d_1}{\frozentrue}}{\state{d_2}{\frozentrue}}$.

          By Definition~\ref{def:lubp},
          $\lubp{\state{d_1}{\frozentrue}}{\state{d_2}{\frozentrue}} =
          \state{d_1}{\frozentrue}$.

          Therefore $(\lubstore{S}{S''})(l) =
          \state{d_1}{\frozentrue}$.

          Therefore, by {\sc E-Freeze-Simple}, we have that

          $\config{\lubstore{S}{S''}}{\freeze{l}}
          \parstepsto
          \config{\extS{(\lubstore{S}{S''})}{l}{d_1}{\frozentrue}}{d_1}$.

        \item $S''(l) = \state{d_2}{\frozentrue}$, where $d_2 \neq d_1$:

          By Definition~\ref{def:lubp},
          $\lubp{\state{d_1}{\frozentrue}}{\state{d_2}{\frozentrue}}
          = \state{\top}{\frozenfalse}$.

          By Definition~\ref{def:lattice-with-status-bits},
          $\state{\top}{\frozenfalse} = \topp$.

          Therefore $\lubp{S(l)}{S''(l)} = \topp$.

          Therefore, by Definition~\ref{def:lubstore},
          $\lubstore{S}{S''} = \topS$.

          This is a contradiction.

          Therefore,

          $\config{\lubstore{S}{S''}}{\freeze{l}}
          \parstepsto
          \config{\extS{(\lubstore{S}{S''})}{l}{d_1}{\frozentrue}}{d_1}$.
        \end{itemize}
      \end{itemize}

      In each case we have shown that

      $\config{\lubstore{S}{S''}}{\freeze{l}} \parstepsto
      \config{\extS{(\lubstore{S}{S''})}{l}{d_1}{\frozentrue}}{d_1}$.

      Note that:
      \begin{align*}
        \extS{(\lubstore{S}{S''})}{l}{d_1}{\frozentrue} &=
        \lubstore{\extS{S}{l}{d_1}{\frozentrue}}{\extS{S''}{l}{d_1}{\frozentrue}} \\
        &= \lubstore{\lubstore{S}{\store{\storebinding{l}{d_1}{\frozentrue}}}}{\lubstore{S''}{\store{\storebinding{l}{d_1}{\frozentrue}}}} \\
        &= \lubstore{\lubstore{S}{\store{\storebinding{l}{d_1}{\frozentrue}}}}{S''} \\
        &= \lubstore{\extS{S}{l}{d_1}{\frozentrue}}{S''}.
      \end{align*}
      Therefore
      $\config{\lubstore{S}{S''}}{\freeze{l}}
      \parstepsto
      \config{\lubstore{\extS{S}{l}{d_1}{\frozentrue}}{S''}}{d_1}$,
      as we were required to show.
  \end{itemize}
\end{proof}


\section{Proof of Lemma~\ref{lem:strong-local-quasi-confluence}}\label{section:strong-local-quasi-confluence-proof}
\begin{proof}
  Suppose $\conf \ctxstepsto \conf_a$ and $\conf \ctxstepsto \conf_b$.

  We have to show that either there exist $\conf_c, i, j, \pi$ such
  that $\conf_a \ctxstepsto^i \conf_c$ and $\pi(\conf_b) \ctxstepsto^j
  \conf_c$ and $i \leq 1$ and $j \leq 1$, or that $\conf_a \ctxstepsto
  \error$ or $\conf_b \ctxstepsto \error$.

  By inspection of the operational semantics, it must be the case that
  $\conf$ steps to $\conf_a$ by the {\sc E-Eval-Ctxt} rule.

  Let $\conf = \config{S}{\evalctxt{E_a}{e_{a_1}}}$ and let $\conf_a =
  \config{S_a}{\evalctxt{E_a}{e_{a_2}}}$.

  Likewise, it must be the case that $\conf$ steps to $\conf_b$ by the
  {\sc E-Eval-Ctxt} rule.

  Let $\conf = \config{S}{\evalctxt{E_b}{e_{b_1}}}$ and let $\conf_b =
  \config{S_b}{\evalctxt{E_b}{e_{b_2}}}$.

  Note that $\conf = \config{S}{\evalctxt{E_a}{e_{a_1}}} =
  \config{S}{\evalctxt{E_b}{e_{b_1}}}$, and so
  $\evalctxt{E_a}{e_{a_1}} = \evalctxt{E_b}{e_{b_1}}$, but $E_a$ and
  $E_b$ may differ and $e_{a_1}$ and $e_{b_1}$ may differ.

  First, consider the possibility that $E_a = E_b$ (and $e_{a_1} =
  e_{b_1}$).

  Since $\config{S}{\evalctxt{E_a}{e_{a_1}}} \ctxstepsto
  \config{S_a}{\evalctxt{E_a}{e_{a_2}}}$ by {\sc E-Eval-Ctxt} and
  $\config{S}{\evalctxt{E_b}{e_{b_1}}} \ctxstepsto
  \config{S_b}{\evalctxt{E_b}{e_{b_2}}}$ by {\sc E-Eval-Ctxt}, we have
  from the premise of {\sc E-Eval-Ctxt} that $\config{S}{e_{a_1}}
  \parstepsto \config{S_a}{e_{a_2}}$ and $\config{S}{e_{b_1}}
  \parstepsto \config{S_b}{e_{b_2}}$.

  But then, since $e_{a_1} = e_{b_1}$, by Internal Determinism
  (Lemma~\ref{lem:internal-determinism}) there is a permutation $\pi'$
  such that $\config{S_a}{e_{a_2}} = \pi'(\config{S_b}{e_{b_2}})$,
  modulo choice of events.

  We have two cases:

  \begin{itemize}
  \item In the case where the steps $\conf \ctxstepsto \conf_a$ and
    $\conf \ctxstepsto \conf_b$ are both by {\sc E-Spawn-Handler} and
    they handle different events $d_2$ and $d'_2$, then we can satisfy
    the proof by choosing the final configuration $\conf_c$ as the
    configuration where both $d_2$ and $d'_2$ have been handled.

    Both $\conf_a$ and $\conf_b$ can step to this configuration by
    {\sc E-Spawn-Handler}: if the step from $\conf$ to $\conf_a$
    handles $d_2$ then the step from $\conf_a$ to $\conf_c$ handles
    $d'_2$, while if the step from $\conf$ to $\conf_b$ handles $d'_2$
    then the step from $\conf_b$ to $\conf_c$ handles $d_2$.

    The store in the final configuration is $S_a$ or $S_b$, which are
    equal because {\sc E-Spawn-Handler} does not affect the store, and
    we can satisfy the proof by choosing $i = 1$ and $j = 0$ and $\pi
    = \id$.

  \item Otherwise, we can satisfy the proof by choosing $\conf_c =
    \config{S_a}{e_{a_2}}$ and $i = 0$ and $j = 0$ and $\pi = \id$.
  \end{itemize}

  The rest of this proof deals with the more interesting case in which
  $E_a \neq E_b$ (and $e_{a_1} \neq e_{b_1}$).

  Since $\config{S}{\evalctxt{E_a}{e_{a_1}}} \ctxstepsto
  \config{S_a}{\evalctxt{E_a}{e_{a_2}}}$ and
  $\config{S}{\evalctxt{E_b}{e_{b_1}}} \ctxstepsto
  \config{S_b}{\evalctxt{E_b}{e_{b_2}}}$ and $\evalctxt{E_a}{e_{a_1}}
  = \evalctxt{E_b}{e_{b_1}}$, and since $E_a \neq E_b$, we have from
  Lemma~\ref{lem:locality} (Locality) that there exist evaluation
  contexts $E'_a$ and $E'_b$ such that:

  \begin{itemize}
  \item $\evalctxt{E'_a}{e_{a_1}} = \evalctxt{E_b}{e_{b_2}}$, and
  \item $\evalctxt{E'_b}{e_{b_1}} = \evalctxt{E_a}{e_{a_2}}$, and
  \item $\evalctxt{E'_a}{e_{a_2}} =
    \evalctxt{E'_b}{e_{b_2}}$.
  \end{itemize}

  In some of the cases that follow, we will choose $\conf_c = \error$,
  and in some we will prove that one of $\conf_a$ or $\conf_b$ steps
  to $\error$.

  In most cases, however, our approach will be to show that there
  exist $S', i, j, \pi$ such that:
  \begin{itemize}
  \item $\config{S_a}{\evalctxt{E_a}{e_{a_2}}} \ctxstepsto^i
    \config{S'}{\evalctxt{E'_a}{e_{a_2}}}$, and
  \item $\pi(\config{S_b}{\evalctxt{E_b}{e_{b_2}}}) \ctxstepsto^j
    \config{S'}{\evalctxt{E'_a}{e_{a_2}}}$.
  \end{itemize}
  Since $\evalctxt{E'_a}{e_{a_1}} = \evalctxt{E_b}{e_{b_2}}$,
  $\evalctxt{E'_b}{e_{b_1}} = \evalctxt{E_a}{e_{a_2}}$, and
  $\evalctxt{E'_a}{e_{a_2}} = \evalctxt{E'_b}{e_{b_2}}$, it suffices
  to show that:
  \begin{itemize}
  \item $\config{S_a}{\evalctxt{E'_b}{e_{b_1}}} \ctxstepsto^i
    \config{S'}{\evalctxt{E'_b}{e_{b_2}}}$, and
  \item $\pi(\config{S_b}{\evalctxt{E'_a}{e_{a_1}}}) \ctxstepsto^j
    \config{S'}{\evalctxt{E'_a}{e_{a_2}}}$.
  \end{itemize}
  From the premise of {\sc E-Eval-Ctxt}, we have that
  $\config{S}{e_{a_1}} \parstepsto \config{S_a}{e_{a_2}}$ and
  $\config{S}{e_{b_1}} \parstepsto \config{S_b}{e_{b_2}}$.

  We proceed by case analysis on the rule by which
  $\config{S}{e_{a_1}}$ steps to $\config{S_a}{e_{a_2}}$.

  Since the only way an $\error$ configuration can arise is by the
  {\sc E-Put-Err} rule, we can assume in all other cases that $\conf_a
  \neq \error$.

  \begin{enumerate}
  \item Case {\sc E-Beta}: We have $S_a = S$.

    We proceed by case analysis on the rule by which
    $\config{S}{e_{b_1}}$ steps to $\config{S_b}{e_{b_2}}$.

    Since the only way an $\error$ configuration can arise is by the
    {\sc E-Put-Err} rule, we can assume in all other cases that
    $\conf_b \neq \error$.
    \begin{enumerate}
    \item \label{slqc-beta-beta}Case {\sc E-Beta}: We have $S_a = S$
      and $S_b = S$.

      Choose $S' = S = S_a = S_b$, $i = 1$, $j = 1$, and $\pi = \id$.

      We have to show that:
      \begin{itemize}
      \item $\config{S}{\evalctxt{E'_b}{e_{b_1}}} \ctxstepsto
        \config{S_a}{\evalctxt{E'_b}{e_{b_2}}}$, and
      \item $\config{S}{\evalctxt{E'_a}{e_{a_1}}} \ctxstepsto
        \config{S_b}{\evalctxt{E'_a}{e_{a_2}}}$, 
      \end{itemize}

      both of which follow immediately from $\config{S}{e_{a_1}}
      \parstepsto \config{S_a}{e_{a_2}}$ and $\config{S}{e_{b_1}}
      \parstepsto \config{S_b}{e_{b_2}}$ and {\sc E-Eval-Ctxt}.

    \item \label{slqc-beta-new}Case {\sc E-New}: We have $S_a = S$ and
      $S_b = \extS{S}{l}{\bot}{\frozenfalse}$.

      Choose $S' = S_b$, $i = 1$, $j = 1$, and $\pi = \id$.

      We have to show that:
      \begin{itemize}
      \item $\config{S}{\evalctxt{E'_b}{e_{b_1}}} \ctxstepsto
        \config{S_b}{\evalctxt{E'_b}{e_{b_2}}}$, and
      \item $\config{S_b}{\evalctxt{E'_a}{e_{a_1}}} \ctxstepsto
        \config{S_b}{\evalctxt{E'_a}{e_{a_2}}}$.
      \end{itemize}

      The first of these follows immediately from $\config{S}{e_{b_1}}
      \parstepsto \config{S_b}{e_{b_2}}$ and {\sc E-Eval-Ctxt}.

      For the second, consider that $S_b =
      \extS{S}{l}{\bot}{\frozenfalse} = U_S(S)$, where $U_S$ is the
      store update operation that acts as the identity on the contents
      of all existing locations, and adds the binding
      $\storebinding{l}{\bot}{\frozenfalse}$ if no binding for $l$
      exists.

      Note that:
      \begin{itemize}
      \item $U_S$ is non-conflicting with $\config{S}{e_{a_1}}
        \parstepsto \config{S_a}{e_{a_2}}$, since no locations are
        allocated in the transition;
      \item $U_S(S_a) \neq \topS$, since $U_S(S_a) = U_S(S) = S_b$
        and we know that $\conf_b \neq \error$; and
      \item $U_S$ is freeze-safe with $\config{S}{e_{a_1}}
        \parstepsto \config{S_a}{e_{a_2}}$, since $S_a = S$, so
        there are no locations whose contents differ in status
        between them.
      \end{itemize}

      Therefore, by Lemma~\ref{lem:generalized-independence}
      (Generalized Independence), we have that

      $\config{U_S(S)}{e_{a_1}} \parstepsto
      \config{U_S(S_a)}{e_{a_2}}$.

      Hence $\config{S_b}{e_{a_1}} \parstepsto \config{S_b}{e_{a_2}}$.

      By {\sc E-Eval-Ctxt}, it follows that
      $\config{S_b}{\evalctxt{E'_a}{e_{a_1}}} \ctxstepsto
      \config{S_b}{\evalctxt{E'_a}{e_{a_2}}}$,
      as we were required to show.

    \item \label{slqc-beta-put}Case {\sc E-Put}: We have $S_a = S$ and
      $S_b = \extSRaw{S}{l}{u_{p_i}(p_1)}$.

      Choose $S' = S_b$, $i = 1$, $j = 1$, and $\pi = \id$.

      We have to show that:
      \begin{itemize}
      \item $\config{S}{\evalctxt{E'_b}{e_{b_1}}} \ctxstepsto
        \config{S_b}{\evalctxt{E'_b}{e_{b_2}}}$, and
      \item $\config{S_b}{\evalctxt{E'_a}{e_{a_1}}} \ctxstepsto
        \config{S_b}{\evalctxt{E'_a}{e_{a_2}}}$.
      \end{itemize}

      The first of these follows immediately from $\config{S}{e_{b_1}}
      \parstepsto \config{S_b}{e_{b_2}}$ and {\sc E-Eval-Ctxt}.

      For the second, consider that $S_b = U_S(S)$, where $U_S$ is the
      store update operation that applies $u_{p_i}$ to the contents of
      $l$ and acts as the identity on all other locations.

      Note that:
      \begin{itemize}
      \item $U_S$ is non-conflicting with $\config{S}{e_{a_1}}
        \parstepsto \config{S_a}{e_{a_2}}$, since no locations are
        allocated in the transition;
      \item $U_S(S_a) \neq \topS$, since $U_S(S_a) = U_S(S) = S_b$
        and we know that $\conf_b \neq \error$; and
      \item $U_S$ is freeze-safe with $\config{S}{e_{a_1}}
        \parstepsto \config{S_a}{e_{a_2}}$, since $S_a = S$, so
        there are no locations whose contents differ in status
        between them.
      \end{itemize}

      Therefore, by Lemma~\ref{lem:generalized-independence}
      (Generalized Independence), we have that

      $\config{U_S(S)}{e_{a_1}} \parstepsto
      \config{U_S(S_a)}{e_{a_2}}$.

      Hence $\config{S_b}{e_{a_1}} \parstepsto \config{S_b}{e_{a_2}}$.

      By {\sc E-Eval-Ctxt}, it follows that
      $\config{S_b}{\evalctxt{E'_a}{e_{a_1}}} \ctxstepsto
      \config{S_b}{\evalctxt{E'_a}{e_{a_2}}}$, as we were required to
      show.

    \item \label{slqc-beta-put-err}Case {\sc E-Put-Err}: We have $S_a
      = S$ and $\config{S_b}{e_{b_2}} = \error$, and so we choose
      $\conf_c = \error$, $i = 1$, $j = 0$, and $\pi = \id$.

      We have to show that:
      \begin{itemize}
      \item $\config{S}{\evalctxt{E'_b}{e_{b_1}}} \ctxstepsto \error$,
        and
      \item $\config{S_b}{\evalctxt{E'_a}{e_{a_1}}} = \error$.
      \end{itemize}

      The second of these is immediately true because since
      $\config{S_b}{e_{b_2}} = \error$, $S_b = \topS$, and so
      $\config{S_b}{\evalctxt{E'_a}{e_{a_1}}}$ is equal to $\error$ as
      well.

      For the first, observe that $\config{S}{e_{b_1}} \parstepsto
      \config{S_b}{e_{b_2}}$, hence by {\sc E-Eval-Ctxt},
      $\config{S}{\evalctxt{E'_b}{e_{b_1}}} \ctxstepsto
      \config{S_b}{\evalctxt{E'_b}{e_{b_2}}}$.

      But $S_b = \topS$, so $\config{S_b}{\evalctxt{E'_b}{e_{b_2}}}$
      is equal to $\error$, and so
      $\config{S}{\evalctxt{E'_b}{e_{b_1}}} \ctxstepsto \error$, as
      required.

    \item \label{slqc-beta-get}Case {\sc E-Get}: Similar to
      case~\ref{slqc-beta-beta}, since $S_a = S$ and $S_b = S$.
    \item \label{slqc-beta-freeze-init}Case {\sc E-Freeze-Init}:
      Similar to case~\ref{slqc-beta-beta}, since $S_a = S$ and $S_b =
      S$.
    \item \label{slqc-beta-spawn-handler}Case {\sc E-Spawn-Handler}:
      Similar to case~\ref{slqc-beta-beta}, since $S_a = S$ and $S_b =
      S$.
    \item \label{slqc-beta-freeze-final}Case {\sc E-Freeze-Final}: We
      have $S_a = S$ and $S_b = \extS{S}{l}{d_1}{\frozentrue}$.

      Choose $S' = S_b$, $i = 1$, $j = 1$, and $\pi = \id$.

      We have to show that:
      \begin{itemize}
      \item $\config{S}{\evalctxt{E'_b}{e_{b_1}}} \ctxstepsto
        \config{S_b}{\evalctxt{E'_b}{e_{b_2}}}$, and
      \item $\config{S_b}{\evalctxt{E'_a}{e_{a_1}}} \ctxstepsto
        \config{S_b}{\evalctxt{E'_a}{e_{a_2}}}$.
      \end{itemize}

      The first of these follows immediately from $\config{S}{e_{b_1}}
      \parstepsto \config{S_b}{e_{b_2}}$ and {\sc E-Eval-Ctxt}.

      For the second, note that $S_b = U_S(S)$, where $U_S$ is the
      store update operation that freezes the contents of $l$ and acts
      as the identity on the contents of all other locations.

      Note that:
      \begin{itemize}
      \item $U_S$ is non-conflicting with $\config{S}{e_{a_1}}
        \parstepsto \config{S_a}{e_{a_2}}$, since no locations are
        allocated in the transition;
      \item $U_S(S_a) \neq \topS$, since $U_S(S_a) = U_S(S) = S_b$
        and we know that $\conf_b \neq \error$; and
      \item $U_S$ is freeze-safe with $\config{S}{e_{a_1}}
        \parstepsto \config{S_a}{e_{a_2}}$, since $S_a = S$, so
        there are no locations whose contents differ in status
        between them.
      \end{itemize}

      Therefore, by Lemma~\ref{lem:generalized-independence}
      (Generalized Independence), we have that

      $\config{U_S(S)}{e_{a_1}} \parstepsto
      \config{U_S(S_a)}{e_{a_2}}$.

      Hence $\config{S_b}{e_{a_1}} \parstepsto \config{S_b}{e_{a_2}}$.

      By {\sc E-Eval-Ctxt}, it follows that
      $\config{S_b}{\evalctxt{E'_a}{e_{a_1}}} \ctxstepsto
      \config{S_b}{\evalctxt{E'_a}{e_{a_2}}}$, as we were required to
      show.

    \item \label{slqc-beta-freeze-simple}Case {\sc E-Freeze-Simple}:
      Similar to case~\ref{slqc-beta-freeze-final}, since $S_b =
      \extS{S}{l}{d_1}{\frozentrue}$.

    \end{enumerate}
  \item Case {\sc E-New}: We have $S_a = \extS{S}{l}{\bot}{\frozenfalse}$.

    We proceed by case analysis on the rule by which
    $\config{S}{e_{b_1}}$ steps to $\config{S_b}{e_{b_2}}$.

    Since the only way an $\error$ configuration can arise is by the
    {\sc E-Put-Err} rule, we can assume in all other cases that
    $\conf_b \neq \error$.
    \begin{enumerate}
    \item \label{slqc-new-beta}Case {\sc E-Beta}: By symmetry with case~\ref{slqc-beta-new}.
    \item \label{slqc-new-new}Case {\sc E-New}: We have $S_a =
      \extS{S}{l}{\bot}{\frozenfalse}$ and $S_b =
      \extS{S}{l'}{\bot}{\frozenfalse}$.

      Now consider whether $l = l'$:
      \begin{itemize}
      \item If $l \neq l'$:

        Choose $S' =
        \extS{\extS{S}{l'}{\bot}{\frozenfalse}}{l}{\bot}{\frozenfalse}$,
        $i = 1$, $j = 1$, and $\pi = \id$.

        We have to show that:
        \begin{itemize}
        \item $\config{S_a}{\evalctxt{E'_b}{e_{b_1}}} \ctxstepsto
          \config{\extS{\extS{S}{l'}{\bot}{\frozenfalse}}{l}{\bot}{\frozenfalse}}{\evalctxt{E'_b}{e_{b_2}}}$,
          and
        \item $\config{S_b}{\evalctxt{E'_a}{e_{a_1}}} \ctxstepsto
          \config{\extS{\extS{S}{l'}{\bot}{\frozenfalse}}{l}{\bot}{\frozenfalse}}{\evalctxt{E'_a}{e_{a_2}}}$.
        \end{itemize}

        For the first of these, consider that $S_a =
        \extS{S}{l}{\bot}{\frozenfalse} = U_S(S)$, where $U_S$ is
        the store update operation that acts as the identity on the
        contents of all existing locations, and adds the binding
        $\storebinding{l}{\bot}{\frozenfalse}$ if no binding for $l$
        exists.

        Note that:
        \begin{itemize}
        \item $U_S$ is non-conflicting with $\config{S}{e_{b_1}}
          \parstepsto \config{S_b}{e_{b_2}}$, since the only
          location allocated in the transition is $l'$, and $l
          \neq l'$ in this case;
        \item $U_S(S_b) \neq \topS$, since $U_S(S_b) =
          \extS{\extS{S}{l'}{\bot}{\frozenfalse}}{l}{\bot}{\frozenfalse}$
          and we know $S \neq \topS$ and the addition of new
          bindings $\storebinding{l}{\bot}{\frozenfalse}$ and
          $\storebinding{l'}{\bot}{\frozenfalse}$ cannot cause it to
          become $\topS$; and
        \item $U_S$ is freeze-safe with $\config{S}{e_{b_1}}
          \parstepsto \config{S_b}{e_{b_2}}$, since $S_b =
          \extS{S}{l'}{\bot}{\frozenfalse}$ and $l' \notin \dom{S}$,
          so there are no locations whose contents differ in status
          between $S$ and $S_b$.
        \end{itemize}

        Therefore, by Lemma~\ref{lem:generalized-independence}
        (Generalized Independence), we have that

        $\config{U_S(S)}{e_{b_1}} \parstepsto
        \config{U_S(S_b)}{e_{b_2}}$.

        Hence $\config{\extS{S}{l}{\bot}{\frozenfalse}}{e_{b_1}}
        \parstepsto
        \config{\extS{S_b}{l}{\bot}{\frozenfalse}}{e_{b_2}}$.

        By {\sc E-Eval-Ctxt} it follows that

        $\config{\extS{S}{l}{\bot}{\frozenfalse}}{\evalctxt{E'_b}{e_{b_1}}}
        \parstepsto
        \config{\extS{S_b}{l}{\bot}{\frozenfalse}}{\evalctxt{E'_b}{e_{b_2}}}$,
        which, since $S_b = \extS{S}{l'}{\bot}{\frozenfalse}$, is what
        we were required to show.

        The argument for the second is symmetrical.

      \item If $l = l'$:

        In this case, observe that we do \emph{not} want the
        expression in the final configuration to be
        $\evalctxt{E'_a}{e_{a_2}}$ (nor its equivalent,
        $\evalctxt{E'_b}{e_{b_2}}$).

        The reason for this is that $\evalctxt{E'_a}{e_{a_2}}$
        contains both occurrences of $l$.

        Rather, we want both configurations to step to a configuration
        in which exactly one occurrence of $l$ has been renamed to a
        fresh location $l''$.

        Let $l''$ be a location such that $l'' \notin \dom{S}$ and
        $l'' \neq l$ (and hence $l'' \neq l'$, as well).

        Then choose $S' =
        \extS{\extS{S}{l''}{\bot}{\frozenfalse}}{l}{\bot}{\frozenfalse}$,
        $i = 1$, $j = 1$, and $\pi = \setof{(l, l'')}$.

        Either
        $\config{\extS{\extS{S}{l''}{\bot}{\frozenfalse}}{l}{\bot}{\frozenfalse}}{\evalctxt{E'_a}{\pi(e_{a_2})}}$
        or
        $\config{\extS{\extS{S}{l''}{\bot}{\frozenfalse}}{l}{\bot}{\frozenfalse}}{\evalctxt{E'_b}{\pi(e_{b_2})}}$
        would work as a final configuration; we choose

        $\config{\extS{\extS{S}{l''}{\bot}{\frozenfalse}}{l}{\bot}{\frozenfalse}}{\evalctxt{E'_b}{\pi(e_{b_2})}}$.

        We have to show that:
        \begin{itemize}
        \item $\config{S_a}{\evalctxt{E'_b}{e_{b_1}}} \ctxstepsto
          \config{\extS{\extS{S}{l''}{\bot}{\frozenfalse}}{l}{\bot}{\frozenfalse}}{\evalctxt{E'_b}{\pi(e_{b_2})}}$,
          and
        \item $\pi(\config{S_b}{\evalctxt{E'_a}{e_{a_1}}})
          \ctxstepsto
          \config{\extS{\extS{S}{l''}{\bot}{\frozenfalse}}{l}{\bot}{\frozenfalse}}{\evalctxt{E'_b}{\pi(e_{b_2})}}$.
        \end{itemize}

        For the first of these, since $\config{S}{e_{b_1}}
        \parstepsto \config{S_b}{e_{b_2}}$, we have by
        Lemma~\ref{lem:permutability} (Permutability) that
        $\pi(\config{S}{e_{b_1}}) \parstepsto
        \pi(\config{S_b}{e_{b_2}})$.

        Since $\pi = \setof{(l, l'')}$, but $l \notin S$ (from the
        side condition on {\sc E-New}), we have that
        $\pi(\config{S}{e_{b_1}}) = \config{S}{e_{b_1}}$.

        Since $\config{S_b}{e_{b_2}} =
        \config{\extS{S}{l'}{\bot}{\frozenfalse}}{l'}$, and $l = l'$,
        we have that $\pi(\config{S_b}{e_{b_2}}) =
        \config{\extS{S}{l''}{\bot}{\frozenfalse}}{\pi(e_{b_2})}$.

        Hence $\config{S}{e_{b_1}} \parstepsto
        \config{\extS{S}{l''}{\bot}{\frozenfalse}}{\pi(e_{b_2})}$.

        Let $U_S$ be the store update operation that acts as the
        identity on the contents of all existing locations, and adds
        the binding $\storebinding{l}{\bot}{\frozenfalse}$ if no
        binding for $l$ exists.

        Note that:
        \begin{itemize}
        \item $U_S$ is non-conflicting with $\config{S}{e_{b_1}}
          \parstepsto
          \config{\extS{S}{l''}{\bot}{\frozenfalse}}{\pi(e_{b_2})}$,
          since the only location allocated in the transition is
          $l''$;
        \item $U_S(\extS{S}{l''}{\bot}{\frozenfalse}) \neq \topS$,
          since $U_S(\extS{S}{l''}{\bot}{\frozenfalse}) = \\
          \extS{\extS{S}{l''}{\bot}{\frozenfalse}}{l}{\bot}{\frozenfalse}$
          and we know $S \neq \topS$ and the addition of new
          bindings $\storebinding{l}{\bot}{\frozenfalse}$ and
          $\storebinding{l''}{\bot}{\frozenfalse}$ cannot cause it
          to become $\topS$; and
        \item $U_S$ is freeze-safe with $\config{S}{e_{b_1}}
          \parstepsto
          \config{\extS{S}{l''}{\bot}{\frozenfalse}}{\pi(e_{b_2})}$,
          since $l'' \notin \dom{S}$, so there are no locations
          whose contents differ in status between $S$ and
          $\extS{S}{l''}{\bot}{\frozenfalse}$.
        \end{itemize}

        Therefore, by Lemma~\ref{lem:generalized-independence}
        (Generalized Independence), we have that

        $\config{U_S(S)}{e_{b_1}} \parstepsto
        \config{U_S(\extS{S}{l''}{\bot}{\frozenfalse})}{\pi(e_{b_2})}$.

        Hence $\config{\extS{S}{l}{\bot}{\frozenfalse}}{e_{b_1}}
        \parstepsto
        \config{\extS{\extS{S}{l''}{\bot}{\frozenfalse}}{l}{\bot}{\frozenfalse}}{\pi(e_{b_2})}$.

        By {\sc E-Eval-Ctxt} it follows that

        $\config{\extS{S}{l}{\bot}{\frozenfalse}}{\evalctxt{E'_b}{e_{b_1}}}
        \parstepsto
        \config{\extS{\extS{S}{l''}{\bot}{\frozenfalse}}{l}{\bot}{\frozenfalse}}{\evalctxt{E'_b}{\pi(e_{b_2})}}$,

        which, since $\extS{S}{l}{\bot}{\frozenfalse} = S_a$, is what
        we were required to show.

        For the second, observe that since $S_b =
        \extS{S}{l}{\bot}{\frozenfalse}$, we have that $\pi(S_b) =
        \extS{S}{l''}{\bot}{\frozenfalse}$.

        Also, since $l$ does not occur in $e_{a_1}$, we have that
        $\pi(\evalctxt{E'_a}{e_{a_1}}) =
        \evalctxt{(\pi(E'_a))}{e_{a_1}}$.

        Hence we have to show that

        $\config{\extS{S}{l''}{\bot}{\frozenfalse}}{\evalctxt{(\pi(E'_a))}{e_{a_1}}}
        \ctxstepsto \\
        \config{\extS{\extS{S}{l''}{\bot}{\frozenfalse}}{l}{\bot}{\frozenfalse}}{\evalctxt{E'_b}{\pi(e_{b_2})}}$.

        Let $U_S$ be the store update operation that acts as the
        identity on the contents of all existing locations, and adds
        the binding $\storebinding{l''}{\bot}{\frozenfalse}$ if no
        binding for $l''$ exists.

        Note that:
        \begin{itemize}
        \item $U_S$ is non-conflicting with $\config{S}{e_{a_1}}
          \parstepsto \config{S_a}{e_{a_2}}$, since the only
          location allocated in the transition is $l$;
        \item $U_S(S_a) \neq \topS$, since $U_S(S_a) =
          \extS{\extS{S}{l''}{\bot}{\frozenfalse}}{l}{\bot}{\frozenfalse}$
          and we know $S \neq \topS$ and the addition of new
          bindings $\storebinding{l}{\bot}{\frozenfalse}$ and
          $\storebinding{l''}{\bot}{\frozenfalse}$ cannot cause it
          to become $\topS$; and
        \item $U_S$ is freeze-safe with $\config{S}{e_{a_1}}
          \parstepsto \config{S_a}{e_{a_2}}$, since $S_a =
          \extS{S}{l}{\bot}{\frozenfalse}$ and $l \notin \dom{S}$,
          so there are no locations whose contents differ in status
          between $S$ and $S_a$.
        \end{itemize}

        Therefore, by Lemma~\ref{lem:generalized-independence}
        (Generalized Independence), we have that

        $\config{U_S(S)}{e_{a_1}} \parstepsto
        \config{U_S(S_a)}{e_{a_2}}$.

        Hence $\config{\extS{S}{l''}{\bot}{\frozenfalse}}{e_{a_1}}
        \parstepsto
        \config{\extS{\extS{S}{l''}{\bot}{\frozenfalse}}{l}{\bot}{\frozenfalse}}{e_{a_2}}$.

        By {\sc E-Eval-Ctxt} it follows that
        
        $\config{\extS{S}{l''}{\bot}{\frozenfalse}}{\evalctxt{(\pi(E'_a))}{e_{a_1}}}
        \ctxstepsto \\
        \config{\extS{\extS{S}{l''}{\bot}{\frozenfalse}}{l}{\bot}{\frozenfalse}}{\evalctxt{(\pi(E'_a))}{e_{a_2}}}$,

        which completes the case since $\evalctxt{E'_b}{\pi(e_{b_2})}
        = \evalctxt{(\pi(E'_a))}{e_{a_2}}$.

        \lk{This assumes that you believe that
          $\evalctxt{E'_b}{\pi(e_{b_2})} =
          \evalctxt{(\pi(E'_a))}{e_{a_2}}$.}

      \end{itemize}

    \item \label{slqc-new-put}Case {\sc E-Put}: We have $S_a =
      \extS{S}{l}{\bot}{\frozenfalse}$ and $S_b =
      \extSRaw{S}{l'}{u_{p_i}(p_1)}$, where $l \neq l'$ (since $l
      \notin \dom{S}$, but $l' \in \dom{S}$).

      We have to show that:
      \begin{itemize}
      \item $\config{S_a}{\evalctxt{E'_b}{e_{b_1}}} \ctxstepsto
        \config{\extS{S_b}{l}{\bot}{\frozenfalse}}{\evalctxt{E'_b}{e_{b_2}}}$,
        and
      \item $\config{S_b}{\evalctxt{E'_a}{e_{a_1}}} \ctxstepsto
        \config{\extS{S_b}{l}{\bot}{\frozenfalse}}{\evalctxt{E'_a}{e_{a_2}}}$.
      \end{itemize}

      For the first of these, consider that $S_a =
      \extS{S}{l}{\bot}{\frozenfalse} = U_S(S)$, where $U_S$ is the
      store update operation that acts as the identity on the contents
      of all existing locations, and adds the binding
      $\storebinding{l}{\bot}{\frozenfalse}$ if no binding for $l$
      exists.

      Note that:
      \begin{itemize}
      \item $U_S$ is non-conflicting with $\config{S}{e_{b_1}}
        \parstepsto \config{S_b}{e_{b_2}}$, since no locations are
        allocated in the transition;
      \item $U_S(S_b) \neq \topS$, since $U_S(S_b) =
        \extS{S_b}{l}{\bot}{\frozenfalse}$, and we know $S_b \neq
        \topS$ and the addition of a new binding
        $\storebinding{l}{\bot}{\frozenfalse}$ cannot cause it to
        become $\topS$; and
      \item $U_S$ is freeze-safe with $\config{S}{e_{b_1}} \parstepsto
        \config{S_b}{e_{b_2}}$, since $S_b =
        \extSRaw{S}{l'}{u_{p_i}(p_1)}$ and $u_{p_i}$ does not alter
        the status of $p_1$.

        (By Definition~\ref{def:set-of-state-update-operations},
        $u_{p_i}$ can only change the status bit of a location if its
        contents are $\state{d}{\frozentrue}$ and $u_i(d) \neq d$, in
        which case $u_{p_i}$ changes the contents of the location to
        $\state{\top}{\frozenfalse}$; however, that cannot be the case
        here since then $u_{p_i}(p_1)$ would be $\topp$, contradicting
        the premise of {\sc E-Put}.)
      \end{itemize}

      Therefore, by Lemma~\ref{lem:generalized-independence}
      (Generalized Independence), we have that

      $\config{U_S(S)}{e_{b_1}} \parstepsto
      \config{U_S(S_b)}{e_{b_2}}$.

      Hence $\config{\extS{S}{l}{\bot}{\frozenfalse}}{e_{b_1}}
      \parstepsto
      \config{\extS{S_b}{l}{\bot}{\frozenfalse}}{e_{b_2}}$.

      By {\sc E-Eval-Ctxt}, it follows that

      $\config{\extS{S}{l}{\bot}{\frozenfalse}}{\evalctxt{E'_b}{e_{b_1}}}
      \ctxstepsto
      \config{\extS{S_b}{l}{\bot}{\frozenfalse}}{\evalctxt{E'_b}{e_{b_2}}}$,
      
      which, since $S_a = \extS{S}{l}{\bot}{\frozenfalse}$, is what we
      were required to show.

      For the second, let $U_S$ be the store update operation that
      applies $u_{p_i}$ to the contents of $l'$ if it exists, and adds
      a binding $\storebindingRaw{l'}{u_{p_i}(p_1)}$ if no binding for
      $l'$ exists.

      Consider that $S_b = U_S(S)$, and
      $\extS{S_b}{l}{\bot}{\frozenfalse} =
      \extSRaw{S_a}{l'}{u_{p_i}(p_1)} = U_S(S_a)$.

      Note that:
      \begin{itemize}
      \item $U_S$ is non-conflicting with $\config{S}{e_{a_1}}
        \parstepsto \config{S_a}{e_{a_2}}$, since the only location
        allocated in the transition is $l$;
      \item $U_S(S_a) \neq \topS$, since $U_S(S_a) =
        \extSRaw{\extS{S}{l}{\bot}{\frozenfalse}}{l'}{u_{p_i}(p_1)}$
        and we know $S \neq \topS$ and the addition of a new binding
        $\storebinding{l}{\bot}{\frozenfalse}$ and updating the
        contents of location $l'$ to $u_{p_i}(p_1)$ in $S$ cannot
        cause it to become $\topS$ (since if $u_{p_i}(p_1) = \topp$,
        $\config{S}{e_{b_1}}$ would not have been able to step by {\sc
          E-Put}); and
      \item $U_S$ is freeze-safe with $\config{S}{e_{a_1}} \parstepsto
        \config{S_a}{e_{a_2}}$, since $S_a =
        \extS{S}{l}{\bot}{\frozenfalse}$ and $l \notin \dom{S}$, so
        there are no locations whose contents differ in status between
        $S$ and $S_a$.
      \end{itemize}

      Therefore, by Lemma~\ref{lem:generalized-independence}
      (Generalized Independence), we have that

      $\config{U_S(S)}{e_{a_1}} \parstepsto
      \config{U_S(S_a)}{e_{a_2}}$.

      Hence $\config{S_b}{e_{a_1}}
      \parstepsto
      \config{\extS{S_b}{l}{\bot}{\frozenfalse}}{e_{a_2}}$.

      By {\sc E-Eval-Ctxt}, it follows that
      
      $\config{S_b}{\evalctxt{E'_a}{e_{a_1}}} \ctxstepsto
      \config{\extS{S_b}{l}{\bot}{\frozenfalse}}{\evalctxt{E'_a}{e_{a_2}}}$,
      
      as we were required to show.

    \item \label{slqc-new-put-err}Case {\sc E-Put-Err}: We have $S_a =
      \extS{S}{l}{\bot}{\frozenfalse}$ and $\config{S_b}{e_{b_2}} =
      \error$, and so we choose $\conf_c = \error$, $i = 1$, $j = 0$,
      and $\pi = \id$.

      We have to show that:
      \begin{itemize}
      \item $\config{S_a}{\evalctxt{E'_b}{e_{b_1}}} \ctxstepsto
        \error$, and
      \item $\config{S_b}{\evalctxt{E'_a}{e_{a_1}}} = \error$.
      \end{itemize}

      The second of these is immediately true because since
      $\config{S_b}{e_{b_2}} = \error$, $S_b = \topS$, and so
      $\config{S_b}{\evalctxt{E'_a}{e_{a_1}}}$ is equal to $\error$ as
      well.

      For the first, observe that since $\config{S}{e_{a_1}}
      \parstepsto \config{S_a}{e_{a_2}}$, we have by
      Lemma~\ref{lem:monotonicity} (Monotonicity) that
      $\leqstore{S}{S_a}$.

      Therefore, since $\config{S}{e_{b_1}} \parstepsto \error$,

      we have by Lemma~\ref{lem:error-preservation} (Error
      Preservation) that $\config{S_a}{e_{b_1}} \parstepsto \error$.

      Since $\error$ is equal to $\config{\topS}{e}$ for all
      expressions $e$, $\config{S_a}{e_{b_1}} \parstepsto
      \config{\topS}{e}$ for all $e$.

      Therefore, by {\sc E-Eval-Ctxt},
      $\config{S_a}{\evalctxt{E'_b}{e_{b_1}}} \ctxstepsto
      \config{\topS}{\evalctxt{E'_b}{e}}$ for all $e$.

      Since $\config{\topS}{\evalctxt{E'_b}{e}}$ is equal to $\error$,
      we have that $\config{S_a}{\evalctxt{E'_b}{e_{b_1}}} \ctxstepsto
      \error$, as we were required to show.

    \item \label{slqc-new-get}Case {\sc E-Get}: Similar to
      case~\ref{slqc-new-beta}, since $S_a =
      \extS{S}{l}{\bot}{\frozenfalse}$ and $S_b = S$.
    \item \label{slqc-new-freeze-init}Case {\sc E-Freeze-Init}:
      Similar to case~\ref{slqc-new-beta}, since $S_a =
      \extS{S}{l}{\bot}{\frozenfalse}$ and $S_b = S$.
    \item \label{slqc-new-spawn-handler}Case {\sc E-Spawn-Handler}:
      Similar to case~\ref{slqc-new-beta}, since $S_a =
      \extS{S}{l}{\bot}{\frozenfalse}$ and $S_b = S$.
    \item \label{slqc-new-freeze-final}Case {\sc E-Freeze-Final}: We
      have $S_a = \extS{S}{l}{\bot}{\frozenfalse}$ and $S_b =
      \extS{S}{l'}{d_1}{\frozentrue}$, where $l \neq l'$ (since $l
      \notin \dom{S}$, but $l' \in \dom{S}$).

      Choose $S' =
      \extS{\extS{S}{l}{\bot}{\frozenfalse}}{l'}{d_1}{\frozentrue}$,
      $i = i$, $j = 1$, and $\pi = \id$.

      We have to show that:
      \begin{itemize}
      \item
        $\config{\extS{S}{l}{\bot}{\frozenfalse}}{\evalctxt{E'_b}{e_{b_1}}}
        \ctxstepsto
        \config{\extS{\extS{S}{l}{\bot}{\frozenfalse}}{l'}{d_1}{\frozentrue}}{\evalctxt{E'_b}{e_{b_2}}}$,
        and
      \item
        $\config{\extS{S}{l'}{d_1}{\frozentrue}}{\evalctxt{E'_a}{e_{a_1}}}
        \ctxstepsto
        \config{\extS{\extS{S}{l}{\bot}{\frozenfalse}}{l'}{d_1}{\frozentrue}}{\evalctxt{E'_a}{e_{a_2}}}$.
      \end{itemize}

      For the first of these, consider that
      $\extS{S}{l}{\bot}{\frozenfalse} = U_S(S)$, where $U_S$ is the
      store update operation that acts as the identity on the contents
      of all existing locations, and adds the binding
      $\storebinding{l}{\bot}{\frozenfalse}$ if no binding for $l$
      exists.

      Note that:
      \begin{itemize}
      \item $U_S$ is non-conflicting with $\config{S}{e_{b_1}}
        \parstepsto \config{S_b}{e_{b_2}}$, since no locations are
        allocated in the transition;
      \item $U_S(S_b) \neq \topS$, since $U_S(S_b) =
        \extS{S_b}{l}{\bot}{\frozenfalse}$, and we know $S_b \neq
        \topS$ and the addition of a new binding
        $\storebinding{l}{\bot}{\frozenfalse}$ cannot cause it to
        become $\topS$; and
      \item $U_S$ is freeze-safe with $\config{S}{e_{b_1}}
        \parstepsto \config{S_b}{e_{b_2}}$, since $S_b =
        \extS{S}{l'}{d_1}{\frozentrue}$ and so the only location
        that can change in status between $S$ and $S_b$ is $l'$, and
        $U_S$ acts as the identity on $l'$.
      \end{itemize}
      Therefore, by Lemma~\ref{lem:generalized-independence}
      (Generalized Independence), we have that

      $\config{U_S(S)}{e_{b_1}} \parstepsto
      \config{U_S(S_b)}{e_{b_2}}$.

      Hence $\config{\extS{S}{l}{\bot}{\frozenfalse}}{e_{b_1}}
      \parstepsto
      \config{\extS{\extS{S}{l}{\bot}{\frozenfalse}}{l'}{d_1}{\frozentrue}}{e_{b_2}}$.

      By {\sc E-Eval-Ctxt}, it follows that

      $\config{\extS{S}{l}{\bot}{\frozenfalse}}{\evalctxt{E'_b}{e_{b_1}}}
      \ctxstepsto
      \config{\extS{\extS{S}{l}{\bot}{\frozenfalse}}{l'}{d_1}{\frozentrue}}{\evalctxt{E'_b}{e_{b_2}}}$,

      as we were required to show.

      For the second, consider that $\extS{S}{l'}{d_1}{\frozentrue} =
      U_S(S)$, where $U_S$ is the store update operation that freezes
      the contents of $l'$ and acts as the identity on the contents of
      all other locations.

      Note that:
      \begin{itemize}
      \item $U_S$ is non-conflicting with $\config{S}{e_{a_1}}
        \parstepsto \config{S_a}{e_{a_2}}$, since the only location
        allocated in the transition is $l$, and $l \neq l'$;
      \item $U_S(S_a) \neq \topS$, since $U_S(S_a) =
        \extS{S_a}{l'}{d_1}{\frozentrue} =
        \extS{S_b}{l}{\bot}{\frozenfalse}$, and we know $S_b \neq
        \topS$ and the addition of a new binding
        $\storebinding{l}{\bot}{\frozenfalse}$ cannot cause it to
        become $\topS$; and
      \item $U_S$ is freeze-safe with $\config{S}{e_{a_1}}
        \parstepsto \config{S_a}{e_{a_2}}$, since $S_a =
        \extS{S}{l}{\bot}{\frozenfalse}$ and $l \notin \dom{S}$, so
        there are no locations whose contents differ in status
        between $S$ and $S_a$.
      \end{itemize}

      Therefore, by Lemma~\ref{lem:generalized-independence}
      (Generalized Independence), we have that

      $\config{U_S(S)}{e_{a_1}} \parstepsto
      \config{U_S(S_a)}{e_{a_2}}$.

      Hence $\config{\extS{S}{l'}{d_1}{\frozentrue}}{e_{a_1}}
      \parstepsto
      \config{\extS{\extS{S}{l}{\bot}{\frozenfalse}}{l'}{d_1}{\frozentrue}}{e_{a_2}}$.

      By {\sc E-Eval-Ctxt} it follows that

      $\config{\extS{S}{l'}{d_1}{\frozentrue}}{\evalctxt{E'_a}{e_{a_1}}}
      \ctxstepsto
      \config{\extS{\extS{S}{l}{\bot}{\frozenfalse}}{l'}{d_1}{\frozentrue}}{\evalctxt{E'_a}{e_{a_2}}}$,

      as we were required to show.

    \item \label{slqc-new-freeze-simple}Case {\sc E-Freeze-Simple}:
      Similar to case~\ref{slqc-new-freeze-final}, since $S_a =
      \extS{S}{l}{\bot}{\frozenfalse}$ and $S_b =
      \extS{S}{l'}{d_1}{\frozentrue}$, where $l \neq l'$ (since $l
      \notin \dom{S}$, but $l' \in \dom{S}$).

    \end{enumerate}
  \item Case {\sc E-Put}: We have $S_a =
    \extSRaw{S}{l}{u_{p_i}(p_1)}$.

    We proceed by case analysis on the rule by which
    $\config{S}{e_{b_1}}$ steps to $\config{S_b}{e_{b_2}}$.

    Since the only way an $\error$ configuration can arise is by the
    {\sc E-Put-Err} rule, we can assume in all other cases that
    $\conf_b \neq \error$.
    \begin{enumerate}
    \item \label{slqc-put-beta}Case {\sc E-Beta}: By symmetry with case~\ref{slqc-beta-put}.
    \item \label{slqc-put-new}Case {\sc E-New}: By symmetry with case~\ref{slqc-new-put}.
    \item \label{slqc-put-put}Case {\sc E-Put}: We have $S_a =
      \extSRaw{S}{l}{u_{p_i}(p_1)}$ and $S_b =
      \extSRaw{S}{l'}{u_{p_j}(p'_1)}$, where $p'_1 = S(l')$.

      Now consider whether $l = l'$:
      \begin{itemize}
      \item If $l \neq l'$:

        Choose $S' =
        \extSRaw{\extSRaw{S}{l'}{u_{p_j}(p'_1)}}{l}{u_{p_i}(p_1)}$,
        $i = 1$, $j = 1$, and $\pi = \id$.

        We have to show that:
        \begin{itemize}
        \item
          $\config{\extSRaw{S}{l}{u_{p_i}(p_1)}}{\evalctxt{E'_b}{e_{b_1}}}
          \ctxstepsto
          \config{\extSRaw{\extSRaw{S}{l'}{u_{p_j}(p'_1)}}{l}{u_{p_i}(p_1)}}{\evalctxt{E'_b}{e_{b_2}}}$,
          and
        \item
          $\config{\extSRaw{S}{l'}{u_{p_j}(p'_1)}}{\evalctxt{E'_a}{e_{a_1}}}
          \ctxstepsto
          \config{\extSRaw{\extSRaw{S}{l'}{u_{p_j}(p'_1)}}{l}{u_{p_i}(p_1)}}{\evalctxt{E'_a}{e_{a_2}}}$.
        \end{itemize}

        For the first of these, consider that
        $\extSRaw{S}{l}{u_{p_i}(p_1)} = U_S(S)$, where $U_S$ is the
        store update operation that applies $u_{p_i}$ to the
        contents of $l$ if it exists, and adds a binding
        $\storebindingRaw{l}{u_{p_i}(p_1)}$ if no binding for $l$
        exists.

        Note that:
        \begin{itemize}
        \item $U_S$ is non-conflicting with $\config{S}{e_{b_1}}
          \parstepsto
          \config{\extSRaw{S}{l'}{u_{p_j}(p'_1)}}{e_{b_2}}$, since
          no locations are allocated in the transition;
        \item $U_S(\extSRaw{S}{l'}{u_{p_j}(p'_1)}) \neq \topS$,
          since $U_S(\extSRaw{S}{l'}{u_{p_j}(p'_1)}) =
          \extSRaw{\extSRaw{S}{l'}{u_{p_j}(p'_1)}}{l}{u_{p_i}(p_1)}$
          and we know $S \neq \topS$ and updating the contents of
          location $l$ to $u_{p_i}(p_1)$ and the contents of
          location $l'$ to $u_{p_j}(p'_1)$ in $S$ cannot cause it to
          become $\topS$ (because if so, then we would have $S_a =
          \topS$ or $S_b = \topS$, which we know are not the case);
          and
        \item $U_S$ is freeze-safe with $\config{S}{e_{b_1}}
          \parstepsto
          \config{\extSRaw{S}{l'}{u_{p_j}(p'_1)}}{e_{b_2}}$, since
          $u_{p_j}$ does not alter the status of $p'_1$.

          (By Definition~\ref{def:set-of-state-update-operations},
          $u_{p_j}$ can only change the status bit of a location if
          its contents are $\state{d}{\frozentrue}$ and $u_j(d) \neq
          d$, in which case $u_{p_j}$ changes the contents of the
          location to $\state{\top}{\frozenfalse}$; however, that
          cannot be the case here since then $u_{p_j}(p'_1)$ would be
          $\topp$, contradicting the premise of {\sc E-Put}.)
        \end{itemize}

        Therefore, by Lemma~\ref{lem:generalized-independence}
        (Generalized Independence), we have that

        $\config{U_S(S)}{e_{b_1}} \parstepsto
        \config{U_S(\extSRaw{S}{l'}{u_{p_j}(p'_1)})}{e_{b_2}}$.

        Hence $\config{\extSRaw{S}{l}{u_{p_i}(p_1)}}{e_{b_1}}
        \parstepsto
        \config{\extSRaw{\extSRaw{S}{l'}{u_{p_j}(p'_1)}}{l}{u_{p_i}(p_1)}}{e_{b_2}}$.

        By {\sc E-Eval-Ctxt}, it follows that

        $\config{\extSRaw{S}{l}{u_{p_i}(p_1)}}{\evalctxt{E'_b}{e_{b_1}}}
        \ctxstepsto
        \config{\extSRaw{\extSRaw{S}{l'}{u_{p_j}(p'_1)}}{l}{u_{p_i}(p_1)}}{\evalctxt{E'_b}{e_{b_2}}}$,

        as we were required to show.

        The argument for the second is symmetrical.

      \item If $l = l'$:
        Note that since $l = l'$, $p_1 = p'_1$ as well.

        Consider whether $u_{p_i}(u_{p_j}(p_1)) = \topp$:
        \begin{itemize}
        \item If $u_{p_i}(u_{p_j}(p_1)) = \topp$:

          Choose $\conf_c = \error$, $i = 1$, $j = 1$, and $\pi =
          \id$.

          We have to show that:

          \begin{itemize}
          \item
            $\config{\extSRaw{S}{l}{u_{p_i}(p_1)}}{\evalctxt{E'_b}{e_{b_1}}}
            \ctxstepsto \error$, and
          \item
            $\config{\extSRaw{S}{l}{u_{p_j}(p_1)}}{\evalctxt{E'_a}{e_{a_1}}}
            \ctxstepsto \error$.
          \end{itemize}

          For the first of these, consider that
          $\extSRaw{S}{l}{u_{p_i}(p_1)} = U_S(S)$, where $U_S$ is the
          store update operation that applies $u_{p_i}$ to the
          contents of $l$ if it exists.

          Note that:
          \begin{itemize}
          \item $U_S$ is non-conflicting with $\config{S}{e_{b_1}}
            \parstepsto
            \config{\extSRaw{S}{l}{u_{p_j}(p_1)}}{e_{b_2}}$, since
            no locations are allocated in the transition;
          \item $U_S(\extSRaw{S}{l}{u_{p_j}(p_1)}) = \topS$, since
            $U_S(\extSRaw{S}{l}{u_{p_j}(p_1)}) =
            \extSRaw{S}{l}{u_{p_i}(u_{p_j}(p_1))}$ and we know
            $u_{p_i}(u_{p_j}(p_1)) = \topp$ in this case;
          \item $U_S$ is freeze-safe with $\config{S}{e_{b_1}}
            \parstepsto
            \config{\extSRaw{S}{l}{u_{p_j}(p_1)}}{e_{b_2}}$, since
            $u_{p_j}$ does not alter the status of $p_1$.

            (By Definition~\ref{def:set-of-state-update-operations},
            $u_{p_j}$ can only change the status bit of a location if
            its contents are $\state{d}{\frozentrue}$ and $u_j(d) \neq
            d$, in which case $u_{p_j}$ changes the contents of the
            location to $\state{\top}{\frozenfalse}$; however, that
            cannot be the case here since then $u_{p_j}(p_1)$ would be
            $\topp$, contradicting the premise of {\sc E-Put}.)
          \end{itemize}

          Therefore, by Lemma~\ref{lem:generalized-clash}
          (Generalized Clash), we have that there exists $i' \leq 1$
          such that $\config{U_S(S)}{e_{b_1}} \parstepsto^{i'}
          \error$.

          Hence $\config{\extSRaw{S}{l}{u_{p_i}(p_1)}}{e_{b_1}}
          \parstepsto^{i'} \error$.

          If $i' = 0$, we would have
          $\config{\extSRaw{S}{l}{u_{p_i}(p_1)}}{e_{b_1}} =
          \config{S_a}{e_{b_1}} = \error$.

          So we would have $S_a = \topS$ by the definition of
          $\error$, but then we would have $\conf_a = \error$, a
          contradiction.

          Therefore $i' = 1$, and so we have
          $\config{\extSRaw{S}{l}{u_{p_i}(p_1)}}{e_{b_1}} \parstepsto
          \error$.

          Since $\error = \config{\topS}{e}$ for all $e$, we have
          $\config{\extSRaw{S}{l}{u_{p_i}(p_1)}}{e_{b_1}}
          \parstepsto \config{\topS}{e}$ for all $e$.

          So, by {\sc E-Eval-Ctxt}, we have that
          $\config{\extSRaw{S}{l}{u_{p_i}(p_1)}}{\evalctxt{E'_b}{e_{b_1}}}
          \parstepsto \config{\topS}{\evalctxt{E'_b}{e}}$ for all $e$.

          Hence
          $\config{\extSRaw{S}{l}{u_{p_i}(p_1)}}{\evalctxt{E'_b}{e_{b_1}}}
          \parstepsto \error$.

          The argument for the second is symmetrical.

        \item If $u_{p_i}(u_{p_j}(p_1)) \neq \topp$:

          Choose $S' = \extSRaw{S}{l}{u_{p_i}(u_{p_j}(p_1))}$, $i =
          1$, $j = 1$, and $\pi = \id$.

          We have to show that:
          \begin{itemize}
          \item
            $\config{\extSRaw{S}{l}{u_{p_i}(p_1)}}{\evalctxt{E'_b}{e_{b_1}}}
            \ctxstepsto
            \config{\extSRaw{S}{l}{u_{p_i}(u_{p_j}(p_1))}}{\evalctxt{E'_b}{e_{b_2}}}$,
            and
          \item
            $\config{\extSRaw{S}{l}{u_{p_j}(p_1)}}{\evalctxt{E'_a}{e_{a_1}}}
            \ctxstepsto
            \config{\extSRaw{S}{l}{u_{p_i}(u_{p_j}(p_1))}}{\evalctxt{E'_a}{e_{a_2}}}$.
          \end{itemize}

          For the first of these, consider that
          $\extSRaw{S}{l}{u_{p_i}(p_1)} = U_S(S)$, where $U_S$ is the
          store update operation that applies $u_{p_i}$ to the
          contents of $l$ if it exists.

          Note that:
          \begin{itemize}
          \item $U_S$ is non-conflicting with $\config{S}{e_{b_1}}
            \parstepsto
            \config{\extSRaw{S}{l}{u_{p_j}(p_1)}}{e_{b_2}}$, since no
            locations are allocated in the transition;
          \item $U_S(\extSRaw{S}{l}{u_{p_j}(p_1)}) \neq \topS$, since
            $U_S(\extSRaw{S}{l}{u_{p_j}(p_1)}) =
            \extSRaw{S}{l}{u_{p_i}(u_{p_j}(p_1))}$ and we know $S \neq
            \topS$ and $u_{p_i}(u_{p_j}(p_1)) \neq \topp$ in this
            case;
          \item $U_S$ is freeze-safe with $\config{S}{e_{b_1}}
            \parstepsto
            \config{\extSRaw{S}{l}{u_{p_j}(p_1)}}{e_{b_2}}$, since
            $u_{p_j}$ does not alter the status of $p_1$.

            (By Definition~\ref{def:set-of-state-update-operations},
            $u_{p_j}$ can only change the status bit of a location if
            its contents are $\state{d}{\frozentrue}$ and $u_j(d) \neq
            d$, in which case $u_{p_j}$ changes the contents of the
            location to $\state{\top}{\frozenfalse}$; however, that
            cannot be the case here since then $u_{p_j}(p_1)$ would be
            $\topp$, contradicting the premise of {\sc E-Put}.)
          \end{itemize}

          Therefore, by Lemma~\ref{lem:generalized-independence}
          (Generalized Independence), we have that

          $\config{U_S(S)}{e_{b_1}} \parstepsto
          \config{U_S(\extSRaw{S}{l}{u_{p_j}(p_1)})}{e_{b_2}}$.

          Hence $\config{\extSRaw{S}{l}{u_{p_i}(p_1)}}{e_{b_1}}
          \parstepsto
          \config{\extSRaw{S}{l}{u_{p_i}(u_{p_j}(p_1))}}{e_{b_2}}$.

          By {\sc E-Eval-Ctxt}, it follows that

          $\config{\extSRaw{S}{l}{u_{p_i}(p_1)}}{\evalctxt{E'_b}{e_{b_1}}}
          \ctxstepsto
          \config{\extSRaw{S}{l}{u_{p_i}(u_{p_j}(p_1))}}{\evalctxt{E'_b}{e_{b_2}}}$,

          as we were required to show.

          The argument for the second is symmetrical.

        \end{itemize}

      \end{itemize}

    \item \label{slqc-put-put-err}Case {\sc E-Put-Err}: We have $S_a =
      \extSRaw{S}{l}{u_{p_i}(p_1)}$ and $\config{S_b}{e_{b_2}} =
      \error$, and so we choose $\conf_c = \error$, $i = 1$, $j = 0$,
      and $\pi = \id$.

      We have to show that:
      \begin{itemize}
      \item $\config{S_a}{\evalctxt{E'_b}{e_{b_1}}} \ctxstepsto
        \error$, and
      \item $\config{S_b}{\evalctxt{E'_a}{e_{a_1}}} = \error$.
      \end{itemize}

      The second of these is immediately true because since
      $\config{S_b}{e_{b_2}} = \error$, $S_b = \topS$, and so
      $\config{S_b}{\evalctxt{E'_a}{e_{a_1}}}$ is equal to $\error$ as
      well.

      For the first, observe that since $\config{S}{e_{a_1}}
      \parstepsto \config{S_a}{e_{a_2}}$, we have by
      Lemma~\ref{lem:monotonicity} (Monotonicity) that
      $\leqstore{S}{S_a}$.

      Therefore, since $\config{S}{e_{b_1}} \parstepsto \error$,

      we have by Lemma~\ref{lem:error-preservation} (Error
      Preservation) that $\config{S_a}{e_{b_1}} \parstepsto \error$.
      
      Since $\error$ is equal to $\config{\topS}{e}$ for all
      expressions $e$, $\config{S_a}{e_{b_1}} \parstepsto
      \config{\topS}{e}$ for all $e$.

      Therefore, by {\sc E-Eval-Ctxt},
      $\config{S_a}{\evalctxt{E'_b}{e_{b_1}}} \ctxstepsto
      \config{\topS}{\evalctxt{E'_b}{e}}$ for all $e$.

      Since $\config{\topS}{\evalctxt{E'_b}{e}}$ is equal to $\error$,
      we have that $\config{S_a}{\evalctxt{E'_b}{e_{b_1}}} \ctxstepsto
      \error$, as we were required to show.

    \item \label{slqc-put-get}Case {\sc E-Get}: Similar to
      case~\ref{slqc-put-beta}, since $S_a =
      \extSRaw{S}{l}{u_{p_i}(p_1)}$ and $S_b = S$.
    \item \label{slqc-put-freeze-init}Case {\sc E-Freeze-Init}:
      Similar to case~\ref{slqc-put-beta}, since $S_a =
      \extSRaw{S}{l}{u_{p_i}(p_1)}$ and $S_b = S$.
    \item \label{slqc-put-spawn-handler}Case {\sc E-Spawn-Handler}:
      Similar to case~\ref{slqc-put-beta}, since $S_a =
      \extSRaw{S}{l}{u_{p_i}(p_1)}$ and $S_b = S$.
    \item \label{slqc-put-freeze-final}Case {\sc E-Freeze-Final}: We
      have $S_a = \extSRaw{S}{l}{u_{p_i}(p_1)}$ and $S_b =
      \extS{S}{l'}{d_1}{\frozentrue}$.

      Now consider whether $l = l'$:
      \begin{itemize}
      \item If $l \neq l'$:

        Choose $S' =
        \extS{\extSRaw{S}{l}{u_{p_i}(p_1)}}{l'}{d_1}{\frozentrue}$,
        $i = 1$, $j = 1$, and $\pi = \id$.

        We have to show that:
        \begin{itemize}
        \item
          $\config{\extSRaw{S}{l}{u_{p_i}(p_1)}}{\evalctxt{E'_b}{e_{b_1}}}
          \ctxstepsto
          \config{\extS{\extSRaw{S}{l}{u_{p_i}(p_1)}}{l'}{d_1}{\frozentrue}}{\evalctxt{E'_b}{e_{b_2}}}$,
          and
        \item
          $\config{\extS{S}{l'}{d_1}{\frozentrue}}{\evalctxt{E'_a}{e_{a_1}}}
          \ctxstepsto
          \config{\extS{\extSRaw{S}{l}{u_{p_i}(p_1)}}{l'}{d_1}{\frozentrue}}{\evalctxt{E'_a}{e_{a_2}}}$.
        \end{itemize}

        For the first of these, consider that
        $\extSRaw{S}{l}{u_{p_i}(p_1)} = U_S(S)$, where $U_S$ is the
        store update operation that applies $u_{p_i}$ to the
        contents of $l$ if it exists, and adds a binding
        $\storebindingRaw{l}{u_{p_i}(p_1)}$ if no binding for $l$
        exists, and acts as the identity on all other locations.

        Note that:
        \begin{itemize}
        \item $U_S$ is non-conflicting with $\config{S}{e_{b_1}}
          \parstepsto
          \config{\extS{S}{l'}{d_1}{\frozentrue}}{e_{b_2}}$, since
          no locations are allocated in the transition;
        \item $U_S(\extS{S}{l'}{d_1}{\frozentrue}) \neq \topS$,

          since $U_S(\extS{S}{l'}{d_1}{\frozentrue}) =
          \extSRaw{\extS{S}{l'}{d_1}{\frozentrue}}{l}{u_{p_i}(p_1)}$
          and we know $S \neq \topS$ and updating the contents of
          location $l$ to $u_{p_i}(p_1)$ and freezing the contents
          of location $l'$ in $S$ cannot cause it to become $\topS$
          (because if so, then we would have $S_a = \topS$ or $S_b =
          \topS$, which we know are not the case); and
        \item $U_S$ is freeze-safe with $\config{S}{e_{b_1}}
          \parstepsto
          \config{\extS{S}{l'}{d_1}{\frozentrue}}{e_{b_2}}$, since
          the only location that can change in status between $S$
          and $\extS{S}{l'}{d_1}{\frozentrue}$ is $l'$, and $U_S$
          acts as the identity on $l'$.
        \end{itemize}
        Therefore, by Lemma~\ref{lem:generalized-independence}
        (Generalized Independence), we have that

        $\config{U_S(S)}{e_{b_1}} \parstepsto
        \config{U_S(\extS{S}{l'}{d_1}{\frozentrue})}{e_{b_2}}$.

        Hence $\config{\extSRaw{S}{l}{u_{p_i}(p_1)}}{e_{b_1}}
        \parstepsto
        \config{\extSRaw{\extS{S}{l'}{d_1}{\frozentrue}}{l}{u_{p_i}(p_1)}}{e_{b_2}}$.

        By {\sc E-Eval-Ctxt}, it follows that

        $\config{\extSRaw{S}{l}{u_{p_i}(p_1)}}{\evalctxt{E'_b}{e_{b_1}}}
        \ctxstepsto
        \config{\extSRaw{\extS{S}{l'}{d_1}{\frozentrue}}{l}{u_{p_i}(p_1)}}{\evalctxt{E'_b}{e_{b_2}}}$,
        
        as we were required to show.

        For the second, consider that
        $\extS{S}{l'}{d_1}{\frozentrue} = U_S(S)$, where $U_S$ is
        the store update operation that freezes the contents of $l'$
        and acts as the identity on the contents of all other
        locations.

        Note that:
        \begin{itemize}
        \item $U_S$ is non-conflicting with $\config{S}{e_{a_1}}
          \parstepsto
          \config{\extSRaw{S}{l}{u_{p_i}(p_1)}}{e_{a_2}}$, since no
          locations are allocated in the transition;
        \item $U_S(\extSRaw{S}{l}{u_{p_i}(p_1)}) \neq \topS$, since
          $U_S(\extSRaw{S}{l}{u_{p_i}(p_1)}) =
          \extS{\extSRaw{S}{l}{u_{p_i}(p_1)}}{l'}{d_1}{\frozentrue}$,
          and we know $S \neq \topS$ and updating the contents of
          location $l$ to $u_{p_i}(p_1)$ and freezing the contents
          of location $l$ in $S$ cannot cause it to become $\topS$
          (because if so, then we would have $S_a = \topS$ or $S_b =
          \topS$, which we know are not the case); and
        \item $U_S$ is freeze-safe with $\config{S}{e_{a_1}}
          \parstepsto
          \config{\extSRaw{S}{l}{u_{p_i}(p_1)}}{e_{a_2}}$, since
          $u_{p_i}$ does not alter the status of $p_1$.

          (By Definition~\ref{def:set-of-state-update-operations},
          $u_{p_i}$ can only change the status bit of a location if
          its contents are $\state{d}{\frozentrue}$ and $u_i(d) \neq
          d$, in which case $u_{p_i}$ changes the contents of the
          location to $\state{\top}{\frozenfalse}$; however, that
          cannot be the case here since then $u_{p_i}(p_1)$ would be
          $\topp$, and we would have $S_a = \topS$, a contradiction.)
        \end{itemize}
        Therefore, by Lemma~\ref{lem:generalized-independence}
        (Generalized Independence), we have that

        $\config{U_S(S)}{e_{a_1}} \parstepsto
        \config{U_S(\extSRaw{S}{l}{u_{p_i}(p_1)})}{e_{a_2}}$.

        Hence $\config{\extS{S}{l'}{d_1}{\frozentrue}}{e_{a_1}}
        \parstepsto
        \config{\extS{\extSRaw{S}{l}{u_{p_i}(p_1)}}{l'}{d_1}{\frozentrue}}{e_{a_2}}$.

        By {\sc E-Eval-Ctxt}, it follows that
        $\config{\extS{S}{l'}{d_1}{\frozentrue}}{\evalctxt{E'_a}{e_{a_1}}}
        \ctxstepsto
        \config{\extS{\extSRaw{S}{l}{u_{p_i}(p_1)}}{l'}{d_1}{\frozentrue}}{\evalctxt{E'_a}{e_{a_2}}}$,
        
        as we were required to show.

      \item If $l = l'$:

        We have two cases to consider:

        \begin{itemize}
        \item $u_{p_i}(\state{d_1}{\frozentrue}) = \topp$:

          \lk{This is the interesting case: the potential
            put-after-freeze case.  It's important to note that this
            case doesn't necessarily end in a put-after-freeze (and
            hence an error); all we're required to show is that it
            \emph{can} end that way.}

          Since $(\extSRaw{S}{l}{\state{d_1}{\frozentrue}})(l) =
          \state{d_1}{\frozentrue}$ and
          $u_{p_i}(\state{d_1}{\frozentrue}) = \topp$, by {\sc
            E-Put-Err} we have that
          $\config{\extSRaw{S}{l}{\state{d_1}{\frozentrue}}}{\putiexp{l}}
          \parstepsto \error$.

          Since $S_b = \extSRaw{S}{l}{\state{d_1}{\frozentrue}}$,
          we have that $\config{S_b}{\putiexp{l}} \parstepsto
          \error$.

          Since $\config{S}{e_{a_1}} \parstepsto
          \config{S_a}{e_{a_2}}$ by {\sc E-Put}, it must be the
          case that $e_{a_1} = \putiexp{l}$.

          Hence $\config{S_b}{e_{a_1}} \parstepsto \error$.

          Since $\error$ is equal to $\config{\topS}{e}$ for all
          expressions $e$, $\config{S_b}{e_{a_1}} \parstepsto
          \config{\topS}{e}$ for all $e$.

          Therefore, by {\sc E-Eval-Ctxt},
          $\config{S_b}{\evalctxt{E'_a}{e_{a_1}}} \ctxstepsto
          \config{\topS}{\evalctxt{E'_a}{e}}$ for all $e$.

          Since $\config{\topS}{\evalctxt{E'_a}{e}}$ is equal to
          $\error$, we have that
          $\config{S_b}{\evalctxt{E'_a}{e_{a_1}}} \ctxstepsto \error$.

          Since $\evalctxt{E'_a}{e_{a_1}} =
          \evalctxt{E_b}{e_{b_2}}$, we have that
          $\config{S_b}{\evalctxt{E_b}{e_{b_2}}} \ctxstepsto
          \error$.

          Since $\conf_b = \config{S_b}{\evalctxt{E_b}{e_{b_2}}}$,
          we therefore have that $\conf_b \ctxstepsto \error$, and
          the case is satisfied.

        \item $u_{p_i}(\state{d_1}{\frozentrue}) \neq \topp$:

          \lk{This is the case where there's a conflicting put and
            freeze, but the put is a no-op, so it doesn't matter.}

          In this case, by the definition of $U_p$
          (Definition~\ref{def:set-of-state-update-operations}),
          
          it must be the case that $u_{p_i}(\state{d_1}{\frozentrue})
          = \state{d_1}{\frozentrue}$.

          Choose $S' = \extS{S}{l}{d_1}{\frozentrue}$, $i = 1$, $j
          = 1$, and $\pi = \id$.

          We have to show that:
          \begin{itemize}
          \item
            $\config{\extSRaw{S}{l}{u_{p_i}(p_1)}}{\evalctxt{E'_b}{e_{b_1}}}
            \ctxstepsto
            \config{\extS{S}{l}{d_1}{\frozentrue}}{\evalctxt{E'_b}{e_{b_2}}}$,
            and
          \item
            $\config{\extS{S}{l}{d_1}{\frozentrue}}{\evalctxt{E'_a}{e_{a_1}}}
            \ctxstepsto
            \config{\extS{S}{l}{d_1}{\frozentrue}}{\evalctxt{E'_a}{e_{a_2}}}$.
          \end{itemize}

          For the first of these, consider that
          $\extSRaw{S}{l}{u_{p_i}(p_1)} = U_S(S)$, where $U_S$ is
          the store update operation that applies $u_{p_i}$ to the
          contents of $l$ if it exists, and adds a binding
          $\storebindingRaw{l}{u_{p_i}(p_1)}$ if no binding for
          $l$ exists, and acts as the identity on all other
          locations.

          Note that:
          \begin{itemize}
          \item $U_S$ is non-conflicting with $\config{S}{e_{b_1}}
            \parstepsto
            \config{\extS{S}{l}{d_1}{\frozentrue}}{e_{b_2}}$,
            since no locations are allocated in the
            transition;
          \item $U_S(\extS{S}{l}{d_1}{\frozentrue}) \neq \topS$,
            
            since $U_S(\extS{S}{l}{d_1}{\frozentrue}) =
            \extSRaw{S}{l}{u_{p_i}(\state{d_1}{\frozentrue})}$ and
            we know $S \neq \topS$ and
            $u_{p_i}(\state{d_1}{\frozentrue}) \neq \topp$; and
          \item $U_S$ is freeze-safe with $\config{S}{e_{b_1}}
            \parstepsto
            \config{\extS{S}{l}{d_1}{\frozentrue}}{e_{b_2}}$, since
            the only location that can change in status between $S$
            and $\extS{S}{l}{d_1}{\frozentrue}$ is $l$, and $U_S$
            acts as the identity on $l$.
          \end{itemize}
          Therefore, by Lemma~\ref{lem:generalized-independence}
          (Generalized Independence), we have that

          $\config{U_S(S)}{e_{b_1}} \parstepsto
          \config{U_S(\extS{S}{l}{d_1}{\frozentrue})}{e_{b_2}}$.

          Hence $\config{\extSRaw{S}{l}{u_{p_i}(p_1)}}{e_{b_1}}
          \parstepsto
          \config{\extSRaw{S}{l}{u_{p_i}(\state{d_1}{\frozentrue})}}{e_{b_2}}$.

          Since $u_{p_i}(\state{d_1}{\frozentrue}) =
          \state{d_1}{\frozentrue}$,

          we have that
          $\config{\extSRaw{S}{l}{u_{p_i}(p_1)}}{e_{b_1}}
          \parstepsto
          \config{\extS{S}{l}{d_1}{\frozentrue}}{e_{b_2}}$.

          By {\sc E-Eval-Ctxt}, it follows that

          $\config{\extSRaw{S}{l}{u_{p_i}(p_1)}}{\evalctxt{E'_b}{e_{b_1}}}
          \ctxstepsto
          \config{\extS{S}{l}{d_1}{\frozentrue}}{\evalctxt{E'_b}{e_{b_2}}}$,

          as we were required to show.

          For the second, consider that
          $\extS{S}{l}{d_1}{\frozentrue} = U_S(S)$, where $U_S$ is
          the store update operation that freezes the contents of $l$
          and acts as the identity on the contents of all other
          locations.

          Note that:
          \begin{itemize}
          \item $U_S$ is non-conflicting with $\config{S}{e_{a_1}}
            \parstepsto
            \config{\extSRaw{S}{l}{u_{p_i}(p_1)}}{e_{a_2}}$, since no
            locations are allocated in the transition;
          \item $U_S(\extSRaw{S}{l}{u_{p_i}(p_1)}) \neq \topS$,
            since $U_S(\extSRaw{S}{l}{u_{p_i}(p_1)}) =
            \extS{S}{l}{d_1}{\frozentrue}$ (since, by
            Definition~\ref{def:set-of-state-update-operations},
            $u_i(d_1) = d_1$; otherwise we would have
            $u_{p_i}(\state{d_1}{\frozentrue}) = \topp$, a
            contradiction), and we know $S \neq \topS$ and
            freezing the contents of location $l$ in $S$ cannot
            cause it to become $\topS$; and
          \item $U_S$ is freeze-safe with $\config{S}{e_{a_1}}
            \parstepsto
            \config{\extSRaw{S}{l}{u_{p_i}(p_1)}}{e_{a_2}}$, since
            $u_{p_i}$ does not alter the status of $p_1$.

            (By Definition~\ref{def:set-of-state-update-operations},
            $u_{p_i}$ can only change the status bit of a location if
            its contents are $\state{d}{\frozentrue}$ and $u_i(d) \neq
            d$, in which case $u_{p_i}$ changes the contents of the
            location to $\state{\top}{\frozenfalse}$; however, that
            cannot be the case here since then $u_{p_i}(p_1)$ would be
            $\topp$, and we would have $S_a = \topS$, a
            contradiction.)
          \end{itemize}
          Therefore, by Lemma~\ref{lem:generalized-independence}
          (Generalized Independence), we have that

          $\config{U_S(S)}{e_{a_1}} \parstepsto
          \config{U_S(\extSRaw{S}{l}{u_{p_i}(p_1)})}{e_{a_2}}$.

          Hence $\config{\extS{S}{l}{d_1}{\frozentrue}}{e_{a_1}}
          \parstepsto
          \config{\extS{S}{l}{d_1}{\frozentrue}}{e_{a_2}}$.

          By {\sc E-Eval-Ctxt}, it follows that

          $\config{\extS{S}{l}{d_1}{\frozentrue}}{\evalctxt{E'_a}{e_{a_1}}}
          \ctxstepsto
          \config{\extS{S}{l}{d_1}{\frozentrue}}{\evalctxt{E'_a}{e_{a_2}}}$,

          as we were required to show.
        \end{itemize}

      \end{itemize}

    \item \label{slqc-put-freeze-simple}Case {\sc E-Freeze-Simple}:
      Similar to case~\ref{slqc-put-freeze-final}, since $S_a =
      \extSRaw{S}{l}{u_{p_i}(p_1)}$ and $S_b =
      \extS{S}{l'}{d_1}{\frozentrue}$.

    \end{enumerate}
  \item Case {\sc E-Put-Err}: We have $\config{S_a}{e_{a_2}} =
    \error$.

    We proceed by case analysis on the rule by which
    $\config{S}{e_{b_1}}$ steps to $\config{S_b}{e_{b_2}}$.

    Since the only way an $\error$ configuration can arise is by the
    {\sc E-Put-Err} rule, we can assume in all other cases that
    $\conf_b \neq \error$.
    \begin{enumerate}
    \item \label{slqc-put-err-beta}Case {\sc E-Beta}: By symmetry with case~\ref{slqc-beta-put-err}.
    \item \label{slqc-put-err-new}Case {\sc E-New}: By symmetry with case~\ref{slqc-new-put-err}.
    \item \label{slqc-put-err-put}Case {\sc E-Put}: By symmetry with case~\ref{slqc-put-put-err}.
    \item \label{slqc-put-err-put-err}Case {\sc E-Put-Err}: We have
      $\config{S_a}{e_{a_2}} = \error$ and $\config{S_b}{e_{b_2}} =
      \error$, and so we choose $\conf_c = \error$, $i = 0$, $j = 0$,
      and $\pi = \id$.

      We have to show that:
      \begin{itemize}
      \item $\config{S_a}{\evalctxt{E'_b}{e_{b_1}}} = \error$, and
      \item $\config{S_b}{\evalctxt{E'_a}{e_{a_1}}} = \error$.
      \end{itemize}

      Since $\config{S_a}{e_{a_2}} = \error$, $S_a = \topS$, and since
      $\config{S_b}{e_{b_2}} = \error$, $S_b = \topS$, so both of the
      above follow immediately.

    \item \label{slqc-put-err-get}Case {\sc E-Get}: Similar to
      case~\ref{slqc-put-err-beta}, since $\config{S_a}{e_{a_2}} =
      \error$ and $S_b = S$.
    \item \label{slqc-put-err-freeze-init}Case {\sc E-Freeze-Init}:
      Similar to case~\ref{slqc-put-err-beta}, since
      $\config{S_a}{e_{a_2}} = \error$ and $S_b = S$.
    \item \label{slqc-put-err-spawn-handler}Case {\sc
      E-Spawn-Handler}: Similar to case~\ref{slqc-put-err-beta}, since
      $\config{S_a}{e_{a_2}} = \error$ and $S_b = S$.
    \item \label{slqc-put-err-freeze-final}Case {\sc E-Freeze-Final}:
      We have $\config{S_a}{e_{a_2}} = \error$ and $S_b =
      \extS{S}{l}{d_1}{\frozentrue}$, and so we choose $\conf_c =
      \error$, $i = 0$, $j = 1$, and $\pi = \id$.

      We have to show that:
      \begin{itemize}
      \item $\config{S_a}{\evalctxt{E'_b}{e_{b_1}}} = \error$,
        and
      \item $\config{S_b}{\evalctxt{E'_a}{e_{a_1}}} \ctxstepsto
        \error$.
      \end{itemize}

      The first of these is immediately true because since
      $\config{S_a}{e_{a_2}} = \error$, $S_a = \topS$, and so
      $\config{S_a}{\evalctxt{E'_b}{e_{b_1}}}$ is equal to $\error$ as
      well.

      For the second, observe that since $\config{S}{e_{b_1}}
      \parstepsto \config{S_b}{e_{b_2}}$, we have by
      Lemma~\ref{lem:monotonicity} (Monotonicity) that
      $\leqstore{S}{S_b}$.

      Therefore, since $\config{S}{e_{a_1}} \parstepsto \error$, we
      have by Lemma~\ref{lem:error-preservation} that
      $\config{S_b}{e_{a_1}} \parstepsto \error$.

      Since $\error$ is equal to $\config{\topS}{e}$ for all
      expressions $e$, $\config{S_b}{e_{a_1}} \parstepsto
      \config{\topS}{e}$ for all $e$.

      Therefore, by {\sc E-Eval-Ctxt},
      $\config{S_b}{\evalctxt{E'_a}{e_{a_1}}} \ctxstepsto
      \config{\topS}{\evalctxt{E'_a}{e}}$ for all $e$.

      Since $\config{\topS}{\evalctxt{E'_a}{e}}$ is equal to $\error$,
      we have that $\config{S_b}{\evalctxt{E'_a}{e_{a_1}}} \ctxstepsto
      \error$, as we were required to show.

    \item \label{slqc-put-err-freeze-simple}Case {\sc
      E-Freeze-Simple}: Similar to
      case~\ref{slqc-put-err-freeze-final}, since $S_b =
      \extS{S}{l}{d_1}{\frozentrue}$.

    \end{enumerate}
  \item Case {\sc E-Get}: We have $S_a = S$.

    We proceed by case analysis on the rule by which
    $\config{S}{e_{b_1}}$ steps to $\config{S_b}{e_{b_2}}$.

    Since the only way an $\error$ configuration can arise is by the
    {\sc E-Put-Err} rule, we can assume in all other cases that
    $\conf_b \neq \error$.
    \begin{enumerate}
    \item \label{slqc-get-beta}Case {\sc E-Beta}: By symmetry with case~\ref{slqc-beta-get}.
    \item \label{slqc-get-new}Case {\sc E-New}: By symmetry with case~\ref{slqc-new-get}.
    \item \label{slqc-get-put}Case {\sc E-Put}: By symmetry with case~\ref{slqc-put-get}.
    \item \label{slqc-get-put-err}Case {\sc E-Put-Err}: By symmetry with case~\ref{slqc-put-err-get}.
    \item \label{slqc-get-get}Case {\sc E-Get}: Similar to
      case~\ref{slqc-get-beta}, since $S_a = S$ and $S_b = S$.
    \item \label{slqc-get-freeze-init}Case {\sc E-Freeze-Init}:
      Similar to case~\ref{slqc-get-beta}, since $S_a = S$ and $S_b = S$.
    \item \label{slqc-get-spawn-handler}Case {\sc E-Spawn-Handler}:
      Similar to case~\ref{slqc-get-beta}, since $S_a = S$ and $S_b = S$.
    \item \label{slqc-get-freeze-final}Case {\sc E-Freeze-Final}:
      Similar to case~\ref{slqc-beta-freeze-final}, since $S_a = S$
      and $S_b = \extS{S}{l}{d_1}{\frozentrue}$.
    \item \label{slqc-get-freeze-simple}Case {\sc E-Freeze-Simple}:
      Similar to case~\ref{slqc-beta-freeze-simple}, since $S_a = S$
      and $S_b = \extS{S}{l}{d_1}{\frozentrue}$.
    \end{enumerate}

  \item Case {\sc E-Freeze-Init}: We have $S_a = S$.

    We proceed by case analysis on the rule by which
    $\config{S}{e_{b_1}}$ steps to $\config{S_b}{e_{b_2}}$.

    Since the only way an $\error$ configuration can arise is by the
    {\sc E-Put-Err} rule, we can assume in all other cases that
    $\conf_b \neq \error$.
    \begin{enumerate}
    \item \label{slqc-freeze-init-beta}Case {\sc E-Beta}: By symmetry with case~\ref{slqc-beta-freeze-init}.
    \item \label{slqc-freeze-init-new}Case {\sc E-New}: By symmetry with case~\ref{slqc-new-freeze-init}.
    \item \label{slqc-freeze-init-put}Case {\sc E-Put}: By symmetry with case~\ref{slqc-put-freeze-init}.
    \item \label{slqc-freeze-init-put-err}Case {\sc E-Put-Err}: By symmetry with case~\ref{slqc-put-err-freeze-init}.
    \item \label{slqc-freeze-init-get}Case {\sc E-Get}: By symmetry with case~\ref{slqc-get-freeze-init}.
    \item \label{slqc-freeze-init-freeze-init}Case {\sc
      E-Freeze-Init}: Similar to case~\ref{slqc-freeze-init-beta},
      since $S_a = S$ and $S_b = S$.
    \item \label{slqc-freeze-init-spawn-handler}Case {\sc
      E-Spawn-Handler}: Similar to case~\ref{slqc-freeze-init-beta},
      since $S_a = S$ and $S_b = S$.
    \item \label{slqc-freeze-init-freeze-final}Case {\sc
      E-Freeze-Final}: Similar to case~\ref{slqc-beta-freeze-final},
      since $S_a = S$ and $S_b = \extS{S}{l}{d_1}{\frozentrue}$.
    \item \label{slqc-freeze-init-freeze-simple}Case {\sc
      E-Freeze-Simple}: Similar to case~\ref{slqc-beta-freeze-simple},
      since $S_a = S$ and $S_b = \extS{S}{l}{d_1}{\frozentrue}$.
    \end{enumerate}

  \item Case {\sc E-Spawn-Handler}: We have $S_a = S$.

    We proceed by case analysis on the rule by which
    $\config{S}{e_{b_1}}$ steps to $\config{S_b}{e_{b_2}}$.

    Since the only way an $\error$ configuration can arise is by the
    {\sc E-Put-Err} rule, we can assume in all other cases that
    $\conf_b \neq \error$.
    \begin{enumerate}
    \item \label{slqc-spawn-handler-beta}Case {\sc E-Beta}: By symmetry with case~\ref{slqc-beta-spawn-handler}.
    \item \label{slqc-spawn-handler-new}Case {\sc E-New}: By symmetry with case~\ref{slqc-new-spawn-handler}.
    \item \label{slqc-spawn-handler-put}Case {\sc E-Put}: By symmetry with case~\ref{slqc-put-spawn-handler}.
    \item \label{slqc-spawn-handler-put-err}Case {\sc E-Put-Err}: By symmetry with case~\ref{slqc-put-err-spawn-handler}.
    \item \label{slqc-spawn-handler-get}Case {\sc E-Get}: By symmetry with case~\ref{slqc-get-spawn-handler}.
    \item \label{slqc-spawn-handler-freeze-init}Case {\sc E-Freeze-Init}: By symmetry with case~\ref{slqc-freeze-init-spawn-handler}.
    \item \label{slqc-spawn-handler-spawn-handler}Case {\sc
      E-Spawn-Handler}: Similar to case~\ref{slqc-spawn-handler-beta},
      since $S_a = S$ and $S_b = S$.
    \item \label{slqc-spawn-handler-freeze-final}Case {\sc
      E-Freeze-Final}: Similar to case~\ref{slqc-beta-freeze-final},
      since $S_a = S$ and $S_b = \extS{S}{l}{d_1}{\frozentrue}$.
    \item \label{slqc-spawn-handler-freeze-simple}Case {\sc
      E-Freeze-Simple}: Similar to case~\ref{slqc-beta-freeze-simple},
      since $S_a = S$ and $S_b = \extS{S}{l}{d_1}{\frozentrue}$.
    \end{enumerate}

  \item Case {\sc E-Freeze-Final}: We have $S_a =
    \extS{S}{l}{d_1}{\frozentrue}$.

    We proceed by case analysis on the rule by which
    $\config{S}{e_{b_1}}$ steps to $\config{S_b}{e_{b_2}}$.

    Since the only way an $\error$ configuration can arise is by the
    {\sc E-Put-Err} rule, we can assume in all other cases that
    $\conf_b \neq \error$.
    \begin{enumerate}
    \item \label{slqc-freeze-final-beta}Case {\sc E-Beta}: By symmetry with case~\ref{slqc-beta-freeze-final}.
    \item \label{slqc-freeze-final-new}Case {\sc E-New}: By symmetry with case~\ref{slqc-new-freeze-final}.
    \item \label{slqc-freeze-final-put}Case {\sc E-Put}: By symmetry with case~\ref{slqc-put-freeze-final}.
    \item \label{slqc-freeze-final-put-err}Case {\sc E-Put-Err}: By symmetry with case~\ref{slqc-put-err-freeze-final}.
    \item \label{slqc-freeze-final-get}Case {\sc E-Get}: By symmetry with case~\ref{slqc-get-freeze-final}.
    \item \label{slqc-freeze-final-freeze-init}Case {\sc E-Freeze-Init}: By symmetry with case~\ref{slqc-freeze-init-freeze-final}.
    \item \label{slqc-freeze-final-spawn-handler}Case {\sc E-Spawn-Handler}: By symmetry with case~\ref{slqc-spawn-handler-freeze-final}.
    \item \label{slqc-freeze-final-freeze-final}Case {\sc
      E-Freeze-Final}: We have $S_a = \extS{S}{l}{d_1}{\frozentrue}$
      and $S_b = \extS{S}{l'}{d'_1}{\frozentrue}$.

      Now consider whether $l = l'$:
      \begin{itemize}
      \item If $l \neq l'$:

        Choose $S' =
        \extS{\extS{S}{l'}{d'_1}{\frozentrue}}{l}{d_1}{\frozentrue}$,
        $i = 1$, $j = 1$, and $\pi = \id$.

        We have to show that:
        \begin{itemize}
        \item
          $\config{\extS{S}{l}{d_1}{\frozentrue}}{\evalctxt{E'_b}{e_{b_1}}}
          \ctxstepsto
          \config{\extS{\extS{S}{l'}{d'_1}{\frozentrue}}{l}{d_1}{\frozentrue}}{\evalctxt{E'_b}{e_{b_2}}}$,
          and
        \item
          $\config{\extS{S}{l'}{d'_1}{\frozentrue}}{\evalctxt{E'_a}{e_{a_1}}}
          \ctxstepsto
          \config{\extS{\extS{S}{l'}{d'_1}{\frozentrue}}{l}{d_1}{\frozentrue}}{\evalctxt{E'_a}{e_{a_2}}}$.
        \end{itemize}

        For the first of these, consider that
        $\extS{S}{l}{d_1}{\frozentrue} = U_S(S)$, where $U_S$ is the
        store update operation that freezes the contents of $l$
        and acts as the identity on the contents of all other
        locations.

        Note that:
        \begin{itemize}
        \item $U_S$ is non-conflicting with $\config{S}{e_{b_1}}
          \parstepsto
          \config{\extS{S}{l'}{d'_1}{\frozentrue}}{e_{b_2}}$, since
          no locations are allocated in the transition;
        \item $U_S(\extS{S}{l'}{d'_1}{\frozentrue}) \neq \topS$,

          since $U_S(\extS{S}{l'}{d'_1}{\frozentrue}) =
          \extS{\extS{S}{l'}{d'_1}{\frozentrue}}{l}{d_1}{\frozentrue}$
          and we know $S \neq \topS$ and freezing the contents of
          locations $l$ and $l'$ in $S$ cannot cause it to become
          $\topS$ (because if so, then we would have $S_a = \topS$
          or $S_b = \topS$, which we know are not the case); and
        \item $U_S$ is freeze-safe with $\config{S}{e_{b_1}}
          \parstepsto
          \config{\extS{S}{l'}{d'_1}{\frozentrue}}{e_{b_2}}$, since
          the only location that can change in status between $S$
          and $\extS{S}{l'}{d'_1}{\frozentrue}$ is $l'$, and $U_S$
          acts as the identity on $l'$.
        \end{itemize}
        Therefore, by Lemma~\ref{lem:generalized-independence}
        (Generalized Independence), we have that

        $\config{U_S(S)}{e_{b_1}} \parstepsto
        \config{U_S(\extS{S}{l'}{d'_1}{\frozentrue})}{e_{b_2}}$.

        Hence $\config{\extS{S}{l}{d_1}{\frozentrue}}{e_{b_1}}
        \parstepsto
        \config{\extS{\extS{S}{l'}{d'_1}{\frozentrue}}{l}{d_1}{\frozentrue}}{e_{b_2}}$.

        By {\sc E-Eval-Ctxt}, it follows that

        $\config{\extS{S}{l}{d_1}{\frozentrue}}{\evalctxt{E'_b}{e_{b_1}}}
        \ctxstepsto
        \config{\extS{\extS{S}{l'}{d'_1}{\frozentrue}}{l}{d_1}{\frozentrue}}{\evalctxt{E'_b}{e_{b_2}}}$,
        
        as we were required to show.

        The argument for the second is symmetrical.

      \item If $l = l'$:

        \lk{This is the case where we freeze the same location twice,
          which is no problem; the second freeze is a no-op.}

        Note that since $l = l'$, $d_1 = d'_1$ as well.

        Choose $S' = \extS{S}{l}{d_1}{\frozentrue}$, $i = 1$, $j =
        1$, and $\pi = \id$.

        We have to show that:
        \begin{itemize}
        \item
          $\config{\extS{S}{l}{d_1}{\frozentrue}}{\evalctxt{E'_b}{e_{b_1}}}
          \ctxstepsto
          \config{\extS{S}{l}{d_1}{\frozentrue}}{\evalctxt{E'_b}{e_{b_2}}}$,
          and
        \item
          $\config{\extS{S}{l'}{d'_1}{\frozentrue}}{\evalctxt{E'_a}{e_{a_1}}}
          \ctxstepsto
          \config{\extS{S}{l}{d_1}{\frozentrue}}{\evalctxt{E'_a}{e_{a_2}}}$.
        \end{itemize}

        For the first of these, consider that
        $\extS{S}{l}{d_1}{\frozentrue} = U_S(S)$, where $U_S$ is the
        store update operation that freezes the contents of $l$ and
        acts as the identity on the contents of all other locations.

        Note that:
        \begin{itemize}
        \item $U_S$ is non-conflicting with $\config{S}{e_{b_1}}
          \parstepsto
          \config{\extS{S}{l}{d_1}{\frozentrue}}{e_{b_2}}$, since no
          locations are allocated in the transition;
        \item $U_S(\extS{S}{l}{d_1}{\frozentrue}) \neq \topS$, since
          $U_S(\extS{S}{l}{d_1}{\frozentrue}) =
          \extS{S}{l}{d_1}{\frozentrue}$, and we know $S \neq \topS$
          and freezing the contents of location $l$ in $S$ cannot
          cause it to become $\topS$; and
        \item $U_S$ is freeze-safe with $\config{S}{e_{b_1}}
          \parstepsto
          \config{\extS{S}{l}{d_1}{\frozentrue}}{e_{b_2}}$, since
          the only location that can change in status between $S$
          and $\extS{S}{l}{d_1}{\frozentrue}$ is $l$, and $U_S$
          freezes the contents of $l$ but has no other effect on
          them.
        \end{itemize}

        Therefore, by Lemma~\ref{lem:generalized-independence}
        (Generalized Independence), we have that

        $\config{U_S(S)}{e_{b_1}} \parstepsto
        \config{U_S(\extS{S}{l}{d_1}{\frozentrue})}{e_{b_2}}$.

        Hence $\config{\extS{S}{l}{d_1}{\frozentrue}}{e_{b_1}}
        \parstepsto
        \config{\extS{S}{l}{d_1}{\frozentrue}}{e_{b_2}}$.

        By {\sc E-Eval-Ctxt}, it follows that

        $\config{\extS{S}{l}{d_1}{\frozentrue}}{\evalctxt{E'_b}{e_{b_1}}}
        \ctxstepsto
        \config{\extS{S}{l}{d_1}{\frozentrue}}{\evalctxt{E'_b}{e_{b_2}}}$,

        as we were required to show.

        The argument for the second is symmetrical.

      \end{itemize}

    \item \label{slqc-freeze-final-freeze-simple}Case {\sc
      E-Freeze-Simple}: Similar to
      case~\ref{slqc-freeze-final-freeze-final}, since $S_a =
      \extS{S}{l}{d_1}{\frozentrue}$ and $S_b =
      \extS{S}{l'}{d'_1}{\frozentrue}$.
    \end{enumerate}

  \item Case {\sc E-Freeze-Simple}: We have $S_a =
    \extS{S}{l}{d_1}{\frozentrue}$.

    \begin{enumerate}
    \item \label{slqc-freeze-simple-beta}Case {\sc E-Beta}: By symmetry with case~\ref{slqc-beta-freeze-simple}.
    \item \label{slqc-freeze-simple-new}Case {\sc E-New}: By symmetry with case~\ref{slqc-new-freeze-simple}.
    \item \label{slqc-freeze-simple-put}Case {\sc E-Put}: By symmetry with case~\ref{slqc-put-freeze-simple}.
    \item \label{slqc-freeze-simple-put-err}Case {\sc E-Put-Err}: By symmetry with case~\ref{slqc-put-err-freeze-simple}.
    \item \label{slqc-freeze-simple-get}Case {\sc E-Get}: By symmetry with case~\ref{slqc-get-freeze-simple}.
    \item \label{slqc-freeze-simple-freeze-init}Case {\sc E-Freeze-Init}: By symmetry with case~\ref{slqc-freeze-init-freeze-simple}.
    \item \label{slqc-freeze-simple-spawn-handler}Case {\sc E-Spawn-Handler}: By symmetry with case~\ref{slqc-spawn-handler-freeze-simple}.
    \item \label{slqc-freeze-simple-freeze-final}Case {\sc E-Freeze-Final}: By symmetry with case~\ref{slqc-freeze-final-freeze-simple}.
    \item \label{slqc-freeze-simple-freeze-simple}Case {\sc
      E-Freeze-Simple}: Similar to
      case~\ref{slqc-freeze-simple-freeze-final}, since $S_a =
      \extS{S}{l}{d_1}{\frozentrue}$ and $S_b =
      \extS{S}{l'}{d'_1}{\frozentrue}$.
    \end{enumerate}

  \end{enumerate}
\end{proof}



\section{Proof of Lemma~\ref{lem:strong-one-sided-quasi-confluence}}\label{section:strong-one-sided-quasi-confluence-proof}
\begin{proof}
  Suppose $\conf \ctxstepsto \conf'$ and $\conf \ctxstepsto^m
  \conf''$, where $1 \leq m$.

  We are required to show that either:
  \begin{enumerate}
  \item there exist $\conf_c, i, j, \pi$ such that $\conf'
    \ctxstepsto^i \conf_c$ and $\pi(\conf'') \ctxstepsto^j \conf_c$
    and $i \leq m$ and $j \leq 1$, or
  \item there exists $k \leq m$ such that $\conf' \ctxstepsto^k
    \textup{\error}$, or there exists $k \leq 1$ such that $\conf''
    \ctxstepsto^k \textup{\error}$.
  \end{enumerate}

  We proceed by induction on $m$.

  In the base case of $m = 1$, the result is immediate from
  Lemma~\ref{lem:strong-local-quasi-confluence}, with $k = 1$.

  For the induction step, suppose $\conf \ctxstepsto^m \conf''
  \ctxstepsto \conf'''$ and suppose the lemma holds for $m$.

  We show that it holds for $m + 1$, as follows.

  From the induction hypothesis, we have that either:
  \begin{enumerate}
  \item there exist $\conf_c', i', j', \pi'$ such that $\conf'
    \ctxstepsto^{i'} \conf_c'$ and $\pi'(\conf'') \ctxstepsto^{j'}
    \conf_c'$ and $i' \leq m$ and $j' \leq 1$, or
  \item there exists $k' \leq m$ such that $\conf'
    \ctxstepsto^{k'} \error$, or there exists $k' \leq 1$ such that
    $\conf'' \ctxstepsto^{k'} \error$.
  \end{enumerate}

  We consider these two cases in turn:
  \begin{enumerate}
  \item There exist $\conf_c', i', j', \pi'$ such that $\conf'
    \ctxstepsto^{i'} \conf_c'$ and $\pi'(\conf'') \ctxstepsto^{j'}
    \conf_c'$ and $i' \leq m$ and $j' \leq 1$:

    We proceed by cases on $j'$:
    \begin{itemize}

    \item If $j' = 0$, then $\pi'(\conf'') = \conf_c'$.

      Since $\conf'' \ctxstepsto \conf'''$, we have that
      $\pi'(\conf'') \ctxstepsto \pi'(\conf''')$ by
      Lemma~\ref{lem:permutability} (Permutability).

      We can then choose $\conf_c = \pi'(\conf''')$ and $i = i' + 1$
      and $j = 0$ and $\pi = \pi'$.

      The key is that $\conf' \ctxstepsto^{i'} \conf'_c =
      \pi'(\conf'') \ctxstepsto \pi'(\conf''')$ for a total of $i' +
      1$ steps.
      
    \item If $j' = 1$:

      First, since $\pi'(\conf'') \ctxstepsto^{j'} \conf'_c$, then
      by Lemma~\ref{lem:permutability} (Permutability) we have that
      $\conf'' \ctxstepsto^{j'} \piprimeinv(\conf'_c)$.
      
      Then, by $\conf'' \ctxstepsto^{j'} \piprimeinv(\conf'_c)$ and
      $\conf'' \ctxstepsto \conf'''$ and
      Lemma~\ref{lem:strong-local-quasi-confluence}, one of the
      following two cases is true:
      \begin{enumerate}
      \item There exist $\conf_c''$ and $i''$ and $j''$ and $\pi''$
        such that $\piprimeinv(\conf'_c) \ctxstepsto^{i''}
        \conf_c''$ and $\pi''(\conf''') \ctxstepsto^{j''} \conf_c''$
        and $i'' \leq 1$ and $j'' \leq 1$.

        Since $\piprimeinv(\conf'_c) \ctxstepsto^{i''} \conf_c''$,
        by Lemma~\ref{lem:permutability} (Permutability) we have
        that $\conf'_c \ctxstepsto^{i''} \pi'(\conf_c'')$.

        So we also have $\conf' \ctxstepsto^{i'} \conf_c'
        \ctxstepsto^{i''} \pi'(\conf_c'')$.

        Since $\pi''(\conf''') \ctxstepsto^{j''} \conf_c''$, by
        Lemma~\ref{lem:permutability} (Permutability) we have that
        $\pi'(\pi''(\conf''')) \ctxstepsto^{j''} \pi'(\conf_c'')$.

        In summary, we pick $\conf_c = \pi'(\conf_c'')$ and $i = i' + i''$
        and $j = j''$ and $\pi = \pi'' \circ \pi'$, which is sufficient
        because $i = i' + i'' \leq m + 1$ and $j = j'' \leq 1$.

      \item $\piprimeinv(\conf'_c) \ctxstepsto \error$ or $\conf'''
        \ctxstepsto \error$.

        If $\conf''' \ctxstepsto \error$, then choosing $k = 1$
        satisfies the proof.

        Otherwise, $\piprimeinv(\conf'_c) \ctxstepsto \error$.

        Then, by Lemma~\ref{lem:permutability} we have that
        $\conf'_c \ctxstepsto \pi'(\error)$.

        By Definition~\ref{def:permutation-configuration},
        $\pi'(\error) = \error$, and so $\conf'_c \ctxstepsto
        \error$.

        Therefore $\conf' \ctxstepsto^{i'} \conf'_c \ctxstepsto
        \error$.

        Hence $\conf' \ctxstepsto^{i'+1} \error$.

        Since $i' \leq m$, we have that $i' + 1 \leq m + 1$, and
        so choosing $k = i' + 1$ satisfies the proof.
        
      \end{enumerate}

    \end{itemize}

  \item There exists $k' \leq m$ such that $\conf' \ctxstepsto^{k'}
    \error$, or there exists $k' \leq 1$ such that $\conf''
    \ctxstepsto^{k'} \error$:

    If there exists $k' \leq m$ such that $\conf' \ctxstepsto^{k'}
    \error$, then choosing $k = k'$ satisfies the proof.

    Otherwise, there exists $k' \leq 1$ such that $\conf''
    \ctxstepsto^{k'} \error$.

    We proceed by cases on $k'$:

    \begin{itemize}

    \item If $k' = 0$, then $\conf'' = \error$.

      Hence this case is not possible, since $\conf'' \ctxstepsto
      \conf'''$ and $\error$ cannot step.

    \item If $k' = 1$:

      From $\conf'' \ctxstepsto \conf'''$ and $\conf''
      \ctxstepsto^{k'} \error$ and
      Lemma~\ref{lem:strong-local-quasi-confluence}, one of the
      following two cases is true:

      \begin{enumerate}
      \item There exist $\conf_c''$ and $i''$ and $j''$ and $\pi''$
        such that $\error \ctxstepsto^{i''} \conf_c''$ and
        $\pi''(\conf''') \ctxstepsto^{j''} \conf_c''$ and $i'' \leq
        1$ and $j'' \leq 1$.

        Since $\error$ cannot step, $i'' = 0$ and $\conf''_c =
        \error$.

        By Definition~\ref{def:permutation-configuration},
        $\pi''(\conf''') = \conf'''$.

        Hence $\conf''' \ctxstepsto^{j''} \error$.

        \lk{This is the one place that we need to allow $k$ to be
          $\leq$ 1 instead of exactly 1.}

        Since $j'' \leq 1$, choosing $k = j''$ satisfies the proof.

      \item $\error \ctxstepsto \error$ or $\conf''' \ctxstepsto
        \error$.

        Since $\error$ cannot step, $\conf''' \ctxstepsto \error$.

        Hence choosing $k = 1$ satisfies the proof.

      \end{enumerate}

    \end{itemize}

  \end{enumerate}

\end{proof}


\section{Proof of Lemma~\ref{lem:strong-quasi-confluence}}\label{section:strong-quasi-confluence-proof}
\begin{proof}
  We proceed by induction on $n$.  In the base case of $n = 1$, the
  result is immediate from Lemma~\ref{lem:strong-one-sided-quasi-confluence}.

  For the induction step, suppose $\conf \parstepsto^n \conf'
  \parstepsto \conf'''$ and suppose the lemma holds for $n$.

  We show that it holds for $n + 1$, as follows.

  We are required to show that either:
  \begin{enumerate}
  \item there exist $\conf_c, i, j$ such that $\conf''' \parstepsto^i
    \conf_c$ and $\conf'' \parstepsto^j \conf_c$ and $i \leq m$ and $j
    \leq n + 1$, or
  \item there exists $k \leq m$ such that $\conf''' \parstepsto^k
    \error$, or there exists $k \leq n + 1$ such that $\conf''
    \parstepsto^k \error$.
  \end{enumerate}

  From the induction hypothesis, we have that either:
  \begin{enumerate}
  \item there exist $\conf'_c, i', j'$ such that $\conf'
    \parstepsto^{i'} \conf'_c$ and $\conf'' \parstepsto^{j'} \conf'_c$
    and $i' \leq m$ and $j' \leq n$, or
  \item there exists $k' \leq m$ such that $\conf' \parstepsto^{k'}
    \error$, or there exists $k' \leq n$ such that $\conf''
    \parstepsto^{k'} \error$.
  \end{enumerate}

  We consider these two cases in turn:

  \begin{enumerate}
  \item There exist $\conf'_c, i', j'$ such that $\conf'
    \parstepsto^{i'} \conf'_c$ and $\conf'' \parstepsto^{j'} \conf'_c$
    and $i' \leq m$ and $j' \leq n$:

    We proceed by cases on $i'$:
    \begin{itemize}

    \item If $i' = 0$, then $\conf' = \conf_c'$.  We can then choose
      $\conf_c = \conf'''$ and $i = 0$ and $j = j' + 1$.

    \item If $i' \geq 1$:

      From $\conf' \parstepsto \conf'''$ and $\conf' \parstepsto^{i'}
      \conf_c'$ and Lemma~\ref{lem:strong-one-sided-quasi-confluence},
      one of the following two cases is true:
      \begin{enumerate}
        \item There exist $\conf_c''$ and $i''$ and $j''$ such that
          $\conf''' \parstepsto^{i''} \conf_c''$ and $\conf_c'
          \parstepsto^{j''} \conf_c''$ and $i'' \leq i'$ and $j'' \leq
          1$.  So we also have $\conf'' \parstepsto^{j'} \conf_c'
          \parstepsto^{j''} \conf_c''$.  In summary, we pick $\conf_c
          = \conf_c''$ and $i = i''$ and $j = j' + j''$, which is
          sufficient because $i = i'' \leq i' \leq m$ and $j = j' +
          j'' \leq n + 1$.
        \item There exists $k'' \leq i'$ such that $\conf'''
          \parstepsto^{k''} \error$, or there exists $k'' \leq 1$ such
          that $\conf'_c \parstepsto^{k''} \error$.

          If there exists $k'' \leq i'$ such that $\conf'''
          \parstepsto^{k''} \error$, then choosing $k = k''$ satisfies
          the proof, since $k'' \leq i' \leq m$.

          Otherwise, there exists $k'' \leq 1$ such
          that $\conf'_c \parstepsto^{k''} \error$.

          Therefore, $\conf'' \parstepsto^{j'} \conf_c'
          \parstepsto^{k''} \error$.

          Hence $\conf'' \parstepsto^{j' + k''} \error$.

          Since $j' \leq n$ and $k'' \leq 1$, $j' + k'' \leq n + 1$.

          Hence choosing $k = j' + k''$ satisfies the proof.

      \end{enumerate}
    \end{itemize}

  \item There exists $k' \leq m$ such that $\conf' \parstepsto^{k'}
    \error$, or there exists $k' \leq n$ such that $\conf''
    \parstepsto^{k'} \error$:

    If there exists $k' \leq n$ such that $\conf'' \parstepsto^{k'}
    \error$, then choosing $k = k'$ satisfies the proof.

    Otherwise, there exists $k' \leq m$ such that $\conf'
    \parstepsto^{k'} \error$.  We proceed by cases on $k'$:

    \begin{itemize}

    \item If $k' = 0$, then $\conf' = \error$.

      Hence this case is not possible, since $\conf' \parstepsto
      \conf'''$ and $\error$ cannot step.

    \item If $k' \geq 1$:

      From $\conf' \parstepsto \conf'''$ and $\conf' \parstepsto^{k'}
      \error$ and Lemma~\ref{lem:strong-one-sided-quasi-confluence},
      one of the following two cases is true:

      \begin{enumerate}
        \item There exist $\conf''_c$ and $i''$ and $j''$ such that
          $\conf''' \parstepsto^{i''} \conf''_c$ and $\error
          \parstepsto^{j''} \conf''_c$ and $i'' \leq k'$ and $j'' \leq
          1$.

          Since $\error$ cannot step, $j'' = 0$ and $\conf''_c =
          \error$.

          Hence $\conf''' \parstepsto^{i''} \error$.

          Since $i'' \leq k' \leq m$, choosing $k = i''$ satisfies the
          proof.

        \item There exists $k'' \leq k'$ such that $\conf'''
          \parstepsto^{k''} \error$, or there exists $k'' \leq 1$ such
          that $\error \parstepsto^{k''} \error$.

          Since $\error$ cannot step, there exists $k'' \leq k'$ such
          that $\conf''' \parstepsto^{k''} \error$.

          Since $k'' \leq k' \leq m$, choosing $k = k''$ satisfies the
          proof.
      \end{enumerate}
    \end{itemize}
  \end{enumerate}

\end{proof}


\section{Proof of Theorem~\ref{thm:determinism-of-threshold-queries}}\label{section:determinism-of-threshold-queries-proof}
\begin{proof}
  Consider replica $i$ of a threshold CvRDT $(S, \leq, s^0, q, t, u,
  m)$.

  Let $\mathcal{S}$ be a threshold set with respect to
  $(S, \leq)$.

  Consider a method execution $t^{k+1}_i(\mathcal{S})$ (\ie, a
  threshold query that is the $k+1$th method execution on replica $i$,
  with threshold set $\mathcal{S}$ as its argument) that returns some
  set of activation states $S_a \in \mathcal{S}$.

  For part~\ref{thm:this-replica} of the theorem, we have to show that
  threshold queries with $\mathcal{S}$ as their argument will always
  return $S_a$ on subsequent executions at $i$.

  That is, we have to show that, for all $k' > (k+1)$, the threshold
  query $t^{k'}_i(\mathcal{S})$ on $i$ returns $S_a$.

  Since $t^{k+1}_i(\mathcal{S})$ returns $S_a$, from
  Definition~\ref{def:cvrdt-with-threshold-queries} we have that for
  some activation state $s_a \in S_a$, the condition $s_a \leq s^k_i$
  holds.

  Consider arbitrary $k' > (k+1)$.

  Since state is inflationary across updates, we know that the state
  $s^{k'}_i$ after method execution $k'$ is at least $s^k_i$.

  That is, $s^k_i \leq s^{k'}_i$.

  By transitivity of $\leq$, then, $s_a \leq s^{k'}_i$.

  Hence, by Definition~\ref{def:cvrdt-with-threshold-queries},
  $t^{k'}_i(\mathcal{S})$ returns $S_a$.

  For part~\ref{thm:any-replica} of the theorem, consider some replica
  $j$ of $(S, \leq, s^0, q, t, u, m)$, located at process $p_j$.

  We are required to show that, for all $x \geq 0$, the threshold
  query $t^{x+1}_j(\mathcal{S})$ returns $S_a$ eventually, and blocks
  until it does.\footnote{The occurrences of $k+1$ and $x+1$ in this
    proof are an artifact of how we index method executions starting
    from $1$, but states starting from $0$.  The initial state (of
    every replica) is $s^0$, and so $s^k_i$ is the state of replica
    $i$ after method execution $k$ has completed at $i$.}

  That is, we must show that, for all $x \geq 0$, there exists some
  finite $n \geq 0$ such that
  \begin{itemize}
  \item 
    for all $i$ in the range $0 \leq i \leq n-1$, the threshold query
    $t^{x+1+i}_j(\mathcal{S})$ returns $\block$, and
  \item
    for all $i \geq n$, the threshold query $t^{x+1+i}_j(\mathcal{S})$
    returns $S_a$.
  \end{itemize}
  Consider arbitrary $x \geq 0$.

  Recall that $s^x_j$ is the state of replica $j$ after the $x$th
  method execution, and therefore $s^x_j$ is also the state of $j$
  when $t^{x+1}_j(\mathcal{S})$ runs.
  %
  We have three cases to consider:
  \begin{itemize}
  \item $s^k_i \leq s^x_j$.

    (That is, replica $i$'s state after the $k$th method execution on $i$
    is \emph{at or below} replica $j$'s state after the $x$th method
    execution on $j$.)

    Choose $n = 0$.

    We have to show that, for all $i \geq n$, the threshold query
    $t^{x+1+i}_j(\mathcal{S})$ returns $S_a$.

    Since $t^{k+1}_i(\mathcal{S})$ returns $S_a$, we know that there
    exists an $s_a \in S_a$ such that $s_a \leq s^k_i$.

    Since $s^k_i \leq s^x_j$, we have by transitivity of $\leq$ that
    $s_a \leq s^x_j$.

    Therefore, by Definition~\ref{def:cvrdt-with-threshold-queries},
    $t^{x+1}_j(\mathcal{S})$ returns $S_a$.

    Then, by part~\ref{thm:this-replica} of the theorem, we have that
    subsequent executions $t^{x+1+i}_j(\mathcal{S})$ at replica $j$
    will also return $S_a$, and so the case holds.

    (Note that this case includes the possibility $s^k_i \equiv s^0$,
    in which no updates have executed at replica $i$.)

  \item $s^k_i > s^x_j$.

    (That is, replica $i$'s state after the $k$th method execution on $i$
    is \emph{above} replica $j$'s state after the $x$th method execution
    on $j$.)

    We have two subcases:

    \begin{itemize}
    \item
      There exists some activation state $s'_a \in S_a$ for which $s'_a \leq
      s^x_j$.

      In this case, we choose $n = 0$.

      We have to show that, for all $i \geq n$, the threshold query
      $t^{x+1+i}_j(\mathcal{S})$ returns $S_a$.

      Since $s'_a \leq s^x_j$, by
      Definition~\ref{def:cvrdt-with-threshold-queries},
      $t^{x+1}_j(\mathcal{S})$ returns $S_a$.

      Then, by part~\ref{thm:this-replica} of the theorem, we have
      that subsequent executions $t^{x+1+i}_j(\mathcal{S})$ at replica
      $j$ will also return $S_a$, and so the case holds.

    \item
      There is no activation state $s'_a \in S_a$ for which $s'_a \leq
      s^x_j$.

      Since $t^{k+1}_i(\mathcal{S})$ returns $S_a$, we know that there
      is some update $u^{k'}_i(a)$ in $i$'s causal history, for some
      $k' < (k+1)$, that updates $i$ from a state at or below $s^x_j$
      to $s^k_i$.\footnote{We know that $i$'s state was once at or
        below $s^x_j$, because $i$ and $j$ started at the same state
        $s^0$ and can both only grow.  Hence the least that $s^x_j$
        can be is $s^0$, and we know that $i$ was originally $s^0$ as
        well.}

      By eventual delivery, $u^{k'}_i(a)$ is eventually delivered at
      $j$.

      Hence some update or updates that will increase $j$'s state from
      $s^x_j$ to a state at or above some $s'_a$ must reach replica
      $j$.\footnote{We say ``some update or updates'' because the
        exact update $u^{k'}_i(a)$ may not be the update that causes
        the threshold query at $j$ to unblock; a different update or
        updates could do it.  Nevertheless, the existence of
        $u^{k'}_i(a)$ means that there is at least one update that
        will suffice to unblock the threshold query.}

      Let the $x+1+r$th method execution on $j$ be the first update on $j$
      that updates its state to some $s^{x+1+r}_j \geq s'_a$, for some
      activation state $s'_a \in S_a$.

      Choose $n = r+1$.

      We have to show that, for all $i$ in the range $0 \leq i \leq
      r$, the threshold query $t^{x+1+i}_j(\mathcal{S})$ returns
      $\block$, and that for all $i \geq r+1$, the threshold query
      $t^{x+1+i}_j(\mathcal{S})$ returns $S_a$.

      For the former, since the $x+1+r$th method execution on $j$ is the
      first one that updates its state to $s^{x+1+r}_j \geq s'_a$, we have
      by Definition~\ref{def:cvrdt-with-threshold-queries} that for all $i$
      in the range $0 \leq i \leq r$, the threshold query
      $t^{x+1+i}_j(\mathcal{S})$ returns $\block$.

      For the latter, since $s^{x+1+r}_j \geq s'_a$, by
      Definition~\ref{def:cvrdt-with-threshold-queries} we have that
      $t^{x+1+r+1}_j(\mathcal{S})$ returns $S_a$, and by
      part~\ref{thm:this-replica} of the theorem, we have that for $i \geq
      r+1$, subsequent executions $t^{x+1+i}_j(\mathcal{S})$ at replica $j$
      will also return $S_a$, and so the case holds.
    \end{itemize}

  \item $s^k_i \nleq s^x_j$ and $s^x_j \nleq s^k_i$.

    (That is, replica $i$'s state after the $k$th method execution on $i$
    is \emph{not comparable} to replica $j$'s state after the $x$th method
    execution on $j$.)

    Similar to the previous case.
  \end{itemize}
\end{proof}




% Put this bak the way it was before typesetting refs.
\renewcommand{\section}{\oldsection}

\bibliography{../latex_common/refs,../latex_common/lkuper}

\end{document}
