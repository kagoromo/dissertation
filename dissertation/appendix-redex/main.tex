\chapter{A PLT Redex Model of $\lambdaLVish$}\label{app:plt-redex}

I have developed a runnable version\footnote{Available at
  \url{http://github.com/lkuper/lvar-semantics}.} of the
$\lambdaLVish$ calculus of Chapter~\ref{ch:quasi} using the PLT Redex
semantics engineering toolkit~\cite{redex-book}.  In the Redex of
today, it is not possible to directly parameterize a language
definition by a lattice.\footnote{See discussion at
  \url{http://lists.racket-lang.org/users/archive/2013-April/057075.html}.}
Instead, taking advantage of Racket's syntactic abstraction
capabilities, I define a Racket macro, @define-lambdaLVish-language@,
that serves as a wrapper for a template that implements the
lattice-agnostic semantics of $\lambdaLVish$.  The
@define-lambdaLVish-language@ macro takes the following arguments:
\begin{itemize}
\item a \emph{name}, which becomes the \emph{lang-name} passed to
  Redex's \texttt{define-language} form;
\item a \emph{``downset'' operation}, a Racket-level procedure that
  takes a lattice element and returns the (finite) set of all lattice
  elements that are less than or equal to that element (this operation
  is used to implement the semantics of $\FAW$, in particular, to
  determine when the {\sc E-Freeze-Final} rule can fire);
\item a \emph{lub operation}, a Racket procedure that takes two
  lattice elements and returns a lattice element;
\item a list of \emph{update operations}, Racket procedures that each
  take a lattice element and return a lattice element; and
\item a (possibly infinite) set of \emph{lattice elements} represented
  as Redex \emph{patterns}.
\end{itemize}

\noindent Given these arguments, @define-lambdaLVish-language@
generates a Redex model that is specialized to the appropriate lattice
and set of update operations. For instance, to generate a Redex model
called @lambdaLVish-nat@ where the lattice is the non-negative
integers ordered by $\leq$, and the set of update operations is $\{
u_{(+1)}, u_{(+2)} \}$ where $u_{(+n)}(d)$ increments $d$'s contents
by $n$, one could write:

\singlespacing
\begin{lstlisting}[language=Lisp]
(define-lambdaLVish-language lambdaLVish-nat downset-op max update-ops natural)
\end{lstlisting}
\doublespacing

\noindent where @max@ is a built-in Racket procedure, and @downset-op@
and @update-ops@ can be defined in Racket as follows:

\singlespacing
\begin{lstlisting}[language=Lisp]
(define downset-op
  (lambda (d)
    (if (number? d)
        (append '(Bot) (iota d) `(,d))
        '(Bot))))

(define update-op-1
  (lambda (d)
    (match d
      ['Bot 1]
      [number (add1 d)])))

(define update-op-2
  (lambda (d)
    (match d
      ['Bot 2]
      [number (add1 (add1 d))])))

(define update-ops `(,update-op-1 ,update-op-2))
\end{lstlisting}
\doublespacing

\noindent The last argument, @natural@, is a Redex pattern that
matches any exact non-negative integer.  @natural@ has no meaning to
Racket outside of Redex, but since @define-lambdaLVish-language@ is a
macro rather than an ordinary procedure, its arguments are not
evaluated until they are in the context of Redex, and so passing
@natural@ as an argument to the macro gives us the behavior we want.

The Redex model is not completely faithful to the $\lambdaLVish$
calculus of Chapter~\ref{ch:quasi}: it requires us to specify a finite
list of update operations rather than a possibly infinite set of them.
However, the list of update operations can be arbitrarily long, and
the definitions of the update operations could, at least in principle,
be generated programmatically as well.
