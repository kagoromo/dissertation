\begin{proof}
  Consider an arbitrary permutation $\pi$.  For
  part~\ref{thm:quasi-permutable-reduction-transitions}, we have to
  show that if $\conf \parstepsto \conf'$ then $\pi(\conf) \parstepsto
  \pi(\conf')$, and that if $\pi(\conf) \parstepsto \pi(\conf')$ then
  $\conf \parstepsto \conf'$.

  For the forward direction of
  part~\ref{thm:quasi-permutable-reduction-transitions}, suppose
  $\conf \parstepsto \conf'$.  We have to show that $\pi(\conf)
  \parstepsto \pi(\conf')$.  We proceed by cases on the rule by which
  $\conf$ steps to $\conf'$.

  \begin{itemize}
    \item Case {\sc E-Beta}: $\conf =
      \config{S}{\app{(\lam{x}{e})}{v}}$, and $\conf' =
      \config{S}{\subst{e}{x}{v}}$.

      To show: $\pi(\config{S}{\app{(\lam{x}{e})}{v}}) \parstepsto
      \pi(\config{S}{\subst{e}{x}{v}})$.

      By Definitions~\ref{def:permutation-configuration}
      and~\ref{def:permutation-expression}, $\pi(\conf) =
      \config{\pi(S)}{\app{(\lam{x}{\pi(e)})}{\pi(v)}}$.

      By {\sc E-Beta},
      $\config{\pi(S)}{\app{(\lam{x}{\pi(e)})}{\pi(v)}}$ steps to
      $\config{\pi(S)}{\subst{\pi(e)}{x}{\pi(v)}}$.

      By Definition~\ref{def:permutation-expression},
      $\config{\pi(S)}{\subst{\pi(e)}{x}{\pi(v)}}$ is equal to
      $\config{\pi(S)}{\pi(\subst{e}{x}{v})}$.

      Hence $\config{\pi(S)}{\app{(\lam{x}{\pi(e)})}{\pi(v)}}$ steps
      to $\config{\pi(S)}{\pi(\subst{e}{x}{v})}$, which is equal to
      $\pi(\config{S}{\subst{e}{x}{v}})$ by
      Definition~\ref{def:permutation-configuration}.  Hence the case
      is satisfied.

    \item Case {\sc E-New}: $\conf = \config{S}{\NEW}$, and $\conf' =
      \config{\extS{S}{l}{\bot}{\frozenfalse}}{l}$.

      To show: $\pi(\config{S}{\NEW}) \parstepsto
      \pi(\config{\extS{S}{l}{\bot}{\frozenfalse}}{l})$.

      By Definitions~\ref{def:permutation-configuration}
      and~\ref{def:permutation-expression}, $\pi(\conf) =
      \config{\pi(S)}{\NEW}$.

      By {\sc E-New}, $\config{\pi(S)}{\NEW}$ steps to
      $\config{\extS{(\pi(S))}{l'}{\bot}{\frozenfalse}}{l'}$, where
      $l' \notin \dom{\pi(S)}$.
      
      It remains to show that
      $\config{\extS{(\pi(S))}{l'}{\bot}{\frozenfalse}}{l'}$ is equal
      to $\pi(\config{\extS{S}{l}{\bot}{\frozenfalse}}{l})$.

      By Definition~\ref{def:permutation-configuration},
      $\pi(\config{\extS{S}{l}{\bot}{\frozenfalse}}{l})$ is equal to
      $\config{\pi(\extS{S}{l}{\bot}{\frozenfalse})}{\pi(l)}$, which
      is equal to
      $\config{\extS{(\pi(S))}{\pi(l)}{\bot}{\frozenfalse}}{\pi(l)}$.

      So, we have to show that
      $\config{\extS{(\pi(S))}{l'}{\bot}{\frozenfalse}}{l'}$ is equal
      to
      $\config{\extS{(\pi(S))}{\pi(l)}{\bot}{\frozenfalse}}{\pi(l)}$.
      Since we know (from the side condition of {\sc E-New}) that $l
      \notin \dom{S}$, it follows that $\pi(l) \notin \pi(\dom{S})$.
      Therefore, in
      $\config{\extS{(\pi(S))}{l'}{\bot}{\frozenfalse}}{l'}$, we can
      $\alpha$-rename $l'$ to $\pi(l)$, and so the two configurations
      are equal and the case is satisfied.

    \item Case {\sc E-Put}: $\conf = \config{S}{\putiexp{l}}$, and
      $\conf' = \config{\extSRaw{S}{l}{u_{p_i}(p_1)}}{\unit}$.

      To show: $\pi(\config{S}{\putiexp{l}}) \parstepsto
      \pi(\config{\extSRaw{S}{l}{u_{p_i}(p_1)}}{\unit})$.

      By Definition~\ref{def:permutation-configuration}, $\pi(\conf) =
      \config{\pi(S)}{\putiexp{\pi(l)}}$.

      By {\sc E-Put}, $\config{\pi(S)}{\putiexp{\pi(l)}}$ steps to
      $\config{\extSRaw{(\pi(S))}{\pi(l)}{u_{p_i}(p_1)}}{\unit}$,
      since $S(l) = (\pi(S))(\pi(l)) = p_1$.

      It remains to show that
      $\config{\extSRaw{(\pi(S))}{\pi(l)}{u_{p_i}(p_1)}}{\unit}$ is
      equal to $\pi(\config{\extSRaw{S}{l}{u_{p_i}(p_1)}}{\unit})$.

      By Definitions~\ref{def:permutation-configuration}
      and~\ref{def:permutation-expression},
      $\pi(\config{\extSRaw{S}{l}{u_{p_i}(p_1)}}{\unit})$ is equal to
      $\config{\extSRaw{(\pi(S))}{\pi(l)}{u_{p_i}(p_1)}}{\unit}$, and
      so the two configurations are equal and the case is satisfied.
    \item Case {\sc E-Put-Err}: $\conf = \config{S}{\putiexp{l}}$,
      and $\conf' = \error$.

      To show: $\pi(\config{S}{\putiexp{l}}) \parstepsto
      \pi(\error)$.

      By Definition~\ref{def:permutation-configuration}, $\pi(\conf) =
      \config{\pi(S)}{\putiexp{\pi(l)}}$.

      By {\sc E-Put-Err}, $\config{\pi(S)}{\putiexp{\pi(l)}}$ steps to
      $\error$, since $S(l) = (\pi(S))(\pi(l)) = p_1$.

      Since $\pi(\error) = \error$ by
      Definition~\ref{def:permutation-configuration}, the case is
      complete.
    \item Case {\sc E-Get}: $\conf = \config{S}{\getexp{l}{P}}$, and
      $\conf' = \config{S}{p_2}$.

      To show: $\pi(\config{S}{\getexp{l}{P}}) \parstepsto
      \pi(\config{S}{p_2})$.

      By Definitions~\ref{def:permutation-configuration}
      and~\ref{def:permutation-expression}, $\pi(\conf) =
      \config{\pi(S)}{\getexp{\pi(l)}{P}}$.

      By {\sc E-Get}, $\config{\pi(S)}{\getexp{\pi(l)}{P}}$ steps to
      $\config{\pi(S)}{p_2}$, since $S(l) = (\pi(S))(\pi(l)) = p_1$.

      By Definitions~\ref{def:permutation-configuration}
      and~\ref{def:permutation-expression}, $\pi(\config{S}{p_2}) =
      \config{\pi(S)}{p_2}$.  Therefore the case is complete.
    \item Case {\sc E-Freeze-Init}: $\conf =
      \config{S}{\freezeafter{l}{Q}{\lam{x}{e}}}$, and $\conf' =
      \config{S}{\freezeafterfull{l}{Q}{\lam{x}{e}}{\setof{}}{\setof{}}}$.

      To show: $\pi(\config{S}{\freezeafter{l}{Q}{\lam{x}{e}}})
      \parstepsto
      \pi(\config{S}{\freezeafterfull{l}{Q}{\lam{x}{e}}{\setof{}}{\setof{}}})$.

      By Definitions~\ref{def:permutation-configuration}
      and~\ref{def:permutation-expression}, $\pi(\conf) =
      \config{\pi(S)}{\freezeafter{\pi(l)}{Q}{\lam{x}{\pi(e)}}}$.

      By {\sc E-Freeze-Init},
      $\config{\pi(S)}{\freezeafter{\pi(l)}{Q}{\lam{x}{\pi(e)}}}
      \parstepsto
      \config{\pi(S)}{\freezeafterfull{\pi(l)}{Q}{\lam{x}{\pi(e)}}{\setof{}}{\setof{}}}$.

      By Definitions~\ref{def:permutation-configuration}
      and~\ref{def:permutation-expression},
      $\pi(\config{S}{\freezeafterfull{l}{Q}{\lam{x}{e}}{\setof{}}{\setof{}}})
      =
      \config{\pi(S)}{\freezeafterfull{\pi(l)}{Q}{\lam{x}{\pi(e)}}{\setof{}}{\setof{}}}$.
      Therefore the case is complete.

    \item Case {\sc E-Spawn-Handler}: $\conf =
      \config{S}{\freezeafterfull{l}{Q}{\lam{x}{e_0}}{\setof{e,
            \dots}}{H}}$, and $\conf' =
      \config{S}{\freezeafterfull{l}{Q}{\lam{x}{e_0}}{\setof{\subst{e_0}{x}{d_2},
            e, \dots}} {\{d_2\}\cup H}}$.

      To show:
      $\pi(\config{S}{\freezeafterfull{l}{Q}{\lam{x}{e_0}}{\setof{e,
            \dots}}{H}}) \parstepsto
      \pi(\config{S}{\freezeafterfull{l}{Q}{\lam{x}{e_0}}{\setof{\subst{e_0}{x}{d_2},
            e, \dots}} {\{d_2\}\cup H}})$.

      By Definitions~\ref{def:permutation-configuration}
      and~\ref{def:permutation-expression}, $\pi(\conf) =
      \config{\pi(S)}{\freezeafterfull{\pi(l)}{Q}{\lam{x}{\pi(e_0)}}{\setof{\pi(e),
            \dots}}{H}}$.

      Since $(\pi(S))(\pi(l)) = \state{d_1}{\status_1}$, by {\sc
        E-Spawn-Handler} we have that
      $\config{\pi(S)}{\freezeafterfull{\pi(l)}{Q}{\lam{x}{\pi(e_0)}}{\setof{\pi(e),
            \dots}}{H}} \parstepsto
      \config{\pi(S)}{\freezeafterfull{\pi(l)}{Q}{\lam{x}{\pi(e_0)}}{\setof{\subst{\pi(e_0)}{x}{d_2},
            \pi(e), \dots}} {\{d_2\}\cup H}}$.

      By Definitions~\ref{def:permutation-configuration}
      and~\ref{def:permutation-expression},
      $\pi(\config{S}{\freezeafterfull{l}{Q}{\lam{x}{e_0}}{\setof{\subst{e_0}{x}{d_2},
            e, \dots}} {\{d_2\}\cup H}}) =
      \config{\pi(S)}{\freezeafterfull{\pi(l)}{Q}{\lam{x}{\pi(e_0)}}{\setof{\subst{\pi(e_0)}{x}{d_2},
            \pi(e), \dots}} {\{d_2\}\cup H}}$.  Therefore the case is
      complete.

    \item Case {\sc E-Freeze-Final}: $\conf =
      \config{S}{\freezeafterfull{l}{Q}{\lam{x}{e_0}}{\setof{v,
            \dots}}{H}}$, and $\conf' =
      \config{\extS{S}{l}{d_1}{\frozentrue}}{d_1}$.

      To show:
      $\pi(\config{S}{\freezeafterfull{l}{Q}{\lam{x}{e_0}}{\setof{v,
            \dots}}{H}}) \parstepsto
      \pi(\config{\extS{S}{l}{d_1}{\frozentrue}}{d_1})$.

      By Definitions~\ref{def:permutation-configuration}
      and~\ref{def:permutation-expression}, $\pi(\conf) =
      \config{\pi(S)}{\freezeafterfull{\pi(l)}{Q}{\lam{x}{\pi(e_0)}}{\setof{\pi(v),
            \dots}}{H}}$.

      \lk{How do we handle applying $\pi$ to an arbitrary value $v$?
        I'm just writing $\pi(v)$ because we don't know what kind of
        value it is.}

      \TODO{.}
    \item Case {\sc E-Freeze-Simple}: $\conf =
      \config{S}{\freeze{l}}$, and $\conf' =
      \config{\extS{S}{l}{d_1}{\frozentrue}}{d_1}$.

      To show: $\pi(\config{S}{\freeze{l}}) \parstepsto
      \pi(\config{\extS{S}{l}{d_1}{\frozentrue}}{d_1})$.

      By Definitions~\ref{def:permutation-configuration}
      and~\ref{def:permutation-expression}, $\pi(\conf) =
      \config{\pi(S)}{\freeze{\pi(l)}}$.

      \TODO{.}
  \end{itemize}

  For the reverse direction of
  part~\ref{thm:quasi-permutable-reduction-transitions}, suppose
  $\pi(\conf) \parstepsto \pi(\conf')$.  We have to show that $\conf
  \parstepsto \conf'$.

  We know from the forward direction of the proof that for all
  configurations $\conf$ and $\conf'$ and permutations $\pi$, if
  $\conf \parstepsto \conf'$ then $\pi(\conf) \parstepsto
  \pi(\conf')$.  Hence since $\pi(\conf) \parstepsto \pi(\conf')$, and
  since $\piinv$ is also a permutation, we have that
  $\piinv(\pi(\conf)) \parstepsto \piinv(\pi(\conf'))$.  Since
  $\piinv(\pi(l)) = l$ for every $l \in \Loc$, and that property lifts
  to configurations as well, we have that $\conf \parstepsto \conf'$.

  \lk{Is the above enough of a proof?}

  For the forward direction of
  part~\ref{thm:quasi-permutable-context-transitions}, suppose $\conf
  \ctxstepsto \conf'$.  We have to show that $\pi(\conf) \ctxstepsto
  \pi(\conf')$.

  By inspection of the operational semantics, $\conf$ must be of the
  form $\config{S}{\E{e}}$, and $\conf'$ must be of the form
  $\config{S'}{\E{e'}}$.  Hence we have to show that
  $\pi(\config{S}{\E{e}}) \ctxstepsto \pi(\config{S'}{\E{e'}})$.

  By Definition~\ref{def:permutation-configuration},
  $\pi(\config{S}{\E{e}})$ is equal to $\config{\pi(S)}{\pi(\E{e})}$,
  and $\pi(\config{S'}{\E{e'}})$ is equal to
  $\config{\pi(S')}{\pi(\E{e'})}$.  Furthermore,
  $\config{\pi(S)}{\pi(\E{e})}$ is equal to
  $\config{\pi(S)}{\evalctxt{(\pi(E))}{\pi(e)}}$ and
  $\config{\pi(S')}{\pi(\E{e'})}$ is equal to
  $\config{\pi(S')}{\evalctxt{(\pi(E))}{\pi(e')}}$.

  So we have to show that
  $\config{\pi(S)}{\evalctxt{(\pi(E))}{\pi(e)}} \ctxstepsto
  \config{\pi(S')}{\evalctxt{(\pi(E))}{\pi(e')}}$.

  From the premise of {\sc E-Eval-Ctxt}, $\config{S}{e} \parstepsto
  \config{S'}{e'}$.  Hence, by
  part~\ref{thm:quasi-permutable-reduction-transitions},
  $\pi(\config{S}{e}) \parstepsto \pi(\config{S'}{e'})$.  By
  Definition~\ref{def:permutation-configuration}, $\pi(\config{S}{e})$
  is equal to $\config{\pi(S)}{\pi(e)}$ and $\pi(\config{S'}{e'})$ is
  equal to $\config{\pi(S')}{\pi(e')}$.

  Hence $\config{\pi(S)}{\pi(e)} \parstepsto
  \config{\pi(S')}{\pi(e')}$.  Therefore, by {\sc E-Eval-Ctxt},
  $\config{\pi(S)}{\E{\pi(e)}} \ctxstepsto
  \config{\pi(S')}{\E{\pi(e')}}$ for all evaluation contexts $E$.

  In particular, it is true that
  $\config{\pi(S)}{\evalctxt{(\pi(E))}{\pi(e)}} \ctxstepsto
  \config{\pi(S')}{\evalctxt{(\pi(E))}{\pi(e')}}$, as we were required
  to show.

  For the reverse direction of
  part~\ref{thm:quasi-permutable-context-transitions}, suppose
  $\pi(\conf) \ctxstepsto \pi(\conf')$.  We have to show that $\conf
  \ctxstepsto \conf'$.

  We know from the forward direction of the proof that for all
  configurations $\conf$ and $\conf'$ and permutations $\pi$, if
  $\conf \ctxstepsto \conf'$ then $\pi(\conf) \ctxstepsto
  \pi(\conf')$.  Hence since $\pi(\conf) \ctxstepsto \pi(\conf')$, and
  since $\piinv$ is also a permutation, we have that
  $\piinv(\pi(\conf)) \ctxstepsto \piinv(\pi(\conf'))$.  Since
  $\piinv(\pi(l)) = l$ for every $l \in \Loc$, and that property lifts
  to configurations as well, we have that $\conf \ctxstepsto \conf'$.

  \lk{Is the above enough of a proof?}
\end{proof}
