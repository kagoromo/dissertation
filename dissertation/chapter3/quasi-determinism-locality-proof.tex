\begin{proof}
  Suppose $\config{S}{\evalctxt{E_1}{e_1}} \ctxstepsto \config{S_1}{\evalctxt{E_1}{e'_1}}$ and
  $\config{S}{\evalctxt{E_2}{e_2}} \ctxstepsto
  \config{S_2}{\evalctxt{E_2}{e'_2}}$ and $\evalctxt{E_1}{e_1} =
  \evalctxt{E_2}{e_2}$.

  We are required to show that if $E_1 \neq E_2$, then there exist
  evaluation contexts $E'_1$ and $E'_2$ such that:
  \begin{itemize}
  \item $\evalctxt{E'_1}{e_1} = \evalctxt{E_2}{e'_2}$, and
  \item $\evalctxt{E'_2}{e_2} = \evalctxt{E_1}{e'_1}$, and
  \item $\evalctxt{E'_1}{e'_1} = \evalctxt{E'_2}{e'_2}$.
  \end{itemize}

  Let $e = \evalctxt{E_1}{e_1} = \evalctxt{E_2}{e_2}$.  The proof is
  by induction on the structure of the expression $e$.  Proceeding by
  cases on $e$:

  \begin{itemize}

    \item Cases $e = x$, $e = v$, $e = \app{e_a}{e_b}$, $e = \getexp{e_a}{e_b}$, and
      $e = \NEW$ are identical to their corresponding cases in the proof
      of Lemma~\ref{lem:lvars-locality}.

    \item Case $e = \putiexp{e_a}$:

      From the grammar of evaluation contexts, we know that either:
      \begin{itemize}
        \item $\putiexp{e_a} = \evalctxt{E_1}{e_1} = \evalctxt{E_1}{\putiexp{e_a}}$, where $E_1 = [~]$, or
        \item $\putiexp{e_a} = \evalctxt{E_1}{e_1} =
          \putiexp{\evalctxt{E_{11}}{e_1}}$, where
          $\evalctxt{E_{11}}{e_1} = e_a$.
      \end{itemize}

      Similarly, we know that $\putiexp{e_a} = \evalctxt{E_2}{e_2}$.
      From the grammar of evaluation contexts, we know that either:
      \begin{itemize}
        \item $\putiexp{e_a} = \evalctxt{E_2}{e_2} = \evalctxt{E_2}{\putiexp{e_a}}$, where $E_2 = [~]$, or
        \item $\putiexp{e_a} = \evalctxt{E_2}{e_2} =
          \putiexp{\evalctxt{E_{21}}{e_2}}$, where
          $\evalctxt{E_{21}}{e_2} = e_a$.
      \end{itemize}

      However, if $E_1 = [~]$ or $E_2 = [~]$, then $\putiexp{e_a}$
      must be $\putiexp{v}$ for some $v$, and $v$ cannot step
      individually, so the other of $E_1$ or $E_2$ must be $[~]$ as
      well, and so $E_1 = E_2$.  Therefore the only case that we have
      to consider is the case in which $\evalctxt{E_1}{e_1} =
      \putiexp{\evalctxt{E_{11}}{e_1}}$, where $\evalctxt{E_{11}}{e_1}
      = e_a$, and $\putiexp{e_a} = \evalctxt{E_2}{e_2} =
      \putiexp{\evalctxt{E_{21}}{e_2}}$, where $\evalctxt{E_{21}}{e_2}
      = e_a$.

      So, we have $\evalctxt{E_{11}}{e_1} = e_a$ and
      $\evalctxt{E_{21}}{e_2} = e_a$.  In this case, we know that
      $E_{11} \neq E_{21}$, because if $E_{11} = E_{21}$, we would
      have $e_1 = e_2$, which would mean that $E_1 = E_2$, a
      contradiction.  So, since $E_{11} \neq E_{21}$, by IH we have
      that there exist evaluation contexts $E'_{11}$ and $E'_{21}$
      such that:
      \begin{itemize}
      \item $\evalctxt{E'_{11}}{e_1} = \evalctxt{E_{21}}{e'_2}$, and
      \item $\evalctxt{E'_{21}}{e_2} = \evalctxt{E_{11}}{e'_1}$, and
      \item $\evalctxt{E'_{11}}{e'_1} = \evalctxt{E'_{21}}{e'_2}$.
      \end{itemize}

      Hence we can choose $E'_1 = \putiexp{E'_{11}}$ and $E'_2 =
      \putiexp{E'_{21}}$, which satisfy the criteria for $E'_1$ and
      $E'_2$.

    \item Case $e = \freeze{e_a}$: Similar to the case for $\putiexp{e_a}$.

    \item Case $e = \freezeafter{e_a}{e_b}{e_c}$: Similar to the case
      where $e = \app{e_a}{e_b}$.

    \item Case $e = \freezeafterfull{l}{Q}{\lam{x}{e_a}}{\setof{e_b,
        \dots}}{H}$: \TODO{this case}

  \end{itemize}
\end{proof}
