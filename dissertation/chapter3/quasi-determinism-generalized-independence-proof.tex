\begin{proof}
  Consider arbitrary $U_S$ such that $U_S$ is non-conflicting with
  $\config{S}{e} \parstepsto \config{S'}{e'}$ and $U_S(S') \neq \topS$
  and and $U_S$ is freeze-safe with $\config{S}{e} \parstepsto
  \config{S'}{e'}$.

  To show: $\config{U_S(S)}{e} \parstepsto \config{U_S(S')}{e'}$.

  The proof is by cases on the rule of the reduction semantics by
  which $\config{S}{e}$ steps to $\config{S'}{e'}$.  Since
  $\config{S'}{e'} \neq \error$, we do not need to consider the {\sc
    E-Put-Err} rule.  The assumption that $U_S$ is freeze-safe with
  $\config{S}{e} \parstepsto \config{S'}{e'}$ is only needed in the
  {\sc E-Freeze-Final} and {\sc E-Freeze-Simple} cases.

  \begin{itemize}

    \item Case {\sc E-Beta}:

      Given: $\config{S}{\app{(\lam{x}{e})}{v}} \parstepsto
      \config{S}{\subst{e}{x}{v}}$.

      To show: $\config{U_S(S)}{\app{(\lam{x}{e})}{v}} \parstepsto
      \config{U_S(S)}{\subst{e}{x}{v}}$.

      Immediate by {\sc E-Beta}.

    \item Case {\sc E-New}:

      Given: $\config{S}{\NEW} \parstepsto
      \config{\extS{S}{l}{\bot}{\frozenfalse}}{l}$.

      To show: $\config{U_S(S)}{\NEW} \parstepsto
      \config{U_S(\extS{S}{l}{\bot}{\frozenfalse})}{l}$.

      By {\sc E-New}, we have that $\config{U_S(S)}{\NEW} \parstepsto
      \config{\extS{(U_S(S))}{l'}{\bot}{\frozenfalse}}{l'}$, where $l'
      \notin \dom{U_S(S)}$.

      By assumption, $U_S$ is non-conflicting with $\config{S}{\NEW}
      \parstepsto \config{\extS{S}{l}{\bot}{\frozenfalse}}{l}$.
      Therefore $l \notin \dom{U_S(S)}$.

      Therefore, in
      $\config{\extS{(U_S(S))}{l'}{\bot}{\frozenfalse}}{l'}$, we
      can $\alpha$-rename $l'$ to $l$.

      Therefore $\config{U_S(S)}{\NEW} \parstepsto
      \config{\extS{(U_S(S))}{l}{\bot}{\frozenfalse}}{l}$.

      Also, since $U_S$ is non-conflicting with $\config{S}{\NEW}
      \parstepsto \config{\extS{S}{l}{\bot}{\frozenfalse}}{l}$, we
      have that $(U_S(\extS{S}{l}{\bot}{\frozenfalse}))(l) =
      (\extS{S}{l}{\bot}{\frozenfalse})(l) =
      \state{\bot}{\frozenfalse}$.

      Hence $\extS{(U_S(S))}{l}{\bot}{\frozenfalse} =
      U_S(\extS{S}{l}{\bot}{\frozenfalse})$.

      Therefore $\config{U_S(S)}{\NEW} \parstepsto
      \config{U_S(\extS{S}{l}{\bot}{\frozenfalse})}{l}$, as we were
      required to show.

    \item Case {\sc E-Put}:

      Given: $\config{S}{\putiexp{l}} \parstepsto
      \config{\extSRaw{S}{l}{u_{p_i}(p_1)}}{\unit}$.

      To show: $\config{U_S(S)}{\putiexp{l}} \parstepsto
      \config{U_S(\extSRaw{S}{l}{u_{p_i}(p_1)})}{\unit}$.

      From the premises of {\sc E-Put}, $S(l) = p_1$.

      Hence $(U_S(S))(l) = p'_1$, where $p_1 \leqp p'_1$.

      Next, we want to show that $u_{p_i}(p'_1) \neq \topp$.

      Assume for the sake of a contradiction that $u_{p_i}(p'_1) =
      \topp$.

      Then $u_{p_i}((U_S(S))(l)) = \topp$.

      Let $u_{p_j}$ be the state update operation in $U_S$ that
      affects the contents of $l$.  Hence $(U_S(S))(l) =
      u_{p_j}(p_1)$. Then $u_{p_i}(u_{p_j}(p_1)) = \topp$.

      Since state update operations commute, $u_{p_j}(u_{p_i}(p_1)) =
      \topp$.
      
      But then $U_S(\extSRaw{S}{l}{u_{p_i}(p_1)}) = \topS$,
      which contradicts the assumption that
      $U_S(\extSRaw{S}{l}{u_{p_i}(p_1)}) \neq \topS$.

      Hence, $u_{p_i}(p'_1) \neq \topp$.

      Therefore, by {\sc E-Put}, $\config{U_S(S)}{\putiexp{l}}
      \parstepsto
      \config{\extSRaw{(U_S(S))}{l}{u_{p_i}(p'_1)}}{\unit}$.

      Since $p'_1 = u_{p_j}(p_1)$, we have that
      $\extSRaw{(U_S(S))}{l}{u_{p_i}(p'_1)} =
      \extSRaw{(U_S(S))}{l}{u_{p_i}(u_{p_j}(p_1))}$, which, since
      $u_{p_i}$ and $u_{p_j}$ commute, is equal to
      $\extSRaw{(U_S(S))}{l}{u_{p_j}(u_{p_i}(p_1))}$.

      Finally, since $u_{p_j}$ is the update operation in $U_S$ that
      affects the contents of $l$, we have that
      $\extSRaw{(U_S(S))}{l}{u_{p_j}(u_{p_i}(p_1))} =
      U_S(\extSRaw{S}{l}{u_{p_i}(p_1)})$, and so the case is
      satisfied.

    \item Case {\sc E-Get}:

      Given: $\config{S}{\getexp{l}{P}} \parstepsto \config{S}{p_2}$.

      To show: $\config{U_S(S)}{\getexp{l}{P}} \parstepsto
      \config{U_S(S)}{p_2}$.

      From the premises of {\sc E-Get}, $S(l) = p_1$ and $\incomp{P}$
      and $p_2 \in P$ and $p_2 \leqp p_1$.

      By assumption, $U_S(S) \neq \topS$.

      Hence $(U_S(S))(l) = p'_1$, where $p_1 \leqp p'_1$.

      By the transitivity of $\leqp$, $p_2 \leqp p'_1$.

      Hence, $(U_S(S))(l) = p'_1$ and $\incomp{P}$ and $p_2 \in P$ and
      $p_2 \leqp p'_1$.

      Therefore, by {\sc E-Get},

      $\config{U_S(S)}{\getexp{l}{P}} \parstepsto
      \config{U_S(S)}{p_2}$,

      as we were required to show.

    \item Case {\sc E-Freeze-Init}:

      Given: $\config{S}{\freezeafter{l}{Q}{\lam{x}{e}}} \parstepsto
      \config{S}{\freezeafterfull{l}{Q}{\lam{x}{e}}{\setof{}}{\setof{}}}$.

      To show: $\config{U_S(S)}{\freezeafter{l}{Q}{\lam{x}{e}}}
      \parstepsto
      \config{U_S(S)}{\freezeafterfull{l}{Q}{\lam{x}{e}}{\setof{}}{\setof{}}}$.

      Immediate by {\sc E-Freeze-Init}.

    \item Case {\sc E-Spawn-Handler}:

      Given:

      $\config{S}{\freezeafterfull{l}{Q}{\lam{x}{e_0}}{\setof{e,
            \dots}}{H}} \parstepsto
      \config{S}{\freezeafterfull{l}{Q}{\lam{x}{e_0}}{\setof{\subst{e_0}{x}{d_2},
            e, \dots}} {\{d_2\}\cup H}}$.

      To show:

      $\config{U_S(S)}{\freezeafterfull{l}{Q}{\lam{x}{e_0}}{\setof{e,
            \dots}}{H}} \parstepsto
      \config{U_S(S)}{\freezeafterfull{l}{Q}{\lam{x}{e_0}}{\setof{\subst{e_0}{x}{d_2},
            e, \dots}} {\{d_2\}\cup H}}$.

      From the premises of {\sc E-Spawn-Handler}, $S(l) =
      \state{d_1}{\status_1}$ and $d_2 \userleq d_1$ and $d_2 \notin
      H$ and $d_2 \in Q$.

      By assumption, $U_S(S) \neq \topS$.

      Hence $(U_S(S))(l) = \state{d'_1}{\status'_1}$ where
      $\state{d_1}{\status_1} \leqp \state{d'_1}{\status'_1}$.

      By Definition~\ref{def:lattice-with-status-bits}, $d_1 \userleq
      d'_1$.

      By the transitivity of $\userleq$, $d_2 \userleq d'_1$.

      Hence $(U_S(S))(l) = \state{d'_1}{\status'_1}$ and $d_2 \userleq
      d'_1$ and $d_2 \notin H$ and $d_2 \in Q$.

      Therefore, by {\sc E-Spawn-Handler},

      $\config{U_S(S)}{\freezeafterfull{l}{Q}{\lam{x}{e_0}}{\setof{e,
            \dots}}{H}} \parstepsto
      \config{U_S(S)}{\freezeafterfull{l}{Q}{\lam{x}{e_0}}{\setof{\subst{e_0}{x}{d_2},
            e, \dots}} {\{d_2\}\cup H}}$,

      as we were required to show.

    \item Case {\sc E-Freeze-Final}:

      Given:
      $\config{S}{\freezeafterfull{l}{Q}{\lam{x}{e_0}}{\setof{v,
            \dots}}{H}} \parstepsto
      \config{\extS{S}{l}{d_1}{\frozentrue}}{d_1}$.

      To show:
      $\config{U_S(S)}{\freezeafterfull{l}{Q}{\lam{x}{e_0}}{\setof{v,
            \dots}}{H}} \parstepsto
      \config{U_S(\extS{S}{l}{d_1}{\frozentrue})}{d_1}$.

      From the premises of {\sc E-Freeze-Final}, $S(l) =
      \state{d_1}{\status_1}$.

      We have two cases to consider:
      \begin{itemize}
        \item $\status_1 = \frozentrue$:

          In this case, $S(l) = \state{d_1}{\frozentrue}$.

          Let $u_{p_i}$ be the state update operation in $U_S$ that
          affects the contents of $l$.  Hence $(U_S(S))(l) =
          u_{p_i}(\state{d_1}{\frozentrue})$.

          We know from
          Definition~\ref{def:set-of-state-update-operations} that
          $u_{p_i}(\state{d_1}{\frozentrue})$ is either
          $\state{d_1}{\frozentrue}$ or $\state{\top}{\frozenfalse}$.

          But if $u_{p_i}(\state{d_1}{\frozentrue}) =
          \state{\top}{\frozenfalse}$, then
          $U_S(\extS{S}{l}{d_1}{\frozentrue}) = \topS$, which
          contradicts our assumption that
          $U_S(\extS{S}{l}{d_1}{\frozentrue}) \neq \topS$.  Hence
          $u_{p_i}(\state{d_1}{\frozentrue}) =
          \state{d_1}{\frozentrue}$.

          Hence $(U_S(S))(l) = \state{d_1}{\frozentrue}$, and we
          already have from the premises of {\sc E-Freeze-Final} that
          $\forall{d_2} ~.~ ( {d_2 \userleq d_1 \land d_2 \in Q}
          \Rightarrow d_2 \in H)$.  Hence, by {\sc E-Freeze-Final}, we
          have that
          $\config{U_S(S)}{\freezeafterfull{l}{Q}{\lam{x}{e_0}}{\setof{v,
                \dots}}{H}} \parstepsto
          \config{\extS{(U_S(S))}{l}{d_1}{\frozentrue}}{d_1}$.

          Finally, since $u_{p_i}$ is the state update operation in
          $U_S$ that affects the contents of $l$, and
          $u_{p_i}(\state{d_1}{\frozentrue}) =
          \state{d_1}{\frozentrue}$, we have that
          $\extS{(U_S(S))}{l}{d_1}{\frozentrue}$ is equal to
          $U_S(\extS{S}{l}{d_1}{\frozentrue})$, and so the case is
          satisfied.

        \item $\status_1 = \frozenfalse$:

          By assumption, $U_S$ is freeze-safe with
          $\config{S}{\freezeafterfull{l}{Q}{\lam{x}{e_0}}{\setof{v,
                \dots}}{H}} \parstepsto
          \config{\extS{S}{l}{d_1}{\frozentrue}}{d_1}$.  Therefore
          $U_S$ acts as the identity on the contents of any locations
          that change status during the transition.  Since $\status_1
          = \frozenfalse$, the contents of $l$ change status during
          the transition.  Therefore $U_S$ acts as the identity on the
          contents of $l$.

          Hence $(U_S(S))(l) = S(l) = \state{d_1}{\status_1}$, and we
          already have from the premises of {\sc E-Freeze-Final} that
          $\forall{d_2} ~.~ ( {d_2 \userleq d_1 \land d_2 \in Q}
          \Rightarrow d_2 \in H)$.  Hence, by {\sc E-Freeze-Final}, we
          have that
          $\config{U_S(S)}{\freezeafterfull{l}{Q}{\lam{x}{e_0}}{\setof{v,
                \dots}}{H}} \parstepsto
          \config{\extS{(U_S(S))}{l}{d_1}{\frozentrue}}{d_1}$.

          Finally, since $U_S$ acts as the identity on the contents of
          $l$, we have that $\extS{(U_S(S))}{l}{d_1}{\frozentrue}$ is
          equal to $U_S(\extS{S}{l}{d_1}{\frozentrue})$, and so the
          case is satisfied.
      \end{itemize}

    \item Case {\sc E-Freeze-Simple}:

      Given: $\config{S}{\freeze{l}} \parstepsto
      \config{\extS{S}{l}{d_1}{\frozentrue}}{d_1}$.

      To show: $\config{U_S(S)}{\freeze{l}} \parstepsto
      \config{U_S(\extS{S}{l}{d_1}{\frozentrue})}{d_1}$.

      From the premises of {\sc E-Freeze-Simple}, $S(l) =
      \state{d_1}{\status_1}$.

      We have two cases to consider:
      \begin{itemize}
        \item $\status_1 = \frozentrue$:

          In this case, $S(l) = \state{d_1}{\frozentrue}$.

          Let $u_{p_i}$ be the state update operation in $U_S$ that
          affects the contents of $l$.  Hence $(U_S(S))(l) =
          u_{p_i}(\state{d_1}{\frozentrue})$.

          We know from
          Definition~\ref{def:set-of-state-update-operations} that
          $u_{p_i}(\state{d_1}{\frozentrue})$ is either
          $\state{d_1}{\frozentrue}$ or $\state{\top}{\frozenfalse}$.

          But if $u_{p_i}(\state{d_1}{\frozentrue}) =
          \state{\top}{\frozenfalse}$, then
          $U_S(\extS{S}{l}{d_1}{\frozentrue}) = \topS$, which
          contradicts our assumption that
          $U_S(\extS{S}{l}{d_1}{\frozentrue}) \neq \topS$.  Hence
          $u_{p_i}(\state{d_1}{\frozentrue}) =
          \state{d_1}{\frozentrue}$.

          Hence $(U_S(S))(l) = \state{d_1}{\frozentrue}$.  Hence, by
          {\sc E-Freeze-Simple}, we have that
          $\config{U_S(S)}{\freeze{l}} \parstepsto
          \config{\extS{(U_S(S))}{l}{d_1}{\frozentrue}}{d_1}$.

          Finally, since $u_{p_i}$ is the state update operation in
          $U_S$ that affects the contents of $l$, and
          $u_{p_i}(\state{d_1}{\frozentrue}) =
          \state{d_1}{\frozentrue}$, we have that
          $\extS{(U_S(S))}{l}{d_1}{\frozentrue}$ is equal to
          $U_S(\extS{S}{l}{d_1}{\frozentrue})$, and so the case is
          satisfied.

        \item $\status_1 = \frozenfalse$:

          By assumption, $U_S$ is freeze-safe with
          $\config{S}{\freeze{l}} \parstepsto
          \config{\extS{S}{l}{d_1}{\frozentrue}}{d_1}$.  Therefore
          $U_S$ acts as the identity on the contents of any locations
          that change status during the transition.  Since $\status_1
          = \frozenfalse$, the contents of $l$ change status during
          the transition.  Therefore $U_S$ acts as the identity on the
          contents of $l$.

          Hence $(U_S(S))(l) = S(l) = \state{d_1}{\status_1}$.  Hence,
          by {\sc E-Freeze-Simple}, we have that
          $\config{U_S(S)}{\freeze{l}} \parstepsto
          \config{\extS{(U_S(S))}{l}{d_1}{\frozentrue}}{d_1}$.

          Finally, since $U_S$ acts as the identity on the contents of
          $l$, we have that $\extS{(U_S(S))}{l}{d_1}{\frozentrue}$ is
          equal to $U_S(\extS{S}{l}{d_1}{\frozentrue})$, and so the
          case is satisfied.
      \end{itemize}

  \end{itemize}
\end{proof}
