\section{Adding threshold queries to CvRDTs}\label{s:distributed-model}

In Shapiro \etal's CvRDT model, the query operation $q$ reads the
exact contents of its local replica, and therefore different replicas
may see different states at the same time, if not all updates have
been propagated yet.  That is, it is possible to observe intermediate
states of a CvRDT replica.  Such intermediate observations are not
possible with threshold queries. In this section, we show how to
extend the CvRDT model to accommodate threshold queries.

\subsection{Objects with threshold queries}

Definition~\ref{def:state-based-object-with-threshold-queries}
extends Shapiro \etal's definition of a state-based object with
a threshold query method $t$:
\begin{definition}[state-based object with threshold queries]
  \label{def:state-based-object-with-threshold-queries}
  A \emph{state-based object with threshold queries} (henceforth
  \emph{object}) is a tuple $(S, s^0, q, t, u, m)$, where $S$ is a set
  of states, $s^0 \in S$ is the initial state, $q$ is a \emph{query
    method}, $t$ is a \emph{threshold query
    method}, $u$ is an \emph{update method}, and $m$ is a \emph{merge
    method}.
\end{definition}

In order to give a semantics to the threshold query method $t$, we
need to formally define the notion of a threshold set.  The notion of
``threshold set'' that I use here is the generalized formulation of
threshold sets, based on \emph{activation sets}, that I described
previously in
Section~\ref{subsection:lvars-a-more-general-formulation-of-threshold-sets}.

\begin{definition}[threshold set]
  \label{def:threshold-set}
  A \emph{threshold set with respect to a lattice $(S, \leq)$} is a
  set $\mathcal{S} = \{ S_a, S_b, \dots \}$ of one or more sets of
  \emph{activation states}, where each set of activation states is a
  subset of $S$, the set of lattice elements, and where the following
  \emph{pairwise incompatibility} property holds:

  For all $S_a, S_b \in \mathcal{S}$, if $S_a \neq S_b$, then for all
  activation states $s_a \in S_a$ and for all activation states $s_b \in
  S_b$, $s_a \sqcup s_b = \top$, where $\sqcup$ is the join operation
  induced by $\leq$ and $\top$ is the greatest element of $(S, \leq)$.
\end{definition}

In our model, we assume a finite set of $n$ processes $p_1, \dots,
p_n$, and consider a single replicated object with one replica at each
process, with replica $i$ at process $p_i$.  Processes may crash
silently; we say that a non-crashed process is \emph{correct}.

Every replica has initial state $s^0$.  Methods execute at individual
replicas, possibly updating that replica's state.  The $k$th method
execution at replica $i$ is written $f^{k}_{i}(a)$, where $k$ is $\geq
1$ and $f$ is either $q$, $t$, $u$, or $m$, and $a$ is the arguments
to $f$, if any.  Methods execute sequentially at each replica.  The
state of replica $i$ after the $k$th method execution at $i$ is
$s^{k}_{i}$.  We say that states $s$ and $s'$ are equivalent, written
$s \equiv s'$, if $q(s) = q(s')$.

\subsection{Causal histories}

An object's \emph{causal history} is a record of all the updates that
have happened at all replicas.  The causal history does not track the
order in which updates happened, merely that they did happen.  The
\emph{causal history at replica $i$ after execution $k$} is the set of
all updates that have happened at replica $i$ after execution $k$.
Definition~\ref{def:causal-history} updates Shapiro \etal's definition
of causal history for a state-based object to account for $t$ (a
trivial change, since execution of $t$ does not change a replica's
causal history):
\begin{definition}[causal history]
  \label{def:causal-history}
  A \emph{causal history} is a sequence $[c_1, \dots, c_n]$, where
  $c_i$ is a set of the updates that have occurred at
  replica $i$.  Each $c_i$ is initially $\emptyset$.  If the $k$th
  method execution at replica $i$ is:
  \begin{itemize}
    \item a query $q$ or a threshold query $t$, then the causal
      history at replica $i$ after execution $k$ does not change:
      $c^{k}_{i} = c^{k-1}_{i}$.
    \item an update $u^{k}_{i}(a)$, then the causal history at replica
      $i$ after execution $k$ is $c^{k}_{i} = c^{k-1}_{i} \cup {
      u^{k}_{i}(a) }$.
    \item a merge $m^{k}_{i}(s^{k'}_{i'})$, then the causal history at
      replica $i$ after execution $k$ is the union of the local and
      remote histories: $c^{k}_{i} = c^{k-1}_{i} \cup c^{k'}_{i'}$.
  \end{itemize}
\end{definition}
We say that an update is \emph{delivered at replica $i$} if it is in
the causal history at replica $i$.

\subsection{Threshold CvRDTs and the semantics of blocking}\label{subsection:distributed-blocking}

With the previous definitions in place, we can give the definition of
a CvRDT that supports threshold queries:

\begin{definition}[CvRDT with threshold queries]
  \label{def:cvrdt-with-threshold-queries}
  A \emph{convergent replicated data type with threshold queries}
  (henceforth \emph{threshold CvRDT}) is an object equipped
  with a partial order $\leq$, written $(S, \leq, s^0, q, t, u, m)$,
  that has the following properties:
  \begin{itemize}
    \item $S$ forms a join-semilattice ordered by $\leq$.
    \item $S$ has a greatest element $\top$ according to $\leq$.
    \item The query method $q$ takes no arguments and returns the
      local state.
    \item The threshold query method $t$ takes a threshold set
      $\mathcal{S}$ as its argument, and has the following semantics:
      let $t^{k+1}_i(\mathcal{S})$ be the $k+1$th method execution at
      replica $i$, where $k \geq 0$.  If, for some activation state
      $s_a$ in some (unique) set of activation states $S_a \in
      \mathcal{S}$, the condition $s_a \leq s^{k}_{i}$ is met,
      $t^{k+1}_i(\mathcal{S})$ returns the set of activation states
      $S_a$.  Otherwise, $t^{k+1}_i(\mathcal{S})$ returns the
      distinguished value $\block$.
    \item The update method $u$ takes a state as argument and updates
      the local state to it.
    \item State is \emph{inflationary} across updates: if $u$ updates
      a state $s$ to $s'$, then $s \leq s'$.
    \item The merge method $m$ takes a remote state as its argument,
      computes the join of the remote state and the local state with
      respect to $\leq$, and updates the local state to the result.
  \end{itemize}
  and the $q$, $t$, $u$, and $m$ methods have no side effects other
  than those listed above.
\end{definition}
We use the $\block$ return value to model $t$'s ``blocking'' behavior
as a mathematical function with no intrinsic notion of running
duration.  When we say that a call to $t$ ``blocks'', we mean that it
immediately returns $\block$, and when we say that a call to $t$
``unblocks'', we mean that it returns a set of activation states
$S_a$.

Modeling blocking as a distinguished value introduces a new
complication: we lose determinism, because a call to $t$ at a
particular replica may return either $\block$ or a set of activation
states $S_a$, depending on the replica's state at the time it is
called.  However, we can conceal this nondeterminism with an
additional layer over the nondeterministic API exposed by $t$.  This
additional layer simply \emph{polls} $t$, calling it repeatedly until
it returns a value other than $\block$.  Calls to $t$ at a replica
that are made by this ``polling layer'' count as method executions at
that replica, and are arbitrarily interleaved with other method
executions at the replica, including updates and merges.  The polling
layer itself need not do any computation other than checking to see
whether $t$ returns $\block$ or something else; in particular, the
polling layer does not need to compare activation states to replica
states, since that comparison is done by $t$ itself.

The set of activation states $S_a$ that a call to $t$ returns when it
unblocks is unique because of the pairwise incompatibility property
required of threshold sets: without it, different orderings of updates
could allow the same threshold query to unblock in different ways,
introducing nondeterminism that would be observable beyond the polling
layer.

\subsection{Threshold CvRDTs are strongly eventually consistent}

We can define eventual consistency and strong eventual consistency
exactly as Shapiro \etal~do in their model.  In the following
definitions, a \emph{correct replica} is a replica at a correct
process, and the  symbol $\eventually$ means ``eventually'':

\begin{definition}[eventual consistency (EC)]
  \label{def:eventual-consistency}
  An object is \emph{eventually consistent} (EC) if the following three
  conditions hold:
  \begin{itemize}
    \item \emph{Eventual delivery}: An update delivered at some
      correct replica is eventually delivered at all correct replicas:
      $\forall i, j : f \in c_i \implies \eventually f \in c_j$.
    \item \emph{Convergence}: Correct replicas at which the same
      updates have been delivered eventually have equivalent state:
      $\forall i, j : c_i = c_j \implies \eventually s_i \equiv s_j$.
    \item \emph{Termination}: All method executions halt.
  \end{itemize}
\end{definition}

\begin{definition}[strong eventual consistency (SEC)]
  \label{def:strong-eventual-consistency}
  An object is \emph{strongly eventually consistent} (SEC) if it is
  eventually consistent and the following condition holds:
  \begin{itemize}
  \item \emph{Strong convergence}: Correct replicas at which the same
    updates have been delivered have equivalent state:
    $\forall i, j : c_i = c_j \implies s_i \equiv s_j$.
  \end{itemize}
\end{definition}
Since we model blocking threshold queries with $\block$, we need not
be concerned with threshold queries not necessarily terminating.
Determinism does \emph{not} rule out queries that return $\block$
every time they are called (and would therefore cause the polling
layer to block forever).  However, we guarantee that if a threshold
query returns $\block$ every time it is called during a complete run
of the system, it will do so on \emph{every} run of the system,
regardless of scheduling.  That is, it is not possible for a query to
cause the polling layer to block forever on some runs, but not on
others.

Finally, we can directly leverage Shapiro \etal's SEC result for
CvRDTs to show that a threshold CvRDT is SEC:
\begin{thm}[Strong Eventual Consistency of Threshold CvRDTs]
  \label{thm:strong-eventual-consistency-of-threshold-cvrdts}
  Assuming eventual delivery and termination, an object that meets
  the criteria for a threshold CvRDT is SEC.
\end{thm}
\begin{proof}
From Shapiro \etal, we have that an object that meets the criteria for
a CvRDT is SEC~\cite{crdts}.  Shapiro \etal's proof also assumes that
eventual delivery and termination hold for the object, and proves that
strong convergence holds --- that is, that given causal histories
$c_i$ and $c_j$ for respective replicas $i$ and $j$, that their states
$s_i$ and $s_j$ are equivalent.  The proof relies on the commutativity
of the least upper bound operation.  Since, according to our
Definition~\ref{def:causal-history}, threshold queries do not affect
causal history, we can leverage Shapiro \etal's result to say that a
threshold CvRDT is also SEC.
\end{proof}
