\documentclass{article}
\usepackage{amsmath,amssymb,amsthm}
\usepackage{color}
\usepackage{hyperref}
\usepackage{stmaryrd}
\usepackage{float}
\usepackage{listings}
\usepackage{graphicx}
\usepackage{fullpage}
\usepackage{times}

%%% Macros for typesetting terms of the LVish language itself.

% Sans-serif font for terms of the language
\newcommand\termfont[1]{\mbox{\texttt{#1}}}

% Amount to indent the next line of a 'let' expression.
\newcommand{\letspace}{\hspace{1.1em}}
\newcommand{\letparspace}{\hspace{3.7em}}

%%% Language forms.

\newcommand{\lam}[2]{\ensuremath{\lambda#1.\,#2}}
\newcommand{\app}[2]{\ensuremath{#1~#2}}

\newcommand{\unit}{\termfont{()}}

\newcommand{\NEW}{\termfont{new}}

\newcommand{\PUT}{\termfont{put}}
\newcommand{\putexp}[2]{\ensuremath{\PUT~#1~#2}}

\newcommand{\GET}{\termfont{get}}
\newcommand{\getexp}[2]{\ensuremath{\GET~#1~#2}}

\newcommand{\BUMP}{\termfont{bump}}
\newcommand{\bumpexp}[1]{\ensuremath{\BUMP~#1}}

\newcommand{\GETFSTORSND}{\termfont{getFstOrSnd}}
\newcommand{\GETFST}{\termfont{getFst}}
\newcommand{\getfstexp}[1]{\ensuremath{\GETFST~#1}}
\newcommand{\GETSND}{\termfont{getSnd}}
\newcommand{\getsndexp}[1]{\ensuremath{\GETSND~#1}}

\newcommand{\FREEZE}{\termfont{freeze}}
\newcommand{\AFTER}{~{\termfont{after}}}
\newcommand{\WITH}{~{\termfont{with}}}

\newcommand{\ADDHANDLER}{{\termfont{addHandler}}}
\newcommand{\NEWPOOL}{{\termfont{newPool}}}
\newcommand{\ADDHANDLERINPOOL}{{\termfont{addInPool}}}
\newcommand{\QUIESCE}{{\termfont{quiesce}}}

\newcommand{\FAW}{{\termfont{freeze}-\termfont{after}-\termfont{with}}}
\newcommand{\freeze}[1]{\ensuremath{\FREEZE~#1}}
\newcommand{\freezeafter}[3]{\ensuremath{\FREEZE~#1~\AFTER~#2~\WITH~#3}}
\newcommand{\freezeafterfull}[5]{\ensuremath{\FREEZE~#1~\AFTER~#2~\WITH~#3, #4, #5}}

\newcommand{\LET}{\termfont{let}}
\newcommand{\PAR}{\termfont{par}}
\newcommand{\LETPAR}{\termfont{\LET~\PAR}}
\newcommand{\IN}{\termfont{in}}
\newcommand{\letexp}[3]{\ensuremath{\LET~#1=#2~\IN~#3}}
\newcommand{\letparexp}[5]{\ensuremath{\LETPAR~#1=#2;~#3=#4~\IN~#5}}

\newcommand{\stateset}[1]{\ensuremath{\lbrace #1 \rbrace}}

\newcommand{\UNIQUE}{\termfont{unique}}

\newcommand{\lambdalvar}{\ensuremath{\lambda_{\textsf{LVar}}}}



% Editing marks
%%% Macros for editing marks.

\newcommand{\TBD}[0]{{\color{red}TBD}}
\newcommand{\TODO}[1]{\noindent{\color{red} TODO: #1}\\}

\newcommand{\lk}[1]{\pgwrapper{LK}{#1}}
\newcommand{\rn}[1]{\pgwrapper{RRN}{#1}}
\newcommand{\rnote}[1]{\pgwrapper{RRN}{#1}}
\newcommand{\ajt}[1]{\pgwrapper{AJT}{#1}}

\InputIfFileExists{editingmarks}{}{
    \def\noeditingmarks{}
}

\definecolor{comment-red}{rgb}{0.8,0,0}
\ifx\noeditingmarks\undefined
   \newcommand{\textred}[1]{\textcolor{comment-red}{#1}}
   \newcommand{\pgwrapper}[2]{\textred{#1: #2}}
   \newcommand{\new}[1]{\textcolor{blue}{#1}}
\else
   \newcommand{\textred}[1]{\textcolor{comment-red}{#1}}
   \newcommand{\pgwrapper}[2]{}
   \newcommand{\new}[1]{#1}
\fi


% Other presentational stuff
%% Assorted style and presentation stuff.

\newcommand{\figsepr}{\vspace{0.5em}\hrulefill\vspace{0.5em}}
\newcommand{\etal}{\textit{et al}.}
\newcommand{\ie}{\textit{i}.\textit{e}.}
\newcommand{\eg}{\textit{e}.\textit{g}.}
\newcommand{\il}[1]{\lstinline{#1}}

\newenvironment{blockquote}{%
  \par%
  \medskip
  \leftskip=4em\rightskip=2em%
  \noindent\ignorespaces}{%
  \par\medskip}

%% LK: See the definition of `\theequation`.
\newcommand{\eref}[1]{\ref{#1}}

%====Set up Listings===============================================================

\definecolor{darkgreen}{rgb}{0,0.5,0}
\definecolor{darkred}{rgb}{0.5,0,0}
\lstloadlanguages{Haskell}
\lstnewenvironment{code}
    { % \centering
      \lstset{}%
      \csname lst@SetFirstLabel\endcsname}
    { %\centering
      \csname lst@SaveFirstLabel\endcsname}
    \lstset{
      language=Haskell,
%      basicstyle=\footnotesize\ttfamily,
      basicstyle=\small\ttfamily,
      flexiblecolumns=false,
      basewidth={0.5em,0.45em},
      aboveskip={3pt},
      belowskip={3pt},
      keywordstyle=\color{black},  
      commentstyle=\color{darkgreen},
      literate={+}{{$+$}}1 {/}{{$/$}}1 {*}{{$*$}}1 % {=}{{$=$}}1
               {>}{{$>$}}1 {<}{{$<$}}1 {\\}{{$\lambda$}}1
               {\\\\}{{\char`\\\char`\\}}1
               {->}{{$\rightarrow$}}2 {>=}{{$\geq$}}2 {<-}{{$\leftarrow$}}2
               {<=}{{$\leq$}}2 {=>}{{$\Rightarrow$}}2 
               {\ .}{{$\circ$}}2 {\ .\ }{{$\circ$}}2
               {>>}{{>>}}2 {>>=}{{>>=}}2 {=<<}{{=<<}}2
               {|}{{$\mid$}}1
               {`member`}{{$\in$}}1
               {leftBrace}{\{}1
               {rightBrace}{\}}1
               {\$singleton\$startV}{{ \hspace{2.4em} \{{\tt startV}\}}}1
               {\$singleton\$n}{{  \{{\tt n}\}}}1
               {dotdotdot}{{$\ldots$}}3
    }

%================================================================================

% Penalty for line-breaking inline math
\relpenalty=9999
\binoppenalty=9999


\begin{document}

\title{Research Statement}

\author{Lindsey Kuper}

\date{\today}

\maketitle

\lk{Some of this wording is borrowed from Aaron Turon's research statement.}

My research lies broadly in the area of programming languages, with
the goal of creating tools, techniques, and abstractions that support
\emph{compositional reasoning} about programs, proofs, and processes,
especially those that do not immediately appear to compose well.  For
the last few years, my focus has been on parallel and distributed
programming models.  Below I discuss my work in more detail, beginning
in Sections~\ref{lvars} and \ref{quasi} with my work on \emph{LVars},
lattice-based data structures for deterministic and
\emph{quasi-deterministic} parallelism.  Then in Section~\ref{crdts}, I discuss
the relationship between LVars and \emph{conflict-free replicated data
  types}.  For brevity, I do not discuss my previous work on
\emph{relational programming}~\cite{lambdae}, \emph{multi-language
  parametricity}~\cite{multilang-talk}, or \emph{verifying CPU
  semantics}~\cite{tsl-tr}.

\section{LVars: lattice-based data structures for deterministic parallelism}\label{lvars}

\lk{Something like this?
A fundamental challenge of parallel programming is that programs can
behave in unexpected ways as a result of unpredictable interactions
between parallel tasks.  So-called \emph{irregular} parallel
applications---in which the work an algorithm must do and the
potential for parallelizing it depend on the particular input being
processed---exacerbate this challenge: since work has to be
dynamically scheduled, unpredictable inter-task interactions are
inevitable.}

\emph{Deterministic-by-construction} parallel programming models
guarantee that programs written using them will have the same
observable behavior on every run, offering the promise of freedom from
subtle, hard-to-reproduce bugs caused by schedule nondeterminism.  In
order to guarantee determinism, though, deterministic-by-construction
models must sharply restrict the sharing of state between parallel
tasks.  Shared state, if not disallowed entirely~\cite{dph,
  dpj-oopsla}, is restricted to a single type of shared data
structure, such as single-assignment locations, sometimes known as
\emph{IVars}~\cite{IStructures, CnC}, or blocking FIFO queues, as in
Kahn process networks~\cite{Kahn-1974}.  These approaches limit the
kinds of deterministic algorithms that can be expressed---efficiently
or at all---within the model.

\lk{TODO: say something about irregular parallel applications, where
  the amount of available parallelism depends on the data being
  processed.}

My work introduced \emph{lattice-based data structures}, or
\emph{LVars}~\cite{LVars-paper, LVars-TR}, which generalize the above
approaches to allow multiple assignments that are monotonically
increasing with respect to an application-specific \emph{lattice}.
LVars ensure determinism by allowing only monotonically increasing
writes and ``threshold'' reads that block until a lower bound in the
lattice is reached.  Together, monotonic writes and threshold reads
yield a guaranteed-deterministic programming model: programs will have
the same observable result in spite of parallel execution and schedule
nondeterminism.  Moreover, our determinism result is
\emph{lattice-generic}: threads can communicate through any shared
data structure, so long as the states the data structure can take on
can be viewed as elements of a lattice and updates are monotonically
increasing with respect to that lattice.  This genericity allows a
programming model based on LVars to subsume existing deterministic
parallel programming models: for example, a lattice of channel
histories with a prefix ordering allows LVars to represent FIFO
channels that implement a Kahn process network, whereas instantiating
the model with a lattice with ``empty'' and ``full'' states (where
$\mathit{empty} < \mathit{full}$) results in a parallel
single-assignment language.  Different instantiations of the LVars
model result in a wide-ranging family of deterministic parallel
languages.

We put LVars into practice in \emph{LVish}~\cite{LVish}, a Haskell
library for guaranteed-deterministic parallel programming with LVars.
The LVish library provides a monad for encapsulating parallel
computation and enables a notion of lightweight, library-level threads
to be employed with a custom work-stealing scheduler.  LVish also
provides a variety of lattice-based data structures (\eg, sets, maps,
graphs) that support concurrent insertion, but not deletion, during
monadic computations.  Users may implement their own lattice-based
data structures as well, and LVish provides tools to facilitate the
definition of such user-defined LVars.

\section{Quasi-deterministic programming with LVars and LVish}\label{quasi}

Although it guarantees determinism, the LVars interface is quite
limited.  My work in POPL '14~\cite{Freeze-paper, Freeze-TR} extends
LVars in two ways.  First, we add the ability to ``freeze'' and then
read the contents of an LVar directly.  Second, we add the ability to
attach event handlers to an LVar, triggering a callback when the
LVar's value changes.  Together, handlers and freezing enable an
expressive and useful style of parallel programming.  We prove that in
a language where communication takes place through these extended
LVars, programs are at worst \emph{quasi-deterministic}: on every run,
they either produce the same answer or raise an error.  We demonstrate
the viability of our approach by extending the LVish library with
support for handlers and freezing.

\section{Joining forces: LVars and conflict-free replicated data types}\label{crdts}

\bibliographystyle{plain}
\newcommand{\myname}[0]{\textbf{Lindsey Kuper}}
\bibliography{../latex_common/refs}

\end{document}
