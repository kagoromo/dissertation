\documentclass{article}

\usepackage{hyperref}
\usepackage{geometry}
\usepackage{amssymb}
\usepackage{xcolor}
\usepackage{listings}

\usepackage{xltxtra,fontspec,xunicode}
\defaultfontfeatures{Scale=MatchLowercase}
\setromanfont[Mapping=tex-text]{PT Serif}
\setsansfont[Mapping=tex-text]{PT Serif}

% Set your name here
\def\name{Lindsey Kuper}

% The following metadata will show up in the PDF properties
\hypersetup{
  colorlinks = true,
  urlcolor = cyan,
  pdfauthor = {\name},
  pdftitle = {\name: Research Statement},
  pdfsubject = {Research Statement},
  pdfpagemode = UseNone
}
\urlstyle{same}

\geometry{
  body={6.6in, 9.0in},
  left=0.95in,
  top=1.0in
}

% Customize page headers
\pagestyle{myheadings}
\markright{\name: Research Statement} 
\thispagestyle{empty}

% Custom section fonts
\usepackage{sectsty}
\sectionfont{\mdseries\Large}
\subsectionfont{\mdseries\itshape\large}

%%% Macros for typesetting terms of the LVish language itself.

% Sans-serif font for terms of the language
\newcommand\termfont[1]{\mbox{\texttt{#1}}}

% Amount to indent the next line of a 'let' expression.
\newcommand{\letspace}{\hspace{1.1em}}
\newcommand{\letparspace}{\hspace{3.7em}}

%%% Language forms.

\newcommand{\lam}[2]{\ensuremath{\lambda#1.\,#2}}
\newcommand{\app}[2]{\ensuremath{#1~#2}}

\newcommand{\unit}{\termfont{()}}

\newcommand{\NEW}{\termfont{new}}

\newcommand{\PUT}{\termfont{put}}
\newcommand{\putexp}[2]{\ensuremath{\PUT~#1~#2}}

\newcommand{\GET}{\termfont{get}}
\newcommand{\getexp}[2]{\ensuremath{\GET~#1~#2}}

\newcommand{\BUMP}{\termfont{bump}}
\newcommand{\bumpexp}[1]{\ensuremath{\BUMP~#1}}

\newcommand{\GETFSTORSND}{\termfont{getFstOrSnd}}
\newcommand{\GETFST}{\termfont{getFst}}
\newcommand{\getfstexp}[1]{\ensuremath{\GETFST~#1}}
\newcommand{\GETSND}{\termfont{getSnd}}
\newcommand{\getsndexp}[1]{\ensuremath{\GETSND~#1}}

\newcommand{\FREEZE}{\termfont{freeze}}
\newcommand{\AFTER}{~{\termfont{after}}}
\newcommand{\WITH}{~{\termfont{with}}}

\newcommand{\ADDHANDLER}{{\termfont{addHandler}}}
\newcommand{\NEWPOOL}{{\termfont{newPool}}}
\newcommand{\ADDHANDLERINPOOL}{{\termfont{addInPool}}}
\newcommand{\QUIESCE}{{\termfont{quiesce}}}

\newcommand{\FAW}{{\termfont{freeze}-\termfont{after}-\termfont{with}}}
\newcommand{\freeze}[1]{\ensuremath{\FREEZE~#1}}
\newcommand{\freezeafter}[3]{\ensuremath{\FREEZE~#1~\AFTER~#2~\WITH~#3}}
\newcommand{\freezeafterfull}[5]{\ensuremath{\FREEZE~#1~\AFTER~#2~\WITH~#3, #4, #5}}

\newcommand{\LET}{\termfont{let}}
\newcommand{\PAR}{\termfont{par}}
\newcommand{\LETPAR}{\termfont{\LET~\PAR}}
\newcommand{\IN}{\termfont{in}}
\newcommand{\letexp}[3]{\ensuremath{\LET~#1=#2~\IN~#3}}
\newcommand{\letparexp}[5]{\ensuremath{\LETPAR~#1=#2;~#3=#4~\IN~#5}}

\newcommand{\stateset}[1]{\ensuremath{\lbrace #1 \rbrace}}

\newcommand{\UNIQUE}{\termfont{unique}}

\newcommand{\lambdalvar}{\ensuremath{\lambda_{\textsf{LVar}}}}



% Editing marks
%%% Macros for editing marks.

\newcommand{\TBD}[0]{{\color{red}TBD}}
\newcommand{\TODO}[1]{\noindent{\color{red} TODO: #1}\\}

\newcommand{\lk}[1]{\pgwrapper{LK}{#1}}
\newcommand{\rn}[1]{\pgwrapper{RRN}{#1}}
\newcommand{\rnote}[1]{\pgwrapper{RRN}{#1}}
\newcommand{\ajt}[1]{\pgwrapper{AJT}{#1}}

\InputIfFileExists{editingmarks}{}{
    \def\noeditingmarks{}
}

\definecolor{comment-red}{rgb}{0.8,0,0}
\ifx\noeditingmarks\undefined
   \newcommand{\textred}[1]{\textcolor{comment-red}{#1}}
   \newcommand{\pgwrapper}[2]{\textred{#1: #2}}
   \newcommand{\new}[1]{\textcolor{blue}{#1}}
\else
   \newcommand{\textred}[1]{\textcolor{comment-red}{#1}}
   \newcommand{\pgwrapper}[2]{}
   \newcommand{\new}[1]{#1}
\fi


% Other presentational stuff
%% Assorted style and presentation stuff.

\newcommand{\figsepr}{\vspace{0.5em}\hrulefill\vspace{0.5em}}
\newcommand{\etal}{\textit{et al}.}
\newcommand{\ie}{\textit{i}.\textit{e}.}
\newcommand{\eg}{\textit{e}.\textit{g}.}
\newcommand{\il}[1]{\lstinline{#1}}

\newenvironment{blockquote}{%
  \par%
  \medskip
  \leftskip=4em\rightskip=2em%
  \noindent\ignorespaces}{%
  \par\medskip}

%% LK: See the definition of `\theequation`.
\newcommand{\eref}[1]{\ref{#1}}

%====Set up Listings===============================================================

\definecolor{darkgreen}{rgb}{0,0.5,0}
\definecolor{darkred}{rgb}{0.5,0,0}
\lstloadlanguages{Haskell}
\lstnewenvironment{code}
    { % \centering
      \lstset{}%
      \csname lst@SetFirstLabel\endcsname}
    { %\centering
      \csname lst@SaveFirstLabel\endcsname}
    \lstset{
      language=Haskell,
%      basicstyle=\footnotesize\ttfamily,
      basicstyle=\small\ttfamily,
      flexiblecolumns=false,
      basewidth={0.5em,0.45em},
      aboveskip={3pt},
      belowskip={3pt},
      keywordstyle=\color{black},  
      commentstyle=\color{darkgreen},
      literate={+}{{$+$}}1 {/}{{$/$}}1 {*}{{$*$}}1 % {=}{{$=$}}1
               {>}{{$>$}}1 {<}{{$<$}}1 {\\}{{$\lambda$}}1
               {\\\\}{{\char`\\\char`\\}}1
               {->}{{$\rightarrow$}}2 {>=}{{$\geq$}}2 {<-}{{$\leftarrow$}}2
               {<=}{{$\leq$}}2 {=>}{{$\Rightarrow$}}2 
               {\ .}{{$\circ$}}2 {\ .\ }{{$\circ$}}2
               {>>}{{>>}}2 {>>=}{{>>=}}2 {=<<}{{=<<}}2
               {|}{{$\mid$}}1
               {`member`}{{$\in$}}1
               {leftBrace}{\{}1
               {rightBrace}{\}}1
               {\$singleton\$startV}{{ \hspace{2.4em} \{{\tt startV}\}}}1
               {\$singleton\$n}{{  \{{\tt n}\}}}1
               {dotdotdot}{{$\ldots$}}3
    }

%================================================================================

% Penalty for line-breaking inline math
\relpenalty=9999
\binoppenalty=9999


%  Use the @ symbol for simple inline code within prose:
\lstMakeShortInline[]@

\begin{document}

\title{Research Statement}

\author{\name}

\date{\today}

\maketitle

\noindent 
The goal of my research is to identify and create tools, techniques,
and abstractions that support \emph{compositional reasoning} about
programs, proofs, and processes, especially those that do not
immediately appear to compose well.  To that end, I study programming
languages.

For the last few years, I have focused on concurrent and parallel
programming models.  Parallel tasks updating shared state present a
challenge for compositional reasoning. Programs can behave in
unexpected ways, or even crash, as a result of unpredictable
interactions between parallel tasks.  The increasing ubiquity of
\emph{irregular} parallel applications---in which the work an
algorithm must do, and the potential for parallelizing it, depend on
the particular input being processed---exacerbate this challenge:
since the work to be done must be dynamically scheduled, unpredictable
inter-task interactions at runtime are inevitable.
\emph{Deterministic-by-construction} parallel programming models,
though, are guaranteed to have the same observable behavior on every
execution, despite such unpredictable runtime interactions.

In this statement, I discuss my contributions to
deterministic-by-construction parallel programming, starting with my
work on \emph{lattice-based data structures}, or
\emph{LVars}~(Section~\ref{lvars}), and on combining LVars with other
parallel effects in a broadly applicable, guaranteed-deterministic
programming model~(Section~\ref{effectzoo}).  Then, in
Section~\ref{crdts}, I discuss my ongoing work on the relationship
between LVars and \emph{conflict-free replicated data types} for
eventually consistent distributed systems.  For brevity, I omit my
previous work on \emph{relational programming}~\cite{lambdae},
\emph{multi-language parametricity}~\cite{multilang-talk}, and
\emph{testing CPU semantic specifications}~\cite{tsl-tr}.

\section{LVars: lattice-based data structures for deterministic parallelism}\label{lvars}

\emph{Deterministic-by-construction} parallel programming models
guarantee that programs written using them will have the same
observable behavior on every run, offering the promise of freedom from
subtle, hard-to-reproduce bugs caused by schedule nondeterminism.  In
order to guarantee determinism, though, deterministic-by-construction
models must sharply restrict the sharing of state between parallel
tasks.  Shared state, when not disallowed entirely, is restricted to a
single type of shared data structure, such as single-assignment
locations, known as \emph{IVars}~\cite{IStructures, CnC}, or blocking
FIFO queues, as in Kahn process networks~\cite{Kahn-1974}.  These
approaches limit the kinds of deterministic algorithms that can be
expressed---efficiently or at all---within the model.

My work in POPL '14~\cite{Freeze-paper, Freeze-TR} and FHPC
'13~~\cite{LVars-paper, LVars-TR} introduces \emph{lattice-based data
  structures}, or \emph{LVars}, which generalize the above approaches
to allow multiple assignments to shared locations.  Writes to an LVar
update its value to the \emph{least upper bound} of the old and new
values, ensuring that its state is monotonically increasing with
respect to an application-specific \emph{lattice}.  Along with
monotonically increasing writes, LVars ensure determinism by allowing
only ``threshold'' reads that block until a lower bound (\ie, an
element belonging to a restricted subset of the lattice) is reached.
Together, monotonic writes and threshold reads yield a provably
deterministic parallel programming model.

Our determinism result is \emph{lattice-generic}: threads can
communicate through \emph{any} shared data structure, so long as the
states the data structure can take on can be viewed as elements of a
lattice and updates are monotonically increasing.  This generality
allows a programming model based on LVars to subsume existing
deterministic parallel programming models.  For example, a lattice of
channel histories with a prefix ordering allows LVars to represent
FIFO channels that implement a Kahn process network, whereas
instantiating the model with a lattice with one ``empty'' state and
multiple ``full'' states (where $\forall{i}.\; \mathit{empty} <
\mathit{full_i}$) results in a parallel single-assignment language.
Different instantiations of the LVars model result in a wide-ranging
family of deterministic parallel languages.

In order to handle irregular parallel programs, such as graph
algorithms, that traditionally present a challenge for
guaranteed-deterministic parallel programming, LVars support a
\emph{freeze} operation that allows its contents to be read
immediately and exactly without blocking. Freezing imposes the
trade-off that, once an LVar has been frozen, any further writes that
would change its value instead raise an exception.  Moreover, it is
possible to support \emph{event handlers} to an LVar, which trigger
callbacks to run asynchronously whenever events arrive (in the form of
monotonic updates to the LVar).  Crucially, it is possible to check
for \emph{quiescence} of a group of handlers, which can be used to
tell that a fixpoint has been reached.  Together, handlers,
quiescence, and freezing enable an expressive and useful style of
parallel programming.  We prove that in the presence of handlers,
quiescence, and freezing, programs are at worst
\emph{quasi-deterministic}: on every run, they either produce the same
answer or raise an error.  The error case can only be caused by a
write-after-freeze exception, and error messages can pinpoint the
exact racing pair of write and freeze operations so that the
programmer can easily correct the bug.

\section{LVish: deterministic programming with a suite of parallel effects}\label{effectzoo}

Language-level enforcement of determinism \emph{is} possible, but
existing deterministic-by-construction parallel programming models
tend to lack features that would make them applicable to a broad range
of real-world problems. Moreover, they lack \emph{extensibility}: it
is difficult to add or change language features without breaking the
determinism guarantee.

In my work in PLDI '14 (\cite{effectzoo}, to appear), we address these
applicability and extensibility challenges by putting LVars into
practice in \emph{LVish}~\cite{LVish}, a Haskell library for
guaranteed-deterministic parallel programming.  The LVish library
provides the \emph{\lstinline|Par| monad}, a mechanism for
encapsulating parallel computation, and enables a notion of
lightweight, library-level threads to be employed with a custom
work-stealing scheduler.  LVish provides a variety of lattice-based
data structures (\eg, sets, maps, graphs) that support concurrent
insertion, but not deletion, during @Par@ computations.  Users may
implement their own lattice-based data structures as well, and LVish
provides tools to facilitate the definition of such user-defined
LVars.

While applying the LVars model to real problems, we have identified
and implemented three capabilities that extend its reach:
monotonically increasing updates other than least-upper-bound writes;
transitive task cancellation; and parallel mutation of non-overlapping
memory locations. The unifying abstraction we use in LVish to add
these capabilities---without suffering added complexity or cost in the
core LVish implementation, or compromising determinism---is a form of
\emph{monad transformer}, extended to handle the @Par@ monad.  With
these extensions, LVish provides, to our knowledge, the most broadly
applicable guaranteed-deterministic parallel programming interface
available to date. We demonstrate the viability of our approach both
with traditional parallel benchmarks and with results from a
real-world case study: a bioinformatics application that we
parallelized using LVish~\cite{phybin}.

An important aspect of programming with LVish is the ability to
annotate a @Par@ computation with an \emph{effect level}, allowing
fine-grained specification of the effects that a given computation is
allowed to perform.  For instance, since freezing introduces
quasi-determinism, a computation annotated with a deterministic effect
level is not allowed to perform a freeze.  Thus, the \emph{static
  type} of an LVish computation reflects its determinism or
quasi-determinism guarantee.  Furthermore, if a freeze is guaranteed
to be the \emph{last} effect that occurs in an LVar computation, then
it is impossible for that freeze to race with a write, ruling out the
possibility of a run-time write-after-freeze exception.  LVish exposes
a way to run @Par@ computations that captures this ``freeze-last''
idiom and has a deterministic effect level.

\section{Joining forces: LVars and convergent replicated data types}\label{crdts}

Replication of data across physical locations is of fundamental
importance in distributed systems: it makes systems more robust to
data loss and allows for good data locality.  However, it presents the
dilemma of how to resolve conflicts between replicas that differ,
particularly in the case of \emph{highly available} distributed
systems~\cite{dynamo} that must give up on strict consistency in favor
of \emph{eventual consistency}~\cite{vogels-ec}.  Fortunately,
\emph{conflict-free replicated data types} (CRDTs)~\cite{crdts}
provide a simple mathematical framework for reasoning about and
enforcing the eventual consistency of replicated objects in a
distributed system.  In particular, \emph{state-based} or
\emph{convergent} replicated data types, abbreviated as \emph{CvRDTs},
leverage the mathematical properties of lattices to guarantee that all
replicas of an object (for instance, in a distributed database)
eventually agree.

CvRDTs use lattice properties to ensure eventual consistency across
replicas in a way that is closely analogous to how LVars use lattice
properties to ensure determinism across parallel executions.
Therefore a reasonable next research question is: how can we take
inspiration from CvRDTs to improve the LVars model, and vice versa?
In my current work (\cite{joining-wodet}, to appear in WoDet '14), I
am approaching this question in both directions.  In one direction, I
consider how LVar-style threshold reads apply to the setting of
CvRDTs.  Since threshold reads guarantee that the order in which
updates occur cannot be observed, they can prevent queries from
returning inconsistent intermediate states, ensuring a greater degree
of consistency in CvRDTs.  When \emph{all} reads are threshold reads,
we should be able to guarantee strong consistency---and only at the
(potential) price of read availability, never write
availability. Moreover, in practice, a combination of threshold reads
and regular reads should enable a notion of dynamically configurable,
per-query consistency.  In the other direction, I consider how to
formally extend the LVars model with CvRDT-style general inflationary
updates beyond least-upper-bound writes.  Finally, in ongoing work, I
am using techniques from the CRDT literature \cite{crdts-tr} to
develop LVars that \emph{emulate} non-monotonic updates, such as sets
that support removal of elements and counters that can be decremented
as well as incremented.

\bibliographystyle{plain}
\newcommand{\myname}[0]{\textbf{Lindsey Kuper}}
\bibliography{../latex_common/refs}

\end{document}
