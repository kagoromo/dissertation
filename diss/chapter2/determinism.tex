\section{Proof of determinism for $\lambdaLVar$}\label{s:lvars-proof}

The main technical result of this chapter is a proof of determinism
for the $\lambdaLVar$ language.  \TODO{Say something about how later
  I'll show determinism for generalized monotonic writes and
  generalized threshold reads, if in fact I'm going to do that!}

\TODO{The entire determinism proof, and hence the rest of this
  section, needs to be redone.}

\subsection{Discussion: termination}

I have followed Budimli\'c \etal~\cite{CnC} in treating
\emph{determinism} separately from the issue of \emph{termination}.
Yet one might legitimately be concerned that in $\lambdaLVar$, a
configuration could have both an infinite reduction path and one that
terminates with a value.  Theorem~\ref{thm:determinism} says that if
two runs of a given $\lambdaLVar$ program reach configurations where
no more reductions are possible (except by reflexive rules), then they
have reached the same configuration.  Hence
Theorem~\ref{thm:determinism} handles the case of \emph{deadlocks}
already: a $\lambdaLVar$ program can deadlock (\eg, with a blocked
$\GET$), but it will do so deterministically.

However, Theorem~\ref{thm:determinism} has nothing to say about
\emph{livelocks}, in which a program reduces infinitely.  It would be
desirable to have a {\em consistent termination} property which would
guarantee that if one run of a given $\lambdaLVar$ program terminates
with a non-$\error$ result, then every run will.  I conjecture (but do
not prove) that such a consistent termination property holds for
$\lambdaLVar$.  Such a property could be paired with
Theorem~\ref{thm:determinism} to guarantee that if one run of a given
$\lambdaLVar$ program terminates in a non-$\error$ configuration
$\sigma$, then every run of that program terminates in $\sigma$.  (The
``non-$\error$ configuration'' condition is necessary because it is
possible to construct a $\lambdaLVar$ program that can terminate in
$\error$ on some runs and diverge on others.  By contrast, the
existing determinism theorem does not have to treat $\error$
specially.)
