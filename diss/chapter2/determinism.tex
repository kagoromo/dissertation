\section{Proof of determinism for
  $\lambdaLVar$}\label{s:lvars-proof}

The main technical result of this chapter is a proof of determinism
for the $\lambdaLVar$ language.  The determinism theorem says that if
two executions starting from a given configuration $\conf$ terminate
in configurations $\conf'$ and $\conf''$, then $\conf'$ and $\conf''$
are the same configuration, up to a permutation on locations.

In order to prove determinism for $\lambdaLVar$, I first prove several
supporting lemmas.  The structure of the proof is similar to that of
the proof of determinism for Featherweight CnC given by Budimli\'c
\etal~\cite{CnC}.  I reuse the naming conventions of Budimli\'c
\etal~for Lemmas~\ref{lem:lvars-monotonicity} through
\ref{lem:lvars-diamond}, although the statements and proofs of those
lemmas differ considerably in the setting of $\lambdaLVar$, due to the
generality of LVars.

However, Lemmas~\ref{lem:lvars-strong-one-sided-confluence},
\ref{lem:lvars-strong-confluence}, and \ref{lem:lvars-confluence} are
nearly identical to the corresponding lemmas in the Featherweight CnC
determinism proof.  This is the case because, once
Lemmas~\ref{lem:lvars-monotonicity} through \ref{lem:lvars-diamond}
are established, the remainder of the determinism proof does not need
to deal specifically with the semantics of LVars. \TODO{Something like
  this is true, but audit this paragraph to make sure after the proof
  is done.}

\TODO{Say something about how later I'll show determinism for
  generalized monotonic writes and generalized threshold reads, if in
  fact I'm going to do that.}

\subsection{Monotonicity lemma}

The Monotonicity lemma says that, as evaluation proceeds according to
the $\parstepsto$ relation, the store can only grow with respect to
the $\leqstore{}{}$ ordering.

\LVarsLemMonotonicity
\begin{proof}
  Suppose $\config{S}{e} \parstepsto \config{S'}{e'}$.  We are
  required to show that $\leqstore{S}{S'}$.  The proof is by cases on
  the rule by which $\config{S}{e}$ steps to $\config{S'}{e'}$.

  \begin{itemize}
    \item Case {\sc E-Beta}:

      Immediate by the definition of $\leqstore{}{}$, since $S$ does
      not change.

    \item Case {\sc E-New}:

      Given: $\config{S}{\NEW} \parstepsto
      \config{\extSRaw{S}{l}{\bot}}{l}$.

      To show: $\leqstore{S}{\extSRaw{S}{l}{\bot}}$.

      By Definition~\ref{def:lvars-leqstore}, we have to show that
      $\dom{S} \subseteq \dom{\extSRaw{S}{l}{\bot}}$ and
      that for all $l' \in \dom{S}$, $S(l') \userleq
      (\extSRaw{S}{l}{\bot})(l')$.

      By definition, a store update operation on $S$ can only either
      update an existing binding in $S$ or extend $S$ with a new
      binding.  Hence $\dom{S} \subseteq \dom{\extSRaw{S}{l}{\bot}}$.

      From the side condition of {\sc E-New}, $l \notin \dom{S}$.
      Hence $\extSRaw{S}{l}{\bot}$ adds a new binding for $l$ in $S$.

      Hence $\extSRaw{S}{l}{\bot}$ does not update any existing
      bindings in $S$.

      Hence, for all $l' \in \dom{S}, S(l') \userleq
      (\extSRaw{S}{l}{\bot})(l')$.

      Therefore $\leqstore{S}{\extSRaw{S}{l}{\bot}}$, as
      required.

    \item Case {\sc E-Put}:

      Given: $\config{S}{\putexp{l}{d_2}} \parstepsto
      \config{\extSRaw{S}{l}{\userlub{d_1}{d_2}}}{\unit}$.

      To show: $\leqstore{S}{\extSRaw{S}{l}{\userlub{d_1}{d_2}}}$.

      By Definition~\ref{def:lvars-leqstore}, we have to show that
      $\dom{S} \subseteq \dom{\extSRaw{S}{l}{\userlub{d_1}{d_2}}}$ and
      that for all $l' \in \dom{S}$, $S(l') \userleq
      (\extSRaw{S}{l}{\userlub{d_1}{d_2}})(l')$.

      By definition, a store update operation on $S$ can only either
      update an existing binding in $S$ or extend $S$ with a new
      binding.  Hence $\dom{S} \subseteq
      \dom{\extSRaw{S}{l}{\userlub{d_1}{d_2}}}$.

      From the premises of {\sc E-Put}, $S(l) = d_1$.  Therefore $l
      \in \dom{S}$.

      Hence $\extSRaw{S}{l}{\userlub{d_1}{d_2}}$ updates the existing
      binding for $l$ in $S$ from $d_1$ to $\userlub{d_1}{d_2}$.

      By the definition of $\userlub{}{}$, $d_1 \userleq
      (\userlub{d_1}{d_2})$.  $\extSRaw{S}{l}{\userlub{d_1}{d_2}}$
      does not update any other bindings in $S$, hence, for all $l'
      \in \dom{S}, S(l') \userleq
      (\extSRaw{S}{l}{\userlub{d_1}{d_2}})(l')$.

      Hence $\leqstore{S}{\extSRaw{S}{l}{\userlub{d_1}{d_2}}}$, as
      required.

    \item Case {\sc E-Put-Err}:

      Given: $\config{S}{\putexp{l}{d_2}} \parstepsto \error$.

      By the definition of $\error$, $\error$ is equal to
      $\config{\topS}{e}$ for all $e$.

      To show: $\leqstore{S}{\topS}$.

      Immediate by the definition of $\leqstore{}{}$.

    \item Case {\sc E-Get}:

      Immediate by the definition of $\leqstore{}{}$, since $S$ does
      not change.

  \end{itemize}

\end{proof}


\subsection{Independence lemma}

In order to show
Lemma~\ref{lem:lvars-strong-local-confluence}\TODO{audit}, I first
establish a ``frame property'' for $\lambdaLVar$ that captures the
idea that independent effects commute with each other.
Lemma~\ref{lem:lvars-independence}, the Independence lemma, is this
frame property.  Consider an expression $e$ that runs starting in
store $S$ and steps to $e'$, updating the store to $S'$.  The
Independence lemma provides a double-edged guarantee about what will
happen if one runs $e$ starting from a larger store
$\lubstore{S}{S''}$: first, $e$ will update the store to
$\lubstore{S'}{S''}$; second, it will step to $e'$ as it did before.
Here $\lubstore{S}{S''}$ is the least upper bound of the original $S$
and some other store $S''$ that is ``framed on'' to $S$; intuitively,
$S''$ is the store resulting from some other independently-running
computation.

Lemma~\ref{lem:lvars-independence} requires as a precondition that the
store $S''$ must be \emph{non-conflicting} with the original
transition from $\config{S}{e}$ to $\config{S'}{e'}$, meaning that
locations in $S''$ cannot share names with locations newly allocated
during the transition; this rules out location name conflicts caused
by allocation.

\LVarsDefNonConflicting

\LVarsLemIndependence
\begin{proof}
  Consider arbitrary $S''$ such that $S''$ is non-conflicting with
  $\config{S}{e} \parstepsto \config{S'}{e'}$ and $\lubstore{S'}{S''}
  \neq \topS$.

  To show: $\config{\lubstore{S}{S''}}{e} \parstepsto
  \config{\lubstore{S'}{S''}}{e'}$.

  The proof is by induction on the derivation of $\config{S}{e}
  \parstepsto \config{S'}{e'}$, by cases on the last rule in the
  derivation.  In every case we may assume that $\config{S'}{e'} \neq
  \error$.  Since $\config{S'}{e'} \neq \error$, we do not need to
  consider the {\sc E-Put-Err} rule.
  \begin{itemize}

    \item Case {\sc E-Eval-Ctxt}:

      Given: $\config{S}{\E{e}} \parstepsto \config{S'}{\E{e'}}$.

      To show: $\config{\lubstore{S}{S''}}{\E{e}} \parstepsto
      \config{\lubstore{S'}{S''}}{\E{e'}}$.

      From the premise of {\sc E-Eval-Ctxt}, we have that
      $\config{S}{e} \parstepsto \config{S'}{e'}$.

      Therefore, by IH, we have that $\config{\lubstore{S}{S''}}{e}
      \parstepsto \config{\lubstore{S'}{S''}}{e'}$.

      Therefore, by {\sc E-Eval-Ctxt}, we have that
      $\config{\lubstore{S}{S''}}{\E{e}} \parstepsto
      \config{\lubstore{S'}{S''}}{\E{e'}}$, as we were required to
      show.

    \item Case {\sc E-Beta}:

      Given: $\config{S}{\app{(\lam{x}{e})}{v}} \parstepsto
      \config{S}{\subst{e}{x}{v}}$.

      To show: $\config{\lubstore{S}{S''}}{\app{(\lam{x}{e})}{v}}
      \parstepsto \config{\lubstore{S}{S''}}{\subst{e}{x}{v}}$.

      Immediate by {\sc E-Beta}.

    \item Case {\sc E-New}:

      Given: $\config{S}{\NEW} \parstepsto
      \config{\extSRaw{S}{l}{\bot}}{l}$.

      To show: $\config{\lubstore{S}{S''}}{\NEW} \parstepsto
      \config{\lubstore{(\extSRaw{S}{l}{\bot})}{S''}}{l}$.

      By {\sc E-New}, we have that $\config{\lubstore{S}{S''}}{\NEW}
      \parstepsto \config{\extSRaw{(\lubstore{S}{S''})}{l'}{\bot}}{l'}$,
      where $l' \notin \dom{\lubstore{S}{S''}}$.

      By assumption, $S''$ is non-conflicting with $\config{S}{\NEW}
      \parstepsto \config{\extSRaw{S}{l}{\bot}}{l}$.
 
      Therefore $l \notin \dom{S''}$.

      From the side condition of {\sc E-New}, $l \notin \dom{S}$.

      Therefore $l \notin \dom{\lubstore{S}{S''}}$.

      Therefore, in
      $\config{\extSRaw{(\lubstore{S}{S''})}{l'}{\bot}}{l'}$, we can
      $\alpha$-rename $l'$ to $l$, \\ resulting in
      $\config{\extSRaw{(\lubstore{S}{S''})}{l}{\bot}}{l}$.

      Therefore $\config{\lubstore{S}{S''}}{\NEW} \parstepsto
      \config{\extSRaw{(\lubstore{S}{S''})}{l}{\bot}}{l}$.

      Note that:
      \begin{align*}
        \extSRaw{(\lubstore{S}{S''})}{l}{\bot} &=
        \lubstore{\extSRaw{S}{l}{\bot}}{\extSRaw{S''}{l}{\bot}} \\ &=
        \lubstore{\lubstore{S}{\store{\storebindingRaw{l}{\bot}}}}{\lubstore{S''}{\store{\storebindingRaw{l}{\bot}}}}
        \\ &=
        \lubstore{\lubstore{S}{\store{\storebindingRaw{l}{\bot}}}}{S''}
        \\ &= \lubstore{\extSRaw{S}{l}{\bot}}{S''}.
      \end{align*}
      Therefore $\config{\lubstore{S}{S''}}{\NEW} \parstepsto
      \config{\lubstore{\extSRaw{S}{l}{\bot}}{S''}}{l}$, as we were
      required to show.

    \item Case {\sc E-Put}:

      Given: $\config{S}{\putexp{l}{d_2}} \parstepsto
      \config{\extSRaw{S}{l}{d_2}}{\unit}$.

      To show: $\config{\lubstore{S}{S''}}{\putexp{l}{d_2}}
      \parstepsto
      \config{\lubstore{\extSRaw{S}{l}{d_2}}{S''}}{\unit}$.

      We will first show that

      $\config{\lubstore{S}{S''}}{\putexp{l}{d_2}} \parstepsto
      \config{\extSRaw{(\lubstore{S}{S''})}{l}{d_2}}{\unit}$

      and then show why this is sufficient.

      We proceed by cases on $l$:

      \begin{itemize}
        \item $l \notin \dom{S''}$:

          By assumption, $\lubstore{\extSRaw{S}{l}{d_2}}{S''} \neq
          \topS$.

          By Lemma~\ref{lem:lvars-monotonicity},
          $\leqstore{S}{\extSRaw{S}{l}{d_2}}$.

          Hence $\lubstore{S}{S''} \neq \topS$.

          Therefore, by Definition~\ref{def:lvars-lubstore},
          $(\lubstore{S}{S''})(l) = S(l)$.

          From the premises of {\sc E-Put}, $S(l) = d_1$.

          Hence $(\lubstore{S}{S''})(l) = d_1$.

          From the premises of {\sc E-Put}, $d_2 = \userlub{d_1}{d_2}$
          and $d_2 \neq \top$.

          Therefore, by {\sc E-Put}, we have:
          $\config{\lubstore{S}{S''}}{\putexp{l}{d_2}} \parstepsto
          \config{\extSRaw{(\lubstore{S}{S''})}{l}{d_2}}{\unit}$.

        \item $l \in \dom{S''}$:

          By assumption, $\lubstore{\extSRaw{S}{l}{d_2}}{S''} \neq
          \topS$.

          By Lemma~\ref{lem:lvars-monotonicity},
          $\leqstore{S}{\extSRaw{S}{l}{d_2}}$.

          Hence $\lubstore{S}{S''} \neq \topS$.

          Therefore $(\lubstore{S}{S''})(l) = \userlub{S(l)}{S''(l)}$.

          From the premises of {\sc E-Put}, $S(l) = d_1$.
          
          Hence $(\lubstore{S}{S''})(l) = d'_1$, where $d_1 \userleq
          d'_1$.

          From the premises of {\sc E-Put}, $d_2 =
          \userlub{d_1}{d_2}$.

          Let $d'_2 = \userlub{d'_1}{d_2}$.

          Hence $d_2 \userleq d'_2$.

          By assumption, $\lubstore{\extSRaw{S}{l}{d_2}}{S''} \neq
          \topS$.

          Therefore, by Definition~\ref{def:lvars-lubstore},
          $\lubstore{d_2}{S''(l)} \neq \top$.

          Note that:
          \begin{align*}
            \top &\neq \lubstore{d_2}{S''(l)} \\ &=
            \userlub{\userlub{d_1}{d_2}}{S''(l)} \\ &=
            \userlub{\userlub{S(l)}{d_2}}{S''(l)} \\ &=
            \userlub{\userlub{S(l)}{S''(l)}}{d_2} \\ &=
            \userlub{(\lubstore{S}{S''})(l)}{d_2} \\ &=
            \userlub{d'_1}{d_2} \\ &= d'_2. \\
          \end{align*}
          Hence $d'_2 \neq \top$.

          Hence $(\lubstore{S}{S''})(l) = d'_1$ and $d'_2 =
          \userlub{d'_1}{d_2}$ and $d'_2 \neq \top$.

          Therefore, by {\sc E-Put} we have:
          $\config{\lubstore{S}{S''}}{\putexp{l}{d_2}} \parstepsto
          \config{\extSRaw{(\lubstore{S}{S''})}{l}{d'_2}}{\unit}$.

          \lk{If we really wanted to be pedantic here, we'd actually
            prove that the stores are equal.  I'm assuming that if I
            can show that $\extSRaw{(\lubstore{S}{S''})}{l}{d'_2}$ and
            $\extSRaw{(\lubstore{S}{S''})}{l}{d_2}$ bind $l$ to the
            same value, then it will be obvious that they're equal.}

          Note that:
          \begin{align*}
            (\extSRaw{(\lubstore{S}{S''})}{l}{d'_2})(l) &=
            \userlub{(\lubstore{S}{S''})(l)}{(\store{\storebindingRaw{l}{d'_2}})(l)}
            \\ &= \userlub{d'_1}{d'_2} \\ &=
            \userlub{d'_1}{\userlub{d'_1}{d_2}} \\ &=
            \userlub{d'_1}{d_2}
          \end{align*}
          and
          \begin{align*}
            (\extSRaw{(\lubstore{S}{S''})}{l}{d_2})(l) &=
            \userlub{(\lubstore{S}{S''})(l)}{(\store{\storebindingRaw{l}{d_2}})(l)}
            \\ &= \userlub{d'_1}{d_2} \\ &=
            \userlub{d'_1}{\userlub{d_1}{d_2}} \\ &=
            \userlub{d'_1}{d_2} & \textrm{(since $d_1 \userleq
              d'_1$).}
          \end{align*}
          Therefore $\extSRaw{(\lubstore{S}{S''})}{l}{d'_2} =
          \extSRaw{(\lubstore{S}{S''})}{l}{d_2}$.

          Therefore, $\config{\lubstore{S}{S''}}{\putexp{l}{d_2}}
          \parstepsto
          \config{\extSRaw{(\lubstore{S}{S''})}{l}{d_2}}{\unit}$.
      \end{itemize}

      Note that:
      \begin{align*}
        \extSRaw{(\lubstore{S}{S''})}{l}{d_2} &=
        \lubstore{\extSRaw{S}{l}{d_2}}{\extSRaw{S''}{l}{d_2}} \\ &=
        \lubstore{\lubstore{S}{\store{\storebindingRaw{l}{d_2}}}}{\lubstore{S''}{\store{\storebindingRaw{l}{d_2}}}}
        \\ &=
        \lubstore{\lubstore{S}{\store{\storebindingRaw{l}{d_2}}}}{S''}
        \\ &= \lubstore{\extSRaw{S}{l}{d_2}}{S''}.
      \end{align*}
      Therefore $\config{\lubstore{S}{S''}}{\putexp{l}{d_2}}
      \parstepsto
      \config{\lubstore{\extSRaw{S}{l}{d_2}}{S''}}{\unit}$, as we were
      required to show.

    \item Case {\sc E-Get}:

      Given: $\config{S}{\getexp{l}{T}} \parstepsto \config{S}{d_2}$.

      To show: $\config{\lubstore{S}{S''}}{\getexp{l}{T}} \parstepsto
      \config{\lubstore{S}{S''}}{d_2}$.

      From the premises of {\sc E-Get}, $S(l) = d_1$ and $\incomp{T}$
      and $d_2 \in T$ and $d_2 \userleq d_1$.

      By assumption, $\lubstore{S}{S''} \neq \topS$.

      Hence $(\lubstore{S}{S''}) = d'_1$, where $d_1 \userleq d'_1$.

      By the transitivity of $\userleq$, $d_2 \userleq d'_1$.

      Hence, $S(l) = d'_1$ and $\incomp{T}$ and $d_2 \in T$ and $d_2
      \userleq d'_1$.

      Therefore, by {\sc E-Get},

      $\config{\lubstore{S}{S''}}{\getexp{l}{T}} \parstepsto
      \config{\lubstore{S}{S''}}{d_2}$,

      as we were required to show.
  \end{itemize}
\end{proof}


\subsection{Clash lemma}

The Clash lemma, Lemma~\ref{lem:lvars-clash}, is similar to the
Independence lemma, but handles the case where $\lubstore{S'}{S''} =
\topS$.  It establishes that, in that case,
$\config{\lubstore{S}{S''}}{e}$ steps to $\error$.

\lk{We didn't need the Clash lemma for the quasi-determinism proof,
  but I'm pretty sure the only reason we didn't need it was because
  quasi-determinism is ``...or $\error$''.  I think we actually do
  need it here.}

\LVarsLemClash
\begin{proof}
  Consider arbitrary $S''$ such that $S''$ is non-conflicting with
  $\config{S}{e} \parstepsto \config{S'}{e'}$ and $\lubstore{S'}{S''}
  = \topS$.

  To show: $\config{\lubstore{S}{S''}}{e} \parstepsto \error$.

  The proof is by induction on the derivation of $\config{S}{e}
  \parstepsto \config{S'}{e'}$, by cases on the last rule in the
  derivation.  In every case we may assume that $\config{S'}{e'} \neq
  \error$.  Since $\config{S'}{e'} \neq \error$, we do not need to
  consider the {\sc E-Put-Err} rule.

  \begin{itemize}

    \item Case {\sc E-Eval-Ctxt}:

      Given: $\config{S}{\E{e}} \parstepsto \config{S'}{\E{e'}}$.

      To show: $\config{\lubstore{S}{S''}}{\E{e}} \parstepsto^i
      \error$, where $i \leq 1$.

      From the premise of {\sc E-Eval-Ctxt}, we have that
      $\config{S}{e} \parstepsto \config{S'}{e'}$.

      Therefore, by IH, we have that $\config{\lubstore{S}{S''}}{e}
      \parstepsto^{i'} \error$, where $i' \leq 1$.

      We proceed by cases on $i'$:

      \begin{itemize}
        \item $i' = 0$:

          In this case, $\config{\lubstore{S}{S''}}{e} = \error$.

          Hence, by the definition of $\error$, $\lubstore{S}{S''} =
          \topS$.

          Hence $\config{\lubstore{S}{S''}}{\E{e}} = \error$.

          Hence $\config{\lubstore{S}{S''}}{\E{e}} \parstepsto^i
          \error$, with $i = 0$.

        \item $i' = 1$:

          In this case, $\config{\lubstore{S}{S''}}{e} \parstepsto
          \error$.

          By the definition of $\error$, $\error =
          \config{\topS}{e''}$ for any $e''$.

          Hence $\config{\lubstore{S}{S''}}{e} \parstepsto
          \config{\topS}{e''}$.

          Hence, by {\sc E-Eval-Ctxt},
          $\config{\lubstore{S}{S''}}{\E{e}} \parstepsto
          \config{\topS}{\E{e''}}$.

          By the definition of $\error$, $\config{\topS}{\E{e''}} =
          \error$.

          Hence $\config{\lubstore{S}{S''}}{\E{e}} \parstepsto
          \error$.

          Hence $\config{\lubstore{S}{S''}}{\E{e}} \parstepsto^i
          \error$, with $i = 1$.

      \end{itemize}

    \item Case {\sc E-Beta}:

      Given: $\config{S}{\app{(\lam{x}{e})}{v}} \parstepsto
      \config{S}{\subst{e}{x}{v}}$.

      To show: $\config{\lubstore{S}{S''}}{\app{(\lam{x}{e})}{v}}
      \parstepsto^i \error$, where $i \leq 1$.

      By assumption, $\lubstore{S}{S''} = \topS$.

      Hence, by the definition of $\error$,
      $\config{\lubstore{S}{S''}}{\app{(\lam{x}{e})}{v}} = \error$.

      Hence $\config{\lubstore{S}{S''}}{\app{(\lam{x}{e})}{v}}
      \parstepsto^i \error$, with $i = 0$.

    \item Case {\sc E-New}:

      Given: $\config{S}{\NEW} \parstepsto
      \config{\extSRaw{S}{l}{\bot}}{l}$.

      To show: $\config{\lubstore{S}{S''}}{\NEW} \parstepsto^i
      \error$, where $i \leq 1$.

      By {\sc E-New}, $\config{\lubstore{S}{S''}}{\NEW} \parstepsto
      \config{\extSRaw{(\lubstore{S}{S''})}{l'}{\bot}}{l'}$, where $l'
      \notin \dom{\lubstore{S}{S''}}$.

      By assumption, $S''$ is non-conflicting with $\config{S}{\NEW}
      \parstepsto \config{\extSRaw{S}{l}{\bot}}{l}$.
 
      Therefore $l \notin \dom{S''}$.

      From the side condition of {\sc E-New}, $l \notin \dom{S}$.

      Therefore $l \notin \dom{\lubstore{S}{S''}}$.

      Therefore, in
      $\config{\extSRaw{(\lubstore{S}{S''})}{l'}{\bot}}{l'}$, we can
      $\alpha$-rename $l'$ to $l$, \\ resulting in
      $\config{\extSRaw{(\lubstore{S}{S''})}{l}{\bot}}{l}$.

      Therefore $\config{\lubstore{S}{S''}}{\NEW} \parstepsto
      \config{\extSRaw{(\lubstore{S}{S''})}{l}{\bot}}{l}$.

      By assumption, $\lubstore{\extSRaw{S}{l}{\bot}}{S''}
      = \topS$.

      Note that:
      \begin{align*}
        \topS &= \lubstore{\extSRaw{S}{l}{\bot}}{S''} \\ &=
        \lubstore{\lubstore{S}{\store{\storebindingRaw{l}{\bot}}}}{S''}
        \\ &=
        \lubstore{\lubstore{S}{S''}}{\store{\storebindingRaw{l}{\bot}}}
        \\ &=
        \lubstore{(\lubstore{S}{S''})}{\store{\storebindingRaw{l}{\bot}}}
        \\ &= \extSRaw{(\lubstore{S}{S''})}{l}{\bot} .
      \end{align*}

      Hence $\config{\lubstore{S}{S''}}{\NEW} \parstepsto
      \config{\topS}{l}$.

      Hence, by the definition of $\error$,
      $\config{\lubstore{S}{S''}}{\NEW} \parstepsto \error$.

      Hence $\config{\lubstore{S}{S''}}{\NEW} \parstepsto^i \error$,
      with $i = 1$.

    \item Case {\sc E-Put}:

      Given: $\config{S}{\putexp{l}{d_2}} \parstepsto
      \config{\extSRaw{S}{l}{d_2}}{\unit}$.

      To show: $\config{\lubstore{S}{S''}}{\putexp{l}{d_2}}
      \parstepsto^i \error$, where $i \leq 1$.

      We proceed by cases on $\lubstore{S}{S''}$:

      \begin{itemize}

        \item $\lubstore{S}{S''} = \topS$:

          In this case, by the definition of $\error$,
          $\config{\lubstore{S}{S''}}{\putexp{l}{d_2}} = \error$.

          Hence $\config{\lubstore{S}{S''}}{\putexp{l}{d_2}}
          \parstepsto^i \error$, with $i = 0$.

        \item $\lubstore{S}{S''} \neq \topS$:

          From the premises of {\sc E-Put}, we have that $S(l) = d_1$.

          Hence $(\lubstore{S}{S''})(l) = d'_1$, where $d_1 \userleq
          d'_1$.

          We show that $\userlub{d'_1}{d_2} =
          \top$, as follows:

          By assumption, $\lubstore{\extSRaw{S}{l}{d_2}}{S''} = \topS$.

          Hence, by Definition~\ref{def:lvars-lubstore}, there exists
          some $l' \in \dom{\extSRaw{S}{l}{d_2}} \cap \dom{S''}$ such
          that $\userlub{(\extSRaw{S}{l}{d_2})(l')}{S''(l')} = \top$.

          Now case on $l'$:

          \begin{itemize}
            \item $l' \neq l$:

              In this case, $(\extSRaw{S}{l}{d_2})(l') = S(l')$.

              Since $\userlub{(\extSRaw{S}{l}{d_2})(l')}{S''(l')} = \top$,
              we then have that $\userlub{S(l')}{S''(l')} = \top$.

              However, this is a contradiction since
              $\lubstore{S}{S''} \neq \topS$.

              Hence this case cannot occur.

            \item $l' = l$:

              Then $\userlub{(\extSRaw{S}{l}{d_2})(l)}{S''(l)} = \top$.

              Note that:
              \begin{align*}
                \top &= \userlub{(\extSRaw{S}{l}{d_2})(l)}{S''(l)} \\ &=
                \userlub{d_2}{S''(l)} \\ &=
                \userlub{\userlub{d_1}{d_2}}{S''(l)}
                \\ &=
                \userlub{\userlub{S(l)}{d_2}}{S''(l)}
                \\ &=
                \userlub{\userlub{S(l)}{S''(l)}}{d_2}
                \\ &=
                \userlub{(\lubstore{S}{S''})(l)}{d_2}
                \\ &= \userlub{d'_1}{d_2}.
              \end{align*}
              Hence $\userlub{d'_1}{d_2} = \top$.

              Hence, by {\sc E-Put-Err},
              $\config{\lubstore{S}{S''}}{\putexp{l}{d_2}} \parstepsto
              \error$.

              Hence $\config{\lubstore{S}{S''}}{\putexp{l}{d_2}}
              \parstepsto^i \error$, with $i = 1$.

          \end{itemize}

      \end{itemize}

    \item Case {\sc E-Get}:

      Given: $\config{S}{\getexp{l}{T}} \parstepsto \config{S}{d_2}$.

      To show: $\config{\lubstore{S}{S''}}{\getexp{l}{T}}
      \parstepsto^i \error$, where $i \leq 1$.

      By assumption, $\lubstore{S}{S''} = \topS$.

      Hence, by the definition of $\error$,
      $\config{\lubstore{S}{S''}}{\getexp{l}{T}} = \error$.

      Hence $\config{\lubstore{S}{S''}}{\getexp{l}{T}} \parstepsto^i
      \error$, with $i = 0$.
  \end{itemize}
\end{proof}


\subsection{Error Preservation lemma}

Lemma~\ref{lem:lvars-error-preservation}, Error Preservation, says
that if a configuration $\config{S}{e}$ steps to $\error$, then
evaluating $e$ in the context of some larger store will also result in
$\error$.

\lk{We didn't need the Error Preservation lemma for the
  quasi-determinism proof, but I'm pretty sure the only reason we
  didn't need it was because quasi-determinism is ``...or $\error$''.
  I think we actually do need it here.}

\LVarsLemErrorPreservation
\begin{proof}

  Given: $\config{S}{e} \parstepsto \error$ and $\leqstore{S}{S'}$.

  To show: $\config{S'}{e} \parstepsto \error$.

  \TODO{Figure out what to do here.  I think we need to handle both
    E-Eval-Ctxt and E-Put-Err.}
\end{proof}


\subsection{Diamond lemma}

\LVarsLemDiamond
\begin{proof}
  By induction on the derivation of $\conf \parstepsto \conf_a$, by
  cases on the last rule in the derivation.

  \begin{itemize}
    \item {\sc E-Eval-Ctxt}: \TODO{Figure out what to do here.}

    \item {\sc E-Beta}: $\conf = \config{S}{\app{(\lam{x}{e})}{v}}$,
      and $\conf_a = \config{S}{\subst{e}{x}{v}}$.

      Given:
      \begin{itemize}
      \item $\config{S}{\app{(\lam{x}{e})}{v}} \parstepsto
        \config{S}{\subst{e}{x}{v}}$, and
      \item $\config{S}{\app{(\lam{x}{e})}{v}} \parstepsto \conf_b$.
      \end{itemize}

      To show: There exist $\conf_c, i, j$ such that

      $\config{S}{\subst{e}{x}{v}} \parstepsto^i \conf_c$ and $\conf_b
      \parstepsto^j \conf_c$ and $i \leq 1$ and $j \leq 1$.

      By inspection of the operational semantics, $\conf_b =
      \config{S}{\subst{e}{x}{v}}$.

      Choose $\conf_c = \config{S}{\subst{e}{x}{v}}$, $i = 0$ and $j =
      0$.

      Then $\config{S}{\subst{e}{x}{v}} = \conf_c$ and $\conf_b =
      \conf_c$, as required.

    \item {\sc E-New}: $\conf = \config{S}{\NEW}$, and $\conf_a =
      \config{\extSRaw{S}{l}{\bot}}{l}$.

      Given:
      \begin{itemize}
      \item $\config{S}{\NEW} \parstepsto
        \config{\extSRaw{S}{l}{\bot}}{l}$, and
      \item $\config{S}{\NEW} \parstepsto \conf_b$.
      \end{itemize}

      To show: There exist $\conf_c, i, j$ such that

      $\config{\extSRaw{S}{l}{\bot}}{l} \parstepsto^i
      \conf_c$ and $\conf_b \parstepsto^j \conf_c$ and $i \leq 1$ and
      $j \leq 1$.

      By inspection of the operational semantics, $\conf_b =
      \config{\extSRaw{S}{l'}{\bot}}{l'}$.

      From the side condition of {\sc E-New}, $l \notin S$.

      Therefore, in $\config{\extSRaw{S}{l'}{\bot}}{l'}$,
      we can $\alpha$-rename $l'$ to $l$, resulting in
      $\config{\extSRaw{S}{l}{\bot}}{l}$.

      Choose $\conf_c = \config{\extSRaw{S}{l}{\bot}}{l}$,
      $i = 0$ and $j = 0$.

      Then $\config{S}{\subst{e}{x}{v}} = \conf_c$ and $\conf_b =
      \conf_c$, as required.

    \item {\sc E-Put}: $\conf = \config{S}{\putexp{l}{d_2}}$, and
      $\conf_a = \config{\extSRaw{S}{l}{d_2}}{\unit}$.

      Given:
      \begin{itemize}
      \item $\config{S}{\putexp{l}{d_2}} \parstepsto
        \config{\extSRaw{S}{l}{d_2}}{\unit}$, and
      \item $\config{S}{\putexp{l}{d_2}} \parstepsto \conf_b$.
      \end{itemize}

      To show: There exist $\conf_c, i, j$ such that

      $\config{\extSRaw{S}{l}{d_2}}{\unit} \parstepsto^i \conf_c$ and
      $\conf_b \parstepsto^j \conf_c$ and $i \leq 1$ and $j \leq 1$.

      By inspection of the operational semantics, $\conf_b =
      \config{\extSRaw{S}{l}{d_2}}{\unit}$.

      Choose $\conf_c = \config{\extSRaw{S}{l}{d_2}}{\unit}$, $i = 0$
      and $j = 0$.

      Then $\config{\extSRaw{S}{l}{d_2}}{\unit} = \conf_c$ and $\conf_b
      = \conf_c$, as required.

    \item {\sc E-Put-Err}: $\conf = \config{S}{\putexp{l}{d_2}}$, and
      $\conf_a = \error$.

      Given:
      \begin{itemize}
      \item $\config{S}{\putexp{l}{d_2}} \parstepsto \error$, and
      \item $\config{S}{\putexp{l}{d_2}} \parstepsto \conf_b$.
      \end{itemize}

      To show: There exist $\conf_c, i, j$ such that

      $\error \parstepsto^i \conf_c$ and $\conf_b \parstepsto^j
      \conf_c$ and $i \leq 1$ and $j \leq 1$.

      By inspection of the operational semantics, $\conf_b = \error$.

      Choose $\conf_c = \error$, $i = 0$ and $j = 0$.

      Then $\error = \conf_c$ and $\conf_b = \conf_c$, as required.

    \item {\sc E-Get}: $\conf = \config{S}{\getexp{l}{T}}$, and
      $\conf_a = \config{S}{d_2}$.

      Given:
      \begin{itemize}
      \item $\config{S}{\getexp{l}{T}} \parstepsto \config{S}{d_2}$,
        and
      \item $\config{S}{\getexp{l}{T}} \parstepsto \conf_b$.
      \end{itemize}

      To show: There exist $\conf_c, i, j$ such that

      $\config{S}{d_2} \parstepsto^i \conf_c$ and $\conf_b
      \parstepsto^j \conf_c$ and $i \leq 1$ and $j \leq 1$.

      By inspection of the operational semantics, $\conf_b =
      \config{S}{d_2}$.

      Choose $\conf_c = \config{S}{d_2}$, $i = 0$ and $j = 0$.

      Then $\config{S}{d_2} = \conf_c$ and $\conf_b = \conf_c$, as
      required.
  \end{itemize}
\end{proof}


\subsection{Confluence lemmas and determinism theorem}

\LVarsCorStrongLocalConfluence

\TODO{Deal with permutation stuff here?}

\LVarsLemStrongOneSidedConfluence

\TODO{Deal with permutation stuff here?}

\TODO{proof}

\LVarsLemStrongConfluence

\TODO{Deal with permutation stuff here?}

\TODO{proof}

\LVarsLemConfluence

\TODO{Deal with permutation stuff here?}

\LVarsThmDeterminism

\TODO{Deal with permutation stuff here?}

\subsection{Discussion: termination}

I have followed Budimli\'c \etal~\cite{CnC} in treating
\emph{determinism} separately from the issue of \emph{termination}.
Yet one might legitimately be concerned that in $\lambdaLVar$, a
configuration could have both an infinite reduction path and one that
terminates with a value.  Theorem~\ref{thm:lvars-determinism} says
that if two runs of a given $\lambdaLVar$ program reach configurations
where no more reductions are possible, then they have reached the same
configuration.  Hence Theorem~\ref{thm:lvars-determinism} handles the
case of \emph{deadlocks} already: a $\lambdaLVar$ program can deadlock
(\eg, with a blocked $\GET$), but it will do so deterministically.

However, Theorem~\ref{thm:lvars-determinism} has nothing to say about
\emph{livelocks}, in which a program reduces infinitely.  It would be
desirable to have a \emph{consistent termination} property which would
guarantee that if one run of a given $\lambdaLVar$ program terminates
with a non-$\error$ result, then every run will.  I conjecture (but do
not prove) that such a consistent termination property holds for
$\lambdaLVar$.  Such a property could be paired with
Theorem~\ref{thm:lvars-determinism} to guarantee that if one run of a
given $\lambdaLVar$ program terminates in a non-$\error$ configuration
$\sigma$, then every run of that program terminates in $\sigma$.  (The
``non-$\error$ configuration'' condition is necessary because it is
possible to construct a $\lambdaLVar$ program that can terminate in
$\error$ on some runs and diverge on others.  By contrast, the
existing determinism theorem does not have to treat $\error$
specially.)
