\section{Proof of determinism for $\lambdaLVar$}\label{s:lvars-proof}

The main technical result of this chapter is a proof of determinism
for the $\lambdaLVar$ language.  The determinism theorem says that if
two executions starting from a given configuration $\conf$ terminate
in configurations $\conf'$ and $\conf''$, then $\conf'$ and $\conf''$
are the same configuration, up to a permutation on locations.  (I
discuss permutations in more detail below, in
Section~\ref{subsection:lvars-permutations}.)

In order to prove determinism for $\lambdaLVar$, I first prove several
supporting lemmas.

Lemma~\ref{lem:lvars-permutability} (Permutability) deals with
location names, and Lemma~\ref{lem:lvars-locality} (Locality)
establishes a useful property for dealing with expressions that
decompose into redex and context in multiple ways.

After that point, the structure of the proof is similar to that of the
proof of determinism for Featherweight CnC given by Budimli\'c
\etal~\cite{CnC}.  I reuse the naming conventions of Budimli\'c
\etal~for Lemmas~\ref{lem:lvars-monotonicity} (Monotonicity),
\ref{lem:lvars-independence} (Independence), \ref{lem:lvars-clash}
(Clash), \ref{lem:lvars-error-preservation} (Error Preservation), and
\ref{lem:lvars-strong-local-confluence} (Strong Local
Confluence). However, the statements and proofs of those lemmas differ
considerably in the setting of $\lambdaLVar$, due to the generality of
LVars and other differences between the $\lambdaLVar$ language and
Featherweight CnC.

However, Lemmas~\ref{lem:lvars-strong-one-sided-confluence} (Strong
One-Sided Confluence), \ref{lem:lvars-strong-confluence} (Strong
Confluence), and \ref{lem:confluence} (Confluence) are nearly
identical to the corresponding lemmas in the Featherweight CnC
determinism proof.  This is the case because, once
Lemmas~\ref{lem:lvars-monotonicity} through
\ref{lem:lvars-strong-local-confluence} are established, the remainder
of the determinism proof does not need to deal specifically with the
semantics of LVars, lattices, or the store, and instead deals only
with execution steps at a high level.

\TODO{Say something about how later I'll show determinism for
  generalized monotonic writes and generalized threshold reads, if in
  fact I'm going to do that.}

\subsection{Permutations and permutability}\label{subsection:lvars-permutations}

The {\sc E-New} rule allocates a fresh location $l \in \Loc$ in the
store, with the only requirement on $l$ being that it is not (yet) in
the domain of the store.  Therefore, multiple runs of the same program
may differ in what locations they allocate, and therefore the
reduction semantics is nondeterministic with respect to allocation.
However, this is not a kind of nondeterminism that we care about, so
we work modulo an arbitrary \emph{permutation} on locations.

Recall from Section~\ref{subsection:lvars-stores} that we have a
countable set of locations $\Loc$.  Then, a permutation is defined as
follows:

\LVarsDefPermutation

Condition (1) in Definition~\ref{def:lvars-permutation} ensures that
we only consider location renamings that we can ``undo'', and
condition (2) ensures that we only consider renamings of a finite
number of locations.  Equivalently, we can say that $\pi$ is a
bijection from $\Loc$ to $\Loc$ such that it is the identity on all
but finitely many elements.

We can straightforwardly lift Definition~\ref{def:lvars-permutation}
to expressions and stores, and hence also to configurations.  To lift
a permutation $\pi$ to expressions, we over the syntax of terms
structurally and apply $\pi$ to any locations that occur in the term.
We can also lift $\pi$ to evaluation contexts, structurally: $\pi([~])
= [~]$, $\pi(\app{E}{e} = \app{\pi(E)}{\pi(e)}$, and so on.

To lift $\pi$ to stores, we apply $\pi$ to all locations in the domain
of the store.  (We do not have to do any renaming in the codomain of
the store, since locations cannot occur in elements of the
application-specific lattice $D$.)  Since $\pi$ is a bijection, it
follows that if some location $l$ is not in the domain of some store
$S$, then $\pi(l) \notin \dom{(\pi(S)}$,\lk{Do I need to spell out why
  this is true?} a fact that will be useful to us shortly.

\LVarsDefPermutationExpression

\LVarsDefPermutationStore

\LVarsDefPermutationConfiguration

With these definitions in place, I can prove a lemma that says that
the names of locations in a configuration do not affect whether or not
it can take a step. Lemma~\ref{lem:lvars-permutability} says that a
configuration $\conf$ can step to $\conf$ exactly when $\pi(\conf)$
can step to $\pi(\conf)$.

\LVarsLemPermutability
\begin{proof}
  See Section~\ref{section:lvars-permutability-proof}.  The forward
  direction of part~\ref{thm:permutable-reduction-transitions} is by
  cases on the rule in the reduction semantics by which $\conf$ steps
  to $\conf'$; the only interesting case is the {\sc E-New} case, in
  which we make use of the fact that if $l \notin \dom{S}$, then
  $\pi(l) \notin \dom{\pi(S)}$. The reverse direction of
  part~\ref{thm:permutable-reduction-transitions} relies on the fact
  that if $\pi$ is a permutation, then $\piinv$ is also a permutation.
  Part~\ref{thm:permutable-context-transitions} of the proof builds on
  part~\ref{thm:permutable-reduction-transitions}.
\end{proof}

\subsection{Internal determinism of the reduction semantics}

My goal is to show that $\lambdaLVar$ is deterministic according to
the definition of \emph{observable determinism} that I gave in
Chapter~\ref{ch:intro}---that is, that a $\lambdaLVar$ program always
evaluates to the same value.  In the context of $\lambdaLVar$, a
``program'' can be understood as a configuration, and a ``value'' can
be understood as a configuration that cannot step, either because the
expression in that configuration is actually a value, or because it is
a ``stuck'' configuration that cannot step because no rule of the
operational semantics applies.  In $\lambdaLVar$, the latter situation
could occur if, for instance, a configuration contains a blocking
@get@ expression and there are no other expressions left to evaluate
that might cause it to unblock.

This definition of observable determinism does \emph{not} require that
a configuration takes the same sequence of steps on the way to
reaching its value at the end of every run, and in fact, the
$\lambdaLVar$ operational semantics does not have that property.
Borrowing terminology from
Blelloch~\etal~\cite{blelloch-internally-deterministic}, I will use
the term \emph{internally deterministic} to describe a program that
does, in fact, take the same sequence of steps on every
run.\footnote{I am using ``internally deterministic'' in a more
  specific way than Blelloch~\etal: they define an internally
  deterministic program to be one for which the \emph{trace} of the
  program is the same on every run, where a trace is a directed
  acyclic graph of the operations executed by the program and the
  control dependencies among them.  This definition, in turn, depends
  on the definition of ``operation'', which might be defined in a
  fine-grained way or a coarse-grained, abstract way, depending on
  which aspects of program execution one wants the notion of internal
  determinism to capture.  The important point is that internal
  determinism is a stronger property than observable determinism.}
Although $\lambdaLVar$ is not internally deterministic, all of its
internal nondeterminism is due to the {\sc E-Eval-Ctxt} rule!  This is
the case because the {\sc E-Eval-Ctxt} rule is the only rule in the
operational semantics by which a particular configuration can step in
multiple ways.  The multiple ways in which a configuration can step
via {\sc E-Eval-Ctxt} correspond to the ways in which the expression
in that configuration can be decomposed into a redex and an evaluation
context.  In fact, it is exactly this property that makes it possible
for multiple subexpressions of a $\lambdaLVar$ expression (a @let par@
expression, for instance) to be evaluated in parallel.

But, if we leave aside evaluation contexts for the moment (we'll
return to them in the following section) and focus on only the rules
of the reduction semantics in
Figure~\ref{f:lvars-lambdaLVar-reduction-semantics}, we see that there
is only one rule by which a given configuration can step, and only one
configuration to which it can step.  The exception is the {\sc E-New}
rule, which nondeterministically allocates locations and returns
pointers to them---but we can account for this by saying that the
reduction semantics is internally deterministic up to a permutation on
locations.  Lemma~\ref{lem:lvars-internal-determinism} formalizes this
claim.

\lk{We don't actually use this fact anywhere!  Could I have used it to
  simplify the proof of Lemma~\ref{lem:lvars-strong-local-confluence}}
or something?}

\LVarsLemInternalDeterminism
\begin{proof}
  Straightforward by cases on the rule of the reduction semantics by
  which $\conf$ steps to $\conf'$; the only interesting case is for
  the {\sc E-New} rule.  See
  Section~\ref{section:lvars-internal-determinism-proof}.
\end{proof}

\subsection{Locality lemma}

In order to prove determinism for $\lambdaLVar$ we will have to
consider situations in which we have an expression that decomposes
into redex and context in multiple ways.  Suppose that we have an
expression $e$ such that $e = \evalctxt{E_1}{e_1} =
\evalctxt{E_2}{e_2}$.  The configuration $\config{S}{e}$ can then step
in two different ways by the {\sc E-Eval-Ctxt} rule:
$\config{S}{\evalctxt{E_1}{e_1}} \ctxstepsto
\config{S_1}{\evalctxt{E_1}{e'_1}}$, and
$\config{S}{\evalctxt{E_2}{e_2}} \ctxstepsto
\config{S_2}{\evalctxt{E_2}{e'_2}}$.

The key observation we can make here is that the $\ctxstepsto$
relation acts ``locally''.  That is, when $e_1$ steps to $e'_1$ within
its context, the expression $e_2$ will be left alone, because it
belongs to the context.  Likewise, when $e_2$ steps to $e'_2$ within
its context, the expression $e_1$ will be left alone.
Lemma~\ref{lem:lvars-locality} formalizes this claim.

\LVarsLemLocality
\begin{proof}
  \TODO{Add short description here once the proof is done.} See
  Section~\ref{section:lvars-locality-proof}.
\end{proof}

\subsection{Monotonicity lemma}

The Monotonicity lemma says that, as evaluation proceeds according to
the $\parstepsto$ relation, the store can only grow with respect to
the $\leqstore{}{}$ ordering.

\LVarsLemMonotonicity
\begin{proof}
  Straightforward by cases on the rule of the reduction semantics by
  which $\config{S}{e}$ steps to $\config{S'}{e'}$. The interesting
  cases are for the {\sc E-New} and {\sc E-Put} rules.  See
  Section~\ref{section:lvars-monotonicity-proof}.
\end{proof}

\subsection{Independence lemma}\label{subsection:lvars-independence}

The Independence lemma establishes a ``frame property'' for
$\lambdaLVar$ that captures the idea that independent effects commute
with each other.  Consider an expression $e$ that runs starting in
store $S$ and steps to $e'$, updating the store to $S'$.  The
Independence lemma provides a double-edged guarantee about what will
happen if we evaluate $e$ starting from a larger store
$\lubstore{S}{S''}$: first, $e$ will update the store to
$\lubstore{S'}{S''}$; second, $e$ will step to $e'$ as it did before.
Here $\lubstore{S}{S''}$ is the least upper bound of the original $S$
and some other store $S''$ that is ``framed on'' to $S$; intuitively,
$S''$ is the store resulting from some other independently-running
computation.

Lemma~\ref{lem:lvars-independence} requires as a precondition that the
store $S''$ must be \emph{non-conflicting} with the original
transition from $\config{S}{e}$ to $\config{S'}{e'}$, meaning that
locations in $S''$ cannot share names with locations newly allocated
during the transition; this rules out location name conflicts caused
by allocation.

\LVarsDefNonConflicting

\LVarsLemIndependence
\begin{proof}
  By cases on the rule of the reduction semantics by which
  $\config{S}{e}$ steps to $\config{S'}{e'}$. The interesting cases
  are for the {\sc E-New} and {\sc E-Put} rules.  Since
  $\config{S'}{e'} \neq \error$, we do not need to consider the {\sc
    E-Put-Err} rule.  See
  Section~\ref{section:lvars-independence-proof}.
\end{proof}

\subsection{Clash lemma}

The Clash lemma, Lemma~\ref{lem:lvars-clash}, is similar to the
Independence lemma, but handles the case where $\lubstore{S'}{S''} =
\topS$.  It establishes that, in that case,
$\config{\lubstore{S}{S''}}{e}$ steps to $\error$.

\LVarsLemClash
\begin{proof}
  By cases on the rule of the reduction semantics by which
  $\config{S}{e}$ steps to $\config{S'}{e'}$. As with
  Lemma~\ref{lem:lvars-independence}, the interesting cases are for
  the {\sc E-New} and {\sc E-Put} rules, and since $\config{S'}{e'}
  \neq \error$, we do not need to consider the {\sc E-Put-Err} rule.
  See Section~\ref{section:lvars-clash-proof}.
\end{proof}

\subsection{Error Preservation lemma}

Lemma~\ref{lem:lvars-error-preservation}, Error Preservation, says
that if a configuration $\config{S}{e}$ steps to $\error$, then
evaluating $e$ in the context of some larger store will also result in
$\error$.

\LVarsLemErrorPreservation
\begin{proof}
  Suppose $\config{S}{e} \parstepsto \error$ and
  $\leqstore{S}{S'}$. We are required to show that $\config{S'}{e}
  \parstepsto \textup{\error}$.

  By inspection of the operational semantics, the only rule by which
  $\config{S}{e}$ can step to $\error$ is {\sc E-Put-Err}.  Hence $e =
  \putexp{l}{d_2}$.  From the premises of {\sc E-Put-Err}, we have
  that $S(l) = d_1$.  Since $\leqstore{S}{S'}$, it must be the case
  that $S'(l) = d'_1$, where $d_1 \userleq d'_1$.  Since
  $\userlub{d_1}{d_2} = \top$, we have that $\userlub{d'_1}{d_2} =
  \top$.  Hence, by {\sc E-Put-Err}, $\config{S'}{\putexp{l}{d_2}}
  \parstepsto \error$, as we were required to show.
\end{proof}

\subsection{Confluence lemmas}\label{subsection:lvars-confluence}

Lemma~\ref{lem:lvars-strong-local-confluence}, the Strong Local
Confluence lemma, says that if a configuration $\conf$ can step to
configurations $\conf_a$ and $\conf_b$, then there exists an
configuration $\conf_c$ that $\conf_a$ and $\conf_b$ can each reach in
at most one step, modulo a permutation on the locations in $\conf_b$.
Lemmas~\ref{lem:lvars-strong-one-sided-confluence}
and~\ref{lem:lvars-strong-confluence} then generalize that result to
arbitrary numbers of steps.

The structure of this part of the proof differs slightly from the
Budimli\'c \etal~determinism proof for Featherweight CnC.  Budimli\'c
\etal~prove a \emph{diamond} property, in which $\conf_a$ and
$\conf_b$ each step to $\conf_c$ in \emph{exactly} one step.  They
then get a property like Lemma~\ref{lem:lvars-strong-local-confluence}
as an immediate consequence of the diamond property, by choosing $i =
j = 1$.  But a true diamond property with exactly one step ``on each
side of the diamond'' is stronger than we need here, and, in fact,
does not hold for $\lambdaLVar$; so, instead, I prove the weaker ``at
most one step'' property directly.

\LVarsLemStrongLocalConfluence
\begin{proof}
  Since the original configuration $\conf$ can step in two different
  ways, its expression decomposes into redex and context in two
  different ways: $\conf = \config{S}{\evalctxt{E_a}{e_{a_1}}} =
  \config{S}{\evalctxt{E_b}{e_{b_1}}}$, where $\evalctxt{E_a}{e_{a_1}}
  = \evalctxt{E_b}{e_{b_1}}$, but $E_a$ and $E_b$ may differ and
  $e_{a_1}$ and $e_{b_1}$ may differ.  We can then apply the
  Lemma~\ref{lem:lvars-locality} (Locality) lemma; at a high level, it
  shows that $e_{a_1}$ and $e_{b_1}$ can step independently within
  their contexts.

  The proof is then by a double case analysis on the rule of the
  reduction semantics by which $e_{a_1}$ steps and by which $e_{b_1}$
  steps.  In order to combine the results of the two steps, the proof
  makes use of Lemma~\ref{lem:lvars-independence} (Independence).  The
  most interesting case is that in which both expressions step by the
  {\sc E-New} rule and they allocate locations with the same name.  In
  that case, we can use Lemma~\ref{lem:lvars-permutability}
  (Permutability) to rename locations so as not to conflict.  See
  Section~\ref{section:lvars-strong-local-confluence-proof}.
\end{proof}

\LVarsLemStrongOneSidedConfluence
\begin{proof}
  By induction on $m$; see
  Section~\ref{section:lvars-strong-one-sided-confluence-proof}.
\end{proof}

\LVarsLemStrongConfluence
\begin{proof}
  By induction on $n$; see
  Section~\ref{section:lvars-strong-confluence-proof}.
\end{proof}

\LVarsLemConfluence

\subsection{Determinism theorem}

Finally, the determinism theorem, Theorem~\ref{thm:lvars-determinism},
is a direct result of Lemma~\ref{lem:lvars-confluence}:

\LVarsThmDeterminism

\subsection{Discussion: termination}

I have followed Budimli\'c \etal~\cite{CnC} in treating
\emph{determinism} separately from the issue of \emph{termination}.
Yet one might legitimately be concerned that in $\lambdaLVar$, a
configuration could have both an infinite reduction path and one that
terminates with a value.  Theorem~\ref{thm:lvars-determinism} says
that if two runs of a given $\lambdaLVar$ program reach configurations
where no more reductions are possible, then they have reached the same
configuration.  Hence Theorem~\ref{thm:lvars-determinism} handles the
case of \emph{deadlocks} already: a $\lambdaLVar$ program can deadlock
(\eg, with a blocked $\GET$), but it will do so deterministically.

However, Theorem~\ref{thm:lvars-determinism} has nothing to say about
\emph{livelocks}, in which a program reduces infinitely.  It would be
desirable to have a \emph{consistent termination} property which would
guarantee that if one run of a given $\lambdaLVar$ program terminates
with a non-$\error$ result, then every run will.  I conjecture (but do
not prove) that such a consistent termination property holds for
$\lambdaLVar$.  Such a property could be paired with
Theorem~\ref{thm:lvars-determinism} to guarantee that if one run of a
given $\lambdaLVar$ program terminates in a non-$\error$ configuration
$\sigma$, then every run of that program terminates in $\sigma$.  (The
``non-$\error$ configuration'' condition is necessary because it is
possible to construct a $\lambdaLVar$ program that can terminate in
$\error$ on some runs and diverge on others.  By contrast, the
existing determinism theorem does not have to treat $\error$
specially.)
