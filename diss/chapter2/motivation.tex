\section{Motivating example: a parallel, pipelined graph computation}\label{s:lvars-motivation}

What applications motivate going beyond IVars and FIFO streams?
Consider applications in which independent subcomputations contribute
information to shared mutable data structures that change
monotonically.  Hindley-Milner type inference is one example: in a
parallel type-inference algorithm, each type variable monotonically
acquires information through unification (which can be represented as
a lattice). Likewise, in control-flow analysis, the \emph{set} of
locations to which a variable refers monotonically \emph{shrinks}.  In
logic programming, a parallel implementation of conjunction might
asynchronously add information to a logic variable from different
threads.

To illustrate the issues that arise in computations of this nature,
consider a specific problem, drawn from the domain of \emph{graph
  algorithms}, where issues of ordering create a tension between
parallelism and determinism:
\begin{quote}
  In a directed graph, find the connected component containing a
  vertex $v$, and compute a (possibly expensive) function $f$ over all
  vertices in that component, making the set of results available
  asynchronously to other computations.
\end{quote}
For example, in a directed graph representing user profiles on a
social network and the connections between them, where $v$ represents
a particular profile, we might wish to find all (or the first $k$
degrees of) profiles connected to $v$, then map a function $f$ over
each profile in that set in parallel.

This is a challenge problem for deterministic-by-construction parallel
programming: existing parallel solutions (such as, for instance, the
parallel breadth-first graph traversal implementation from the
Parallel Boost Graph Library~\cite{bfs-pbgl}) often traverse the
connected component in a nondeterministic order (even though the
outcome of the program---that is, the final connected component---is
deterministic), and IVars and streams provide no obvious aid.  For
example, IVars cannot accumulate sets of visited nodes, nor can they
be used as ``mark bits'' on visited nodes, since they can only be
written once and not tested for emptiness.  Streams, on the other
hand, impose an excessively strict ordering for computing the
unordered \emph{set} of vertex labels in a connected component.
Before turning to an LVar-based approach, though, let's consider
whether a purely functional (and therefore deterministic by
construction) program can meet the specification.

\subsection{A purely functional attempt}

\begin{figure}
  \lstinputlisting{chapter2/code/bfs_pure.hs}
  \caption{A purely functional Haskell program that maps the
    \lstinline|analyze| function over the connected component of the
    \lstinline|profiles| graph that is reachable from the node
    \lstinline|profile0|.  Although component discovery proceeds in
    parallel, results of \lstinline|analyze| are not asynchronously
    available to other computations, inhibiting pipelining.}
  \label{f:bfs-pure}
\end{figure}

Figure~\ref{f:bfs-pure} gives a Haskell implementation of a
\emph{level-synchronized} breadth-first graph traversal that finds the
connected component reachable from a starting vertex.  Nodes at
distance one from the starting vertex are discovered---and set-unioned
into the connected component---before nodes of distance two are
considered.  Level-synchronization is a popular strategy for parallel
breadth-first graph traversal (in fact, the aforementioned
implementation from the Parallel Boost Graph Library~\cite{bfs-pbgl}
uses level-synchronization), although it necessarily sacrifices some
parallelism for determinism: parallel tasks cannot continue
discovering nodes in the component before synchronizing with all other
tasks at a given distance from the start.

Unfortunately, the code given in Figure~\ref{f:bfs-pure} does not
quite implement the problem specification given above.  Even though
connected-component discovery is parallel, members of the output set
do not become available to other computations until component
discovery is \emph{finished}, limiting parallelism.  We could
manually push the @analyze@ invocation inside the @bf_traverse@
function, allowing the @analyze@ computation to start sooner, but
doing so just passes the problem to the downstream consumer, unless
we are able to perform a heroic whole-program fusion.

If @bf_traverse@ returned a list, lazy evaluation could make it
possible to \emph{stream} results to consumers incrementally.  But
since it instead returns a \emph{set}, such pipelining is not
generally possible: consuming the results early would create a proof
obligation that the determinism of the consumer does not depend on the
order in which results emerge from the producer.\footnote{As intuition
  for this idea, consider that purely functional set data structures,
  such as Haskell's \lstinline|Data.Set|, are typically represented
  with balanced trees.  Unlike with lists, the structure of the tree
  is not known until all elements are present.}

A compromise would be for @bf_traverse@ to return a list of
``level-sets'': distance one nodes, distance two nodes, and so on.
Thus level-one results could be consumed before level-two results are
ready.  Still, at the granularity of the individual level-sets, the
problem would remain: within each level-set, one would not be able to
launch all instances of @analyze@ and asynchronously use those results
that finished first.  Moreover, we would still have to contend with
the previously-mentioned difficulty of separating producers and
consumers when expressing producer-consumer computations when using
pure programming with futures~\cite{monad-par}.

\subsection{An LVar-based solution}

Consider a version of @bf_traverse@ written using a programming model
with limited effects that allows \emph{any} data structure to be
shared among tasks, including sets and graphs, so long as that data
structure grows monotonically.  Consumers of the data structure may
execute as soon as data is available, enabling pipelining, but may
only observe irrevocable, monotonic properties of it. This is possible
with a programming model based on LVars.  In the rest of this chapter,
I will formally introduce LVars and the $\lambdaLVar$ language and
give a proof of determinism for $\lambdaLVar$.  Then, in
Chapter~\ref{ch:quasi}, I will extend the basic LVars model with
additional features that make it easier to implement parallel graph
traversal algorithms and other fixpoint computations, and I will
return to @bf_traverse@ and show how to implement a version of it
using LVars that makes pipelining possible and truly satisfies the
above specification.
