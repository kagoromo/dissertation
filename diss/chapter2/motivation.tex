\section{Motivating Example: A Parallel, Pipelined Graph Computation}\label{s:lvars-motivation}

What applications motivate going beyond IVars and FIFO streams?

Consider applications in which independent subcomputations contribute
information to shared data structures that change monotonically with
respect to some order.  Hindley-Milner type inference is one example:
in a parallel type-inference algorithm, each type variable
monotonically acquires information through unification (which can be
represented as a lattice). Likewise, in control-flow analysis, the
\emph{set} of locations to which a variable refers monotonically
\emph{shrinks}.  In logic programming, a parallel implementation of
conjunction might asynchronously add information to a logic variable
from different threads.

To illustrate the issues that arise in computations of this nature, we
consider a specific problem, drawn from the domain of {\em graph
  algorithms}, where issues of ordering create a tension between
parallelism and determinism:
\begin{itemize}
\item 
  In a directed graph, find the connected component containing a
  vertex $v$, and compute a (possibly expensive) function $f$ over all
  vertices in that component, making the set of results available
  asynchronously to other computations.
\end{itemize}
For example, in a directed graph representing user profiles on a
social network and the connections between them, where $v$ represents
a particular profile, we might wish to find all (or the first $k$
degrees of) profiles connected to $v$, then map a function $f$ over
each profile in that set in parallel.

This is a challenge problem for deterministic-by-construction parallel
programming: existing parallel solutions~\cite{bfs-pbgl} often use a
nondeterministic traversal of the connected component (even though the
final connected component is deterministic), and IVars and streams
provide no obvious aid.  For example, IVars cannot accumulate sets of
visited nodes, nor can they be used as ``mark bits'' on visited nodes,
since they can only be written once and not tested for emptiness.
Streams, on the other hand, impose an excessively strict ordering for
computing the unordered \emph{set} of vertex labels in a connected
component.  Yet before considering \emph{new} mechanisms, we must also
ask if a purely functional program can do the job.

\paragraph{A purely functional attempt}

Figure~\ref{f:bfs-pure} gives a Haskell implementation of a
\emph{level-synchronized} breadth-first graph traversal that finds the
connected component reachable from a starting vertex.  Nodes at
distance one from the starting vertex are discovered---and set-unioned
into the connected component---before nodes of distance two are
considered.  Level-synchronization is a popular strategy for parallel
breadth-first graph traversal (see, for instance, the Parallel Boost
Graph Library \cite{bfs-pbgl}), although it necessarily sacrifices
some parallelism for determinism: parallel tasks cannot continue
discovering nodes in the component (racing to visit neighbor vertices)
before synchronizing with all other tasks at a given distance from the
start.

Unfortunately, the code given in Figure~\ref{f:bfs-pure} does not
quite implement the problem specification given above.  Even though
connected-component discovery is parallel, members of the output set
do not become available to other computations until component
discovery is \emph{finished}, limiting parallelism.  We could manually
push the @analyze@ invocation inside the @bf_traverse@ function,
allowing the @analyze@ computation to start sooner, but then we push
the same problem to the downstream consumer, unless we are able to
perform a heroic whole-program fusion.  \TODO{Emphasize the following
  point: we're prevented from PIPELINING the work here because we have
  no monotonicity guarantee.  Kahn 1974 makes a point of talking about
  how in KPNs, monotonicity enables both pipelining and determinism.}

If @bf_traverse@ returned a list, lazy evaluation could make it
possible to \emph{stream} results to consumers incrementally.  But
since it instead returns a \emph{set}, such pipelining is not
generally possible: consuming the results early would create a proof
obligation that the determinism of the consumer does not depend on the
order in which results emerge from the producer.\footnote{As intuition
  for this idea, consider that purely functional set data structures,
  such as Haskell's \lstinline|Data.Set|, are typically represented
  with balanced trees.  Unlike with lists, the structure of the tree
  is not known until all elements are present.}

A compromise would be for @bf_traverse@ to return a list of
``level-sets'': distance one nodes, distance two nodes, and so on.
Thus level-one results could be consumed before level-two results are
ready.  Still, the problem would remain: within each level-set, we
cannot launch all instances of @analyze@ and asynchronously use those
results that finish \emph{first}.  Moreover, we would still have to
contend with the previously-mentioned difficulty of separating
producers and consumers when expressing producer-consumer computations
using pure programming with futures~\cite{monad-par}.

\begin{figure}
  \lstinputlisting{chapter2/code/bfs_pure.hs}
  \caption{\footnotesize A purely functional Haskell program that maps
    the \lstinline|analyze| function over the connected component of
    the \lstinline|profiles| graph that is reachable from the node
    \lstinline|profile0|.  Although component discovery proceeds in
    parallel, results of \lstinline|analyze| are not asynchronously
    available to other computations, inhibiting pipelining.}
  \label{f:bfs-pure}
\end{figure}

\paragraph{Our solution}

Suppose that we could write a breadth-first traversal in a programming
model with limited effects that allows \emph{any} data structure to be
shared among tasks, including sets and graphs, so long as that data
structure grows \emph{monotonically}.  Consumers of the data structure
may execute as soon as data is available, but may only observe
irrevocable, monotonic properties of it. This is possible with a
programming model based on LVars.

\lk{The following bit needs to go somewhere, but I'm not sure where.}
I define a \emph{monotonic data structure} to be a data structure that
meets these criteria: it is a data structure to which information can
only be added and never removed, and for which the order in which
information is added is not observable.
