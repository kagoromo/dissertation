\begin{proof}
  Suppose $\conf \ctxstepsto \conf_a$ and $\conf \ctxstepsto \conf_b$.
  We have to show that there exist $\conf_c, i, j, \pi$ such that
  $\conf_a \ctxstepsto^i \conf_c$ and $\pi(\conf_b) \ctxstepsto^j
  \pi(\conf_c)$ and $i \leq 1$ and $j \leq 1$.

  By inspection of the operational semantics, it must be the case that
  $\conf$ steps to $\conf_a$ by the {\sc E-Eval-Ctxt} rule.  Let
  $\conf = \config{S}{\evalctxt{E_a}{e_{a_1}}}$ and let $\conf_a =
  \config{S_a}{\evalctxt{E_a}{e_{a_2}}}$.

  Likewise, it must be the case that $\conf$ steps to $\conf_b$ by the
  {\sc E-Eval-Ctxt} rule.  Let $\conf =
  \config{S}{\evalctxt{E_b}{e_{b_1}}}$ and let $\conf_b =
  \config{S_b}{\evalctxt{E_b}{e_{b_2}}}$.

  Note that $\conf = \config{S}{\evalctxt{E_a}{e_{a_1}}} =
  \config{S}{\evalctxt{E_b}{e_{b_1}}}$, and so
  $\evalctxt{E_a}{e_{a_1}} = \evalctxt{E_b}{e_{b_1}}$, but $E_a$ and
  $E_b$ may differ and $e_{a_1}$ and $e_{b_1}$ may differ.

  Since $\config{S}{\evalctxt{E_a}{e_{a_1}}} \ctxstepsto
  \config{S_a}{\evalctxt{E_a}{e_{a_2}}}$ and
  $\config{S}{\evalctxt{E_b}{e_{b_1}}} \ctxstepsto
  \config{S_b}{\evalctxt{E_b}{e_{b_2}}}$ and $\evalctxt{E_a}{e_{a_1}}
  = \evalctxt{E_b}{e_{b_1}}$, we have from
  Lemma~\ref{lem:lvars-locality} (Locality) that there exist
  evaluation contexts $E'_a$ and $E'_b$ such that:

  \begin{itemize}
  \item $\evalctxt{E'_a}{e_{a_1}} = \evalctxt{E_b}{e_{b_2}}$, and
  \item $\evalctxt{E'_b}{e_{b_1}} = \evalctxt{E_a}{e_{a_2}}$, and
  \item $\evalctxt{E'_a}{e_{a_2}} =
  \evalctxt{E'_b}{e_{b_2}}$.
  \end{itemize}

  Our approach will be to show that there exist $S', i, j, \pi$ such
  that:
  \begin{itemize}
  \item $\config{S_a}{\evalctxt{E_a}{e_{a_2}}} \ctxstepsto^i
    \config{S'}{\evalctxt{E'_a}{e_{a_2}}}$, and
  \item $\pi(\config{S_b}{\evalctxt{E_b}{e_{b_2}}}) \ctxstepsto^j
    \pi(\config{S'}{\evalctxt{E'_a}{e_{a_2}}})$.
  \end{itemize}
  Since $\evalctxt{E'_a}{e_{a_1}} = \evalctxt{E_b}{e_{b_2}}$,
  $\evalctxt{E'_b}{e_{b_1}} = \evalctxt{E_a}{e_{a_2}}$, and
  $\evalctxt{E'_a}{e_{a_2}} = \evalctxt{E'_b}{e_{b_2}}$, it suffices
  to show that:
  \begin{itemize}
  \item $\config{S_a}{\evalctxt{E'_b}{e_{b_1}}} \ctxstepsto^i
    \config{S'}{\evalctxt{E'_b}{e_{b_2}}}$, and
  \item $\pi(\config{S_b}{\evalctxt{E'_a}{e_{a_1}}}) \ctxstepsto^j
    \pi(\config{S'}{\evalctxt{E'_a}{e_{a_2}}})$.
  \end{itemize}

  From the premise of {\sc E-Eval-Ctxt}, we have that
  $\config{S}{e_{a_1}} \parstepsto \config{S_a}{e_{a_2}}$ and
  $\config{S}{e_{b_1}} \parstepsto \config{S_b}{e_{b_2}}$.  We proceed
  by case analysis on the rule by which $\config{S}{e_{a_1}}$ steps to
  $\config{S_a}{e_{a_2}}$.

  \begin{enumerate}
  \item Case {\sc E-Beta}:

    We have:
    \begin{itemize}
      \item $e_{a_1} = \app{\lam{x}{e'_a}}{v_a}$,
      \item $e_{a_2} = \subst{e'_a}{x}{v_a}$, and
      \item $S_a = S$.
    \end{itemize}

    Now, we proceed by case analysis on the rule by which
    $\config{S}{e_{b_1}}$ steps to $\config{S_b}{e_{b_2}}$:
    \begin{enumerate}
    \item Case {\sc E-Beta}:

      We have:
      \begin{itemize}
      \item $e_{b_1} = \app{\lam{x}{e'_b}}{v_b}$,
      \item $e_{b_2} = \subst{e'_b}{x}{v_b}$, and
      \item $S_b = S$.
      \end{itemize}

      Choose $S' = S$, $i = 1$, $j = 1$, and $\pi = \id$.

      We have to show that:

      \begin{itemize}
      \item $\config{S}{\evalctxt{E'_b}{e_{b_1}}} \ctxstepsto
        \config{S}{\evalctxt{E'_b}{e_{b_2}}}$, and
      \item $\config{S}{\evalctxt{E'_a}{e_{a_1}}} \ctxstepsto
        \config{S}{\evalctxt{E'_a}{e_{a_2}}}$, 
      \end{itemize}

      both of which follow immediately from $\config{S}{e_{a_1}}
      \parstepsto \config{S_a}{e_{a_2}}$ and $\config{S}{e_{b_1}}
      \parstepsto \config{S_b}{e_{b_2}}$ and {\sc E-Eval-Ctxt}.

    \item Case {\sc E-New}:

      We have:
      \begin{itemize}
      \item $e_{b_1} = \NEW$,
      \item $e_{b_2} = l$, and
      \item $S_b = \extSRaw{S}{l}{\bot}$.
      \end{itemize}

      Choose $S' = S_b$, $i = 1$, $j = 1$, and $\pi = \id$.

      We have to show that:

      \begin{itemize}
      \item $\config{S}{\evalctxt{E'_b}{e_{b_1}}} \ctxstepsto
        \config{S_b}{\evalctxt{E'_b}{e_{b_2}}}$, and
      \item
        $\config{S_b}{\evalctxt{E'_a}{e_{a_1}}} \ctxstepsto
        \config{S_b}{\evalctxt{E'_a}{e_{a_2}}}$.
      \end{itemize}

      The first of these follows immediately from $\config{S}{e_{b_1}}
      \parstepsto \config{S_b}{e_{b_2}}$ and {\sc E-Eval-Ctxt}.  For
      the second, consider that $S_b = \extSRaw{S}{l}{\bot} =
      \lubstore{S}{\store{\storebindingRaw{l}{\bot}}}$.  Furthermore, we
      know from the side condition of {\sc E-New} that $l \notin
      \dom{S}$, so $\store{\storebindingRaw{l}{\bot}}$ is non-conflicting
      with the transition $\config{S}{e_{a_1}} \parstepsto
      \config{S_a}{e_{a_2}}$, and we know that
      $\lubstore{S_a}{\store{\storebindingRaw{l}{\bot}}} \neq \topS$
      since $S_a$ is just $S$.  Therefore, by
      Lemma~\ref{lem:lvars-independence} (Independence), we have that
      $\config{\lubstore{S}{\store{\storebindingRaw{l}{\bot}}}}{e_{a_1}}
      \parstepsto
      \config{\lubstore{S_a}{\store{\storebindingRaw{l}{\bot}}}}{e_{a_2}}$.
      Hence $\config{S_b}{e_{a_1}} \parstepsto \config{S_b}{e_{a_2}}$.
      By {\sc E-Eval-Ctxt}, it follows that
      $\config{S_b}{\evalctxt{E'_a}{e_{a_1}}} \ctxstepsto
      \config{S_b}{\evalctxt{E'_a}{e_{a_2}}}$, as we were required to
      show.

    \item Case {\sc E-Put}: \TODO{}
    \item Case {\sc E-Put-Err}: \TODO{}
    \item Case {\sc E-Get}:\TODO{}
    \end{enumerate}
  \item Case {\sc E-New}:

    Now, we proceed by case analysis on the rule by which
    $\config{S}{e_{b_1}}$ steps to $\config{S_b}{e_{b_2}}$:
    \begin{enumerate}
    \item Case {\sc E-Beta}: \TODO{}
    \item Case {\sc E-New}: \TODO{}
    \item Case {\sc E-Put}: \TODO{}
    \item Case {\sc E-Put-Err}: \TODO{}
    \item Case {\sc E-Get}: \TODO{}
    \end{enumerate}
  \item Case {\sc E-Put}:

    Now, we proceed by case analysis on the rule by which
    $\config{S}{e_{b_1}}$ steps to $\config{S_b}{e_{b_2}}$:
    \begin{enumerate}
    \item Case {\sc E-Beta}: \TODO{}
    \item Case {\sc E-New}: \TODO{}
    \item Case {\sc E-Put}: \TODO{}
    \item Case {\sc E-Put-Err}: \TODO{}
    \item Case {\sc E-Get}: \TODO{}
    \end{enumerate}
  \item Case {\sc E-Put-Err}:

    Now, we proceed by case analysis on the rule by which
    $\config{S}{e_{b_1}}$ steps to $\config{S_b}{e_{b_2}}$:
    \begin{enumerate}
    \item Case {\sc E-Beta}: \TODO{}
    \item Case {\sc E-New}: \TODO{}
    \item Case {\sc E-Put}: \TODO{}
    \item Case {\sc E-Put-Err}: \TODO{}
    \item Case {\sc E-Get}: \TODO{}
    \end{enumerate}
  \item Case {\sc E-Get}:

    Now, we proceed by case analysis on the rule by which
    $\config{S}{e_{b_1}}$ steps to $\config{S_b}{e_{b_2}}$:
    \begin{enumerate}
    \item Case {\sc E-Beta}: \TODO{}
    \item Case {\sc E-New}: \TODO{}
    \item Case {\sc E-Put}: \TODO{}
    \item Case {\sc E-Put-Err}: \TODO{}
    \item Case {\sc E-Get}: \TODO{}
    \end{enumerate}
  \end{enumerate}

  \lk{I think we also still have to separately deal with cases where
    $\conf_a = \error$ or $\conf_b = \error$.}
\end{proof}
