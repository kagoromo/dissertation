\section{Related work}

The extended CvRDT model we present in this paper is based on Shapiro
\etal's work on conflict-free replicated data types
\cite{crdts,crdts-tr}, discussed in Section~\ref{s:cvrdts}.  Various
other authors
\cite{eventually-consistent-transactions,semantics-concurrent-revisions,blooml}
have used lattices as a framework for establishing formal guarantees
about eventually consistent systems and distributed programs.
Burckhardt \etal~\cite{eventually-consistent-transactions} propose a
formalism for eventual consistency based on graphs called
\emph{revision diagrams}.  Burckhardt and
Leijen~\cite{semantics-concurrent-revisions} show that revision
diagrams are semilattices, and Leijen, Burckhardt, and Fahndrich apply
the revision diagrams approach to guaranteed-deterministic concurrent
functional programming~\cite{concurrent-revisions-haskell11}. Conway
\etal's Bloom$^L$ language for distributed programming leverages the
lattice-based semantics of CvRDTs to guarantee
confluence~\cite{blooml}.  The concept of threshold queries 
comes from our previous work on the LVars model for
lattice-based deterministic parallel
programming~\cite{LVars-paper,Freeze-paper,effectzoo}.

As mentioned in Section~\ref{s:intro}, database services such as
Amazon's SimpleDB~\cite{simpledb-vogels-article} allow for both
eventually consistent and strongly consistent reads, chosen at a
per-query granularity.  Terry \etal's Pileus key-value
store~\cite{pileus} takes the idea of mixing consistency levels
further: instead of requiring the application developer to choose the
consistency level of a particular query at development time, the
system allows the developer to specify a service-level agreement that
can dynamically adapt to changing network conditions, for instance.
%%  a query might
%% be strongly consistent if it is possible to complete the query within
%% a given response time and otherwise fall back to a weaker consistency
%% guarantee.
However, we are not aware of previous work on using
lattice-based data structures as a foundation for both eventually
consistent and strongly consistent queries.

