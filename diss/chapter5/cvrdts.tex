\section{Background: CvRDTs and eventual consistency}\label{s:cvrdts}

Shapiro \etal~\cite{crdts,crdts-tr} define an \emph{eventually consistent}
object as one that
meets three conditions.  One of these conditions is
 the property of \emph{convergence}: all correct
replicas of an object at which the same updates have been delivered
eventually have equivalent state.  The other two conditions
are \emph{eventual delivery}, meaning that all replicas receive all
update messages, and \emph{termination}, meaning that all method
executions terminate (we discuss methods in more detail below).

\rn{But we kind of break this termination property!  The ``microops''
  of individual ``t'' methods complete immediately, but at the polling
layer the ``macro-op'' can block forever.  Do we need to say something
about how that is ok because the merges keep happening?}

\lk{I think that this is actually just fine, because all our proofs
  are about the inner layer, *not* the polling layer.  The argument
  that we are making is that if the inner layer behaves in the way
  that we prove it does, then we get determinism by layering a polling
  layer on top of it -- and the polling layer does not have to know
  anything about threshold sets or lattices or anything like that, so
  it's very conceptually simple.}

Shapiro \etal~further define a \emph{strongly eventually consistent}
(SEC) object as one that is eventually consistent and, in addition to
being merely convergent, is \emph{strongly convergent}, meaning that
correct replicas at which the same updates have been delivered have
equivalent state.\footnote{
  Strong eventual consistency is not to be confused with strong
  consistency: it is the combination of eventual consistency and
  strong convergence.
  Contrast with ordinary convergence, in
  which replicas only \emph{eventually} have equivalent state.
  In a strongly convergent object, knowing that the same
  updates have been delivered to all correct replicas is sufficient to
  ensure that those replicas have equivalent state, whereas in an object
  that is merely convergent, there might be some further delay before
  all replicas agree.}
A \emph{conflict-free replicated data type} (CRDT), then,
is a data type (\ie, a specification for an object) satisfying certain
conditions that are sufficient to guarantee that the object is SEC.
(The term ``CRDT'' is used interchangeably to mean a specification for
an object, or an object meeting that specification.)

There are two ``styles'' of specifying a CRDT:
\emph{state-based}, also known as \emph{convergent}\footnote{There is
  a potentially misleading terminology overlap here: the definitions
  of convergence and strong convergence above pertain not only to
  CvRDTs (where the C stands for ``Convergent''), but to \emph{all}
  CRDTs.}; or \emph{operation-based} (or ``op-based''), also known as
\emph{commutative}.  CRDTs specified in the state-based style are
called \emph{convergent replicated data types}, abbreviated
\emph{CvRDTs}, while those specified in the op-based style are
called \emph{commutative replicated data types}, abbreviated
\emph{CmRDTs}.
Of the two styles, we focus on the CvRDT style in this paper because
CvRDTs are lattice-based data structures and therefore amenable to
threshold queries described in
Section~\ref{s:threshold-reads}---although,
as Shapiro \etal~show, CmRDTs can emulate CvRDTs and vice
versa.

\subsection{State-based objects}

The Shapiro \etal~model specifies a \emph{state-based object} as a tuple $(S, s^0,
q, u, m)$, where $S$ is a set of states, $s^0$ is the initial state,
$q$ is the \emph{query method}, $u$ is the \emph{update method}, and
$m$ is the \emph{merge method}.  Objects are replicated across some
finite number of processes, with one replica at each process,
and each replica begins in the initial state $s^0$.  The state
of a local replica may be queried via the method $q$ and updated via
the method $u$.  Methods execute locally, at a single replica, but the
merge method $m$ can merge the state from a remote replica with the
local replica.  The model assumes that each replica
sends its state to the other replicas infinitely often,
and that eventually every
update reaches every replica, whether directly or indirectly.

The assumption that replicas send their state to one another
``infinitely often'' refers not to the \emph{frequency} of these state
transmissions; rather; it says that, regardless of what event (such as
an update, via the $u$ method) occurs at a replica, a state
transmission is guaranteed to occur after that event.  We can
therefore conclude that all updates eventually reach all replicas in a
state-based object, meeting the ``eventual delivery'' condition
discussed above.  However, we still have no guarantee of strong
convergence or even convergence.  This is where Shapiro \etal's notion
of a CvRDT comes in: a state-based object that meets the criteria for
a CvRDT is guaranteed to have the strong-convergence property.

A \emph{state-based} or \emph{convergent} replicated data type (CvRDT)
is a state-based object equipped with a partial order $\leq$, written
as a tuple
$(S, \leq, s^0, q, u, m)$, that has the following properties:
\begin{itemize}
\item $S$ forms a join-semilattice ordered by $\leq$.
\item The merge method $m$ computes the join of two
  states with respect to $\leq$.
\item State is \emph{inflationary} across updates: if $u$ updates a
  state $s$ to $s'$, then $s \leq s'$.
\end{itemize}
Shapiro \etal~show that a state-based object that meets the criteria
for a CvRDT is strongly convergent.  Therefore, given the eventual
delivery guarantee that all state-based objects have, and given an
additional assumption that all method executions terminate, a
state-based object that meets the criteria for a CvRDT is
SEC~\cite{crdts}.

\subsection{Discussion: the need for inflationary updates}

\lk{I think this is true, but I want someone else to check my
  reasoning.}
Although CvRDT updates are required to be inflationary, we note that
it is not clear that inflationary updates are necessarily required for
convergence.  Consider, for example, a scenario in which replicas 1
and 2 both have the state $\{a, b\}$. Replica 1 updates its state to
$\{a\}$, a non-inflationary update, and then sends its updated state
to replica 2.  Replica 2 merges the received state $\{a\}$ with $\{a,
b\}$, and its state remains $\{a, b\}$. Then replica 2 sends its state
back to replica 1; replica 1 merges $\{a, b\}$ with $\{a\}$, and its
state becomes $\{a, b\}$.  The non-inflationary update has been lost,
and was, perhaps, nonsensical---but the replicas are nevertheless
convergent.

However, once we introduce threshold queries of CvRDTs, as
we will do in the following section, inflationary updates become
\emph{necessary} for the determinism of threshold queries.  This is
because a non-inflationary update could cause a threshold query that
had been unblocked to block again, and so arbitrary interleaving of
non-inflationary writes and threshold queries would lead to
nondeterministic behavior.  Therefore the requirement that updates be
inflationary will not only be sensible, but actually crucial.

