\chapter{Related Work\label{ch:related}} % 6

\lk{Following stuff is from my proposal.}

Work on deterministic parallel programming models is long-standing.
As discussed in Section~\ref{s:intro}, the LVars model builds on long
traditions of work on parallel programming models based on
monotonically-growing shared data structures, and it provides a
framework for generalizing and unifying these existing approaches.  In
this section I describe some more recent contributions to the
literature.

As we have seen, what deterministic parallel programming models have
in common is that they all must do something to restrict access to
mutable state shared among concurrent computations so that schedule
nondeterminism cannot be observed.  Depending on the model,
restricting access to shared mutable state might involve disallowing
sharing entirely \cite{dph}, only allowing single assignments to
shared references \cite{Tesler-1968, IStructures, CnC}, allowing
sharing only by a limited form of message passing \cite{Kahn-1974},
ensuring that concurrent accesses to shared state are disjoint
\cite{dpj-oopsla}, resolving conflicting updates after the fact
\cite{concurrent-revisions-haskell11}, or some combination of these
approaches.  These constraints can be imposed at the language or API
level, within a type system, or at runtime.

\subsection{Deterministic Parallel Java (DPJ)}

DPJ \cite{dpj-oopsla, dpj-hotpar09} is a deterministic language
consisting of a system of annotations for Java code.  A sophisticated
region-based type system ensures that a mutable region of the heap is,
essentially, passed linearly to an exclusive writer, thereby ensuring
that the state accessed by concurrent threads is disjoint.  DPJ does,
however, provide a way to unsafely assert that operations commute with
one another (using the @commuteswith@ form) to enable concurrent
mutation.

The LVars model differs from DPJ in that it allows overlapping shared
state between threads as the default.  Moreover, since LVar effects
are already commutative, we avoid the need for @commuteswith@
annotations.  Finally, it is worth noting that while in DPJ,
commutativity annotations have to appear in application-level code, in
LVish only the data-structure author needs to write trusted code. The
application programmer can run untrusted code that still enjoys a
(quasi-)determinism guarantee, because only (quasi-)deterministic
programs can be expressed as LVish @Par@ computations.

More recently, Bocchino \etal~\cite{dpj-popl} proposed a type and
effect system that allows for the incorporation of nondeterministic
sections of code in DPJ.  The goal here is different from ours: while
they aim to support \emph{intentionally} nondeterministic computations
such as those arising from optimization problems like branch-and-bound
search, the quasi-determinism in LVish arises as a result of schedule
nondeterminism.

\subsection{FlowPools}

Prokopec \etal~\cite{flowpools} recently proposed a data structure
with an API closely related to LVars extended with freezing and
handlers: a FlowPool is a bag (that is, a multiset) that allows
concurrent insertions but forbids removals, a @seal@ operation that
forbids further updates, and combinators like @foreach@ that invoke
callbacks as data arrives in the pool.  To retain determinism, the
@seal@ operation requires explicitly passing the expected bag
\emph{size} as an argument, and the program will raise an exception if
the bag goes over the expected size.

While this interface has a flavor similar to that of LVars, it lacks
the ability to detect quiescence, which is crucial for expressing
algorithms like graph traversal, and the @seal@ operation is awkward
to use when the structure of data is not known in advance.  By
contrast, the @freeze@ operation on LVars does not require such
advance knowledge, but moves the model into the realm of
quasi-determinism.  Another important difference is the fact that
LVars are \emph{data structure-generic}: both our formalism and our
library support an unlimited collection of data structures, whereas
FlowPools are specialized to bags.

\subsection{Concurrent Revisions}

The Concurrent Revisions (CR)~\cite{concurrent-revisions-haskell11}
programming model uses isolation types to distinguish regions of the
heap shared by multiple mutators.  Rather than enforcing exclusive
access in the style of DPJ, CR clones a copy of the state for each
mutator, using a deterministic ``merge function'' for resolving
conflicts in local copies at join points.

In CR, variables can be annotated as being shared between a ``joiner''
thread and a ``joinee'' thread.  Unlike the least-upper-bound writes
of LVars, CR merge functions are \emph{not} necessarily commutative;
indeed, the default CR merge function is ``joiner wins''.  Determinism
is enforced by the programming model allowing the programmer to
specify which of two writing threads should prevail, regardless of the
order in which those writes arrive, and the states that a shared
variable can take on need not form a lattice.  Still, semilattices
turn up in the metatheory of CR: in particular, Burckhardt and
Leijen~\cite{semantics-concurrent-revisions} show that, for any two
vertices in a CR revision diagram, there exists a \emph{greatest
  common ancestor} state that can be used to determine what changes
each side has made---an interesting duality with our model (in which
any two LVar states have a lub).\lk{TODO: Write to them and ask them
  if it's a join-semilattice or a meet-semilattice!}

Although versioned variables in CR could model lattice-based data
structures---if they used least upper bound as their merge function
for conflicts---the programming model nevertheless differs from the
LVars model in that effects only become visible at the end of parallel
regions, as opposed to the asynchronous communication within parallel
regions that the LVars model allows.  This semantics precludes the use
of traditional lock-free data structures for representing versioned
variables.

\subsection{Bloom and Bloom$^L$}

The Bloom language for distributed database programming guarantees
eventual consistency for distributed data collections that are updated
monotonically.  The initial formulation of Bloom~\cite{bloom-cidr} had
a notion of monotonicity based on set inclusion, analogous to the
store ordering used in the proof of determinism for the (IVar-based)
CnC system~\cite{CnC}.  More recently, Conway \etal~\cite{blooml}
generalized Bloom to a more flexible lattice-parameterized system,
Bloom$^L$, in a manner analogous to our generalization from IVars to
LVars.  Bloom$^L$ combines ideas from CRDTs with \emph{monotonic
  logic}, resulting in a lattice-parameterized, confluent language
that is a close relative of LVish.  A monotonicity analysis pass rules
out programs that would perform non-monotonic operations on
distributed data collections, whereas in the LVars model, monotonicity
is enforced by the API presented by LVars.  Another difference between
Bloom($^L$) and the LVars model is that the former does not have a
notion of quasi-determinism.\lk{TODO: make sure this is the case---I
  think there might be some sort of way to unsafely peek in Bloom, but
  it's ruled out by the CALM analysis.}  Finally, since LVish is
implemented as a Haskell library (whereas Bloom($^L$) is implemented
as a domain-specific language embedded in Ruby), we can rely on
Haskell's static type system for fine-grained effect tracking and
monadic encapsulation of LVar effects.

\lk{Following stuff is from the FHPC paper, and might get axed.}

\if{0}
Work on deterministic parallel programming
models is long-standing; some milestones include:

\begin{itemize}
\item 1968: {\em single-assignment} parallel languages \cite{Tesler-1968};
\item 1974: Kahn process networks \cite{Kahn-1974};
\item 1980s: data-flow computing \cite{IStructures}, data-parallel
              languages \cite{Sabot_Paralation};
\item 1990s: continued progress on parallel functional programming 
      \cite{ph,blelloch_something,hammond_par_90s};
\item 2000s: introduction of type systems powerful enough to track
  and restrict effects in imperative programs \cite{par_by_default};
  new data-parallel models such as MapReduce \cite{MapReduce}.
\end{itemize}
\fi{}

Work on deterministic parallel programming
models is long-standing.
In addition to the single-assignment and KPN models already
discussed, here we consider a few recent contributions to the literature.

\paragraph{Deterministic Parallel Java (DPJ)}

% As we have presented it, $\lambdaLVar$ enforces single assignment
% dynamically, but it is also compatible with static techniques, such as
% affine type systems, to enforce that LVar locations are handled
% linearly.
% \lk{I don't think we should say this!  Affine types, linear types,
%   etc., would be fine for enforcing single assignment \emph{if} that's
%   all we had to do, but I'm pretty sure they can't enforce monotonic
%   multiple assignment; you would need something fancier for that.}
%   Moreover, in the presence of linearity guarantees it is
% possible to remove the restrictions on $\GET$ and $\PUT$.  The proof
% of determinism would rest on the fact that no other reduction would be
% concurrently possible on a linear location.\rn{Lindsey -- what do you
%   think of that?  Fair statement?  Tweak?}
% \lk{Wait.  An affine type system is enough to enforce single
%   assignment, but how would it be enough to enforce monotonic multiple
%   assignment?  Moreover, even if the affine type system can enforce
%   single assignment in a particular thread, how would it prevent
%   different threads from writing to the same location?  I don't know
%   how you would enforce that prior to runtime except by serializing
%   the whole program.}

DPJ \cite{dpj-hotpar09} is a deterministic language
consisting of a system of annotations for Java code.  A sophisticated
region-based type system ensures that a mutable region of the heap is,
essentially, passed linearly to an exclusive writer.
While a linear type system or region system like that of DPJ
could be used to enforce single assignment statically, accommodating $\lambdaLVar$'s
semantics would involve parameterizing the
type system by the user-specified lattice.
%% ---a direction of
%% inquiry that we leave for future work.

DPJ also
provides a way to unsafely assert that operations commute with one another (using the @commuteswith@ form) to enable
concurrent mutation.  However, DPJ does not provide direct support for
modeling message-passing (\eg, KPNs) or asynchronous communication
within parallel regions.  
\rn{DOUBLE CHECK THIS.}
Finally, a key difference between the $\lambdaLVar$ model and DPJ 
is that $\lambdaLVar$ retains determinism by restricting {\em
  what} can be read or written, rather than by restricting 
%% {\em who} can read or write.
the semantics of reads and writes themselves.


%% We believe that DPJ's techniques are largely orthogonal to, and
%% compatible with, the work we have presented on $\lambdaLVar$.
%
%% \new{For example, while $\lambdaLVar$, like most systems based on
%%   $IVars$, relies on dynamic enforcement of $\PUT$ errors, it would
%%   also be possible to use DPJ-style region annotations to remove the
%%   need for dynamic checks.}

% \rn{One distinction vs. DPJ and other strictly nested or BSP models
%   is that lambda-par models programs in the asynchronous/dataflow
%   tradition, which is getting a LOT of traction recently... Akka
%   dataflow concurrency, Chapel, etc etc, making it important to have
%   semantic foundations for this class of programming models.}

% \rn{I think something we need to argue is that we are more EXPRESSIVE
%   than DPJ.  DPJ is in some sense not as well motivated.  They want to
%   make certain restricted imperative programs deterministic (mostly
%   divide-and-conquer it seems), but what's the strong motivation when
%   you can just write those in a purely functional way?  We, on the
%   other hand, are talking about something DOES NOT come for free with
%   using Haskell.. communicating, asynchronous agents.}

% \note{Essentially the OPPOSITE of $\lambdaLVar$, focused primarily on
%   inplace operations on disjoint memory, no communication and
%   synchronization within parallel regions.  We focus on which building
%   up a class of safe operations on shared state.}


\paragraph{Concurrent Revisions}
The Concurrent Revisions (CR) 
% \cite{concurrent-revisions-oopsla} 
\cite{concurrent-revisions-haskell11}
%http://research.microsoft.com/pubs/151805/PrettierConcurrency-Haskell2011.pdf
programming model uses isolation types to distinguish regions of the heap shared
by multiple mutators.  Rather than enforcing exclusive access, CR
clones a copy of the state for each mutator, using a
deterministic policy for resolving conflicts in local copies.
% (hence the name of the system, which draws inspiration from version control systems).  
%
The management of shared variables in CR is tightly coupled to a
fork-join control structure, and the implementation of these variables
is similar to reduction variables in other languages (\eg, Cilk
{\em hyperobjects}).
% \cite{cilk-hyperobjects}).
%
CR charts an important new area in the deterministic-parallelism
design space, but one that differs significantly from
% explores a very different part of the design space that
$\lambdaLVar$.  CR could be used to model similar types of data
structures---if versioned variables used least upper bound as their merge function
for conflicts---but effects would only become visible at the end of
parallel regions, rather than $\lambdaLVar$'s asynchronous communication within
parallel regions.

% , targeting imperative multiple-assignment and {\em not}
% asynchronous, data-driven communication.

% \note{Nevertheless, CR could be used to build a practical
%   implementation of a $\lambdaLVar$-like programming model by defining
%   LVars as shared CR versioned variables the merge operation as lub.}

\paragraph{Bloom and Bloom$^L$}
\lk{Augh, I really want to mention CRDTs, but no room!}
% Turning our attention to research from the distributed systems community,
% Bloom \cite{bloom-website} is a declarative, domain-specific language 
% embedded in Ruby, designed for expressing order-independent, distributed
% computations that reach \emph{eventual consistency} \cite{eventual-consistency}.

%% In the database literature, there is a body of work on {\em monotonic
%%   logic} that has recently been leveraged by the {Bloom language} 
%% to achieve \emph{eventual consistency} in distributed programs.  
% Bloom \cite{bloom-website} is a language, embedded in Ruby, for programming such systems.
%
In the distributed systems literature, \emph{eventually
  consistent} systems \cite{eventual-consistency} 
leverage the idea of monotonicity to guarantee that,
for instance, nodes in a distributed database eventually agree.
The Bloom language for distributed database programming \cite{bloom-cidr} guarantees eventual consistency for
distributed data collections that are updated monotonically.
The initial formulation of Bloom 
had a notion of monotonicity based on \emph{set containment},
analogous to the store ordering for single-assignment languages given in 
Definition~\ref{def:leqstore-cnc}.
However, recent work
by Conway \etal~\cite{blooml} generalizes Bloom to a more flexible
lattice-parameterized system, Bloom$^L$, in a manner analogous
to our generalization from IVars to LVars.
Bloom$^L$ comes with a library of built-in lattice types and also
allows for users to implement their own lattice types as Ruby classes.
Although Conway \etal~do not give a proof of eventual consistency for
Bloom$^L$, our determinism result for $\lambdaLVar$ suggests that
%% generalizing from a set-containment-based 
%% notion of monotonicity to
%% one parameterized by a user-defined lattice 
their generalization is indeed safe.
Moreover, although the goals of Bloom$^L$
% (being a pure language with no shared state)
differ from those of LVars, we believe that Bloom$^L$ bodes well for
programmers' willingness to use lattice-based data structures,
and lattice-parameterized languages based on them, to address
real-world programming challenges.

\if{0}
Since the goal of Bloom$^L$ is to enforce eventual consistency among
distributed replicas (rather than determinism), a read of a variable
in Bloom$^L$ is essentially a write into a local replica of the least
upper bound of all other replicas' copies of its value.  Therefore
Bloom$^L$ does not use threshold reads, which are necessary in the
shared-state LVar setting.
\fi{}

%% LK: Axed for space. :(
%% \paragraph{Quantum programming}
%% The $\lambdaLVar$ semantics is reminiscent of the semantics of quantum
%% programming languages
%% that extend a conventional $\lambda$-calculus with a store
%% that maintains the quantum state. Because of quantum parallelism, the
%% quantum state can be accessed by many threads in parallel, but
%% only through a restricted interface.
%% As a concrete example, the language
%% designed by Selinger and Valiron~\cite{valiron} allows only the following
%%  operations on quantum data:
%% (1) ``appending'' to the current data using the tensor product;
%% (2) performing a unitary operation that must, by definition, act linearly and
%% uniformly on the data; and
%% (3) selecting a set of orthogonal subspaces and performing a measurement
%% that projects the quantum state onto one of the subspaces.
%% These operations correspond roughly to $\lambdaLVar$'s $\NEW$, $\PUT$, and $\GET$.  
%% Quantum mechanics may serve as a source of inspiration
%% when designing operations like $\BUMP$ that introduce limited
%% nondeterminism.

\lk{Following stuff is from the POPL paper, and might get axed.}

\section{Related Work}
\label{section:related}

%% Cut for space -- LK

%% As we discussed in Section~\ref{section:intro}, what deterministic
%% parallel programming models have in common is that they all must do
%% something to restrict access to mutable state shared among
%% concurrent computations so that schedule nondeterminism cannot be
%% observed.  Depending on the model, restricting access to shared
%% mutable state might involve disallowing sharing entirely \cite{dph},
%% only allowing single assignments to shared references
%% \cite{Tesler-1968, IStructures, CnC}, allowing sharing only by a
%% limited form of message passing \cite{Kahn-1974}, ensuring that
%% concurrent accesses to shared state are disjoint \cite{dpj-oopsla},
%% resolving conflicting updates after the fact
%% \cite{concurrent-revisions-haskell11}, or some combination of these
%% approaches.  These constraints can be imposed at the language or API
%% level, within a type system, or at runtime.  In this section, we
%% compare LVish with various examples of previous work in the area.

\paragraph{Monotonic data structures: traditional approaches}

LVish builds on two long traditions of work on parallel programming
models based on monotonically-growing shared data structures:
\begin{itemize}
\item In {\em Kahn process networks} (KPNs) \cite{Kahn-1974}, as well
  as in the more restricted {\em synchronous data flow} systems
  \cite{Lee-sdn}, a network of processes communicate with each other
  through blocking FIFO channels with ever-growing {\em channel
    histories}.  Each process computes a sequential, monotonic
  function from the history of its inputs to the history of its
  outputs, enabling pipeline parallelism.  KPNs are the basis for
  deterministic stream-processing languages such as StreamIt
  \cite{streamit-asplos}.
\item In parallel {\em single-assignment languages}
  \cite{Tesler-1968}, ``full/empty'' bits are associated with heap
  locations so that they may be written to at most once.
  Single-assignment locations with blocking read semantics---that is,
  \emph{IVars} \cite{IStructures}---have appeared in Concurrent ML as
  @SyncVar@s \cite{reppy-cml-book}; in the Intel Concurrent
  Collections system \cite{CnC}; in languages and libraries for
  high-performance computing, such as Chapel \cite{chapel} and the
  Qthreads library \cite{qthreads}; and have even been implemented in
  hardware in Cray MTA machines \cite{cray-mta}.  Although most of
  these uses incorporate IVars into already-nondeterministic
  programming environments, Haskell's @Par@
  monad~\cite{monad-par}---on which our LVish implementation is
  based---uses IVars in a deterministic-by-construction setting,
  allowing user-created threads to communicate through IVars without
  requiring @IO@, so that such communication can occur anywhere inside
  pure programs.
\end{itemize}
LVars are general enough to subsume both IVars and KPNs: a lattice of
channel histories with a prefix ordering allows LVars to represent
FIFO channels that implement a Kahn process network, whereas an LVar
with ``empty'' and ``full'' states (where $\mathit{empty} <
\mathit{full}$) behaves like an IVar, as we described in
Section~\ref{section:lvars-refresher}.  Hence LVars provide a
framework for generalizing and unifying these two existing approaches
to deterministic parallelism.

\paragraph{Deterministic Parallel Java (DPJ)}

DPJ \cite{dpj-oopsla, dpj-hotpar09} is a deterministic language
consisting of a system of annotations for Java code.  A sophisticated
region-based type system ensures that a mutable region of the heap is,
essentially, passed linearly to an exclusive writer, thereby ensuring
that the state accessed by concurrent threads is disjoint.  DPJ does,
however, provide a way to unsafely assert that operations commute with
one another (using the @commuteswith@ form) to enable concurrent
mutation.

LVish differs from DPJ in that it allows overlapping shared state
between threads as the default.  Moreover, since LVar effects are
already commutative, we avoid the need for @commuteswith@ annotations.
Finally, it is worth noting that while in DPJ, commutativity
annotations have to appear in application-level code, in LVish only
the data-structure author needs to write trusted code. The application
programmer can run untrusted code that still enjoys a (quasi-)determinism guarantee, because only (quasi-)deterministic programs can be expressed as LVish @Par@
computations.

More recently, Bocchino \etal~\cite{dpj-popl} proposed a type and effect system that
allows for the incorporation of nondeterministic
sections of code in DPJ.  The goal here is different from ours: while they
aim to support \emph{intentionally} nondeterministic computations such
as those arising from optimization problems like branch-and-bound
search, LVish's quasi-determinism arises as a result of schedule
nondeterminism.

\paragraph{FlowPools}

Prokopec \etal~\cite{flowpools} recently proposed a data structure with
an API closely related to ideas in LVish: a FlowPool is a bag that allows
concurrent insertions but forbids removals, a {\tt seal} operation that forbids
further updates, and combinators like {\tt foreach} that invoke callbacks as
data arrives in the pool.  To retain determinism, the {\tt seal} operation
requires explicitly passing the expected bag \emph{size} as an argument, and the
program will raise an exception if the bag goes over the expected size.

While
this interface has a flavor similar to LVish, it lacks the ability to
detect quiescence, which is crucial for supporting examples like graph
traversal, and the {\tt seal} operation is awkward to use when the structure of
data is not known in advance.  By contrast, our @freeze@ operation
is more expressive and convenient, but moves the
model into the realm of quasi-determinism.  Another important difference is the
fact that LVish is \emph{data structure-generic}: both our formalism and our library
support an unlimited collection of data structures, whereas FlowPools 
are specialized to bags.
Nevertheless, FlowPools represent a ``sweet spot'' in the deterministic parallel design space: by
allowing handlers but not general freezing, they retain determinism
while improving on the expressivity of the original LVars model.  We
claim that, with our addition of handlers, LVish generalizes FlowPools
to add support for arbitrary lattice-based data structures.

\paragraph{Concurrent Revisions}

The Concurrent Revisions (CR)~\cite{concurrent-revisions-haskell11} programming model
 uses isolation types to distinguish regions of the heap shared
by multiple mutators.  Rather than enforcing exclusive access, CR
clones a copy of the state for each mutator, using a
deterministic ``merge function'' for resolving conflicts in local copies at join points.
Unlike LVish's least-upper-bound writes, CR merge functions are \emph{not} necessarily
commutative; the default CR merge function is ``joiner wins''.
Still, semilattices turn up in the metatheory of CR: in particular,
Burckhardt and Leijen~\cite{semantics-concurrent-revisions} show that,
for any two vertices in a CR revision diagram, there exists a
\emph{greatest common ancestor} state which can be used to determine
what changes each side has made---an interesting duality with our
model (in which any two LVar states have a lub).

While CR could be used to model similar types of data structures to LVish---if
versioned variables used least upper bound as their merge function for
conflicts---effects would only become visible at the end of parallel regions,
rather than LVish's asynchronous communication within parallel regions.  This
precludes the use of traditional lock-free data structures as a representation.

%% The management of shared variables in CR is tightly coupled to a
%% fork-join control structure, and the implementation of these variables
%% is similar to reduction variables in other languages (\eg, Cilk
%% {\em hyperobjects}).
%% CR charts an important new area in the deterministic-parallelism
%% design space, but one that differs significantly from
%% LVish.

\paragraph{Conflict-free replicated data types}

In the distributed systems literature, \emph{eventually consistent}
systems based on \emph{conflict-free replicated data types}
(CRDTs)~\cite{crdts} leverage lattice properties to guarantee that
replicas in a distributed database eventually agree.  Unlike LVars,
CRDTs allow intermediate states to be observed: if two replicas are
updated independently, reads of those replicas may disagree until a
(least-upper-bound) merge operation takes place.  Various
data-structure-specific techniques can ensure that non-monotonic
updates (such as removal of elements from a set) are not lost.

The Bloom$^L$ language for distributed database programming
\cite{blooml} combines CRDTs with \emph{monotonic logic}, resulting in a
lattice-parameterized, confluent language that is a close relative of
LVish.  A monotonicity analysis pass rules out programs
that would perform non-monotonic operations on distributed data
collections, whereas in LVish, monotonicity is enforced by the LVar
API.

Future work will further explore the relationship between LVars and
CRDTs: in one direction, we will investigate LVar-based data
structures inspired by CRDTs that support non-monotonic operations; in
the other direction, we will investigate the feasibility and
usefulness of LVar threshold reads in a distributed setting.

\TODO{Add stuff from PLDI paper, maybe?}

\TODO{Add stuff from DISC submission.}
