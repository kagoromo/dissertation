\begin{abstract}
Deterministic-by-construction parallel programming models offer the advantages of parallel speedup while avoiding the nondeterministic, hard-to-reproduce bugs that plague fully concurrent code.
A principled approach to deterministic-by-construction parallel programming with
shared state is offered by \emph{LVars}: shared memory locations whose semantics
are defined in terms of an application-specific lattice.  Writes to an LVar take the
least upper bound of the old and new values with respect to the lattice, while
reads from an LVar can observe only that its contents have crossed a specified
threshold in the lattice.  Although it guarantees determinism, this interface is
quite limited.

We extend LVars in two ways.  First, we add the ability to ``freeze'' and then
read the contents of an LVar directly.  Second, we add the ability to attach
event handlers to an LVar, triggering a callback when the LVar's value changes.
Together, handlers and freezing enable an expressive and useful style of
parallel programming.  We prove that in a language where communication takes
place through these extended LVars, programs are at worst \emph{quasi-deterministic}: on
every run, they either produce the same answer or raise an error.  We
demonstrate the viability of our approach by implementing a library for Haskell
supporting a variety of LVar-based data structures, together with a case
study that illustrates the programming model and yields promising parallel
speedup.
\end{abstract}
