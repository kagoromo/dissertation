\section{LVish, formally}\label{s:quasi-formal}

In this section, I present $\lambdaLVish$, a core calculus for the
LVish programming model. $\lambdaLVish$ is a quasi-deterministic,
parallel, call-by-value $\lambda$-calculus extended with a store
containing LVars.  It extends the original $\lambdaLVar$ language of
Chapter~\ref{ch:lvars} to support the new features in the LVish model
that I described in the previous section.  In comparison to the
informal description in Section~\ref{s:quasi-informal}, though, I make
two simplifications to keep the model lightweight:
\begin{itemize}
\item As with $\lambdaLVar$ in Chapter~\ref{ch:lvars}, I parameterize
  the definition of $\lambdaLVish$ by a \emph{single}
  application-specific lattice, representing the set of states that
  LVars in the calculus can take on. Therefore $\lambdaLVish$ is
  really a \emph{family} of calculi, varying by choice of lattice.
  Multiple lattices can in principle be encoded using a sum
  construction, so this modeling choice is just to keep the
  presentation simple; in any case, our Haskell implementation
  supports multiple lattices natively.
\item Rather than modeling the full ensemble of event handlers,
  handler pools, quiescence, and freezing as separate primitives, I
  instead formalize the ``freeze-after'' pattern---which combined
  them---directly as a primitive.  This greatly simplifies the
  calculus, while still capturing the essence of the programming
  model.
\end{itemize}

\subsection{Lattices}

Just as with $\lambdaLVar$, the application-specific lattice is given
as a 4-tuple $(D, \userleq, \bot, \top)$ where $D$ is a set,
$\userleq$ is a partial order on the elements of $D$, $\bot$ is the
least element of $D$ according to $\userleq$ and $\top$ is the
greatest.  The $\bot$ element represents the initial ``empty'' state
of every LVar, while $\top$ represents the ``error'' state that would
result from conflicting updates to an LVar.  The partial order
$\userleq$ represents the order in which an LVar may take on states.
It induces a binary \emph{least upper bound} (lub) operation
$\userlub{}{}$ on the elements of $D$.

Every two elements of $D$ must have a least upper bound in $D$.
Intuitively, the existence of a lub for every two elements of $D$
means that it is possible for two subcomputations to independently
update an LVar, and then deterministically merge the results by taking
the lub of the resulting two states.  Formally, this makes $(D,
\userleq, \bot, \top)$ a \emph{bounded join-semilattice} with a
designated greatest element $\top$; I continue to use ``lattice'' as
shorthand, and I use $D$ as a shorthand for the entire 4-tuple $(D,
\userleq, \bot, \top)$ when its meaning is clear from the context.

\subsection{Freezing}

To model freezing, we need to generalize the notion of the state of an
LVar to include information about whether it is ``frozen'' or not.
Thus, in $\lambdaLVish$ an LVar's \emph{state} is a pair
$\state{d}{\status}$, where $d$ is an element of the
application-specific set $D$ and $\status$ is a ``status bit'' of
either $\frozentrue$ or $\frozenfalse$.  A state where $\status$ is
$\frozenfalse$ is ``unfrozen'', and one where $\status$ is
$\frozentrue$ is ``frozen''.

I define an ordering $\leqp$ on LVar states $\state{d}{\status}$ in
terms of the application-specific ordering $\userleq$ on elements of
$D$.  Every element of $D$ is ``freezable'' except $\top$.
Informally:
\begin{itemize}
\item Two unfrozen states are ordered according to the
  application-specific $\userleq$; that is, $\state{d}{\frozenfalse}
  \leqp \state{d'}{\frozenfalse}$ exactly when $d \userleq d'$.
\item Two frozen states do not have an order, unless they are equal:
  $\state{d}{\frozentrue} \leqp \state{d'}{\frozentrue}$ exactly when
  $d = d'$.
\item An unfrozen state $\state{d}{\frozenfalse}$ is less than or
  equal to a frozen state $\state{d'}{\frozentrue}$ exactly when $d
  \userleq d'$.
\item The only situation in which a frozen state is less than an
  unfrozen state is if the unfrozen state is $\top$; that is,
  $\state{d}{\frozentrue} \leqp \state{d'}{\frozenfalse}$ exactly when
  $d' = \top$.
\end{itemize}
Adding status bits to each element (except $\top$) of the lattice $(D,
\userleq, \bot, \top)$ results in a new lattice $(D_p, \leqp, \botp,
\topp)$.  (The $p$ stands for pair, since elements of this new lattice
are pairs $\state{d}{\status}$.) I write $\lubp{}{}$ for the least
upper bound operation that $\leqp$ induces.
Definitions~\ref{def:lattice-with-status-bits} and \ref{def:lubp} and
Lemmas~\ref{lem:partition-of-Dp} and~\ref{lem:lattice-structure}
formalize this notion.

\DefLatticeWithStatusBits

\LemPartitionOfDp

\DefLubP

Lemma~\ref{lem:lattice-structure} says that if $(D, \leq, \bot, \top)$
is a lattice, then $(D_p, \leqp, \botp, \topp)$ is as well:

\LemLatticeStructure
\begin{proof}
See Section~\ref{section:lattice-structure-proof}.
\end{proof}

\subsection{Stores}

During the evaluation of $\lambdaLVish$ programs, a \emph{store} $S$
keeps track of the states of LVars.  Each LVar is represented by a
binding from a location $l$, drawn from a set $\Loc$, to its state,
which is some pair $\state{d}{\status}$ from the set $D_p$.  The way
that stores are handled in $\lambdaLVish$ is very similar to how they
are handled in $\lambdaLVar$, except that store bindings now point to
states $\state{d}{\status}$, that is, elements of $D_p$, instead of
merely to $d$, that is, elements of $D$.

\DefStore

I use the notation $\extS{S}{l}{d}{\status}$ to denote extending $S$
with a binding from $l$ to $\state{d}{\status}$.  If $l \in \dom{S}$,
then $\extS{S}{l}{d}{\status}$ denotes an update to the existing
binding for $l$, rather than an extension.  Another way to denote a
store is by explicitly writing out all its bindings, using the
notation $\store{\storebinding{l_1}{d_1}{\status_1},
  \storebinding{l_2}{d_2}{\status_2}, \dots}$.

We can lift the $\leqp$ and $\lubp{}{}$ operations defined on elements
of $D_p$ to the level of stores:

\DefLeqStore

\DefLubStore

If, for example,
\[ \lubp{\state{d_1}{\status_1}}{\state{d_2}{\status_2}} = \topp, \]
then
\[ \lubstore{\store{\storebinding{l}{d_1}{\status_1}}}{\store{\storebinding{l}{d_2}{\status_2}}} =
\topS. \]

Just as a a store containing a binding $\storebindingRaw{l}{\top}$ can
never arise during the execution of a $\lambdaLVar$ program, a store
containing a binding $\storebinding{l}{\top}{\status}$ can never arise
during the execution of a $\lambdaLVish$ program. An attempted @put@
that would take the value of $l$ to
$\state{\top}{\frozenfalse}$---that is, $\topp$---will raise an error,
and there is no $\state{\top}{\frozentrue}$ element of $D_p$.

\subsection{$\lambdaLVish$: syntax and semantics}

\TODO{Split the semantics, as in Chapter 2, and update the text to go
  with the split semantics.}

\FigLambdaLVishGrammar

\FigLambdaLVishSemantics

The syntax and operational semantics of $\lambdaLVish$ appear in
Figures \ref{f:lambdaLVish-syntax} and \ref{f:lambdaLVish-semantics},
respectively.  As with $\lambdaLVar$, both the syntax and semantics
are parameterized by the lattice $(D, \userleq, \bot, \top)$, and the
reduction relation $\parstepsto$ is defined on \emph{configurations}
$\config{S}{e}$, made up of a store $S$ and an expression $e$.  The
\emph{error configuration}, written $\error$, is a unique element
added to the set of configurations, but $\config{\topS}{e}$ is equal
to $\error$ for all expressions $e$.  The metavariable $\conf$ ranges
over configurations.

$\lambdaLVish$ has all the expression forms of
$\lambdaLVar$---variables, values, application expressions, @get@
expressions, @put@ expressions, and @new@---and adds two new forms:
the @freeze@ expression and the $\FAW$ expression, which I discuss in
more detail below.  Values in $\lambdaLVish$ include all those from
$\lambdaLVar$---the unit value $\unit$, lattice elements $d$,
locations $l$, threshold sets $P$, and $\lambda$ expressions---as well
as states $p$, which are pairs $\state{d}{\status}$, and event sets
$Q$.  Instead of $T$, I now use the metavariable $P$ for threshold
sets, in keeping with the fact that in $\lambdaLVish$, members of
threshold sets are states $p$.
 
As with $\lambdaLVar$, $\lambdaLVish$ uses a reduction semantics based
on evaluation contexts.  The {\sc E-Eval-Ctxt} rule allows us to apply
reductions within a context, and is identical to the corresponding
rule in $\lambdaLVar$, although the set of evaluation contexts that
the metavariable $E$ ranges over has grown to accommodate @freeze@ and
$\FAW$.  The {\sc E-Beta} rule is also identical to its counterpart in
$\lambdaLVar$.

\subsection{Semantics of \lstinline|new|, \lstinline|put|, and \lstinline|get|}\label{subsection:quasi-semantics-of-new-put-and-get}

Because of the addition of status bits, the {\sc E-New}, {\sc E-Put},
{\sc E-Put-Err}, {\sc E-Get} rules have changed slightly from their
counterparts in $\lambdaLVar$:

\begin{itemize}
\item @new@ (implemented by the {\sc E-New} rule) extends the store
  with a binding for a new LVar whose initial state is $(\bot,
  \frozenfalse)$, and returns the location $l$ of that LVar (\ie, a
  pointer to the LVar).
\item @put@ (implemented by the {\sc E-Put} and {\sc E-Put-Err} rules)
  takes a pointer to an LVar and a new lattice element $d_2$ and
  updates the LVar's state to the \emph{least upper bound} of the
  current state and $\state{d_2}{\frozenfalse}$, potentially pushing
  the state of the LVar upward in the lattice.  Any update that would
  take the state of an LVar to $\topp$ results in the program
  immediately stepping to $\error$.
\item @get@ (implemented by the {\sc E-Get} rule) performs a blocking
  threshold read.  It takes a pointer to an LVar and a \emph{threshold
    set} $P$, which is a non-empty set of LVar states that must be
  \emph{pairwise incompatible}, expressed by the premise $\incomp{P}$.
  A threshold set $P$ is pairwise incompatible iff the lub of any two
  distinct elements in $P$ is $\topp$.  If the LVar's state $p_1$ in
  the lattice is \emph{at or above} some $p_2 \in P$, the @get@
  operation unblocks and returns $p_2$.  Note that $p_2$ is a unique
  element of $P$, for if there is another $p'_2 \neq p_2$ in the
  threshold set such that $p'_2 \leqp p_1$, it would follow that
  $\lubp{p_2}{p'_2} = p_1 \neq \topp$, which contradicts the
  requirement that $P$ be pairwise incompatible.\footnote{Although
    $\incomp{P}$ is given as a premise of the {\sc E-Get} reduction
    rule (suggesting that it is checked at runtime), in a real
    implementation threshold sets need not be written explicitly, and
    it is the data structure author's responsibility to ensure that
    any provided read operations have threshold semantics; see
    Chapter~\ref{ch:lvish}.\TODO{Point to a more specific part of
      Chapter~\ref{ch:lvish}.}}
\end{itemize}

\subsection{Freezing and the $\FAW$ primitive}\label{subsection:quasi-freezing}

The {\sc E-Freeze-Init}, {\sc E-Spawn-Handler}, {\sc E-Freeze-Final},
and {\sc E-Freeze-Simple} rules are all new additions to
$\lambdaLVish$.  The {\sc E-Freeze-Simple} rule gives the semantics
for the @freeze@ expression, which takes an LVar as argument and
immediately freezes and returns its contents.

More interesting is the $\FAW$ primitive, which models the
``freeze-after'' pattern I described in
Section~\ref{subsection:quasi-freeze-after}.  The expression
\[ \freezeafter{e_{\rm lv}}{e_{\rm events}}{e_{\rm cb}} \]
has the following semantics:
\begin{itemize}
\item It attaches the callback $e_{\rm cb}$ to the LVar $e_{\rm lv}$.
  The expression $e_{\rm events}$ must evaluate to a event set $Q$;
  the callback will be executed, once, for each lattice element in $Q$
  that the LVar's state reaches or surpasses.  The callback $e_{\rm
    cb}$ is a function that takes a lattice element as its argument.
  Its return value is ignored, so it runs solely for effect.  For
  instance, a callback might itself do a @put@ to the LVar to which it
  is attached, triggering yet more callbacks.
\item If the handler reaches a quiescent state, the LVar $e_{\rm lv}$
  is frozen, and its \emph{exact} state is returned (rather than an
  underapproximation of the state, as with @get@).
\end{itemize}
To keep track of the running callbacks, $\lambdaLVish$ includes an
auxiliary form,
\[
\freezeafterfull{l}{Q}{\lam{x}{e_0}}{\setof{e, \dots}}{H}
\]
where:
\begin{itemize}
\item The value $l$ is the LVar being handled/frozen;
\item The set $Q$ (a subset of the lattice $D$) is the event set;
\item The value $\lam{x}{e_0}$ is the callback function;
\item The set of expressions $\setof{e, \dots}$ are the running
  callbacks; and
\item The set $H$ (a subset of the lattice $D$) represents those
  values in $Q$ for which callbacks have already been launched.
\end{itemize}
Due to $\lambdaLVish$'s use of evaluation contexts, any running
callback can execute at any time, as if each is running in its own
thread.

The rule {\sc E-Spawn-Handler} launches a new callback thread any time
the LVar's current value is at or above some element in $Q$ that has
not already been handled.  This step can be taken nondeterministically
at any time after the relevant @put@ has been performed.

The rule {\sc E-Freeze-Final} detects quiescence by checking that two
properties hold.  First, every event of interest (lattice element in
$Q$) that has occurred (is bounded by the current LVar state) must be
handled (be in $H$).  Second, all existing callback threads must have
terminated with a value.  In other words, every enabled callback has
completed.  When such a quiescent state is detected, {\sc
  E-Freeze-Final} freezes the LVar's state.  Like {\sc
  E-Spawn-Handler}, the rule can fire at any time,
nondeterministically, that the handler appears quiescent---a transient
property!  But after being frozen, any further @put@s that would have
enabled additional callbacks will instead fault, causing the program
to step to $\error$.

Therefore, freezing is a way of ``betting'' that once a collection of
callbacks have completed, no further @put@s that change the LVar's
value will occur.  For a given run of a program, either all @put@s to
an LVar arrive before it has been frozen, in which case the value
returned by $\FAW$ is the lub of those values, or some @put@ arrives
after the LVar has been frozen, in which case the program will fault.
And thus we have arrived at \emph{quasi-determinism}: a program will
always either evaluate to the same answer or it will fault.

To ensure that we will win our bet, we need to guarantee that
quiescence is a \emph{permanent} state, rather than a transient
one---that is, we need to perform all @put@s either prior to $\FAW$,
or by the callback function within it (as will be the case for
fixpoint computations).  In practice, freezing is usually the very
last step of an algorithm, permitting its result to be extracted. As
we will see in Chapter~\ref{ch:lvish}, our implementation provides a
special @runParThenFreeze@ function that does so, and thereby
guarantees full determinism.\TODO{Point to a more specific part of
  Chapter~\ref{ch:lvish}.}
