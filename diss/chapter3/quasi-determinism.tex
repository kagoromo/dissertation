\section{Proof of quasi-determinism for $\lambdaLVish$}\label{s:quasi-quasi-determinism}

In this section, I give a proof of quasi-determinism for
$\lambdaLVish$ that formalizes the claim made earlier in this chapter:
that, for a given program, although some executions may raise
exceptions, all executions that produce a final result will produce
the same final result.

The quasi-determinism theorem says that if two executions starting
from a configuration $\conf$ terminate in configurations $\conf'$ and
$\conf''$, then either $\conf'$ and $\conf''$ are the same
configuration, or one of them is $\error$.  As with the determinism
result for $\lambdaLVar$, quasi-determinism follows from a series of
supporting lemmas which I first prove.  Unlike with the determinism
proof for $\lambdaLVar$, though, for $\lambdaLVish$ I do not need to
prove a Clash lemma (\eg, Lemma~\ref{lem:lvars-clash}) or an Error
Preservation lemma (\eg, Lemma~\ref{lem:lvars-error-preservation}).
These properties aren't necessary because quasi-determinism, with its
``...or one of them is $\error$'', is a weaker property than
determinism.

\subsection{Monotonicity lemma}\label{subsection:quasi-monotonicity}

The Monotonicity lemma says that, as evaluation proceeds according to
the $\parstepsto$ relation, the store can only grow with respect to
the $\leqstore{}{}$ ordering.

\LemMonotonicity
\begin{proof}
  By induction on the derivation of $\config{S}{e} \parstepsto
  \config{S'}{e'}$, by cases on the last rule in the derivation; see
  Section~\ref{section:monotonicity-proof}.
\end{proof}

\subsection{Independence lemma}\label{subsection:quasi-independence}

Like the Independence lemma for $\lambdaLVar$
(Lemma~\ref{lem:lvars-independence}), Lemma~\ref{lem:independence}
establishes a ``frame property'' for $\lambdaLVish$ that captures the
idea that independent effects commute with each other.  Just as with
the $\lambdaLVar$ version of the Independence lemma,
Lemma~\ref{lem:independence} requires as a precondidion that the store
$S''$ must be \emph{non-conflicting}
(Definition~\ref{def:lvars-non-conflicting}) with the original
transition from $\config{S}{e}$ to $\config{S'}{e'}$, meaning that
locations in $S''$ cannot share names with locations newly allocated
during the transition, which rules out location name conflicts caused
by allocation.

In the context of $\lambdaLVish$, where freezing is possible, we have
an additional precondition that the stores $\lubstore{S'}{S''}$ and
$S$ are \emph{equal in status}---that, for all the locations shared
between them, the status bits of those locations agree.  This
assumption rules out interference from freezing.

\DefEqualStatus

\LemIndependence
\begin{proof}
   By induction on the derivation of $\config{S}{e} \parstepsto
   \config{S'}{e'}$, by cases on the last rule in the derivation; see
   Section~\ref{section:independence-proof}.
\end{proof}

\subsection{Quasi-confluence lemmas}\label{subsection:quasi-quasi-confluence}

\lk{In the POPL paper and TR, there was a lemma called ``Strong Local
  Quasi-Diamond'' and another, almost identical one called ``Strong
  Local Quasi-Confluence'' that followed almost directly from Strong
  Local Quasi-Diamond.  In the $\lambdaLVar$ determinism proof,
  ``Strong Local Confluence'' followed directly from ``Diamond'', but
  this was the case because ``Strong Local Confluence'' was a special
  case of ``Diamond''.  Here, I actually think we should have any
  property that we call ``diamond'', because to me that word connotes
  ``exactly one step'', and we should instead just have a ``Strong
  Local Quasi-Confluence'' that is $\lambdaLVish$'s counterpart to
  ``Diamond'' and ``Strong Local Confluence'' taken together.}

\TODO{Add some text here explaining the above.}

\LemStrongLocalQuasiConfluence
\begin{proof}
  By induction on the derivation of $\conf \parstepsto \conf_a$, by
  cases on the last rule in the derivation; see
  Section~\ref{section:strong-local-quasi-confluence-proof}.
  \TODO{Figure out if this is a correct summary.}
\end{proof}

\LemStrongOneSidedQuasiConfluence
\begin{proof}
  By induction on $m$; see
  Section~\ref{section:strong-one-sided-quasi-confluence-proof}.
  \TODO{Figure out if this is a correct summary.}
\end{proof}

\LemStrongQuasiConfluence
\begin{proof}
  By induction on $n$; see
  Section~\ref{section:strong-quasi-confluence-proof}.
  \TODO{Figure out if this is a correct summary.}
\end{proof}

\LemQuasiConfluence
 
\subsection{Quasi-determinism theorem}\label{subsection:quasi-quasi-determinism}

\ThmQuasiDeterminism
