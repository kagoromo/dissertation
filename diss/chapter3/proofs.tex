\section{Proof of quasi-determinism for $\lambdaLVish$}\label{s:quasi-proof}

\TODO{Revise and expand this section.}

Our proof of quasi-determinism for LVish formalizes the claim we make
in Section~\ref{section:intro}: that, for a given program, although
some executions may raise exceptions, all executions that produce a
final result will produce the same final result.

In this section, we give the statements of the main
quasi-determinism theorem and the two most important supporting
lemmas.  The statements of the remaining lemmas,
and proofs of all our theorems and lemmas,
are included in
\ifx\fulltr\undefined
%%% Text for paper
the companion
technical report \cite{Freeze-TR}.
\else
%%% Text for TR
Appendix~\ref{app:quasi-determinism-for-lvish}.
\fi

\subsection{Quasi-determinism and quasi-confluence}

Our main result, Theorem~\ref{thm:quasi-determinism}, says that if two
executions starting from a configuration $\conf$ terminate in
configurations $\conf'$ and $\conf''$, then $\conf'$ and $\conf''$ are
the same configuration, or one of them is $\error$.

\ThmQuasiDeterminism

\noindent Theorem~\ref{thm:quasi-determinism} follows from a series of
\emph{quasi-confluence} lemmas.  The most important of these,
Strong Local Quasi-Confluence (Lemma~\ref{lem:strong-local-quasi-confluence}), says that if a
configuration steps to two different configurations, then either there
exists a single third configuration to which they both step (in at
most one step), or one of them steps to $\error$.
Additional lemmas generalize Lemma~\ref{lem:strong-local-quasi-confluence}'s result to multiple
steps by induction on the number of steps, eventually building up to Theorem~\ref{thm:quasi-determinism}.

\LemStrongLocalQuasiConfluence

\if 0
\noindent The Strong Quasi-Confluence lemma,
Lemma~\ref{lem:strong-quasi-confluence}, immediately implies
Quasi-Confluence, Lemma~\ref{lem:quasi-confluence}.  Although the
weaker property Quasi-Confluence is all that is necessary to show
Theorem~\ref{thm:quasi-determinism}, Strong Quasi-Confluence is an
interesting result in itself because it bounds the number of steps
that an execution will take to reach $\error$.
\fi

\subsection{Independence}\label{subsection:quasi-independence}

In order to show Lemma~\ref{lem:strong-local-quasi-confluence},
%% on which the rest of
%% the quasi-confluence lemmas and ultimately
%% Theorem~\ref{thm:quasi-determinism} depend,
%% we are required to
we need a ``frame property'' for LVish that captures the idea that
independent effects commute with each other.
Lemma~\ref{lem:independence}, the Independence lemma, establishes this
property.  Consider an expression $e$ that runs starting in store $S$
and steps to $e'$, updating the store to $S'$.  The Independence lemma
allows us to make a double-edged guarantee about what will happen if
we run $e$ starting from a larger store $\lubstore{S}{S''}$: first, it
will update the store to $\lubstore{S'}{S''}$; second, it will step to
$e'$ as it did before.\lk{Aaron: Do you think it's fair to claim that
  these respectively boil down to the Safety Monotonicity and Frame
  Property claims from the ``Local Action'' paper?}  Here
$\lubstore{S}{S''}$ is the least upper bound of the original $S$ and
some other store $S''$ that is ``framed on'' to $S$; intuitively,
$S''$ is the store resulting from some other independently-running computation.

\LemIndependence

\noindent Lemma~\ref{lem:independence} requires as a precondition that
the stores $\lubstore{S'}{S''}$ and $S$ are \emph{equal in
  status}---that, for all the locations shared between them, the
status bits of those locations agree.  This assumption rules out
interference from freezing.  Finally, the store $S''$ must be
\emph{non-conflicting} with the original transition from
$\config{S}{e}$ to $\config{S'}{e'}$, meaning that locations in $S''$
cannot share names with locations newly allocated during the
transition; this rules out location name conflicts caused by
allocation.

\DefEqualStatus

\DefNonConflicting
