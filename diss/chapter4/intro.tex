\TODO{This intro is really rough and probably needs editing.}

We want to realize the programming model of Chapters~\ref{ch:lvars}
and~\ref{ch:quasi} in a practical programming tool.  In this chapter,
I describe the LVish library, a Haskell library for practical
deterministic and quasi-deterministic parallle programming with LVars.

From the example shown in Section~\ref{s:quasi-informal}, we've
already had a taste of what it is like to program with the LVish
programming model.  Indeed, Section~\ref{s:quasi-informal} even gives
an example of an LVish Haskell program.  However, we haven't yet seen
a complete picture of the LVish Haskell API.  In
Section~\ref{s:lvish-api}, I discuss the high-level design goals of
LVish and describe the LVish API, and I discuss the parallel
breadth-first traversal of Section~\ref{s:quasi-informal} in more
detail.

In Section~\ref{s:lvish-internals}, I describe how the LVish library
itself is implemented. \TODO{More about this.}

Finally, in Sections~\ref{s:lvish-k-cfa} and~\ref{s:lvish-phybin}, I
illustrate how LVish is like through two case studies that involve
parallelizing existing Haskell programs by porting them to LVish.
First, in Section~\ref{s:lvish-k-cfa}, I describe using LVish to
parallelize a control-flow analysis ($k$-CFA) algorithm.  Second, in
Section~\ref{s:lvish-phybin}, I describe using LVish to parallelize
\emph{PhyBin}, a bioinformatics application that relies heavily on a
(parallelizable) tree-edit distance algorithm.
