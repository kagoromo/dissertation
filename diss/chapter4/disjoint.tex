\section{\il{Par}-monad transformers and disjoint parallel update}\label{section:lvish-par-monad-transformers-and-disjoint-parallel-update}

\lk{This section is from sections 4 and 5 of the PLDI paper.  I'm
  generally a bit unhappy with it---it's too Haskelly compared to the
  surrounding material, and there's probably more detail than there
  needs to be; I'd like to pare it down to the bare minimum needed to
  explain \il{ParST} (can we get away with not explaining monad
  transformers?).  Also, the PLDI paper had an argument here as to why
  \il{ParST} retains determinism, which I've left out because it's
  kind of handwavy and unconvincing.}

\lk{Really, all we want to say is this: We can apply something called
  a ParST transformer to the Par monad that will let us thread some
  state through, and at a fork the state has to be either split or
  duplicated.  If we're splitting, say, a vector of locations, the
  Haskell type system is powerful enough to ensure at compile time
  that neither of the child computations can access the original
  complete vector.}

The effect-tracking system of the previous section provides a way to
toggle on and off a fixed set of basic capabilities using the type
system---that is, with the switches embedded in the @e@ effect
signature that parameterizes the @Par@ type.  These type-level
distinctions are needed for defining restricted but safe idioms, but
they do not address \emph{extensibility}.  For that, we turn to
multiple monads rather than a single parameterized @Par@ monad.

\TODO{This ``extensibility'' point is kind of out of place here---it
  made sense in the context of the PLDI paper, but here it's not clear
  what I'm talking about.}

\subsection{Monad transformers and their use in LVish}

Haskell programmers use a variety of different monads: @Reader@ for
threading parameters, @State@ for in-place update, @Cont@ for
continuations, and so on.  All monads support the same core operations
(@bind@ and @return@ from the @Monad@ type class) and satisfy the
three monad laws.  However, each monad must also provide other
operations that make it worth using.  Most famously, the @IO@ monad
provides various input-output operations.

A monad \emph{transformer}, on the other hand, is a type constructor
that adds ``plug-in'' capabilities to an underlying monad.  For
example, the @StateT@ monad transformer adds an extra piece of
implicit, modifiable state to an underlying monad.  Adding a monad
transformer to a type always returns another monad (preserving the
@Monad@ instance).

In the same way, we can define a \emph{@Par@-monad transformer} as a
type constructor @T@, where, for all @Par@ monads @m@, @T m@ is
another @Par@ monad with additional capabilities, and where a value of
type @T m a@, for instance, @T (Par e s) a@, is a computation in that
monad.  Indeed, @Par@-monad transformers \emph{are} valid monad
transformers (in the sense of providing a standard @MonadTrans@
instance).

The semantics of a @Par@ monad is captured by a series of type
classes, all of which are closed under @Par@-monad transformer
application.  At minimum, an instance of the @Par@ monad type class
must provide the @fork@ operation that we have seen in our examples so
far:

\singlespacing
\begin{lstlisting}
   class (Monad m) => ParMonad m where
     fork :: m () -> m ()
\end{lstlisting}
\doublespacing

Programs with @fork@ create a binary tree of monadic actions with @()@
(unit) return values.  Additional type classes capture the interfaces
to basic parallel data structures and control constructs such as
futures (@ParFuture@), @IVar@s (@ParIVar@), and more general LVars
(@ParLVar@). For example, the class @ParIVar@ provides @new@ and @put@
operations with the signatures below.\footnote{Although it may appear
  that generic treatment of \il{Par} monads as type variables \il{m}
  removes the additional metadata in a type such as \il{Par e s a},
  note that it is possible to recover this information with type-level
  functions.}

\singlespacing
\begin{lstlisting}[mathescape=true]
   class (ParMonad m) => ParFuture m  where
     $\ldots$ 
   class (ParMonad m) => ParIVar m  where
     type IVar m :: * -> *
     new :: m (IVar m a)
     put :: IVar m a -> a -> m ()
     get :: IVar m a -> m a
\end{lstlisting}
\doublespacing

The \il{ParFuture}, \il{ParIVar}, and \il{ParLVar} type classes form a
hierarchy: any implementation that can support LVars can support
IVars, and any that can support IVars can support futures.  Taken
together, this framework for generic @Par@ programming makes it
possible for LVish programs to be reusable across a variety of
schedulers.  This can be quite useful; for example, LVish provides a
\il{ParFuture} instance for the native GHC work-stealing
scheduler~\cite{runtime-support-multicore-haskell}.

\subsection{Example: threading state in parallel}

Perhaps the simplest example of a @Par@-monad transformer is the
standard @StateT@ monad transformer (provided by Haskell's
@Control.Monad.State@ package).  However, even if @m@ is a @Par@
monad, for @StateT s m@ to also be a @Par@ monad, the state @s@ must
be \emph{splittable}; that is, it must be specified what is to be done
with the state at @fork@ points in the control flow.  For example, the
state may be duplicated, split, or otherwise updated to note the fork.

The below code promotes @StateT@ to be a @Par@-monad transformer:

\singlespacing
\begin{lstlisting}
  class SplittableState a where
    splitState :: a -> (a,a)

  instance (SplittableState s, ParMonad m) => 
           ParMonad (StateT s m) where  
    fork task =
       do s <- State.get 
          let (s1,s2) = splitState s
          State.put s2
          lift (fork (do runStateT task s1; return ()))
\end{lstlisting}
\doublespacing

Note that here, @put@ and @get@ are not LVar operations, but the
standard procedures for setting and retrieving the state in a
@StateT@.

\subsection{Determinism guarantee}

The @StateT@ transformer preserves determinism because it is
effectively \emph{syntactic sugar}.  That is, @StateT@ does not allow
one to write any program that could not already be written using the
underlying @Par@ monad, simply by passing around an extra argument.

This is because @StateT@ only provides a \emph{functional} state (an
implicit argument and return value), not actual mutable heap
locations.  Genuine mutable locations in pure computations, on the
other hand, require Haskell's @ST@ monad, the safer sister monad to
@IO@.

\subsection{Disjoint parallel update with \il{ParST}}

The LVars model is based on the notion that it is fine for multiple
threads to access and update shared memory, so long as updates commute
and ``build on'' one another, only adding information rather than
destroying it.  Yet it should be possible for threads to update memory
destructively, so long as the memory updated by different threads is
\emph{disjoint}.  This is the approach to deterministic parallelism
taken by, for example, Deterministic Parallel Java
(DPJ)~\cite{dpj-oopsla}, which uses a region-based type and effect
system to ensure that each mutable region of the heap is passed
linearly to a thread that then gains exclusive permission to update
that region.  In order to add this capability to the LVish library,
though, we need destructive updates to interoperate with LVar effects.
Moreover, we wish to do so at the library level, without requiring
language extensions.

Our solution is to provide a @ParST@ transformer, a variant of the
@StateT@ transformer of Section~\ref{s:transformers}.  @ParST@ allows
arbitrarily complex mutable state, such as tuples of vectors (arrays).
However, @ParST@ enforces the restriction that {every} memory location
in the state is reachable by only one pointer: alias freedom.

Previous approaches to integrating mutable memory with pure functional
code (\ie, the @ST@ monad) work with LVish, but only allow
thread-private memory.  There is no way to operate on the same
structure (for instance, on two halves of an array) from different
threads.  @ParST@ exploits the fact that it is perfectly safe to do so
as long as the different threads are accessing disjoint parts of the
data structure.  Below we demonstrate the idea using a simplified
convenience module provided alongside the general (@ParST@) library,
which handles the specific case of a single vector as the mutable
state being shared.

\singlespacing
\begin{lstlisting}
  runParVecT 10 (
    do -- Fill all 10 slots with "a":
       set "a"
       -- Get a pointer to the state:
       ptr <- reify 
       -- Call pre-existing ST code:
       new <- pickLetter ptr
       forkSTSplit (SplitAt 5)
         (write 0 new)
         (write 0 "c")
       -- ptr is again accessible here
       ellipses)
\end{lstlisting}
\doublespacing

This program demonstrates running a parallel, stateful session within
a @Par@ computation.  The shared mutable vector is implicit and global
within the monadic @do@ block.  We fork the control flow of the
program with @forkSTSplit@, where @(write 0 new)@ and @(write 0 "c")@
are the two forked child computations.  The @SplitAt@ value describes
how to partition the state into disjoint pieces: @(SplitAt 5)@
indicates that the element at index $5$ in the vector is the ``split
point'', and hence the first child computation passed to @forkSTSplit@
may access only the first half of the vector, while the other may
access only the second half.  (We will see shortly how this
generalizes.)  Each child computation sees only a \emph{local} view of
the vector, so writing @"c"@ to index @0@ in the second child
computation is really writing to index $5$ of the global vector.

Ensuring the safety of @ParST@ hinges on two requirements:

\begin{itemize}
\item \emph{Disjointness:} Any thread can get a direct pointer to its
  state.  In the above example, @ptr@ is an @STVector@ that can be
  passed to any standard library procedures in the @ST@ monad.
  However, it must \emph{not} be possible to access @ptr@ from
  @forkSTSplit@'s child computations.  We accomplish this using
  Haskell's support for higher-rank types,\footnote{That is, the type
    of a child computation begins with \il{(forall s DOT ParST
      ellipses)}.} ensuring that accessing @ptr@ from a child
  computation causes a type error.  Finally, @forkSTSplit@ is a
  fork-join construct; after it completes the parent thread again has
  full access to @ptr@.

\item \emph{Alias freedom:} Imagine that we expanded the example above
  to have as its state a \emph{tuple} of two vectors: $(v_1,v_2)$.
  \emph{If} we allowed the user to supply an arbitrary initial state
  to their @ParST@ computation, then they might provide the state
  $(v_1, v_1)$, \ie, two copies of the same pointer.  This breaks the
  abstraction, enabling them to reach the same mutable location from
  multiple threads (by splitting the supposedly-disjoint vectors at a
  different index).  Thus, in LVish, users do not populate the state
  directly, but only describe a \emph{recipe} for its creation.  Each
  type used as a @ParST@ state has an associated type for descriptions
  of (1) how to create an initial structure, and (2) how to split it
  into disjoint pieces.  We provide a trusted library of instances for
  commonly used types.
\end{itemize}

\subsection{Inter-thread communication}

Disjoint state update does not solve the problem of communication
between threads.  Hence systems built around this idea often include
other means for performing reductions, or require ``commutativity
annotations'' for operations such as adding to a set.  For instance,
DPJ provides a @commuteswith@ form for asserting that operations
commute with one another to enable concurrent mutation.  In LVish,
however, such annotations are unnecessary, because LVish already
provides a language-level guarantee that all effects commute!  Thus, a
programmer using LVish with @ParST@ can use any LVar to communicate
results between threads performing disjoint updates, without requiring
trusted code or annotations.  Moreover, LVish with @ParST@ is unique
among deterministic parallel programming models in that it allows both
DPJ-style, disjoint destructive parallel updates, and blocking,
dataflow-style communication between threads (through LVars).
