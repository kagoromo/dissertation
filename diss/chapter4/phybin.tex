\section{Case study: parallelizing PhyBin with LVish}\label{s:lvish-phybin}

\TODO{Edit this section.}

\begin{figure}
  \centering
\begin{lstlisting}
  global: biptable, distmat
  (1) for t `member` alltrees:
        for bip `member` t:
          insert(biptable, (t, bip))
  (2) for (_, trset) `member` biptable: 
        for t1 `member` alltrees:
          for t2 `member` alltrees:
            if t1 `member` trset `xor` t2 `member` trset
            then increment(distmat[t1,t2])
\end{lstlisting}  
  \caption{Pseudocode of the HashRF algorithm for computing a tree-edit-distance matrix.}
  \label{f:hashrf-alg}
\end{figure}


A {\em phylogenetic tree} represents a possible ancestry for a set of $N$ species.
% which form the leaf nodes of the tree. 
Leaf nodes in the tree are labeled with species' names, and the structure of the
tree represents a hypothesis about common ancestors. For a variety of reasons, biologists often end up
with many alternative trees, whose relationships they need to then analyze.
PhyBin\footnote{\url{http://hackage.haskell.org/package/phybin}} is a medium-sized (3500-line)
bioinformatics program for this purpose, initially released in 2010.
%
The primary output of the 
software is a hierarchical clustering of the input tree set (a tree
of trees), but most of its computational effort is spent computing an
$N{\times}N$ 
distance matrix, which records the pairwise 
{\em edit distance} between trees.
It is this distance computation that we parallelize in our case study.

The distance metric itself is called {\em Robinson-Foulds} (RF) distance, and
the fastest algorithm for all-to-all RF distance computation is the {\em HashRF}
algorithm~\cite{hashrf}, introduced by a software package of the same name.\footnote{\url{https://code.google.com/p/hashrf/}}  HashRF is about
2-3$\times$ as fast as PhyBin.  Both packages are dozens or hundreds of times faster
than the more widely-used software that computes RF distance matrices (\eg{}, Phylip\footnote{\url{http://evolution.genetics.washington.edu/phylip.html}}, DendroPy\footnote{\url{http://pythonhosted.org/DendroPy/}}).  These
slower packages use $\frac{N^2-N}{2}$ full applications of the distance metric, which has poor
locality in that it reads all trees in from memory $\frac{N^2-N}{2}$ times.
%

Before describing how the HashRF algorithm improves on this, we must observe that {edit
  distance} between trees (number of modifications to transform one to the
other) can be reduced to symmetric set difference between sets of bipartitions.
That is, each intermediate node of a tree can be seen as partitioning the set of
leaves into those below and above the node, respectively.  For example, with
leaves $A$, $B$, $C$, $D$, and $E$, one bipartition would be {\tt ``AB|CDE''},
while another would be {\tt ``ABC|DE''}.  Identical trees, of course, convert to the
same set of bipartitions.  Furthermore, after converting trees to sets of
bipartitions, set difference may be computed using standard set data structures.
% the size of the symmetric set difference is exactly the Robinson Foulds distance.

The HashRF algorithm makes use of this fact and adds a clever trick that greatly improves
locality.  Before computing the actual distances between trees, it populates a
table mapping each observed bipartition to the set of trees that contain it.  In
the original PhyBin source:
%
\begin{lstlisting}
  type BipTable = Map DenseLabelSet (Set TreeID)
\end{lstlisting}
% 
Above, a @DenseLabelSet@ encodes an individual bipartition as a bit
vector.  PhyBin uses purely functional data structures for the @Map@
and @Set@ types, whereas HashRF uses a mutable hash table.  Yet in
both cases, these structures {\em grow monotonically} during
execution.  The full algorithm for computing the distance matrix is 
shown in Figure~\ref{f:hashrf-alg}.
The second phase of the algorithm is still $O(N^2)$, but it only needs
to read from the much smaller @trset@ during this phase.  All loops
in Figure~\ref{f:hashrf-alg} are potentially parallel.


\paragraph{Parallelization}
The LVish methodology applies directly to this application: 
\begin{itemize}
\item The @biptable@ in the first phase is a map of sets, which are
  directly replaced by their LVar counterparts.
\item The @distmat@ in the second phase is a vector of monotonic
      @bump@ counters.
\end{itemize}
In fact, the parallel port of PhyBin using LVish was so straightforward that, after reading the code,
parallelizing the first phase took only 29 minutes.\footnote{Git commit range: \url{https://github.com/rrnewton/PhyBin/compare/5cbf7d26c07a...6a05cfab490a7a}}  Once the second
phase was ported, the distance computation sped up by a factor of $3.35\times$ on 
8 cores (Table~\ref{t:phybin-bench}).
% 
% This is a heavily data-stucture intensive and allocation intensive
% program, the sort that is unlikely to see perfect scaling.  Nevertheless, 
This is exactly where we would like to use LVish---to achieve modest
speedups for modest effort, in programs with complicated data
structures (and high allocation rates), and {\em without} changing the
determinism guarantee of the original functional code.

  %% Performance vs. HashRF, DendroPy, PAUP
  %% Parallel Speedup.
% \begin{center}
%   \includegraphics[width=2.5in]{figures/vs_phylip_dendropy.pdf}
% \end{center}

\begin{table}
\begin{tabular}{| l | l | l | l | l | l | l |}
\hline
Trees     & Species    & \multicolumn{3}{|l|}{PhyBin} & DendroPy & Phylip \\ \hline
100       & 150        & \multicolumn{3}{|l|}{0.269}  & 22.1     & 12.8   \\ \hline
          &            & \multicolumn{4}{|l|}{PhyBin 1, 2, 4, 8 core} & HashRF \\ \hline
1000      & 150        & 4.7      & 3     & 1.9     & 1.4      & 1.7      \\ \hline
\end{tabular}
\caption{PhyBin performance comparison with DendroPy, Phylip, and HashRF.  All
  times in seconds.}
\label{t:phybin-bench}
\end{table}
%% FINISH TABLE
%% \begin{verbatim}
%%           phybin  dendro phylip
%% 100  150  0.269   22.1   12.8  

%%           phybin-1,2,4,8  hashrf
%% 1000 150  4.7 3 1.9 1.4     1.7
%% \end{verbatim}



%% PhyBin was selected as an interesting target because the {\em HashRF} algorithm
%% on which it is based is monotonic.  
%% %
%% In fact, even the HashRF software of
%% the same name, which is a C++ implementation of the algorithm, spends all of its%% time doing monotonic additions to a table.


%% \begin{figure}
%% -----------------------------\\
%% (PLACEHOLDER)  \\
%% Show the first phase -- monotonically filling a growing Map-of-Sets \\
%% And the second phase -- bumping cells in a distance matrix of integers.\\
%% -----------------------------\\
%%   \caption{The core algorithms in PhyBin.}  
%% \end{figure}

%% -----------------------
%% As an example of an application dependent on $\BUMP$, we introduce the first of
%% our case studies: the bioinformatics application {\em PhyBin}.  PhyBin's core
%% algorithms are pictured in Figure \ref{f:phybin-arch}.  PhyBin was initially
%% released in 2010, and we attempted to modify and parallelize it in 2013 using
%% LVish.



