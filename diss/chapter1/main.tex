\chapter{Introduction}\label{ch:intro} % 1

Parallel programming---that is, writing programs that can take
advantage of parallel hardware to go faster---is notoriously
difficult.  A fundamental reason for this difficulty is that programs
can yield inconsistent results, or even crash, due to unpredictable
interactions between parallel tasks.

\emph{Deterministic-by-construction} parallel programming models,
though, offer the promise of freedom from subtle, hard-to-reproduce
nondeterministic bugs in parallel code.  While there are many ways to
construct deterministic parallel programs and to verify the
determinism of individual programs, only a
deterministic-by-construction programming model can provide a
\emph{language-level} guarantee of determinism: deterministic programs
are the only programs that can be expressed within the model.

A deterministic-by-construction programming model is one that ensures
that all programs written using the model have the same
\emph{observable behavior} every time they are run.  How do we define
what is observable about a program's behavior?  Certainly, we do
\emph{not} wish to preserve behaviors such as running time across
multiple runs---ideally, a deterministic-by-construction parallel
program will run faster when more parallel resources are available.
Moreover, we do not want to count scheduling behavior as
observable---in fact, we want to specifically \emph{allow} tasks to be
scheduled dynamically and unpredictably, without allowing such
\emph{schedule nondeterminism} to affect the observable behavior of a
program.  Therefore, in this dissertation I will define the observable
behavior of a program to be \emph{the value to which the program
  evaluates}.

\lk{In my proposal, I had a footnote here: ``We assume that programs
  have no side effects other than state effects.'' I think I instead
  just want to say that we \emph{ignore} other side effects.  They can
  \emph{happen}; it's just that they don't count.}

This definition of observable behavior ignores side effects other than
\emph{state}.  Even so, sharing of state between parallel computations
raises the possibility of \emph{race conditions} that allow schedule
nondeterminism to be observed in the outcome of a program.  For
instance, if a computation writes $3$ to a shared location while
another computation writes $4$, then a subsequent third computation
that reads and returns the location's contents will
nondeterministically return $3$ or $4$, depending on the order in
which the first two computations ran.  Therefore, if a parallel
programming model is to guarantee determinism by construction, it must
necessarily limit sharing of mutable state between parallel tasks in
some way.

\section{The deterministic-by-construction parallel programming landscape}\label{s:intro-landscape}

There is long-standing work on deterministic-by-construction parallel
programming models that limit sharing of state between tasks. The
possibilities include:

\begin{itemize}
\item \emph{No-shared-state parallelism.}  One classic approach to
  guaranteeing determinism in a parallel programming model is to allow
  \emph{no} shared mutable state between tasks, forcing tasks to
  produce values independently.  An example of no-shared-state
  parallelism is pure functional programming with function-level task
  parallelism, or \emph{futures}---for instance, in Haskell programs
  that use the @par@ and @pseq@ combinators~\cite{marlow-par}.  The
  key characteristic of this style of programming is lack of side
  effects: because programs don't have side effects, expressions can
  evaluate simultaneously without affecting the eventual value of the
  program.  Also belonging in this category are parallel programming
  models based on \emph{pure data parallelism}, such as Data Parallel
  Haskell~\cite{dph, dph-status} or the River Trail API for
  JavaScript~\cite{river-trail}, each of which extend existing
  languages with \emph{parallel array} data types and (observably)
  pure operations on them.  \lk{Does it make sense to say that DPH is
    observably pure?  It does mutate arrays.}

\item \emph{Data-flow parallelism.}  In \emph{Kahn process networks}
  (KPNs)~\cite{Kahn-1974}, as well as in the more restricted
  \emph{synchronous data flow} systems~\cite{Lee-sdn}, a network of
  independent ``computing stations'' communicate with each other
  through first-in first-out (FIFO) queues, or \emph{channels}.
  Reading data out of such a FIFO queue is a \emph{blocking}
  operation: once an attempt to read has started, a computing station
  cannot do anything else until the data to be read is available.
  Each station computes a sequential, monotonic function from the
  \emph{history} of its input channels (\ie, the input it has received
  so far) to the history of its output channels (the output it has
  produced so far).  KPNs are the basis for deterministic
  stream-processing languages such as StreamIt~\cite{streamit-asplos}.

\item \emph{Single-assignment parallelism.}  In parallel
  \emph{single-assignment} languages, ``full/empty'' bits are
  associated with memory locations so that they may be written to at
  most once. Single-assignment locations with blocking read semantics
  are known as \emph{IVars}~\cite{IStructures} and are a
  well-established mechanism for enforcing determinism in parallel
  settings: they have appeared in Concurrent ML as
  @SyncVar@s~\cite{reppy-cml-book}; in the Intel Concurrent
  Collections (abbreviated ``CnC'') system~\cite{CnC}; and have even
  been implemented in hardware in Cray MTA machines~\cite{cray-mta}.
  Although most of these uses incorporate IVars into
  already-nondeterministic programming environments, the
  \emph{monad-par} Haskell library~\cite{monad-par} uses IVars in a
  deterministic-by-construction setting, allowing user-created threads
  to communicate through IVars without requiring the @IO@ monad.
  Rather, operations that read and write IVars must run inside a @Par@
  monad, thus encapsulating them inside otherwise pure programs, and
  hence a program in which the only effects are @Par@ effects is
  guaranteed to be deterministic.

\item \emph{Imperative disjoint parallelism.}  Finally, yet another
  approach to guaranteeing determinism is to ensure that the state
  accessed by concurrent threads is \emph{disjoint}.\lk{This is the
    first place I've used the word ``concurrent''.  Should I explain
    concurrency vs. parallelism in a footnote or something?  Is it
    even a useful distinction in this context?}  Sophisticated
  permissions systems and type systems make it possible for imperative
  programs to mutate state in parallel, while guaranteeing that the
  same state is not accessed simultaneously by multiple threads.  I
  will refer to this style of programming as \emph{imperative disjoint
    parallelism}, with Deterministic Parallel Java
  (DPJ)~\cite{dpj-oopsla, dpj-hotpar09} as a prominent example.
\end{itemize}
The four parallel programming models listed above---no-shared-state
parallelism, data-flow parallelism, single-assignment parallelism, and
imperative disjoint parallelism---all seem to embody rather different
mechanisms for exposing parallelism and for ensuring determinism.  If
we view these different programming models as a toolkit of unrelated
choices, though, it is not clear how to proceed when we want to
implement an application with multiple parallelizable components that
are best suited to different programming models.  For example, suppose
we have an application in which we wish to use data-flow pipeline
parallelism via FIFO queues, but also disjoint parallel mutation of
arrays.  It is not obvious how to compose two programming models that
each only allow communication through a single type of shared data
structure---or, if we do manage to compose them, whether or not the
determinism guarantee of the individual models is preserved by their
composition.  Therefore, we seek a general, broadly-applicable model
for deterministic parallel programming that is not tied to a
particular data structure.

\section{Lattice-based, monotonic data structures as a basis for deterministic parallelism}\label{s:intro-monotonic}

In KPNs and other data-flow models, communication takes place over
blocking FIFO queues with ever-increasing \emph{channel histories},
while in IVar-based programming models such as CnC and monad-par, a
shared data store of blocking single-assignment memory locations grows
monotonically.  Hence \emph{monotonic data structures}---data
structures to which information can only be added and never
removed---emerge as a common theme of guaranteed-deterministic
programming models.\footnote{In Chapter~\ref{ch:lvars}, I will refine
  this definition of ``monotonic data structures'' to mean data
  structures to which information can only be added and never removed,
  \emph{and} for which the order in which information is added is not
  observable.}

In this dissertation, I show that \emph{lattice-based} data
structures, or \emph{LVars}, are the foundation for a model of
deterministic-by-construction parallel programming that allows a more
general form of communication between tasks than previously existing
guaranteed-deterministic models allowed.  LVars generalize IVars and
are so named because the states an LVar can take on are elements of an
application-specific \emph{lattice}.\footnote{As I will explain in
  Chapter~\ref{ch:lvars}, what I am calling a ``lattice'' here really
  need only be a \emph{bounded join-semilattice} augmented with a
  greatest element $\top$.}  This application-specific lattice
determines the semantics of the @put@ and @get@ operations that
comprise the interface to LVars (which I will explain in detail in
Chapter~\ref{ch:lvars}):
\begin{itemize}
\item The @put@ operation can only change an LVar's state in a way
  that is \emph{monotonically increasing} with respect to the lattice,
  because it takes the least upper bound of the current state and the
  new state.
\item The @get@ operation allows only limited observations of the
  state of an LVar.  It requires the user to specify a \emph{threshold
    set} of minimum values that can be read from the LVar, where every
  two elements in the threshold set must have the lattice's greatest
  element $\top$ as their least upper bound.  A call to @get@ blocks until the
  LVar in question reaches a (unique) value in the threshold set, then
  unblocks and returns that value.
\end{itemize}
Together, least-upper-bound writes via @put@ and threshold reads via
@get@ yield a deterministic-by-construction programming model.  That
is, a program in which @put@ and @get@ operations on LVars are the
only side effects will have the same observable result in spite of
parallel execution and schedule nondeterminism.  \lk{Maybe I don't
  need the next sentence at all; maybe it's covered by the
  ``organization'' bullet points below.} As we will see in
Chapter~\ref{ch:lvars}, no-shared-state parallelism, data-flow
parallelism and single-assignment parallelism are all subsumed by the
LVars programming model, and as we will see in
Chapter~\ref{ch:lvish}\lk{Maybe refer to a specific section?},
imperative disjoint parallelism is compatible with LVars as well.
Furthermore, as I show in Section~\ref{s:lvars-generalizing}, we can
generalize the behavior of the @put@ and @get@ operations while
retaining determinism: we can generalize @put@ from least-upper-bound
writes to arbitrary inflationary and commutative writes, and we can
generalize the @get@ operation to allow a more general form of
threshold reads.

\section{Quasi-deterministic and event-driven programming with LVars}\label{s:intro-quasi}

The LVars model described above guarantees determinism and supports an
unlimited variety of shared data structures: anything viewable as a
lattice.  However, it is not as general-purpose as one might hope.
Consider, for instance, an algorithm for unordered graph traversal.  A
typical implementation involves a monotonically growing set of ``seen
nodes''; neighbors of seen nodes are fed back into the set until it
reaches a fixed point.  Such fixpoint computations are ubiquitous, and
would seem to be a perfect match for the LVars model due to their use
of monotonicity.  But they are not expressible using the threshold
@get@ and least-upper-bound @put@ operations, nor even with the more
general alternatives to @get@ and @put@ mentioned above.

The problem is that these computations rely on \emph{negative}
information about a monotonic data structure, \ie, on the
\emph{absence} of certain writes to the data structure.  In a graph
traversal, for example, neighboring nodes should only be explored if
the current node is \emph{not yet} in the set; a fixpoint is reached
only if no new neighbors are found; and, of course, at the end of the
computation it must be possible to learn exactly which nodes were
reachable (which entails learning that certain nodes were not).  In
Chapter~\ref{ch:quasi}, I describe two extensions to the basic LVars
model that make such computations possible:
\begin{itemize}
\item First, I extend the model with a primitive operation @freeze@
  for \emph{freezing} an LVar, which allows its contents to be read
  immediately and exactly, rather than the blocking threshold read
  that @get@ allows.  The @freeze@ primitive imposes the following
  trade-off: once an LVar has been frozen, any further writes that
  would change its value instead raise an exception; on the other
  hand, it becomes possible to discover the exact value of the LVar,
  learning both positive and negative information about it, without
  blocking.  Therefore, LVar programs that use @freeze@ are \emph{not}
  guaranteed to be deterministic, because they could
  nondeterministically raise an exception depending on how @put@ and
  @freeze@ operations are scheduled.  However, such programs satisfy
  \emph{quasi-determinism}: all executions that produce a final value
  produce the \emph{same} final value.
\item Second, I add the ability to attach \emph{event handlers} to an
  LVar.  When an event handler has been registered with an LVar, it
  invokes a \emph{callback function} to run, asynchronously, whenever
  events arrive (in the form of monotonic updates to the LVar).
  Crucially, it is possible to check for \emph{quiescence} of a group
  of handlers, discovering that no callbacks are currently enabled---a
  transient, negative property.  Since quiescence means that there are
  no further changes to respond to, it can be used to tell that a
  fixpoint has been reached.
\end{itemize}
Of course, since more events could arrive later, there is no way to
guarantee that quiescence is permanent---but since the contents of the
LVar being written to can only be read through @get@ or @freeze@
operations anyway, early quiescence poses no risk to determinism or
quasi-determinism, respectively.  In fact, freezing and quiescence
work particularly well together because freezing provides a mechanism
by which the programmer can safely ``place a bet'' that all writes
have completed.  Hence freezing and handlers make possible fixpoint
computations like the graph traversal described above.  Moreover, if
we can ensure that the freeze does indeed happen after all writes have
completed, then we can ensure that the computation is deterministic,
and it is possible to enforce this ``freeze-last'' idiom at the
implementation level, as I discuss below (and, in more detail, in
Chapter~\ref{ch:lvish}\lk{Refer to a specific section?}).

\section{The LVish library}\label{s:intro-lvish}

\lk{I'm not sure yet if this is a problem, but in this section I just
  call the library ``LVish'', which may or may not be distinct enough
  from ``the LVish programming model'' in Chapter~\ref{ch:quasi}.}

To demonstrate the practicality of the LVars programming model, in
Chapter~\ref{ch:lvish} I will describe
\emph{LVish},\footnote{Available at
  \url{http://hackage.haskell.org/package/lvish}.} a Haskell library
for deterministic and quasi-deterministic programming with LVars.

LVish provides a @Par@ monad for encapsulating parallel computation
and enables a notion of lightweight, library-level threads to be
employed with a custom work-stealing scheduler.\footnote{The
  \lstinline|Par| monad exposed by LVish generalizes the original
  \lstinline|Par| monad exposed by the \emph{monad-par} library
  ({\url{http://hackage.haskell.org/package/monad-par}}, described by
  Marlow \etal~\cite{monad-par}), which allows determinism-preserving
  communication between threads, but only through IVars, rather than
  LVars.}  LVar computations run inside the @Par@ monad, which is
indexed by an \emph{effect level}, allowing fine-grained specification
of the effects that a given computation is allowed to perform.  For
instance, since @freeze@ introduces quasi-determinism, a computation
indexed with a deterministic effect level is not allowed to use
@freeze@.  Thus, the \emph{type} of an LVish computation reflects its
determinism or quasi-determinism guarantee.  Furthermore, if a
@freeze@ is guaranteed to be the \emph{last} effect that occurs in a
computation, then it is impossible for that @freeze@ to race with a
@put@, ruling out the possibility of a run-time @put@-after-@freeze@
exception.  LVish exposes a @runParThenFreeze@ operation that captures
this ``freeze-last'' idiom and has a deterministic effect level.

LVish also provides a variety of lattice-based data structures (\eg,
sets, maps, graphs) that support concurrent insertion, but not
deletion, during @Par@ computations.  In addition to those that LVish
provides, users may implement their own lattice-based data structures,
and LVish provides tools to facilitate the definition of user-defined
LVars.  I will describe the proof obligations for data structure
implementors and give examples of applications that use user-defined
LVars as well as those that the library provides.

In addition to discussing the implementation of the LVish library and
introducing the above features with examples, Chapter~\ref{ch:lvish}
illustrates LVish through two case studies, drawn from my
collaborators' and my experience using the LVish library, both of which
make use of handlers and freezing:
\begin{itemize}
\item First, I describe using LVish to parallelize a control flow
  analysis ($k$-CFA) algorithm.  The goal of $k$-CFA is to compute the
  flow of values to expressions in a program.  The $k$-CFA algorithm
  proceeds in two phases: first, it explores a graph of \emph{abstract
    states} of the program; then, it summarizes the results of the
  first phase.  Using LVish, these two phases can be pipelined in a
  manner similar to the pipelined breadth-first graph traversal
  described above; moreover, the original graph exploration phase can
  be internally parallelized.  I contrast the LVish implementation
  with the original sequential implementation and give performance
  results.
\item Second, I describe using LVish to parallelize
  \emph{PhyBin}~\cite{PhyBin}, a bioinformatics application for
  comparing sets of phylogenetic trees that relies heavily on a
  parallel tree-edit distance algorithm~\cite{hashrf}.  In addition to
  handlers and freezing, the PhyBin application crucially relies on
  the aforementioned ability to perform generalized writes to LVars
  that are commutative and inflationary with respect to the lattice in
  question, but \emph{not} idempotent (in contrast to
  least-upper-bound writes discussed above, which are idempotent as
  well as commutative and inflationary).
\end{itemize}

\section{Deterministic threshold queries of distributed data structures}\label{s:intro-cvrdts}

The LVars model is closely related to the concept of
\emph{conflict-free replicated data types} (CRDTs)~\cite{crdts} for
enforcing \emph{eventual consistency}~\cite{vogels-ec} of replicated
objects in a distributed system.  In particular, \emph{state-based} or
\emph{convergent} replicated data types, abbreviated as
\emph{CvRDTs}~\cite{crdts, crdts-tr}, leverage the mathematical
properties of join-semilattices to guarantee that all replicas of an
object (for instance, in a distributed database) eventually agree.

Although CvRDTs are provably eventually consistent, queries of CvRDTs
(unlike threshold reads of LVars) nevertheless allow inconsistent
intermediate states of replicas to be observed.  That is, if two
replicas of a CvRDT object are updated independently, reads of those
replicas may disagree until a (least-upper-bound) \emph{merge}
operation takes place.

Taking inspiration from LVar-style threshold reads, in
Chapter~\ref{ch:distributed} I show how to extend CvRDTs to support
deterministic, \emph{strongly consistent} queries using a mechanism
called \emph{threshold queries} (or, seen from another angle, I show
how to port threshold reads from a shared-memory setting to a
distributed one).  The threshold query technique generalizes to any
lattice, and hence any CvRDT, and allows deterministic observations to
be made of replicated objects before the replicas' states have
converged.  This work has practical relevance since, while real
distributed database applications call for a combination of eventually
consistent and strongly consistent queries, CvRDTs only support the
former, and threshold queries extend the CvRDT model to support both
kinds of queries within a single, lattice-based reasoning framework.
Furthermore, since threshold queries behave deterministically
regardless of whether all replicas agree, they suggest a way to save
on synchronization costs: existing operations that require all
replicas to agree could be done with threshold queries instead, and
retain behavior that is \emph{observably} strongly consistent while
avoiding unnecessary synchronization.

\section{Thesis statement, and organization of the rest of this dissertation}\label{s:intro-thesis}

With the above background, I can state my thesis:\lk{This format
  ripped off from Josh Dunfield.}
\begin{quote}
  Lattice-based data structures are a general and practical foundation
  for deterministic and quasi-deterministic parallel and distributed
  programming.
\end{quote}
The rest of this dissertation supports my thesis as follows:
\begin{itemize}
  \item \emph{Lattice-based data structures}: In
    Chapter~\ref{ch:lvars}, I formally define LVars and use them to
    define $\lambdaLVar$, a call-by-value parallel calculus with a
    store of LVars that support least-upper-bound @put@ and threshold
    @get@ operations. In Chapter~\ref{ch:quasi}, I extend
    $\lambdaLVar$ to add support for event handlers and the @freeze@
    operation, calling the resulting language $\lambdaLVish$.
    Appendix~\ref{app:plt-redex} contains runnable versions of
    $\lambdaLVar$ and $\lambdaLVish$ implemented using the PLT Redex
    semantics engineering system~\cite{redex-book} for interactive
    experimentation.

  \item \emph{general}: In Chapter~\ref{ch:lvars}, I show how
    previously existing deterministic parallel programming models
    (single-assignment languages, Kahn process networks) are subsumed
    by the lattice-generic LVars model.  Additionally, I show how to
    generalize the @put@ and @get@ operations on LVars while
    preserving their determinism.

  \item \emph{deterministic}: In Chapter~\ref{ch:lvars}, I show that
    the basic LVars model guarantees determinism by giving a proof of
    determinism for the $\lambdaLVar$ language with @put@ and @get@.

  \item \emph{quasi-deterministic}: In Chapter~\ref{ch:quasi}, I
    define quasi-determinism and give a proof of quasi-determinism for
    $\lambdaLVish$, which adds the @freeze@ operation and event
    handlers to the $\lambdaLVar$ language of Chapter~\ref{ch:lvars}.
    $\lambdaLVish$ also generalizes the language from allowing only
    least-upper-bound writes to allowing arbitrary inflationary and
    commutative writes.

  \item \emph{practical}: In Chapter~\ref{ch:lvish}, I describe the
    LVish Haskell library, which is based on the LVars programming
    model, and demonstrate how it is used for practical programming
    with the two case studies described above, including performance
    results.

  \item \emph{distributed programming}: In
    Chapter~\ref{ch:distributed}, I show how LVar-style threshold
    reads apply to the setting of distributed, replicated data
    structures.  In particular, I extend convergent replicated data
    types (CvRDTs) to support strongly consistent threshold queries,
    which take advantage of the existing lattice structure of CvRDTs
    and allow deterministic observations to be made of their contents
    without requiring all replicas to agree.
\end{itemize}

\section{Previously published work}

This dissertation draws heavily on the earlier work and writing
appearing in the following papers, written jointly with several
collaborators:\lk{This exact wording stolen from Aaron Turon.  maybe
  tweak it?}

\begin{itemize}
\item Lindsey Kuper and Ryan
  R. Newton. 2013. \href{http://doi.acm.org/10.1145/2502323.2502326}{LVars:
    lattice-based data structures for deterministic parallelism.} In
  \emph{Proceedings of the 2nd ACM SIGPLAN Workshop on Functional
    High-Performance Computing} (FHPC '13).

\item Lindsey Kuper, Aaron Turon, Neelakantan R. Krishnaswami, and
  Ryan
  R. Newton. 2014. \href{http://doi.acm.org/10.1145/2535838.2535842
  }{Freeze after writing: quasi-deterministic parallel programming
    with LVars.} In \emph{Proceedings of the 41st ACM SIGPLAN-SIGACT
    Symposium on Principles of Programming Languages} (POPL '14).

\item Lindsey Kuper, Aaron Todd, Sam Tobin-Hochstadt, and Ryan
  R. Newton. 2014. \href{http://doi.acm.org/10.1145/2594291.2594312
  }{Taming the parallel effect zoo: extensible deterministic
    parallelism with LVish.} In \emph{Proceedings of the 35th ACM
    SIGPLAN Conference on Programming Language Design and
    Implementation} (PLDI '14).

\end{itemize}

\lk{I was going to put something like, ``The material in this chapter
  is based on research done jointly with...'' in individual chapters,
  and cite each paper at the start of its chapter.  But chapters don't
  really correspond to the papers, so I'm just going to try citing all
  the papers up front like this.}
