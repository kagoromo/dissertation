%%% Macros for typesetting lambdaLVar and lambdaLVish semantics and metatheory.

%%% Names of the languages.
\newcommand{\lambdaLVar}{\ensuremath{\lambda_{\textrm{LVar}}}}
\newcommand{\lambdaLVish}{\ensuremath{\lambda_{\textrm{LVish}}}}

%%% BNF grammar stuff
\newcommand{\bnfdef}{\ensuremath{::=}}
\newcommand{\bnfsep}{\ensuremath{\ \ | \ \ }}
\newcommand{\setsep}{\ensuremath{\ | \ }}
\newcommand{\sep}{\bnfsep}

%%% Metafunctions
\newcommand{\subst}[3]{\ensuremath{#1[#2 := #3]}}
\newcommand{\dom}[1]{\ensuremath{\mathit{dom}(#1)}}
\newcommand{\incomp}[1]{\ensuremath{\mathit{incomp}(#1)}}
\newcommand{\pred}[1]{\ensuremath{\mathit{pred}(#1)}}
\newcommand{\userlub}[2]{\ensuremath{#1 \sqcup #2}}
\newcommand{\lubp}[2]{\ensuremath{#1 \sqcup_p #2}}
\newcommand{\qexist}[2]{\ensuremath{\exists\,#1.~#2}}
\newcommand{\qforall}[2]{\ensuremath{\forall\,#1.~#2}}
\newcommand{\userelements}{\ensuremath{\mathit{Elements}}}
\newcommand{\userleq}{\ensuremath{\sqsubseteq}}
\newcommand{\nuserleq}{\ensuremath{\not \sqsubseteq}}
\newcommand{\usergeq}{\ensuremath{\sqsupseteq}}
\newcommand{\userlt}{\ensuremath{\sqsubset}}
\newcommand{\leqp}{\ensuremath{\sqsubseteq_p}}
\newcommand{\ltp}{\ensuremath{\sqsubset_p}}
\newcommand{\botp}{\ensuremath{\bot_{p}}}
\newcommand{\topp}{\ensuremath{\top\hspace{-0.85mm}_{p}}}

%%% Evaluation / operational semantics
\newcommand*{\longhookrightarrow}{\ensuremath{\lhook\joinrel\relbar\joinrel\rightarrow}}

%%% Reduction semantics evaluation relation
\newcommand{\parstepsto}{\ensuremath{\longhookrightarrow}}
\newcommand{\parstepstoeq}{\ensuremath{\longhookrightarrow^{?}}}

%%% Context semantics evaluation relation
\newcommand{\ctxstepsto}{\ensuremath{\longmapsto}}

\newcommand{\evalctxt}[2]{\ensuremath{#1\hspace{-1mm}\left[#2\right]}}
\newcommand{\E}[1]{\ensuremath{\evalctxt{E}{#1}}}

%%% Stores and store operations
\newcommand{\StoreVal}{\ensuremath{\mathit{StoreVal}}}
\newcommand{\fmap}{\ensuremath{\stackrel{\textrm{fin}}{\rightarrow}}}
\newcommand{\eqstore}[2]{\ensuremath{#1 =_S #2}}
\newcommand{\leqstore}[2]{\ensuremath{#1 \userleq_S #2}}
\newcommand{\neqstore}[2]{\ensuremath{#1 \not =_S #2}}
\newcommand{\nleqstore}[2]{\ensuremath{#1 \not \userleq_S #2}}
\newcommand{\ltstore}[2]{\ensuremath{#1 \userlt_S #2}}
\newcommand{\lubstore}[2]{\ensuremath{#1 \sqcup_{S} #2}}
\newcommand{\glbstore}[2]{\ensuremath{#1 \sqcap_{S} #2}}
\newcommand{\storebindingRaw}[2]{\ensuremath{#1 \mapsto #2}}
\newcommand{\storebinding}[3]{\storebindingRaw{#1}{\state{#2}{#3}}}
\newcommand{\state}[2]{\ensuremath{(#1, #2)}}
\newcommand{\store}[1]{\left[ #1 \right]}
\newcommand{\extS}[4]{#1[\storebinding{#2}{#3}{#4}]}
\newcommand{\extSRaw}[3]{#1[\storebindingRaw{#2}{#3}]}
\newcommand{\storeset}{\mathcal{S}}
\newcommand{\status}{\ensuremath{\mathit{frz}}}
\newcommand{\statuseq}{\ensuremath{=_{\textit{frz}}}} % textit looks better than mathit at the subscript size
\newcommand{\statusneq}{\ensuremath{\neq_{\textit{frz}}}} % textit looks better than mathit at the subscript size
\newcommand{\frozentrue}{\ensuremath{\textsf{true}}}
\newcommand{\frozenfalse}{\ensuremath{\textsf{false}}}
\newcommand{\Loc}{\mathit{Loc}}
\newcommand{\topS}{\ensuremath{\top\hspace{-0.85mm}_{S}}}
\newcommand{\LVar}{\mathit{LVar}}

%% It would be really nice to be able to typeset stores like

%%        \store{l_1, l_2, l_3}{3, 4, 5}

%% producing

%%        {l_1 -> 3, l_2 -> 4, l_3 -> 5}

%% but to do that, I think we need something like what's done in
%% http://stackoverflow.com/questions/2402354/split-comma-separated-parameters-in-latex,
%% and I haven't figured out how to do it just yet :(

%%% States and configurations
\newcommand{\conf}{\ensuremath{\sigma}}
\newcommand{\config}[2]{\ensuremath{\langle #1;\, #2 \rangle}}
\newcommand{\error}{\ensuremath{\textbf{\textsf{error}}}}

\newcommand{\handledBy}{\hookrightarrow}

%%% Assorted math stuff
\newcommand{\defeq}{\stackrel{\triangle}{=}} % "defined-as"
\newcommand{\pfn}{\rightharpoonup}
\newcommand{\bag}[1]{\Lbag #1 \Rbag}
\newcommand{\setof}[1]{\left\{#1\right\}}
\newcommand{\comprehend}[2]{\setof{{#1}\;\middle|\;{#2}}}
\newcommand{\power}[1]{\mathcal{P}(#1)}
\newcommand{\inlop}{\mathsf{inl}}
\newcommand{\inl}[1]{\inlop\,{#1}}
\newcommand{\inrop}{\mathsf{inr}}
\newcommand{\inr}[1]{\inrop\,{#1}}
\newcommand{\letv}[2]{\mathsf{let}\;{#1} = {#2}}
\newcommand{\Freeze}[1]{\mathrm{Freeze}(#1)}
\newcommand{\piinv}{\pi^{-1}}

%%% Formatting of definitions, theorems, etc.
\newtheorem{lem}{Lemma}[chapter]
\newtheorem{cor}{Corollary}[chapter]
\newtheorem{thm}{Theorem}[chapter]
