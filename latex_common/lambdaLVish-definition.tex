%%% The syntax and semantics of lambdaLVish.

\newcommand{\FigLambdaLVishGrammar}[1][t]{
\begin{figure}[#1]
  \small
  \setlength{\arraycolsep}{1pt} % A hack to squeeze everything into a column
  Given a lattice $(D, \userleq, \bot, \top)$ with elements $d \in D$:
    \[
    \begin{array}{@{}c@{}}
    \begin{array}{rlcl}
      \mbox{configurations} & \conf & \bnfdef & \config{S}{e} \sep \error \\
      \mbox{expressions} & e & \bnfdef & 
           x \sep 
           v \sep 
           \app{e}{e} \sep 
%% \\
%%            & & &
           \getexp{e}{e} \sep 
           \putexp{e}{e} \sep
           \NEW \sep
           \freeze{e} \\
           & & | &
           \freezeafter{e}{e}{e} \\
           & & | &
           \freezeafterfull{l}{Q}{\lam{x}{e}}{\setof{e, \dots}}{H} \\

      \mbox{stores} & S & \bnfdef &
        \store{\storebindingRaw{l_1}{p_1},\,\dots, \storebindingRaw{l_n}{p_n}} \sep \topS \\

      \mbox{values} & v & \bnfdef & \unit \sep d \sep p \sep l \sep P \sep Q \sep \lam{x}{e} \\[0.3em]

      \mbox{eval contexts} & E & \bnfdef &
           [~] \sep
           \app{E}{e} \sep
           \app{e}{E} \sep
           \getexp{E}{e} \sep
           \getexp{e}{E} \sep 
           \putexp{E}{e} \\ \multicolumn{4}{@{}r@{}}{
           \begin{array}{@{\sep\;}l@{}}
             \putexp{e}{E} \sep
             \freeze {E} \sep
             \freezeafter{E}{e}{e}
             \\
             \freezeafter{e}{E}{e} \sep
             \freezeafter{e}{e}{E}  \\
             \freezeafterfull{v}{v}{v}{\setof{e\dots~E~e\dots}}{H}
           \end{array}}

    \end{array}\\ \\[-4pt]
    \begin{array}{@{}c@{}}

      % N.B. In Redex we actually rule out store values being Top in
      % the grammar, and have a special StoreVal type for elements
      % other than Top.  Here we don't bother, and we just say that
      % stores contain bindings from locations l to pairs p.

      \begin{array}{rlcl}

        \mbox{``handled'' sets} & H & \bnfdef & \setof{d_1,\,\dots, d_n} \\

        \mbox{threshold sets} & P & \bnfdef & \stateset{p_1,\,p_2,\,\dots} \\

        \mbox{event sets} & Q & \bnfdef & \stateset{d_1,\,d_2,\,\dots} 

      \end{array}\quad
      \begin{array}{rlcl}

        \mbox{states} & p & \bnfdef & \state{d}{\status} \\


        \mbox{status bits} & \status & \bnfdef & \frozentrue \sep \frozenfalse 

      \end{array}

    \end{array}
    \end{array}
    \]
  \caption{Syntax for $\lambdaLVish$.}
  \label{f:lambdaLVish-syntax}
\end{figure}
}

\newcommand{\FigLambdaLVishSemantics}[1][t]{
\begin{figure*}[#1]
  Given a lattice $(D, \userleq, \bot, \top)$ with elements $d \in D$:

$\incomp{P} \defeq \qforall{p_1,p_2 \in P}{(p_1 \neq p_2 \implies \lubp{p_1}{p_2}
  = \topp)}$\hfill \fbox{$\conf \parstepsto \conf'$}
{\small
  
  \begin{mathpar}
    \inferrule*[lab=E-Eval-Ctxt]
        {\config{S}{e} \parstepsto \config{S'}{e'}}
        {\config{S}{\E{e}} \parstepsto \config{S'}{\E{e'}}}

    \inferrule*[lab=E-Beta]
        {~}
        {\config{S}{\app{(\lam{x}{e})}{v}} \parstepsto \config{S}{\subst{e}{x}{v}}}

    \inferrule*[lab=E-New, right=\textnormal{($l \notin \dom{S}$)}]
        {~}
        {\config{S}{\NEW} \parstepsto \config{\extS{S}{l}{\bot}{\frozenfalse}}{l}}

    \inferrule*[lab=E-Put]
        {S(l) = p_1 \\ p_2 = \lubp{p_1}{\state{d_2}{\frozenfalse}} \\ p_2 \neq \topp}
        {\config{S}{\putexp{l}{d_2}} \parstepsto
          \config{\extSp{S}{l}{p_2}}{\unit}}

    \inferrule*[lab=E-Put-Err]
        {S(l) = p_1 \\ \lubp{p_1}{\state{d_2}{\frozenfalse}} = \topp}
        {\config{S}{\putexp{l}{d_2}} \parstepsto 
          \error}

    \inferrule*[lab=E-Get]
        {S(l) = p_1 \\ \incomp{P} \\ p_2 \in P \\ p_2 \leqp p_1}
        {\config{S}{\getexp{l}{P}} \parstepsto \config{S}{p_2}}

    \inferrule*[lab=E-Freeze-Init]
        {~}
        {\config{S}{\freezeafter{l}{Q}{\lam{x}{e}}} \parstepsto
          \config{S}{\freezeafterfull{l}{Q}{\lam{x}{e}}{\setof{}}{\setof{}}}}

    \inferrule*[lab=E-Spawn-Handler]
        { S(l) = \state{d_1}{\status_1} \\ 
          d_2 \userleq d_1 \\
          d_2 \notin H \\
          d_2 \in Q
        }
        {
          \config{S}{\freezeafterfull{l}{Q}{\lam{x}{e_0}}{\setof{e, \dots}}{H}}
          \parstepsto
          \config{S}{\freezeafterfull{l}{Q}{\lam{x}{e_0}}{\setof{\subst{e_0}{x}{d_2}, e, \dots}}
            {\{d_2\}\cup H}}
        }

    \inferrule*[lab=E-Freeze-Final]
        { S(l) = \state{d_1}{\status_1} \\ 
          \forall{d_2} ~.~ ( {d_2 \userleq d_1 \land d_2 \in Q} \Rightarrow 
             d_2 \in H) }
        {
          \config{S}{\freezeafterfull{l}{Q}{v}{\setof{v\dots}}{H}}
          \parstepsto
          \config{\extS{S}{l}{d_1}{\frozentrue}}{d_1}
        }

    \inferrule*[lab=E-Freeze-Simple]
        { S(l) = \state{d_1}{\status_1} }
        {
          \config{S}{\freeze{l}}
          \parstepsto
          \config{\extS{S}{l}{d_1}{\frozentrue}}{d_1}
        }
        
  \end{mathpar}}
  \caption{An operational semantics for $\lambdaLVish$.}
  \label{f:lambdaLVish-semantics}
\end{figure*}
}
